\lecdate{24.05.2017}
%\section{Überblick}
%\slides{06-filesystems}{2}

\section{Implementierungen von Dateisystemen}
\subsection{Kontinuierliche Allokation}
\slides{06-filesystems}{3}

\subsection{Verkettete Liste}
\slides{06-filesystems}{4}
Beispiel ist nicht so schnell, da gesprungen werden muss (0015$\to$FFFF$\to$0014).
\subsubsection{Nachteile}
\slides{06-filesystems}{5}
Wichtiger konzeptioneller Nachteil: Dadurch, dass in jedem Nutzdatenblock eine Verwaltungsinformation gespeichert ist, ist der Netto-Nutzdatenblock keine 2er-Potenz mehr!
\subsubsection{Liste mit Zuordnungstabelle}
\slides{06-filesystems}{6}
Größe bspw. FAT16: $2^{16}\cdot 32 \unit{KiB}$ ($32\unit{KiB}$: Clustergröße)

\subsection{Indizierte Speicherung}
\slides{06-filesystems}{7}
\subsubsection{Speicherung mit variablen Indexblocks}
\slides{06-filesystems}{8}
\subsubsection{Indirekt-indizierte Speicherung}
\slides{06-filesystems}{9}
\slides{06-filesystems}{10}

%\subsubsection{Beispiel ISO 9660 (1988)}
%\slides{06-filesystems}{11}
%\slides{06-filesystems}{12}

\subsubsection{Beispiel Unix Dateisystem}
\slides{06-filesystems}{13}
\subsubsection*{Dateiadressierung mittels Inodes}
\slides{06-filesystems}{14}
(graue Datenblöcke sind gleich groß wie die weiß gezeichneten Datenblöcke!)
\slides{06-filesystems}{15}
\paragraph{Klausurfrage} Zwischen welchen Parametern wird im Unix-Dateisystem ein Kompromiss gefunden:
\begin{itemize}
\item kurze Zugriffszeit auf kleine Dateien
\item große Dateien realisierbar
\end{itemize}

\subsection{Verweise (Links)}
\begin{itemize}
\item \emph{Hard Links}\\
Link auf Adresse wird erstellt. Wenn „Original“ gelöscht wird, bleibt Datei durch Link erhalten (Löschen ist viel mehr \lstinline`unlink()`, wo geprüft wird, ob noch ein Verweis auf Adresse besteht ($\to$ Link-Counter, Info über \lstinline`stat Datei`). Nur wenn kein Verweis mehr besteht, werden Daten gelöscht, sonst nur der Verweis).\\
Alle Attribute (Änderungsdatum usw.) sind gleich.\\
Im Prinzip wird nur ein neuer Name für die Datei erstellt.\\
Es kann nicht mehr herausgefunden werden, welches das „Original“ ist: Beide Dateien/Verweise sind gleichberechtigt.\\
Funktioniert nur auf gleicher Partition.
\item \emph{Soft/Symbolic Links}\\
Link auf Datei wird erstellt. Dieser Link ist eine Textdatei, die die Pfadangabe enthält.\\
Funktioniert über Dateisystem-Grenzen hinweg.
\end{itemize}

\subsection{Journaling}
\subsubsection{Nachteil traditioneller Dateisysteme}
\slides{06-filesystems}{16}
\subsubsection{Idee des Journals}
\slides{06-filesystems}{17}
$\to$ nach Absturtz muss nicht (mit \lstinline`fsck`) das gesamte Dateisystem überprüft werden, sondern nur das Journal erneut erstellt werden.
\subsubsection{Betriebsmodi}
\slides{06-filesystems}{18}
\subsubsection{Zusammenfassung}
\slides{06-filesystems}{19}
\subsubsection{Erweiterte Attribute}
\slides{06-filesystems}{20}
\subsubsection*{… in Linux-Dateisystemen}
\slides{06-filesystems}{21}
\subsubsection*{Benötigte Systemrufe}
\slides{06-filesystems}{22}

\section{I/O-Scheduling}
\subsubsection*{Prinzip der Festplatte}
\slides{06-filesystems}{31}
Zwei Komponenten, die Zugriffszeiten beeinflussen:
\begin{itemize}
\item Seek: Zeit, die der Lesekopf braucht, um auf die richtige Spur zu kommen.
\item Rot: Zeit, die die Platte zur Rotation braucht. Rotationslatenz liegt zwischen $0$ (Sektor zufällig unter Lesekopf), und $\frac{1}{f}$ (Sektor gerade am Lesekopf vorbei).
\end{itemize}

\subsubsection*{Optimierung von Massenspeicherzugriffen}
\slides{06-filesystems}{23}

\subsection{FCFS, SSTF}

\subsection{SCAN (Elevator) und Varianten}

\subsection{Shortest Access Time First (SATF)}

\subsection{Verfahren im Linux-Kernel}




