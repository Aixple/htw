\lecdate{06.12.2017}
\section{Grundlagen}
\slides{V6_Codegen}{1}
\section{Funktionen zur Codegenerierung}
\slides{V6_Codegen}{3}
\begin{lstlisting}[language=C]
// Schreibt am aktuellen Programmcounter
void wr2ToCode(short x){
	*pCode++=(unsigned char)(x & 0xff);
	*pCode++=(unsigned char)(x >> 8);
}
// Schreibt an der übergebenen Stelle
void wr2ToCodeAtP(short x,char*pD){
	* pD =(unsigned char)(x & 0xff);
	*(pD+1)=(unsigned char)(x >> 8);
}
// Schreibt Befehl mit 0, 1, 2 oder 3 Parametern in den Codeausgabepuffer
int code(tCode Code,...){
	va_list ap;
	short sarg;
	if (pCode-vCode+MAX_LEN_OF_CODE>=LenCode){
		char* xCode=realloc(vCode,(LenCode+=1024));
		if (xCode==NULL) Error(ENoMem);
		pCode=xCode+(pCodevCode);
		vCode=xCode;
	}
	*pCode++=(char)Code;
	va_start(ap,Code);
	switch (Code){
		/* Befehle mit 3 Parametern */
		case entryProc:
			sarg=va_arg(ap,int);
			wr2ToCode(sarg);
		/* Befehle mit 2 Parametern */
		case puValVrGlob:
		case puAdrVrGlob:
			sarg=va_arg(ap,int);
			wr2ToCode(sarg);
		/* Befehle mit 1 Parameter */
		case puValVrMain:
		case puAdrVrMain:
		case puValVrLocl:
		case puAdrVrLocl:
		case puConst:
		case jmp :
		case jnot:
		case call:
			sarg=va_arg(ap,int); // Prozedurnummer
			wr2ToCode(sarg);
			break;
		/* keine Parameter */
		default :
			break;
		}
	va_end (ap);
	return OK;
}
\end{lstlisting}

\section{Variablen zur Codegenerierung}
\slides{V6_Codegen}{7}
\subsection{EntryProc}
Vor der eigentlichen Codegenerierung ist der Befehl \lstinline`entryProc(Codelen, ProcIdx, VarLen)` zu generieren, dies erfolgt in dem Nil-Bogen vor statement (siehe \autoref{6-block}: bl6). CodeLen bleibt zunächst 0, ProcEdx ist die aktuelle Prozedur und VarLen ist dem SpzzVar zu entnehmen.
\subsection{programm}
Nur eine Aktion, wenn die Kompilierung bis hier gekommen ist $\to$ Abschlussarbeiten.
\slides[.5]{V6_Codegen}{8}
\subsection{block}
\label{6-block}
\slides[.5]{V6_Codegen}{9}
\slides{V6_Codegen}{10}
\slides{V6_Codegen}{11}
\slides{V6_Codegen}{12}
\subsection{statement}
\slides[.5]{V6_Codegen}{13}
\subsection{expression}
\slides[.5]{V6_Codegen}{16}
\subsection{term}
\slides[.5]{V6_Codegen}{17}
\subsection{factor}
\slides[.5]{V6_Codegen}{18}
\subsection{condition}
\slides[.5]{V6_Codegen}{19}

\subsection{Auszuführende Funktionen in statement}
\subsubsection{Zuweisungen}
stx wird jeweils ausgeführt, nachdem die beschriebenen Handlungen schon vollzogen wurden (ident, expression, …)
\slides{V6_Codegen}{14}
\subsubsection{conditional statement}
\slides[.5]{V6_Codegen}{22}
Beispielcode:
\begin{lstlisting}[language=C]
// Procedure
0000:	1A	EntryProc							0028,0000,0004
// ?a
0007:	04	PushAdrVarMain				0000
000A:	09	GetVal
// if a>10 then
000B:	01	PushValVarMain				0000
000E:	06	PushConst							0000
0011:	13	CmpGreaterThen
0012:	19	JmpNot								0004
// !00
0015:	06	PushConst							0001
0018:	08	PutVal
// if a<=10 then
0019:	01	PushValVarMain				0000
001C:	06	PushConst							0000
001F:	14	CmpLessEqual
0020:	19	JmpNot								0004
// !1
0023:	06	PushConst							0002
0026:	08	PutVal

0027:	17	ReturnProc
Const	0000:0010
Const	0001:0100
Const	0002:0001
\end{lstlisting}
\subsubsection{loop statement}
\slides[.5]{V6_Codegen}{23}
\subsubsection{procedure call}
\slides[.5]{V6_Codegen}{24}
%s8 Prozeduraufruf:
%\begin{itemize}
%\item Bezeichner global suchen
%\begin{itemize}
%\item Nicht gefunden $\to$ Fehlerbehandlung
%\item Gefunden $\to$ ok.
%\end{itemize}
%\item Bezeichnet der Bezeichner eine Prozedur?
%\begin{itemize}
%\item Nein, eine Konstante oder Variable $\to$ Fehlebehandlung
%\item Ja $\to$ ok
%\end{itemize}
%\item Codegenerierung call procedurenumber
%\end{itemize}
\subsubsection{Eingabe/Ausgabe}
\slides{V6_Codegen}{15}
