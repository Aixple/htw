\section{Mathematische Notation}
\slides{WS2016_Kap1_Kap2}{4}
\subsection{Neuronale Bezeichner}
\slides{WS2016_Kap1_Kap2}{5}
$w_{ij}$: Sender $i$, Empfänger $j$
\slides{WS2016_Kap1_Kap2}{6}
$d$: Abstandsmaß (bspw. Euklidisch oder Hamming-Distanz)
\section{Neuronenmodell}
\slides{WS2016_Kap1_Kap2}{7}
\slides{WS2016_Kap1_Kap2}{8}
\unimptnt{
\section{Netzwerk- \& Lernparadigmen}
\slides{WS2016_Kap1_Kap2}{9}
}
\section{Lernparadigmen}
\subsection{Hebb'sche  Lernregeln}
Korrelationslernen, unüberwacht
\slides{WS2016_Kap1_Kap2}{10}
\begin{center}
\begin{tikzpicture}
\draw (0,0) node{$i$} circle (0.5) node[above = .5]{Sendeneuron};
\draw (4,0) node{$j$} circle (0.5) node[above = .5]{Empfangsneuron};
\draw [-hooks] (0.5,0) -- (2,0);
\draw [hooks-](2,0) -- (3.5,0);
\draw [decorate, decoration={brace, amplitude=5pt}](2.5,-0.5) -- (1.5,-0.5);
\node at (2,-1) {$w_{ij}$};
\end{tikzpicture}
\end{center}
\paragraph{Frage:} Wann soll sich der Gewichtswert der Verbindungssynapse $w_{ij}$ ändern?
\paragraph{Idee:} Gewicht proportional zur Aktivität prä- und postsynaptische Seite erhöhen.\bigskip\\
Als binäre Wertetabelle:\\
\begin{tabular}{c | c c}
$y_i / y_j$ & 0 & 1\\
\hline
0 & 0 & 0\\
1 & 0 & 1
\end{tabular}\\
$\Delta w_{ij} = \mu \cdot y_i \cdot y_j$ \qquad ($\mu$: Lernrate mit $0 < \mu \leq 1$)\\
$\to$ Gewicht der Synapse wird verstärkt, wenn \emph{beide} Neuronen aktiv sind.\\
$\Rightarrow$ Korrelations-LR\\
\emph{Nachteil:} keine Änderung $\Delta w_{ij}$ in negative Richtung möglich!

\unimptnt{
\subsection{SOFM}
Kohnen-Algorithmus, unüberwacht
\slides{WS2016_Kap1_Kap2}{11}
}

\subsection{Error-Driven}
überwacht
\slides{WS2016_Kap1_Kap2}{12}









