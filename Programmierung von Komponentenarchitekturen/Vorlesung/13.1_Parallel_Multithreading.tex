\section{Multithreading}
\lecdate{12.05.2017}
\subsection{Verschiedene Methoden zur Parallelisierung:}

\begin{itemize}
\item \emph{Hochleistungsrechner}: Verteilter Speicher, Kommunikation zwischen Rechnern durch Nachrichten (Message Passing, MPI)
\item \emph{Einzelrechner/Smartphones}: gemeinsamer Speicher, Aufteilung des Programmablauf in Threads
\end{itemize}
Wir beschränken uns auf letzteres!

\subsection{Thread}
\begin{itemize}
\item Anwendung
\begin{itemize}
\item Prozess (Task)
\begin{itemize}
\item Thread 1
\item Thread 2
\item Thread 3
\end{itemize}
\end{itemize}
\end{itemize}
Thread (Faden) = Teilprozess
\begin{itemize}
\item zum Start des Programms: 1 Thread (Masterthread)
\item serielle Programmierung: es gibt nur diesen einen Thread
\item Multithreading: es werden neue Threads gebildet die parallel abgearbeitet werden können
\end{itemize}

\begin{center}
\includegraphics[width=.9\textwidth]{Vorlesung/a131ParallelMultithreading-img004.png} 
\end{center}
veranschaulicht:
\begin{itemize}[label=$\to$]
\item fork (verzweigen)
\item join (zusammenführen)
\end{itemize}

\begin{itemize}
\item In einem Prozessor kann zu jedem Zeitpunkt nur 1 Thread bearbeitet werden
\item sind mehrere Threads $\to$ Warteschlange, Austausch in kurzen Zeitintervallen (Standard: 20ms), Scheduling durch das System
\item Threads haben verschiedene Prioritätsstufen (Windows: 1..31) die beim Scheduling berücksichtigt werden
\end{itemize}

\subsection{Threadzustände}

\begin{itemize}
\item \emph{laufend} (running) - läuft gerade in der CPU
\item \emph{bereit} (ready) - wartet in der Warteschlange auf CPU-Zeit
\item \emph{wartend} (waiting) - im Leerlauf, benötigt im Moment keine CPU-Zeit, kann erst durch einen anderen Thread wieder in den Zustand \emph{bereit}
\end{itemize}

\subsection{Threads starten}

Eine gegebene Methode TestMethod, soll nebenläufig in einem neuen Thread abgearbeitet werden:

\begin{lstlisting}[language={[Sharp]C}]
using System.Threading;
// ...
ThreadStart del = new ThreadStart(TestMethod);
Thread thread = new Thread(del);
thread.Start();
\end{lstlisting}

Alternativ: Thread starten mit Lambda-Expressions

\begin{lstlisting}[language={[Sharp]C}]
Thread thread = new Thread( () => TestMethod() );
Thread thread = new Thread( () => TestMethod(param1, param2, ...) );
// Achtung: Parameter werden erst zum Start des Threads ausgewertet und dürfen sich bis dahin nicht ändern 
\end{lstlisting}

\subsection{Beispiel}
\begin{lstlisting}[language={[Sharp]C}]
using System.Threading;

class Program {
	static void Main(string[] args) {
		Thread thread = new Thread(new ThreadStart(TestMethod));
		thread.Start();
		for(int i = 0; i <= 100; i++) {
			for(int k = 1; k <= 5; k++)
				Console.Write(".");
			Console.WriteLine("Primär-Thread " + i);
		}
	}

	public static void TestMethod() {
		for(int i = 0; i <= 100; i++) {
			for(int k = 1; k <= 5; k++)
			Console.Write("X");
			Console.WriteLine("Sekundär-Thread " + i);
		}
	}
}
\end{lstlisting}

Ausgabe (Beispiel):

\begin{lstlisting}[language={[Sharp]C}]
...XXXXXXSekundär-Thread 0
XX..Primär-Thread 0
.XXXSekundär-Thread 1
XX.. \\ usw.
\end{lstlisting}

\subsection{Weitere Methoden von Thread}

\begin{itemize}
\item \lstinline[language={[Sharp]C}]!Thread.Sleep(int time)! Thread schläft für time Millisekunden
\item \lstinline[language={[Sharp]C}]!thread.Abort()! Thread bricht ab, nicht mehr unterstützt in .Net Core
\item \lstinline[language={[Sharp]C}]!thread.Join()! Blockiert den aufrufenden Thread (!=thread), bis der Thread thread beendet ist
\end{itemize}