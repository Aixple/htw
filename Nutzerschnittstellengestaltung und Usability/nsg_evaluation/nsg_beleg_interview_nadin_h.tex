\subsection*{Interview-Fragen}
\subsubsection*{Zur Person}
\begin{enumerate}
\item Was ist dein Name?
\interviewText{Nadin Hahn}
\item Wie alt bist du?
\interviewText{28}
\item Was arbeitest/studierst du?
\interviewText{Ich habe Architektur studiert und arbeite an der TU Dresden.}
\item Wie oft bzw. wie lange browst du Webseiten im Internet?
\interviewText{Browse täglich zwischen 4 und 5 Stunden. Allein schon für die Arbeit.}
\item Auf welchen Seiten bist du regelmäßig unterwegs?
\interviewText{
\begin{itemize}
\item Uni-Seiten (TUD)
\item Opal
\item Blogs
\item Youtube
\item Unterhaltung (9gag, …)
\item Zur Recherche (google, …)
\item Zum Shopping (amazon, …)
\end{itemize}
}
\item Wie schätzt Du Deine Kenntnisse im Umgang mit Webseiten ein?
\interviewText{Gut.}
\item Wofür nutzen Du das Internet (außer zum Browsen)?
\interviewText{Nichts (bewusst). Benutze keine Social-Media Kanäle.}
\item War Dir diese Webseite im vornherein bekannt? Wenn ja: Woher?
\interviewText{Nein.}
\end{enumerate}

\subsubsection*{Allgemeines}
\begin{enumerate}
\item Wie leicht fiel es Dir dich auf der Seite zu orientieren?
%               -        ~        +  von bis
\interviewScala{ }{ }{ }{ }{ }{6}{ }{schwer}{leicht}
Was viel dir besonders schwer/leicht:
\interviewText{
\begin{itemize}
\item Seite ist klar strukturiert in Menüs mit Kategorien. Die Zuordnung ist aber nicht immer logisch oder konsistent.
\item Da der Umfang der Seite recht klein ist, ist die Navigation einfach.
\end{itemize}}
\item Gab es Probleme beim Bedienen der Website?
%               -        ~        +  von bis
\interviewScala{ }{ }{ }{ }{ }{ }{7}{viele}{wenige}
Anmerkungen:
\interviewText{
\begin{itemize}
\item hat alles funktioniert, Verlinkungen waren erkennbar.
\item Gruppierung auf der rechten Seite (in den Übersichtsseiten) sind positiv aufgefallen.
\end{itemize}
}
\item Konntest Du die Website auch gut auf dem Handy bedienen?
%               -        ~        +  von bis
\interviewScala{ }{ }{ }{ }{ }{ }{7}{schwer}{leicht}
Anmerkungen:
\interviewText{Es war anders und wirkte insgesamt gleich, aber nicht besser/schlechter.

AWS hatte einen zu großen Button und die zu kleinen Textfelder.}
\item Würdest Du etwas an der Webseite verändern?
%               -        ~        +  von bis
\interviewScala{ }{ }{ }{3}{ }{ }{ }{viel}{wenig}
Wenn ja, was?
\interviewText{Das Design: Die Farben sind sehr trist, es fehlen Bilder. Überladen mit Text.}
\item Wie würdest du die Seite beschreiben? Was ist der Zweck der Seite?
\interviewText{Darstellung der Interessen des Autors. Geht Richtung Blog und scheint an Wissensweitergabe interessiert zu sein.}
\item Kennst Du Webseiten, die mit dieser vergleichbar sind? Wenn ja: Wie lauten diese? In wie fern sind sie besser/schlechter/ähnlich gestaltet?
\interviewText{Eher weniger. Wenn dann sind es richtige Blogs.}
\end{enumerate}

\subsubsection*{Zum Design}
\begin{enumerate}
\item Wie bewertest du das Design der Seite?
\begin{enumerate}
\item Der Gesamteindruck war…
%               -        ~        +  von bis
\interviewScala{ }{ }{ }{ }{5}{ }{ }{schlecht}{gut}
Anmerkungen:
\interviewText{Das Design war sehr trist. Dafür aber aufgeräumt.

Bei der Menge an Schrift, würde sich vielleicht eher eine serifenschrift zur besseren Lesbarkeit anbieten.

Foto auf der Start- und den Übersichtsseiten würde es etwas einladender wirken lassen. Fotos zwischen dem ganzen Text würde die Menge etwas entschärfen (im positiven Beispiel: Counter-Strike Source Seite).

Positiv ist auf jeden Fall, dass es nicht zu vollgepackt ist und auch „Mut zur Lücke“ bewiesen wurde. Bei Blogs o.ä. hat man sonst immer tausend Seitenleisten mit unnützem Inhalt (twitter-feed, stichwort-cloud) der den Eindruck stört.}
\item Die Farbwahl und -gestaltung war…
%               -        ~        +  von bis
\interviewScala{ }{2}{ }{ }{ }{ }{ }{schlecht}{gut}
Anmerkungen:
\interviewText{Die Farben verschwimmen, weil sie alle so eine geringe Sättigung haben. Das grau vom Hintergrund verschwimmt mit dem blau (eigentlich grün) von den Balken. Kräftige Farben wären vielleicht besser.}
\item Die Anordnung der Inhalte war…
%               -        ~        +  von bis
\interviewScala{ }{ }{ }{ }{5}{ }{ }{schlecht}{gut}
Anmerkungen:
\interviewText{Keine Anmerkungen. Wiederholung: „Mut zur Lücke“ $\to$ gut}
\item Die Präsentation der Webseite war im Bezug auf den Inhalt…
%               -        ~        +  von bis
\interviewScala{ }{ }{ }{ }{ }{6}{ }{unangemessen}{angemessen}
Anmerkungen:
\interviewText{Wie zuvor erwähnt: Eine Unterlegung mit Bildern würde die Seite auflockern. Vielleicht wären ein paar Icons zur besseren Übersicht hilfreich. Ansonsten sind die Bilder (und deren Anordnung) in Ordnung.}
\item Die Lesbarkeit der Webseite war…
%               -        ~        +  von bis
\interviewScala{ }{ }{ }{ }{ }{ }{7}{schlecht}{gut}
Anmerkungen:
\interviewText{Keine Anmerkungen.}
\end{enumerate}
\end{enumerate}

\subsubsection*{Automatisches Wichtel System}
\begin{enumerate}
\item Sind Dir bei der Bedienung des Automatisierten Wichtel Systems Besonderheiten aufgefallen, die Du noch nicht erwähnt hast?
\interviewText{Siehe oben.}
\end{enumerate}

\subsubsection*{Abschließende Fragen}
\begin{enumerate}
\item Würdest du die Website weiter empfehlen? Warum? Warum nicht?
\interviewText{Das AWS ist ok, die Inhalte scheinen interessant.

Beim AWS bestehen bedenken bezüglich der Sicherheit (was passiert mit den Daten/Email-Adressen). Hier wäre ein Disclaimer, dass nichts gespeichert wird angemessen.}
\item Würdest du die Funktionen der Seite erneut nutzen?
\interviewText{Bei AWS schon, andere Inhalte sind nicht relevant.}
\end{enumerate}
