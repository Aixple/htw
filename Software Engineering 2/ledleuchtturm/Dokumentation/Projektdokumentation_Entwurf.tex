\subsection{Aufgabe}
Die Aufgabe des Entwerfenden war die Überwachung des Entwurfsprozess zur Planung einer Software für die gegebene Aufgabenstellung. Im Zentrum stand die Entwicklung eines Entwurfes, der die Komplexität während der Entwicklung handhabbar und überschaubar hält. Mit dem Pflichtenheft als Grundlage wurde der exakte Programmablauf, mögliche auftretende Probleme und Schwachstellen analysiert. Unter Absprache mit dem Implementierer, Datenbankbeauftragten und Projektleiter wurden Diagramme, Schaltpläne, Vorgaben und Dateien erstellt. Diese sollten für die folgende Implementation als wichtige Grundlage und Leitfaden dienen.

\subsection{Ablauf}
Zu Beginn wurden alle Anforderungen erneut analysiert. Danach wurde ein Aktivitätsdiagramm, welches den grundsätzlichen Programmablauf verdeutlichte, erstellt. Dieses diente als eine angemessen Grundlage, um den Implementierer,  Projektleiter und Datenbankbeauftragten die grundsätzliche Idee zu veranschaulichen und Unklarheiten und Anregungen zu klären: Der  Sachverhalt konnte wesentlich besser analysiert werden. Mit voranschreitenden Analysen wurden Funktionen hinzugefügt, entfernt oder modifiziert. Jede Funktion in dem Aktivitätsdiagramm wurde noch einmal im Zusammenhang mit Ein- und Ausgangsdaten, Meldungen und einer Kennung in zwei Tabellen dargestellt. Dies sollte die Verständlichkeit und die Aufgabe der Funktionen besser beschreiben. Nachdem die Arbeiten mit dem Aktivitätsdiagramm abgeschlossen waren, wurden die einzelnen Zustände, auf die die Software reagiert, analysiert. Einige Unklarheiten konnten in Rücksprache mit dem Kunden geklärt und das Zustandsdiagramm mit fünf verschiedenen Zuständen erstellt werden. Eine Aufgabe der Software war es, eine Konfigurationsdatei einzulesen, die bestimmte Informationen enthält. Für diese musste ein Schema erarbeitet werden. Es entstanden einige Fragen: Ist der inhaltliche Aufbau verständlich? Ist der Name der Konfigurationsdatei eindeutig und steht dieser auch in Verbindung zu der Software? Da auf die inhaltliche Korrektheit geachtet werden musste, wurde wieder ein Treffen mit dem Implementierer vereinbart. Zuletzt wurde ein konzeptioneller Schaltplan entworfen. Hier konnte auf Tests und Ideen aufgebaut werden, die von dem Hardwarebeauftragen durchgeführt wurden.

\subsection{Herausforderungen}
Der Entwurf spielt eine entscheidende Rolle in Bezug auf eine Software die allen  Anforderungen, Rahmenbedingungen und Kundenwünschen gerecht werden soll und in einem gegebenen Zeitraum zu realisieren ist. Deshalb war es wichtig, die Kommunikation zum Kunden und zu allen Gruppenmitgliedern, die bei dem Entwurfsprozess eine entscheidende Rolle spielten, aufrecht zu halten. Der klare Vorteil war, dass Missstände mit dem Kunden ausgeräumt und der Entwurf auf den Implementierer wesentlich besser zugeschnitten werden konnte.

Da der Zeitrahmen für den Entwurf sehr knapp gewählt war, stellte sich die Frage der Notwendigkeit von einigen Diagrammen: Ist der Aufwand ein Diagramm zu konzeptionieren vielleicht größer als der tatsächlich Nutzen? Werden Sachverhalte durch ein Diagramm vielleicht unnötig kompliziert gemacht? Letztendlich wurde sich auf einige wenige Diagrammtypen beschränkt, um einen guten Überblick zu bieten und den Sachverhalt trotzdem einfach zu halten.
%Um dem engen Zeitplan gerecht zu werden, hinterfragte ich ebenso die Notwendigkeit einiger Diagramme. Hier ging es um die Fragen, ob nicht der Aufwand, bestimmte Diagramme zu entwickeln, größer wäre als der tatsächliche Nutzen oder ob man so vielleicht Missstände vermeiden könnte, weil man das Problem durch ein Diagramm nur unnötig komplizierter machte. 

\subsection{Persönliches Fazit}
Als Entwerfender habe ich gelernt, dass es wichtig ist sich viel Zeit für den Entwurf zu nehmen. Durch die Analyse deckt man Schwachstellen auf und kann sie beseitigen. Auch wenn manche Probleme und Situationen trivial erscheinen, ist es wesentlich zielführender diese anzusprechen und drauf hinzuweisen. So werden Missstände in der Entwicklung vermieden und ein reibungsloser Ablauf garantiert. Es ist also besser einen trivialen Sachverhalt zu schildern, statt anzunehmen, dass Implementierendem und Entwerfendem das gleiche Verständnis dieses Sachverhaltes zu Grunde liegen. Ein Entwurf ist die Grundlage für das erfolgreiche Entwickeln einer Software. Die Verantwortung eines Entwerfenden empfinde ich rückblickend als sehr wichtig. 