\section*{Struktur der Protokolle}
\begin{tabular}[t]{p{.25\linewidth} p{.35\linewidth}}
\emph{Beginn:}				& Beginn des Treffens\\
\emph{Ende:}					& Ende des Treffen\\
\emph{Ort:}						& Ort des Treffens\\
\emph{Anwesend:}	& alle Anwesenden\\
\emph{Abwesend:}		 & alle Abwesenden\footnote{wenn Treffen ein Gesamtgruppentreffen/Kundengespräch war}\\
\emph{Protokollant:}& der(die) Protokollant(en)
\end{tabular}
\paragraph*{Tagesordnungspunkte}
Hier werden zur Übersicht alle Punkte gelistet, die an dem Treffen besprochen werden sollen.
\begin{itemize}
\item TOP 1
\item TOP 2
\end{itemize}

\subsection*{TOP 1}
Hier werden die Tagesordnungspunkte abgearbeitet. Diese Sektionen haben in der Regel gleiche oder ähnliche Überschriften wie die zugehörigen TOPs. Gegebenenfalls, je nach Dynamik des Treffens, werden auch mehrere TOPs in einer Sektion zusammengefasst oder auf mehrere aufgeteilt.
%\subsection*{TOP 2}

\subsection*{Nächste Schritte}
Hier werden tabellarisch die nächsten Schritte mit einer entsprechenden Beschreibung (Betreff) und der Frist festgehalten. Weiterhin werden Verantwortliche (in der Regel der Gruppenrolle entsprechend) und weitere Involvierte zugewiesen (wenn Aufgaben delegiert werden oder Abhängigkeiten zwischen Aufgaben bestehen).

Bei einigen Treffen gibt es keine nächsten Schritte festzuhalten, da alle nächsten Schritte bereits aus vergangenen Treffen klar sind oder sich aus dem Projektplan ergeben.\\
\begin{tabular}{L{.249} L{.2} L{.25} L{.25}}
Betreff & Frist & Verantwortliche(r) & Involviert\\\hline
-- & -- & -- & --\\
\end{tabular}