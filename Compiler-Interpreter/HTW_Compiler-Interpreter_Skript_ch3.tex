\section{Lexikalische Analyse}
\slides{V3_Lexer}{1}

\subsection{Einordnung in den Compiler}
\slides{V3_Lexer}{2}

\subsection{Funktionsweise}
\slides{V3_Lexer}{3}

\section{Grammatik der PL/0 Token}
\slides{V3_Lexer}{4}
Buzi: Buchstabe/Ziffer; Bu: Buchstabe
\subsection{Regeln der Grammatik der Morpheme}
\slides{V3_Lexer}{5}
\subsubsection*{Syntaxgraph}
\label{pl0syntaxgraph}
\slides[.4]{V3_Lexer}{6}

\section{Endlicher deterministischer Automat}
\slides{V3_Lexer}{7}
\slides{V3_Lexer}{8}
\slides{V3_Lexer}{9}

\subsection{Automatentabelle}
\slides{V3_Lexer}{10}
\subsubsection{Zeichenklassenvektor}
\slides{V3_Lexer}{11}
\slides[.4]{V3_Lexer}{12}
$\to$Jedem ASCII-Zeichen wird eine Spalte („Handlung“) in der Automatentabelle zugewiesen (mit Hilfe des Syntaxgraphs in \autoref{pl0syntaxgraph}):
\slides[.4]{V3_Lexer}{15}
s: Schreibe des betrachteten Symbols in die Ausgabe\\
g/sg: Schreibe Großbuchstabe des betrachteten Buchstaben\\
l: Lese nächstes Symbol\\
b: Beende das Erkennen dieses Tokens

\subsubsection{Aktionen}
\slides{V3_Lexer}{16}

\subsubsection{Fazit}
\slides{V3_Lexer}{17}

\subsubsection*{Variablen Beispiellexer}
\slides{V3_Lexer}{18}

\section{Implementierung des Lexers}
\slides{V3_Lexer}{19}
\subsection{Datentypen}
\slides{V3_Lexer}{20}
\slides[.4]{V3_Lexer}{21}
%\slides{V3_Lexer}{22}
%\slides{V3_Lexer}{23}
\begin{lstlisting}[language=C]
/* Schalt und Ausgabefunktionen des Automaten */
static void fl (void);
static void fb (void);
static void fgl (void);
static void fsl (void);
static void fslb(void);
typedef void (*FX)(void);
static FX vfx[]={fl,fb,fgl,fsl,fslb};

/* Funktionsindex *0x10, bzw. *16 */
typedef enum T_Fx{
	ifl=0x0,
	ifb=0x10,
	ifgl=0x20,
	ifsl=0x30,
	ifslb=0x40
}tFx;

/* Morphemcodes */
typedef enum T_MC{
	mcEmpty, mcSymb, mcNum, mcIdent
}tMC;
typedef enum T_ZS{
	zNIL,
	zERG=128,zLE,zGE,
	zBGN,zCLL,zCST,zDO,zEND,zIF,zODD,zPRC,zTHN,zVAR,zWHL
}tZS;

typedef struct{
	tMC MC; /* Morphemcode */
	int PosLine; /* Zeile */
	int PosCol; /* Spalte */
	union VAL	{
		long Num;
		char*pStr;
		int Symb;
	}Val;
	int MLen; /* Morphemlnge*/
}tMorph;

int initLex(char* fname);
tMorph* Lex(void);
\end{lstlisting}
\slides[.4]{V3_Lexer}{24}
\slides[.4]{V3_Lexer}{25}
\subsection{Initialisierung}
\begin{lstlisting}[language=C]
static FILE *pIF;
static tMorph MorphInit;
extern tMorph Morph;
static int X;
static int Z;
static char vBuf[1024+1];
static char* pBuf;
static int line,col;
static int Ende; // Entfällt, wenn Zustand 9 -> Ende
/* Initialisierung der lexiaklischen Analyse */
int initLex(char* fname){
	char vName[128+1];
	strcpy(vName,fname);
	if (strstr(vName,".pl0")==NULL)
		strcat(vName,".pl0");
	pIF=fopen(vName,"r+t");
	if (pIF!=NULL){
		X=fgetc(pIF); /* Lesen des ersten Zeichens */
		return OK;
	}
	return FAIL;
}
\end{lstlisting}

\subsection{Algorithmus}
\begin{lstlisting}[language=C]
tMorph* Lex(void){
	Z=0; // Anfangszustand, wenn nicht aus Tabelle
	int zx;
	Morph=MorphInit;
	Morph.PosLine=line;
	Morph.PosCol =col;
	pBuf=vBuf;
	End=0;
	do{
		zx= vSMatrix[Z][vZKl[X&0x7f]]&0xF;
		vfx[(vSMatrix[Z][vZKl[X&0x7f]])>>4]();
		Z=zx;
	}while (End==0); // (Z!=zEnd) // zEnd:9
	return &Morph;
}
\end{lstlisting}

Hinweis Praktikum: Klassifizierung soll über Zeichenklassenvektor passieren!

\subsection{Funktionen}
\lecdate{01.11.2017}
\begin{lstlisting}[language=C]
/* lesen */
static void fl (void){
	X=fgetc(pIF);
	if (X=='\n') 
		line++,
		col=0;
	else 
		col++;
}
/* schreiben als Großbuchstabe, lesen */
static void fgl (void){
	*pBuf=(char)toupper(X);	// oder *Buf=(char)X&0xdf;
	*(++pBuf)=0;
	fl();
}
/* schreiben, lesen */
static void fsl (void){
	*pBuf=(char)X;
	*(++pBuf)=0;
	fl();
}
/*schreiben, lesen, beenden */
static void fslb(void){
	fsl();
	fb();
}
\end{lstlisting}

\subsubsection{Beenden}
\begin{lstlisting}[language=C]
static void fb (){
	int i,j;
	switch (Z){
		/* Symbol */
		case 3: // :
		case 4: // <
		case 5: // >
		case 0:
			Morph.Val.Symb=vBuf[0];
			Morph.MC =mcSymb;
			break;
		/* Zahl */
		case 1:
			Morph.Val.Num=atol(vBuf);
			Morph.MC =mcNum;
			break;
		/* Ergibtzeichen */
		case 6:
			Morph.Val.Symb=(long)zErg;
			Morph.MC =mcSymb;
			break;
		/* KleinerGleich */
		case 7:
			Morph.Val.Symb=(long)zle;
			Morph.MC =mcSymb;
			break;
		/* GroesserGleich */
		case 8:
			Morph.Val.Symb=(long)zge;
			Morph.MC =mcSymb;
			break;
	}
	Ende=1; // entfällt bei Variante mit Zustand zEnd
}
\end{lstlisting}
\slides[.5]{V3_Lexer}{30}

\section{Schlüsselworterkennung}
\slides{V3_Lexer}{31}
\slides{V3_Lexer}{32}

\subsection*{Implementationsvarianten}
\slides{V3_Lexer}{33}
\subsection{Binäre Suche}
\begin{lstlisting}[language=C]
int main(int argc, char*argv[]){
	/* sortierter Vektor der Keywords: */
	const char* Keyw[]= {"BEGIN","CALL","CONST","DO","END","IF","ODD","PROCEDURE","THEN","VAR","WHILE"};
	int n=sizeof Keyw/sizeof(char*);
	int i;
	printf("Anz:%d\n",n);
	i=binary_search(Keyw,n,argv[1]);
	if (i>=0)
		printf("Ergebnis: %d %s\n",i, Keyw[i]);
	else 
		printf("Ergebnis: not found\n");
	return 0;
}
\end{lstlisting}
\subsubsection*{Nicht-Rekursiv}
\begin{lstlisting}[language=C]
int binary_search(
	const char** M,	// stock
	int n,					// number
	const char* X){	// needle
	unsigned mitte;
	unsigned links = 0; /* man beginne beim kleinsten Index */
	unsigned rechts = n 1;
	/* Arrays sind 0basiert
	*/
	int ret=1;
	int bool;
	do{
		if (n<=0)
			break;
		mitte = links + ((rechts links)/ 2); // Bereich halbieren
		if (rechts < links)
			break; // alles wurde durchsucht, X nicht gefunden
		bool=strcmp(M[mitte],X);
		if (bool==0) // Element X gefunden?
			ret=mitte; 
		else if (bool >0) // im linken Abschnitt weitersuchen
			rechts = mitte; 
		else  // im rechten Abschnitt weitersuchen
			links = mitte + 1;
		n=(n)/2;
	} while (bool!=0);
	return ret;
}
\end{lstlisting}
\subsubsection*{Rekursiv}
\begin{lstlisting}[language=C]
int count=0;
int binary_search(
	const char** M,	// stock
	int n,					// number
	const char* X,	// needle
	int offset){		// offset fuer return
	count ++;
	unsigned mitte, index, ret=1;
	unsigned links = 0; // man beginne beim kleinsten Index
	unsigned rechts = n 1;
	// Arrays sind 0basiert
	if (M == NULL || n <= 0) // Bereichsüberprüfung
		printf("out of Range n:%d M[0]:%s m[%d]:%s\n",n, M[0], n1, n>0? M[n1]:"OutOfRange");
	else{
		int bool;
		mitte = links + ((rechts links) / 2); // Bereich halbieren
		if (rechts >= links){
			bool=strcmp(M[mitte],X);
			if (bool==0) // gefunden
				ret= mitte+offset; 
			else if (bool >0) // links weiter
				ret= binary_search(M, n/2, X,offset); 
			else // rechts weiter
				ret= binary_search(M+mitte+1, n/2, X, offset+=mitte+1);
		}
	}
	return ret;
}
\end{lstlisting}

\subsubsection{Tabelle}
\slides[.5]{V3_Lexer}{37}
\subsubsection{Zeichenklassen für Buchstaben der Schlüsselwörter}
\slides[.5]{V3_Lexer}{38}
\subsubsection{Extra Zeichenklasse für 1. Buchstaben von Schlüsselwörtern}
\slides[.5]{V3_Lexer}{39}
\subsection{Automatentabelle}
mit Schlüsselwörtern \lstinline`BEGIN`, \lstinline`CALL`, \lstinline`CONST`
\slides[.5]{V3_Lexer}{40}

\subsection{Schlüsselworttabelle mit einfacher Hashtechnik}
\slides[.5]{V3_Lexer}{41}

\begin{lstlisting}[language=C]
/* Bezeichner/Wortsymbol */
case 2:
	i=vBuf[0]-'A';
	j=strlen(vBuf)-2;
	if (j>=0)
		if (mSclW[i][j].pKeyWord)
			if (strcmp(vBuf+1,mSclW[i][j].pKeyWord)==0){
				Morph.MC =mcSymb;
				Morph.Val.Symb=mSclW[i][j].KWCode;
				break;
			}
	Morph.Val.pStr=vBuf;
	Morph.MC =mcIdent;
	break;
\end{lstlisting}
\begin{lstlisting}[language=C]	
typedef enum T_ZS{
	zNIL,
	zErg=128,zle,zge,
	zBGN,zCLL,zCST,zDO,zEND,zIF,zODD,zPRC,zTHN,zVAR,zWHL
}tZS;
\end{lstlisting}


