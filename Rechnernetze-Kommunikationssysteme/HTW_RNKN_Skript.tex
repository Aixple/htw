\newcommand{\customDir}{../}
\RequirePackage{ifthen,xifthen}

% Input inkl. Umlaute, Silbentrennung
\RequirePackage[T1]{fontenc}
\RequirePackage[utf8]{inputenc}

% Arbeitsordner (in Abhängigkeit vom Master) Standard: .LateX_master Ordner liegt im Eltern-Ordner
\providecommand{\customDir}{../}
\newcommand{\setCustomDir}[1]{\renewcommand{\customDir}{#1}}
%%% alle Optionen:
% Doppelseitig (mit Rand an der Innenseite)
\newboolean{twosided}
\setboolean{twosided}{false}
% Eigene Dokument-Klasse (alle KOMA möglich; cheatsheet für Spicker [3 Spalten pro Seite, alles kleiner])
\newcommand{\customDocumentClass}{scrreprt}
\newcommand{\setCustomDocumentClass}[1]{\renewcommand{\customDocumentClass}{#1}}
% Unterscheidung verschiedener Designs: htw, fjs
\newcommand{\customDesign}{htw}
\newcommand{\setCustomDesign}[1]{\renewcommand{\customDesign}{#1}}
% Dokumenten Metadaten
\newcommand{\customTitle}{}
\newcommand{\setCustomTitle}[1]{\renewcommand{\customTitle}{#1}}
\newcommand{\customSubtitle}{}
\newcommand{\setCustomSubtitle}[1]{\renewcommand{\customSubtitle}{#1}}
\newcommand{\customAuthor}{}
\newcommand{\setCustomAuthor}[1]{\renewcommand{\customAuthor}{#1}}
%	Notiz auf der Titelseite (A: vor Autor, B: nach Autor)
\newcommand{\customNoteA}{}
\newcommand{\setCustomNoteA}[1]{\renewcommand{\customNoteA}{#1}}
\newcommand{\customNoteB}{}
\newcommand{\setCustomNoteB}[1]{\renewcommand{\customNoteB}{#1}}
% Format der Signatur in Fußzeile:
\newcommand{\customSignature}{\ifthenelse{\equal{\customAuthor}{}} {} {\footnotesize{\textcolor{darkgray}{Mitschrift von\\ \customAuthor}}}}
\newcommand{\setCustomSignature}[1]{\renewcommand{\customSignature}{#1}}
% Format des Autors auf dem Titelblatt:
\newcommand{\customTitleAuthor}{\textcolor{darkgray}{Mitschrift von \customAuthor}}
\newcommand{\setCustomTitleAuthor}[1]{\renewcommand{\customTitleAuthor}{#1}}
% Standard Sprache
\newcommand{\customDefaultLanguage}[1]{}
\newcommand{\setCustomDefaultLanguage}[1]{\renewcommand{\customDefaultLanguage}{#1}}
% Folien-Pfad (inkl. Dateiname ohne Endung und ggf. ohne Nummerierung)
\newcommand{\customSlidePath}{}
\newcommand{\setCustomSlidePath}[1]{\renewcommand{\customSlidePath}{#1}}
% Folien Eigenschaften
\newcommand{\customSlideScale}{0.5}
\newcommand{\setCustomSlideScale}[1]{\renewcommand{\customSlideScale}{#1}}
\newcommand{\customSlideHeight}{9.63cm}
\newcommand{\setCustomSlideHeight}[1]{\renewcommand{\customSlideHeight}{#1}}
\newcommand{\customSlideWidth}{12.8cm}
\newcommand{\setCustomSlideWidth}[1]{\renewcommand{\customSlideWidth}{#1}}

%\setboolean{twosided}{true}
%\setCustomDocumentClass{scrartcl}
%\setCustomDesign{htw}
\setCustomSlidePath{Vorlesung/rn-}
\setCustomSlideScale{1}

\setCustomTitle{\texorpdfstring{Rechnernetze /\\ Kommunikationssysteme}{Rechnernetze / Kommunikationssysteme}}
\setCustomSubtitle{Vorlesungsskript}
\setCustomAuthor{Falk-Jonatan Strube}
%\setCustomNoteA{TitlepageNoteBeforeAuthor}
\setCustomNoteB{Vorlesung von }

%\setcustomSignature{\footnotesize{\textcolor{darkgray}{Mitschrift von\\ \customAuthor}}	% Formatierung der Signatur in der Fußzeile
%\setcustomTitleAuthor{\textcolor{darkgray}{Mitschrift von #1}}	% Formatierung des Autors auf dem Titelblatt

%-- Prüfen, ob Beamer
\ifthenelse{\equal{\customDocumentClass}{beamer}}{
%%% TODO: andere Layouts für Beamer außer HTW
	\documentclass[ignorenonframetext, 11pt, table]{beamer}
	
	\usenavigationsymbolstemplate{}
	\setbeamercolor{author in head/foot}{fg=black}
	\setbeamercolor{title}{fg=black}
	\setbeamercolor{bibliography entry author}{fg=htworange!70}
	%\setbeamercolor{bibliography entry title}{fg=blue} 
	\setbeamercolor{bibliography entry location}{fg=htworange!60} 
	\setbeamercolor{bibliography entry note}{fg=htworange!60}  
	
	\setbeamertemplate{itemize item}{\color{black}$\bullet$}
	\setbeamertemplate{itemize subitem}{\color{black}--}
	\setbeamertemplate{itemize subsubitem}{\color{black}$\bullet$}
	\makeatother
	\setbeamertemplate{footline}
	{
	\leavevmode
	\def\arraystretch{1.2}
	\arrayrulecolor{gray}
	\begin{tabular}{ p{0.167\textwidth} | p{0.491\textwidth} | p{0.089\textwidth} | p{0.103\textwidth}}
	\hline
	\strut\insertshortauthor & \insertshorttitle & Slide \insertframenumber{}% / \inserttotalframenumber{}
	 & May 4, 2016\\
	\end{tabular}
	}
	\setbeamertemplate{headline}
	{
	\leavevmode
	\setlength{\arrayrulewidth}{1pt}
	\hspace*{2em}	
	\begin{tabular}{p{0.63\textwidth}}
	\rule{0pt}{3em}\normalsize{\textbf{\insertsection\strut}}\\
	\arrayrulecolor{htworange}
	\hline
	\end{tabular}
	\begin{tabular}{l}
	\rule{0pt}{4em}\includegraphics[width=3.25cm]{\customDir .LaTeX_master/HTW_GESAMTLOGO_CMYK.eps}\\
	\end{tabular}
	}
	\makeatletter	
}{	
	%-- Für Spicker einiges anders:
	\ifthenelse{\equal{\customDocumentClass}{cheatsheet}}{
		\documentclass[a4paper,10pt,landscape]{scrartcl}
		\usepackage{geometry}
		\geometry{top=2mm, bottom=2mm, headsep=0mm, footskip=0mm, left=2mm, right=2mm}
		
		% Für Spicker \spsection für Section, zur Strukturierung \HRule oder \HDRule Linie einsetzen
		\usepackage{multicol}
		\newcommand{\spsection}[1]{\textbf{#1}}	% Platzsparende "section" für Spicker
	}{	%-- Ende Spicker-Unterscheidung-if
		%-- Unterscheidung Doppelseitig
		\ifthenelse{\boolean{twosided}}{
			\documentclass[a4paper,11pt, footheight=26pt,twoside]{\customDocumentClass}
			\usepackage[head=23pt]{geometry}	% head=23pt umgeht Fehlerwarnung, dafür größeres "top" in geometry
			\geometry{top=30mm, bottom=22mm, headsep=10mm, footskip=12mm, inner=27mm, outer=13mm}
		}{
			\documentclass[a4paper,11pt, footheight=26pt]{\customDocumentClass}
			\usepackage[head=23pt]{geometry}	% head=23pt umgeht Fehlerwarnung, dafür größeres "top" in geometry
			\geometry{top=30mm, bottom=22mm, headsep=10mm, footskip=12mm, left=20mm, right=20mm}
		}
		%-- Nummerierung bis Subsubsection für Report
		\ifthenelse{\equal{\customDocumentClass}{report} \OR \equal{\customDocumentClass}{scrreprt}}{
			\setcounter{secnumdepth}{3}	% zählt auch subsubsection
			\setcounter{tocdepth}{3}	% Inhaltsverzeichnis bis in subsubsection
		}{}
	}%-- Ende Spicker-Unterscheidung-else
	
	\usepackage{scrlayer-scrpage}	% Kopf-/Fußzeile
	\renewcommand*{\thefootnote}{\fnsymbol{footnote}}	% Fußnoten-Symbole anstatt Zahlen
	\renewcommand*{\titlepagestyle}{empty} % Keine Seitennummer auf Titelseite
	\usepackage[perpage]{footmisc}	% Fußnotenzählung Seitenweit, nicht Dokumentenweit
}

% Input inkl. Umlaute, Silbentrennung
\RequirePackage[T1]{fontenc}
\RequirePackage[utf8]{inputenc}
\usepackage[english,ngerman]{babel}
\usepackage{csquotes}	% Anführungszeichen
\RequirePackage{marvosym}
\usepackage{eurosym}

% Style-Aufhübschung
\usepackage{soul, color}	% Kapitälchen, Unterstrichen, Durchgestrichen usw. im Text
%\usepackage{titleref}

% Mathe usw.
\usepackage{amssymb}
\usepackage{amsthm}
\ifthenelse{\equal{\customDocumentClass}{beamer}}{}{
\usepackage[fleqn,intlimits]{amsmath}	% fleqn: align-Umgebung rechtsbündig; intlimits: Integralgrenzen immer ober-/unterhalb
}
%\usepackage{mathtools} % u.a. schönere underbraces
\usepackage{xcolor}
\usepackage{esint}	% Schönere Integrale, \oiint vorhanden
\everymath=\expandafter{\the\everymath\displaystyle}	% Mathe Inhalte werden weniger verkleinert
\usepackage{wasysym}	% mehr Symbole, bspw \lightning
% Auch arcus-Hyperbolicus-Funktionen
\DeclareMathOperator{\arccot}{arccot}
\DeclareMathOperator{\arccosh}{arccosh}
\DeclareMathOperator{\arcsinh}{arcsinh}
\DeclareMathOperator{\arctanh}{arctanh}
\DeclareMathOperator{\arccoth}{arccoth} 
%\renewcommand{\int}{\int\limits}
%\usepackage{xfrac}	% mehr fracs: sfrac{}{}
\let\oldemptyset\emptyset	% schöneres emptyset
\let\emptyset\varnothing
%\RequirePackage{mathabx}	% mehr Symbole
\mathchardef\mhyphen="2D	% Hyphen in Math

% tikz usw.
\usepackage{tikz}
\usepackage{pgfplots}
\pgfplotsset{compat=1.11}	% Umgeht Fehlermeldung
\usetikzlibrary{graphs}
%\usetikzlibrary{through}	% ???
\usetikzlibrary{arrows}
\usetikzlibrary{arrows.meta}	% Pfeile verändern / vergrößern: \draw[-{>[scale=1.5]}] (-3,5) -> (-3,3);
\usetikzlibrary{automata,positioning} % Zeilenumbruch im Node node[align=center] {Text\\nächste Zeile} automata für Graphen
\usetikzlibrary{matrix}
\usetikzlibrary{patterns}	% Schraffierte Füllung
\usetikzlibrary{shapes.geometric}	% Polygon usw.
\tikzstyle{reverseclip}=[insert path={	% Inverser Clip \clip
	(current page.north east) --
	(current page.south east) --
	(current page.south west) --
	(current page.north west) --
	(current page.north east)}
% Nutzen: 
%\begin{tikzpicture}[remember picture]
%\begin{scope}
%\begin{pgfinterruptboundingbox}
%\draw [clip] DIE FLÄCHE, IN DER OBJEKT NICHT ERSCHEINEN SOLL [reverseclip];
%\end{pgfinterruptboundingbox}
%\draw DAS OBJEKT;
%\end{scope}
%\end{tikzpicture}
]	% Achtung: dafür muss doppelt kompliert werden!
\usepackage{graphpap}	% Grid für Graphen
\tikzset{every state/.style={inner sep=2pt, minimum size=2em}}
\usetikzlibrary{mindmap, backgrounds}
%\usepackage{tikz-uml}	% braucht Dateien: http://perso.ensta-paristech.fr/~kielbasi/tikzuml/

% Tabular
\usepackage{longtable}	% Große Tabellen über mehrere Seiten
\usepackage{multirow}	% Multirow/-column: \multirow{2[Anzahl der Zeilen]}{*[Format]}{Test[Inhalt]} oder \multicolumn{7[Anzahl der Reihen]}{|c|[Format]}{Test2[Inhalt]}
\renewcommand{\arraystretch}{1.3} % Tabellenlinien nicht zu dicht
\usepackage{colortbl}
\arrayrulecolor{gray}	% heller Tabellenlinien
\usepackage{array}	% für folgende 3 Zeilen (für Spalten fester breite mit entsprechender Ausrichtung):
\newcolumntype{L}[1]{>{\raggedright\let\newline\\\arraybackslash\hspace{0pt}}m{\dimexpr#1\columnwidth-2\tabcolsep-1.5\arrayrulewidth}}
\newcolumntype{C}[1]{>{\centering\let\newline\\\arraybackslash\hspace{0pt}}m{\dimexpr#1\columnwidth-2\tabcolsep-1.5\arrayrulewidth}}
\newcolumntype{R}[1]{>{\raggedleft\let\newline\\\arraybackslash\hspace{0pt}}m{\dimexpr#1\columnwidth-2\tabcolsep-1.5\arrayrulewidth}}
\usepackage{caption}	% Um auch unbeschriftete Captions mit \caption* zu machen

% Nützliches
\usepackage{verbatim}	% u.a. zum auskommentieren via \begin{comment} \end{comment}
\usepackage{tabto}	% Tabs: /tab zum nächsten Tab oder /tabto{.5 \CurrentLineWidth} zur Stelle in der Linie
\NumTabs{6}	% Anzahl von Tabs pro Zeile zum springen
\usepackage{listings} % Source-Code mit Tabs
\usepackage{lstautogobble} 
\ifthenelse{\equal{\customDocumentClass}{beamer}}{}{
\usepackage{enumitem}	% Anpassung der enumerates
%\setlist[enumerate,1]{label=(\arabic*)}	% global andere Enum-Items
\renewcommand{\labelitemiii}{$\scriptscriptstyle ^\blacklozenge$} % global andere 3. Item-Aufzählungszeichen
}
\newenvironment{anumerate}{\begin{enumerate}[label=(\alph*)]}{\end{enumerate}} % Alphabetische Aufzählung
\usepackage{letltxmacro} % neue Definiton von Grundbefehlen
% Nutzen:
%\LetLtxMacro{\oldemph}{\emph}
%\renewcommand{\emph}[1]{\oldemph{#1}}
\RequirePackage{xpatch}	% ua. Konkatenieren von Strings/Variablen (etoolbox)


% Einrichtung von lst
\lstset{
basicstyle=\ttfamily, 
%mathescape=true, 
%escapeinside=^^, 
autogobble, 
tabsize=2,
basicstyle=\footnotesize\sffamily\color{black},
frame=single,
rulecolor=\color{lightgray},
numbers=left,
numbersep=5pt,
numberstyle=\tiny\color{gray},
commentstyle=\color{gray},
keywordstyle=\color{green},
stringstyle=\color{orange},
morecomment=[l][\color{magenta}]{\#}
showspaces=false,
showstringspaces=false,
breaklines=true,
literate=%
    {Ö}{{\"O}}1
    {Ä}{{\"A}}1
    {Ü}{{\"U}}1
    {ß}{{\ss}}1
    {ü}{{\"u}}1
    {ä}{{\"a}}1
    {ö}{{\"o}}1
    {~}{{\textasciitilde}}1
}
\usepackage{scrhack} % Fehler umgehen
\def\ContinueLineNumber{\lstset{firstnumber=last}} % vor lstlisting. Zum wechsel zum nicht-kontinuierlichen muss wieder \StartLineAt1 eingegeben werden
\def\StartLineAt#1{\lstset{firstnumber=#1}} % vor lstlisting \StartLineAt30 eingeben, um bei Zeile 30 zu starten
\let\numberLineAt\StartLineAt

% BibTeX
\usepackage[backend=bibtex8, bibencoding=ascii,
%style=authortitle, citestyle=authortitle-ibid,
%doi=false,
%isbn=false,
%url=false
]{biblatex}	% BibTeX
\usepackage{makeidx}
%\makeglossary
%\makeindex

% Grafiken
\usepackage{graphicx}
\usepackage{epstopdf}	% eps-Vektorgrafiken einfügen
%\epstopdfsetup{outdir=\customDir}

% pdf-Setup
\usepackage{pdfpages}
\ifthenelse{\equal{\customDocumentClass}{beamer}}{}{
\usepackage[bookmarks,%
bookmarksopen=false,% Klappt die Bookmarks in Acrobat aus
colorlinks=true,%
linkcolor=black,%
citecolor=red,%
urlcolor=green,%
]{hyperref}
}

%-- Unterscheidung des Stils
\newcommand{\customLogo}{}
\newcommand{\customPreamble}{}
\ifthenelse{\equal{\customDesign}{htw}}{
	% HTW Corporate Design: Arial (Helvetica)
	\usepackage{helvet}
	\renewcommand{\familydefault}{\sfdefault}
	\renewcommand{\customLogo}{HTW-Logo}
	\renewcommand{\customPreamble}{HTW Dresden}
}{
% \renewcommand{\customLogo}{HTW-Logo.eps}
}

% Nach Dokumentenbeginn ausführen:
\AtBeginDocument{
	% Autor und Titel für pdf-Eigenschaften festlegen, falls noch nicht geschehen
	\providecommand{\pdfAuthor}{John Doe}
	\ifdefempty{\customAuthor} {} {\renewcommand{\pdfAuthor}{\customAuthor}}
	\providecommand{\pdfTitle}{}
	\providecommand{\pdfTitleA}{}
	\providecommand{\pdfTitleB}{}
	\providecommand{\pdfTitleC}{}	
	\ifdefempty{\pdfTitle}{
		\ifdefempty{\customPreamble} {} {\renewcommand{\pdfTitleA}{\customPreamble{} | }}
		\ifdefempty{\customTitle} {\renewcommand{\pdfTitleB}{No Title}} {\renewcommand{\pdfTitleB}{\customTitle}}
		\ifdefempty{\customSubtitle} {} {\renewcommand{\pdfTitleC}{ - \customSubtitle}}
	}{}
	
	\newcommand{\customLogoLocation}{\customDir .LaTeX_master/\customLogo}
	\hypersetup{
		pdfauthor={\pdfAuthor},
		pdftitle={\pdfTitleA\pdfTitleB\pdfTitleC},
	}
	\ifthenelse{\equal{\customDocumentClass}{beamer}}{
		\title{\customTitle}
		\author{\customAuthor}
	}{
		\automark[section]{section}
		\automark*[subsection]{subsection}
		\pagestyle{scrheadings}
		\ifthenelse{\equal{\customDocumentClass}{report} \OR \equal{\customDocumentClass}{scrreprt}}{
		\renewcommand*{\chapterpagestyle}{scrheadings}
		}{}
		%\renewcommand*{\titlepagestyle}{scrheadings}
		\ihead{\includegraphics[height=1.7em]{\customLogoLocation}}
		%\ohead{\truncate{5cm}{\customTitle}}
		\ohead{\customTitle}
		\cfoot{\pagemark}
		\ofoot{\customSignature}
		% Titelseite
		\title{
		\includegraphics[width=0.35\textwidth]{\customDir .LaTeX_master/\customLogo}\\\vspace{0.5em}
		\Huge\textbf{\customTitle}
		\ifdefempty{\customSubtitle} {} {\\\vspace*{0.7em}\Large \customSubtitle}
		\\\vspace*{5em}}
		\author{
		\ifdefempty{\customNoteA} {} {\customNoteA \vspace*{1em}}\\ 
		\ifdefempty{\customAuthor} {} {\customTitleAuthor}
		\ifdefempty{\customNoteB}{}{\vspace*{1em}\\\customNoteB}
		}
		
		\ifthenelse{\equal{\customDocumentClass}{cheatsheet}}{
			\pagestyle{empty}
			\setlist{nolistsep}
	%		\usepackage{parskip}	% Aufzählung Abstand
	%		\setlength{\parskip}{0em}
			\lstset{
	    belowcaptionskip=0pt,
	    belowskip=0pt,
	    aboveskip=0pt,
			tabsize=2,
			frame=none,
			numbers=none,
			showspaces=false,
			showstringspaces=false,
			breaklines=true,
			}
		}{}
	}
}

% Unterabschnitte
%\newtheorem{example}{Beispiel}%[section]
%\newtheorem{definition}{Definition}[section]
%\newtheorem{discussion}{Diskussion}[section]
%\newtheorem{remark}{Bemerkung}[section]
%\newtheorem{proof}{Beweis}[section]
%\newtheorem{notation}{Schreibweise}[section]
\RequirePackage{xcolor}
\RequirePackage{amsmath}

% Horizontale Linie:
\newcommand{\HRule}[1][\medskipamount]{\par
  \vspace*{\dimexpr-\parskip-\baselineskip+#1}
  \noindent\rule[0.2ex]{\linewidth}{0.2mm}\par
  \vspace*{\dimexpr-\parskip-.5\baselineskip+#1}}
% Gestrichelte horizontale Linie:
\RequirePackage{dashrule}
\newcommand{\HDRule}[1][\medskipamount]{\par
  \vspace*{\dimexpr-\parskip-\baselineskip+#1}
  \noindent\hdashrule[0.2ex]{\linewidth}{0.2mm}{1mm} \par
  \vspace*{\dimexpr-\parskip-.5\baselineskip+#1}}
% Mathe in Anführungszeichen:
\newsavebox{\mathbox}\newsavebox{\mathquote}
\makeatletter
\newcommand{\mq}[1]{% \mathquotes{<stuff>}
  \savebox{\mathquote}{\text{"}}% Save quotes
  \savebox{\mathbox}{$\displaystyle #1$}% Save <stuff>
  \raisebox{\dimexpr\ht\mathbox-\ht\mathquote\relax}{"}#1\raisebox{\dimexpr\ht\mathbox-\ht\mathquote\relax}{''}
}
\makeatother

% Paragraph mit Zähler (Section-Weise)
\newcounter{cparagraphC}
\newcommand{\cparagraph}[1]{
\stepcounter{cparagraphC}
\paragraph{\thesection{}-\thecparagraphC{} #1}
%\addcontentsline{toc}{subsubsection}{\thesection{}-\thecparagraphC{} #1}
\label{\thesection-\thecparagraphC}
}
\makeatletter
\@addtoreset{cparagraphC}{section}
\makeatother


% (Vorlesungs-)Folien einbinden:
% Folien von einer Datei skaliert
\newcommand{\slide}[2][\customSlideScale]{\slides[#1]{}{#2}}
\newcommand{\slideTrim}[6][\customSlideScale]{\slides[#1 , clip,  trim = #5cm #4cm #6cm #3cm]{}{#2}}
% Folien von mehreren nummerierten Dateien skaliert
\newcommand{\slides}[3][\customSlideScale]{\begin{center}
\includegraphics[page=#3, scale=#1]{\customSlidePath #2.pdf}
\end{center}}

% \emph{} anders definieren
\makeatletter
\DeclareRobustCommand{\em}{%
  \@nomath\em \if b\expandafter\@car\f@series\@nil
  \normalfont \else \scshape \fi}
\makeatother

% unwichtiges
\newcommand{\unimptnt}[1]{{\transparent{0.5}#1}}

% alph. enumerate
\newenvironment{anumerate}{\begin{enumerate}[label=(\alph*)]}{\end{enumerate}} % Alphabetische Aufzählung

%% EINFACHE BEFEHLE

% Abkürzungen Mathe
\newcommand{\EE}{\mathbb{E}}
\newcommand{\QQ}{\mathbb{Q}}
\newcommand{\RR}{\mathbb{R}}
\newcommand{\CC}{\mathbb{C}}
\newcommand{\NN}{\mathbb{N}}
\newcommand{\ZZ}{\mathbb{Z}}
\newcommand{\PP}{\mathbb{P}}
\renewcommand{\SS}{\mathbb{S}}
\newcommand{\cA}{\mathcal{A}}
\newcommand{\cB}{\mathcal{B}}
\newcommand{\cC}{\mathcal{C}}
\newcommand{\cD}{\mathcal{D}}
\newcommand{\cE}{\mathcal{E}}
\newcommand{\cF}{\mathcal{F}}
\newcommand{\cG}{\mathcal{G}}
\newcommand{\cH}{\mathcal{H}}
\newcommand{\cI}{\mathcal{I}}
\newcommand{\cJ}{\mathcal{J}}
\newcommand{\cM}{\mathcal{M}}
\newcommand{\cN}{\mathcal{N}}
\newcommand{\cP}{\mathcal{P}}
\newcommand{\cR}{\mathcal{R}}
\newcommand{\cS}{\mathcal{S}}
\newcommand{\cZ}{\mathcal{Z}}
\newcommand{\cL}{\mathcal{L}}
\newcommand{\cT}{\mathcal{T}}
\newcommand{\cU}{\mathcal{U}}
\newcommand{\cV}{\mathcal{V}}
\renewcommand{\phi}{\varphi}
\renewcommand{\epsilon}{\varepsilon}

% Verschiedene als Mathe-Operatoren
\DeclareMathOperator{\arccot}{arccot}
\DeclareMathOperator{\arccosh}{arccosh}
\DeclareMathOperator{\arcsinh}{arcsinh}
\DeclareMathOperator{\arctanh}{arctanh}
\DeclareMathOperator{\arccoth}{arccoth} 
\DeclareMathOperator{\var}{Var} % Varianz 
\DeclareMathOperator{\cov}{Cov} % Co-Varianz 

% Farbdefinitionen
\definecolor{red}{RGB}{180,0,0}
\definecolor{green}{RGB}{75,160,0}
\definecolor{blue}{RGB}{0,75,200}
\definecolor{orange}{RGB}{255,128,0}
\definecolor{yellow}{RGB}{255,245,0}
\definecolor{purple}{RGB}{75,0,160}
\definecolor{cyan}{RGB}{0,160,160}
\definecolor{brown}{RGB}{120,60,10}

\definecolor{itteny}{RGB}{244,229,0}
\definecolor{ittenyo}{RGB}{253,198,11}
\definecolor{itteno}{RGB}{241,142,28}
\definecolor{ittenor}{RGB}{234,98,31}
\definecolor{ittenr}{RGB}{227,35,34}
\definecolor{ittenrp}{RGB}{196,3,125}
\definecolor{ittenp}{RGB}{109,57,139}
\definecolor{ittenpb}{RGB}{68,78,153}
\definecolor{ittenb}{RGB}{42,113,176}
\definecolor{ittenbg}{RGB}{6,150,187}
\definecolor{itteng}{RGB}{0,142,91}
\definecolor{ittengy}{RGB}{140,187,38}

\definecolor{htworange}{RGB}{249,155,28}

% Textfarbe ändern
\newcommand{\tred}[1]{\textcolor{red}{#1}}
\newcommand{\tgreen}[1]{\textcolor{green}{#1}}
\newcommand{\tblue}[1]{\textcolor{blue}{#1}}
\newcommand{\torange}[1]{\textcolor{orange}{#1}}
\newcommand{\tyellow}[1]{\textcolor{yellow}{#1}}
\newcommand{\tpurple}[1]{\textcolor{purple}{#1}}
\newcommand{\tcyan}[1]{\textcolor{cyan}{#1}}
\newcommand{\tbrown}[1]{\textcolor{brown}{#1}}

% Umstellen der Tabellen Definition
\newcommand{\mpb}[1][.3]{\begin{minipage}{#1\textwidth}\vspace*{3pt}}
\newcommand{\mpe}{\vspace*{3pt}\end{minipage}}

\newcommand{\resultul}[1]{\underline{\underline{#1}}}
\newcommand{\parskp}{$ $\\}	% new line after paragraph
\newcommand{\corr}{\;\widehat{=}\;}
\newcommand{\mdeg}{^{\circ}}

\newcommand{\nok}[2]{\binom{#1}{#2}}	% n über k BESSER: \binom{n}{k}
\newcommand{\mtr}[1]{\begin{pmatrix}#1\end{pmatrix}}	% Matrix
\newcommand{\dtr}[1]{\begin{vmatrix}#1\end{vmatrix}}	% Determinante (Betragsmatrix)
\renewcommand{\vec}[1]{\underline{#1}}	% Vektorschreibweise
\newcommand{\imptnt}[1]{\colorbox{red!30}{#1}}	% Wichtiges
\newcommand{\intd}[1]{\,\mathrm{d}#1}
\newcommand{\diffd}[1]{\mathrm{d}#1}
% für Module-Rechnung: \pmod{}
\newcommand{\unit}[1]{\,\mathrm{#1}}

%\bibliography{\customDir .Literatur/HTW_Literatur.bib}
\setlength{\headheight}{10mm}	% default: ca. 8mm

\begin{document}

%\selectlanguage{english}
\maketitle
\newpage
\tableofcontents
\newpage

\chapter*{Vorbemerkung}

% User: rn PW: VLRN
Zur Prüfung: alles erlaubt (Taschenrechner, Unterlagen, …)

\chapter{Einführung}

% Folien

\section{Moderne Kommunikation}
Bandbreite: zur Zeit noch beschränkt, denkbar auch, dass jeder auf allen Bandbreiten. WLAN und Bluetooth sind schon auf einer Bandbreite und bekommen es auch hin.\\
Sendeleistung: Nicht „zu laut reden“, damit sich andere auch noch verstehen.

\chapter{Sockets}

\section{Einführung}
\subsection{Schnittstelle auf einem Host zur Datenübertragung an andere Prozesse}
\slidesScale{10-sockets_print}{3}
\subsection{Socketschnittstelle: Abgrenzung}
\slidesScale{10-sockets_print}{4}
\subsection{IP-Adressen und Ports}
\slidesScale{10-sockets_print}{5}
\section{Demultiplexing}
\slidesScale{10-sockets_print}{6}
\subsection{Verbindungsloses Demultiplexing (UDP)}
\slidesScale{10-sockets_print}{7}
\paragraph{Testfrage:} Benötigte UDP-Sockets: 2 Sockets (beliebig viele Partner können auf einen Port/Socket kommen, dafür muss darauf geachtet werden, dass das Paket an den richtigen Absender zurück geschickt wird)
\subsection{Verbindungsorientiertes Demultiplexing (TCP)}
\slidesScale{10-sockets_print}{8}
\paragraph{Testfrage:} Benötigte TCP-Sockets: 3 Sockets (jeder Verbindung ist eigenständig und vom Rechner verwaltet, es muss sich nicht drum gekümmert werden, dass es auch an den richtigen Partner zurück geschickt wird)
\slidesScale{10-sockets_print}{9}
\section{Socket-Programmierung}
\slidesScale{10-sockets_print}{10}
\subsection{UDP-Sockets}
\slidesScale{10-sockets_print}{11}
\slidesScale{10-sockets_print}{12}
\subsubsection{UDP-Demo Datenströme}
Datenströme müssen explizit in Pakete gewandelt werden:\\
UDP ist kein Stream $\to$ Konvertierung IO-Stream $\Leftrightarrow$ UDP.
\slidesScale{10-sockets_print}{13}
\subsubsection{UDP-Demo Client}
\slidesScale{10-sockets_print}{14}
\slidesScale{10-sockets_print}{15}
Achtung: Wenn Paket verloren geht, würde Programm stehen bleiben! Dafür gibt es in der Funktion \emph{.recieve()} auch die Option eines Timeouts.\\
Hinweis nebenbei: sendData-Array hätte gar nicht initialisiert werden müssen, weil er in der Zuweisung später sowieso einen neuen Zeiger bekommt.
\subsubsection{UDP-Demo Server}
\slidesScale{10-sockets_print}{16}
\slidesScale{10-sockets_print}{17}
\slidesScale{10-sockets_print}{18}
UDP vor allem bei Anfragen sinnvoll, die in ein Paket passen. Nicht so sinnvoll für mehrere Pakete/Streams.

\subsection{TCP-Sockets}
\slidesScale{10-sockets_print}{19}
\slidesScale{10-sockets_print}{20}
\slidesScale{10-sockets_print}{21}
\subsubsection*{Dreiwege-Handshake}
\slidesScale{10-sockets_print}{22}
\slidesScale{10-sockets_print}{23}
\subsubsection{TCP-Demo Datenströme}
Eingabestrom: Quelle - Tastatur\\
Ausgabestrom: Senke - Socket
\slidesScale{10-sockets_print}{24}
\slidesScale{10-sockets_print}{25}
\subsubsection{TCP-Demo Client}
\slidesScale{10-sockets_print}{26}
\slidesScale{10-sockets_print}{27}
\subsubsection{TCP-Demo Server}
\slidesScale{10-sockets_print}{28}
\slidesScale{10-sockets_print}{29}

\section{Werkzeug Netcat}
\slidesScale{10-sockets_print}{30}

\section*{Zusammenfassung}
\slidesScale{10-sockets_print}{31}

\chapter{Grundlagen OSI}

\section{Verzögerung}

\subsection{Verzögerung in paketorientierten Netzen}
\slidesScale{12-grundlagen_print}{3}

\subsubsection*{Beispiele}
\slidesScale{12-grundlagen_print}{4}

\subsubsection*{Grafik Verzögerung}
\slidesScale{12-grundlagen_print}{5}
(Hinweis: Interessant für Prüfung)

\subsubsection{Datenrate-Verzögerungsprodukt}
bandwitdh-delay product, BDP (Begriff Bandbreite hier korrekt)
\slidesScale{12-grundlagen_print}{6}
Wichtig: Es gibt einen Unterschied zwischen dem physischen Durchsatz und dem, der tatsächlich nur ankommt. In diesem Fall ist der Unterschied sehr groß: da muss die Fenstergröße größer gewählt werden.

\section{Standardisierung}
\subsection{Standardisierungskommision}
\subsubsection{International}
\slidesScale{12-grundlagen_print}{7}
\subsubsection{National/Europäisch}
\slidesScale{12-grundlagen_print}{8}
\subsection{Standards}
\slidesScale{12-grundlagen_print}{9}

\section{OSI Modell}
\subsection*{ISO OSI 7-Schichtenprotokoll}
\slidesScale{12-grundlagen_print}{10}
OSI: Open System Interconnection
\subsection*{Trennung in Anwendungs- und Transportverbindung}
\slidesScale{12-grundlagen_print}{11}
\subsection{Bitübertragungsschicht (Physical Layer)}
\slidesScale{12-grundlagen_print}{12}
\subsubsection*{Beispiel RS232}
\slidesScale{12-grundlagen_print}{13}
\subsection{Sicherungsschicht (Data Link Layer)}
\slidesScale{12-grundlagen_print}{14}
\subsection{Vermittlungsschicht (Network Layer)}
\slidesScale{12-grundlagen_print}{15}
\subsection{Transportschicht (Transport Layer)}
\slidesScale{12-grundlagen_print}{16}
\subsection{Sitzungsschicht (Session Layer)}
\slidesScale{12-grundlagen_print}{17}
\subsection{Darstellungsschicht (Presentation Layer)}
\slidesScale{12-grundlagen_print}{18}
\subsection{Anwendungsschicht (Application Layer)}
\slidesScale{12-grundlagen_print}{19}

\section{Protokolle und Dienste}
\subsection{Protokolle}
\subsubsection{Protokollbegriff}
\slidesScale{12-grundlagen_print}{20}
\subsubsection{Grundfunktionen in Protokollen}
\slidesScale{12-grundlagen_print}{21}
\subsection{Dienste}
\slidesScale{12-grundlagen_print}{22}
\subsubsection{Begriffe}
\slidesScale{12-grundlagen_print}{23}
\subsection{Dienst und Protokoll}
\slidesScale{12-grundlagen_print}{24}
\subsubsection*{Beispiel: Gespräch der Philosophen}
\slidesScale{12-grundlagen_print}{25}
\begin{tabular}{c c c r}
Chinesischer Philosoph & $\overset{\text{Thema}}{\longleftrightarrow}$ & Spanischer Philosoph\\
$\downarrow$ &  &  $\downarrow$\\
Übersetzung Englisch & $\overset{\text{Englisch}}{\longleftrightarrow}$ & Übersetzung Englisch\\
$\downarrow$ & & $\downarrow$\\
Funker & $\overset{\text{Morse}}{\longleftrightarrow}$ & Funker
\end{tabular}\\
Veranschaulichung: Horizontal ist das Protokoll und vertikal der Dienst.
\subsubsection{Dienstelemente: Funktionsprinzip}
\slidesScale{12-grundlagen_print}{26}
\subsubsection{Protokolldateneinheit / Interfacedateneinheit}
\slidesScale{12-grundlagen_print}{27}
\subsubsection{Zusammenhang: SDU - PCI - PDU - IDU}
\slidesScale{12-grundlagen_print}{28}
\subsubsection{Schichten und PDUs}
\slidesScale{12-grundlagen_print}{29}
\subsubsection{Beispiel für Dienstelemente}
\slidesScale{12-grundlagen_print}{30}
\subsubsection{Zeitdiagramm mit Dienstelementen}
Verbindungsaufbau zu Dienstelementen
\slidesScale{12-grundlagen_print}{31}
\section{Protokollbeschreibungen}
(Möglichkeiten, vgl. Software Engineering)
\slidesScale{12-grundlagen_print}{32}
\subsection{Systementwurf / Detailentwurf}
\slidesScale{12-grundlagen_print}{33}
\subsubsection{TCP-Zustände des Servers}
\slidesScale{12-grundlagen_print}{34}
\subsubsection{TCP-Zustandsmaschine}
\slidesScale{12-grundlagen_print}{35}
\subsection{Zusammenfassung}
\slidesScale{12-grundlagen_print}{36}
\subsubsection{Übung für Dienst-Primitiven (Operationen)}
\slidesScale{12-grundlagen_print}{37}
\slidesScale{12-grundlagen_print}{38}

\chapter{Transportschicht UDP/TCP}
\section{Aufgaben der Transportschicht}
\slidesScale{15-transportschicht_print}{3}
\subsection{Transportschichten im Internet}
\slidesScale{15-transportschicht_print}{4}
\subsection{Multiplexing / Demultiplexing}
\slidesScale{15-transportschicht_print}{5}
\subsection{Verbindungsloses Multiplexing}
\slidesScale{15-transportschicht_print}{6}
\section{UDP}
\subsection{UDP-Header}
(RFC 768)
\slidesScale{15-transportschicht_print}{7}
\subsection{UDP-Lite}
(RFC 3828, UDP für geringe Verzögerungen $\to$ Mediendaten)
\slidesScale{15-transportschicht_print}{8}

\subsection{Netzwerk-Datenformat}
\slidesScale{15-transportschicht_print}{9}
\section{TCP}
\subsection{Verbindungsorientiertes Multiplexing}
\slidesScale{15-transportschicht_print}{10}
\slidesScale{15-transportschicht_print}{11}
\subsection{TCP-Nutzung}
\slidesScale{15-transportschicht_print}{12}
\subsection{TCP-Eigenschaften}
\slidesScale{15-transportschicht_print}{13}
\subsection{TCP-Header}
\slidesScale{15-transportschicht_print}{14}
\slidesScale{15-transportschicht_print}{15}

\subsection{TCP Verbindungsstart}
(Drei-Wege-Handshake)
\slidesScale{15-transportschicht_print}{16}
\subsubsection{Sequenznummern}
\slidesScale{15-transportschicht_print}{17}
\subsection{Freigabe von Verbindungen}
\slidesScale{15-transportschicht_print}{18}
\subsubsection{Zwei-Armeen-Problem}
\slidesScale{15-transportschicht_print}{19}

\subsection{TCP Verbindungsende}
\slidesScale{15-transportschicht_print}{20}
\subsubsection{TCP Zustände des Servers}
\slidesScale{15-transportschicht_print}{21}
\subsubsection{TCP Zustände des Clients}
\slidesScale{15-transportschicht_print}{22}

\subsection{TCP Datenübertragung}
\slidesScale{15-transportschicht_print}{23}
\subsubsection{Beispiel Datenübertragung}
\slidesScale{15-transportschicht_print}{24}
\subsubsection{Beispiel Segmentverlust}
\slidesScale{15-transportschicht_print}{25}
\subsubsection{Beispiel zu knapper Timeout}
\slidesScale{15-transportschicht_print}{26}
\subsubsection{Beispiel Kumulative ACKs}
\slidesScale{15-transportschicht_print}{27}

\subsubsection{Test TCP-Segmenterzeugung}
\slidesScale{15-transportschicht_print}{28}
(potentielle Prüfungsaufgabe)

\subsection{TCP-ACK-Erzeugung}
(RFC 1122, RFC 2581)
\slidesScale{15-transportschicht_print}{29}
\subsubsection{Beispiel TCP Fast Retransmit}
\slidesScale{15-transportschicht_print}{30}

\subsection{TCP Flusskontrolle}
Angleichen der Sende-/Empfangsgeschwindigkeit
\slidesScale{15-transportschicht_print}{31}
\subsubsection{TCP-RcvWindow}
\slidesScale{15-transportschicht_print}{32}
\subsubsection{Beispiel TCP Verbindungsaufbau}
Paketsniffer Wireshark
\slidesScale{15-transportschicht_print}{33}
\slidesScale{15-transportschicht_print}{34}
\subsubsection{Retransmission-Timer}
\slidesScale{15-transportschicht_print}{35}
\subsubsection*{RTT Bestimmung}
\slidesScale{15-transportschicht_print}{36}
\subsubsection*{RTO Bestimmung}
\slidesScale{15-transportschicht_print}{37}
Achtung: Startwerte und Minimum RTO sind hier sehr konservativ angegeben!\\
Hinweis: RTO Formel nützlich für Beleg. Startwerte kürzer setzen und Minimum nicht nötig.

\subsection{TCP Überlastkontrolle}
\slidesScale{15-transportschicht_print}{38}
1. Bild: Flusskontrolle, 2. Bild: Überlastkontrolle\\
Daran stellt man Überlast fest:
\begin{itemize}
\item Paketverlust
\end{itemize}
\subsubsection{Ansätze zur Überlastkontrolle}
\slidesScale{15-transportschicht_print}{39}
\subsubsection{Überlastkontrolle bei TCP}
\slidesScale{15-transportschicht_print}{40}
\subsubsection{TCP Überlastkontrolle durch variable Fenstergröße}
Erhöhung der Fenstergröße bis Verlust eintritt
\slidesScale{15-transportschicht_print}{41}
\subsubsection{TCP Slow Start}
\slidesScale{15-transportschicht_print}{42}
\slidesScale{15-transportschicht_print}{43}
\subsubsection{Fast Recovery}
\slidesScale{15-transportschicht_print}{44}
\subsubsection{Optimale TCP Buffergröße}
Sende- und Empfangspuffer des Sockets
\slidesScale{15-transportschicht_print}{45}
\subsubsection{Zusammenfassung TCP Überlastkontrolle}
\slidesScale{15-transportschicht_print}{46}
\subsubsection{Verfahren zur Verstopfungsvermeidung}
Forschungsthema
\slidesScale{15-transportschicht_print}{47}

\subsection{TCP Reset}
\slidesScale{15-transportschicht_print}{48}
Verfahren beim Empfang in Abhängigkeit vom Zustand:
\slidesScale{15-transportschicht_print}{49}

\subsection{Keepalive}
\slidesScale{15-transportschicht_print}{50}

\section{Zusammenfassung Transportschicht}
\slidesScale{15-transportschicht_print}{51}

\section{Ergänzungen}
\subsection{Erweiterung Zeitstempel}
\slidesScale{15-transportschicht_print}{52}
\subsection{Wireshark}
Erkenntnisgewinn durch Paketanalyse
\slidesScale{15-transportschicht_print}{53}
\subsection{TCP-Fairness}
\slidesScale{15-transportschicht_print}{54}
\subsection{TCP offload engine (TOE)}
\slidesScale{15-transportschicht_print}{55}
\subsection{TCP Zustandsmaschine}
\slidesScale{15-transportschicht_print}{56}

\chapter{Sicherungsprotokolle}
\section{Überblick}
\subsection{FEC-Verfahren (Forward Error Correction)}
\slidesScale{20-sicherungsprotokolle_print}{3}
Potentielle als Echtzeitanwendung.

\subsection{ARQ-Verfahren (Automatic Repeat Request)}
\slidesScale{20-sicherungsprotokolle_print}{4}
Nicht gut für Echtzeitanwendungen, da Verzögerung.

\subsection{Hybride FEC/ARQ-Systeme}
\slidesScale{20-sicherungsprotokolle_print}{5}

\subsection{Coderate}
\slidesScale{20-sicherungsprotokolle_print}{6}

\subsection{Details zum ARQ-Verfahren}
\subsubsection{Fehlerereignisse}
\slidesScale{20-sicherungsprotokolle_print}{7}
\subsubsection{Paketfehlerrate}
Bei der Paketfehlerrate muss berücksichtigt werden, wie oft das Paket erfolgreich als fehlerhaft erkannt wird und erneut gesendet wird:
\slidesScale{20-sicherungsprotokolle_print}{8}
\subsubsection{Durchsatz}
\slidesScale{20-sicherungsprotokolle_print}{9}
\subsubsection{Beispiel}
\slidesScale{20-sicherungsprotokolle_print}{10}
D.h. maximal $84\%$ sind als Durchsatz zu erwarten!
\subsubsection{Unterteilung}
\slidesScale{20-sicherungsprotokolle_print}{11}

\section{Stop-and-Wait}
\slidesScale{20-sicherungsprotokolle_print}{12}
\subsection{Durchsatz}
\slidesScale{20-sicherungsprotokolle_print}{13}
$T_W=2T_{AB}+T_{ACK}$
\subsection{Beispiel Satellitenkanal}
\slidesScale{20-sicherungsprotokolle_print}{14}
Sehr ungünstig für Stop-and-Wait Protokoll, da das Verhältnis der Paketgröße zur Wartezeit zu ungünstig ist (zu kleine Pakete/zu lange Wartezeit) $\Rightarrow$ die Übertragungszeit ist zu lang.
\subsection{Beispiel Rechnernetz}
Abschätzung des Durchsatzes:
\slidesScale{20-sicherungsprotokolle_print}{15}

\section{Go-Back-N-Protokoll (GBN)}
\slidesScale{20-sicherungsprotokolle_print}{16}
\subsection{Optimale Fenstergröße}
\slidesScale{20-sicherungsprotokolle_print}{17}
Hinweise Notation: $\lceil a \rceil \Rightarrow a$ wird aufgerundet 
\subsection{Sendefenster der Größe N}
\slidesScale{20-sicherungsprotokolle_print}{18}
\subsection{Durchsatz}
\slidesScale{20-sicherungsprotokolle_print}{19}
Ist zwar langsamer als folgendes (SR), dafür braucht Empfänger aber keinen Puffer!

\section{Selective Repeat-Protokoll (SR)}
\slidesScale{20-sicherungsprotokolle_print}{20}
\subsection{Sendefenster}
\slidesScale{20-sicherungsprotokolle_print}{21}
\subsection{Empfangsfenster}
\slidesScale{20-sicherungsprotokolle_print}{22}
\subsection{Durchsatz}
\slidesScale{20-sicherungsprotokolle_print}{23}

\section{Ergänzungen}
\subsection{Paketnummerierung}
\slidesScale{20-sicherungsprotokolle_print}{24}
\subsection{Bidirektionale Datenübertragung}
\slidesScale{20-sicherungsprotokolle_print}{25}
\subsection{Vergleich der Sicherungsprotokolle}
\slidesScale{20-sicherungsprotokolle_print}{26}
\subsection{Beispiel mit verschiedenen ARQ-Protokollen}
(Richtfunkanlage)
\slidesScale{20-sicherungsprotokolle_print}{27}
\subsection{Zusammenfassung}
\slidesScale{20-sicherungsprotokolle_print}{28}
\subsection{Beleg Stop-and-Wait}
Zustände Sender:
\begin{itemize}
\item Wait for Data
\item Wait for ACK
\end{itemize}
Übergange:\\
Wait for Data $\to$ SW called send / send packet 0/1 $\to$ Wait for ACK\\
Wait for ACK $\to$ received ACK /  $\to$ Wait for Data\\
Wait for ACK $\to$ timeout / resend packet || stop $\to$ Wait for ACK \\
Wait for ACK $\to$ wrong ACK || wrong Session ID / $\to$ Wait for ACK

\chapter{Vermittlungsschicht}
\slidesScale{24-vermittlung_print}{3}
\section{Leitungsvermittlung}
\slidesScale{24-vermittlung_print}{4}
\slidesScale{24-vermittlung_print}{5}
\section{Paketvermittlung}
\slidesScale{24-vermittlung_print}{6}
\slidesScale{24-vermittlung_print}{7}
\section{Mischformen}
Betriebsarten paketvermittelter Netze
\slidesScale{24-vermittlung_print}{8}
\section{Zellvermittlung}
\slidesScale{24-vermittlung_print}{9}
\section{Routing}
Vermittlungskonzepte
\slidesScale{24-vermittlung_print}{10}
\subsection{Wegesuchung}
\slidesScale{24-vermittlung_print}{11}
\subsection{Überlastungsprobleme}
\subsubsection{flow control}
\slidesScale{24-vermittlung_print}{12}
\subsubsection{Verstopfungskontrolle}
(congestion control)
\slidesScale{24-vermittlung_print}{13}
\subsection{Pufferspeicherverwaltung}
\slidesScale{24-vermittlung_print}{14}

\section{Internetprotokoll}
\slidesScale{25-ip-schicht_print}{3}
\subsection{Einführung}
\subsubsection{Aufgaben der IP-Schicht}
\slidesScale{25-ip-schicht_print}{4}
\subsubsection{Vergleich OSI -- IP}
\slidesScale{25-ip-schicht_print}{5}
\subsubsection{Internetadressen}
Drei unabhängige Namens-/Adressebenen
\slidesScale{25-ip-schicht_print}{6}
\subsubsection{Domain Name System (DNS)}
\slidesScale{25-ip-schicht_print}{7}

\subsubsection{IP-Vergabe}
\slidesScale{25-ip-schicht_print}{8}

\subsubsection{Internetadressen (IPv4)}
\slidesScale{25-ip-schicht_print}{9}
\subsection{Adressklassen}
(classfull addressing) Unterteilung in Netz und Host
\slidesScale{25-ip-schicht_print}{10}

\subsubsection{Besondere Internet-Adressen}
\slidesScale{25-ip-schicht_print}{11}
\subsubsection{Subnetting}
Aufteilung eines Netzes in mehrere Netze
\slidesScale{25-ip-schicht_print}{12}
\subsection{Classless Interdomain Routing (CIDR)}
\slidesScale{25-ip-schicht_print}{13}
\subsubsection*{Beispiel}
Vergabe von Adressen aus dem Pool eines Internet-Providers
\slidesScale{25-ip-schicht_print}{14}
\slidesScale{25-ip-schicht_print}{15}
\slidesScale{25-ip-schicht_print}{16}
Hostadressen in einem /31 Netz: 0 (nur ein Bit für Adresse übrig: 0 steht für Netzadresse, 1 für Broadcastadresse $\to$ keine Hostadresse mehr übrig)\\
$\Rightarrow$ /30 Netz: 2 Hostadressen
\subsection{IP-Netz}
\subsubsection{Definition IP-Netz}
Hosts mit gleicher Netzadresse befinden sich im gleichen Netz
\slidesScale{25-ip-schicht_print}{17}
\subsubsection{Adressarten}
\slidesScale{25-ip-schicht_print}{18}
\subsubsection{Hardwareadressen}
\slidesScale{25-ip-schicht_print}{19}
\subsubsection{Adress Resolution Protokoll}
Beispiel: A möchte HW-Adresse von B erhalten
\slidesScale{25-ip-schicht_print}{20}
\subsubsection*{Anzeigen}
\slidesScale{25-ip-schicht_print}{21}
\subsubsection{Übertragung eines Datagramms}
Übertragung 111.111.111.111 zu 222.222.222.222 (Maske 255.255.255.0)
\slidesScale{25-ip-schicht_print}{22}








%\newpage
%\printbibliography

\end{document}