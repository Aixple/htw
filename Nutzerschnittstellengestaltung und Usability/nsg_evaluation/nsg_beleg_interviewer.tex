\section{Leitfaden für die Interviewer}
Aufgrund der Anwendungskontexte vergangener Benutzer liegt das Hauptaugenmerk auf der Desktop-Version der Webseite.

Als einen Teil der Evaluation soll der Nutzer mit möglichst wenig Vorwissen die Webseite benutzen und Aufgaben bearbeiten. Aus diesem Grund ist die Einführung nur mit Erklärungen zum Ablauf und nicht zum Inhalt zu gestalten. Des Weiteren werden alle Fragen, auch die zur Person, erst im anschließenden Interview gestellt.

Der Interviewer soll, neben allgemeinen Beobachtungen, auch Verhalten bezüglich folgender Features protokollieren (die teils nicht explizit getestet werden):
\begin{itemize}
\item Werden Breadcrumbs benutzt/beachtet?
\item Wird die Suche benutzt/beachtet?
\item Wird die Sitemap benutzt/beachtet?
\item Gibt es eine Reaktion auf die Cookie-Einblendung? Wenn ja: Welche?
\item Gibt es Kommentare zum Inhalt selbst? Zu guten/schlechten Formulierungen im Text?
\item Wird die SSL-Verschlüsselung wahrgenommen?
\end{itemize}

Nach Möglichkeit soll zusätzlich zur Audio- und Videoaufnahme ein Protokoll dessen geführt werden, was der Nutzer (erwähnenswertes) tut und sagt. Dazu sind, ähnlich wie beim Teilnehmer, Textboxen vorgesehen.

Beim Interview-Teil sind einige Antworten zum besseren Vergleich in einer Skala von 1 bis 7 einzuordnen.

Vor Beginn der Evaluation muss die Kamera entsprechend vorbereitet werden, dass sie den Computerbildschirm gut erfasst und der Proband auch sein Handy bequem so halten kann, dass es ebenfalls von der Kamera erfasst wird.