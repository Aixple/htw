\lecdate{10.10.2017}
\slides{intro_bio}{1}
\section{Einordnung der Neuroinformatik}
Künstliche Intelligenz:
\begin{itemize}
\item Logik und Deduktion
\item Mustererkennung
\item Maschinelles Lernen
\item Spracherkennung
\item …
\item \emph{Neuroinformationsverarbeitung}
\begin{itemize}
\item Biologie
\item Psychologie
\item Computational Neuroscience
\item …
\item \emph{Neuronalen Netze}
\end{itemize}
\end{itemize}

\section{Geschichte der Neuroinformatik}
\slides{intro_bio}{2}
\subsection{Lernen}
\begin{center}
\fbox{Biologisches System} $\longleftrightarrow$ \fbox{Umwelt}
\end{center}
\emph{Lernen}: Optimierung der „Fitness“\\
(Biologisches System soll besser in seiner Umwelt zurecht kommen).\\
Benötigt dafür: Funktion, die diese Fitness beschreibt.
\subsection{Klassifikation}
Einfache Klassifikation: Die, die links und die, die rechts von $\Theta$ liegen.
\begin{center}
\begin{tikzpicture}[scale=.5]
% Koordinatenachsen:
\draw [-latex] (-5.5,0) -- (7,0) node[below right]{$y$};
\draw [-latex] (0,-0.5) node[right]{$0$} -- (0,3.5) node[above left]{$z$};
% Funktion:
\draw [thick] (-3.5,0) -- (2.5,0) node[below]{$\Theta$} -- (2.5,2) -- (6,2);
\draw (0,2) -- (-0.2,2) node[left]{$1$};
\end{tikzpicture}
\end{center}
Eine Gerade als einfachste Form einer Diskriminanzfunktion (einer Punktwolke). Orthogonal zu dieser Geraden kann wieder die klassifizierende Funktion mit dem $\Theta$ an der Diskriminanzfunktion anlegen.
\begin{center}
\begin{tikzpicture}[scale=.5]
% Koordinatenachsen:
\draw [-latex] (-5.5,0) -- (7,0) node[below right]{$x_1$};
\draw [-latex] (0,-0.5) -- (0,3.5) node[above left]{$x_2$};
% Funktion:
\draw (-2,-1.5) -- (5.5,3.5);
% Punkte oben:
\draw [orange] (2,3) circle (0.2);
\draw [orange] (2.75,2.5) circle (0.2);
\draw [orange] (1,1.5) circle (0.2);
\draw [orange] (1.5,2) circle (0.2);
\draw [orange] (1,2.5) circle (0.2);
% Punkte unten
\draw [blue] (4.25,2) circle (0.2);
\draw [blue] (4.5,1) circle (0.2);
\draw [blue] (2.5,0.5) circle (0.2);
\draw [blue] (3.75,1) circle (0.2);
\draw [blue] (3.25,1.5) circle (0.2);
\end{tikzpicture}
\end{center}
Kontrast: XOR -- Welche lineare Funktion kann diese Klassen separieren? Keine!
\begin{center}
\begin{tikzpicture}[scale=1]
% Koordinatenachsen:
\draw [-latex] (-0.5,0) -- (2,0) node[below right]{$x_1$};
\draw [-latex] (0,-0.5) -- (0,2) node[above left]{$x_2$};
% Funktion:
% Punkte oben:
\draw [orange] (1,0) circle (0.1);
\draw [orange] (0,1) circle (0.1);
% Punkte unten
\draw [blue] (1,1) circle (0.1);
\draw [blue] (0,0) circle (0.1);
\end{tikzpicture}
\end{center}
Lösungsmöglichkeiten: Einteilung in Mengen -- die Diagonale und der Rest. 

\section{Das Gehirn}
\subsection{Anatomie}
\slides{intro_bio}{3}
\subsection{Vorstellung der Funktionalität Anfang des 20. Jahrhunderts}
\slides{intro_bio}{4}

\subsection{Neuronen}
\subsubsection{Experimenteller Nachweis von Einzelneuronen}
\slides{intro_bio}{6}
\subsubsection{Strukturelle Vielfalt der Neuronentypen}
\slides{intro_bio}{7}
\subsubsection{Strukturelle und funktionelle Details}
\slides[.7]{intro_bio}{10}
Wichtige Bestandteile:
\begin{itemize}
\item Dendrit mit Kontaktstellen (Axon von anderer Zelle)\\
Kopplung durch Synapsen beschreibt man mit einem Gewichtsvektor $\vec{w}$, dessen Komponenten adaptierbar sind $\to$ lernen!
\item Axon\\
Jeder Eingang bildet \emph{eine} Komponente eines Eingabevektors $\vec{x}$.
\item Zellkörper\\
„verrechnet“ $\vec{x}$ und $\vec{w}$
\end{itemize}
Die Nervenzelle hat eine besondere Art der Zellmembran: Semipermeable Membran mit Ionen-Kanälen (Teilbild rechts oben).

\subsection{Medizinische Verfahren für funktionelle Untersuchungen}
\slides{intro_bio}{8}

\subsection{Hirnaktivität bei Intellektuellen Aufgaben}
\slides{intro_bio}{9}

\section{Neuronale Netze in der Robotik}
\subsection{Lokale Fahrzeugnavigation durch Experten-Cloning}
\slides{intro_bio}{12}
\subsection[360 Grad Geräusch-Lokalisationssystems für Roboter]{Biologisch motiviertes Modell eines 360 Grad Geräusch-Lokalisationssystems für Roboter}
\slides{intro_bio}{13}
\subsubsection*{Experimentelle Ergebnisse}
\slides{intro_bio}{14}