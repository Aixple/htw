
\slides{24-vermittlung_print}{3}
\section{Leitungsvermittlung}
\slides{24-vermittlung_print}{4}
\slides{24-vermittlung_print}{5}
\section{Paketvermittlung}
\slides{24-vermittlung_print}{6}
\slides{24-vermittlung_print}{7}
\section{Mischformen}
Betriebsarten paketvermittelter Netze
\slides{24-vermittlung_print}{8}
\section{Zellvermittlung}
\slides{24-vermittlung_print}{9}
\section{Routing}
Vermittlungskonzepte
\slides{24-vermittlung_print}{10}
\subsection{Wegesuchung}
\slides{24-vermittlung_print}{11}
\subsection{Überlastungsprobleme}
\subsubsection{flow control}
\slides{24-vermittlung_print}{12}
\subsubsection{Verstopfungskontrolle}
(congestion control)
\slides{24-vermittlung_print}{13}
\subsection{Pufferspeicherverwaltung}
\slides{24-vermittlung_print}{14}

\section{Internetprotokoll}
\slides{25-ip-schicht_print}{3}
\subsection{Einführung}
\subsubsection{Aufgaben der IP-Schicht}
\slides{25-ip-schicht_print}{4}
\subsubsection{Vergleich OSI -- IP}
\slides{25-ip-schicht_print}{5}
\subsubsection{Internetadressen}
Drei unabhängige Namens-/Adressebenen
\slides{25-ip-schicht_print}{6}
\subsubsection{Domain Name System (DNS)}
\slides{25-ip-schicht_print}{7}

\subsubsection{IP-Vergabe}
\slides{25-ip-schicht_print}{8}

\subsubsection{Internetadressen (IPv4)}
\slides{25-ip-schicht_print}{9}
\subsection{Adressklassen}
(classfull addressing) Unterteilung in Netz und Host
\slides{25-ip-schicht_print}{10}

\subsubsection{Besondere Internet-Adressen}
\slides{25-ip-schicht_print}{11}
\subsubsection{Subnetting}
Aufteilung eines Netzes in mehrere Netze
\slides{25-ip-schicht_print}{12}
\subsection{Classless Interdomain Routing (CIDR)}
\slides{25-ip-schicht_print}{13}
\subsubsection*{Beispiel}
Vergabe von Adressen aus dem Pool eines Internet-Providers
\slides{25-ip-schicht_print}{14}
\slides{25-ip-schicht_print}{15}
\slides{25-ip-schicht_print}{16}
Hostadressen in einem /31 Netz: 0 (nur ein Bit für Adresse übrig: 0 steht für Netzadresse, 1 für Broadcastadresse $\to$ keine Hostadresse mehr übrig)\\
$\Rightarrow$ /30 Netz: 2 Hostadressen\bigskip\\
Beispiel $\underbrace{141.56}_{Netz}.\underbrace{0.0}_{Host}/\underbrace{16}_{Netzanteil}$: $2^{16}-2$ Hosts möglich (die zwei Adressen $0.0$ als Netzschlüssel und $255.255$ als Broadcast vergeben)
\subsection{IP-Netz}
\subsubsection{Definition IP-Netz}
Hosts mit gleicher Netzadresse befinden sich im gleichen Netz
\slides{25-ip-schicht_print}{17}
\subsubsection{Adressarten}
\slides{25-ip-schicht_print}{18}
\subsubsection{Hardwareadressen}
\slides{25-ip-schicht_print}{19}
\subsubsection{Adress Resolution Protokoll}
Beispiel: A möchte HW-Adresse von B erhalten
\slides{25-ip-schicht_print}{20}
\subsubsection*{Anzeigen}
\slides{25-ip-schicht_print}{21}
\subsubsection{Übertragung eines Datagramms}
Übertragung 111.111.111.111 zu 222.222.222.222 (Maske 255.255.255.0)
\slides{25-ip-schicht_print}{22}

\subsection{Weiterleitung und Wegewahl}
\slides{25-ip-schicht_print}{23}
\subsubsection{Analyse des Verbindungsweges}
Beispiel für Nutzung von \lstinline|traceroute|
\slides{25-ip-schicht_print}{24}
\subsection{Dynamic Host Configuration Protocoll (DHCP)}
RFC 2131, 2132
\slides{25-ip-schicht_print}{25}
\subsection{IP-Paketaufbau (IPv4)}
Header 20 Byte + optionaler Anteil
\slides{25-ip-schicht_print}{26}
\slides{25-ip-schicht_print}{27}
\subsubsection{IP-Fragmentierung}
\slides{25-ip-schicht_print}{28}
\subsubsection*{Beispiel}
\slides{25-ip-schicht_print}{29}

\subsection{Komponenten des Internet Protokolls}
\slides{25-ip-schicht_print}{30}

\subsubsection{Internet Contronl Message Protocoll (ICMP)}
RFC 792
\slides{25-ip-schicht_print}{31}
\subsubsection*{ICMP Unreachable}
\slides{25-ip-schicht_print}{32}

\subsection{Weiterleitung}
\slides{25-ip-schicht_print}{33}
\subsubsection*{Genereller Aufbau}
\slides{25-ip-schicht_print}{34}
\subsubsection*{Beispiel}
Routingtabelle für mittlere Router
\slides{25-ip-schicht_print}{35}
\subsubsection*{Befehler zur Konfiguration}
\slides{25-ip-schicht_print}{36}
\subsection{IPv6}
Probleme von IPv4:
\slides{25-ip-schicht_print}{37}

\subsection{Aufbau und Funktion eines Routers}
\slides{25-ip-schicht_print}{38}
\subsubsection{Eingangsport}
\slides{25-ip-schicht_print}{39}
\subsubsection{Switching Fabric}
\slides{25-ip-schicht_print}{40}
\slides{25-ip-schicht_print}{41}
\subsubsection{Ausgangsport}
\slides{25-ip-schicht_print}{42}
\subsubsection{Queueing am Ausgangsport}
\slides{25-ip-schicht_print}{43}
\subsubsection{Router Bauformen}
\slides{25-ip-schicht_print}{44}

\subsection{Network Address Translation (NAT)}
\slides{25-ip-schicht_print}{45}
\subsubsection*{Funktion ohne NAT}
\slides{25-ip-schicht_print}{46}
\subsubsection{Funktionsweise}
\slides{25-ip-schicht_print}{47}
\subsubsection*{Funktion mit NAT}
\slides{25-ip-schicht_print}{48}
\subsubsection{Probleme mit NAT}
\slides{25-ip-schicht_print}{49}
\subsubsection{Lösungsansätze zur Erreichbarkeit}
Ein Server innerhalb von NAT soll erreicht werden
\slides{25-ip-schicht_print}{50}
\subsection{Zusammenfassung}
\slides{25-ip-schicht_print}{51}

\section{Routing-Protokolle}
\subsection{Netzwerke als Graph}
\slides{26-ip-routing_print}{3}

\subsection{Metriken bei Netzwerken}
Maß für die Güte einer Verbindung
\slides{26-ip-routing_print}{4}

\subsection{Routinalgorithmen}
Wie erfolgt die Wegesuche im Internet?
\slides{26-ip-routing_print}{5}

\subsection{Distanzvektor Routing}
\slides{26-ip-routing_print}{6}

\subsubsection{Prinzip: Bellmann-Ford-Gleichung}
\slides{26-ip-routing_print}{7}
\slides{26-ip-routing_print}{8}
\subsubsection{Ablauf}
\slides{26-ip-routing_print}{9}
\subsubsection{Beispiel für Distanzvektor Routing}
Entfernungsmaß sei Übertragungszeit
\slides{26-ip-routing_print}{10}
\subsubsection{Änderung der Kosten:}
\slides{26-ip-routing_print}{11}
\subsubsection{Problem: Count-to-Infinity}
\slides{26-ip-routing_print}{12}
\subsubsection{RIP-Protokoll (RIPv2)}
\slides{26-ip-routing_print}{13}
\subsubsection*{RIP-Timer}
\slides{26-ip-routing_print}{14}

\subsection{Link State Routing}
Routing in Fünf Schritten
\slides{26-ip-routing_print}{15}
\subsubsection{Dijkstra-Algorithmus}
Berechnung des kürzesten Pfades von A nach B
\slides{26-ip-routing_print}{16}
\slides{26-ip-routing_print}{17}
\subsubsection*{Beispiel}
\slides{26-ip-routing_print}{18}
Tipp: Nexthop und Wegelänge an Knoten ran schreiben.
\subsubsection*{Komplexitätsbetrachtung}
\slides{26-ip-routing_print}{19}
\subsubsection*{Probleme mit Oszillation bei netzlastabhängigen Metriken}
\slides{26-ip-routing_print}{20}

\subsubsection{Open Shortest Path First OSPF}
\slides{26-ip-routing_print}{21}

\subsection{Vergleich Routingprotokolle (Intra-AS)}
\slides{26-ip-routing_print}{22}

\subsection{Hierarchisches Routing -- Autonome Systeme (AS)}
\slides{26-ip-routing_print}{23}
\slides{26-ip-routing_print}{24}
\slides{26-ip-routing_print}{25}
\subsubsection*{Autonome Systeme}
\slides{26-ip-routing_print}{26}

\subsubsection{Beziehungen}
Provider, Peers, Kunden
\slides{26-ip-routing_print}{27}
\subsubsection{3-Stufen Hierarchie}
\slides{26-ip-routing_print}{28}

\subsection{Border-Gateway-Protokoll (BGP)}
\slides{26-ip-routing_print}{29}
\subsubsection{Grundlagen:}
\slides{26-ip-routing_print}{30}
\subsubsection{Supernetting}
\slides{26-ip-routing_print}{31}
\subsubsection{Routenauswahl}
Vereinfachte Auswahlregeln
\slides{26-ip-routing_print}{32}
\subsubsection{Routingrichtlinien: Beispiel mit 6 AS}
\slides{26-ip-routing_print}{33}

\subsection{AS in Deutschland}
\slides{26-ip-routing_print}{34}

\subsection{Zusammenfassung}
\slides{26-ip-routing_print}{35}

