\lecdate{19.12.2017}
\section{Einführung}
\slides{it2-50-galois_print}{3}
\subsection{Definition Galois-Feld}
\slides{it2-50-galois_print}{4}
\subsection{Definition Prim- und Erweiterungskörper}
\slides{it2-50-galois_print}{5}
\subsubsection*{Beispiel GF(3)}
\slides{it2-50-galois_print}{6}
\subsubsection*{Beispiel Nichtexistenz von GF(4)}
\slides{it2-50-galois_print}{7}
\subsubsection*{Beispiel GF(5)}
\slides{it2-50-galois_print}{8}

\subsection{Primitives Element}
\slides{it2-50-galois_print}{9}
\subsubsection*{Ordnung eines Elements}
\slides{it2-50-galois_print}{10}
\subsection{Irreduzible Polynome}
\slides{it2-50-galois_print}{11}
\subsection{Erweiterungskörper}
\slides{it2-50-galois_print}{12}
\subsubsection{Existenz}
\slides{it2-50-galois_print}{13}
\subsubsection{Konstruktion}
\slides{it2-50-galois_print}{14}
\subsubsection{Rechnung}
\slides{it2-50-galois_print}{15}
\subsection{Primitives Polynom}
\slides{it2-50-galois_print}{16}
\slides{it2-50-galois_print}{17}
\subsection{Alternative Berechnung Modulo}
\slides{it2-50-galois_print}{18}
\subsubsection*{Beispiel GF(4)}
\slides{it2-50-galois_print}{19}
\slides{it2-50-galois_print}{20}

\section{Darstellungsarten der Erweiterungskörper}
\slides{it2-50-galois_print}{21}

\subsection{Beispiel GF(8)}
\slides{it2-50-galois_print}{22}
\slides{it2-50-galois_print}{23}
\slides{it2-50-galois_print}{24}
\slides{it2-50-galois_print}{25}

\subsection{Beispiel GF(16)}
\slides{it2-50-galois_print}{26}

\section{Implementierung}
\slides{it2-50-galois_print}{27}
\subsubsection*{Beispiel}
\slides{it2-50-galois_print}{28}









