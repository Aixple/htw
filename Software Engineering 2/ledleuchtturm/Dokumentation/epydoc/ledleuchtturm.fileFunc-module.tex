%
% API Documentation for API Documentation
% Module ledleuchtturm.fileFunc
%
% Generated by epydoc 3.0.1
% [Tue Jul 18 00:55:24 2017]
%

%%%%%%%%%%%%%%%%%%%%%%%%%%%%%%%%%%%%%%%%%%%%%%%%%%%%%%%%%%%%%%%%%%%%%%%%%%%
%%                          Module Description                           %%
%%%%%%%%%%%%%%%%%%%%%%%%%%%%%%%%%%%%%%%%%%%%%%%%%%%%%%%%%%%%%%%%%%%%%%%%%%%

    \index{ledleuchtturm \textit{(package)}!ledleuchtturm.fileFunc \textit{(module)}|(}
\section{Modul ledleuchtturm.fileFunc}

    \label{ledleuchtturm:fileFunc}
Funktionen zur Arbeit mit Dateien.

\textbf{Autor:} Simon Retsch




%%%%%%%%%%%%%%%%%%%%%%%%%%%%%%%%%%%%%%%%%%%%%%%%%%%%%%%%%%%%%%%%%%%%%%%%%%%
%%                               Functions                               %%
%%%%%%%%%%%%%%%%%%%%%%%%%%%%%%%%%%%%%%%%%%%%%%%%%%%%%%%%%%%%%%%%%%%%%%%%%%%

  \subsection{Funktionen}

    \label{ledleuchtturm:fileFunc:checkArgumentForConfigFile}
    \index{ledleuchtturm \textit{(package)}!ledleuchtturm.fileFunc \textit{(module)}!ledleuchtturm.fileFunc.checkArgumentForConfigFile \textit{(function)}}

    \vspace{0.5ex}

\hspace{.8\funcindent}\begin{boxedminipage}{\funcwidth}

    \raggedright \textbf{checkArgumentForConfigFile}(\textit{argv})

    \vspace{-1.5ex}

    \rule{\textwidth}{0.5\fboxrule}
\setlength{\parskip}{2ex}
    Überprüft ob dem Programm ein Pfad einer Konfigurationsdatei mit der 
    Endung .json als Argument übergeben wurde.

\setlength{\parskip}{1ex}
      \textbf{Parameter}
      \vspace{-1ex}

      \begin{quote}
        \begin{Ventry}{xxxx}

          \item[argv]

          Programm Übergabeparameter.

            {\it (type=string[])}

        \end{Ventry}

      \end{quote}

      \textbf{R"uckgabewert}
    \vspace{-1ex}

      \begin{quote}
      Pfad der Konfigurationsdatei falls es erfolgreich ist, -1 falls nicht
      die richtige Anzahl von Argumenten mitgegeben wurde.

      {\it (type=int / string)}

      \end{quote}

\textbf{Autor:} Raphael Pour



    \end{boxedminipage}

    \label{ledleuchtturm:fileFunc:configFileExists}
    \index{ledleuchtturm \textit{(package)}!ledleuchtturm.fileFunc \textit{(module)}!ledleuchtturm.fileFunc.configFileExists \textit{(function)}}

    \vspace{0.5ex}

\hspace{.8\funcindent}\begin{boxedminipage}{\funcwidth}

    \raggedright \textbf{configFileExists}(\textit{configFile})

    \vspace{-1.5ex}

    \rule{\textwidth}{0.5\fboxrule}
\setlength{\parskip}{2ex}
    Überprüft ob die Konfigurationsdatei existiert.

\setlength{\parskip}{1ex}
      \textbf{Parameter}
      \vspace{-1ex}

      \begin{quote}
        \begin{Ventry}{xxxxxxxxxx}

          \item[configFile]

          Pfad der Konfigurationsdatei.

            {\it (type=string)}

        \end{Ventry}

      \end{quote}

      \textbf{R"uckgabewert}
    \vspace{-1ex}

      \begin{quote}
      1 falls die Datei existiert, -1 falls die Datei nicht existiert.

      {\it (type=int)}

      \end{quote}

\textbf{Autor:} Raphael Pour



    \end{boxedminipage}

    \label{ledleuchtturm:fileFunc:configFileValidJsonFile}
    \index{ledleuchtturm \textit{(package)}!ledleuchtturm.fileFunc \textit{(module)}!ledleuchtturm.fileFunc.configFileValidJsonFile \textit{(function)}}

    \vspace{0.5ex}

\hspace{.8\funcindent}\begin{boxedminipage}{\funcwidth}

    \raggedright \textbf{configFileValidJsonFile}(\textit{configFile})

    \vspace{-1.5ex}

    \rule{\textwidth}{0.5\fboxrule}
\setlength{\parskip}{2ex}
    Überprüft ob es sich bei configFile wirklich um eine json Datei 
    handelt.

\setlength{\parskip}{1ex}
      \textbf{Parameter}
      \vspace{-1ex}

      \begin{quote}
        \begin{Ventry}{xxxxxxxxxx}

          \item[configFile]

          Pfad der Konfigurationsdatei.

            {\it (type=string)}

        \end{Ventry}

      \end{quote}

      \textbf{R"uckgabewert}
    \vspace{-1ex}

      \begin{quote}
      1 falls configFile eine Json Datei ist, -1 falls configFile keine 
      Json Datei ist.

      {\it (type=int)}

      \end{quote}

\textbf{Autor:} Raphael Pour



    \end{boxedminipage}

    \label{ledleuchtturm:fileFunc:createStdConf}
    \index{ledleuchtturm \textit{(package)}!ledleuchtturm.fileFunc \textit{(module)}!ledleuchtturm.fileFunc.createStdConf \textit{(function)}}

    \vspace{0.5ex}

\hspace{.8\funcindent}\begin{boxedminipage}{\funcwidth}

    \raggedright \textbf{createStdConf}()

    \vspace{-1.5ex}

    \rule{\textwidth}{0.5\fboxrule}
\setlength{\parskip}{2ex}
    Schreibt ein Python Objekt in eine Datei mit dem Namen 
    Std-Conf-Ledleuchtturm.json. In dem Objekt befinden sich Standartwerte.

\setlength{\parskip}{1ex}
      \textbf{R"uckgabewert}
    \vspace{-1ex}

      \begin{quote}
      1, falls das Objekt in die Datei geschrieben werden kann, -1 falls es
      dabei Probleme gibt.

      {\it (type=int)}

      \end{quote}

\textbf{Autor:} Simon Retsch



    \end{boxedminipage}

    \label{ledleuchtturm:fileFunc:getJsonFileObject}
    \index{ledleuchtturm \textit{(package)}!ledleuchtturm.fileFunc \textit{(module)}!ledleuchtturm.fileFunc.getJsonFileObject \textit{(function)}}

    \vspace{0.5ex}

\hspace{.8\funcindent}\begin{boxedminipage}{\funcwidth}

    \raggedright \textbf{getJsonFileObject}(\textit{fileName})

    \vspace{-1.5ex}

    \rule{\textwidth}{0.5\fboxrule}
\setlength{\parskip}{2ex}
    Lädt die Struktur einer Json Datei in ein Objekt.

\setlength{\parskip}{1ex}
      \textbf{Parameter}
      \vspace{-1ex}

      \begin{quote}
        \begin{Ventry}{xxxxxxxx}

          \item[fileName]

          Pfad der Json Datei.

            {\it (type=string)}

        \end{Ventry}

      \end{quote}

      \textbf{R"uckgabewert}
    \vspace{-1ex}

      \begin{quote}
      Objekt welches aus einer Json Datei geladen wurde.

      {\it (type=Json Object)}

      \end{quote}

\textbf{Autor:} Simon Retsch



    \end{boxedminipage}


%%%%%%%%%%%%%%%%%%%%%%%%%%%%%%%%%%%%%%%%%%%%%%%%%%%%%%%%%%%%%%%%%%%%%%%%%%%
%%                               Variables                               %%
%%%%%%%%%%%%%%%%%%%%%%%%%%%%%%%%%%%%%%%%%%%%%%%%%%%%%%%%%%%%%%%%%%%%%%%%%%%

  \subsection{Variablen}

    \vspace{-1cm}
\hspace{\varindent}\begin{longtable}{|p{\varnamewidth}|p{\vardescrwidth}|l}
\cline{1-2}
\cline{1-2} \centering \textbf{Name} & \centering \textbf{Beschreibung}& \\
\cline{1-2}
\endhead\cline{1-2}\multicolumn{3}{r}{\small\ldots}\\\endfoot\cline{1-2}
\endlastfoot\raggedright g\-l\-o\-b\-a\-l\-F\-i\-l\-e\-O\-p\-e\-n\- & \raggedright Dateiobjekt einer geöffneten Datei.

\textbf{Wert:} 
{\tt -1}            {\it (type=File object)}&\\
\cline{1-2}
\raggedright \_\-\_\-p\-a\-c\-k\-a\-g\-e\-\_\-\_\- & \raggedright \textbf{Wert:} 
{\tt \texttt{'}\texttt{ledleuchtturm}\texttt{'}}&\\
\cline{1-2}
\end{longtable}

    \index{ledleuchtturm \textit{(package)}!ledleuchtturm.fileFunc \textit{(module)}|)}
