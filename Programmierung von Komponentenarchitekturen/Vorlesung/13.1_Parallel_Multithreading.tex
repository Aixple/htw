Multithreading

Donnerstag, 2. Februar 2017

17:06

~

Verschiedene Methoden zur Parallelisierung: 

\begin{itemize}
\item Hochleistungsrechner: Verteilter Speicher, Kommunikation zwischen Rechnern durch Nachrichten (Message Passing, MPI)
\item Einzelrechner/Smartphones: gemeinsamer Speicher, Aufteilung des Programmablauf in Threads
\end{itemize}
~

wir beschränken uns auf letzteres!

Anwendung

{}-

Prozess (Task)

Thread Threadl Threadb

 \includegraphics{a131ParallelMultithreading-img001.png}  \includegraphics{a131ParallelMultithreading-img002.png}  \includegraphics{a131ParallelMultithreading-img003.png} 

~

Thread (Faden) = Teilprozess

\begin{itemize}
\item zum Start des Programms: 1 Thread (Masterthread)
\item serielle Programmierung: es gibt nur diesen einen Thread
\item Multithreading: es werden neue Threads gebildet die parallel abgearbeitet werden können
\end{itemize}
~

 \includegraphics{a131ParallelMultithreading-img004.png}  \includegraphics{a131ParallelMultithreading-img005.png}  \includegraphics{a131ParallelMultithreading-img006.png}  \includegraphics{a131ParallelMultithreading-img007.png} 

joinzusannnenführen)

fork (verzweigen)

\begin{itemize}
\item In einem Prozessor kann zu jedem Zeitpunkt nur 1 Thread bearbeitet werden
\item sind mehrere Threads -{\textgreater} Warteschlange, Austausch in kurzen Zeitintervallen (Standard: 20ms), Scheduling durch das System
\item Threads haben verschiedene Prioritätsstufen (Windows: 1..31) die beim Scheduling berücksichtigt werden
\end{itemize}
~

Threadzustände (die wichtigsten 3)

\begin{itemize}
\item laufend (running) - läuft gerade in der CPU
\item bereit (ready) - wartet in der Warteschlange auf CPU-Zeit
\item wartend (waiting) - im Leerlauf, benötigt im Moment keine CPU-Zeit, kann erst durch einen anderen Thread wieder in den Zustand bereit 
\end{itemize}
~

Threads starten

eine gegebene Methode TestMethod, soll nebenläufig in einem neuen Thread abgearbeitet werden:

~

using System.Threading;

…

ThreadStart del = new ThreadStart(TestMethod);

Thread thread = new Thread(del);

thread.Start();

~

Alternativ: Thread starten mit Lambda-Expressions

~

Thread thread = new Thread( () ={\textgreater} TestMethod() );

Thread thread = new Thread( () ={\textgreater} TestMethod(param1, param2, …) );

// Achtung: Parameter werden erst zum Start des Threads ausgewertet und dürfen sich bis dahin nicht ändern 

~

Ein einfaches Beispiel:

using System.Threading;

~

class Program \{

 static void Main(string[] args) \{

 Thread thread = new Thread(new ThreadStart(TestMethod));

 thread.Start();

 for(int i = 0; i {\textless}= 100; i++) \{

 for(int k = 1; k {\textless}= 5; k++)

 Console.Write({\textquotedbl}.{\textquotedbl});

 Console.WriteLine({\textquotedbl}Primär-Thread {\textquotedbl} + i);

 \}

 \}

~

 public static void TestMethod() \{

 for(int i = 0; i {\textless}= 100; i++) \{

 for(int k = 1; k {\textless}= 5; k++)

 Console.Write({\textquotedbl}X{\textquotedbl});

 Console.WriteLine({\textquotedbl}Sekundär-Thread {\textquotedbl} + i);

 \}

 \}

\}

~

Ausgabebeispiel:

~

Weitere Methoden von Thread

\begin{itemize}
\item Thread.Sleep(int time)  // Thread schläft für time Millisekunden
\item thread.Abort() // Thread bricht ab, nicht mehr unterstützt in .Net Core
\item thread.Join() // Blockiert den aufrufenden Thread (!=thread), bis der Thread thread beendet ist 
\end{itemize}
\endinput
