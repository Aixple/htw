\newcommand{\customDir}{../}
\RequirePackage{ifthen,xifthen}

% Input inkl. Umlaute, Silbentrennung
\RequirePackage[T1]{fontenc}
\RequirePackage[utf8]{inputenc}

% Arbeitsordner (in Abhängigkeit vom Master) Standard: .LateX_master Ordner liegt im Eltern-Ordner
\providecommand{\customDir}{../}
\newcommand{\setCustomDir}[1]{\renewcommand{\customDir}{#1}}
%%% alle Optionen:
% Doppelseitig (mit Rand an der Innenseite)
\newboolean{twosided}
\setboolean{twosided}{false}
% Eigene Dokument-Klasse (alle KOMA möglich; cheatsheet für Spicker [3 Spalten pro Seite, alles kleiner])
\newcommand{\customDocumentClass}{scrreprt}
\newcommand{\setCustomDocumentClass}[1]{\renewcommand{\customDocumentClass}{#1}}
% Unterscheidung verschiedener Designs: htw, fjs
\newcommand{\customDesign}{htw}
\newcommand{\setCustomDesign}[1]{\renewcommand{\customDesign}{#1}}
% Dokumenten Metadaten
\newcommand{\customTitle}{}
\newcommand{\setCustomTitle}[1]{\renewcommand{\customTitle}{#1}}
\newcommand{\customSubtitle}{}
\newcommand{\setCustomSubtitle}[1]{\renewcommand{\customSubtitle}{#1}}
\newcommand{\customAuthor}{}
\newcommand{\setCustomAuthor}[1]{\renewcommand{\customAuthor}{#1}}
%	Notiz auf der Titelseite (A: vor Autor, B: nach Autor)
\newcommand{\customNoteA}{}
\newcommand{\setCustomNoteA}[1]{\renewcommand{\customNoteA}{#1}}
\newcommand{\customNoteB}{}
\newcommand{\setCustomNoteB}[1]{\renewcommand{\customNoteB}{#1}}
% Format der Signatur in Fußzeile:
\newcommand{\customSignature}{\ifthenelse{\equal{\customAuthor}{}} {} {\footnotesize{\textcolor{darkgray}{Mitschrift von\\ \customAuthor}}}}
\newcommand{\setCustomSignature}[1]{\renewcommand{\customSignature}{#1}}
% Format des Autors auf dem Titelblatt:
\newcommand{\customTitleAuthor}{\textcolor{darkgray}{Mitschrift von \customAuthor}}
\newcommand{\setCustomTitleAuthor}[1]{\renewcommand{\customTitleAuthor}{#1}}
% Standard Sprache
\newcommand{\customDefaultLanguage}[1]{}
\newcommand{\setCustomDefaultLanguage}[1]{\renewcommand{\customDefaultLanguage}{#1}}
% Folien-Pfad (inkl. Dateiname ohne Endung und ggf. ohne Nummerierung)
\newcommand{\customSlidePath}{}
\newcommand{\setCustomSlidePath}[1]{\renewcommand{\customSlidePath}{#1}}
% Folien Eigenschaften
\newcommand{\customSlideScale}{0.5}
\newcommand{\setCustomSlideScale}[1]{\renewcommand{\customSlideScale}{#1}}
\newcommand{\customSlideHeight}{9.63cm}
\newcommand{\setCustomSlideHeight}[1]{\renewcommand{\customSlideHeight}{#1}}
\newcommand{\customSlideWidth}{12.8cm}
\newcommand{\setCustomSlideWidth}[1]{\renewcommand{\customSlideWidth}{#1}}

%\setboolean{twosided}{true}
%\setCustomDocumentClass{scrartcl}
%\setCustomDesign{htw}
%\setCustomSlidePath{Folien}

\setCustomTitle{Elektronik}
\setCustomSubtitle{Vorlesungsskript}
\setCustomAuthor{Falk-Jonatan Strube}
\setCustomNoteA{(Auszug)}
\setCustomNoteB{Vorlesung von Prof. Dr.-Ing. Flach}

%\setcustomSignature{\footnotesize{\textcolor{darkgray}{Mitschrift von\\ \customAuthor}}	% Formatierung der Signatur in der Fußzeile
%\setcustomTitleAuthor{\textcolor{darkgray}{Mitschrift von #1}}	% Formatierung des Autors auf dem Titelblatt

%-- Prüfen, ob Beamer
\ifthenelse{\equal{\customDocumentClass}{beamer}}{
%%% TODO: andere Layouts für Beamer außer HTW
	\documentclass[ignorenonframetext, 11pt, table]{beamer}
	
	\usenavigationsymbolstemplate{}
	\setbeamercolor{author in head/foot}{fg=black}
	\setbeamercolor{title}{fg=black}
	\setbeamercolor{bibliography entry author}{fg=htworange!70}
	%\setbeamercolor{bibliography entry title}{fg=blue} 
	\setbeamercolor{bibliography entry location}{fg=htworange!60} 
	\setbeamercolor{bibliography entry note}{fg=htworange!60}  
	
	\setbeamertemplate{itemize item}{\color{black}$\bullet$}
	\setbeamertemplate{itemize subitem}{\color{black}--}
	\setbeamertemplate{itemize subsubitem}{\color{black}$\bullet$}
	\makeatother
	\setbeamertemplate{footline}
	{
	\leavevmode
	\def\arraystretch{1.2}
	\arrayrulecolor{gray}
	\begin{tabular}{ p{0.167\textwidth} | p{0.491\textwidth} | p{0.089\textwidth} | p{0.103\textwidth}}
	\hline
	\strut\insertshortauthor & \insertshorttitle & Slide \insertframenumber{}% / \inserttotalframenumber{}
	 & May 4, 2016\\
	\end{tabular}
	}
	\setbeamertemplate{headline}
	{
	\leavevmode
	\setlength{\arrayrulewidth}{1pt}
	\hspace*{2em}	
	\begin{tabular}{p{0.63\textwidth}}
	\rule{0pt}{3em}\normalsize{\textbf{\insertsection\strut}}\\
	\arrayrulecolor{htworange}
	\hline
	\end{tabular}
	\begin{tabular}{l}
	\rule{0pt}{4em}\includegraphics[width=3.25cm]{\customDir .LaTeX_master/HTW_GESAMTLOGO_CMYK.eps}\\
	\end{tabular}
	}
	\makeatletter	
}{	
	%-- Für Spicker einiges anders:
	\ifthenelse{\equal{\customDocumentClass}{cheatsheet}}{
		\documentclass[a4paper,10pt,landscape]{scrartcl}
		\usepackage{geometry}
		\geometry{top=2mm, bottom=2mm, headsep=0mm, footskip=0mm, left=2mm, right=2mm}
		
		% Für Spicker \spsection für Section, zur Strukturierung \HRule oder \HDRule Linie einsetzen
		\usepackage{multicol}
		\newcommand{\spsection}[1]{\textbf{#1}}	% Platzsparende "section" für Spicker
	}{	%-- Ende Spicker-Unterscheidung-if
		%-- Unterscheidung Doppelseitig
		\ifthenelse{\boolean{twosided}}{
			\documentclass[a4paper,11pt, footheight=26pt,twoside]{\customDocumentClass}
			\usepackage[head=23pt]{geometry}	% head=23pt umgeht Fehlerwarnung, dafür größeres "top" in geometry
			\geometry{top=30mm, bottom=22mm, headsep=10mm, footskip=12mm, inner=27mm, outer=13mm}
		}{
			\documentclass[a4paper,11pt, footheight=26pt]{\customDocumentClass}
			\usepackage[head=23pt]{geometry}	% head=23pt umgeht Fehlerwarnung, dafür größeres "top" in geometry
			\geometry{top=30mm, bottom=22mm, headsep=10mm, footskip=12mm, left=20mm, right=20mm}
		}
		%-- Nummerierung bis Subsubsection für Report
		\ifthenelse{\equal{\customDocumentClass}{report} \OR \equal{\customDocumentClass}{scrreprt}}{
			\setcounter{secnumdepth}{3}	% zählt auch subsubsection
			\setcounter{tocdepth}{3}	% Inhaltsverzeichnis bis in subsubsection
		}{}
	}%-- Ende Spicker-Unterscheidung-else
	
	\usepackage{scrlayer-scrpage}	% Kopf-/Fußzeile
	\renewcommand*{\thefootnote}{\fnsymbol{footnote}}	% Fußnoten-Symbole anstatt Zahlen
	\renewcommand*{\titlepagestyle}{empty} % Keine Seitennummer auf Titelseite
	\usepackage[perpage]{footmisc}	% Fußnotenzählung Seitenweit, nicht Dokumentenweit
}

% Input inkl. Umlaute, Silbentrennung
\RequirePackage[T1]{fontenc}
\RequirePackage[utf8]{inputenc}
\usepackage[english,ngerman]{babel}
\usepackage{csquotes}	% Anführungszeichen
\RequirePackage{marvosym}
\usepackage{eurosym}

% Style-Aufhübschung
\usepackage{soul, color}	% Kapitälchen, Unterstrichen, Durchgestrichen usw. im Text
%\usepackage{titleref}

% Mathe usw.
\usepackage{amssymb}
\usepackage{amsthm}
\ifthenelse{\equal{\customDocumentClass}{beamer}}{}{
\usepackage[fleqn,intlimits]{amsmath}	% fleqn: align-Umgebung rechtsbündig; intlimits: Integralgrenzen immer ober-/unterhalb
}
%\usepackage{mathtools} % u.a. schönere underbraces
\usepackage{xcolor}
\usepackage{esint}	% Schönere Integrale, \oiint vorhanden
\everymath=\expandafter{\the\everymath\displaystyle}	% Mathe Inhalte werden weniger verkleinert
\usepackage{wasysym}	% mehr Symbole, bspw \lightning
% Auch arcus-Hyperbolicus-Funktionen
\DeclareMathOperator{\arccot}{arccot}
\DeclareMathOperator{\arccosh}{arccosh}
\DeclareMathOperator{\arcsinh}{arcsinh}
\DeclareMathOperator{\arctanh}{arctanh}
\DeclareMathOperator{\arccoth}{arccoth} 
%\renewcommand{\int}{\int\limits}
%\usepackage{xfrac}	% mehr fracs: sfrac{}{}
\let\oldemptyset\emptyset	% schöneres emptyset
\let\emptyset\varnothing
%\RequirePackage{mathabx}	% mehr Symbole
\mathchardef\mhyphen="2D	% Hyphen in Math

% tikz usw.
\usepackage{tikz}
\usepackage{pgfplots}
\pgfplotsset{compat=1.11}	% Umgeht Fehlermeldung
\usetikzlibrary{graphs}
%\usetikzlibrary{through}	% ???
\usetikzlibrary{arrows}
\usetikzlibrary{arrows.meta}	% Pfeile verändern / vergrößern: \draw[-{>[scale=1.5]}] (-3,5) -> (-3,3);
\usetikzlibrary{automata,positioning} % Zeilenumbruch im Node node[align=center] {Text\\nächste Zeile} automata für Graphen
\usetikzlibrary{matrix}
\usetikzlibrary{patterns}	% Schraffierte Füllung
\usetikzlibrary{shapes.geometric}	% Polygon usw.
\tikzstyle{reverseclip}=[insert path={	% Inverser Clip \clip
	(current page.north east) --
	(current page.south east) --
	(current page.south west) --
	(current page.north west) --
	(current page.north east)}
% Nutzen: 
%\begin{tikzpicture}[remember picture]
%\begin{scope}
%\begin{pgfinterruptboundingbox}
%\draw [clip] DIE FLÄCHE, IN DER OBJEKT NICHT ERSCHEINEN SOLL [reverseclip];
%\end{pgfinterruptboundingbox}
%\draw DAS OBJEKT;
%\end{scope}
%\end{tikzpicture}
]	% Achtung: dafür muss doppelt kompliert werden!
\usepackage{graphpap}	% Grid für Graphen
\tikzset{every state/.style={inner sep=2pt, minimum size=2em}}
\usetikzlibrary{mindmap, backgrounds}
%\usepackage{tikz-uml}	% braucht Dateien: http://perso.ensta-paristech.fr/~kielbasi/tikzuml/

% Tabular
\usepackage{longtable}	% Große Tabellen über mehrere Seiten
\usepackage{multirow}	% Multirow/-column: \multirow{2[Anzahl der Zeilen]}{*[Format]}{Test[Inhalt]} oder \multicolumn{7[Anzahl der Reihen]}{|c|[Format]}{Test2[Inhalt]}
\renewcommand{\arraystretch}{1.3} % Tabellenlinien nicht zu dicht
\usepackage{colortbl}
\arrayrulecolor{gray}	% heller Tabellenlinien
\usepackage{array}	% für folgende 3 Zeilen (für Spalten fester breite mit entsprechender Ausrichtung):
\newcolumntype{L}[1]{>{\raggedright\let\newline\\\arraybackslash\hspace{0pt}}m{\dimexpr#1\columnwidth-2\tabcolsep-1.5\arrayrulewidth}}
\newcolumntype{C}[1]{>{\centering\let\newline\\\arraybackslash\hspace{0pt}}m{\dimexpr#1\columnwidth-2\tabcolsep-1.5\arrayrulewidth}}
\newcolumntype{R}[1]{>{\raggedleft\let\newline\\\arraybackslash\hspace{0pt}}m{\dimexpr#1\columnwidth-2\tabcolsep-1.5\arrayrulewidth}}
\usepackage{caption}	% Um auch unbeschriftete Captions mit \caption* zu machen

% Nützliches
\usepackage{verbatim}	% u.a. zum auskommentieren via \begin{comment} \end{comment}
\usepackage{tabto}	% Tabs: /tab zum nächsten Tab oder /tabto{.5 \CurrentLineWidth} zur Stelle in der Linie
\NumTabs{6}	% Anzahl von Tabs pro Zeile zum springen
\usepackage{listings} % Source-Code mit Tabs
\usepackage{lstautogobble} 
\ifthenelse{\equal{\customDocumentClass}{beamer}}{}{
\usepackage{enumitem}	% Anpassung der enumerates
%\setlist[enumerate,1]{label=(\arabic*)}	% global andere Enum-Items
\renewcommand{\labelitemiii}{$\scriptscriptstyle ^\blacklozenge$} % global andere 3. Item-Aufzählungszeichen
}
\newenvironment{anumerate}{\begin{enumerate}[label=(\alph*)]}{\end{enumerate}} % Alphabetische Aufzählung
\usepackage{letltxmacro} % neue Definiton von Grundbefehlen
% Nutzen:
%\LetLtxMacro{\oldemph}{\emph}
%\renewcommand{\emph}[1]{\oldemph{#1}}
\RequirePackage{xpatch}	% ua. Konkatenieren von Strings/Variablen (etoolbox)


% Einrichtung von lst
\lstset{
basicstyle=\ttfamily, 
%mathescape=true, 
%escapeinside=^^, 
autogobble, 
tabsize=2,
basicstyle=\footnotesize\sffamily\color{black},
frame=single,
rulecolor=\color{lightgray},
numbers=left,
numbersep=5pt,
numberstyle=\tiny\color{gray},
commentstyle=\color{gray},
keywordstyle=\color{green},
stringstyle=\color{orange},
morecomment=[l][\color{magenta}]{\#}
showspaces=false,
showstringspaces=false,
breaklines=true,
literate=%
    {Ö}{{\"O}}1
    {Ä}{{\"A}}1
    {Ü}{{\"U}}1
    {ß}{{\ss}}1
    {ü}{{\"u}}1
    {ä}{{\"a}}1
    {ö}{{\"o}}1
    {~}{{\textasciitilde}}1
}
\usepackage{scrhack} % Fehler umgehen
\def\ContinueLineNumber{\lstset{firstnumber=last}} % vor lstlisting. Zum wechsel zum nicht-kontinuierlichen muss wieder \StartLineAt1 eingegeben werden
\def\StartLineAt#1{\lstset{firstnumber=#1}} % vor lstlisting \StartLineAt30 eingeben, um bei Zeile 30 zu starten
\let\numberLineAt\StartLineAt

% BibTeX
\usepackage[backend=bibtex8, bibencoding=ascii,
%style=authortitle, citestyle=authortitle-ibid,
%doi=false,
%isbn=false,
%url=false
]{biblatex}	% BibTeX
\usepackage{makeidx}
%\makeglossary
%\makeindex

% Grafiken
\usepackage{graphicx}
\usepackage{epstopdf}	% eps-Vektorgrafiken einfügen
%\epstopdfsetup{outdir=\customDir}

% pdf-Setup
\usepackage{pdfpages}
\ifthenelse{\equal{\customDocumentClass}{beamer}}{}{
\usepackage[bookmarks,%
bookmarksopen=false,% Klappt die Bookmarks in Acrobat aus
colorlinks=true,%
linkcolor=black,%
citecolor=red,%
urlcolor=green,%
]{hyperref}
}

%-- Unterscheidung des Stils
\newcommand{\customLogo}{}
\newcommand{\customPreamble}{}
\ifthenelse{\equal{\customDesign}{htw}}{
	% HTW Corporate Design: Arial (Helvetica)
	\usepackage{helvet}
	\renewcommand{\familydefault}{\sfdefault}
	\renewcommand{\customLogo}{HTW-Logo}
	\renewcommand{\customPreamble}{HTW Dresden}
}{
% \renewcommand{\customLogo}{HTW-Logo.eps}
}

% Nach Dokumentenbeginn ausführen:
\AtBeginDocument{
	% Autor und Titel für pdf-Eigenschaften festlegen, falls noch nicht geschehen
	\providecommand{\pdfAuthor}{John Doe}
	\ifdefempty{\customAuthor} {} {\renewcommand{\pdfAuthor}{\customAuthor}}
	\providecommand{\pdfTitle}{}
	\providecommand{\pdfTitleA}{}
	\providecommand{\pdfTitleB}{}
	\providecommand{\pdfTitleC}{}	
	\ifdefempty{\pdfTitle}{
		\ifdefempty{\customPreamble} {} {\renewcommand{\pdfTitleA}{\customPreamble{} | }}
		\ifdefempty{\customTitle} {\renewcommand{\pdfTitleB}{No Title}} {\renewcommand{\pdfTitleB}{\customTitle}}
		\ifdefempty{\customSubtitle} {} {\renewcommand{\pdfTitleC}{ - \customSubtitle}}
	}{}
	
	\newcommand{\customLogoLocation}{\customDir .LaTeX_master/\customLogo}
	\hypersetup{
		pdfauthor={\pdfAuthor},
		pdftitle={\pdfTitleA\pdfTitleB\pdfTitleC},
	}
	\ifthenelse{\equal{\customDocumentClass}{beamer}}{
		\title{\customTitle}
		\author{\customAuthor}
	}{
		\automark[section]{section}
		\automark*[subsection]{subsection}
		\pagestyle{scrheadings}
		\ifthenelse{\equal{\customDocumentClass}{report} \OR \equal{\customDocumentClass}{scrreprt}}{
		\renewcommand*{\chapterpagestyle}{scrheadings}
		}{}
		%\renewcommand*{\titlepagestyle}{scrheadings}
		\ihead{\includegraphics[height=1.7em]{\customLogoLocation}}
		%\ohead{\truncate{5cm}{\customTitle}}
		\ohead{\customTitle}
		\cfoot{\pagemark}
		\ofoot{\customSignature}
		% Titelseite
		\title{
		\includegraphics[width=0.35\textwidth]{\customDir .LaTeX_master/\customLogo}\\\vspace{0.5em}
		\Huge\textbf{\customTitle}
		\ifdefempty{\customSubtitle} {} {\\\vspace*{0.7em}\Large \customSubtitle}
		\\\vspace*{5em}}
		\author{
		\ifdefempty{\customNoteA} {} {\customNoteA \vspace*{1em}}\\ 
		\ifdefempty{\customAuthor} {} {\customTitleAuthor}
		\ifdefempty{\customNoteB}{}{\vspace*{1em}\\\customNoteB}
		}
		
		\ifthenelse{\equal{\customDocumentClass}{cheatsheet}}{
			\pagestyle{empty}
			\setlist{nolistsep}
	%		\usepackage{parskip}	% Aufzählung Abstand
	%		\setlength{\parskip}{0em}
			\lstset{
	    belowcaptionskip=0pt,
	    belowskip=0pt,
	    aboveskip=0pt,
			tabsize=2,
			frame=none,
			numbers=none,
			showspaces=false,
			showstringspaces=false,
			breaklines=true,
			}
		}{}
	}
}

% Unterabschnitte
%\newtheorem{example}{Beispiel}%[section]
%\newtheorem{definition}{Definition}[section]
%\newtheorem{discussion}{Diskussion}[section]
%\newtheorem{remark}{Bemerkung}[section]
%\newtheorem{proof}{Beweis}[section]
%\newtheorem{notation}{Schreibweise}[section]
\RequirePackage{xcolor}
\RequirePackage{amsmath}

% Horizontale Linie:
\newcommand{\HRule}[1][\medskipamount]{\par
  \vspace*{\dimexpr-\parskip-\baselineskip+#1}
  \noindent\rule[0.2ex]{\linewidth}{0.2mm}\par
  \vspace*{\dimexpr-\parskip-.5\baselineskip+#1}}
% Gestrichelte horizontale Linie:
\RequirePackage{dashrule}
\newcommand{\HDRule}[1][\medskipamount]{\par
  \vspace*{\dimexpr-\parskip-\baselineskip+#1}
  \noindent\hdashrule[0.2ex]{\linewidth}{0.2mm}{1mm} \par
  \vspace*{\dimexpr-\parskip-.5\baselineskip+#1}}
% Mathe in Anführungszeichen:
\newsavebox{\mathbox}\newsavebox{\mathquote}
\makeatletter
\newcommand{\mq}[1]{% \mathquotes{<stuff>}
  \savebox{\mathquote}{\text{"}}% Save quotes
  \savebox{\mathbox}{$\displaystyle #1$}% Save <stuff>
  \raisebox{\dimexpr\ht\mathbox-\ht\mathquote\relax}{"}#1\raisebox{\dimexpr\ht\mathbox-\ht\mathquote\relax}{''}
}
\makeatother

% Paragraph mit Zähler (Section-Weise)
\newcounter{cparagraphC}
\newcommand{\cparagraph}[1]{
\stepcounter{cparagraphC}
\paragraph{\thesection{}-\thecparagraphC{} #1}
%\addcontentsline{toc}{subsubsection}{\thesection{}-\thecparagraphC{} #1}
\label{\thesection-\thecparagraphC}
}
\makeatletter
\@addtoreset{cparagraphC}{section}
\makeatother


% (Vorlesungs-)Folien einbinden:
% Folien von einer Datei skaliert
\newcommand{\slide}[2][\customSlideScale]{\slides[#1]{}{#2}}
\newcommand{\slideTrim}[6][\customSlideScale]{\slides[#1 , clip,  trim = #5cm #4cm #6cm #3cm]{}{#2}}
% Folien von mehreren nummerierten Dateien skaliert
\newcommand{\slides}[3][\customSlideScale]{\begin{center}
\includegraphics[page=#3, scale=#1]{\customSlidePath #2.pdf}
\end{center}}

% \emph{} anders definieren
\makeatletter
\DeclareRobustCommand{\em}{%
  \@nomath\em \if b\expandafter\@car\f@series\@nil
  \normalfont \else \scshape \fi}
\makeatother

% unwichtiges
\newcommand{\unimptnt}[1]{{\transparent{0.5}#1}}

% alph. enumerate
\newenvironment{anumerate}{\begin{enumerate}[label=(\alph*)]}{\end{enumerate}} % Alphabetische Aufzählung

%% EINFACHE BEFEHLE

% Abkürzungen Mathe
\newcommand{\EE}{\mathbb{E}}
\newcommand{\QQ}{\mathbb{Q}}
\newcommand{\RR}{\mathbb{R}}
\newcommand{\CC}{\mathbb{C}}
\newcommand{\NN}{\mathbb{N}}
\newcommand{\ZZ}{\mathbb{Z}}
\newcommand{\PP}{\mathbb{P}}
\renewcommand{\SS}{\mathbb{S}}
\newcommand{\cA}{\mathcal{A}}
\newcommand{\cB}{\mathcal{B}}
\newcommand{\cC}{\mathcal{C}}
\newcommand{\cD}{\mathcal{D}}
\newcommand{\cE}{\mathcal{E}}
\newcommand{\cF}{\mathcal{F}}
\newcommand{\cG}{\mathcal{G}}
\newcommand{\cH}{\mathcal{H}}
\newcommand{\cI}{\mathcal{I}}
\newcommand{\cJ}{\mathcal{J}}
\newcommand{\cM}{\mathcal{M}}
\newcommand{\cN}{\mathcal{N}}
\newcommand{\cP}{\mathcal{P}}
\newcommand{\cR}{\mathcal{R}}
\newcommand{\cS}{\mathcal{S}}
\newcommand{\cZ}{\mathcal{Z}}
\newcommand{\cL}{\mathcal{L}}
\newcommand{\cT}{\mathcal{T}}
\newcommand{\cU}{\mathcal{U}}
\newcommand{\cV}{\mathcal{V}}
\renewcommand{\phi}{\varphi}
\renewcommand{\epsilon}{\varepsilon}

% Verschiedene als Mathe-Operatoren
\DeclareMathOperator{\arccot}{arccot}
\DeclareMathOperator{\arccosh}{arccosh}
\DeclareMathOperator{\arcsinh}{arcsinh}
\DeclareMathOperator{\arctanh}{arctanh}
\DeclareMathOperator{\arccoth}{arccoth} 
\DeclareMathOperator{\var}{Var} % Varianz 
\DeclareMathOperator{\cov}{Cov} % Co-Varianz 

% Farbdefinitionen
\definecolor{red}{RGB}{180,0,0}
\definecolor{green}{RGB}{75,160,0}
\definecolor{blue}{RGB}{0,75,200}
\definecolor{orange}{RGB}{255,128,0}
\definecolor{yellow}{RGB}{255,245,0}
\definecolor{purple}{RGB}{75,0,160}
\definecolor{cyan}{RGB}{0,160,160}
\definecolor{brown}{RGB}{120,60,10}

\definecolor{itteny}{RGB}{244,229,0}
\definecolor{ittenyo}{RGB}{253,198,11}
\definecolor{itteno}{RGB}{241,142,28}
\definecolor{ittenor}{RGB}{234,98,31}
\definecolor{ittenr}{RGB}{227,35,34}
\definecolor{ittenrp}{RGB}{196,3,125}
\definecolor{ittenp}{RGB}{109,57,139}
\definecolor{ittenpb}{RGB}{68,78,153}
\definecolor{ittenb}{RGB}{42,113,176}
\definecolor{ittenbg}{RGB}{6,150,187}
\definecolor{itteng}{RGB}{0,142,91}
\definecolor{ittengy}{RGB}{140,187,38}

\definecolor{htworange}{RGB}{249,155,28}

% Textfarbe ändern
\newcommand{\tred}[1]{\textcolor{red}{#1}}
\newcommand{\tgreen}[1]{\textcolor{green}{#1}}
\newcommand{\tblue}[1]{\textcolor{blue}{#1}}
\newcommand{\torange}[1]{\textcolor{orange}{#1}}
\newcommand{\tyellow}[1]{\textcolor{yellow}{#1}}
\newcommand{\tpurple}[1]{\textcolor{purple}{#1}}
\newcommand{\tcyan}[1]{\textcolor{cyan}{#1}}
\newcommand{\tbrown}[1]{\textcolor{brown}{#1}}

% Umstellen der Tabellen Definition
\newcommand{\mpb}[1][.3]{\begin{minipage}{#1\textwidth}\vspace*{3pt}}
\newcommand{\mpe}{\vspace*{3pt}\end{minipage}}

\newcommand{\resultul}[1]{\underline{\underline{#1}}}
\newcommand{\parskp}{$ $\\}	% new line after paragraph
\newcommand{\corr}{\;\widehat{=}\;}
\newcommand{\mdeg}{^{\circ}}

\newcommand{\nok}[2]{\binom{#1}{#2}}	% n über k BESSER: \binom{n}{k}
\newcommand{\mtr}[1]{\begin{pmatrix}#1\end{pmatrix}}	% Matrix
\newcommand{\dtr}[1]{\begin{vmatrix}#1\end{vmatrix}}	% Determinante (Betragsmatrix)
\renewcommand{\vec}[1]{\underline{#1}}	% Vektorschreibweise
\newcommand{\imptnt}[1]{\colorbox{red!30}{#1}}	% Wichtiges
\newcommand{\intd}[1]{\,\mathrm{d}#1}
\newcommand{\diffd}[1]{\mathrm{d}#1}
% für Module-Rechnung: \pmod{}
\newcommand{\unit}[1]{\,\mathrm{#1}}

%\bibliography{\customDir _Literatur/HTW_Literatur.bib}

\begin{document}

%\selectlanguage{english}
\maketitle
\newpage
\tableofcontents
\newpage

\chapter*{Einführung}

Passwort Materialien: lvf\_ws2015

Prüfung: 1 Blatt A4 hanbeschrieben, doppelseitig beschrieben

\chapter{Anliegen der Lehrveranstaltung}

\begin{tabular}{c c}
\mpb[0.4]
Analyse/Synthese
\begin{itemize}
\item Modellbildung
\item unterschiedliche Anregungen
\end{itemize}
\hspace*{1em}
\mpe& \mpb[0.4]
\begin{itemize}
\item Bauelemente\\
aktiv, passiv, Halbleiter
\item Netzwerke (linear, nichtlinear)
\item Schaltungen (analog, digital)
\end{itemize}
\mpe\\
\end{tabular}\\
\begin{tabular}{c c c}
&\boxed{SYSTEM}&\\ 
& analog/digital \emph{WIRKUNG}&\\
$\swarrow$&$\downarrow$&$\searrow$\\ \hline
Kommunikation \vline & Informationsverarbeitung \vline& Steuern / Regeln\\ \hline
&COMPUTER&\\
\hline
\end{tabular}

Informatik: automatisierte Informationsverarbeitung

\emph{FAZIT: Grundkenntnisse, gemeinsames Vokabular mit HW-Ingenieuren}

\chapter{Grundlagen der Elektrotechnik}

\section{Grundgrößen und Grundbeziehungen}

\paragraph{Bsp.:} Elektrophor mit Bernsteinplatte und Katzenfell\\
\begin{tikzpicture}[scale=0.8, remember picture]%, overlay]
\draw (0,0) -- (0,0.5);
\draw (6,0) -- (6,0.5);
\draw (3,0.5) ellipse (3 and 1);
\begin{scope}
\begin{pgfinterruptboundingbox}
\path[clip] (0,0) -- (6,0) -- (6,1.5) -- (0,1.5)-- cycle [reverseclip];
\end{pgfinterruptboundingbox}
\draw(3,0) ellipse (3 and 1);
\end{scope}
\begin{scope}
\draw[-latex] (3,-1.7) node[below]{Bernstein} -- (3,-1.2);
\draw[color=red] (3,-.8) node{$-$};
\draw[color=red] (2,-.7) node{$-$};
\draw[color=red] (1,-.5) node{$-$};
\draw[color=red] (4,-.7) node{$-$};
\draw[color=red] (5,-.5) node{$-$};


\draw (1,0.5) -- (1,0.8);
\draw (5,0.5) -- (5,0.8);
\draw(3,.8) ellipse (2 and .6);
\begin{pgfinterruptboundingbox}
\path[clip] (0,.5) -- (6,.5) -- (6,3.5) -- (0,3.5)-- cycle [reverseclip];
\end{pgfinterruptboundingbox}
\draw(3,0.5) ellipse (2 and .6);
\end{scope}
\draw[-latex] (5,2) node[above right]{Metallplatte} -- (4,1);
\draw[color=blue] (3,0) node{$+$};
\draw[color=blue] (3.7,.1) node{$+$};
\draw[color=blue] (4.4,.2) node{$+$};
\draw[color=blue] (2.3,.1) node{$+$};
\draw[color=blue] (1.6,.2) node{$+$};

\draw[very thick] (3,0.8) -- (3,2.5) node{$\bullet$};
\draw[-latex] (2.5,3) node[above left]{isoliert} -- (2.8,2.7);

\draw (7,-1) node[right]{+ Katzenfell};
\draw[color=red] (7.5,-.9) node[above right]{$- \; -$};
\end{tikzpicture}\\
Bernstein mit Katzenfell einreiben, dann Metallplatte anfassen: Elektronen fließen ab, Bernstein ist positiv geladen.\\
\begin{tikzpicture}[scale=0.8, remember picture]%, overlay]
\draw (0,0) -- (0,0.5);
\draw (6,0) -- (6,0.5);
\draw (3,0.5) ellipse (3 and 1);
\begin{scope}
\begin{pgfinterruptboundingbox}
\path[clip] (0,0) -- (6,0) -- (6,1.5) -- (0,1.5)-- cycle [reverseclip];
\end{pgfinterruptboundingbox}
\draw(3,0) ellipse (3 and 1);
\end{scope}
\draw[color=blue] (3,-.8) node{$+$};
\draw[color=blue] (2,-.7) node{$+$};
\draw[color=blue] (1,-.5) node{$+$};
\draw[color=blue] (4,-.7) node{$+$};
\draw[color=blue] (5,-.5) node{$+$};
\draw[very thick] (3,0.8) -- (3,2.5) node{$\bullet$};

\draw[red, -latex] (7.25,.5) -- (6.25,.5);
\begin{scope}
\begin{pgfinterruptboundingbox}
\draw[clip] (8,.25) ellipse (.5 and .5) [reverseclip];
\end{pgfinterruptboundingbox}
\draw (6,.25) -- (10,.25) -- (10,-2);
\draw (9.5,-2) -- (10.5,-2);
\end{scope}
\draw (8,.25) ellipse (.5 and .5);
\begin{scope}
\draw[clip] (8,.25) ellipse (.5 and .5);
\draw (7.5,-.25) -- (8.5,.75);
\draw (7.5,.75) -- (8.5,-.25);
\end{scope}

\draw (8.5,.25) node[above right]{kurzes Aufleuchten};
\end{tikzpicture}

\paragraph{Modellbildung:} Erklärung für beobachteten Sachverhalt
\begin{itemize}
\item möglichtst einfaches Modell
\item vollständige widerspruchsfreie Definition
\item Beschreibung über mathematische Gleichung
\end{itemize}

\paragraph{Bohr-Sommerfeldsches Atommodell:} \parskp
\begin{tikzpicture}[scale = 0.3, remember picture]
\begin{scope}
\begin{pgfinterruptboundingbox}
\draw[clip] (9,0) ellipse (1 and 1)[reverseclip];
\draw[clip] (-9,0) ellipse (1 and 1)[reverseclip];
\end{pgfinterruptboundingbox}
\draw[dashed] (0,0) ellipse (3 and 3);
\draw[dashed] (0,0) ellipse (9 and 9);
\end{scope}
\draw (9,0) node{$-$};
\draw (-9,0) node{$-$};

\draw (1.1,1.1) node{$+$} ellipse (1 and 1);
\draw (-1.1,-1.1) node{$+$} ellipse (1 and 1);

\draw[-latex] (8,-8) node[below right]{Kern (Protonen, pos. Ladungen)} -- (2.5,-2.5);
\draw[-latex] (13,1) node[right]{Elektron} -- (10.5,0);
\draw[-latex] (8,8) node[right]{Elektronenschale} -- (6.5,6.5);
\end{tikzpicture}

Atommodell ist elektrisch neutral. Aber:
\begin{itemize}
\item unter bestimmten Bediengungen entstehen positive und negative Ladungen (Energiezufuhr)
\item Elementarladung $e=1,6\cdot 10^{-19}C$
\end{itemize}

\paragraph{Beobachtung:} Ladungen ziehen sich an / Ladung stoßen sich ab.

\begin{tikzpicture} [scale=0.15]
\draw  plot[smooth cycle, tension=.7] coordinates {(-4,-2) (-1,-6) (5,-2) (5,4) (2,4) (0,2) (-3,1)};

\draw  plot[smooth cycle, tension=.7] coordinates {(12,-13) (12,-18) (16,-20) (22,-18) (23,-13) (18,-13)(15,-10)};

\node at (1,-1) {$+Q_1$};
\node at (16,-17) {$-Q_2$};
\draw[-latex] (3,-5) -- (6,-8);
\draw[latex-] (7,-9) -- (11,-13);

\node at (9,-7) {$\overrightarrow{F}$};
\draw[dashed] (7,0) -- (37,0);
\draw[dashed] (37,-16) -- (25,-16);

\draw [latex-latex] (35,-1)  -- (35,-15);
\node[right] at (35,-8) {Potentialdifferenz};
\node[right] at (37,0) {$\varphi_1$ (Potential)};
\node[right] at (37,-16) {$\varphi_1=0V$ (per Def.; Masseverbindung)};
\end{tikzpicture}

Kraftwirkung\\
$F \sim Q_1 \cdot Q_2$\\
$F \sim \frac{1}{r^2}$\\
\boxed{F=k\frac{Q_1 \cdot Q_2}{r^2}}

\section{Potential und Spannung}

\begin{itemize}
\item Ladungen im elektrischen Feld haben unterschiedliche Potenziale.
\item Einheit des Potenzial: Volt $[V]$
\item Spannung ist Potentialdifferenz
\item Einführen eines Bezugspotentials $\varphi = 0 V$
\end{itemize}

\paragraph{Beispiele für Spannungen:}
\begin{itemize}
\item Antennen … $\mu V$
\item Microfon … $m V $
\item Batterie (AA) … $1,2 V $
\item Netzteile … $\pm 5 V, \pm 12 V$
\item Haushalt … $230V$
\item Freileitungen … $380 kV$
\end{itemize}

\section{Stromfluss, Ladungsausgleich}

\begin{tikzpicture} [scale=0.15]

\draw  plot[smooth cycle, tension=.7] coordinates {(-5,5) (-9,2) (-6,-3) (3,-1) (3,3) (0,6)};

\draw  plot[smooth cycle, tension=.7] coordinates {(-5,-21) (-11,-26) (-5,-31) (1,-27) (3,-22)};
\node at (-3,3) {$Q_1$};
\node at (-4,-28) {$Q_2$};
\draw (0,0) node{$+$} ellipse (1 and 1);
\draw (-3,0) node{$+$} ellipse (1 and 1);
\draw (-6,0) node{$+$} ellipse (1 and 1);

\draw (0,-24) node{$-$} ellipse (1 and 1);
\draw (-3,-24) node{$-$} ellipse (1 and 1);
\draw (-6,-24) node{$-$} ellipse (1 and 1);

\draw[purple] (-6,-4) -- (-6,-19);

\draw[purple]  (-4,-15) node{$-$} ellipse (1 and 1);
\draw[purple]  (-3,-6) node{$+$} ellipse (1 and 1);

\draw[purple] (-1,-4) -- (-1,-19);
\draw[-latex, purple] (-3,-8) -- (-3,-11);

\draw[-latex ,purple] (-4,-13) -- (-4,-10);
\node[right, purple] at (-1,-11) {leitfähiger Kanal};

\draw[dashed] (7,2) -- (21,2);
\draw[dashed] (6,-27) -- (21,-27);

\draw[-latex] (21,1) -- (21,-26);
\node[right] at (21,-11) {Spannung $U$};
\end{tikzpicture}

\begin{itemize}
\item Strom $I=\frac{\Delta Q}{\Delta t}$, $i(t)=\frac{dQ}{dt}$
\item Ursache: Potentialdifferenz
\item Voraussetzung: leitfähiger Kanal, bewegliche Ladungen
\item „Fließgeschwindigkeit“ bestimmt Größe des Stroms
\end{itemize}

Analogie: Fluß\\
Höhenunterschied - Potential\\
Flussbett - Leitung\\
Wasser - Leiter

\section{Widerstand}

\paragraph{Beobachtung:} $I\sim U$, $I=G\cdot U$ mit \quad $G$ … Leitwert

je größer der Leitwert, desto kleiner der Widerstand $\Rightarrow G=\frac{1}{R}$ mit \quad $R$ … ohmscher Widerstand

\paragraph{Ohmsches Gesetz:}

$R\left(=\frac{U}{I}\right)=const. \quad U= R\cdot I \quad I = \frac{U}{R} $\\
mit $[I]=A$ (Ampere) \quad $[U]=V$ (Volt) \quad $[R]=\frac{V}{A}=\Omega$ (Ohm) \quad $[G]=\frac{A}{V} = S$ (Siemens)

Wiederstand ist eine Materialeigenschaft.\\
$R \sim l$ \quad $R\sim \frac{1}{A}$ \quad $R=k \cdot \frac{l}{A}$ mit  \quad $k=\varrho$ … spezifischer Widerstand $[\varrho]=\Omega \cdot m = \Omega \frac{mm^2}{m}$\\
$G=\frac{1}{R}=\frac{A}{\varrho \cdot l}=\frac{\kappa A}{l}$ mit \quad $\kappa = \frac{1}{\varrho}$\smallskip\\
Widerstand ist …
\begin{itemize}
\item Materialeigenschaft
\item Bauelement \\
\begin{tikzpicture} [scale=0.1]

\draw  (-4,3) rectangle (7,-1);
\draw (-11,1) -- (-4,1);

\draw (7,1) -- (12,1);
\node at (1,5) {$R$};

\draw (-7,0) -- (-6,1) -- (-7,2);
\node at (-7,4) {$I$};
\draw[-latex]  plot[smooth, tension=.7] coordinates {(-8,-3) (1,-5) (10,-3)};
\node at (1,-7) {$U$};
\end{tikzpicture}
\end{itemize}

\section{Zusammenschaltung von Widerständen}

\begin{enumerate} [label=\alph*)]
\item Reihenschaltung\\
\begin{tikzpicture}[scale=0.2]

\draw  (-9,5) rectangle (-3,3);
\node[above] at (-6,5) {$R_1$};
\draw[-latex]  plot[smooth, tension=1] coordinates {(-10,3) (-6,2) (-2,3)};
\node[below] at (-6,2) {$U_1$};
\draw  (2,5) rectangle (8,3);
\node[above] at (5,5) {$R_2$};
\draw[-latex]  plot[smooth, tension=1] coordinates {(1,3) (5,2) (9,3)};
\node[below] at (5,2) {$U_2$};
\draw  (26,5) rectangle (32,3);
\node[above] at (29,5) {$R_n$};
\draw[-latex]  plot[smooth, tension=1] coordinates {(25,3) (29,2) (33,3)};
\node[below] at (29,2) {$U_n$};
\draw (-16,4) -- (-9,4);
\draw (-3,4) -- (2,4);
\draw (8,4) -- (13,4);
\draw (26,4) -- (21,4);
\draw[dotted] (13,4) -- (21,4);
\draw (32,4) -- (40,4);
\node at (40,4) {$\bullet$};
\node at (-16,4) {$\bullet$};


\draw  (-12,10) rectangle (36,-3);
\node[above] at (12,10) {$R_{ges}$};
\draw[-latex] plot[smooth, tension=1] coordinates {(-14,-4) (9,-6) (37,-4)};
\node[below] at (9,-6) {$U_{ges}$};

\draw (-14,3) -- (-13,4) -- (-14,5);
\node at (-14,6) {$I$};
\end{tikzpicture}
\\
\emph{Maschensatz}: $\sum_{\circlearrowleft}U=0$
\begin{align*}
U_{ges}&= U_1 + U_2+...+U_n\\
U_{ges} &= I R_1+IR_2+...+IR_n\\
\frac{U_{ges}}{I}&=R_{ges}=R_1+R_2+...+R_n
\end{align*}
$\boxed{R_{ges}=\sum_{i=1}^n R_i}$
\item Parallelschaltung\\
\begin{tikzpicture}[scale=0.2]

\draw  (2,8) rectangle (8,6);
\node[above] at (5,8) {$R_1$};
\draw (-1,6) -- (0,7) -- (-1,8);
\node at (-1,9) {$I_1$};
\draw (-3,7) -- (2,7);
\draw (8,7) -- (13,7);
\draw  (2,-1) rectangle (8,-3);
\node[above] at (5,-1) {$R_2$};
\draw (-1,-3) -- (0,-2) -- (-1,-1);
\node at (-1,0) {$I_2$};
\draw (-3,-2) -- (2,-2);
\draw (8,-2) -- (13,-2);
\draw  (2,-15) rectangle (8,-17);
\node[above] at (5,-15) {$R_n$};
\draw (-1,-17) -- (0,-16) -- (-1,-15);
\node at (-1,-14) {$I_n$};
\draw (-3,-16) -- (2,-16);
\draw (8,-16) -- (13,-16);

\draw[dotted] (5,-6) -- (5,-11);
\draw (-3,7) -- (-3,-16);
\draw (13,7) -- (13,-16);
\draw (13,-4) -- (21,-4);
\draw (-3,-4) -- (-11,-4);
\draw (-8,-5) -- (-7,-4) -- (-8,-3);
\node at (-8,-2) {$I_{ges}$};
\draw[-latex]  plot[smooth, tension=1] coordinates {(-6,-17) (5,-19) (16,-17)};
\node[below] at (5,-19) {$U$};
\node at (13,-4) {$\bullet$};
\node at (13,-2) {$\bullet$};
\node at (-3,-2) {$\bullet$};
\node at (-3,-4) {$\bullet$};
\node at (21,-4) {$\circ$};
\node at (-11,-4) {$\circ$};
\end{tikzpicture}\\
\emph{Knotensatz:} $\sum_{\cdot}I=0$
\begin{align*}
I_{ges}&= I_1 + I_2 + ... + I_n\\
I_{ges} &= \frac{U}{R_1}+\frac{U}{R_2}+...+\frac{U}{R_n}\\
\frac{I_{ges}}{U}&=\frac{1}{R_{ges}}=\frac{1}{R_1}+\frac{1}{R_2}+...+\frac{1}{R_n}
\end{align*}
$\boxed{\frac{1}{R_{ges}}=\sum_{i=1}^n \frac{1}{R_i}}$
\end{enumerate}

\paragraph{Beispiele:} \parskp
\begin{tikzpicture}[scale=0.2]

\draw  (-9,5) rectangle (-3,3);
\node[above] at (-6,5) {$R_1$};

\draw  (2,5) rectangle (8,3);
\node[above] at (5,5) {$R_2$};

\draw (-14,4) -- (-9,4);
\draw (-3,4) -- (2,4);
\draw (8,4) -- (13,4);


\node at (13,4) {$\bullet$};
\node at (-14,4) {$\bullet$};

\end{tikzpicture}\\
$R_{ges}= R_1 + R_2\\
R_1=R_2 = R \quad \Rightarrow \quad R_{ges}=2R\\
R_1\gg R_2 \quad \Rightarrow \quad R_{ges}\approx R_1$\\
\begin{tikzpicture}[scale=0.2]

\draw  (-9,5) rectangle (-3,3);
\node[above] at (-6,5) {$R_1$};

\draw  (-9,-1) rectangle (-3,-3);
\node[above] at (-6,-1) {$R_2$};

\node at (0,1) {$\bullet$};
\node at (-12,1) {$\bullet$};
\node at (-16,1) {$\circ$};
\node at (4,1) {$\circ$};

\draw (-16,1) -- (-12,1);
\draw (0,1) -- (4,1);
\draw (-3,4) -- (0,4) -- (0,-2) -- (-3,-2);
\draw (-9,-2) -- (-12,-2) -- (-12,4) -- (-9,4);
\end{tikzpicture}\smallskip\\
$R_{ges}=\frac{R_1 R_2}{R_1+R_2}\\
R_1=R_2=R  \quad \Rightarrow \quad  R_{ges}=\frac{R}{2}\\
R_1 \gg R_2  \quad \Rightarrow \quad  R_{ges} \approx R_2$\\
\begin{tikzpicture}[scale=0.2]

\draw  (-9,-12) rectangle (-3,-14);
\node[above] at (-6,-12) {$R$};
\draw  (-9,5) rectangle (-3,3);
\node[above] at (-6,5) {$R$};
\draw  (5,-8) rectangle (3,-2);
\node[right] at (5,-5) {$R$};
\draw  (11,5) rectangle (17,3);
\node[above] at (14,5) {$R$};
\draw  (25,-8) rectangle (23,-2);
\node[right] at (25,-5) {$R$};
\draw  (31,5) rectangle (37,3);
\node[above] at (34,5) {$R$};
\draw  (45,-8) rectangle (43,-2);
\node[right] at (45,-5) {$R$};
\draw (-13,4) -- (-9,4);

\draw (-3,4) -- (11,4);
\draw (17,4) -- (31,4);
\draw (37,4) -- (44,4) -- (44,-2);
\draw (44,-8) -- (44,-13) -- (-3,-13);
\draw (24,4) -- (24,-2);
\draw (24,-8) -- (24,-13);
\draw (4,-8) -- (4,-13);
\draw (4,-2) -- (4,4);
\draw (-9,-13) -- (-13,-13);

\node  at (-13,4) {$\circ$};
\node at (-13,-13) {$\circ$};
\node at (4,4) {$\bullet$};
\node at (4,-13) {$\bullet$};
\node at (24,4) {$\bullet$};
\node at (24,-13) {$\bullet$};

\draw [dashed] (34,-5) ellipse (16 and 15);
\draw [dashed]  (25,-5) ellipse (26 and 21);
\node at (30,-17) {$R^*=R||2R$};
\node at (18,-22) {$R'=R||(R+R^*) $};
\end{tikzpicture}\\
$R^*=R||2R=\frac{2R \cdot R}{3R}=\frac{2}{3}R\\
R'=R|| (R+R^*)=\frac{R\cdot \frac{5}{3}R}{\frac{8}{3}R}=\frac{5}{8}R\\
R_{ges}=R+ R' + R = 2R + \frac{5}{8}R = \frac{21}{8}R$

\section{Leistung und Energie}
\begin{tikzpicture}[scale=0.2]

\draw  (5,-8) rectangle (3,-2);
\node[right] at (5,-5) {$R$};
\draw (4,3) -- (4,-2);
\draw (4,-8) -- (4,-12);

\draw (3,1) -- (4,0) -- (5,1);
\node at (7,1) {$I_{ges}$};
\node at (4,3) {$\bullet$};
\node at (4,-12) {$\bullet$};
\draw[-latex]  plot[smooth, tension=1] coordinates {(1,1) (-1,-5) (1,-11)};
\node at (-2,-5) {$U$};    

\node at (10,-4) {$T$};
\draw[-latex] (11,-4) -- (13,-2);
\end{tikzpicture}\\
$P=U\cdot I \overset{U=R\cdot I}{=}I^2 \cdot R = \frac{U^2}{R}\\
W=P\cdot t=U \cdot I \cdot t = \frac{U^2}{R}\cdot t = I^2 \cdot R \cdot t$

\section{Stromkreise und Schaltbilder}
\begin{itemize}
\item Modellierung elektronischer Erscheinungen
\item Berechnung von Stromkreisen
\end{itemize}
\includegraphics[scale=1.5]{Abbildungen/ABB207}

\paragraph{Bsp.:} Ein Kondensator wird zum Aufladen an eine Spannungsquelle mit dem Innenwiderstand $R_i$ angeschlossen und zum Entladen an einen Widerstand $R_E$. Das Laden erfolgt über den Strombegrenzungswiderstand $R_L$.
\begin{itemize}
\item Umschalter
\item Kondensater $C$, Widerstand $R_i$, $R_E$, $R_L$
\item Spannungsquelle
\end{itemize}
\includegraphics[scale=1.5]{Abbildungen/ABB208}

\chapter{Berechnung von Stromkreisen}

\section{Spannungsteiler}

\includegraphics[scale=1.5]{Abbildungen/ABB209}\\
$U_{R_1}=IR_1 \quad U_{R_2}=IR_2 \quad U_q= I(R_1+R_2)$\\
$\frac{U_{R_2}}{U_q}=\frac{R_2}{R_1+R_2} \quad \frac{U_{R_1}}{U_{R_2}}=\frac{R_1}{R_2}$\\
Anwendungsbeispiel: Potenziometer\\
\includegraphics[scale=1.5]{Abbildungen/ABB210}\\
$\frac{U_{out}}{U_{in}}=\frac{x\cdot R_{Pot}}{R_{Pot}}\Rightarrow U_{out} = x \cdot U_{in}$
\paragraph{belasteter Spannungsteiler:}\parskp
\includegraphics[scale=1.5]{Abbildungen/ABB211}\\
$\frac{U_{out}}{U_{in}}=\frac{R_2||R_L}{R_1+R_2||R_L}$\\
Bspw.: $R_1=5 \Omega$, $R_2=R_L=5 \Omega$\\
unbelasteter Fall: $\frac{U_{out}}{U_{in}}=\frac{5 \Omega}{10 \Omega} \Rightarrow U_{out}=5V$\\
belasteter Fall: $\frac{U_{out}}{U_{in}}=\frac{2,5 \Omega}{7,5 \Omega}\Rightarrow U_{out}=3,33V$

\paragraph{doppelter Spannungsteiler} \parskp
\includegraphics[scale=1.5]{Abbildungen/ABB212}\\
$\frac{U_{out}}{U_{in}}=\underbrace{\frac{R_2|| (R_3+R_4)}{R_1+R_2||(R_3+R_4}}_{\frac{U_{R_2}}{U_{in}}} \underbrace{\frac{R_4}{R_3+R_4}}_{=\frac{U_{out}}{U_{R_2}}}$

\paragraph{gesteuerter Spannungsteiler}\parskp
\includegraphics[scale=1.5]{Abbildungen/ABB301}

\section{Stromteiler}

\includegraphics[scale=1.5]{Abbildungen/ABB302}\\
$U_{out}=I_3 \cdot R_3 = I_2 \cdot R_2 = I_1 \cdot (R_2||R_3)$\\
$\frac{I_3}{I_2}=\frac{R_2}{R_3}$\\
$\frac{I_3}{I_1}=\frac{R_2||R_3}{R_3}=\frac{R_2 \cdot R_3}{(R_2+R_3)\cdot R_3}=\frac{R_2}{R_2+R_3}$\\
$\frac{I_2}{I_1}=\frac{R_2||R_3}{R_2}=\frac{R_3}{R_2+R_3}$

\subparagraph{Beispiel:}
geg.: \includegraphics[scale=1.5]{Abbildungen/ABB303}\\
ges.: $R_{AB}, R_{CD}, u_{out}$, alle Ströme\medskip\\
$R_{AB}=R_1+R_2||(R_3+R_4)$\\
$R_{CD}=R_4||(R_3+R_2)$\\
$\frac{U_{out}}{U_{h}}=\frac{R_4}{R_3+R_4}$
$\frac{U_h}{U_{in}}=\frac{R_2||(R_3+R_4)}{R_1+R_2||(R_3+R_4)}$\\
$U_{out}=U_{in}\frac{R_2||(R_3+R_4)}{R_1+R_2||(R_3+R_4)}\cdot\frac{R_4}{R_3+R_4}$\\
$I_1=\frac{U_{in}}{R_{AB}}$\\
$\frac{I_2}{I_1}=\frac{R_3+R_4}{R_2+R_3+R_4} \Rightarrow I_2=\frac{U_{in}}{R_{AB}}\cdot \frac{R_3+R_4}{R_2+R_3+R_4}$\\
$\frac{I_3}{I_1}=\frac{R_2}{R_2+R_3+R_4}\Rightarrow I_3 = \frac{U_{in}}{R_{AB}}\cdot \frac{R_2}{R_2+R_3+R_4}$

\section{Strom-Spannungskennlinie}
Ziel: anschauliche Beschreibung des Klemmverhaltens von Bauelementen\\
\includegraphics[scale=1.5]{Abbildungen/ABB304}
\begin{itemize}
\item Verbraucher: ohmscher Widerstand\\
\includegraphics[scale=1.5]{Abbildungen/ABB305}\\
$R=\frac{U}{I}$
$I=f(U)=\frac{1}{R}\cdot U = G \cdot U$
\item Verbraucher: Diode\\
\includegraphics[scale=1.5]{Abbildungen/ABB306}\\
? $\Rightarrow$ nichtlinear\\
$I=f(U)=I_s\left(e^{\frac{U}{U_T}}-1\right)$\\
$I_S$ … Sperrstrom\\
$U_T$ … Temperaturspannung 
\end{itemize}

\section{Spannungsquelle}

Was ist eine Spannungsquelle?

Batterie, Netzteil, Antenne, Mikrophon, Steckdose, …\smallskip\\
Unterteilung in:
\begin{itemize}
\item Signalquellen (irgendein $u(t)$, wenig Energie)
\item Spannungsquellen (Gleichspannung/Wechselspannung)
\end{itemize}
\emph{Modell:}
\begin{enumerate}
\item \includegraphics[scale=1.5]{Abbildungen/ABB307} Quelle im Leerlauf
\item \includegraphics[scale=1.5]{Abbildungen/ABB308} Quelle kurzgeschlossen
\end{enumerate}
Ersatzschaltbild einer realen Quelle:\\
\includegraphics[scale=1.5]{Abbildungen/ABB309} (mit $I_k=\frac{U_q}{R_i}$)\\
$U_q$ … Leerlaufspannung\\
$R_i$ … Innenwiderstand\\
$I_k$ … Kurzschlussstrom

\section{Grundstromkreis}
reale Quelle + Verbraucher\\
\includegraphics[scale=1.5]{Abbildungen/ABB310}\\
Strom-Spannungs-Kennlinienfeld des Grundstromkreises\\
Last: $I=f(U)=\frac{U_{AB}}{R_V}$\\
Quelle: $I=f(U)$ \\
(mit Maschensatz: $I\cdot R_i + U_{AB}-U_q=0$ \qquad $I=\frac{1}{R_i}(U_q-U_{AB})=-\frac{1}{R_i}U_{AB}+I_k$)\\
\includegraphics[scale=1.5]{Abbildungen/ABB311}\\
Grundstomkreis mit nichtlinearem Verbraucher\\
\includegraphics[scale=1.5]{Abbildungen/ABB312}\\
Leistung am Lastwiderstand\\
\includegraphics[scale=1.5]{Abbildungen/ABB313}\\
$\rightarrow P_V = I\cdot U_{AB}$ \quad mit $U_{AB}=U_q-I\cdot R_i$\quad $I\cdot R_i + U_{AB}-U_q=0$\\
$P_V=U_q\cdot I + I^2 \cdot R_i$ \quad $I=\frac{U_{AB}}{R_V}$\\
$P_V=f(R_V)$\\
maximale Leistung am Verbraucher: $\frac{d P_V}{dR_V}=0 \Rightarrow P_{V,max}$ für $R_V=R_i$ (dann $P_{V,max}=\frac{I_k \cdot U_q}{4}$)

\subsection{Betrachtung der Leistung im Grundstromkreis}

\includegraphics[scale=1.5]{Abbildungen/ABB401}\\
\begin{tabular}{l l}
\emph{Generator} & \emph{Verbraucher}\\
\mpb \begin{itemize}
\item soll sich nicht erwärmen
\item $P_i$ möglichst klein
\end{itemize} \mpe &
\mpb \begin{itemize}
\item $P_L$ möglichst groß
\item Wirkungsgrad groß
\end{itemize} \mpe
\end{tabular}\\
Leistung am Lastwiderstand:\\
$P_L=U_{AB}\cdot I = \frac{U_{AB}^2}{R_L}=I^2\cdot R_L$\\
mögliche Lastfälle:\\
\includegraphics[scale=1.5]{Abbildungen/ABB402}
\begin{enumerate}
\item Leerlauf: $R_L \rightarrow \infty$\\
$AP_{Leerlauf}$: $U_{AP}=U_q$, $I_{AP}=0$, $P_{AP}=0$
\item Kurzschluss: $R_L = 0$\\
$AP_{Kurzschluss}$: $U_{AP}=0$, $I_{AP}=I_K$, $P_{AP}=0$
\item großer Lastwiderstand: \\
$AP_{gr}$: $U_{AP}=U_{AP, gr}$, $I_{AP}=I_{AP, gr}$, $P_{AP, gr}>0$
\item kleiner Lastwiderstand: \\
$AP_{kl}$: $U_{AP}=U_{AP, kl}$, $I_{AP}=I_{AP, kl}$, $P_{AP, gr}>0$
\end{enumerate}
Zwei realistische Betriebsfälle:
\begin{itemize}
\item Wirkungsgrad groß, dafür nicht maximale Leistung
\item Wirkungsgrad bei 50\% und maximale Leistung
\end{itemize}

\section{Spannungszeitfunktion}
Verlauf einer Spannung über der Zeit.
\begin{itemize}
\item Gleichspannung $U= const \not = f(t)$ (Batterien, Stromversorgung für elektrische Geräte
\item Wechselspannung (Steckdose)\\
\includegraphics[scale=1.5]{Abbildungen/ABB403}\\
Kenngrößen: $\hat{U}=325V$ (Spitzenwert), $U=230V$ (Effektivwert), $f=50Hz$ ($T=\frac{1}{f}=20ms$, $\omega = 2 \pi f$[Kreisfrequenz]), $\varphi_0$ Phasenverschiebung/Nullphasenwinkel\\
$u(t)=\hat{U}\cdot sin(\omega t+\varphi_o)$
\begin{itemize}
\item zur Informationsübertragung können $\hat{U}$, $\omega$ und $\varphi_0$ variiert werden.
\item unterschiedliche Wechselspannungen können gemischt werden.
\item harmonischee Spannungen (bestehen aus Sinussschwingungen).
\end{itemize}
\end{itemize}

\paragraph{Grundtypen von Spannungszeitfunktionen}
\begin{itemize}
\item periodische Spannungen\\
\includegraphics[scale=1.5]{Abbildungen/ABB404}
\item impulsförmige Spannungen\\
\includegraphics[scale=1.5]{Abbildungen/ABB405}
\end{itemize}

\section{Kondensator, Kapazität}

Kapazität \\
$\rightarrow$ Fähigkeit, Ladungen zu speichern\\
$\rightarrow$ konkretes elektrisches Bauelement (kann Ladungen speichern) $\Rightarrow$ Kondensator \medskip\\
Einsatz: Energiespeicherung, Ausnutzung des frequenzabhängigen Verhaltens

\emph{Wirkungsweise:}\\
\includegraphics[scale=1.5]{Abbildungen/ABB406}\\
Beobachtung: $Q\sim U \Rightarrow Q=C\cdot U$ \\
mit $C = $ Proportionalitätsfaktor $\Rightarrow$ Kapazität $C$ mit $[C]=\frac{[Q]}{[U]}=\frac{As}{V}=F$ (Farrad)\\
relevante Werte: zwischen $\underset{\mu}{10^{-6}}... \underset{n}{10^{-9}}...\underset{p}{10^{-6}}F$
\paragraph{Bemessungsgleichung} (für C)\\
$C \sim A$, $C\sim \frac{1}{d}$, $C\sim  \frac{A}{d} \Rightarrow C=\varepsilon \cdot \frac{A}{d}$ \\
($\varepsilon = \varepsilon_r \cdot \varepsilon_0$ … $\varepsilon_0$: Dielektrizitätskonstate des Vakuums$=8,856\cdot 10^{-12}\frac{As}{Vm}$)\\
Symbol: \includegraphics[scale=1.5]{Abbildungen/ABB407}
\subsection{Strom-Spannungs-Beziehung am Kondensator}
$u_C(t)=\frac{1}{C}\int i_C(t) dt$\\
$i_C(t)=C\cdot \frac{du_c(t)}{dt}$\\
\emph{Konsequenzen:} $u(t)=\hat{U}\cdot sin (\omega t) \Rightarrow i_C(t)=\underbrace{\hat{U}\cdot \omega C}_{=\hat{I}}(sin (\omega t) + 90^{\circ})$\\
Scheinwiderstand des Kondensators: $X_C=\frac{\hat{U}}{\hat{I}}=\frac{\hat{U}}{\hat{U}\omega C}=\frac{1}{\omega C}$\\
\includegraphics[scale=1.5]{Abbildungen/ABB408}\\
Anwendung des frequenzebhängigen Verhaltens:\\
\includegraphics[scale=1.5]{Abbildungen/ABB409}




%\newpage
%\printbibliography
\end{document}