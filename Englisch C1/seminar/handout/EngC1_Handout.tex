% Header aus der Vorlage
\documentclass[a4paper,11pt, footheight=26pt
%,twoside
]{scrreprt}
\usepackage[head=23pt]{geometry}	% head=23pt umgeht Fehlerwarnung, dafür größeres "top" in geometry
\geometry{a4paper, top=30mm, bottom=22mm,headsep=10mm, footskip=12mm
, left=20mm, right=20mm
%, inner=27mm, outer=13mm
}

% Zeile 2 (,twoside) und 7 (inner=...) für eine Druckversion (doppelseitig) ent-kommentieren (Rand für Hefter)

\setcounter{secnumdepth}{3}	% zählt auch subsubsection
\setcounter{tocdepth}{3}	% Inhaltsverzeichnis bis in subsubsection

% Input inkl. Umlaute, Silbentrennung
\usepackage[T1]{fontenc}
\usepackage[utf8]{inputenc}
\usepackage[english]{babel}
\usepackage{csquotes}	% Anführungszeichen
\usepackage{eurosym}

% HTW Corporate Design: Arial (Helvetica)
\usepackage{helvet}
\renewcommand{\familydefault}{\sfdefault}

% Style-Aufhübschung
\usepackage{soul, color}	% Kapitälchen, Unterstrichen, Durchgestrichen usw. im Text
\usepackage{scrlayer-scrpage}	% Kopf-/Fußzeile
%\usepackage{titleref}
\usepackage[perpage]{footmisc}	% Fußnotenzählung Seitenweit, nicht Dokumentenweit
\renewcommand*{\thefootnote}{\fnsymbol{footnote}}	% Fußnoten-Symbole anstatt Zahlen
\renewcommand*{\titlepagestyle}{empty} % Keine Seitennummer auf Titelseite

% Mathe usw.
\usepackage{amssymb}
\usepackage[fleqn, intlimits]{amsmath}	% fleqn: align-Umgebung rechtsbündig; intlimits: Integralgrenzen immer ober-/unterhalb
%\usepackage{mathtools} % u.a. schönere underbraces
\usepackage{xcolor}
\usepackage{esint}	% Schönere Integrale, \oiint vorhanden
\everymath=\expandafter{\the\everymath\displaystyle}	% Mathe Inhalte werden weniger verkleinert
\usepackage{wasysym}	% mehr Symbole, bspw \lightning
% Auch arcus-Hyperbolicus-Funktionen
\DeclareMathOperator{\arccot}{arccot}
\DeclareMathOperator{\arccosh}{arccosh}
\DeclareMathOperator{\arcsinh}{arcsinh}
\DeclareMathOperator{\arctanh}{arctanh}
\DeclareMathOperator{\arccoth}{arccoth} 
% Mathe in Anführungszeichen:
\newsavebox{\mathbox}\newsavebox{\mathquote}
\makeatletter
\newcommand{\mq}[1]{% \mathquotes{<stuff>}
  \savebox{\mathquote}{\text{"}}% Save quotes
  \savebox{\mathbox}{$\displaystyle #1$}% Save <stuff>
  \raisebox{\dimexpr\ht\mathbox-\ht\mathquote\relax}{"}#1\raisebox{\dimexpr\ht\mathbox-\ht\mathquote\relax}{''}
}
\makeatother

% tikz usw.
\usepackage{tikz}
\usepackage{pgfplots}
\pgfplotsset{compat=1.11}	% Umgeht Fehlermeldung
\usetikzlibrary{graphs}
%\usetikzlibrary{through}	% ???
\usetikzlibrary{arrows}
\usetikzlibrary{arrows.meta}	% Pfeile verändern / vergrößern: \draw[-{>[scale=1.5]}] (-3,5) -> (-3,3);
\usetikzlibrary{automata,positioning} % Zeilenumbruch im Node node[align=center] {Text\\nächste Zeile} automata für Graphen
\usetikzlibrary{matrix}
\usetikzlibrary{patterns}	% Schraffierte Füllung
\tikzstyle{reverseclip}=[insert path={	% Inverser Clip \clip
	(current page.north east) --
	(current page.south east) --
	(current page.south west) --
	(current page.north west) --
	(current page.north east)}
% Nutzen: 
%\begin{tikzpicture}[remember picture]
%\begin{scope}
%\begin{pgfinterruptboundingbox}
%\draw [clip] DIE FLÄCHE, IN DER OBJEKT NICHT ERSCHEINEN SOLL [reverseclip];
%\end{pgfinterruptboundingbox}
%\draw DAS OBJEKT;
%\end{scope}
%\end{tikzpicture}
]	% Achtung: dafür muss doppelt kompliert werden!
\usepackage{graphpap}	% Grid für Graphen
\tikzset{every state/.style={inner sep=2pt, minimum size=2em}}

% Tabular
\usepackage{longtable}	% Große Tabellen über mehrere Seiten
\usepackage{multirow}	% Multirow/-column: \multirow{2[Anzahl der Zeilen]}{*[Format]}{Test[Inhalt]} oder \multicolumn{7[Anzahl der Reihen]}{|c|[Format]}{Test2[Inhalt]}
\renewcommand{\arraystretch}{1.3} % Tabellenlinien nicht zu dicht
\usepackage{colortbl}
\arrayrulecolor{gray}	% heller Tabellenlinien
\usepackage{array}	% für folgende 3 Zeilen (für Spalten fester breite mit entsprechender Ausrichtung):
\newcolumntype{L}[1]{>{\raggedright\let\newline\\\arraybackslash\hspace{0pt}}m{\dimexpr#1\columnwidth-2\tabcolsep-1.5\arrayrulewidth}}
\newcolumntype{C}[1]{>{\centering\let\newline\\\arraybackslash\hspace{0pt}}m{\dimexpr#1\columnwidth-2\tabcolsep-1.5\arrayrulewidth}}
\newcolumntype{R}[1]{>{\raggedleft\let\newline\\\arraybackslash\hspace{0pt}}m{\dimexpr#1\columnwidth-2\tabcolsep-1.5\arrayrulewidth}}

% Nützliches
\usepackage{verbatim}	% u.a. zum auskommentieren via \begin{comment} \end{comment}
\usepackage{tabto}	% Tabs: /tab zum nächsten Tab oder /tabto{.5 \CurrentLineWidth} zur Stelle in der Linie
\NumTabs{6}	% Anzahl von Tabs pro Zeile zum springen
\usepackage{listings} % Source-Code mit Tabs
\usepackage{lstautogobble} 
\usepackage{enumitem}	% Anpassung der enumerates
\setlist[enumerate,1]{label=\arabic*.)}	% global andere Enum-Items
\newenvironment{anumerate}{\begin{enumerate}[label=\alph*.)]}{\end{enumerate}} % Alphabetische Aufzählung
\renewcommand{\labelitemiii}{$\scriptscriptstyle ^\blacklozenge$} % global andere 3. Item-Aufzählungszeichen
\usepackage{letltxmacro} % neue Definiton von Grundbefehlen
% Nutzen:
%\LetLtxMacro{\oldemph}{\emph}
%\renewcommand{\emph}[1]{\oldemph{#1}}

% Einrichtung von lst
\lstset{
basicstyle=\ttfamily, 
mathescape=true, 
%escapeinside=^^, 
autogobble, 
tabsize=2,
basicstyle=\footnotesize\sffamily\color{black},
frame=single,
rulecolor=\color{lightgray},
numbers=left,
numbersep=5pt,
numberstyle=\tiny\color{gray},
commentstyle=\color{gray},
keywordstyle=\color{green},
stringstyle=\color{orange},
morecomment=[l][\color{magenta}]{\#}
%showspaces=false,
showstringspaces=false,
breaklines=true,
literate=%
    {Ö}{{\"O}}1
    {Ä}{{\"A}}1
    {Ü}{{\"U}}1
    {ß}{{\ss}}1
    {ü}{{\"u}}1
    {ä}{{\"a}}1
    {ö}{{\"o}}1
    {~}{{\textasciitilde}}1
}
\usepackage{scrhack} % Fehler umgehen
\def\ContinueLineNumber{\lstset{firstnumber=last}} % vor lstlisting. Zum wechsel zum nicht-kontinuierlichen muss wieder \StartLineAt1 eingegeben werden
\def\StartLineAt#1{\lstset{firstnumber=#1}} % vor lstlisting \StartLineAt30 eingeben, um bei Zeile 30 zu starten
\let\numberLineAt\StartLineAt

% BibTeX
\usepackage[backend=bibtex, bibencoding=ascii,
%style=authortitle, citestyle=authortitle-ibid,
%doi=false,
%isbn=false,
%url=false
]{biblatex}	% BibTeX
\usepackage{makeidx}
%\makeglossary
%\makeindex

% Grafiken
\usepackage{graphicx}
\usepackage{epstopdf}	% eps-Vektorgrafiken einfügen

% pdf-Setup
\usepackage{pdfpages}
\usepackage[bookmarks,%
bookmarksopen=false,% Klappt die Bookmarks in Acrobat aus
colorlinks=true,%
linkcolor=black,%
citecolor=red,%
urlcolor=black,%
]{hyperref}

% Titel, Autor usw. werden vor dem Anfang des Dokuments in einem Rutsch definiert…
\newcommand{\DTitel}[1]{\newcommand{\Dokumententitel}{#1}}
\newcommand{\DUntertitel}[1]{\newcommand{\Dokumentenuntertitel}{#1}}
\newcommand{\DAutor}[1]{\newcommand{\Dokumentenautor}{#1}}
\newcommand{\DNotiz}[1]{\newcommand{\Dokumentennotiz}{#1}}
\newcommand{\DSign}[1]{\newcommand{\Dokumentensignatur}{#1}}
\DSign{\footnotesize{\textcolor{darkgray}{Mitschrift von\\ \Dokumentenautor}}}
\newcommand{\Autorformat}[1]{\textcolor{darkgray}{Mitschrift von #1}}
\newcommand{\workingdir}{../}	% Arbeitsordner (in Abhängigkeit vom Master) Standard: LateX_master Ordner liegt im Eltern-Ordner
% … Deswegen folgendes erst Nach Dokumentenbeginn ausführen:
\AtBeginDocument{
	\hypersetup{
		pdfauthor={\Dokumentenautor},
		pdftitle={HTW Dresden | \Dokumententitel - \Dokumentenuntertitel},
	}
	\automark[section]{section}
	\automark*[subsection]{subsection}
	\pagestyle{scrheadings}
	\ihead{\includegraphics[height=1.7em]{\workingdir LaTeX_master/HTW-Logo.eps}}
	\ohead{\Dokumententitel}
	\cfoot{\pagemark}
	\ofoot{\Dokumentensignatur}
	% Titelseite
	\title{\includegraphics[width=0.35\textwidth]{\workingdir LaTeX_master/HTW-Logo.eps}\\\vspace{0.5em}
	\Huge\textbf{\Dokumententitel} \\\vspace*{0,5cm}
	\Large \Dokumentenuntertitel \\\vspace*{4cm}}
	\author{\Autorformat{\Dokumentenautor} \vspace*{1cm}\\\Dokumentennotiz}
}

%% EINFACHE BEFEHLE

% Abkürzungen Mathe
\newcommand{\EE}{\mathbb{E}}
\newcommand{\QQ}{\mathbb{Q}}
\newcommand{\RR}{\mathbb{R}}
\newcommand{\CC}{\mathbb{C}}
\newcommand{\NN}{\mathbb{N}}
\newcommand{\ZZ}{\mathbb{Z}}
\newcommand{\PP}{\mathbb{P}}
\renewcommand{\SS}{\mathbb{S}}
\newcommand{\cA}{\mathcal{A}}
\newcommand{\cB}{\mathcal{B}}
\newcommand{\cC}{\mathcal{C}}
\newcommand{\cD}{\mathcal{D}}
\newcommand{\cE}{\mathcal{E}}
\newcommand{\cF}{\mathcal{F}}
\newcommand{\cG}{\mathcal{G}}
\newcommand{\cH}{\mathcal{H}}
\newcommand{\cI}{\mathcal{I}}
\newcommand{\cJ}{\mathcal{J}}
\newcommand{\cM}{\mathcal{M}}
\newcommand{\cN}{\mathcal{N}}
\newcommand{\cP}{\mathcal{P}}
\newcommand{\cR}{\mathcal{R}}
\newcommand{\cS}{\mathcal{S}}
\newcommand{\cZ}{\mathcal{Z}}
\newcommand{\cL}{\mathcal{L}}
\newcommand{\cT}{\mathcal{T}}
\newcommand{\cU}{\mathcal{U}}
\newcommand{\cV}{\mathcal{V}}
\renewcommand{\phi}{\varphi}
\renewcommand{\epsilon}{\varepsilon}

% Farbdefinitionen
\definecolor{red}{RGB}{180,0,0}
\definecolor{green}{RGB}{75,160,0}
\definecolor{blue}{RGB}{0,75,200}
\definecolor{orange}{RGB}{255,128,0}
\definecolor{yellow}{RGB}{255,245,0}
\definecolor{purple}{RGB}{75,0,160}
\definecolor{cyan}{RGB}{0,160,160}
\definecolor{brown}{RGB}{120,60,10}

\definecolor{itteny}{RGB}{244,229,0}
\definecolor{ittenyo}{RGB}{253,198,11}
\definecolor{itteno}{RGB}{241,142,28}
\definecolor{ittenor}{RGB}{234,98,31}
\definecolor{ittenr}{RGB}{227,35,34}
\definecolor{ittenrp}{RGB}{196,3,125}
\definecolor{ittenp}{RGB}{109,57,139}
\definecolor{ittenpb}{RGB}{68,78,153}
\definecolor{ittenb}{RGB}{42,113,176}
\definecolor{ittenbg}{RGB}{6,150,187}
\definecolor{itteng}{RGB}{0,142,91}
\definecolor{ittengy}{RGB}{140,187,38}

% Textfarbe ändern
\newcommand{\tred}[1]{\textcolor{red}{#1}}
\newcommand{\tgreen}[1]{\textcolor{green}{#1}}
\newcommand{\tblue}[1]{\textcolor{blue}{#1}}
\newcommand{\torange}[1]{\textcolor{orange}{#1}}
\newcommand{\tyellow}[1]{\textcolor{yellow}{#1}}
\newcommand{\tpurple}[1]{\textcolor{purple}{#1}}
\newcommand{\tcyan}[1]{\textcolor{cyan}{#1}}
\newcommand{\tbrown}[1]{\textcolor{brown}{#1}}

% Umstellen der Tabellen Definition
\newcommand{\mpb}[1][.3]{\begin{minipage}{#1\textwidth}\vspace*{3pt}}
\newcommand{\mpe}{\vspace*{3pt}\end{minipage}}

\newcommand{\resultul}[1]{\underline{\underline{#1}}}
\newcommand{\parskp}{$ $\\}	% new line after paragraph
\newcommand{\corr}{\;\widehat{=}\;}
\newcommand{\mdeg}{^{\circ}}

\newcommand{\nok}[2]{\begin{pmatrix}#1\\#2\end{pmatrix}}	% n über k BESSER: \binom{n}{k}
\newcommand{\mtr}[1]{\begin{pmatrix}#1\end{pmatrix}}	% Matrix
\newcommand{\dtr}[1]{\begin{vmatrix}#1\end{vmatrix}}	% Determinante (Betragsmatrix)
\renewcommand{\vec}[1]{\underline{#1}}	% Vektorschreibweise
\newcommand{\imptnt}[1]{\colorbox{red!30}{#1}}	% Wichtiges
\newcommand{\intd}[1]{\,\mathrm{d}#1}
\renewcommand{\workingdir}{../../../}

\bibliography{../proposal/EngC1_Proposal_Literature.bib}

% Definition von Titel, Autor usw.
\DTitel{\textsc{Object oriented programming / modeling}}
\DUntertitel{Handout}
\DAutor{Falk-Jonatan Strube}
\DNotiz{}
\renewcommand{\Dokumentensignatur}{}
\renewcommand{\Autorformat}[1]{\textcolor{darkgray}{Seminar by}\\\textcolor{darkgray}{#1}}
\date{May 4, 2016} 

\begin{document}

\maketitle
\newpage
%\tableofcontents
%\newpage

\section*{Seminar outline}
\begin{itemize}
\item Selected glossary
\item Presentation of the topic
\begin{itemize}
\item Generic programming
\item Object oriented programming and modeling
\begin{itemize}
\item Inheritance
\item Polymorphism
\end{itemize}
\item Relevance of the topic
\item Sources
\end{itemize}
\item (Language) practice
\item Discussion
\end{itemize}
\section*{Glossary}
\begin{tabular}{L{0.22} L{0.39}  L{0.39}}
Term & Meaning & In this context\\
\hline
Generic & Relating to a whole group of similar things, rather than to any particular thing & "Normal", nothing special\\
Redundant & Being similar to something else and thereby not needed & Duplicate code\\
Class & A group with certain qualities & Code with a concept\\
Object & A thing & An implemented class \\
Implementation & Putting a plan into action / starting to use something & The concept of a class applied to an object\\
Inheritance & A characteristic passed down from your parents & A shared characteristic between classes\\
Polymorphism & Poly: many, morph: form & Inherited classes may have many different implementations\\
\end{tabular}
\newpage
\section*{Abstract}

With this project the author aims to inform his fellow students about the benefits of object oriented programming and modeling.\medskip

The research was based on items from a diverse bibliography including articles, books and websites. The most important piece of literature is the book \emph{Softwaretechnology for beginners} which is based on a lecture of the TU Dresden.\medskip

After reading the texts it can be stated that object oriented programming is an advanced way of structuring a computer program during development. By modeling dependencies and correlations in classes with the \emph{Unified Modeling Language} (UML), a complex code formation can be made understandable. This does not only benefit team communication and communication with a client, it also improves scalability since the general overview allows specific changes in the model and the program itself.

The key concepts of object oriented programming are inheritance and polymorphism. With inheritance, similar parts of a program may be reused in \emph{child}- and \emph{parent}-objects. Polymorphism allows the modification of a inherited function. Both features add to comprehensibility and scalability.\medskip

It can be concluded that object oriented programming is used in almost every complex computer program. From operating systems such as \emph{Windows} or \emph{Linux} to word processor applications such as \emph{Microsoft Word} -- object oriented programs are everywhere. Though generic programming is more commonly known, since it is easier to grasp at first, object oriented programming is more common overall.

\section*{(Language) practice}
Recall the UML-diagram from the presentation:
\begin{center}
\resizebox{.7\textwidth}{!}{
\begin{tikzpicture}
\tikzumlset{fill class=white!0}
\umlclass{Human}{hunger}{satisfy\_hunger()}
\umlclass[x=-3, y=-3, anchor=north]{Adult}{time\_available}{\strut}
\umlclass[x=3, y=-3, anchor=north]{Student}{motivation}{satisfy\_hunger()}
\umlVHVinherit{Adult}{Human} 
\umlVHVinherit{Student}{Human} 
\tikzumlset{fill class=white!0, text=gray, draw=gray}
\umlclass[x=8, y= -1] {Class name}{properties}{functions}
\end{tikzpicture}
}
\end{center}
\paragraph{Task:} 
\begin{anumerate}
\item Try to fill out the class names of the UML-diagram below based on the following sentences (use the diagram above as a reference on the structure).
\begin{itemize}
\item Students, alumni and employees of an university are persons.
\item Students can be beginners or long-term students.
\item A Professor is an employee.
\end{itemize}
\newpage
\item Now try to fill out the fields for properties and functions ($\corr$ actions).
\begin{itemize}
\item Think of properties or functions that a \emph{parent}-class has (and therefore the child-class too) $\Rightarrow$ inheritance. 
\item What new properties or functions may the child class(es) have, which do not fit to the parent?
\item What properties or functions may be \emph{overloaded} $\Rightarrow$ polymorphism?
\end{itemize}
\end{anumerate}

\begin{center}
\resizebox{.99\textwidth}{!}{
\begin{tikzpicture}
\tikzumlset{fill class=white!0, text=white!0}
\umlclass													 {PersonMMMMMM}{name\\\strut}{eat()\\ work() \\ sleep()}
\umlclass[x=-3, y=-4, anchor=north]{AlumnusMMMMM}{\strut\\\strut}{visitUniversity()\\ drinkBeer()}
\umlclass[x=3, y=-4, anchor=north] {StudentMMMMM}{\strut\\\strut}{visitLecture()\\ drinkBeer()}
\umlclass[x=9, y=-4, anchor=north] {EmployeeMMMM}{salary\\\strut}{\strut\\\strut}
\umlclass[x=9, y=-9, anchor=north] {ProfessorMMM}{\strut\\\strut}{giveLecture()\\\strut}
\umlclass[x=0, y=-9, anchor=north] {BeginnerMMMM}{\strut\\\strut}{visitEverything()\\\strut}
\umlclass[x=5, y=-9, anchor=north] {Long-term student}{salary\\\strut}{ignoreLecture()\\\strut}
\umlVHVinherit{AlumnusMMMMM}{PersonMMMMMM} 
\umlVHVinherit{StudentMMMMM}{PersonMMMMMM} 
\umlVHVinherit{EmployeeMMMM}{PersonMMMMMM} 
\umlVHVinherit{BeginnerMMMM}{StudentMMMMM} 
\umlVHVinherit{Long-term student}{StudentMMMMM}
\umlVHVinherit{ProfessorMMM}{EmployeeMMMM} 
\end{tikzpicture}
}
\end{center}

\section*{Discussion}
\begin{anumerate}
\item What do you think? Is object oriented modeling more comprehensive than the code from generic applications?
\item Have you written a (small) program before? What challenges do you remember? Could they be tackled better with object oriented programming?
\item Do you think the object oriented way of programming is used often in modern applications? Why? Why not?
\end{anumerate}

\section*{Mind map}
\resizebox{0.99\textwidth}{!}{
\begin{tikzpicture}[mindmap, grow cyclic, every node/.style=concept, concept color=black!10, 
    level 1/.append style={level distance=5cm,sibling angle=90},
    level 2/.append style={level distance=3cm,sibling angle=45},
    level 3/.append style={level distance=2.2cm,sibling angle=55},]

\node{Object oriented programming / modelling}
   child { node {Object orientated programming}
        child { node {Im\-ple\-men\-ta\-tion}}
        child { node {Effi\-cien\-cy}}
        child { node {reusabilty}}
        child { node {Pro\-gramm\-ing lang\-uag\-es}
        	child {node{Java}}
        	child {node{C++}}
        	child {node{C\#}}
        }
        child { node (psc) {scope (code)}}
        child { node (pcm) {com\-pre\-hen\-si\-bil\-ity (code)}}
        child { node {Difficult to understand (concept)}}
    }
    child [concept color=black!20] { node {Object oriented modelling}
        child { node {UML}}
        child { node {Com\-mu\-ni\-ca\-tion}
        	child {node{Team}}
        	child {node{Client}}
        }
        child { node {Scalability}}
        child { node (msc) {scope (project)}}
        child { node (mcm) {com\-pre\-hen\-si\-bil\-ity (project)}}
    }
    child [concept color=black!25] { node {Generic programming}
        child { node {Easy to understand (concept)}}
        child { node {Difficult to understand (code)}}
        child { node {Duplicate code}}
        child { node {Pro\-gramm\-ing lang\-uag\-es}
        	child {node{C}}
        	child {node{Haskell}}
        	child {node{SQL}}
        }
        child { node {Lacks scale}}
    }
    child [concept color=black!30] { node {Concepts}
        child [grow=60] { node {In\-heri\-tance}}
        child [grow=0] { node {Poly\-morphism}}
        child [grow=-60] { node {Objects and Classes}}
    };
\end{tikzpicture}
}
\newpage
\section*{Bibliography}
\nocite{*}
\printbibliography[heading=none]
\end{document}