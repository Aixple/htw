
%%%%%%%%%%%%%%%%%%%%%%%%%%%%%%%%%%%%%%%%%%%%%%%%%%%%%%%%%%%%%%%%%%%%%%%%%%%%
\subsection{Schwerpunkte}
\begin{frame}
	\begin{itemize}\pause
		\item Meinung der Probanden zum Test \pause
		\item Persönliche Befragung	\pause
		\item Antworten: \pause
		\begin{itemize}
			\item Skala \pause
			\item Anmerkung \pause
		\end{itemize}
		\item Zielgruppe: \pause
		\begin{itemize}
			\item Studenten / Berufstätige \pause
			\item Interesse an der Person / Vorlesungsmitschriften \pause
			\item Wichtel organisieren
		\end{itemize}
	\end{itemize}
	\note{
		\begin{itemize}
			\item um nicht gesehene Defizite und Probleme im Test zu erfragen
			\item Skala: normierte Einschätzung der Meinung $\rightarrow$ nicht so schwammig \\
					von 1 bis 7 (1 schlecht/schwer, 7 gut/leicht)
			\item Zielgruppen kommen am meisten in der Realität vor
		\end{itemize}}
\end{frame}

%%%%%%%%%%%%%%%%%%%%%%%%%%%%%%%%%%%%%%%%%%%%%%%%%%%%%%%%%%%%%%%%%%%%%%%%%%%%
\subsection{Zur Person}
\begin{frame}
	Beispielfragen: \pause
	\begin{itemize} 
		\item Name, Alter, Beruf \pause
		\item Wie lange und auf welchen Websites bist Du regelmäßig unterwegs? \pause
		\item Wie hoch schätzen Du deine Kenntnisse im Umgang mit Websites ein?
	\end{itemize}

	\note{
	\begin{itemize}
		\item Name, Alter,... $\rightarrow$ zur Einschätzung der Zielgruppe
		\item Einschätzung der Erfahrung mit verschiedener Websites\\
		Interesse in welche Websitetypen?
	\end{itemize}
	}
	
\end{frame}

%%%%%%%%%%%%%%%%%%%%%%%%%%%%%%%%%%%%%%%%%%%%%%%%%%%%%%%%%%%%%%%%%%%%%%%%%%%%
\subsection{Allgemeines}
\begin{frame}
	Beispielfragen: \pause
	\begin{itemize}
		\item Wie leicht fiel es Dir dich auf der Seite zu orientieren? \pause
		\item Konntest Du die Website auch gut auf dem Handy bedienen? \pause
		\item Wie würdest du die Seite beschreiben, was ist der Zweck der Seite?
	\end{itemize}

	\note{
	\begin{itemize}
		\item Funktionieren die gegeben Orientierungsmittel (Suche, Breadcrumbs, Layout, Menüleiste)
		\item Handy:
		\begin{itemize}
			\item Website ist auf Handy skaliert!!!
			\item Kommt der Nutzer genauso gut zurecht wie am Desktop? (Orientierungsmittel und Inhalte)
		\end{itemize}
		\item Welchen Eindruck hat die Website beim Nutzer hinterlassen? Inhalte verstanden?
	\end{itemize}
	}

\end{frame}

%%%%%%%%%%%%%%%%%%%%%%%%%%%%%%%%%%%%%%%%%%%%%%%%%%%%%%%%%%%%%%%%%%%%%%%%%%%%
\subsection{Zum Design}
\begin{frame}
	Beispielfragen: \pause
	\begin{itemize}
		\item Wie bewertest du das Design der Seite? \pause
		\begin{itemize}
			\item Gesamteindruck \pause
			\item Farbwahl \pause
			\item Anordnung der Inhalte \pause
			\item Angemessene Präsentation der Website im Bezug auf den Inhalt
		\end{itemize}
	\end{itemize}

\note{
\begin{itemize}
	\item uns interessiert rein visuelles Feedback
	\item Geschmacksfrage
	\item man erkennt Relation vom Design zu den Problemen im Test\\
	Orientierungsschwierigkeiten? 
\end{itemize}
}
\end{frame}


%%%%%%%%%%%%%%%%%%%%%%%%%%%%%%%%%%%%%%%%%%%%%%%%%%%%%%%%%%%%%%%%%%%%%%%%%%%%
\subsection{Wichtelsystem}
\begin{frame}
	Beispielfragen: \pause
	\begin{itemize}
		\item Sind Dir bei der Bedienung des Automatisierten Wichtel Systems Besonderheiten aufgefallen? \pause
	\end{itemize}
	\begin{center}
	\includegraphics[scale=0.13]{Aufzeichnungen/vlcsnap-2018-01-31-18h08m30s470} \pause
	\includegraphics[scale=0.13]{Aufzeichnungen/vlcsnap-2018-01-31-18h08m00s984}
	\note{
	\begin{itemize}
		\item Warum Befragung für Wichtelsystem minimal?
		\begin{itemize}
			\item Probleme der Probanden konnte gut beobachtet werden
			\item Design des Wichtelsystems entsprach dem der ganzen Seite
		\end{itemize}
		\item Möglichkeit auf Anmerkungen blieb trotzdem bestehen
		\item Testaufbau $\rightarrow$ links Mobil, rechts Desktop
	\end{itemize}}
\end{center}
\end{frame}
