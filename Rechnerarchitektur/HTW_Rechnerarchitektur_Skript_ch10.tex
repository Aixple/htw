\section{Einführung}
\slides{10}{2}
\slides{10}{3}

\section{Programmierte E/A}
\slides{10}{4}
\subsection{Memory Mapped I/O}
\slides{10}{5}
\subsection{I/O Mapping (Isolierte Adressierung)}
\slides{10}{6}
Weiterer Vorteil: Adressräume können dynamisch verschoben werden.
\unimptnt{
\subsection{Befehle}
\slides{10}{7}
}
\subsection[E/A Testschleife -> Polling (Busy Waiting)]{E/A Testschleife $\to$ Polling (Busy Waiting)}
\slides{10}{8}
\slides{10}{9}

\section{Interrupts}
\slides{10}{10}
\subsection{Grundmethode (technische Lösung)}
\slides{10}{11}
\unimptnt{
\subsection{Sprung zur ISR}
\slides{10}{12}
}
\subsection{Identifikation der Quelle}
\slides{10}{13}

\subsubsection{Polling}
\slides{10}{14}
\unimptnt{
\slides{10}{15}
Es wird rundum abgefragt, wer den Interrupt ausgelöst hat. Wenn einer gefunden wurde, wird an der Stelle weiter gesucht.
\slides{10}{16}
Es wird rundum abgefragt, wer den Interrupt ausgelöst hat. Wenn einer gefunden wurde, wird wieder von vorne rundum abgefragt (die letzten können ggf. nie erreicht werden).
}

\subsubsection{Daisy Chain}
\slides{10}{17}

\subsubsection{Multilevel Interrupt}
\slides{10}{18}

\subsection{Ermittlung der Adresse der ISR}
\slides{10}{19}
\subsubsection{Berechnung der ISR-Startadresse}
\slides{10}{21}

\subsection{(Ausnahme-)Vektortabelle}
\slides{10}{20}


\subsection{Maskierung}
\unimptnt{
\slides{10}{22}
}
\slides{10}{23}

\unimptnt{
\subsection{Programmierbare Interrupt-Controller (PIC)}
\slides{10}{24}
}

\subsection{Programmierte E/A mit Interrupts: Vor-/Nachteile}
\slides{10}{25}

\unimptnt{
\subsection{Noch mehr Verbesserungen?}
\slides{10}{26}
}

\section{Direct Memory Access (DMA)}
\slides{10}{27}
\subsection{Definition}
\slides{10}{28}

\subsection{DMA-Controller (DMAC)}
\slides{10}{29}

\subsection{Konfliktlösung bei Speicherzugriff}
\slides{10}{30}

\section{E/A-Prozessoren}
\subsection{Prinzip}
\slides{10}{31}

\unimptnt{
\subsection{Device Driver}
\slides{10}{32}
}

\subsection[Anschlussschemata IOP <=> Geräte]{Anschlussschemata IOP $\Leftrightarrow$ Geräte}
\slides{10}{33}