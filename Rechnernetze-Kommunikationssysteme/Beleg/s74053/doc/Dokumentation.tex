\newcommand{\customDir}{}
\RequirePackage{ifthen,xifthen}

% Input inkl. Umlaute, Silbentrennung
\RequirePackage[T1]{fontenc}
\RequirePackage[utf8]{inputenc}

% Arbeitsordner (in Abhängigkeit vom Master) Standard: .LateX_master Ordner liegt im Eltern-Ordner
\providecommand{\customDir}{../}
\newcommand{\setCustomDir}[1]{\renewcommand{\customDir}{#1}}
%%% alle Optionen:
% Doppelseitig (mit Rand an der Innenseite)
\newboolean{twosided}
\setboolean{twosided}{false}
% Eigene Dokument-Klasse (alle KOMA möglich; cheatsheet für Spicker [3 Spalten pro Seite, alles kleiner])
\newcommand{\customDocumentClass}{scrreprt}
\newcommand{\setCustomDocumentClass}[1]{\renewcommand{\customDocumentClass}{#1}}
% Unterscheidung verschiedener Designs: htw, fjs
\newcommand{\customDesign}{htw}
\newcommand{\setCustomDesign}[1]{\renewcommand{\customDesign}{#1}}
% Dokumenten Metadaten
\newcommand{\customTitle}{}
\newcommand{\setCustomTitle}[1]{\renewcommand{\customTitle}{#1}}
\newcommand{\customSubtitle}{}
\newcommand{\setCustomSubtitle}[1]{\renewcommand{\customSubtitle}{#1}}
\newcommand{\customAuthor}{}
\newcommand{\setCustomAuthor}[1]{\renewcommand{\customAuthor}{#1}}
%	Notiz auf der Titelseite (A: vor Autor, B: nach Autor)
\newcommand{\customNoteA}{}
\newcommand{\setCustomNoteA}[1]{\renewcommand{\customNoteA}{#1}}
\newcommand{\customNoteB}{}
\newcommand{\setCustomNoteB}[1]{\renewcommand{\customNoteB}{#1}}
% Format der Signatur in Fußzeile:
\newcommand{\customSignature}{\ifthenelse{\equal{\customAuthor}{}} {} {\footnotesize{\textcolor{darkgray}{Mitschrift von\\ \customAuthor}}}}
\newcommand{\setCustomSignature}[1]{\renewcommand{\customSignature}{#1}}
% Format des Autors auf dem Titelblatt:
\newcommand{\customTitleAuthor}{\textcolor{darkgray}{Mitschrift von \customAuthor}}
\newcommand{\setCustomTitleAuthor}[1]{\renewcommand{\customTitleAuthor}{#1}}
% Standard Sprache
\newcommand{\customDefaultLanguage}[1]{}
\newcommand{\setCustomDefaultLanguage}[1]{\renewcommand{\customDefaultLanguage}{#1}}
% Folien-Pfad (inkl. Dateiname ohne Endung und ggf. ohne Nummerierung)
\newcommand{\customSlidePath}{}
\newcommand{\setCustomSlidePath}[1]{\renewcommand{\customSlidePath}{#1}}
% Folien Eigenschaften
\newcommand{\customSlideScale}{0.5}
\newcommand{\setCustomSlideScale}[1]{\renewcommand{\customSlideScale}{#1}}
\newcommand{\customSlideHeight}{9.63cm}
\newcommand{\setCustomSlideHeight}[1]{\renewcommand{\customSlideHeight}{#1}}
\newcommand{\customSlideWidth}{12.8cm}
\newcommand{\setCustomSlideWidth}[1]{\renewcommand{\customSlideWidth}{#1}}

%\setboolean{twosided}{true}
%\setCustomDocumentClass{scrartcl}
%\setCustomDesign{htw}
%\setCustomSlidePath{Folien}

\setCustomTitle{Beleg UDP-Dateiübertragung}
\setCustomSubtitle{\texorpdfstring{Rechnernetze /\\ Kommunikationssysteme}{Rechnernetze / Kommunikationssysteme}}
\setCustomAuthor{Falk-Jonatan Strube}
%\setCustomNoteA{TitlepageNoteBeforeAuthor}
\setCustomNoteB{\textcolor{darkgray}{Vorlesung von Prof. Dr.-Ing. Vogt}}

\setCustomSignature{\textcolor{darkgray}{\customAuthor{}\\ s74053}}	% Formatierung der Signatur in der Fußzeile
\setCustomTitleAuthor{Beleg von \customAuthor{} (s74053)}	% Formatierung des Autors auf dem Titelblatt

%-- Prüfen, ob Beamer
\ifthenelse{\equal{\customDocumentClass}{beamer}}{
%%% TODO: andere Layouts für Beamer außer HTW
	\documentclass[ignorenonframetext, 11pt, table]{beamer}
	
	\usenavigationsymbolstemplate{}
	\setbeamercolor{author in head/foot}{fg=black}
	\setbeamercolor{title}{fg=black}
	\setbeamercolor{bibliography entry author}{fg=htworange!70}
	%\setbeamercolor{bibliography entry title}{fg=blue} 
	\setbeamercolor{bibliography entry location}{fg=htworange!60} 
	\setbeamercolor{bibliography entry note}{fg=htworange!60}  
	
	\setbeamertemplate{itemize item}{\color{black}$\bullet$}
	\setbeamertemplate{itemize subitem}{\color{black}--}
	\setbeamertemplate{itemize subsubitem}{\color{black}$\bullet$}
	\makeatother
	\setbeamertemplate{footline}
	{
	\leavevmode
	\def\arraystretch{1.2}
	\arrayrulecolor{gray}
	\begin{tabular}{ p{0.167\textwidth} | p{0.491\textwidth} | p{0.089\textwidth} | p{0.103\textwidth}}
	\hline
	\strut\insertshortauthor & \insertshorttitle & Slide \insertframenumber{}% / \inserttotalframenumber{}
	 & May 4, 2016\\
	\end{tabular}
	}
	\setbeamertemplate{headline}
	{
	\leavevmode
	\setlength{\arrayrulewidth}{1pt}
	\hspace*{2em}	
	\begin{tabular}{p{0.63\textwidth}}
	\rule{0pt}{3em}\normalsize{\textbf{\insertsection\strut}}\\
	\arrayrulecolor{htworange}
	\hline
	\end{tabular}
	\begin{tabular}{l}
	\rule{0pt}{4em}\includegraphics[width=3.25cm]{\customDir .LaTeX_master/HTW_GESAMTLOGO_CMYK.eps}\\
	\end{tabular}
	}
	\makeatletter	
}{	
	%-- Für Spicker einiges anders:
	\ifthenelse{\equal{\customDocumentClass}{cheatsheet}}{
		\documentclass[a4paper,10pt,landscape]{scrartcl}
		\usepackage{geometry}
		\geometry{top=2mm, bottom=2mm, headsep=0mm, footskip=0mm, left=2mm, right=2mm}
		
		% Für Spicker \spsection für Section, zur Strukturierung \HRule oder \HDRule Linie einsetzen
		\usepackage{multicol}
		\newcommand{\spsection}[1]{\textbf{#1}}	% Platzsparende "section" für Spicker
	}{	%-- Ende Spicker-Unterscheidung-if
		%-- Unterscheidung Doppelseitig
		\ifthenelse{\boolean{twosided}}{
			\documentclass[a4paper,11pt, footheight=26pt,twoside]{\customDocumentClass}
			\usepackage[head=23pt]{geometry}	% head=23pt umgeht Fehlerwarnung, dafür größeres "top" in geometry
			\geometry{top=30mm, bottom=22mm, headsep=10mm, footskip=12mm, inner=27mm, outer=13mm}
		}{
			\documentclass[a4paper,11pt, footheight=26pt]{\customDocumentClass}
			\usepackage[head=23pt]{geometry}	% head=23pt umgeht Fehlerwarnung, dafür größeres "top" in geometry
			\geometry{top=30mm, bottom=22mm, headsep=10mm, footskip=12mm, left=20mm, right=20mm}
		}
		%-- Nummerierung bis Subsubsection für Report
		\ifthenelse{\equal{\customDocumentClass}{report} \OR \equal{\customDocumentClass}{scrreprt}}{
			\setcounter{secnumdepth}{3}	% zählt auch subsubsection
			\setcounter{tocdepth}{3}	% Inhaltsverzeichnis bis in subsubsection
		}{}
	}%-- Ende Spicker-Unterscheidung-else
	
	\usepackage{scrlayer-scrpage}	% Kopf-/Fußzeile
	\renewcommand*{\thefootnote}{\fnsymbol{footnote}}	% Fußnoten-Symbole anstatt Zahlen
	\renewcommand*{\titlepagestyle}{empty} % Keine Seitennummer auf Titelseite
	\usepackage[perpage]{footmisc}	% Fußnotenzählung Seitenweit, nicht Dokumentenweit
}

% Input inkl. Umlaute, Silbentrennung
\RequirePackage[T1]{fontenc}
\RequirePackage[utf8]{inputenc}
\usepackage[english,ngerman]{babel}
\usepackage{csquotes}	% Anführungszeichen
\RequirePackage{marvosym}
\usepackage{eurosym}

% Style-Aufhübschung
\usepackage{soul, color}	% Kapitälchen, Unterstrichen, Durchgestrichen usw. im Text
%\usepackage{titleref}

% Mathe usw.
\usepackage{amssymb}
\usepackage{amsthm}
\ifthenelse{\equal{\customDocumentClass}{beamer}}{}{
\usepackage[fleqn,intlimits]{amsmath}	% fleqn: align-Umgebung rechtsbündig; intlimits: Integralgrenzen immer ober-/unterhalb
}
%\usepackage{mathtools} % u.a. schönere underbraces
\usepackage{xcolor}
\usepackage{esint}	% Schönere Integrale, \oiint vorhanden
\everymath=\expandafter{\the\everymath\displaystyle}	% Mathe Inhalte werden weniger verkleinert
\usepackage{wasysym}	% mehr Symbole, bspw \lightning
% Auch arcus-Hyperbolicus-Funktionen
\DeclareMathOperator{\arccot}{arccot}
\DeclareMathOperator{\arccosh}{arccosh}
\DeclareMathOperator{\arcsinh}{arcsinh}
\DeclareMathOperator{\arctanh}{arctanh}
\DeclareMathOperator{\arccoth}{arccoth} 
%\renewcommand{\int}{\int\limits}
%\usepackage{xfrac}	% mehr fracs: sfrac{}{}
\let\oldemptyset\emptyset	% schöneres emptyset
\let\emptyset\varnothing
%\RequirePackage{mathabx}	% mehr Symbole
\mathchardef\mhyphen="2D	% Hyphen in Math

% tikz usw.
\usepackage{tikz}
\usepackage{pgfplots}
\pgfplotsset{compat=1.11}	% Umgeht Fehlermeldung
\usetikzlibrary{graphs}
%\usetikzlibrary{through}	% ???
\usetikzlibrary{arrows}
\usetikzlibrary{arrows.meta}	% Pfeile verändern / vergrößern: \draw[-{>[scale=1.5]}] (-3,5) -> (-3,3);
\usetikzlibrary{automata,positioning} % Zeilenumbruch im Node node[align=center] {Text\\nächste Zeile} automata für Graphen
\usetikzlibrary{matrix}
\usetikzlibrary{patterns}	% Schraffierte Füllung
\usetikzlibrary{shapes.geometric}	% Polygon usw.
\tikzstyle{reverseclip}=[insert path={	% Inverser Clip \clip
	(current page.north east) --
	(current page.south east) --
	(current page.south west) --
	(current page.north west) --
	(current page.north east)}
% Nutzen: 
%\begin{tikzpicture}[remember picture]
%\begin{scope}
%\begin{pgfinterruptboundingbox}
%\draw [clip] DIE FLÄCHE, IN DER OBJEKT NICHT ERSCHEINEN SOLL [reverseclip];
%\end{pgfinterruptboundingbox}
%\draw DAS OBJEKT;
%\end{scope}
%\end{tikzpicture}
]	% Achtung: dafür muss doppelt kompliert werden!
\usepackage{graphpap}	% Grid für Graphen
\tikzset{every state/.style={inner sep=2pt, minimum size=2em}}
\usetikzlibrary{mindmap, backgrounds}
%\usepackage{tikz-uml}	% braucht Dateien: http://perso.ensta-paristech.fr/~kielbasi/tikzuml/

% Tabular
\usepackage{longtable}	% Große Tabellen über mehrere Seiten
\usepackage{multirow}	% Multirow/-column: \multirow{2[Anzahl der Zeilen]}{*[Format]}{Test[Inhalt]} oder \multicolumn{7[Anzahl der Reihen]}{|c|[Format]}{Test2[Inhalt]}
\renewcommand{\arraystretch}{1.3} % Tabellenlinien nicht zu dicht
\usepackage{colortbl}
\arrayrulecolor{gray}	% heller Tabellenlinien
\usepackage{array}	% für folgende 3 Zeilen (für Spalten fester breite mit entsprechender Ausrichtung):
\newcolumntype{L}[1]{>{\raggedright\let\newline\\\arraybackslash\hspace{0pt}}m{\dimexpr#1\columnwidth-2\tabcolsep-1.5\arrayrulewidth}}
\newcolumntype{C}[1]{>{\centering\let\newline\\\arraybackslash\hspace{0pt}}m{\dimexpr#1\columnwidth-2\tabcolsep-1.5\arrayrulewidth}}
\newcolumntype{R}[1]{>{\raggedleft\let\newline\\\arraybackslash\hspace{0pt}}m{\dimexpr#1\columnwidth-2\tabcolsep-1.5\arrayrulewidth}}
\usepackage{caption}	% Um auch unbeschriftete Captions mit \caption* zu machen

% Nützliches
\usepackage{verbatim}	% u.a. zum auskommentieren via \begin{comment} \end{comment}
\usepackage{tabto}	% Tabs: /tab zum nächsten Tab oder /tabto{.5 \CurrentLineWidth} zur Stelle in der Linie
\NumTabs{6}	% Anzahl von Tabs pro Zeile zum springen
\usepackage{listings} % Source-Code mit Tabs
\usepackage{lstautogobble} 
\ifthenelse{\equal{\customDocumentClass}{beamer}}{}{
\usepackage{enumitem}	% Anpassung der enumerates
%\setlist[enumerate,1]{label=(\arabic*)}	% global andere Enum-Items
\renewcommand{\labelitemiii}{$\scriptscriptstyle ^\blacklozenge$} % global andere 3. Item-Aufzählungszeichen
}
\newenvironment{anumerate}{\begin{enumerate}[label=(\alph*)]}{\end{enumerate}} % Alphabetische Aufzählung
\usepackage{letltxmacro} % neue Definiton von Grundbefehlen
% Nutzen:
%\LetLtxMacro{\oldemph}{\emph}
%\renewcommand{\emph}[1]{\oldemph{#1}}
\RequirePackage{xpatch}	% ua. Konkatenieren von Strings/Variablen (etoolbox)


% Einrichtung von lst
\lstset{
basicstyle=\ttfamily, 
%mathescape=true, 
%escapeinside=^^, 
autogobble, 
tabsize=2,
basicstyle=\footnotesize\sffamily\color{black},
frame=single,
rulecolor=\color{lightgray},
numbers=left,
numbersep=5pt,
numberstyle=\tiny\color{gray},
commentstyle=\color{gray},
keywordstyle=\color{green},
stringstyle=\color{orange},
morecomment=[l][\color{magenta}]{\#}
showspaces=false,
showstringspaces=false,
breaklines=true,
literate=%
    {Ö}{{\"O}}1
    {Ä}{{\"A}}1
    {Ü}{{\"U}}1
    {ß}{{\ss}}1
    {ü}{{\"u}}1
    {ä}{{\"a}}1
    {ö}{{\"o}}1
    {~}{{\textasciitilde}}1
}
\usepackage{scrhack} % Fehler umgehen
\def\ContinueLineNumber{\lstset{firstnumber=last}} % vor lstlisting. Zum wechsel zum nicht-kontinuierlichen muss wieder \StartLineAt1 eingegeben werden
\def\StartLineAt#1{\lstset{firstnumber=#1}} % vor lstlisting \StartLineAt30 eingeben, um bei Zeile 30 zu starten
\let\numberLineAt\StartLineAt

% BibTeX
\usepackage[backend=bibtex8, bibencoding=ascii,
%style=authortitle, citestyle=authortitle-ibid,
%doi=false,
%isbn=false,
%url=false
]{biblatex}	% BibTeX
\usepackage{makeidx}
%\makeglossary
%\makeindex

% Grafiken
\usepackage{graphicx}
\usepackage{epstopdf}	% eps-Vektorgrafiken einfügen
%\epstopdfsetup{outdir=\customDir}

% pdf-Setup
\usepackage{pdfpages}
\ifthenelse{\equal{\customDocumentClass}{beamer}}{}{
\usepackage[bookmarks,%
bookmarksopen=false,% Klappt die Bookmarks in Acrobat aus
colorlinks=true,%
linkcolor=black,%
citecolor=red,%
urlcolor=green,%
]{hyperref}
}

%-- Unterscheidung des Stils
\newcommand{\customLogo}{}
\newcommand{\customPreamble}{}
\ifthenelse{\equal{\customDesign}{htw}}{
	% HTW Corporate Design: Arial (Helvetica)
	\usepackage{helvet}
	\renewcommand{\familydefault}{\sfdefault}
	\renewcommand{\customLogo}{HTW-Logo}
	\renewcommand{\customPreamble}{HTW Dresden}
}{
% \renewcommand{\customLogo}{HTW-Logo.eps}
}

% Nach Dokumentenbeginn ausführen:
\AtBeginDocument{
	% Autor und Titel für pdf-Eigenschaften festlegen, falls noch nicht geschehen
	\providecommand{\pdfAuthor}{John Doe}
	\ifdefempty{\customAuthor} {} {\renewcommand{\pdfAuthor}{\customAuthor}}
	\providecommand{\pdfTitle}{}
	\providecommand{\pdfTitleA}{}
	\providecommand{\pdfTitleB}{}
	\providecommand{\pdfTitleC}{}	
	\ifdefempty{\pdfTitle}{
		\ifdefempty{\customPreamble} {} {\renewcommand{\pdfTitleA}{\customPreamble{} | }}
		\ifdefempty{\customTitle} {\renewcommand{\pdfTitleB}{No Title}} {\renewcommand{\pdfTitleB}{\customTitle}}
		\ifdefempty{\customSubtitle} {} {\renewcommand{\pdfTitleC}{ - \customSubtitle}}
	}{}
	
	\newcommand{\customLogoLocation}{\customDir .LaTeX_master/\customLogo}
	\hypersetup{
		pdfauthor={\pdfAuthor},
		pdftitle={\pdfTitleA\pdfTitleB\pdfTitleC},
	}
	\ifthenelse{\equal{\customDocumentClass}{beamer}}{
		\title{\customTitle}
		\author{\customAuthor}
	}{
		\automark[section]{section}
		\automark*[subsection]{subsection}
		\pagestyle{scrheadings}
		\ifthenelse{\equal{\customDocumentClass}{report} \OR \equal{\customDocumentClass}{scrreprt}}{
		\renewcommand*{\chapterpagestyle}{scrheadings}
		}{}
		%\renewcommand*{\titlepagestyle}{scrheadings}
		\ihead{\includegraphics[height=1.7em]{\customLogoLocation}}
		%\ohead{\truncate{5cm}{\customTitle}}
		\ohead{\customTitle}
		\cfoot{\pagemark}
		\ofoot{\customSignature}
		% Titelseite
		\title{
		\includegraphics[width=0.35\textwidth]{\customDir .LaTeX_master/\customLogo}\\\vspace{0.5em}
		\Huge\textbf{\customTitle}
		\ifdefempty{\customSubtitle} {} {\\\vspace*{0.7em}\Large \customSubtitle}
		\\\vspace*{5em}}
		\author{
		\ifdefempty{\customNoteA} {} {\customNoteA \vspace*{1em}}\\ 
		\ifdefempty{\customAuthor} {} {\customTitleAuthor}
		\ifdefempty{\customNoteB}{}{\vspace*{1em}\\\customNoteB}
		}
		
		\ifthenelse{\equal{\customDocumentClass}{cheatsheet}}{
			\pagestyle{empty}
			\setlist{nolistsep}
	%		\usepackage{parskip}	% Aufzählung Abstand
	%		\setlength{\parskip}{0em}
			\lstset{
	    belowcaptionskip=0pt,
	    belowskip=0pt,
	    aboveskip=0pt,
			tabsize=2,
			frame=none,
			numbers=none,
			showspaces=false,
			showstringspaces=false,
			breaklines=true,
			}
		}{}
	}
}

% Unterabschnitte
%\newtheorem{example}{Beispiel}%[section]
%\newtheorem{definition}{Definition}[section]
%\newtheorem{discussion}{Diskussion}[section]
%\newtheorem{remark}{Bemerkung}[section]
%\newtheorem{proof}{Beweis}[section]
%\newtheorem{notation}{Schreibweise}[section]
\RequirePackage{xcolor}
\RequirePackage{amsmath}

% Horizontale Linie:
\newcommand{\HRule}[1][\medskipamount]{\par
  \vspace*{\dimexpr-\parskip-\baselineskip+#1}
  \noindent\rule[0.2ex]{\linewidth}{0.2mm}\par
  \vspace*{\dimexpr-\parskip-.5\baselineskip+#1}}
% Gestrichelte horizontale Linie:
\RequirePackage{dashrule}
\newcommand{\HDRule}[1][\medskipamount]{\par
  \vspace*{\dimexpr-\parskip-\baselineskip+#1}
  \noindent\hdashrule[0.2ex]{\linewidth}{0.2mm}{1mm} \par
  \vspace*{\dimexpr-\parskip-.5\baselineskip+#1}}
% Mathe in Anführungszeichen:
\newsavebox{\mathbox}\newsavebox{\mathquote}
\makeatletter
\newcommand{\mq}[1]{% \mathquotes{<stuff>}
  \savebox{\mathquote}{\text{"}}% Save quotes
  \savebox{\mathbox}{$\displaystyle #1$}% Save <stuff>
  \raisebox{\dimexpr\ht\mathbox-\ht\mathquote\relax}{"}#1\raisebox{\dimexpr\ht\mathbox-\ht\mathquote\relax}{''}
}
\makeatother

% Paragraph mit Zähler (Section-Weise)
\newcounter{cparagraphC}
\newcommand{\cparagraph}[1]{
\stepcounter{cparagraphC}
\paragraph{\thesection{}-\thecparagraphC{} #1}
%\addcontentsline{toc}{subsubsection}{\thesection{}-\thecparagraphC{} #1}
\label{\thesection-\thecparagraphC}
}
\makeatletter
\@addtoreset{cparagraphC}{section}
\makeatother


% (Vorlesungs-)Folien einbinden:
% Folien von einer Datei skaliert
\newcommand{\slide}[2][\customSlideScale]{\slides[#1]{}{#2}}
\newcommand{\slideTrim}[6][\customSlideScale]{\slides[#1 , clip,  trim = #5cm #4cm #6cm #3cm]{}{#2}}
% Folien von mehreren nummerierten Dateien skaliert
\newcommand{\slides}[3][\customSlideScale]{\begin{center}
\includegraphics[page=#3, scale=#1]{\customSlidePath #2.pdf}
\end{center}}

% \emph{} anders definieren
\makeatletter
\DeclareRobustCommand{\em}{%
  \@nomath\em \if b\expandafter\@car\f@series\@nil
  \normalfont \else \scshape \fi}
\makeatother

% unwichtiges
\newcommand{\unimptnt}[1]{{\transparent{0.5}#1}}

% alph. enumerate
\newenvironment{anumerate}{\begin{enumerate}[label=(\alph*)]}{\end{enumerate}} % Alphabetische Aufzählung

%% EINFACHE BEFEHLE

% Abkürzungen Mathe
\newcommand{\EE}{\mathbb{E}}
\newcommand{\QQ}{\mathbb{Q}}
\newcommand{\RR}{\mathbb{R}}
\newcommand{\CC}{\mathbb{C}}
\newcommand{\NN}{\mathbb{N}}
\newcommand{\ZZ}{\mathbb{Z}}
\newcommand{\PP}{\mathbb{P}}
\renewcommand{\SS}{\mathbb{S}}
\newcommand{\cA}{\mathcal{A}}
\newcommand{\cB}{\mathcal{B}}
\newcommand{\cC}{\mathcal{C}}
\newcommand{\cD}{\mathcal{D}}
\newcommand{\cE}{\mathcal{E}}
\newcommand{\cF}{\mathcal{F}}
\newcommand{\cG}{\mathcal{G}}
\newcommand{\cH}{\mathcal{H}}
\newcommand{\cI}{\mathcal{I}}
\newcommand{\cJ}{\mathcal{J}}
\newcommand{\cM}{\mathcal{M}}
\newcommand{\cN}{\mathcal{N}}
\newcommand{\cP}{\mathcal{P}}
\newcommand{\cR}{\mathcal{R}}
\newcommand{\cS}{\mathcal{S}}
\newcommand{\cZ}{\mathcal{Z}}
\newcommand{\cL}{\mathcal{L}}
\newcommand{\cT}{\mathcal{T}}
\newcommand{\cU}{\mathcal{U}}
\newcommand{\cV}{\mathcal{V}}
\renewcommand{\phi}{\varphi}
\renewcommand{\epsilon}{\varepsilon}

% Verschiedene als Mathe-Operatoren
\DeclareMathOperator{\arccot}{arccot}
\DeclareMathOperator{\arccosh}{arccosh}
\DeclareMathOperator{\arcsinh}{arcsinh}
\DeclareMathOperator{\arctanh}{arctanh}
\DeclareMathOperator{\arccoth}{arccoth} 
\DeclareMathOperator{\var}{Var} % Varianz 
\DeclareMathOperator{\cov}{Cov} % Co-Varianz 

% Farbdefinitionen
\definecolor{red}{RGB}{180,0,0}
\definecolor{green}{RGB}{75,160,0}
\definecolor{blue}{RGB}{0,75,200}
\definecolor{orange}{RGB}{255,128,0}
\definecolor{yellow}{RGB}{255,245,0}
\definecolor{purple}{RGB}{75,0,160}
\definecolor{cyan}{RGB}{0,160,160}
\definecolor{brown}{RGB}{120,60,10}

\definecolor{itteny}{RGB}{244,229,0}
\definecolor{ittenyo}{RGB}{253,198,11}
\definecolor{itteno}{RGB}{241,142,28}
\definecolor{ittenor}{RGB}{234,98,31}
\definecolor{ittenr}{RGB}{227,35,34}
\definecolor{ittenrp}{RGB}{196,3,125}
\definecolor{ittenp}{RGB}{109,57,139}
\definecolor{ittenpb}{RGB}{68,78,153}
\definecolor{ittenb}{RGB}{42,113,176}
\definecolor{ittenbg}{RGB}{6,150,187}
\definecolor{itteng}{RGB}{0,142,91}
\definecolor{ittengy}{RGB}{140,187,38}

\definecolor{htworange}{RGB}{249,155,28}

% Textfarbe ändern
\newcommand{\tred}[1]{\textcolor{red}{#1}}
\newcommand{\tgreen}[1]{\textcolor{green}{#1}}
\newcommand{\tblue}[1]{\textcolor{blue}{#1}}
\newcommand{\torange}[1]{\textcolor{orange}{#1}}
\newcommand{\tyellow}[1]{\textcolor{yellow}{#1}}
\newcommand{\tpurple}[1]{\textcolor{purple}{#1}}
\newcommand{\tcyan}[1]{\textcolor{cyan}{#1}}
\newcommand{\tbrown}[1]{\textcolor{brown}{#1}}

% Umstellen der Tabellen Definition
\newcommand{\mpb}[1][.3]{\begin{minipage}{#1\textwidth}\vspace*{3pt}}
\newcommand{\mpe}{\vspace*{3pt}\end{minipage}}

\newcommand{\resultul}[1]{\underline{\underline{#1}}}
\newcommand{\parskp}{$ $\\}	% new line after paragraph
\newcommand{\corr}{\;\widehat{=}\;}
\newcommand{\mdeg}{^{\circ}}

\newcommand{\nok}[2]{\binom{#1}{#2}}	% n über k BESSER: \binom{n}{k}
\newcommand{\mtr}[1]{\begin{pmatrix}#1\end{pmatrix}}	% Matrix
\newcommand{\dtr}[1]{\begin{vmatrix}#1\end{vmatrix}}	% Determinante (Betragsmatrix)
\renewcommand{\vec}[1]{\underline{#1}}	% Vektorschreibweise
\newcommand{\imptnt}[1]{\colorbox{red!30}{#1}}	% Wichtiges
\newcommand{\intd}[1]{\,\mathrm{d}#1}
\newcommand{\diffd}[1]{\mathrm{d}#1}
% für Module-Rechnung: \pmod{}
\newcommand{\unit}[1]{\,\mathrm{#1}}

%\bibliography{\customDir _Literatur/HTW_Literatur.bib}
%\setlength{\headheight}{10mm}	% default: ca. 8mm
\setlength{\footheight}{10mm}	% default: ca. 8mm


\usepackage{tikz-uml}

% TeXDoclet Compatibility:
\makeatletter
\DeclareOldFontCommand{\rm}{\normalfont\rmfamily}{\mathrm}
\DeclareOldFontCommand{\sf}{\normalfont\sffamily}{\mathsf}
\DeclareOldFontCommand{\tt}{\normalfont\ttfamily}{\mathtt}
\DeclareOldFontCommand{\bf}{\normalfont\bfseries}{\mathbf}
\DeclareOldFontCommand{\it}{\normalfont\itshape}{\mathit}
\DeclareOldFontCommand{\sl}{\normalfont\slshape}{\@nomath\sl}
\DeclareOldFontCommand{\sc}{\normalfont\scshape}{\@nomath\sc}
\makeatother

% TeXDoclet Preamble:
%\usepackage{color}
\usepackage{ifthen}
\usepackage{ifpdf}
\usepackage[headings]{fullpage}
\usepackage{listings}
\lstset{language=Java,breaklines=true}
\ifpdf \usepackage[pdftex, pdfpagemode={UseOutlines},bookmarks,colorlinks,linkcolor={blue},plainpages=false,pdfpagelabels,citecolor={red},breaklinks=true]{hyperref}
  \usepackage[pdftex]{graphicx}
  \pdfcompresslevel=9
  \DeclareGraphicsRule{*}{mps}{*}{}
\else
  \usepackage[dvips]{graphicx}
\fi

\newcommand{\entityintro}[3]{%
  \hbox to \hsize{%
    \vbox{%
      \hbox to .2in{}%
    }%
    {\bf  #1}%
    \dotfill\pageref{#2}%
  }
  \makebox[\hsize]{%
    \parbox{.4in}{}%
    \parbox[l]{5in}{%
      \vspace{1mm}%
      #3%
      \vspace{1mm}%
    }%
  }%
}
\newcommand{\refdefined}[1]{
\expandafter\ifx\csname r@#1\endcsname\relax
\relax\else
{$($in \ref{#1}, page \pageref{#1}$)$}\fi}
\date{null}
\chardef\textbackslash=`\\

\usepackage{ifpdf}
\lstset{language=Java,breaklines=true}

\newcommand{\entityintro}[3]{%
  \hbox to \hsize{%
    \vbox{%
      \hbox to .2in{}%
    }%
    {\bf  #1}%
    \dotfill\pageref{#2}%
  }
  \makebox[\hsize]{%
    \parbox{.4in}{}%
    \parbox[l]{5in}{%
      \vspace{1mm}%
      #3%
      \vspace{1mm}%
    }%
  }%
}
\newcommand{\refdefined}[1]{
\expandafter\ifx\csname r@#1\endcsname\relax
\relax\else
{$($in \ref{#1}, page \pageref{#1}$)$}\fi}
\chardef\textbackslash=`\\

\begin{document}

%\selectlanguage{english}
\maketitle
\newpage
\tableofcontents
\newpage

\chapter{Dokumentation}

\section{Aufgabe}

\subsection{Rahmenaufgaben}

Es gibt folgende Rahmenaufgaben, die abseits des Programmierens gelöst werden sollen:
\begin{enumerate}
\setcounter{enumi}{7}
\item Errechneter maximal erzielbarer Durchsatz beim SW-Protokoll bei $10\%$ Paketverlust und $=10\unit{ms}$ Verzögerung:\\
Datenlänge: $400 \unit{Byte} = 3200\unit{Bit}$\\
$R=1$ (Vereinfachung)\\
$v$: Geschwindigkeit\\
$P_{CS}=P_{SC}=0,1$\\
$T_{CS}=T_{SC}=10\unit{Bit}/v$\\
$T_P=(16+8)\unit{Bit}/v+3\,200\unit{Bit}/v=3\,224\unit{Bit}/v$ \\
$T_W=T_{CS}+T_{SC}+T_{ACK}=10\unit{s}+10\unit{s}+(16+8)\unit{Bit}/v=20\unit{s}+24\unit{Bit}/v$
\begin{align*}
\eta_{SW}&=\frac{T_P}{T_P+ T_W}\cdot (1-P_{CS})\cdot(1-P_{SC}) \cdot R \\
&= \frac{3\,224\unit{Bit}/v}{3\,224\unit{Bit}/v+20\unit{s}+24\unit{Bit}/v}\cdot 0,9^2\\
&=\frac{3\,224\unit{Bit}/v}{3\,248\unit{Bit}/v+20\unit{s}}\cdot 0,9^2
\end{align*}
Tatsächlich erzielter Durchsatz:
$$tat$$
Der Unterschied ist damit zu erklären, dass …
\item Dokumentation der Funktionen unter Nutzung von \LaTeX{}.\\
Siehe vor allem Abschnitt \ref{stated} für die Zustandsdiagramme, Abschnitt \ref{ausblick} für Probleme/Limitierungen/Verbesserungsvorschläge  und Kapitel \ref{javadoc} für die Funktionen. Zugunsten der besseren Lesbarkeit im Quellcode wurde die Dokumentation der Funktionen in Englisch gehalten.
\end{enumerate}

\subsection{Programmierung}

Die Programmierung wurde nach den Anforderungen der Aufgabenstellung gelöst. Diese lauten wie folgt:
\begin{itemize}
\item Client:
\begin{itemize}
\item Über Konsole mit Parametern „Zieladresse“, „Portnummer“ und „Dateiname“ aufrufbar (zum debuggen auch „Paketverlust“ und „mittlere Verzögerung“)
\item Zeigt während der Übertragung jede Sekunde die Datenrate an
\item Zeigt nach der Übertragung die Datenrate an
\item Korrigiert Fehler bei verlorenen oder vertauschten Paketen
\end{itemize}
\item Server:
\begin{itemize}
\item Über Konsole mit dem Parameter „Portnummer“ aufrufbar (zum debuggen auch „Paketverlust“ und „mittlere Verzögerung“)
\item Speichert Datei in seinem Pfad unter dem korrekten Dateiname (Zeichen „1“ wird angehängt, wenn Datei bereits existiert)
\item Korrigiert Fehler bei verlorenen oder vertauschten Paketen
\end{itemize}
\end{itemize}

\section{Zustandsdiagramme}
\label{stated}
\subsection{Client}
\begin{center}
\begin{tikzpicture}
\begin{umlstate}[name=Amain]{Etat global de l'objet A}
\begin{umlstate}[name=Bgraph, fill=red!20]{graphe B}
\umlstateinitial[name=Binit]
\umlbasicstate[y=-4, name=test1, fill=white]{test1}
\umltrans{Binit}{test1}
\umltrans[recursive=20|60|2.5cm, recursive direction=right to top, arg={op1}, pos=1.5]{test1}{test1}
\umltrans[recursive=160|120|2.5cm, recursive direction=left to top, arg={op2}, pos=1.5]{test1}{test1}
\umltrans[recursive=-160|-120|2.5cm, recursive direction=left to bottom, arg={op3}, pos=1.5]{test1}{test1}
\umltrans[recursive=-20|-60|2.5cm, recursive direction=right to bottom, arg={op4}, pos=1.5]{test1}{test1}
\umlbasicstate[y=-8, name=test2, fill=white]{test2}
\umltrans[recursive=-160|-120|2.5cm, recursive direction=left to bottom, arg={op5}, pos=1.5]{test2}{test2}
\umltrans{test1}{test2}
\umlstatefinal[x=3, y=-7.75, name=Bfinal]
\umltrans{test2}{Bfinal}
\end{umlstate}
\umlstateinitial[x=6, y=1, name=Ainit]
\umlVHtrans[anchor2=40]{Ainit}{Bgraph}
\umlstatefinal[x=6, y=-3.5, name=Afinal]
\umlHVtrans[anchor1=30]{Bgraph}{Afinal}
\umlbasicstate[x=6, y=-6, name=visu, fill=green!20]{Visualisation}
\umlHVtrans{Bfinal}{visu}
\umltrans{visu}{Afinal}
\umltrans[recursive=-20|-60|2.5cm, recursive direction=right to bottom, arg=a, pos=1.5]{visu}{visu}
\end{umlstate}
\end{tikzpicture}
\end{center}
\subsection{Server}

\section{Ausblick}
\label{ausblick}
\subsection{Probleme}
--

\subsection{Limitierungen}
--

\subsection{Verbesserungsvorschläge}
--

\chapter{JavaDoc}
\selectlanguage{english}
\label{javadoc}
\section*{Class Hierarchy}{
%\thispagestyle{empty}
\markboth{Class Hierarchy}{Class Hierarchy}
\addcontentsline{toc}{section}{Class Hierarchy}
\subsection*{Classes}
{\raggedright
\hspace{0.0cm} $\bullet$ java.lang.Object {\tiny \refdefined{java.lang.Object}} \\
\hspace{1.0cm} $\bullet$ filetransferUDP.FileTransfer {\tiny \refdefined{filetransferUDP.FileTransfer}} \\
\hspace{2.0cm} $\bullet$ filetransferUDP.Client {\tiny \refdefined{filetransferUDP.Client}} \\
\hspace{2.0cm} $\bullet$ filetransferUDP.Server {\tiny \refdefined{filetransferUDP.Server}} \\
}
}
\section{Package filetransferUDP}{
\label{filetransferUDP}\hypertarget{filetransferUDP}{}
\subsection{\label{filetransferUDP.Client}Class Client}{
\hypertarget{filetransferUDP.Client}{}\vskip .1in 
The Client takes a file and transfers it to the server.\vskip .1in 
\subsubsection{Declaration}{
\begin{lstlisting}[frame=none]
public class Client
 extends filetransferUDP.FileTransfer\end{lstlisting}
\subsubsection{Field summary}{
\begin{verse}
\hyperlink{filetransferUDP.Client.ARGUMENT_MESSAGE}{{\bf ARGUMENT\_MESSAGE}} \\
\end{verse}
}
\subsubsection{Constructor summary}{
\begin{verse}
\hyperlink{filetransferUDP.Client(java.lang.String[])}{{\bf Client(String\lbrack \rbrack )}} Executes the client.\\
\end{verse}
}
\subsubsection{Method summary}{
\begin{verse}
\hyperlink{filetransferUDP.Client.displayArguments()}{{\bf displayArguments()}} Displays the client arguments.\\
\hyperlink{filetransferUDP.Client.getNextPacket()}{{\bf getNextPacket()}} Loads the next packet to be send to the server.\\
\hyperlink{filetransferUDP.Client.initializeUpload()}{{\bf initializeUpload()}} Prepares the first packet and sets the socket up.\\
\hyperlink{filetransferUDP.Client.main(java.lang.String[])}{{\bf main(String\lbrack \rbrack )}} Calls Client(args) and thereby starts the client.\\
\hyperlink{filetransferUDP.Client.parseArguments(java.lang.String[])}{{\bf parseArguments(String\lbrack \rbrack )}} Parses the command line arguments.\\
\hyperlink{filetransferUDP.Client.printInfo()}{{\bf printInfo()}} Prints the secondly progress (and speed) info.\\
\hyperlink{filetransferUDP.Client.receiveACK()}{{\bf receiveACK()}} Tries to receive the acknowledge packet from the server.\\
\hyperlink{filetransferUDP.Client.sendPacket(byte[])}{{\bf sendPacket(byte\lbrack \rbrack )}} Sends the packet to the server.\\
\end{verse}
}
\subsubsection{Fields}{
\begin{itemize}
\item{
\index{ARGUMENT\_MESSAGE}
\label{filetransferUDP.Client.ARGUMENT_MESSAGE}\hypertarget{filetransferUDP.Client.ARGUMENT_MESSAGE}{\texttt{static final java.lang.String\ {\bf  ARGUMENT\_MESSAGE}}
}
}
\end{itemize}
}
\subsubsection{Constructors}{
\vskip -2em
\begin{itemize}
\item{ 
\index{Client(String\lbrack \rbrack )}
\hypertarget{filetransferUDP.Client(java.lang.String[])}{{\bf  Client}\\}
\begin{lstlisting}[frame=none]
public Client(java.lang.String[] args)\end{lstlisting} %end signature
\begin{itemize}
\item{
{\bf  Description}

Executes the client.
}
\item{
{\bf  Parameters}
  \begin{itemize}
   \item{
\texttt{args} -- command line arguments.}
  \end{itemize}
}%end item
\end{itemize}
}%end item
\end{itemize}
}
\subsubsection{Methods}{
\vskip -2em
\begin{itemize}
\item{ 
\index{displayArguments()}
\hypertarget{filetransferUDP.Client.displayArguments()}{{\bf  displayArguments}\\}
\begin{lstlisting}[frame=none]
private void displayArguments()\end{lstlisting} %end signature
\begin{itemize}
\item{
{\bf  Description}

Displays the client arguments.
}
\end{itemize}
}%end item
\item{ 
\index{getNextPacket()}
\hypertarget{filetransferUDP.Client.getNextPacket()}{{\bf  getNextPacket}\\}
\begin{lstlisting}[frame=none]
private void getNextPacket()\end{lstlisting} %end signature
\begin{itemize}
\item{
{\bf  Description}

Loads the next packet to be send to the server.
}
\end{itemize}
}%end item
\item{ 
\index{initializeUpload()}
\hypertarget{filetransferUDP.Client.initializeUpload()}{{\bf  initializeUpload}\\}
\begin{lstlisting}[frame=none]
private void initializeUpload()\end{lstlisting} %end signature
\begin{itemize}
\item{
{\bf  Description}

Prepares the first packet and sets the socket up.
}
\end{itemize}
}%end item
\item{ 
\index{main(String\lbrack \rbrack )}
\hypertarget{filetransferUDP.Client.main(java.lang.String[])}{{\bf  main}\\}
\begin{lstlisting}[frame=none]
public static void main(java.lang.String[] args)\end{lstlisting} %end signature
\begin{itemize}
\item{
{\bf  Description}

Calls Client(args) and thereby starts the client.
}
\item{
{\bf  Parameters}
  \begin{itemize}
   \item{
\texttt{args} -- command line arguments}
  \end{itemize}
}%end item
\end{itemize}
}%end item
\item{ 
\index{parseArguments(String\lbrack \rbrack )}
\hypertarget{filetransferUDP.Client.parseArguments(java.lang.String[])}{{\bf  parseArguments}\\}
\begin{lstlisting}[frame=none]
private void parseArguments(java.lang.String[] args)\end{lstlisting} %end signature
\begin{itemize}
\item{
{\bf  Description}

Parses the command line arguments.
}
\item{
{\bf  Parameters}
  \begin{itemize}
   \item{
\texttt{args} -- command line arguments.}
  \end{itemize}
}%end item
\end{itemize}
}%end item
\item{ 
\index{printInfo()}
\hypertarget{filetransferUDP.Client.printInfo()}{{\bf  printInfo}\\}
\begin{lstlisting}[frame=none]
private void printInfo()\end{lstlisting} %end signature
\begin{itemize}
\item{
{\bf  Description}

Prints the secondly progress (and speed) info.
}
\end{itemize}
}%end item
\item{ 
\index{receiveACK()}
\hypertarget{filetransferUDP.Client.receiveACK()}{{\bf  receiveACK}\\}
\begin{lstlisting}[frame=none]
private void receiveACK()\end{lstlisting} %end signature
\begin{itemize}
\item{
{\bf  Description}

Tries to receive the acknowledge packet from the server.
}
\end{itemize}
}%end item
\item{ 
\index{sendPacket(byte\lbrack \rbrack )}
\hypertarget{filetransferUDP.Client.sendPacket(byte[])}{{\bf  sendPacket}\\}
\begin{lstlisting}[frame=none]
private boolean sendPacket(byte[] packet)\end{lstlisting} %end signature
\begin{itemize}
\item{
{\bf  Description}

Sends the packet to the server. If command line arguments are set, it may delay or lose the packet.
}
\item{
{\bf  Parameters}
  \begin{itemize}
   \item{
\texttt{packet} -- the packet to be send.}
  \end{itemize}
}%end item
\item{{\bf  Returns} -- 
true, if the packet has been send, else false. 
}%end item
\end{itemize}
}%end item
\end{itemize}
}
}
\subsection{\label{filetransferUDP.FileTransfer}Class FileTransfer}{
\hypertarget{filetransferUDP.FileTransfer}{}\vskip .1in 
FileTransfer is the base for a Server or Client.\vskip .1in 
\subsubsection{Declaration}{
\begin{lstlisting}[frame=none]
public abstract class FileTransfer
 extends java.lang.Object\end{lstlisting}
\subsubsection{All known subclasses}{Server\small{\refdefined{filetransferUDP.Server}}, Client\small{\refdefined{filetransferUDP.Client}}}
\subsubsection{Field summary}{
\begin{verse}
\hyperlink{filetransferUDP.FileTransfer.activeSession}{{\bf activeSession}} \\
\hyperlink{filetransferUDP.FileTransfer.byteClientFile}{{\bf byteClientFile}} \\
\hyperlink{filetransferUDP.FileTransfer.byteFileLength}{{\bf byteFileLength}} \\
\hyperlink{filetransferUDP.FileTransfer.byteFileName}{{\bf byteFileName}} \\
\hyperlink{filetransferUDP.FileTransfer.byteFileNameLength}{{\bf byteFileNameLength}} \\
\hyperlink{filetransferUDP.FileTransfer.byteFirstPacketCRC}{{\bf byteFirstPacketCRC}} \\
\hyperlink{filetransferUDP.FileTransfer.byteLastPacketCRC}{{\bf byteLastPacketCRC}} \\
\hyperlink{filetransferUDP.FileTransfer.bytePacketNumber}{{\bf bytePacketNumber}} The packet number of the current packet (client) or the expected packet (server).\\
\hyperlink{filetransferUDP.FileTransfer.byteSessionNumber}{{\bf byteSessionNumber}} \\
\hyperlink{filetransferUDP.FileTransfer.bytesProcessed}{{\bf bytesProcessed}} \\
\hyperlink{filetransferUDP.FileTransfer.byteStartIdentifier}{{\bf byteStartIdentifier}} \\
\hyperlink{filetransferUDP.FileTransfer.clientFile}{{\bf clientFile}} \\
\hyperlink{filetransferUDP.FileTransfer.connectionIP}{{\bf connectionIP}} \\
\hyperlink{filetransferUDP.FileTransfer.currentPacket}{{\bf currentPacket}} \\
\hyperlink{filetransferUDP.FileTransfer.dataPacket}{{\bf dataPacket}} \\
\hyperlink{filetransferUDP.FileTransfer.dataSocket}{{\bf dataSocket}} \\
\hyperlink{filetransferUDP.FileTransfer.debug}{{\bf debug}} \\
\hyperlink{filetransferUDP.FileTransfer.fileName}{{\bf fileName}} \\
\hyperlink{filetransferUDP.FileTransfer.filePath}{{\bf filePath}} \\
\hyperlink{filetransferUDP.FileTransfer.finished}{{\bf finished}} \\
\hyperlink{filetransferUDP.FileTransfer.lastInfo}{{\bf lastInfo}} \\
\hyperlink{filetransferUDP.FileTransfer.LINE}{{\bf LINE}} \\
\hyperlink{filetransferUDP.FileTransfer.LINED}{{\bf LINED}} \\
\hyperlink{filetransferUDP.FileTransfer.MAX_TRIES}{{\bf MAX\_TRIES}} \\
\hyperlink{filetransferUDP.FileTransfer.packedAllBytes}{{\bf packedAllBytes}} \\
\hyperlink{filetransferUDP.FileTransfer.packetData}{{\bf packetData}} \\
\hyperlink{filetransferUDP.FileTransfer.PACKETDATALENGTH}{{\bf PACKETDATALENGTH}} \\
\hyperlink{filetransferUDP.FileTransfer.packetDelay}{{\bf packetDelay}} \\
\hyperlink{filetransferUDP.FileTransfer.PACKETLENGTH}{{\bf PACKETLENGTH}} \\
\hyperlink{filetransferUDP.FileTransfer.packetLossRate}{{\bf packetLossRate}} \\
\hyperlink{filetransferUDP.FileTransfer.port}{{\bf port}} The port of the server.\\
\hyperlink{filetransferUDP.FileTransfer.previousPacket}{{\bf previousPacket}} \\
\hyperlink{filetransferUDP.FileTransfer.processedBytes}{{\bf processedBytes}} \\
\hyperlink{filetransferUDP.FileTransfer.sendPrevious}{{\bf sendPrevious}} \\
\hyperlink{filetransferUDP.FileTransfer.speed}{{\bf speed}} \\
\hyperlink{filetransferUDP.FileTransfer.startTime}{{\bf startTime}} \\
\hyperlink{filetransferUDP.FileTransfer.timeout}{{\bf timeout}} \\
\hyperlink{filetransferUDP.FileTransfer.tries}{{\bf tries}} \\
\end{verse}
}
\subsubsection{Constructor summary}{
\begin{verse}
\hyperlink{filetransferUDP.FileTransfer()}{{\bf FileTransfer()}} \\
\end{verse}
}
\subsubsection{Method summary}{
\begin{verse}
\hyperlink{filetransferUDP.FileTransfer.createFirstPacket()}{{\bf createFirstPacket()}} Creates the first packet.\\
\hyperlink{filetransferUDP.FileTransfer.exitApp(java.lang.String, int)}{{\bf exitApp(String, int)}} Closes the Client/Server.\\
\hyperlink{filetransferUDP.FileTransfer.flipPacketNumber()}{{\bf flipPacketNumber()}} Flips the packet number between 0 and 1.\\
\hyperlink{filetransferUDP.FileTransfer.generateCRC(byte[])}{{\bf generateCRC(byte\lbrack \rbrack )}} Generates the CRC value of a byte-array.\\
\hyperlink{filetransferUDP.FileTransfer.getAppLocation()}{{\bf getAppLocation()}} Returns the Location where the Client/Server runs (without /bin).\\
\hyperlink{filetransferUDP.FileTransfer.getByteFileLength()}{{\bf getByteFileLength()}} \\
\hyperlink{filetransferUDP.FileTransfer.getByteFileLengthLong()}{{\bf getByteFileLengthLong()}} \\
\hyperlink{filetransferUDP.FileTransfer.getByteFileName()}{{\bf getByteFileName()}} \\
\hyperlink{filetransferUDP.FileTransfer.getByteFileNameLength()}{{\bf getByteFileNameLength()}} \\
\hyperlink{filetransferUDP.FileTransfer.getByteFileNameLengthShort()}{{\bf getByteFileNameLengthShort()}} \\
\hyperlink{filetransferUDP.FileTransfer.getByteFileNameString()}{{\bf getByteFileNameString()}} \\
\hyperlink{filetransferUDP.FileTransfer.getByteFirstPacketCRC()}{{\bf getByteFirstPacketCRC()}} \\
\hyperlink{filetransferUDP.FileTransfer.getByteLastPacketCRC()}{{\bf getByteLastPacketCRC()}} \\
\hyperlink{filetransferUDP.FileTransfer.getBytePacketNumber()}{{\bf getBytePacketNumber()}} \\
\hyperlink{filetransferUDP.FileTransfer.getByteSessionNumber()}{{\bf getByteSessionNumber()}} \\
\hyperlink{filetransferUDP.FileTransfer.getByteSessionNumberShort()}{{\bf getByteSessionNumberShort()}} \\
\hyperlink{filetransferUDP.FileTransfer.getByteSessionNumberShort(byte[])}{{\bf getByteSessionNumberShort(byte\lbrack \rbrack )}} \\
\hyperlink{filetransferUDP.FileTransfer.getByteStartIdentifier()}{{\bf getByteStartIdentifier()}} \\
\hyperlink{filetransferUDP.FileTransfer.getByteStartIdentifierString()}{{\bf getByteStartIdentifierString()}} \\
\hyperlink{filetransferUDP.FileTransfer.getClientFile()}{{\bf getClientFile()}} \\
\hyperlink{filetransferUDP.FileTransfer.getConnectionIP()}{{\bf getConnectionIP()}} \\
\hyperlink{filetransferUDP.FileTransfer.getCurrentPacket()}{{\bf getCurrentPacket()}} \\
\hyperlink{filetransferUDP.FileTransfer.getDataPacket()}{{\bf getDataPacket()}} \\
\hyperlink{filetransferUDP.FileTransfer.getDataSocket()}{{\bf getDataSocket()}} \\
\hyperlink{filetransferUDP.FileTransfer.getFileName()}{{\bf getFileName()}} \\
\hyperlink{filetransferUDP.FileTransfer.getFilePath()}{{\bf getFilePath()}} \\
\hyperlink{filetransferUDP.FileTransfer.getFirstPacketContents(byte[])}{{\bf getFirstPacketContents(byte\lbrack \rbrack )}} Parses the first packet and saves its contents.\\
\hyperlink{filetransferUDP.FileTransfer.getNewFileName(java.lang.String)}{{\bf getNewFileName(String)}} Generates a new file name for saving a downloaded file.\\
\hyperlink{filetransferUDP.FileTransfer.getPACKETDATALENGTH()}{{\bf getPACKETDATALENGTH()}} \\
\hyperlink{filetransferUDP.FileTransfer.getPacketDelay()}{{\bf getPacketDelay()}} \\
\hyperlink{filetransferUDP.FileTransfer.getPacketLossRate()}{{\bf getPacketLossRate()}} \\
\hyperlink{filetransferUDP.FileTransfer.getPort()}{{\bf getPort()}} \\
\hyperlink{filetransferUDP.FileTransfer.getPreviousPacket()}{{\bf getPreviousPacket()}} \\
\hyperlink{filetransferUDP.FileTransfer.getPrintableDebugStatus()}{{\bf getPrintableDebugStatus()}} Creates a nice line for displaying the debug-status.\\
\hyperlink{filetransferUDP.FileTransfer.getPrintableLossRateDelayValues()}{{\bf getPrintableLossRateDelayValues()}} Packs the loss rate and delay value into a nice line.\\
\hyperlink{filetransferUDP.FileTransfer.isCorrectCRC(byte[], byte[])}{{\bf isCorrectCRC(byte\lbrack \rbrack , byte\lbrack \rbrack )}} Compares the CRC value of a given byte-array with another CRC value.\\
\hyperlink{filetransferUDP.FileTransfer.isCorrectPacketNumber(byte)}{{\bf isCorrectPacketNumber(byte)}} Checks, if the packet number is as expected.\\
\hyperlink{filetransferUDP.FileTransfer.isCorrectSessionNumber(byte[])}{{\bf isCorrectSessionNumber(byte\lbrack \rbrack )}} Checks, if the session number is as expected.\\
\hyperlink{filetransferUDP.FileTransfer.isCorrectStartIdentifier(byte[])}{{\bf isCorrectStartIdentifier(byte\lbrack \rbrack )}} Checks, if the start identifier is correct ASCII "Start".\\
\hyperlink{filetransferUDP.FileTransfer.isDebug()}{{\bf isDebug()}} \\
\hyperlink{filetransferUDP.FileTransfer.printMessage(java.lang.String)}{{\bf printMessage(String)}} Prints a status message.\\
\hyperlink{filetransferUDP.FileTransfer.printMessage(java.lang.String, int)}{{\bf printMessage(String, int)}} Prints a message to the console.\\
\hyperlink{filetransferUDP.FileTransfer.printSessionData()}{{\bf printSessionData()}} Prints the data of a (new) session.\\
\hyperlink{filetransferUDP.FileTransfer.resetSession()}{{\bf resetSession()}} Resets a session.\\
\hyperlink{filetransferUDP.FileTransfer.setActiveSession()}{{\bf setActiveSession()}} \\
\hyperlink{filetransferUDP.FileTransfer.setByteFileLength()}{{\bf setByteFileLength()}} \\
\hyperlink{filetransferUDP.FileTransfer.setByteFileLength(byte[])}{{\bf setByteFileLength(byte\lbrack \rbrack )}} \\
\hyperlink{filetransferUDP.FileTransfer.setByteFileName()}{{\bf setByteFileName()}} \\
\hyperlink{filetransferUDP.FileTransfer.setByteFileName(byte[])}{{\bf setByteFileName(byte\lbrack \rbrack )}} \\
\hyperlink{filetransferUDP.FileTransfer.setByteFileNameLength()}{{\bf setByteFileNameLength()}} \\
\hyperlink{filetransferUDP.FileTransfer.setByteFileNameLength(byte[])}{{\bf setByteFileNameLength(byte\lbrack \rbrack )}} \\
\hyperlink{filetransferUDP.FileTransfer.setByteFirstPacketCRC(byte[])}{{\bf setByteFirstPacketCRC(byte\lbrack \rbrack )}} \\
\hyperlink{filetransferUDP.FileTransfer.setByteLastPacketCRC(byte[])}{{\bf setByteLastPacketCRC(byte\lbrack \rbrack )}} \\
\hyperlink{filetransferUDP.FileTransfer.setBytePacketNumber(byte)}{{\bf setBytePacketNumber(byte)}} \\
\hyperlink{filetransferUDP.FileTransfer.setBytePacketNumber(int)}{{\bf setBytePacketNumber(int)}} \\
\hyperlink{filetransferUDP.FileTransfer.setByteSessionNumber()}{{\bf setByteSessionNumber()}} \\
\hyperlink{filetransferUDP.FileTransfer.setByteSessionNumber(byte[])}{{\bf setByteSessionNumber(byte\lbrack \rbrack )}} \\
\hyperlink{filetransferUDP.FileTransfer.setByteStartIdentifier()}{{\bf setByteStartIdentifier()}} \\
\hyperlink{filetransferUDP.FileTransfer.setByteStartIdentifier(byte[])}{{\bf setByteStartIdentifier(byte\lbrack \rbrack )}} \\
\hyperlink{filetransferUDP.FileTransfer.setClientFile(java.io.File)}{{\bf setClientFile(File)}} \\
\hyperlink{filetransferUDP.FileTransfer.setClientFile(java.lang.String)}{{\bf setClientFile(String)}} \\
\hyperlink{filetransferUDP.FileTransfer.setConnectionIP(java.net.InetAddress)}{{\bf setConnectionIP(InetAddress)}} \\
\hyperlink{filetransferUDP.FileTransfer.setConnectionIP(java.lang.String)}{{\bf setConnectionIP(String)}} \\
\hyperlink{filetransferUDP.FileTransfer.setCurrentPacket(byte[])}{{\bf setCurrentPacket(byte\lbrack \rbrack )}} \\
\hyperlink{filetransferUDP.FileTransfer.setDataPacket(java.net.DatagramPacket)}{{\bf setDataPacket(DatagramPacket)}} \\
\hyperlink{filetransferUDP.FileTransfer.setDataSocket(java.net.DatagramSocket)}{{\bf setDataSocket(DatagramSocket)}} \\
\hyperlink{filetransferUDP.FileTransfer.setDebug(boolean)}{{\bf setDebug(boolean)}} \\
\hyperlink{filetransferUDP.FileTransfer.setFileName(java.lang.String)}{{\bf setFileName(String)}} \\
\hyperlink{filetransferUDP.FileTransfer.setFilePath(java.lang.String)}{{\bf setFilePath(String)}} \\
\hyperlink{filetransferUDP.FileTransfer.setPacketDelay(int)}{{\bf setPacketDelay(int)}} \\
\hyperlink{filetransferUDP.FileTransfer.setPacketLossRate(float)}{{\bf setPacketLossRate(float)}} \\
\hyperlink{filetransferUDP.FileTransfer.setPort(int)}{{\bf setPort(int)}} \\
\hyperlink{filetransferUDP.FileTransfer.setPreviousPacket(byte[])}{{\bf setPreviousPacket(byte\lbrack \rbrack )}} \\
\hyperlink{filetransferUDP.FileTransfer.toggleBytePacketNumber()}{{\bf toggleBytePacketNumber()}} \\
\hyperlink{filetransferUDP.FileTransfer.verifyACK(byte[])}{{\bf verifyACK(byte\lbrack \rbrack )}} Checks, if the acknowledge packet is correct (with the right session and packet number).\\
\end{verse}
}
\subsubsection{Fields}{
\begin{itemize}
\item{
\index{port}
\label{filetransferUDP.FileTransfer.port}\hypertarget{filetransferUDP.FileTransfer.port}{\texttt{ int\ {\bf  port}}
}
\begin{itemize}
\item{\vskip -.9ex 
The port of the server. For the server: After the server is started, this variable is used as the port of the client.}
\end{itemize}
}
\item{
\index{packetLossRate}
\label{filetransferUDP.FileTransfer.packetLossRate}\hypertarget{filetransferUDP.FileTransfer.packetLossRate}{\texttt{ float\ {\bf  packetLossRate}}
}
}
\item{
\index{packetDelay}
\label{filetransferUDP.FileTransfer.packetDelay}\hypertarget{filetransferUDP.FileTransfer.packetDelay}{\texttt{ int\ {\bf  packetDelay}}
}
}
\item{
\index{PACKETDATALENGTH}
\label{filetransferUDP.FileTransfer.PACKETDATALENGTH}\hypertarget{filetransferUDP.FileTransfer.PACKETDATALENGTH}{\texttt{static final int\ {\bf  PACKETDATALENGTH}}
}
}
\item{
\index{PACKETLENGTH}
\label{filetransferUDP.FileTransfer.PACKETLENGTH}\hypertarget{filetransferUDP.FileTransfer.PACKETLENGTH}{\texttt{static final int\ {\bf  PACKETLENGTH}}
}
}
\item{
\index{MAX\_TRIES}
\label{filetransferUDP.FileTransfer.MAX_TRIES}\hypertarget{filetransferUDP.FileTransfer.MAX_TRIES}{\texttt{static final int\ {\bf  MAX\_TRIES}}
}
}
\item{
\index{dataSocket}
\label{filetransferUDP.FileTransfer.dataSocket}\hypertarget{filetransferUDP.FileTransfer.dataSocket}{\texttt{ java.net.DatagramSocket\ {\bf  dataSocket}}
}
}
\item{
\index{dataPacket}
\label{filetransferUDP.FileTransfer.dataPacket}\hypertarget{filetransferUDP.FileTransfer.dataPacket}{\texttt{ java.net.DatagramPacket\ {\bf  dataPacket}}
}
}
\item{
\index{connectionIP}
\label{filetransferUDP.FileTransfer.connectionIP}\hypertarget{filetransferUDP.FileTransfer.connectionIP}{\texttt{ java.net.InetAddress\ {\bf  connectionIP}}
}
}
\item{
\index{previousPacket}
\label{filetransferUDP.FileTransfer.previousPacket}\hypertarget{filetransferUDP.FileTransfer.previousPacket}{\texttt{ byte\lbrack \rbrack \ {\bf  previousPacket}}
}
}
\item{
\index{currentPacket}
\label{filetransferUDP.FileTransfer.currentPacket}\hypertarget{filetransferUDP.FileTransfer.currentPacket}{\texttt{ byte\lbrack \rbrack \ {\bf  currentPacket}}
}
}
\item{
\index{fileName}
\label{filetransferUDP.FileTransfer.fileName}\hypertarget{filetransferUDP.FileTransfer.fileName}{\texttt{ java.lang.String\ {\bf  fileName}}
}
}
\item{
\index{tries}
\label{filetransferUDP.FileTransfer.tries}\hypertarget{filetransferUDP.FileTransfer.tries}{\texttt{ int\ {\bf  tries}}
}
}
\item{
\index{bytesProcessed}
\label{filetransferUDP.FileTransfer.bytesProcessed}\hypertarget{filetransferUDP.FileTransfer.bytesProcessed}{\texttt{ long\ {\bf  bytesProcessed}}
}
}
\item{
\index{processedBytes}
\label{filetransferUDP.FileTransfer.processedBytes}\hypertarget{filetransferUDP.FileTransfer.processedBytes}{\texttt{ byte\lbrack \rbrack \ {\bf  processedBytes}}
}
}
\item{
\index{timeout}
\label{filetransferUDP.FileTransfer.timeout}\hypertarget{filetransferUDP.FileTransfer.timeout}{\texttt{ int\ {\bf  timeout}}
}
}
\item{
\index{byteSessionNumber}
\label{filetransferUDP.FileTransfer.byteSessionNumber}\hypertarget{filetransferUDP.FileTransfer.byteSessionNumber}{\texttt{ byte\lbrack \rbrack \ {\bf  byteSessionNumber}}
}
}
\item{
\index{bytePacketNumber}
\label{filetransferUDP.FileTransfer.bytePacketNumber}\hypertarget{filetransferUDP.FileTransfer.bytePacketNumber}{\texttt{ byte\ {\bf  bytePacketNumber}}
}
\begin{itemize}
\item{\vskip -.9ex 
The packet number of the current packet (client) or the expected packet (server).}
\end{itemize}
}
\item{
\index{byteStartIdentifier}
\label{filetransferUDP.FileTransfer.byteStartIdentifier}\hypertarget{filetransferUDP.FileTransfer.byteStartIdentifier}{\texttt{ byte\lbrack \rbrack \ {\bf  byteStartIdentifier}}
}
}
\item{
\index{byteFileLength}
\label{filetransferUDP.FileTransfer.byteFileLength}\hypertarget{filetransferUDP.FileTransfer.byteFileLength}{\texttt{ byte\lbrack \rbrack \ {\bf  byteFileLength}}
}
}
\item{
\index{byteFileNameLength}
\label{filetransferUDP.FileTransfer.byteFileNameLength}\hypertarget{filetransferUDP.FileTransfer.byteFileNameLength}{\texttt{ byte\lbrack \rbrack \ {\bf  byteFileNameLength}}
}
}
\item{
\index{byteFileName}
\label{filetransferUDP.FileTransfer.byteFileName}\hypertarget{filetransferUDP.FileTransfer.byteFileName}{\texttt{ byte\lbrack \rbrack \ {\bf  byteFileName}}
}
}
\item{
\index{byteFirstPacketCRC}
\label{filetransferUDP.FileTransfer.byteFirstPacketCRC}\hypertarget{filetransferUDP.FileTransfer.byteFirstPacketCRC}{\texttt{ byte\lbrack \rbrack \ {\bf  byteFirstPacketCRC}}
}
}
\item{
\index{byteLastPacketCRC}
\label{filetransferUDP.FileTransfer.byteLastPacketCRC}\hypertarget{filetransferUDP.FileTransfer.byteLastPacketCRC}{\texttt{ byte\lbrack \rbrack \ {\bf  byteLastPacketCRC}}
}
}
\item{
\index{startTime}
\label{filetransferUDP.FileTransfer.startTime}\hypertarget{filetransferUDP.FileTransfer.startTime}{\texttt{ long\ {\bf  startTime}}
}
}
\item{
\index{lastInfo}
\label{filetransferUDP.FileTransfer.lastInfo}\hypertarget{filetransferUDP.FileTransfer.lastInfo}{\texttt{ long\ {\bf  lastInfo}}
}
}
\item{
\index{speed}
\label{filetransferUDP.FileTransfer.speed}\hypertarget{filetransferUDP.FileTransfer.speed}{\texttt{ int\ {\bf  speed}}
}
}
\item{
\index{packedAllBytes}
\label{filetransferUDP.FileTransfer.packedAllBytes}\hypertarget{filetransferUDP.FileTransfer.packedAllBytes}{\texttt{ boolean\ {\bf  packedAllBytes}}
}
}
\item{
\index{finished}
\label{filetransferUDP.FileTransfer.finished}\hypertarget{filetransferUDP.FileTransfer.finished}{\texttt{ boolean\ {\bf  finished}}
}
}
\item{
\index{clientFile}
\label{filetransferUDP.FileTransfer.clientFile}\hypertarget{filetransferUDP.FileTransfer.clientFile}{\texttt{ java.io.File\ {\bf  clientFile}}
}
}
\item{
\index{byteClientFile}
\label{filetransferUDP.FileTransfer.byteClientFile}\hypertarget{filetransferUDP.FileTransfer.byteClientFile}{\texttt{ byte\lbrack \rbrack \ {\bf  byteClientFile}}
}
}
\item{
\index{filePath}
\label{filetransferUDP.FileTransfer.filePath}\hypertarget{filetransferUDP.FileTransfer.filePath}{\texttt{ java.lang.String\ {\bf  filePath}}
}
}
\item{
\index{sendPrevious}
\label{filetransferUDP.FileTransfer.sendPrevious}\hypertarget{filetransferUDP.FileTransfer.sendPrevious}{\texttt{ boolean\ {\bf  sendPrevious}}
}
}
\item{
\index{activeSession}
\label{filetransferUDP.FileTransfer.activeSession}\hypertarget{filetransferUDP.FileTransfer.activeSession}{\texttt{ short\ {\bf  activeSession}}
}
}
\item{
\index{packetData}
\label{filetransferUDP.FileTransfer.packetData}\hypertarget{filetransferUDP.FileTransfer.packetData}{\texttt{ byte\lbrack \rbrack \ {\bf  packetData}}
}
}
\item{
\index{debug}
\label{filetransferUDP.FileTransfer.debug}\hypertarget{filetransferUDP.FileTransfer.debug}{\texttt{ boolean\ {\bf  debug}}
}
}
\item{
\index{LINE}
\label{filetransferUDP.FileTransfer.LINE}\hypertarget{filetransferUDP.FileTransfer.LINE}{\texttt{static final java.lang.String\ {\bf  LINE}}
}
}
\item{
\index{LINED}
\label{filetransferUDP.FileTransfer.LINED}\hypertarget{filetransferUDP.FileTransfer.LINED}{\texttt{static final java.lang.String\ {\bf  LINED}}
}
}
\end{itemize}
}
\subsubsection{Constructors}{
\vskip -2em
\begin{itemize}
\item{ 
\index{FileTransfer()}
\hypertarget{filetransferUDP.FileTransfer()}{{\bf  FileTransfer}\\}
\begin{lstlisting}[frame=none]
public FileTransfer()\end{lstlisting} %end signature
}%end item
\end{itemize}
}
\subsubsection{Methods}{
\vskip -2em
\begin{itemize}
\item{ 
\index{createFirstPacket()}
\hypertarget{filetransferUDP.FileTransfer.createFirstPacket()}{{\bf  createFirstPacket}\\}
\begin{lstlisting}[frame=none]
public byte[] createFirstPacket()\end{lstlisting} %end signature
\begin{itemize}
\item{
{\bf  Description}

Creates the first packet.
}
\item{{\bf  Returns} -- 
the first packet. 
}%end item
\end{itemize}
}%end item
\item{ 
\index{exitApp(String, int)}
\hypertarget{filetransferUDP.FileTransfer.exitApp(java.lang.String, int)}{{\bf  exitApp}\\}
\begin{lstlisting}[frame=none]
 void exitApp(java.lang.String message,int status)\end{lstlisting} %end signature
\begin{itemize}
\item{
{\bf  Description}

Closes the Client/Server.
}
\item{
{\bf  Parameters}
  \begin{itemize}
   \item{
\texttt{message} -- the message.}
   \item{
\texttt{status} -- 0 for a positive exit status, everything else for errors.}
  \end{itemize}
}%end item
\end{itemize}
}%end item
\item{ 
\index{flipPacketNumber()}
\hypertarget{filetransferUDP.FileTransfer.flipPacketNumber()}{{\bf  flipPacketNumber}\\}
\begin{lstlisting}[frame=none]
public void flipPacketNumber()\end{lstlisting} %end signature
\begin{itemize}
\item{
{\bf  Description}

Flips the packet number between 0 and 1.
}
\end{itemize}
}%end item
\item{ 
\index{generateCRC(byte\lbrack \rbrack )}
\hypertarget{filetransferUDP.FileTransfer.generateCRC(byte[])}{{\bf  generateCRC}\\}
\begin{lstlisting}[frame=none]
public byte[] generateCRC(byte[] packetContent)\end{lstlisting} %end signature
\begin{itemize}
\item{
{\bf  Description}

Generates the CRC value of a byte-array.
}
\item{
{\bf  Parameters}
  \begin{itemize}
   \item{
\texttt{packetContent} -- the byte-array.}
  \end{itemize}
}%end item
\item{{\bf  Returns} -- 
The CRC of that byte-array. 
}%end item
\end{itemize}
}%end item
\item{ 
\index{getAppLocation()}
\hypertarget{filetransferUDP.FileTransfer.getAppLocation()}{{\bf  getAppLocation}\\}
\begin{lstlisting}[frame=none]
 java.lang.String getAppLocation()\end{lstlisting} %end signature
\begin{itemize}
\item{
{\bf  Description}

Returns the Location where the Client/Server runs (without /bin).
}
\item{{\bf  Returns} -- 
location. 
}%end item
\end{itemize}
}%end item
\item{ 
\index{getByteFileLength()}
\hypertarget{filetransferUDP.FileTransfer.getByteFileLength()}{{\bf  getByteFileLength}\\}
\begin{lstlisting}[frame=none]
public byte[] getByteFileLength()\end{lstlisting} %end signature
}%end item
\item{ 
\index{getByteFileLengthLong()}
\hypertarget{filetransferUDP.FileTransfer.getByteFileLengthLong()}{{\bf  getByteFileLengthLong}\\}
\begin{lstlisting}[frame=none]
public long getByteFileLengthLong()\end{lstlisting} %end signature
}%end item
\item{ 
\index{getByteFileName()}
\hypertarget{filetransferUDP.FileTransfer.getByteFileName()}{{\bf  getByteFileName}\\}
\begin{lstlisting}[frame=none]
public byte[] getByteFileName()\end{lstlisting} %end signature
}%end item
\item{ 
\index{getByteFileNameLength()}
\hypertarget{filetransferUDP.FileTransfer.getByteFileNameLength()}{{\bf  getByteFileNameLength}\\}
\begin{lstlisting}[frame=none]
public byte[] getByteFileNameLength()\end{lstlisting} %end signature
}%end item
\item{ 
\index{getByteFileNameLengthShort()}
\hypertarget{filetransferUDP.FileTransfer.getByteFileNameLengthShort()}{{\bf  getByteFileNameLengthShort}\\}
\begin{lstlisting}[frame=none]
public int getByteFileNameLengthShort()\end{lstlisting} %end signature
}%end item
\item{ 
\index{getByteFileNameString()}
\hypertarget{filetransferUDP.FileTransfer.getByteFileNameString()}{{\bf  getByteFileNameString}\\}
\begin{lstlisting}[frame=none]
public java.lang.String getByteFileNameString()\end{lstlisting} %end signature
}%end item
\item{ 
\index{getByteFirstPacketCRC()}
\hypertarget{filetransferUDP.FileTransfer.getByteFirstPacketCRC()}{{\bf  getByteFirstPacketCRC}\\}
\begin{lstlisting}[frame=none]
public byte[] getByteFirstPacketCRC()\end{lstlisting} %end signature
}%end item
\item{ 
\index{getByteLastPacketCRC()}
\hypertarget{filetransferUDP.FileTransfer.getByteLastPacketCRC()}{{\bf  getByteLastPacketCRC}\\}
\begin{lstlisting}[frame=none]
public byte[] getByteLastPacketCRC()\end{lstlisting} %end signature
}%end item
\item{ 
\index{getBytePacketNumber()}
\hypertarget{filetransferUDP.FileTransfer.getBytePacketNumber()}{{\bf  getBytePacketNumber}\\}
\begin{lstlisting}[frame=none]
public byte getBytePacketNumber()\end{lstlisting} %end signature
}%end item
\item{ 
\index{getByteSessionNumber()}
\hypertarget{filetransferUDP.FileTransfer.getByteSessionNumber()}{{\bf  getByteSessionNumber}\\}
\begin{lstlisting}[frame=none]
public byte[] getByteSessionNumber()\end{lstlisting} %end signature
}%end item
\item{ 
\index{getByteSessionNumberShort()}
\hypertarget{filetransferUDP.FileTransfer.getByteSessionNumberShort()}{{\bf  getByteSessionNumberShort}\\}
\begin{lstlisting}[frame=none]
public short getByteSessionNumberShort()\end{lstlisting} %end signature
}%end item
\item{ 
\index{getByteSessionNumberShort(byte\lbrack \rbrack )}
\hypertarget{filetransferUDP.FileTransfer.getByteSessionNumberShort(byte[])}{{\bf  getByteSessionNumberShort}\\}
\begin{lstlisting}[frame=none]
public short getByteSessionNumberShort(byte[] sessionNumber)\end{lstlisting} %end signature
}%end item
\item{ 
\index{getByteStartIdentifier()}
\hypertarget{filetransferUDP.FileTransfer.getByteStartIdentifier()}{{\bf  getByteStartIdentifier}\\}
\begin{lstlisting}[frame=none]
public byte[] getByteStartIdentifier()\end{lstlisting} %end signature
}%end item
\item{ 
\index{getByteStartIdentifierString()}
\hypertarget{filetransferUDP.FileTransfer.getByteStartIdentifierString()}{{\bf  getByteStartIdentifierString}\\}
\begin{lstlisting}[frame=none]
public java.lang.String getByteStartIdentifierString()\end{lstlisting} %end signature
}%end item
\item{ 
\index{getClientFile()}
\hypertarget{filetransferUDP.FileTransfer.getClientFile()}{{\bf  getClientFile}\\}
\begin{lstlisting}[frame=none]
public java.io.File getClientFile()\end{lstlisting} %end signature
}%end item
\item{ 
\index{getConnectionIP()}
\hypertarget{filetransferUDP.FileTransfer.getConnectionIP()}{{\bf  getConnectionIP}\\}
\begin{lstlisting}[frame=none]
public java.net.InetAddress getConnectionIP()\end{lstlisting} %end signature
}%end item
\item{ 
\index{getCurrentPacket()}
\hypertarget{filetransferUDP.FileTransfer.getCurrentPacket()}{{\bf  getCurrentPacket}\\}
\begin{lstlisting}[frame=none]
public byte[] getCurrentPacket()\end{lstlisting} %end signature
}%end item
\item{ 
\index{getDataPacket()}
\hypertarget{filetransferUDP.FileTransfer.getDataPacket()}{{\bf  getDataPacket}\\}
\begin{lstlisting}[frame=none]
public java.net.DatagramPacket getDataPacket()\end{lstlisting} %end signature
}%end item
\item{ 
\index{getDataSocket()}
\hypertarget{filetransferUDP.FileTransfer.getDataSocket()}{{\bf  getDataSocket}\\}
\begin{lstlisting}[frame=none]
public java.net.DatagramSocket getDataSocket()\end{lstlisting} %end signature
}%end item
\item{ 
\index{getFileName()}
\hypertarget{filetransferUDP.FileTransfer.getFileName()}{{\bf  getFileName}\\}
\begin{lstlisting}[frame=none]
public java.lang.String getFileName()\end{lstlisting} %end signature
}%end item
\item{ 
\index{getFilePath()}
\hypertarget{filetransferUDP.FileTransfer.getFilePath()}{{\bf  getFilePath}\\}
\begin{lstlisting}[frame=none]
public java.lang.String getFilePath()\end{lstlisting} %end signature
}%end item
\item{ 
\index{getFirstPacketContents(byte\lbrack \rbrack )}
\hypertarget{filetransferUDP.FileTransfer.getFirstPacketContents(byte[])}{{\bf  getFirstPacketContents}\\}
\begin{lstlisting}[frame=none]
public void getFirstPacketContents(byte[] packet)\end{lstlisting} %end signature
\begin{itemize}
\item{
{\bf  Description}

Parses the first packet and saves its contents.
}
\item{
{\bf  Parameters}
  \begin{itemize}
   \item{
\texttt{packet} -- the first packet.}
  \end{itemize}
}%end item
\end{itemize}
}%end item
\item{ 
\index{getNewFileName(String)}
\hypertarget{filetransferUDP.FileTransfer.getNewFileName(java.lang.String)}{{\bf  getNewFileName}\\}
\begin{lstlisting}[frame=none]
 java.lang.String getNewFileName(java.lang.String fileName)\end{lstlisting} %end signature
\begin{itemize}
\item{
{\bf  Description}

Generates a new file name for saving a downloaded file.
}
\item{
{\bf  Parameters}
  \begin{itemize}
   \item{
\texttt{fileName} -- the original file name.}
  \end{itemize}
}%end item
\item{{\bf  Returns} -- 
the file name. 
}%end item
\end{itemize}
}%end item
\item{ 
\index{getPACKETDATALENGTH()}
\hypertarget{filetransferUDP.FileTransfer.getPACKETDATALENGTH()}{{\bf  getPACKETDATALENGTH}\\}
\begin{lstlisting}[frame=none]
public int getPACKETDATALENGTH()\end{lstlisting} %end signature
}%end item
\item{ 
\index{getPacketDelay()}
\hypertarget{filetransferUDP.FileTransfer.getPacketDelay()}{{\bf  getPacketDelay}\\}
\begin{lstlisting}[frame=none]
public int getPacketDelay()\end{lstlisting} %end signature
}%end item
\item{ 
\index{getPacketLossRate()}
\hypertarget{filetransferUDP.FileTransfer.getPacketLossRate()}{{\bf  getPacketLossRate}\\}
\begin{lstlisting}[frame=none]
public float getPacketLossRate()\end{lstlisting} %end signature
}%end item
\item{ 
\index{getPort()}
\hypertarget{filetransferUDP.FileTransfer.getPort()}{{\bf  getPort}\\}
\begin{lstlisting}[frame=none]
public int getPort()\end{lstlisting} %end signature
}%end item
\item{ 
\index{getPreviousPacket()}
\hypertarget{filetransferUDP.FileTransfer.getPreviousPacket()}{{\bf  getPreviousPacket}\\}
\begin{lstlisting}[frame=none]
public byte[] getPreviousPacket()\end{lstlisting} %end signature
}%end item
\item{ 
\index{getPrintableDebugStatus()}
\hypertarget{filetransferUDP.FileTransfer.getPrintableDebugStatus()}{{\bf  getPrintableDebugStatus}\\}
\begin{lstlisting}[frame=none]
 java.lang.String getPrintableDebugStatus()\end{lstlisting} %end signature
\begin{itemize}
\item{
{\bf  Description}

Creates a nice line for displaying the debug-status.
}
\item{{\bf  Returns} -- 
line with debug information. 
}%end item
\end{itemize}
}%end item
\item{ 
\index{getPrintableLossRateDelayValues()}
\hypertarget{filetransferUDP.FileTransfer.getPrintableLossRateDelayValues()}{{\bf  getPrintableLossRateDelayValues}\\}
\begin{lstlisting}[frame=none]
 java.lang.String getPrintableLossRateDelayValues()\end{lstlisting} %end signature
\begin{itemize}
\item{
{\bf  Description}

Packs the loss rate and delay value into a nice line.
}
\item{{\bf  Returns} -- 
Line with packet loss and delay values. 
}%end item
\end{itemize}
}%end item
\item{ 
\index{isCorrectCRC(byte\lbrack \rbrack , byte\lbrack \rbrack )}
\hypertarget{filetransferUDP.FileTransfer.isCorrectCRC(byte[], byte[])}{{\bf  isCorrectCRC}\\}
\begin{lstlisting}[frame=none]
public boolean isCorrectCRC(byte[] packetContent,byte[] crc)\end{lstlisting} %end signature
\begin{itemize}
\item{
{\bf  Description}

Compares the CRC value of a given byte-array with another CRC value.
}
\item{
{\bf  Parameters}
  \begin{itemize}
   \item{
\texttt{packetContent} -- the byte-array.}
   \item{
\texttt{crc} -- the CRC number.}
  \end{itemize}
}%end item
\item{{\bf  Returns} -- 
true, if it is the same CRC number. 
}%end item
\end{itemize}
}%end item
\item{ 
\index{isCorrectPacketNumber(byte)}
\hypertarget{filetransferUDP.FileTransfer.isCorrectPacketNumber(byte)}{{\bf  isCorrectPacketNumber}\\}
\begin{lstlisting}[frame=none]
public boolean isCorrectPacketNumber(byte packetNumber)\end{lstlisting} %end signature
\begin{itemize}
\item{
{\bf  Description}

Checks, if the packet number is as expected.
}
\item{
{\bf  Parameters}
  \begin{itemize}
   \item{
\texttt{packetNumber} -- the packet number of the packet to check.}
  \end{itemize}
}%end item
\item{{\bf  Returns} -- 
true, if it is the correct number. 
}%end item
\end{itemize}
}%end item
\item{ 
\index{isCorrectSessionNumber(byte\lbrack \rbrack )}
\hypertarget{filetransferUDP.FileTransfer.isCorrectSessionNumber(byte[])}{{\bf  isCorrectSessionNumber}\\}
\begin{lstlisting}[frame=none]
public boolean isCorrectSessionNumber(byte[] sessionNumber)\end{lstlisting} %end signature
\begin{itemize}
\item{
{\bf  Description}

Checks, if the session number is as expected.
}
\item{
{\bf  Parameters}
  \begin{itemize}
   \item{
\texttt{sessionNumber} -- the session number of the packet to check.}
  \end{itemize}
}%end item
\item{{\bf  Returns} -- 
true, if it is the correct number. 
}%end item
\end{itemize}
}%end item
\item{ 
\index{isCorrectStartIdentifier(byte\lbrack \rbrack )}
\hypertarget{filetransferUDP.FileTransfer.isCorrectStartIdentifier(byte[])}{{\bf  isCorrectStartIdentifier}\\}
\begin{lstlisting}[frame=none]
public boolean isCorrectStartIdentifier(byte[] startIdentifier)\end{lstlisting} %end signature
\begin{itemize}
\item{
{\bf  Description}

Checks, if the start identifier is correct ASCII "Start".
}
\item{
{\bf  Parameters}
  \begin{itemize}
   \item{
\texttt{startIdentifier} -- the identifier.}
  \end{itemize}
}%end item
\item{{\bf  Returns} -- 
true, if the identifier is correct. 
}%end item
\end{itemize}
}%end item
\item{ 
\index{isDebug()}
\hypertarget{filetransferUDP.FileTransfer.isDebug()}{{\bf  isDebug}\\}
\begin{lstlisting}[frame=none]
public boolean isDebug()\end{lstlisting} %end signature
}%end item
\item{ 
\index{printMessage(String)}
\hypertarget{filetransferUDP.FileTransfer.printMessage(java.lang.String)}{{\bf  printMessage}\\}
\begin{lstlisting}[frame=none]
 void printMessage(java.lang.String message)\end{lstlisting} %end signature
\begin{itemize}
\item{
{\bf  Description}

Prints a status message.
}
\item{
{\bf  Parameters}
  \begin{itemize}
   \item{
\texttt{message} -- the message.}
  \end{itemize}
}%end item
\end{itemize}
}%end item
\item{ 
\index{printMessage(String, int)}
\hypertarget{filetransferUDP.FileTransfer.printMessage(java.lang.String, int)}{{\bf  printMessage}\\}
\begin{lstlisting}[frame=none]
 void printMessage(java.lang.String message,int type)\end{lstlisting} %end signature
\begin{itemize}
\item{
{\bf  Description}

Prints a message to the console.
}
\item{
{\bf  Parameters}
  \begin{itemize}
   \item{
\texttt{message} -- the message.}
   \item{
\texttt{type} -- 1 is debug (only printed in debug mode), 2 for errors, everything else a status.}
  \end{itemize}
}%end item
\end{itemize}
}%end item
\item{ 
\index{printSessionData()}
\hypertarget{filetransferUDP.FileTransfer.printSessionData()}{{\bf  printSessionData}\\}
\begin{lstlisting}[frame=none]
public void printSessionData()\end{lstlisting} %end signature
\begin{itemize}
\item{
{\bf  Description}

Prints the data of a (new) session.
}
\end{itemize}
}%end item
\item{ 
\index{resetSession()}
\hypertarget{filetransferUDP.FileTransfer.resetSession()}{{\bf  resetSession}\\}
\begin{lstlisting}[frame=none]
public void resetSession()\end{lstlisting} %end signature
\begin{itemize}
\item{
{\bf  Description}

Resets a session.
}
\end{itemize}
}%end item
\item{ 
\index{setActiveSession()}
\hypertarget{filetransferUDP.FileTransfer.setActiveSession()}{{\bf  setActiveSession}\\}
\begin{lstlisting}[frame=none]
public void setActiveSession()\end{lstlisting} %end signature
}%end item
\item{ 
\index{setByteFileLength()}
\hypertarget{filetransferUDP.FileTransfer.setByteFileLength()}{{\bf  setByteFileLength}\\}
\begin{lstlisting}[frame=none]
public void setByteFileLength()\end{lstlisting} %end signature
}%end item
\item{ 
\index{setByteFileLength(byte\lbrack \rbrack )}
\hypertarget{filetransferUDP.FileTransfer.setByteFileLength(byte[])}{{\bf  setByteFileLength}\\}
\begin{lstlisting}[frame=none]
public void setByteFileLength(byte[] fileLength)\end{lstlisting} %end signature
}%end item
\item{ 
\index{setByteFileName()}
\hypertarget{filetransferUDP.FileTransfer.setByteFileName()}{{\bf  setByteFileName}\\}
\begin{lstlisting}[frame=none]
public void setByteFileName()\end{lstlisting} %end signature
}%end item
\item{ 
\index{setByteFileName(byte\lbrack \rbrack )}
\hypertarget{filetransferUDP.FileTransfer.setByteFileName(byte[])}{{\bf  setByteFileName}\\}
\begin{lstlisting}[frame=none]
public void setByteFileName(byte[] byteFileName)\end{lstlisting} %end signature
}%end item
\item{ 
\index{setByteFileNameLength()}
\hypertarget{filetransferUDP.FileTransfer.setByteFileNameLength()}{{\bf  setByteFileNameLength}\\}
\begin{lstlisting}[frame=none]
public void setByteFileNameLength()\end{lstlisting} %end signature
}%end item
\item{ 
\index{setByteFileNameLength(byte\lbrack \rbrack )}
\hypertarget{filetransferUDP.FileTransfer.setByteFileNameLength(byte[])}{{\bf  setByteFileNameLength}\\}
\begin{lstlisting}[frame=none]
public void setByteFileNameLength(byte[] fileNameLength)\end{lstlisting} %end signature
}%end item
\item{ 
\index{setByteFirstPacketCRC(byte\lbrack \rbrack )}
\hypertarget{filetransferUDP.FileTransfer.setByteFirstPacketCRC(byte[])}{{\bf  setByteFirstPacketCRC}\\}
\begin{lstlisting}[frame=none]
public void setByteFirstPacketCRC(byte[] byteFirstPacketCRC)\end{lstlisting} %end signature
}%end item
\item{ 
\index{setByteLastPacketCRC(byte\lbrack \rbrack )}
\hypertarget{filetransferUDP.FileTransfer.setByteLastPacketCRC(byte[])}{{\bf  setByteLastPacketCRC}\\}
\begin{lstlisting}[frame=none]
public void setByteLastPacketCRC(byte[] byteLastPacketCRC)\end{lstlisting} %end signature
}%end item
\item{ 
\index{setBytePacketNumber(byte)}
\hypertarget{filetransferUDP.FileTransfer.setBytePacketNumber(byte)}{{\bf  setBytePacketNumber}\\}
\begin{lstlisting}[frame=none]
public void setBytePacketNumber(byte packetNumber)\end{lstlisting} %end signature
}%end item
\item{ 
\index{setBytePacketNumber(int)}
\hypertarget{filetransferUDP.FileTransfer.setBytePacketNumber(int)}{{\bf  setBytePacketNumber}\\}
\begin{lstlisting}[frame=none]
public void setBytePacketNumber(int packetNumber)\end{lstlisting} %end signature
}%end item
\item{ 
\index{setByteSessionNumber()}
\hypertarget{filetransferUDP.FileTransfer.setByteSessionNumber()}{{\bf  setByteSessionNumber}\\}
\begin{lstlisting}[frame=none]
public void setByteSessionNumber()\end{lstlisting} %end signature
}%end item
\item{ 
\index{setByteSessionNumber(byte\lbrack \rbrack )}
\hypertarget{filetransferUDP.FileTransfer.setByteSessionNumber(byte[])}{{\bf  setByteSessionNumber}\\}
\begin{lstlisting}[frame=none]
public void setByteSessionNumber(byte[] sessionNumber)\end{lstlisting} %end signature
}%end item
\item{ 
\index{setByteStartIdentifier()}
\hypertarget{filetransferUDP.FileTransfer.setByteStartIdentifier()}{{\bf  setByteStartIdentifier}\\}
\begin{lstlisting}[frame=none]
public void setByteStartIdentifier()\end{lstlisting} %end signature
}%end item
\item{ 
\index{setByteStartIdentifier(byte\lbrack \rbrack )}
\hypertarget{filetransferUDP.FileTransfer.setByteStartIdentifier(byte[])}{{\bf  setByteStartIdentifier}\\}
\begin{lstlisting}[frame=none]
public void setByteStartIdentifier(byte[] startIdentifier)\end{lstlisting} %end signature
}%end item
\item{ 
\index{setClientFile(File)}
\hypertarget{filetransferUDP.FileTransfer.setClientFile(java.io.File)}{{\bf  setClientFile}\\}
\begin{lstlisting}[frame=none]
public void setClientFile(java.io.File file)\end{lstlisting} %end signature
}%end item
\item{ 
\index{setClientFile(String)}
\hypertarget{filetransferUDP.FileTransfer.setClientFile(java.lang.String)}{{\bf  setClientFile}\\}
\begin{lstlisting}[frame=none]
public void setClientFile(java.lang.String filePath)\end{lstlisting} %end signature
}%end item
\item{ 
\index{setConnectionIP(InetAddress)}
\hypertarget{filetransferUDP.FileTransfer.setConnectionIP(java.net.InetAddress)}{{\bf  setConnectionIP}\\}
\begin{lstlisting}[frame=none]
public void setConnectionIP(java.net.InetAddress connectionIP)\end{lstlisting} %end signature
}%end item
\item{ 
\index{setConnectionIP(String)}
\hypertarget{filetransferUDP.FileTransfer.setConnectionIP(java.lang.String)}{{\bf  setConnectionIP}\\}
\begin{lstlisting}[frame=none]
public void setConnectionIP(java.lang.String connectionName)\end{lstlisting} %end signature
}%end item
\item{ 
\index{setCurrentPacket(byte\lbrack \rbrack )}
\hypertarget{filetransferUDP.FileTransfer.setCurrentPacket(byte[])}{{\bf  setCurrentPacket}\\}
\begin{lstlisting}[frame=none]
public void setCurrentPacket(byte[] currentPacket)\end{lstlisting} %end signature
}%end item
\item{ 
\index{setDataPacket(DatagramPacket)}
\hypertarget{filetransferUDP.FileTransfer.setDataPacket(java.net.DatagramPacket)}{{\bf  setDataPacket}\\}
\begin{lstlisting}[frame=none]
public void setDataPacket(java.net.DatagramPacket dataPacket)\end{lstlisting} %end signature
}%end item
\item{ 
\index{setDataSocket(DatagramSocket)}
\hypertarget{filetransferUDP.FileTransfer.setDataSocket(java.net.DatagramSocket)}{{\bf  setDataSocket}\\}
\begin{lstlisting}[frame=none]
public void setDataSocket(java.net.DatagramSocket dataSocket)\end{lstlisting} %end signature
}%end item
\item{ 
\index{setDebug(boolean)}
\hypertarget{filetransferUDP.FileTransfer.setDebug(boolean)}{{\bf  setDebug}\\}
\begin{lstlisting}[frame=none]
public void setDebug(boolean debug)\end{lstlisting} %end signature
}%end item
\item{ 
\index{setFileName(String)}
\hypertarget{filetransferUDP.FileTransfer.setFileName(java.lang.String)}{{\bf  setFileName}\\}
\begin{lstlisting}[frame=none]
private void setFileName(java.lang.String fileName)\end{lstlisting} %end signature
}%end item
\item{ 
\index{setFilePath(String)}
\hypertarget{filetransferUDP.FileTransfer.setFilePath(java.lang.String)}{{\bf  setFilePath}\\}
\begin{lstlisting}[frame=none]
public void setFilePath(java.lang.String filePath)\end{lstlisting} %end signature
}%end item
\item{ 
\index{setPacketDelay(int)}
\hypertarget{filetransferUDP.FileTransfer.setPacketDelay(int)}{{\bf  setPacketDelay}\\}
\begin{lstlisting}[frame=none]
public void setPacketDelay(int packetDelay)\end{lstlisting} %end signature
}%end item
\item{ 
\index{setPacketLossRate(float)}
\hypertarget{filetransferUDP.FileTransfer.setPacketLossRate(float)}{{\bf  setPacketLossRate}\\}
\begin{lstlisting}[frame=none]
public void setPacketLossRate(float packetLossRate)\end{lstlisting} %end signature
}%end item
\item{ 
\index{setPort(int)}
\hypertarget{filetransferUDP.FileTransfer.setPort(int)}{{\bf  setPort}\\}
\begin{lstlisting}[frame=none]
public void setPort(int port)\end{lstlisting} %end signature
}%end item
\item{ 
\index{setPreviousPacket(byte\lbrack \rbrack )}
\hypertarget{filetransferUDP.FileTransfer.setPreviousPacket(byte[])}{{\bf  setPreviousPacket}\\}
\begin{lstlisting}[frame=none]
public void setPreviousPacket(byte[] previousPacket)\end{lstlisting} %end signature
}%end item
\item{ 
\index{toggleBytePacketNumber()}
\hypertarget{filetransferUDP.FileTransfer.toggleBytePacketNumber()}{{\bf  toggleBytePacketNumber}\\}
\begin{lstlisting}[frame=none]
public void toggleBytePacketNumber()\end{lstlisting} %end signature
}%end item
\item{ 
\index{verifyACK(byte\lbrack \rbrack )}
\hypertarget{filetransferUDP.FileTransfer.verifyACK(byte[])}{{\bf  verifyACK}\\}
\begin{lstlisting}[frame=none]
public int verifyACK(byte[] packet)\end{lstlisting} %end signature
\begin{itemize}
\item{
{\bf  Description}

Checks, if the acknowledge packet is correct (with the right session and packet number).
}
\item{
{\bf  Parameters}
  \begin{itemize}
   \item{
\texttt{packet} -- the packet to check.}
  \end{itemize}
}%end item
\item{{\bf  Returns} -- 
0 if correct, -1 if the session number is wrong and 1 if the packet number is wrong. 
}%end item
\end{itemize}
}%end item
\end{itemize}
}
}
\subsection{\label{filetransferUDP.Server}Class Server}{
\hypertarget{filetransferUDP.Server}{}\vskip .1in 
The server tries to download files from a client.\vskip .1in 
\subsubsection{Declaration}{
\begin{lstlisting}[frame=none]
public class Server
 extends filetransferUDP.FileTransfer\end{lstlisting}
\subsubsection{Field summary}{
\begin{verse}
\hyperlink{filetransferUDP.Server.ARGUMENT_MESSAGE}{{\bf ARGUMENT\_MESSAGE}} \\
\end{verse}
}
\subsubsection{Constructor summary}{
\begin{verse}
\hyperlink{filetransferUDP.Server(java.lang.String[])}{{\bf Server(String\lbrack \rbrack )}} Executes the server.\\
\end{verse}
}
\subsubsection{Method summary}{
\begin{verse}
\hyperlink{filetransferUDP.Server.displayArguments()}{{\bf displayArguments()}} Displays the server arguments.\\
\hyperlink{filetransferUDP.Server.initializeDownload()}{{\bf initializeDownload()}} Sets the Sockets up.\\
\hyperlink{filetransferUDP.Server.main(java.lang.String[])}{{\bf main(String\lbrack \rbrack )}} Calls Server(args) and thereby starts the server.\\
\hyperlink{filetransferUDP.Server.parseArguments(java.lang.String[])}{{\bf parseArguments(String\lbrack \rbrack )}} Parses the command line arguments.\\
\hyperlink{filetransferUDP.Server.proccessPacket()}{{\bf proccessPacket()}} Processes the received packet.\\
\hyperlink{filetransferUDP.Server.receivePacket()}{{\bf receivePacket()}} Tries to receive a packet.\\
\hyperlink{filetransferUDP.Server.sendACK(byte)}{{\bf sendACK(byte)}} Sends an acknowledge packet to the client.\\
\end{verse}
}
\subsubsection{Fields}{
\begin{itemize}
\item{
\index{ARGUMENT\_MESSAGE}
\label{filetransferUDP.Server.ARGUMENT_MESSAGE}\hypertarget{filetransferUDP.Server.ARGUMENT_MESSAGE}{\texttt{static final java.lang.String\ {\bf  ARGUMENT\_MESSAGE}}
}
}
\end{itemize}
}
\subsubsection{Constructors}{
\vskip -2em
\begin{itemize}
\item{ 
\index{Server(String\lbrack \rbrack )}
\hypertarget{filetransferUDP.Server(java.lang.String[])}{{\bf  Server}\\}
\begin{lstlisting}[frame=none]
public Server(java.lang.String[] args)\end{lstlisting} %end signature
\begin{itemize}
\item{
{\bf  Description}

Executes the server.
}
\item{
{\bf  Parameters}
  \begin{itemize}
   \item{
\texttt{args} -- command line arguments}
  \end{itemize}
}%end item
\end{itemize}
}%end item
\end{itemize}
}
\subsubsection{Methods}{
\vskip -2em
\begin{itemize}
\item{ 
\index{displayArguments()}
\hypertarget{filetransferUDP.Server.displayArguments()}{{\bf  displayArguments}\\}
\begin{lstlisting}[frame=none]
private void displayArguments()\end{lstlisting} %end signature
\begin{itemize}
\item{
{\bf  Description}

Displays the server arguments.
}
\end{itemize}
}%end item
\item{ 
\index{initializeDownload()}
\hypertarget{filetransferUDP.Server.initializeDownload()}{{\bf  initializeDownload}\\}
\begin{lstlisting}[frame=none]
private void initializeDownload()\end{lstlisting} %end signature
\begin{itemize}
\item{
{\bf  Description}

Sets the Sockets up.
}
\end{itemize}
}%end item
\item{ 
\index{main(String\lbrack \rbrack )}
\hypertarget{filetransferUDP.Server.main(java.lang.String[])}{{\bf  main}\\}
\begin{lstlisting}[frame=none]
public static void main(java.lang.String[] args)\end{lstlisting} %end signature
\begin{itemize}
\item{
{\bf  Description}

Calls Server(args) and thereby starts the server.
}
\item{
{\bf  Parameters}
  \begin{itemize}
   \item{
\texttt{args} -- command line arguments}
  \end{itemize}
}%end item
\end{itemize}
}%end item
\item{ 
\index{parseArguments(String\lbrack \rbrack )}
\hypertarget{filetransferUDP.Server.parseArguments(java.lang.String[])}{{\bf  parseArguments}\\}
\begin{lstlisting}[frame=none]
private void parseArguments(java.lang.String[] args)\end{lstlisting} %end signature
\begin{itemize}
\item{
{\bf  Description}

Parses the command line arguments.
}
\item{
{\bf  Parameters}
  \begin{itemize}
   \item{
\texttt{args} -- command line arguments.}
  \end{itemize}
}%end item
\end{itemize}
}%end item
\item{ 
\index{proccessPacket()}
\hypertarget{filetransferUDP.Server.proccessPacket()}{{\bf  proccessPacket}\\}
\begin{lstlisting}[frame=none]
private void proccessPacket()\end{lstlisting} %end signature
\begin{itemize}
\item{
{\bf  Description}

Processes the received packet. The packet is either a first packet or a data packet.\mbox{}\newline If everything went o.k., an ACK is sent.\mbox{}\newline If the data packet has the wrong packet number, an ACK with the expected packet number is sent.
}
\end{itemize}
}%end item
\item{ 
\index{receivePacket()}
\hypertarget{filetransferUDP.Server.receivePacket()}{{\bf  receivePacket}\\}
\begin{lstlisting}[frame=none]
private void receivePacket()\end{lstlisting} %end signature
\begin{itemize}
\item{
{\bf  Description}

Tries to receive a packet. If the receiving process times out and a session was active, delete the session.
}
\end{itemize}
}%end item
\item{ 
\index{sendACK(byte)}
\hypertarget{filetransferUDP.Server.sendACK(byte)}{{\bf  sendACK}\\}
\begin{lstlisting}[frame=none]
private void sendACK(byte packetNumber)\end{lstlisting} %end signature
\begin{itemize}
\item{
{\bf  Description}

Sends an acknowledge packet to the client. If command line arguments are set, it may delay or lose the packet.
}
\item{
{\bf  Parameters}
  \begin{itemize}
   \item{
\texttt{packetNumber} -- the packet number which should be acknowledged.}
  \end{itemize}
}%end item
\end{itemize}
}%end item
\end{itemize}
}
}
}


%\newpage
%\printbibliography

\end{document}