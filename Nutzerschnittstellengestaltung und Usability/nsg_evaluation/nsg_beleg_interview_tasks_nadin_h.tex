\subsection*{Begrüßung}
[...]

Wenn Du dem zustimmst, unterschreibe bitte hier:

\vspace*{1em}
Nadin Hahn [13.01.2018]\vspace*{-.9em}\\
\noindent\rule{8cm}{0.4pt}

\subsection*{Einführung}
[...]


\subsection*{Test-Aufgaben}
\subsubsection*{Alle Seiten besuchen}
\begin{enumerate}
\item Wie lautet die Email-Adresse des Admins der Website? Wie ist der Name des Seiteninhabers?
\interviewText{Sucht auf der Startseite nach „Impressum“, findet dies nicht. Per Ausschlussverfahren kommt sie so zum Link „Über“.}
\item Finde eine Seite, die den Text „schwarz-lila“ enthält. Wie ist der Titel dieser Seite?
\interviewText{Geht intuitiv zum Suchfeld, gibt den Begriff ein und findet die Seite.}
\item Für welche Spiele werden auf der Seite Karten/Maps/Level angeboten?
\interviewText{Der erste Impuls ist die Suche: Dort findet sie mit dem Begriff „Karte“ allerdings nur die Counter-Strike Source Karten. Über die Breadcrumbs findet sie im Bereich „Kreativ“ noch die Half-life 2 Mod. Weil sie sich unsicher ist, navigiert sie noch zu der Code Seite und guckt alles durch. Dabei orientiert sie sich an der Zusammenfassung, die in der Übersicht aller Seiten angezeigt wird, und klickt nicht jede einzelne Seite an.

Insgesamt ist sie sich nicht sicher, ob sie alles erwischt hat. Ihr ist nicht ganz klar, war die Inhalte kategorisiert sind. Ihr fällt auf, dass der Inhalt der Webseite sehr gemischt ist.}
\item Finde den Sammelband mit drei Geschichten. Was assoziierst du mit dem Titelbild?
\interviewText{Die Seite ist ihr bei der akribischen Suche der Karten schon über den Weg gelaufen und navigiert gezielt über „Kreativ“ dort hin. Sie ließt die ersten Sätze und möchte sich sofort die pdf angucken und merkt erst dann, dass sie passwortgeschützt ist. Der Text hinter dem Link ist ihr nicht sofort ins Auge gefallen [vielleicht ein Lock-/Schlüsselsymbol dran machen, damit das klar ist?].}
\item Was für Software wird auf der Webseite vorgestellt?
\interviewText{Ihr ist der Begriff Software nicht 100\% geläufig. Sie findet zwar unter Code, was sie sucht, aber weiß nicht, was sie als Antwort notieren soll. Sie schreibt C und C++ Ausleihbibliothek. Wichtelsystem wird korrekt erkannt.}
\item Unter welchem Betriebssystem ist die Ausleihbibliothek in C++ ausführbar?
\interviewText{Liest die entsprechende Seite durch und findet die Antwort.}
\item Welche Medien sind in der Ausleihbibliothek in C ausleihbar?
\interviewText{Sie navigiert zur Ausleihbibliothek in C Seite über das Menü in der Titelleiste, nicht über Breadcrumbs. Findet dort die Antwort.}
\item Wo sind Studiendokumente zu finden?
\interviewText{Den Eintrag hat sie bereits gesehen, wundert sich aber trotzdem über die Kategorie. Warum steht es unter Code?}
\item Stelle die Sprache auf Englisch um. Findest Du eine Seite, die nicht auf Englisch übersetzt ist? Wenn ja, wie ist der Titel dieser Seite?
\interviewText{Option zum Umstellen der Sprache wurde sofort gefunden. Klickt sich durch alle Seiten durch und findet die AWS-Seite durch probieren.}
\end{enumerate}


\subsubsection*{Automatisiertes Wichtel System}
\begin{enumerate}
\item Benutze das Automatisierte Wichtel System:
[...]
\interviewText{
\begin{itemize}
\item Starten und Eintragen der Wichtel intuitiv. Weiter-Button sofort gefunden.
\item Ausnahme hinzufügen: Probiert aus. Ihr wird klar dass „rot“ nicht bewichteln bedeutet und „grün“ bewichteln. An der Stelle wünscht sie sich aber lieber eine Legende.
\item Ihr ist nicht klar, wessen Mail sie am besten bei der Mail-Formulierung eintragen soll (vielleicht noch deutlicher machen, dass es zur Verifizierung benötigt ist).
\end{itemize}
}
\end{enumerate}

\subsubsection*{Responsive Design}
\begin{enumerate}
\item Rufe die Website auf deinem Handy auf:
[...]
\interviewText{
\begin{itemize}
\item Der Banner mit der Cookie-Warnung wird intuitiv und ohne Kommentar weg geklickt.
\item Navigieren klappt problemlos.
\item Beim AWS ist der Button zum Starten des Prozesses zu lang.
\item Beim AWS sind die E-Mail Textänderungsfelder zu klein. man muss im Feld scrollen $\to$ sehr unübersichtlich.
\item Sonst Benutzung des AWS problemlos.
\end{itemize}}
\end{enumerate}