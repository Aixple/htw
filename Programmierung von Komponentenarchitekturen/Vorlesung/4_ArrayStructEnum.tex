\section{Enumeration (Aufzählung)}
\lecdate{31.03.2017}
\subsubsection*{Beispiel}
\begin{lstlisting}[language={[Sharp]C}]
public enum Karte{
	Karo = 9,
	Herz = 10,
	Pik = 11,
	Kreuz = 12
}
\end{lstlisting}

\subsection{Eigenschaften}

\begin{itemize}
\item gruppiert Konstanten die thematisch verwandt sind
\item Konstanten sind (wenn nicht anders angegeben) vom Typ int
\item ohne Angabe von Werten werden automatisch Werte vergeben (0, … , n)
\item ein Objekt von Typ enum hat genau einen Wert
\end{itemize}
\begin{lstlisting}[language={[Sharp]C}]
Karte card1;
card1 = Karte.Herz;
if (card1 == Karte.Herz)
	Console.WriteLine("Sie haben Herz.");
\end{lstlisting}

\subsection{Einen enum durchlaufen}
\begin{lstlisting}[language={[Sharp]C}]
foreach (Karte karte in Enum.GetValues(typeof(Karte)))
	Console.WriteLine(karte);
\end{lstlisting}

\section{Struct}

\begin{itemize}
\item ähnlich wie class, aber Wertetyp (auf dem Stack abgelegt)
\item Unterschiede zu class:
\begin{itemize}
\item kann nicht abgeleitet werden (erbt nur von ValueType)
\item besitzt immer parameterlosen Konstruktor
\item Speicherallokation mit new nicht nötig, wenn auf Initialisierung der Felder des struct verzichtet werden kann
\end{itemize}
\end{itemize}
Beispiele: alle primitiven Datentypen

\section{Array}

Arrays erben direkt von \lstinline$System.Object$, sind also Referenztypen!
\begin{lstlisting}[language={[Sharp]C}]
int[] elements = new int[3];
elements = {23, 9, 7};

// mehrdimensionale Arrays:
int[,] points = new int[4,3];
int[,] points = {{1,2,3},{4,5,6}};

// verzweigte Arrays:
int[][] myArray = new int[3][];
myArray[0] = new int[3];
myArray[1] = new int[4];
myArray[2] = new int[2];
\end{lstlisting}