
\section*{Grundelemente des Prozessors} WICHTIG!
\slides{02}{1}
\section{Rechenwerk}
\subsection*{Aufbau und Ansteuerung}
\slides{02}{2}
\subsection*{Operationen einer (Ganzzahl-)ALU}
\slides{02}{3}

\subsection{Addierer}
\subsubsection*{Addierwerk}
\slides{02}{4}
\subsubsection*{Parallele Addierer: Ripple Carry Addierer (RCA)}
\slides{02}{5}
\subsubsection*{Parallele Addierer: Carry-Look-Ahead-Addierer (CLA)}
\slides{02}{6}

\subsection{Multiplizierer}
\unimptnt{
\subsubsection{Multiplikation von Dualzahlen}
\slides{02}{7}
}

\subsubsection{Multiplizierwerk::seriell}
\slides{02}{8}
\subsubsection{Multiplizierwerk::parallel}
\slides{02}{9}

\unimptnt{
\subsubsection{Weitere Multiplizierwerke}
\begin{itemize}
\item Wallace Tree
\item Binary Tree
\item Booth-Algorithmus
\item Carry Save Multiplier
\end{itemize}
}

\section{Steuerwerk}

\subsection*{Grundaufgaben}
\slides{02}{10}
\subsection*{Phaseneinteilung der Befehlsbearbeitung}
\slides{02}{11}
\subsection*{Steuerwerk und Operationswerk}
\slides{02}{12}
\subsection*{Realisierungstechniken}
\slides{02}{13}

\unimptnt{
\subsection{Steuerwerk als Automat}
\subsubsection*{Finite State Machine (FSM) = endlicher Automat}
\slides{02}{14}
\subsubsection*{Logikrealisierung}
\slides{02}{15}
\slides{02}{16}

\subsection*{Beispiele}
\begin{itemize}
\item „sequentieller Addierer“ Datenpfad\\
Zustände laut Algorithmus:
\begin{enumerate}
\item „Laden von Summand A“
\item „Laden von Summand B“
\item „Übernahme des Additionsergebnisses“
\item „Weiterschalten der Summanden um eine Stelle“
\item „Ausgabe des Ergebnisses“
\end{enumerate}
\end{itemize}
}

\subsection{Steuerwerk mit Mikroprogrammierung}

\subsubsection{Realisierung}
\slides{02}{17}
\unimptnt{
\subsubsection*{Aufbau eines einfachen Mikroprogramm-Steuerwerk}
\slides{02}{18}
}
\subsubsection{Mikroprogrammierung: Vor- und Nachteile}
\slides{02}{19}
\subsubsection{Unterschied mikroprogrammiert/mikroprogrammierbar}
\slides{02}{20}

\section{Adresswerk}
(Address Unit, Address Generation Logic)
\slides{02}{21}

\unimptnt{
\subsection*{Aufbau eines einfachen Adresswerks}
\slides{02}{22}
\slides{02}{23}
}