\chapter{Einführung}
\section*{Inhalte}
\begin{itemize}
\item Wiederholung objectiF\\
Persektive: Entwurf $\to$ technische Sicht\\
Klassendiagramm $\to$ statische Struktur
\item Darstellung des \emph{Verhaltens} (Dynamik, Progress)\\
$\to$ Sequenzdiagramm
\item Konkretes Beispiel: MVC
\end{itemize}
\section*{Notizen}
\begin{itemize}
\item \lstinline$Controller$ und \lstinline$View$ hängen immer zusammen, da die Art, wie ich „kontrolliere“ abhängig ist von der Art, wie mir zu Kontrollierendes angezeigt wird.
\item Das Klassendiagramm(KD) zeigt mögliche Botschaftswege.\\
Das Sequenzdiagramm(SD) zeigt die zeitliche Reihenfolge konkreter Botschaften (d.h. Methodenaufrufe).
\item Modellelemente:
\begin{itemize}
\item Instanz (Objekt) als Lebenslinie\\
$\to$ rot: zugeordnete Klasse), schwarz: ohne zugeordnete Klasse\\
$\to$ verdickt: wenn Instanz an Botschaft beteiligt ist
\item Botschaft
\item Systemgrenze
\end{itemize}
\end{itemize}
