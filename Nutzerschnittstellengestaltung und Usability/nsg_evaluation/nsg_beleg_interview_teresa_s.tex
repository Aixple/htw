\subsection*{Interview-Fragen}
\subsubsection*{Zur Person}
\begin{enumerate}
\item Was ist dein Name?
\interviewText{Teresa Schönherr}
\item Wie alt bist du?
\interviewText{21}
\item Was arbeitest/studierst du?
\interviewText{Azubi Kinderkrankenschwester}
\item Wie oft bzw. wie lange browst du Webseiten im Internet?
\interviewText{2-3 Stunden täglich}
\item Auf welchen Seiten bist du regelmäßig unterwegs?
\interviewText{Pinterest, Chefkoch, Amazon, IKEA, Instagram}
\item Wie schätzt Du Deine Kenntnisse im Umgang mit Webseiten ein?
\interviewText{gut}
\item Wofür nutzen Du das Internet (außer zum Browsen)?
\interviewText{Chatten, Mails}
\item War Dir diese Webseite im vornherein bekannt? Wenn ja: Woher?
\interviewText{Nein.}
\end{enumerate}

\subsubsection*{Allgemeines}
\begin{enumerate}
\item Wie leicht fiel es Dir dich auf der Seite zu orientieren?
%               -        ~        +  von bis
\interviewScala{ }{ }{ }{ }{ }{6}{ }{schwer}{leicht}
Was viel dir besonders schwer/leicht:
\interviewText{Suche war einfach und funktionierte gut. Website ist übersichtlich. Schlicht finde ich gut.}
\item Gab es Probleme beim Bedienen der Website?
%               -        ~        +  von bis
\interviewScala{ }{ }{ }{ }{ }{6}{ }{viele}{wenige}
Anmerkungen:
\interviewText{Keine Anmerkungen.}
\item Konntest Du die Website auch gut auf dem Handy bedienen?
%               -        ~        +  von bis
\interviewScala{ }{ }{ }{ }{ }{6}{ }{schwer}{leicht}
Anmerkungen:
\interviewText{Ja. Da ich die Website vorher am PC genutzt habe, fiel das Bedienen am Handy leicht.}
\item Würdest Du etwas an der Webseite verändern?
%               -        ~        +  von bis
\interviewScala{ }{ }{ }{ }{5}{ }{ }{viel}{wenig}
Wenn ja, was?
\interviewText{Farben sind nicht schön, zu blass.}
\item Wie würdest du die Seite beschreiben? Was ist der Zweck der Seite?
\interviewText{Wissen vermitteln, Wichteln, Mitschriften ansehen}
\item Kennst Du Webseiten, die mit dieser vergleichbar sind? Wenn ja: Wie lauten diese? In wie fern sind sie besser/schlechter/ähnlich gestaltet?
\interviewText{Blogs, die sind aber meist durch Bilder lebendiger gestaltet. }
\end{enumerate}

\subsubsection*{Zum Design}
\begin{enumerate}
\item Wie bewertest du das Design der Seite?
\begin{enumerate}
\item Der Gesamteindruck war…
%               -        ~        +  von bis
\interviewScala{ }{ }{ }{4}{ }{ }{ }{schlecht}{gut}
Anmerkungen:
\interviewText{Sieht unfertig aus.}
\item Die Farbwahl und -gestaltung war…
%               -        ~        +  von bis
\interviewScala{ }{2}{ }{ }{ }{ }{ }{schlecht}{gut}
Anmerkungen:
\interviewText{Zu blass. Eintönig.}
\item Die Anordnung der Inhalte war…
%               -        ~        +  von bis
\interviewScala{ }{ }{ }{ }{ }{6}{ }{schlecht}{gut}
Anmerkungen:
\interviewText{Keine Anmerkungen.}
\item Die Präsentation der Webseite war im Bezug auf den Inhalt…
%               -        ~        +  von bis
\interviewScala{ }{ }{ }{4}{ }{ }{ }{unangemessen}{angemessen}
Anmerkungen:
\interviewText{Keine Anmerkungen.}
\item Die Lesbarkeit der Webseite war…
%               -        ~        +  von bis
\interviewScala{ }{ }{ }{ }{ }{6}{ }{schlecht}{gut}
Anmerkungen:
\interviewText{Keine Anmerkungen.}
\end{enumerate}
\end{enumerate}

\subsubsection*{Automatisches Wichtel System}
\begin{enumerate}
\item Sind Dir bei der Bedienung des Automatisierten Wichtel Systems Besonderheiten aufgefallen, die Du noch nicht erwähnt hast?
\interviewText{Auf dem Handy grenzen die Textboxen den Inhalt ein. Um den Text zu lesen muss darin gescrollt werden.}
\end{enumerate}

\subsubsection*{Abschließende Fragen}
\begin{enumerate}
\item Würdest du die Website weiter empfehlen? Warum? Warum nicht?
\interviewText{Wüsste nicht an wen. Das Wichtelsystem ist cool. Aber mit echten losen attraktiver.}
\item Würdest du die Funktionen der Seite erneut nutzen?
\interviewText{Vielleicht das Wichtelsystem, ja. }
\end{enumerate}
