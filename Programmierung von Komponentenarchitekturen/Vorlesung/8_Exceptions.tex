Ziel: Behandlung von Ausnahmefehlern, Vermeidung von Programmabsturz

\section{Behandlung einer Ausnahme mit try / catch}
Beispiel:
\begin{lstlisting}[language={[Sharp]C}]
using System.IO;

class Program {
	static void Main(string[] args) {
		StreamReader stream = new StreamReader(@"C:\Text.txt");
		Console.WriteLine(stream.ReadToEnd());
		stream.Close();
	}
}
\end{lstlisting}

\section{Arten von Exceptions}
Auszug der Vererbungshierarchie:

\begin{center}
\includegraphics[width=.9\textwidth]{Vorlesung/a8Exceptions-img001.png}
\end{center}
Mehrere Exceptions-Typen können abgefangen werden um gesondert auf sie zu reagieren

\begin{lstlisting}[language={[Sharp]C}]
StreamReader stream = null;
Console.Write("Welche Datei soll geöffnet werden? ... ");
string path = Console.ReadLine();
try {
	stream = new StreamReader(path);
	Console.WriteLine(stream.ReadToEnd());
	stream.Close();
} catch (FileNotFoundException ex) {
	Console.WriteLine(ex.Message);
} catch (DirectoryNotFoundException ex) {
	Console.WriteLine(ex.Message);
}	catch (ArgumentException ex) { //Pfadangabe leer
	Console.WriteLine(ex.Message);
} catch (Exception ex) {
	Console.WriteLine(ex.Message);
}
// ...
\end{lstlisting}

\section{Ausnahmen werfen mit throw}
Zum Test von Code kann man auch selbst ausnahmen erzeugen (werfen)
\lstinline$throw AbcException;$

\section{Finally Block}
Falls nach Behandlung einer Exception etwas unbedingt ausgeführt werden muss (z.B. Freigabe einer Datei, Schließen einer Datenbankverbindung,…)
\begin{lstlisting}[language={[Sharp]C}]
try {
	// ...
} catch {
	// ...
} finally {
	// ...
}
\end{lstlisting}
Der \lstinline$finally$-Block wird immer ausgeführt unabhängig ob:
\begin{itemize}
\item Exception geworfen
\item catch block ein return statement enthielt
\item in catch noch eine Exception geworfen wurde 
\end{itemize}

\section{Eigenschaften der Klasse Exception}

\begin{tabular}{L{.25} L{.75}}
Eigenschaft  & Beschreibung \\\hline
\lstinline$Data$ & Stellt zusätzliche Informationen zu der Ausnahme bereit.\\
\lstinline$HelpLink$ & Verweist auf eine Hilfedatei, die diese Ausnahme beschreibt.\\
\lstinline$InnerException$ & Falls bei der Behandlung einer Ausnahme eine weitere Exception ausgelöst wird, beschreibt diese Eigenschaft die neue (innere) Ausnahme. \\
\lstinline$Message$ & Liefert eine Zeichenfolge mit der Beschreibung des aktuellen Fehlers. Die Information sollte so formuliert sein, dass sie auch von einem Anwender verstanden werden kann. \\
\lstinline$Source$ & Liefert einen String zurück, der die Anwendung angibt, in der die Ausnahme ausgelöst worden ist. \\
\lstinline$StackTrace$ & Beschreibt in einer Zeichenfolge die aktuelle Aufrufreihenfolge aller Methoden.\\
\lstinline$TargetSite$ & Liefert zahlreiche Informationen zu der Methode, in der die Ausnahme ausgelöst worden ist. \\
\end{tabular}\\
Am nützlichsten sind \lstinline$Message$ und \lstinline$StackTrace$, sie liefern beide einen String.

\section{Innere Exceptions}
Tritt innerhalb einer Ausnahmebehandlung (catch) eine Exception auf so sollte auch diese Behandelt werden (innere Exception)\bigskip\\
Beispiel: Was passiert hier?  Was passiert wenn der (int) cast in calculateSomething weggelassen wird?
\begin{lstlisting}[language={[Sharp]C}]
public static void calculateSomething() {
	double x = 1 / 2 * 2;
	int y = 1 / (int) x ;
}

public static void Main(string[] args) {
	try{
		calculateSomething();
	} catch (Exception ex) {
		Console.WriteLine(ex.Message);
		try{
			StreamWriter file = new StreamWriter("error\_output.txt");
			file.WriteLine("Fehler beim Aufruf von calculateSomething");
			file.Close();  
		} catch {
			Console.WriteLine("Fehler-Output konnte nicht geschrieben werden.");
		}
	}
}
\end{lstlisting}
