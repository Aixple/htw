\section{Organisation des DNS-Namensraumes}
\slides{it1-16-dns_print}{3}
\subsubsection*{Verteilte hierarchische Datenbank}
\slides{it1-16-dns_print}{4}

\section{Nameserver}
\subsection{Full-Qualified-Domain-Name (FQDN)}
\slides{it1-16-dns_print}{5}
\subsection{Komponenten des DNS}
\slides{it1-16-dns_print}{6}
\subsection{Root-Nameserver}
\slides{it1-16-dns_print}{7}
\subsection{DNS: Root-Nameserver}
\slides{it1-16-dns_print}{8}
\subsubsection*{Synchronisation}
\slides{it1-16-dns_print}{9}
\subsection{Anycast}
\slides{it1-16-dns_print}{10}
\subsection{TLD- und autorative Server}
\slides{it1-16-dns_print}{11}
\subsection{Lokale Nameserver}
\slides{it1-16-dns_print}{12}
\subsection{DNS-Caching}
\slides{it1-16-dns_print}{13}
\subsection{Resolver}
\slides{it1-16-dns_print}{14}

\section{Namensauflösung}
\slides{it1-16-dns_print}{15}
\subsection{Rekursive Namensauflösung}
\slides{it1-16-dns_print}{16}
\subsection{Programme zur Namensauflösung}
\slides{it1-16-dns_print}{17}
\subsection{Beispiel Namensauflösung}
\slides{it1-16-dns_print}{18}
\subsubsection{Autorativ}
\slides{it1-16-dns_print}{19}
\subsubsection{Cache}
\slides{it1-16-dns_print}{20}
\subsection{Reverse DNS}
\slides{it1-16-dns_print}{21}
\subsubsection*{Test}
\slides{it1-16-dns_print}{22}
\subsection{Beispiel Namensauflösung}
\slides{it1-16-dns_print}{23}
\subsection{Beispiel inverse Namensauflösung}
\slides{it1-16-dns_print}{24}

\section{Ressource Records}
\subsection{DNS-Ressource-Records}
\slides{it1-16-dns_print}{25}
\subsection{Ausgewählte Ressource Records}
\slides{it1-16-dns_print}{26}
\subsubsection*{Beispiel}
\slides{it1-16-dns_print}{27}
\subsection{Erreichbarkeit einer Domain}
\slides{it1-16-dns_print}{28}
\subsection{Neue Einträge hinzufügen}
\slides{it1-16-dns_print}{29}

\section{DNS Nachrichten}
\slides{it1-16-dns_print}{30}
\subsection{Zone}
\slides{it1-16-dns_print}{31}
\subsection{BIND}
\slides{it1-16-dns_print}{32}
\subsubsection*{Beispiel}
\slides{it1-16-dns_print}{33}
\subsection{Zonendatei}
\slides{it1-16-dns_print}{34}
\subsubsection*{Beispiel}
\slides{it1-16-dns_print}{35}

\section{Dynamisches DNS}
\slides{it1-16-dns_print}{36}

\section{DNS zur Spamabwehr}
\slides{it1-16-dns_print}{37}
\subsection{Lastenverteilung}
\slides{it1-16-dns_print}{38}

\section{Content-Distribution-Networks (CDN)}
\slides{it1-16-dns_print}{39}
\subsection{Beispiel}
\slides{it1-16-dns_print}{40}
\subsection{Details}
\slides{it1-16-dns_print}{41}

\section{ENUM}
\slides{it1-16-dns_print}{42}
\slides{it1-16-dns_print}{43}
\slides{it1-16-dns_print}{44}

\section{DNSSEC}
\subsection{DNS und Internetprovider}
\slides{it1-16-dns_print}{45}
\subsection{Sicherheitsprobleme}
\slides{it1-16-dns_print}{46}
\subsection{DNSSEC}
\slides{it1-16-dns_print}{47}
\subsection{RR}
\slides{it1-16-dns_print}{48}
\subsection{Verfahren}
\slides{it1-16-dns_print}{49}
\subsection{Vertrauenskette}
\slides{it1-16-dns_print}{50}
\subsection{Anwendungen}
\slides{it1-16-dns_print}{51}
\subsection{DNS-based Authentication of Named Entities (DANE)}
\slides{it1-16-dns_print}{52}
\subsection{DNS SEC -- mit DIG}
\slides{it1-16-dns_print}{53}
\subsection{Beispiel an TLD DE}
\slides{it1-16-dns_print}{54}

\section{DNS zur Schlüsselübergabe}
\subsection{RR-Typen für Zertifikate}
\slides{it1-16-dns_print}{55}
\subsection{Service-Ressource-Record (SRV)}
\slides{it1-16-dns_print}{56}
\subsection{PGP und DNS}
\slides{it1-16-dns_print}{57}
\subsection{Beispiel PKA}
\slides{it1-16-dns_print}{58}

\section{Zusammenfassung}
\slides{it1-16-dns_print}{59}
\section{Fragen}
\slides{it1-16-dns_print}{60}










