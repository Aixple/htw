\chapter{Einführung MVC}
\section*{Inhalte}
\begin{itemize}
\item Wiederholung objectiF\\
Persektive: Entwurf $\to$ technische Sicht\\
Klassendiagramm $\to$ statische Struktur
\item Darstellung des \emph{Verhaltens} (Dynamik, Progress)\\
$\to$ Sequenzdiagramm
\item Konkretes Beispiel: MVC
\end{itemize}
\section*{Notizen}
\begin{itemize}
\item \lstinline$Controller$ und \lstinline$View$ hängen immer zusammen, da die Art, wie ich „kontrolliere“ abhängig ist von der Art, wie mir zu Kontrollierendes angezeigt wird.
\item Das Klassendiagramm(KD) zeigt mögliche Botschaftswege.\\
Das Sequenzdiagramm(SD) zeigt die zeitliche Reihenfolge konkreter Botschaften (d.h. Methodenaufrufe).
\item Modellelemente:
\begin{itemize}
\item Instanz (Objekt) als Lebenslinie\\
$\to$ rot: zugeordnete Klasse), schwarz: ohne zugeordnete Klasse\\
$\to$ verdickt: wenn Instanz an Botschaft beteiligt ist
\item Botschaft
\item Systemgrenze
\end{itemize}
\end{itemize}

\chapter{Notenverwaltung}

\begin{enumerate}
\item Methodik:\\
\begin{tabular}{L{.3} L{.3} L{0.3}}
Modell\newline (objectiF) & $\overset{\text{Forward Engineering}}{\longrightarrow}$ \newline $\underset{\text{Reverse Engineering}}{\longleftarrow}$ & Implementation\newline (MS VisualStudio .NET)
\end{tabular}\\
$\Rightarrow$ Roundtrip-Verfahren
\item Architektur:\\
3-Schichten-Architektur
\begin{itemize}
\item Präsentation
\item Logik
\item Datenhaltung
\end{itemize}
Im objectiF: Schichten sind Pakete $\to$ Paketdiagramm\\
Im VisualStudio: Schichten sind Projekte $\to$ Projektmappe\\
Vereinbarung: Verbindung der Schichten zwischen objectiF und VisualStudio außer bei der Präsentation
\end{enumerate}

\lecdate{13.04.2017}
\begin{itemize}
\item Dialog bauen $\to$ MS Visual Studie .NET
\item KD: DBService $\to$ objectiF
\item Paket Typen (Student, Fachnote) $\to$ objectiF
\item Synchronisation:\\
Modell -- Implementation
objectiF $\to$ MS Visual Studio
\end{itemize}












