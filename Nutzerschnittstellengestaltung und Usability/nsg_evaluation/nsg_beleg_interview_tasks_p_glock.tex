\subsection*{Begrüßung}
[...]

Wenn Du dem zustimmst, unterschreibe bitte hier:

\vspace*{1em}
P. Glock [16.01.2018]\vspace*{-.9em}\\
\noindent\rule{8cm}{0.4pt}

\subsection*{Einführung}
[...]
\subsection*{Test-Aufgaben}
\subsubsection*{Alle Seiten besuchen}
\begin{enumerate}
\item Wie lautet die Email-Adresse des Admins der Website? Wie ist der Name des Seiteninhabers?
\interviewText{
Findet gleich den Namen, geht dann aber auch auf den Eintrag 'Über' und findet alle restlichen Informationen.
$"$'Über uns' würde ich besser finden$"$. Seiteninhaber ist klar.}

\item Finde eine Seite, die den Text „schwarz-lila“ enthält. Wie ist der Titel dieser Seite?
\interviewText{Klickt auf den Schriftzug neben dem Logo, dann auf Kreativ. Kreativ wegen der Suche nach Farben. Danach geht er zu Home und Code. Auf Englisch will er nicht umstellen weil der Suchbegriff Deutsch ist. $"$Counter-Strike: Soure ist ein Schreibfehler$"$. Durchsucht die aktuelle Seite mit STRG+F. Klickt jeden Link durch und wirkt langsam genervt. Schreibt sein eigenes 'lila-schwarz' in ein Wichtel Feld. Hat kein Suchfeld benutzt!}

\item Für welche Spiele werden auf der Seite Karten/Maps/Level angeboten?
\interviewText{
klickt auf Code, dann Kreativ und öffnet anschließend alle Einträge in Kreativ in verschiedenen Tabs. Findet den Kontent für Counter-Strike und Half Life}

\item Finde den Sammelband mit drei Geschichten. Was assoziierst du mit dem Titelbild?
\interviewText{Klickt sich zielstrebig zu dem Sammelband, da er schon alles gesehen hat und weis wo es steht (durch Aufgabe 2). Bei Kreativ findet er gleich den Sammelband 'Ohne Namen'$"$. Es sieht offensichtlich der Silhouette eines weiblichen Körpers sehr ähnlich. Könnte auch ein Flusslauf oder ein Bild über Strömungslehre sein$"$.}

\item Was für Software wird auf der Webseite vorgestellt?
\interviewText{Geht zu Home, dann zu Code und findet 2 Ausleihbibliotheken und ein Automatisches Wichtel System. Die Mitschriften sieht er eher als Archiv.}

\item Unter welchem Betriebssystem ist die Ausleihbibliothek in C++ ausführbar?
\interviewText{Klickt auf den Programm Link und liest das es wohl für Windows ausführbar ist, vermutet aber auch das es noch für andere Systeme funktionieren könnte}

\item Welche Medien sind in der Ausleihbibliothek in C ausleihbar?
\interviewText{Klickt auf den Programm Link und liest Bücher, CDs und DVDs. $"$'verbort' klingt nicht gut als Begriff (wahrscheinlich Umgangssprache)$"$.}

\item Wo sind Studiendokumente zu finden?
\interviewText{Geht zu Code und dann zu Studienmitschriften$"$. Die Mitschriften finde ich auf der Website der HTW-Dresden mit einem entsprechenden Link$"$.}

\item Stelle die Sprache auf Englisch um. Findest Du eine Seite, die nicht auf Englisch übersetzt ist? Wenn ja, wie ist der Titel dieser Seite?
\interviewText{Klickt gleich auf die Option 'Englisch' in der oberen Ecke$"$. Muss ich jetzt alle Seiten besuchen$"$? Klickt über die Geschichten, Mods, zu Code. Besucht keine externen Links. Findet schließlich die nicht übersetzte Seite des Wichtelsystems.}
\end{enumerate}


\subsubsection*{Automatisiertes Wichtel System}
\begin{enumerate}
\item Benutze das Automatisierte Wichtel System: [..]
\interviewText{Wichtel-Schritte:
\begin{enumerate}
	\item Findet gleich die Namensfelder und fügt weitere Wichtel hinzu. Trägt die Namen und E-Mail Adressen ohne Probleme ein. Geht weiter mit der Anmerkung noch nichts zugewiesen zu haben.
	\item $"$Ganz klar ist mir das System nicht$"$.	Klickt etwas herum in Hoffnung das es klappt und ist schließlich zufrieden. 
	\item Gibt den Namen und die Mail Adresse ein. Wundert sich über den Eintrag '[Name des Wichtels] nach dem ersten Feld in der Mail$"$. Soll beim Hallo Feld der Name des Wichtels geschrieben werden? Ganz klar erscheint mir es das nicht$"$. Wundert weiter über den Eintrag '[Link zur Mail an dich]'. Klickt in ein Text Feld. Versteht die Anweisung nicht, fragt sich welcher Text nun Text ist und welcher nur ein Hinweis ist. Text sollte seiner Meinung nach Grau hinterlegt sein. Vor dem Eintrag '[Link zur Mail an alle]' muss man seiner Meinung nach einen MailTo-Group Link eintragen. Trägt einen fiktiven Link dort ein. Versteht den Sinn und Zweck des Zufall-Seed nicht$"$.Wird schon passen$"$. Geht weiter.
	\item Gibt den Zufalls-Seed ein. $"$Was auch immer ein Seed ist$"$. Es kommt eine Grafik mit Prozenten, die nicht verstanden wird. Er sieht keinen Sinn für die Angabe der Prozente, sie würden eher verwirren. Es wird vermutet das der MailTo-Group Link von ihm selbst angelegt wurde. Anscheinend alles okay gelaufen
\end{enumerate}
Notiz des Probanden:
Das Wichtelsystem ist nicht sehr sensibel. Kevin und Jacqueline erfahren das sie für einander nicht wichteln dürfen! Findet er nicht gut.}
\end{enumerate}

\subsubsection*{Responsive Design}
\begin{enumerate}
\item Rufe die Website auf deinem Handy auf: […]
\interviewText{\begin{itemize}
		\item Geht auf Verzeichnis Code und klickt auf den Programm Link der Ausleihbibliothek. Scrollt etwas, aber klickt nicht weiter auf das eigentliche Programm. Weis nicht was er weiter tun soll.
		\item Es fällt auf das der Button 'Neue Wichtel erstellen und verteilen' ist nicht responsive und ragt über den Bildschirm hinaus.
		\item Funktioniert für einen Wichtel wenn man Zeilen mit den zusätzlichen Wichteln löscht.
		\item nervt das Seite immer wieder nach oben springt und man wieder runter scrollen muss
		\item Nach Eingabe des Keys beim wichteln mit einer Person kommt eine lange Fehlermeldung
	\end{itemize}
\paragraph{Notiz}Proband findet es einfacher wenn man ein paar Zettel erstellt und zieht.}
\end{enumerate}