\section{Grundlagen}
\slides{it2-60-rs-codes_print}{3}
\subsection{Unterschied zu Bose-Chaudhuri-Hocquenghem-Codes(BCH)-Codes}
\slides{it2-60-rs-codes_print}{4}
\subsection{Parameter}
\slides{it2-60-rs-codes_print}{5}
\subsection{Verkürzte Codes}
\slides{it2-60-rs-codes_print}{6}
\section{Definition}
\subsection{Code}
\slides{it2-60-rs-codes_print}{7}
\subsection{Generatorpolynom}
\slides{it2-60-rs-codes_print}{8}
\subsubsection*{Beispiel}
\slides{it2-60-rs-codes_print}{9}
\subsection{Beispiel: Erzeugung eines \texorpdfstring{$(7,5,3)_8$}{(7,5,3)\_8}-RS-Codes}
\slides{it2-60-rs-codes_print}{10}
Divisionsberechnung:
\slides{it2-60-rs-codes_print}{11}
Codewort:
\slides{it2-60-rs-codes_print}{12}

\subsection{Codierung mittels Schieberegisterschaltung}
\slides{it2-60-rs-codes_print}{13}

\section{Prinzip Decodierung}
\slides{it2-60-rs-codes_print}{14}
\subsection{Syndrom}
\slides{it2-60-rs-codes_print}{15}
\subsubsection{Berechnung}
\slides{it2-60-rs-codes_print}{16}
\subsubsection{Fehlermuster}
\slides{it2-60-rs-codes_print}{17}
\subsubsection{Gleichungssystem}
\slides{it2-60-rs-codes_print}{18}

\subsection{Fehlerstellenpolynom}
\slides{it2-60-rs-codes_print}{19}

\subsection{Chin-Suche}
\slides{it2-60-rs-codes_print}{20}
\subsubsection*{Beispiel}
\slides{it2-60-rs-codes_print}{21}
\subsubsection*{Finden des Fehlerstellenpolynoms}
\slides{it2-60-rs-codes_print}{22}

\subsection{Newton-Identität}
\slides{it2-60-rs-codes_print}{23}

\subsection{Peterson-Gorenstein-Zierler-Decoder}
\slides{it2-60-rs-codes_print}{24}

\subsection{Ermittlung der Fehlerwerte}
\slides{it2-60-rs-codes_print}{25}

\subsection{Ausfallstellen-Decodierung}
\slides{it2-60-rs-codes_print}{26}

\subsection{Beispiel Fehlerkorrektur und Ausfallstellenkorrektur}
\subsubsection*{Fehlerposition}
\slides{it2-60-rs-codes_print}{27}
\subsubsection*{Fehlerwerte}
\slides{it2-60-rs-codes_print}{28}
\subsubsection*{Ausfallstellenkorrektur}
\slides{it2-60-rs-codes_print}{29}
\slides{it2-60-rs-codes_print}{30}

\subsection{Ausfallstellenkorrektur für RTP-Pakete}
\slides{it2-60-rs-codes_print}{31}

\subsection{Matrix-Decodierung}
\slides{it2-60-rs-codes_print}{32}
\slides{it2-60-rs-codes_print}{33}

\section{Beispiele}
\subsection{Data Matrix Code ECC200}
\slides{it2-60-rs-codes_print}{35}
\slides{it2-60-rs-codes_print}{36}
\subsection{QR-Code}
\slides{it2-60-rs-codes_print}{37}
\subsubsection*{Fehlerkorrektur}
\slides{it2-60-rs-codes_print}{38}
\subsubsection*{Code-Generator}
\slides{it2-60-rs-codes_print}{39}
\subsubsection*{Beispiel}
\slides{it2-60-rs-codes_print}{40}

\subsection{Checksummenformate PAR1/PAR2}
\slides{it2-60-rs-codes_print}{41}
\subsection{Weitere}
\slides{it2-60-rs-codes_print}{42}

\section{Softwarelösungen}
\subsection{GNU Octave}
\slides{it2-60-rs-codes_print}{43}
\subsection{Weitere}
\slides{it2-60-rs-codes_print}{44}

\section{Zusammenfassung}
\slides{it2-60-rs-codes_print}{45}


















