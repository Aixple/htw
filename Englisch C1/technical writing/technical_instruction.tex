\newcommand{\customDir}{}
\RequirePackage{ifthen,xifthen}

% Input inkl. Umlaute, Silbentrennung
\RequirePackage[T1]{fontenc}
\RequirePackage[utf8]{inputenc}

% Arbeitsordner (in Abhängigkeit vom Master) Standard: .LateX_master Ordner liegt im Eltern-Ordner
\providecommand{\customDir}{../}
\newcommand{\setCustomDir}[1]{\renewcommand{\customDir}{#1}}
%%% alle Optionen:
% Doppelseitig (mit Rand an der Innenseite)
\newboolean{twosided}
\setboolean{twosided}{false}
% Eigene Dokument-Klasse (alle KOMA möglich; cheatsheet für Spicker [3 Spalten pro Seite, alles kleiner])
\newcommand{\customDocumentClass}{scrreprt}
\newcommand{\setCustomDocumentClass}[1]{\renewcommand{\customDocumentClass}{#1}}
% Unterscheidung verschiedener Designs: htw, fjs
\newcommand{\customDesign}{htw}
\newcommand{\setCustomDesign}[1]{\renewcommand{\customDesign}{#1}}
% Dokumenten Metadaten
\newcommand{\customTitle}{}
\newcommand{\setCustomTitle}[1]{\renewcommand{\customTitle}{#1}}
\newcommand{\customSubtitle}{}
\newcommand{\setCustomSubtitle}[1]{\renewcommand{\customSubtitle}{#1}}
\newcommand{\customAuthor}{}
\newcommand{\setCustomAuthor}[1]{\renewcommand{\customAuthor}{#1}}
%	Notiz auf der Titelseite (A: vor Autor, B: nach Autor)
\newcommand{\customNoteA}{}
\newcommand{\setCustomNoteA}[1]{\renewcommand{\customNoteA}{#1}}
\newcommand{\customNoteB}{}
\newcommand{\setCustomNoteB}[1]{\renewcommand{\customNoteB}{#1}}
% Format der Signatur in Fußzeile:
\newcommand{\customSignature}{\ifthenelse{\equal{\customAuthor}{}} {} {\footnotesize{\textcolor{darkgray}{Mitschrift von\\ \customAuthor}}}}
\newcommand{\setCustomSignature}[1]{\renewcommand{\customSignature}{#1}}
% Format des Autors auf dem Titelblatt:
\newcommand{\customTitleAuthor}{\textcolor{darkgray}{Mitschrift von \customAuthor}}
\newcommand{\setCustomTitleAuthor}[1]{\renewcommand{\customTitleAuthor}{#1}}
% Standard Sprache
\newcommand{\customDefaultLanguage}[1]{}
\newcommand{\setCustomDefaultLanguage}[1]{\renewcommand{\customDefaultLanguage}{#1}}
% Folien-Pfad (inkl. Dateiname ohne Endung und ggf. ohne Nummerierung)
\newcommand{\customSlidePath}{}
\newcommand{\setCustomSlidePath}[1]{\renewcommand{\customSlidePath}{#1}}
% Folien Eigenschaften
\newcommand{\customSlideScale}{0.5}
\newcommand{\setCustomSlideScale}[1]{\renewcommand{\customSlideScale}{#1}}
\newcommand{\customSlideHeight}{9.63cm}
\newcommand{\setCustomSlideHeight}[1]{\renewcommand{\customSlideHeight}{#1}}
\newcommand{\customSlideWidth}{12.8cm}
\newcommand{\setCustomSlideWidth}[1]{\renewcommand{\customSlideWidth}{#1}}

%\setboolean{twosided}{true}
%\setCustomDocumentClass{scrartcl}
%\setCustomDesign{htw}
%\setCustomSlidePath{Folien}
\newcommand{\chapterHead}{1}

\setCustomTitle{\textsc{Technical Instruction}}
\setCustomSubtitle{Pokémon reality gear and game}
\setCustomAuthor{Falk-Jonatan Strube}
%\setCustomNoteA{TitlepageNoteBeforeAuthor}
%\setCustomNoteB{Vorlesung von}

\setCustomSignature{}	% Formatierung der Signatur in der Fußzeile
\setCustomTitleAuthor{\customAuthor\\and Anxhela Merko}	% Formatierung des Autors auf dem Titelblatt

%-- Prüfen, ob Beamer
\ifthenelse{\equal{\customDocumentClass}{beamer}}{
%%% TODO: andere Layouts für Beamer außer HTW
	\documentclass[ignorenonframetext, 11pt, table]{beamer}
	
	\usenavigationsymbolstemplate{}
	\setbeamercolor{author in head/foot}{fg=black}
	\setbeamercolor{title}{fg=black}
	\setbeamercolor{bibliography entry author}{fg=htworange!70}
	%\setbeamercolor{bibliography entry title}{fg=blue} 
	\setbeamercolor{bibliography entry location}{fg=htworange!60} 
	\setbeamercolor{bibliography entry note}{fg=htworange!60}  
	
	\setbeamertemplate{itemize item}{\color{black}$\bullet$}
	\setbeamertemplate{itemize subitem}{\color{black}--}
	\setbeamertemplate{itemize subsubitem}{\color{black}$\bullet$}
	\makeatother
	\setbeamertemplate{footline}
	{
	\leavevmode
	\def\arraystretch{1.2}
	\arrayrulecolor{gray}
	\begin{tabular}{ p{0.167\textwidth} | p{0.491\textwidth} | p{0.089\textwidth} | p{0.103\textwidth}}
	\hline
	\strut\insertshortauthor & \insertshorttitle & Slide \insertframenumber{}% / \inserttotalframenumber{}
	 & May 4, 2016\\
	\end{tabular}
	}
	\setbeamertemplate{headline}
	{
	\leavevmode
	\setlength{\arrayrulewidth}{1pt}
	\hspace*{2em}	
	\begin{tabular}{p{0.63\textwidth}}
	\rule{0pt}{3em}\normalsize{\textbf{\insertsection\strut}}\\
	\arrayrulecolor{htworange}
	\hline
	\end{tabular}
	\begin{tabular}{l}
	\rule{0pt}{4em}\includegraphics[width=3.25cm]{\customDir .LaTeX_master/HTW_GESAMTLOGO_CMYK.eps}\\
	\end{tabular}
	}
	\makeatletter	
}{	
	%-- Für Spicker einiges anders:
	\ifthenelse{\equal{\customDocumentClass}{cheatsheet}}{
		\documentclass[a4paper,10pt,landscape]{scrartcl}
		\usepackage{geometry}
		\geometry{top=2mm, bottom=2mm, headsep=0mm, footskip=0mm, left=2mm, right=2mm}
		
		% Für Spicker \spsection für Section, zur Strukturierung \HRule oder \HDRule Linie einsetzen
		\usepackage{multicol}
		\newcommand{\spsection}[1]{\textbf{#1}}	% Platzsparende "section" für Spicker
	}{	%-- Ende Spicker-Unterscheidung-if
		%-- Unterscheidung Doppelseitig
		\ifthenelse{\boolean{twosided}}{
			\documentclass[a4paper,11pt, footheight=26pt,twoside]{\customDocumentClass}
			\usepackage[head=23pt]{geometry}	% head=23pt umgeht Fehlerwarnung, dafür größeres "top" in geometry
			\geometry{top=30mm, bottom=22mm, headsep=10mm, footskip=12mm, inner=27mm, outer=13mm}
		}{
			\documentclass[a4paper,11pt, footheight=26pt]{\customDocumentClass}
			\usepackage[head=23pt]{geometry}	% head=23pt umgeht Fehlerwarnung, dafür größeres "top" in geometry
			\geometry{top=30mm, bottom=22mm, headsep=10mm, footskip=12mm, left=20mm, right=20mm}
		}
		%-- Nummerierung bis Subsubsection für Report
		\ifthenelse{\equal{\customDocumentClass}{report} \OR \equal{\customDocumentClass}{scrreprt}}{
			\setcounter{secnumdepth}{3}	% zählt auch subsubsection
			\setcounter{tocdepth}{3}	% Inhaltsverzeichnis bis in subsubsection
		}{}
	}%-- Ende Spicker-Unterscheidung-else
	
	\usepackage{scrlayer-scrpage}	% Kopf-/Fußzeile
	\renewcommand*{\thefootnote}{\fnsymbol{footnote}}	% Fußnoten-Symbole anstatt Zahlen
	\renewcommand*{\titlepagestyle}{empty} % Keine Seitennummer auf Titelseite
	\usepackage[perpage]{footmisc}	% Fußnotenzählung Seitenweit, nicht Dokumentenweit
}

% Input inkl. Umlaute, Silbentrennung
\RequirePackage[T1]{fontenc}
\RequirePackage[utf8]{inputenc}
\usepackage[english,ngerman]{babel}
\usepackage{csquotes}	% Anführungszeichen
\RequirePackage{marvosym}
\usepackage{eurosym}

% Style-Aufhübschung
\usepackage{soul, color}	% Kapitälchen, Unterstrichen, Durchgestrichen usw. im Text
%\usepackage{titleref}

% Mathe usw.
\usepackage{amssymb}
\usepackage{amsthm}
\ifthenelse{\equal{\customDocumentClass}{beamer}}{}{
\usepackage[fleqn,intlimits]{amsmath}	% fleqn: align-Umgebung rechtsbündig; intlimits: Integralgrenzen immer ober-/unterhalb
}
%\usepackage{mathtools} % u.a. schönere underbraces
\usepackage{xcolor}
\usepackage{esint}	% Schönere Integrale, \oiint vorhanden
\everymath=\expandafter{\the\everymath\displaystyle}	% Mathe Inhalte werden weniger verkleinert
\usepackage{wasysym}	% mehr Symbole, bspw \lightning
% Auch arcus-Hyperbolicus-Funktionen
\DeclareMathOperator{\arccot}{arccot}
\DeclareMathOperator{\arccosh}{arccosh}
\DeclareMathOperator{\arcsinh}{arcsinh}
\DeclareMathOperator{\arctanh}{arctanh}
\DeclareMathOperator{\arccoth}{arccoth} 
%\renewcommand{\int}{\int\limits}
%\usepackage{xfrac}	% mehr fracs: sfrac{}{}
\let\oldemptyset\emptyset	% schöneres emptyset
\let\emptyset\varnothing
%\RequirePackage{mathabx}	% mehr Symbole
\mathchardef\mhyphen="2D	% Hyphen in Math

% tikz usw.
\usepackage{tikz}
\usepackage{pgfplots}
\pgfplotsset{compat=1.11}	% Umgeht Fehlermeldung
\usetikzlibrary{graphs}
%\usetikzlibrary{through}	% ???
\usetikzlibrary{arrows}
\usetikzlibrary{arrows.meta}	% Pfeile verändern / vergrößern: \draw[-{>[scale=1.5]}] (-3,5) -> (-3,3);
\usetikzlibrary{automata,positioning} % Zeilenumbruch im Node node[align=center] {Text\\nächste Zeile} automata für Graphen
\usetikzlibrary{matrix}
\usetikzlibrary{patterns}	% Schraffierte Füllung
\usetikzlibrary{shapes.geometric}	% Polygon usw.
\tikzstyle{reverseclip}=[insert path={	% Inverser Clip \clip
	(current page.north east) --
	(current page.south east) --
	(current page.south west) --
	(current page.north west) --
	(current page.north east)}
% Nutzen: 
%\begin{tikzpicture}[remember picture]
%\begin{scope}
%\begin{pgfinterruptboundingbox}
%\draw [clip] DIE FLÄCHE, IN DER OBJEKT NICHT ERSCHEINEN SOLL [reverseclip];
%\end{pgfinterruptboundingbox}
%\draw DAS OBJEKT;
%\end{scope}
%\end{tikzpicture}
]	% Achtung: dafür muss doppelt kompliert werden!
\usepackage{graphpap}	% Grid für Graphen
\tikzset{every state/.style={inner sep=2pt, minimum size=2em}}
\usetikzlibrary{mindmap, backgrounds}
%\usepackage{tikz-uml}	% braucht Dateien: http://perso.ensta-paristech.fr/~kielbasi/tikzuml/

% Tabular
\usepackage{longtable}	% Große Tabellen über mehrere Seiten
\usepackage{multirow}	% Multirow/-column: \multirow{2[Anzahl der Zeilen]}{*[Format]}{Test[Inhalt]} oder \multicolumn{7[Anzahl der Reihen]}{|c|[Format]}{Test2[Inhalt]}
\renewcommand{\arraystretch}{1.3} % Tabellenlinien nicht zu dicht
\usepackage{colortbl}
\arrayrulecolor{gray}	% heller Tabellenlinien
\usepackage{array}	% für folgende 3 Zeilen (für Spalten fester breite mit entsprechender Ausrichtung):
\newcolumntype{L}[1]{>{\raggedright\let\newline\\\arraybackslash\hspace{0pt}}m{\dimexpr#1\columnwidth-2\tabcolsep-1.5\arrayrulewidth}}
\newcolumntype{C}[1]{>{\centering\let\newline\\\arraybackslash\hspace{0pt}}m{\dimexpr#1\columnwidth-2\tabcolsep-1.5\arrayrulewidth}}
\newcolumntype{R}[1]{>{\raggedleft\let\newline\\\arraybackslash\hspace{0pt}}m{\dimexpr#1\columnwidth-2\tabcolsep-1.5\arrayrulewidth}}
\usepackage{caption}	% Um auch unbeschriftete Captions mit \caption* zu machen

% Nützliches
\usepackage{verbatim}	% u.a. zum auskommentieren via \begin{comment} \end{comment}
\usepackage{tabto}	% Tabs: /tab zum nächsten Tab oder /tabto{.5 \CurrentLineWidth} zur Stelle in der Linie
\NumTabs{6}	% Anzahl von Tabs pro Zeile zum springen
\usepackage{listings} % Source-Code mit Tabs
\usepackage{lstautogobble} 
\ifthenelse{\equal{\customDocumentClass}{beamer}}{}{
\usepackage{enumitem}	% Anpassung der enumerates
%\setlist[enumerate,1]{label=(\arabic*)}	% global andere Enum-Items
\renewcommand{\labelitemiii}{$\scriptscriptstyle ^\blacklozenge$} % global andere 3. Item-Aufzählungszeichen
}
\newenvironment{anumerate}{\begin{enumerate}[label=(\alph*)]}{\end{enumerate}} % Alphabetische Aufzählung
\usepackage{letltxmacro} % neue Definiton von Grundbefehlen
% Nutzen:
%\LetLtxMacro{\oldemph}{\emph}
%\renewcommand{\emph}[1]{\oldemph{#1}}
\RequirePackage{xpatch}	% ua. Konkatenieren von Strings/Variablen (etoolbox)


% Einrichtung von lst
\lstset{
basicstyle=\ttfamily, 
%mathescape=true, 
%escapeinside=^^, 
autogobble, 
tabsize=2,
basicstyle=\footnotesize\sffamily\color{black},
frame=single,
rulecolor=\color{lightgray},
numbers=left,
numbersep=5pt,
numberstyle=\tiny\color{gray},
commentstyle=\color{gray},
keywordstyle=\color{green},
stringstyle=\color{orange},
morecomment=[l][\color{magenta}]{\#}
showspaces=false,
showstringspaces=false,
breaklines=true,
literate=%
    {Ö}{{\"O}}1
    {Ä}{{\"A}}1
    {Ü}{{\"U}}1
    {ß}{{\ss}}1
    {ü}{{\"u}}1
    {ä}{{\"a}}1
    {ö}{{\"o}}1
    {~}{{\textasciitilde}}1
}
\usepackage{scrhack} % Fehler umgehen
\def\ContinueLineNumber{\lstset{firstnumber=last}} % vor lstlisting. Zum wechsel zum nicht-kontinuierlichen muss wieder \StartLineAt1 eingegeben werden
\def\StartLineAt#1{\lstset{firstnumber=#1}} % vor lstlisting \StartLineAt30 eingeben, um bei Zeile 30 zu starten
\let\numberLineAt\StartLineAt

% BibTeX
\usepackage[backend=bibtex8, bibencoding=ascii,
%style=authortitle, citestyle=authortitle-ibid,
%doi=false,
%isbn=false,
%url=false
]{biblatex}	% BibTeX
\usepackage{makeidx}
%\makeglossary
%\makeindex

% Grafiken
\usepackage{graphicx}
\usepackage{epstopdf}	% eps-Vektorgrafiken einfügen
%\epstopdfsetup{outdir=\customDir}

% pdf-Setup
\usepackage{pdfpages}
\ifthenelse{\equal{\customDocumentClass}{beamer}}{}{
\usepackage[bookmarks,%
bookmarksopen=false,% Klappt die Bookmarks in Acrobat aus
colorlinks=true,%
linkcolor=black,%
citecolor=red,%
urlcolor=green,%
]{hyperref}
}

%-- Unterscheidung des Stils
\newcommand{\customLogo}{}
\newcommand{\customPreamble}{}
\ifthenelse{\equal{\customDesign}{htw}}{
	% HTW Corporate Design: Arial (Helvetica)
	\usepackage{helvet}
	\renewcommand{\familydefault}{\sfdefault}
	\renewcommand{\customLogo}{HTW-Logo}
	\renewcommand{\customPreamble}{HTW Dresden}
}{
% \renewcommand{\customLogo}{HTW-Logo.eps}
}

% Nach Dokumentenbeginn ausführen:
\AtBeginDocument{
	% Autor und Titel für pdf-Eigenschaften festlegen, falls noch nicht geschehen
	\providecommand{\pdfAuthor}{John Doe}
	\ifdefempty{\customAuthor} {} {\renewcommand{\pdfAuthor}{\customAuthor}}
	\providecommand{\pdfTitle}{}
	\providecommand{\pdfTitleA}{}
	\providecommand{\pdfTitleB}{}
	\providecommand{\pdfTitleC}{}	
	\ifdefempty{\pdfTitle}{
		\ifdefempty{\customPreamble} {} {\renewcommand{\pdfTitleA}{\customPreamble{} | }}
		\ifdefempty{\customTitle} {\renewcommand{\pdfTitleB}{No Title}} {\renewcommand{\pdfTitleB}{\customTitle}}
		\ifdefempty{\customSubtitle} {} {\renewcommand{\pdfTitleC}{ - \customSubtitle}}
	}{}
	
	\newcommand{\customLogoLocation}{\customDir .LaTeX_master/\customLogo}
	\hypersetup{
		pdfauthor={\pdfAuthor},
		pdftitle={\pdfTitleA\pdfTitleB\pdfTitleC},
	}
	\ifthenelse{\equal{\customDocumentClass}{beamer}}{
		\title{\customTitle}
		\author{\customAuthor}
	}{
		\automark[section]{section}
		\automark*[subsection]{subsection}
		\pagestyle{scrheadings}
		\ifthenelse{\equal{\customDocumentClass}{report} \OR \equal{\customDocumentClass}{scrreprt}}{
		\renewcommand*{\chapterpagestyle}{scrheadings}
		}{}
		%\renewcommand*{\titlepagestyle}{scrheadings}
		\ihead{\includegraphics[height=1.7em]{\customLogoLocation}}
		%\ohead{\truncate{5cm}{\customTitle}}
		\ohead{\customTitle}
		\cfoot{\pagemark}
		\ofoot{\customSignature}
		% Titelseite
		\title{
		\includegraphics[width=0.35\textwidth]{\customDir .LaTeX_master/\customLogo}\\\vspace{0.5em}
		\Huge\textbf{\customTitle}
		\ifdefempty{\customSubtitle} {} {\\\vspace*{0.7em}\Large \customSubtitle}
		\\\vspace*{5em}}
		\author{
		\ifdefempty{\customNoteA} {} {\customNoteA \vspace*{1em}}\\ 
		\ifdefempty{\customAuthor} {} {\customTitleAuthor}
		\ifdefempty{\customNoteB}{}{\vspace*{1em}\\\customNoteB}
		}
		
		\ifthenelse{\equal{\customDocumentClass}{cheatsheet}}{
			\pagestyle{empty}
			\setlist{nolistsep}
	%		\usepackage{parskip}	% Aufzählung Abstand
	%		\setlength{\parskip}{0em}
			\lstset{
	    belowcaptionskip=0pt,
	    belowskip=0pt,
	    aboveskip=0pt,
			tabsize=2,
			frame=none,
			numbers=none,
			showspaces=false,
			showstringspaces=false,
			breaklines=true,
			}
		}{}
	}
}

% Unterabschnitte
%\newtheorem{example}{Beispiel}%[section]
%\newtheorem{definition}{Definition}[section]
%\newtheorem{discussion}{Diskussion}[section]
%\newtheorem{remark}{Bemerkung}[section]
%\newtheorem{proof}{Beweis}[section]
%\newtheorem{notation}{Schreibweise}[section]
\RequirePackage{xcolor}
%% EINFACHE BEFEHLE

% Abkürzungen Mathe
\newcommand{\EE}{\mathbb{E}}
\newcommand{\QQ}{\mathbb{Q}}
\newcommand{\RR}{\mathbb{R}}
\newcommand{\CC}{\mathbb{C}}
\newcommand{\NN}{\mathbb{N}}
\newcommand{\ZZ}{\mathbb{Z}}
\newcommand{\PP}{\mathbb{P}}
\renewcommand{\SS}{\mathbb{S}}
\newcommand{\cA}{\mathcal{A}}
\newcommand{\cB}{\mathcal{B}}
\newcommand{\cC}{\mathcal{C}}
\newcommand{\cD}{\mathcal{D}}
\newcommand{\cE}{\mathcal{E}}
\newcommand{\cF}{\mathcal{F}}
\newcommand{\cG}{\mathcal{G}}
\newcommand{\cH}{\mathcal{H}}
\newcommand{\cI}{\mathcal{I}}
\newcommand{\cJ}{\mathcal{J}}
\newcommand{\cM}{\mathcal{M}}
\newcommand{\cN}{\mathcal{N}}
\newcommand{\cP}{\mathcal{P}}
\newcommand{\cR}{\mathcal{R}}
\newcommand{\cS}{\mathcal{S}}
\newcommand{\cZ}{\mathcal{Z}}
\newcommand{\cL}{\mathcal{L}}
\newcommand{\cT}{\mathcal{T}}
\newcommand{\cU}{\mathcal{U}}
\newcommand{\cV}{\mathcal{V}}
\renewcommand{\phi}{\varphi}
\renewcommand{\epsilon}{\varepsilon}

% Farbdefinitionen
\definecolor{red}{RGB}{180,0,0}
\definecolor{green}{RGB}{75,160,0}
\definecolor{blue}{RGB}{0,75,200}
\definecolor{orange}{RGB}{255,128,0}
\definecolor{yellow}{RGB}{255,245,0}
\definecolor{purple}{RGB}{75,0,160}
\definecolor{cyan}{RGB}{0,160,160}
\definecolor{brown}{RGB}{120,60,10}

\definecolor{itteny}{RGB}{244,229,0}
\definecolor{ittenyo}{RGB}{253,198,11}
\definecolor{itteno}{RGB}{241,142,28}
\definecolor{ittenor}{RGB}{234,98,31}
\definecolor{ittenr}{RGB}{227,35,34}
\definecolor{ittenrp}{RGB}{196,3,125}
\definecolor{ittenp}{RGB}{109,57,139}
\definecolor{ittenpb}{RGB}{68,78,153}
\definecolor{ittenb}{RGB}{42,113,176}
\definecolor{ittenbg}{RGB}{6,150,187}
\definecolor{itteng}{RGB}{0,142,91}
\definecolor{ittengy}{RGB}{140,187,38}

% Textfarbe ändern
\newcommand{\tred}[1]{\textcolor{red}{#1}}
\newcommand{\tgreen}[1]{\textcolor{green}{#1}}
\newcommand{\tblue}[1]{\textcolor{blue}{#1}}
\newcommand{\torange}[1]{\textcolor{orange}{#1}}
\newcommand{\tyellow}[1]{\textcolor{yellow}{#1}}
\newcommand{\tpurple}[1]{\textcolor{purple}{#1}}
\newcommand{\tcyan}[1]{\textcolor{cyan}{#1}}
\newcommand{\tbrown}[1]{\textcolor{brown}{#1}}

% Umstellen der Tabellen Definition
\newcommand{\mpb}[1][.3]{\begin{minipage}{#1\textwidth}\vspace*{3pt}}
\newcommand{\mpe}{\vspace*{3pt}\end{minipage}}

\newcommand{\resultul}[1]{\underline{\underline{#1}}}
\newcommand{\parskp}{$ $\\}	% new line after paragraph
\newcommand{\corr}{\;\widehat{=}\;}
\newcommand{\mdeg}{^{\circ}}

\newcommand{\nok}[2]{\begin{pmatrix}#1\\#2\end{pmatrix}}	% n über k BESSER: \binom{n}{k}
\newcommand{\mtr}[1]{\begin{pmatrix}#1\end{pmatrix}}	% Matrix
\newcommand{\dtr}[1]{\begin{vmatrix}#1\end{vmatrix}}	% Determinante (Betragsmatrix)
\renewcommand{\vec}[1]{\underline{#1}}	% Vektorschreibweise
\newcommand{\imptnt}[1]{\colorbox{red!30}{#1}}	% Wichtiges
\newcommand{\intd}[1]{\,\mathrm{d}#1}
\newcommand{\diffd}[1]{\mathrm{d}#1}

%\bibliography{\customDir .Literatur/HTW_Literatur.bib}

% Run texcount on tex-file and write results to a sum-file
\immediate\write18{texcount \jobname.tex -1 -sum -out=\jobname.sum}
\newcommand\wordcount{\input{\jobname.sum}}

\newcommand{\customLogoLocation}{src/logo_big.jpg}
\newcommand{\customLogoHeadHeight}{3em}
\newcommand{\customLogoWidth}{0.8\linewidth}

\newcommand{\poke}{\,\!\texorpdfstring{\begingroup
\setbox0=\hbox{\includegraphics[width=3.5em]{src/logo_small.jpg}}%
\parbox{\wd0}{\box0}\endgroup%\includegraphics[height=1em]{src/logo_small.jpg}
}{Pokémon}\,}
%\newcommand{\poke}{Pokémon}
\newcommand{\pokeT}{Pokémon}

\usepackage{wrapfig}
\usepackage{multicol}
\begin{document}

\selectlanguage{english}
\maketitle
\newpage
\tableofcontents
\vfill
%Word Count: \wordcount words
\newpage

\chapter{The Gear}

The \emph{\poke{} reality gear} is a virtual reality device that enables you to look around and move in a virtual environment providing an immersive experience for everyone. 
%Once connected to a compatible device you can start playing the \emph{\poke{} reality gear} right away.  

\begin{figure}[!ht]
\begin{center}
\includegraphics[scale=3]{src/inst_gear}
\end{center}
\caption[The \emph{\pokeT{} reality gear}]{The \emph{\poke{} reality gear}}
\label{gear}
\end{figure}

\newpage
\section[Before using the \emph{\pokeT{} reality gear}]{Before using the \emph{\poke{} reality gear}}
\begin{itemize}
\item Carefully read and follow all the instructions provided with the \emph{\poke{} reality gear} before using the product.
\item Please be also aware of the warnings below before using the headset to reduce risk of injury or damage of the device.
%\item It is recommended to see a doctor if you are pregnant, elderly, or suffer from a mental illness or another medical condition.
\end{itemize}

\subsection*{Warnings}
\begin{itemize}
\item The headset produces an immersive virtual reality which is at times hard to differentiate from reality. To avoid possible accidents while using the \emph{\poke{} reality gear} stay away from stairs, balconies, windows etc.
\item Do not use the device for longer than 2 hours without breaks. 
\item Children under the age of 12 and psychologically unstable users are advised to disable the neuro-stimulators.
\item You may wear glasses inside the \emph{\poke{} reality gear}; they should however be removed in case of discomfort. Keeping the glasses on while experiencing discomfort may cause facial injuries. Users with poor eyesight are recommended to wear contact lenses while using the \emph{\poke{} reality gear}.
%\item The \emph{\poke{} reality gear} can be affected by magnetic interference created by devices, such as computers or televisions. Avoid using the Gear in areas affected by magnetic interference.
\item Direct sunlight pointed towards the lenses of the headset may cause damage to the display. When not using the headset, make sure to store it with the lenses pointed away from direct sunlight.
\end{itemize}

\section[Unpacking the \emph{\pokeT{} reality gear}]{Unpacking the \emph{\poke{} reality gear}}

The \emph{\poke{} reality gear} is shipped with the following components:

\begin{itemize}
\item Headset (including 2 lenses)
\item USB connection cable
\item Earphones
\item Cleaning Cloth
\end{itemize}


\subsection*{Parts of the headset}

\begin{figure}[!ht]
\begin{center}
\includegraphics[width=0.7\linewidth]{src/inst_gear1alt}
\includegraphics[width=0.65\linewidth]{src/inst_gear2altb}
\end{center}
\caption{The parts of the headset}
\label{gear_parts}
\end{figure}

\begin{enumerate}
\item On/Off switch for the neuro-stimulator
\item Tape connector
\item USB port
\item Front cover
\item Foam Cushioning 
\item Lenses
\end{enumerate}

\section{Getting started}

\subsection{Connecting the headset to your computer}
Use the USB cable to connect the headset with your computer.
\begin{figure}[!ht]
\begin{center}
\includegraphics[width=0.5\linewidth]{src/inst_connect}
\end{center}
\caption[Connecting the \emph{\pokeT{} reality gear} to your Computer]{Connecting the \emph{\poke{} reality gear} to your Computer}
\label{gear_connect}
\end{figure}
\begin{enumerate}
\item Put the headset plug (the thick end of the USB connection cable) into the USB port of the headset (figure \ref{gear_parts}: 3.).
\item Insert both USB plugs into your computer. Keep in mind that both the power and USB-data plug have to be connected to the USB ports of your computer for the headset to function (figure \ref{gear_connect}).
\end{enumerate}

\subsection{Installing the software}
\label{installSoftware}
Your computer should detect the \emph{\poke{} reality gear} and install the software via \emph{Plug and Play}. 

Please follow these steps, if your computer does not detect the \emph{\poke{} reality gear} automatically:
\begin{figure}[!ht]
\begin{center}
\includegraphics[width=0.5\linewidth]{src/inst_install}
\end{center}
\caption[Installing the \emph{\pokeT{} reality gear} software]{Installing the \emph{\poke{} reality gear} software}
\label{gear_install}
\end{figure}
\begin{enumerate}
\item Visit our Website \emph{pokemonreality.com} to download the \poke{} setup software. 
%Note: You need to register on our page before you can download the software. Please use the username and password provided in the package.
%\item Make sure you have an internet connection before you start with the installation. The setup wizard will guide through the installation step by step so that you do not make any mistake. 
\item Launch the \poke{} setup wizard and click 'next' (figure \ref{gear_install}).
\item Click 'install' to go ahead with the installation.
\item After the installation is finished please restart your computer. 
\item The \emph{\poke{} reality gear} is now ready to use with your computer.
\end{enumerate}

Please note, that the \emph{\poke{} reality game} is installed alongside with the \emph{\poke{} reality gear} software.

\subsection{Putting the Headset on}
The \emph{\poke{} reality gear} should be placed properly on your head during use:
\begin{figure}[!ht]
\begin{center}
\includegraphics[width=0.5\linewidth]{src/inst_gear3}
\end{center}
\caption]{The cable route of the headset}
\label{gear_cable}
\end{figure}
\begin{enumerate}
\item Make sure the straps are connected to the headset (figure \ref{gear_parts}: 2.)
\item Hold the headset in front of your head so that your eyes are on the same level with the lenses.
\item Now slide the straps to the back of your head and fasten them until the headset rests comfortably on your head. Make sure it fits perfectly and is neither too tight nor too loose.
\item Regularly check the headset's fit during usage to avoid dangerous pressure points.
\end{enumerate}

\subsection{Calibrating the interpupillary distance (IPD)}

Before using the \emph{\poke{} reality gear} you have to calibrate the interpupilarry distance (IPD), the distance between the center of the pupils of your eyes.

The \emph{\poke{} reality gear} has a calibration software to measure your IPD:
\begin{figure}[!ht]
\begin{center}
\includegraphics[width=0.5\linewidth]{src/inst_install3}
\end{center}
\caption[Configuring the \emph{\pokeT{} reality gear} software]{Configuring the \emph{\poke{} reality gear} software}
\label{gear_ipd}
\end{figure}
\begin{enumerate}
\item Start the \emph{\poke{} reality gear} software from your computer.
\item Go to the main menu and click 'vr\_calibration': The calibration screen(figure \ref{gear_ipd}) will open.
\item Put the headset on.
\item Follow the instructions visible on the headset screens.

The picture is expected to be blurry at first. At the end of the calibration the picture should be perfectly sharp.

If the picture is still blurry after the calibration try the following steps:
\begin{itemize}
\item Restart the calibration process through the \emph{\poke{} reality gear} software.
\item Manually adjust the IPD settings with options and sliders provided by the calibration screen.
\item Clean the lenses of the headset (see section \ref{cleaningLenses}) and retry the calibration.
\end{itemize}
\end{enumerate}

\subsection[Using the \emph{\pokeT{} reality gear}]{Using the \emph{\poke{} reality gear}}

You can operate the \emph{\poke{} reality gear} with the built-in tracking feature by using your hands to control the virtual surroundings. Just use your fingers to point and click on menu items or other interactive forms in the virtual world.

Since you can not see your actual hands with the headset strapped to your head, a digital representation is stimulated in the virtual reality. If you have activated the neuro-stimulators, you will have additional sensory input while interacting with the virtual reality.

\subsection{The neuro-stimulators}

To use the neuro-stimulators, activate the on/off switch on the left side of your headset (figure \ref{gear_parts}: 1.). 

Using the neuro-stimulators for the first time may result in a brief period of dizziness and lightheadedness. If you feel dizzy or uncomfortable for longer than three minutes, switch the neuro-stimulators off.

\newpage
\section[Cleaning the \emph{\pokeT{} reality gear}]{Cleaning the \emph{\poke{} reality gear}}

Use the included cleaning cloth to clean the \emph{\poke{} reality gear} and follow these instructions:

\begin{itemize}
\item Do not use any cleaning liquid for cleaning the headset; only use water. Some liquids with aggressive chemicals may cause damage to the headset.
\label{cleaningLenses}
\item You may use glass cleaner to clean the lenses. Wipe in circular motion without scratching the lenses.
\end{itemize}
\begin{figure}[!ht]
\begin{center}
\includegraphics[width=0.5\linewidth]{src/inst_clean}
\end{center}
\caption[Cleaning the \emph{\pokeT{} reality gear} lenses]{Cleaning the \emph{\poke{} reality gear} lenses}
\label{gear_clean}
\end{figure}

\chapter{The Game}

The \emph{\poke{} reality game} sets out to be an intuitively controlled game in conjunction with your \emph{\poke{} reality gear}.

%\section{Installing the game}

%The \emph{\poke{} reality game} is automatically installed together with the software for the \emph{\poke{} reality gear}. Please refer towards section \ref{installSoftware} if it is not installed on your computer yet.

\section{Starting the game}
\label{gameStart}

The game is only playable with your \emph{\poke{} reality gear}:
\begin{enumerate}
\item Start the \emph{\poke{} reality game} on your computer and put your \emph{\poke{} reality gear} on.
\item Choose the 'New Game' option appearing on the screen (figure \ref{game_start}).
\begin{figure}[!ht]
\begin{center}
\includegraphics[width=0.5\linewidth]{src/interface4}
\end{center}
\caption[Starting the \emph{\pokeT{} reality game}]{Starting the \emph{\poke{} reality game}}
\label{game_start}
\end{figure}
\item Follow the instructions listed on screen to choose your user name and gender, the \poke{} you want to encounter, and the regions you want to explore.
\item When you are finished the game will begin and place you in your new home in the \poke{} world. 

The controls for movement and interaction are projected in the virtual world and are operated with your gestures. Try to get accustomed with these controls by moving around in the room. 
\end{enumerate}
If you need any help (at this point or any other time in the game) please refer to section \ref{game_stuck}.

\section{Changing game settings}
\label{gameSettings}
The game settings are changeable in the following menus:
\begin{itemize}
\item The main menu before loading or starting a new game (figure \ref{game_start}).
\item The in-game menu during the game (figure \ref{game_menu}).
\end{itemize}

\begin{figure}[!ht]
\begin{center}
\includegraphics[width=0.5\linewidth]{src/interface}
\end{center}
\caption{The in-game menu}
\label{game_menu}
\end{figure}

There you may adjust the visual fidelity and some gameplay features. Please refer to the in-game help for a all the available options.

\section{Exploring}

If you followed the instructions in section \ref{gameStart} you should have already explored your room. Just take the final step through the door and explore the \poke{} world.

\section[\pokeT{} battles]{\poke{} battles}
\poke{} battles happen if you encounter wild \poke{} or a \poke{} trainer who wants to fight. The battle screen will open automatically and prompt you to choose a \poke{} with which you want to fight the opponent.

After you have chosen, the battlefield is shown(figure \ref{battle}). 
\begin{figure}[!ht]
\begin{center}
\includegraphics[width=0.5\linewidth]{src/interface3}
\end{center}
\caption{The battlefield}
\label{battle}
\end{figure}

By using the projected controls (not visible in the picture) you can take the following actions to choose what do in your turn:
\begin{enumerate}
\item Select "Fight" to choose an offensive or defensive action for your \poke{} to execute.
\item Select "Item" to use a special item from the inventory to buff the fighting \poke{}.
\item Select "Pokémon" to exchange the fighting \poke{} for a different \poke{} from your team.
\item Select "Run" to retreat from the battle.
\end{enumerate}
After the selection, both your action and the action of your opponent will be executed.

These turns are repeated until the fight is over. 

\section[Catching \pokeT{}]{Catching \poke{}}

If you want to catch a \poke{} simply select "Item" during one of your turns and choose a Pokéball to use. Remember, that you may only try to catch a wild \poke{} -- the \poke{} of other trainers are off limits.
\newpage
\section{What to do when you are stuck}
\label{game_stuck}

The \emph{\poke{} reality game} is designed to guide you through the game without the need of explicit instructions. If you need any help while playing just remember: Simply indicate a big question mark with your hands or hit the 'F1' key on the keyboard connected with your computer to get in-game help.

\chapter{Final Notes}

Thank you for reading this technical instruction of the \emph{\poke{} reality game and gear}. We hope you enjoy our new revolutionary way of experiencing the \poke{} world.

For an in-depth description of the \emph{\poke{} reality game and gear} please refer to our separately released \emph{technical description}.

\bigskip

If you have any feedback and suggestions regarding the \emph{\poke{} reality game and gear} you may contact us over email or send us a letter:\bigskip\\
\begin{tabular}{l L{.5}}
&Merko\& Strube Inc.\\
\Letter & Charmanderstreet 5\newline
D-04359 Squirtle Town\\
\Email & MandS.inc@pokereality.com\\
\end{tabular}

%\newpage
%\listoffigures
%\newpage
%\printbibliography

\end{document}