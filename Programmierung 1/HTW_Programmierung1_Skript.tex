\newcommand{\customDir}{../}
\RequirePackage{ifthen,xifthen}

% Input inkl. Umlaute, Silbentrennung
\RequirePackage[T1]{fontenc}
\RequirePackage[utf8]{inputenc}

% Arbeitsordner (in Abhängigkeit vom Master) Standard: .LateX_master Ordner liegt im Eltern-Ordner
\providecommand{\customDir}{../}
\newcommand{\setCustomDir}[1]{\renewcommand{\customDir}{#1}}
%%% alle Optionen:
% Doppelseitig (mit Rand an der Innenseite)
\newboolean{twosided}
\setboolean{twosided}{false}
% Eigene Dokument-Klasse (alle KOMA möglich; cheatsheet für Spicker [3 Spalten pro Seite, alles kleiner])
\newcommand{\customDocumentClass}{scrreprt}
\newcommand{\setCustomDocumentClass}[1]{\renewcommand{\customDocumentClass}{#1}}
% Unterscheidung verschiedener Designs: htw, fjs
\newcommand{\customDesign}{htw}
\newcommand{\setCustomDesign}[1]{\renewcommand{\customDesign}{#1}}
% Dokumenten Metadaten
\newcommand{\customTitle}{}
\newcommand{\setCustomTitle}[1]{\renewcommand{\customTitle}{#1}}
\newcommand{\customSubtitle}{}
\newcommand{\setCustomSubtitle}[1]{\renewcommand{\customSubtitle}{#1}}
\newcommand{\customAuthor}{}
\newcommand{\setCustomAuthor}[1]{\renewcommand{\customAuthor}{#1}}
%	Notiz auf der Titelseite (A: vor Autor, B: nach Autor)
\newcommand{\customNoteA}{}
\newcommand{\setCustomNoteA}[1]{\renewcommand{\customNoteA}{#1}}
\newcommand{\customNoteB}{}
\newcommand{\setCustomNoteB}[1]{\renewcommand{\customNoteB}{#1}}
% Format der Signatur in Fußzeile:
\newcommand{\customSignature}{\ifthenelse{\equal{\customAuthor}{}} {} {\footnotesize{\textcolor{darkgray}{Mitschrift von\\ \customAuthor}}}}
\newcommand{\setCustomSignature}[1]{\renewcommand{\customSignature}{#1}}
% Format des Autors auf dem Titelblatt:
\newcommand{\customTitleAuthor}{\textcolor{darkgray}{Mitschrift von \customAuthor}}
\newcommand{\setCustomTitleAuthor}[1]{\renewcommand{\customTitleAuthor}{#1}}
% Standard Sprache
\newcommand{\customDefaultLanguage}[1]{}
\newcommand{\setCustomDefaultLanguage}[1]{\renewcommand{\customDefaultLanguage}{#1}}
% Folien-Pfad (inkl. Dateiname ohne Endung und ggf. ohne Nummerierung)
\newcommand{\customSlidePath}{}
\newcommand{\setCustomSlidePath}[1]{\renewcommand{\customSlidePath}{#1}}
% Folien Eigenschaften
\newcommand{\customSlideScale}{0.5}
\newcommand{\setCustomSlideScale}[1]{\renewcommand{\customSlideScale}{#1}}
\newcommand{\customSlideHeight}{9.63cm}
\newcommand{\setCustomSlideHeight}[1]{\renewcommand{\customSlideHeight}{#1}}
\newcommand{\customSlideWidth}{12.8cm}
\newcommand{\setCustomSlideWidth}[1]{\renewcommand{\customSlideWidth}{#1}}

%\setboolean{twosided}{true}
%\setCustomDocumentClass{scrartcl}
%\setCustomDesign{htw}
%\setCustomSlidePath{Folien}

\setCustomTitle{Programmierung I}
\setCustomSubtitle{Vorlesungsskript}
\setCustomAuthor{Falk-Jonatan Strube}
%\setCustomNoteA{TitlepageNoteBeforeAuthor}
\setCustomNoteB{Vorlesung von Prof. Dr.-Ing. Beck}

%\setcustomSignature{\footnotesize{\textcolor{darkgray}{Mitschrift von\\ \customAuthor}}	% Formatierung der Signatur in der Fußzeile
%\setcustomTitleAuthor{\textcolor{darkgray}{Mitschrift von #1}}	% Formatierung des Autors auf dem Titelblatt

%-- Prüfen, ob Beamer
\ifthenelse{\equal{\customDocumentClass}{beamer}}{
%%% TODO: andere Layouts für Beamer außer HTW
	\documentclass[ignorenonframetext, 11pt, table]{beamer}
	
	\usenavigationsymbolstemplate{}
	\setbeamercolor{author in head/foot}{fg=black}
	\setbeamercolor{title}{fg=black}
	\setbeamercolor{bibliography entry author}{fg=htworange!70}
	%\setbeamercolor{bibliography entry title}{fg=blue} 
	\setbeamercolor{bibliography entry location}{fg=htworange!60} 
	\setbeamercolor{bibliography entry note}{fg=htworange!60}  
	
	\setbeamertemplate{itemize item}{\color{black}$\bullet$}
	\setbeamertemplate{itemize subitem}{\color{black}--}
	\setbeamertemplate{itemize subsubitem}{\color{black}$\bullet$}
	\makeatother
	\setbeamertemplate{footline}
	{
	\leavevmode
	\def\arraystretch{1.2}
	\arrayrulecolor{gray}
	\begin{tabular}{ p{0.167\textwidth} | p{0.491\textwidth} | p{0.089\textwidth} | p{0.103\textwidth}}
	\hline
	\strut\insertshortauthor & \insertshorttitle & Slide \insertframenumber{}% / \inserttotalframenumber{}
	 & May 4, 2016\\
	\end{tabular}
	}
	\setbeamertemplate{headline}
	{
	\leavevmode
	\setlength{\arrayrulewidth}{1pt}
	\hspace*{2em}	
	\begin{tabular}{p{0.63\textwidth}}
	\rule{0pt}{3em}\normalsize{\textbf{\insertsection\strut}}\\
	\arrayrulecolor{htworange}
	\hline
	\end{tabular}
	\begin{tabular}{l}
	\rule{0pt}{4em}\includegraphics[width=3.25cm]{\customDir .LaTeX_master/HTW_GESAMTLOGO_CMYK.eps}\\
	\end{tabular}
	}
	\makeatletter	
}{	
	%-- Für Spicker einiges anders:
	\ifthenelse{\equal{\customDocumentClass}{cheatsheet}}{
		\documentclass[a4paper,10pt,landscape]{scrartcl}
		\usepackage{geometry}
		\geometry{top=2mm, bottom=2mm, headsep=0mm, footskip=0mm, left=2mm, right=2mm}
		
		% Für Spicker \spsection für Section, zur Strukturierung \HRule oder \HDRule Linie einsetzen
		\usepackage{multicol}
		\newcommand{\spsection}[1]{\textbf{#1}}	% Platzsparende "section" für Spicker
	}{	%-- Ende Spicker-Unterscheidung-if
		%-- Unterscheidung Doppelseitig
		\ifthenelse{\boolean{twosided}}{
			\documentclass[a4paper,11pt, footheight=26pt,twoside]{\customDocumentClass}
			\usepackage[head=23pt]{geometry}	% head=23pt umgeht Fehlerwarnung, dafür größeres "top" in geometry
			\geometry{top=30mm, bottom=22mm, headsep=10mm, footskip=12mm, inner=27mm, outer=13mm}
		}{
			\documentclass[a4paper,11pt, footheight=26pt]{\customDocumentClass}
			\usepackage[head=23pt]{geometry}	% head=23pt umgeht Fehlerwarnung, dafür größeres "top" in geometry
			\geometry{top=30mm, bottom=22mm, headsep=10mm, footskip=12mm, left=20mm, right=20mm}
		}
		%-- Nummerierung bis Subsubsection für Report
		\ifthenelse{\equal{\customDocumentClass}{report} \OR \equal{\customDocumentClass}{scrreprt}}{
			\setcounter{secnumdepth}{3}	% zählt auch subsubsection
			\setcounter{tocdepth}{3}	% Inhaltsverzeichnis bis in subsubsection
		}{}
	}%-- Ende Spicker-Unterscheidung-else
	
	\usepackage{scrlayer-scrpage}	% Kopf-/Fußzeile
	\renewcommand*{\thefootnote}{\fnsymbol{footnote}}	% Fußnoten-Symbole anstatt Zahlen
	\renewcommand*{\titlepagestyle}{empty} % Keine Seitennummer auf Titelseite
	\usepackage[perpage]{footmisc}	% Fußnotenzählung Seitenweit, nicht Dokumentenweit
}

% Input inkl. Umlaute, Silbentrennung
\RequirePackage[T1]{fontenc}
\RequirePackage[utf8]{inputenc}
\usepackage[english,ngerman]{babel}
\usepackage{csquotes}	% Anführungszeichen
\RequirePackage{marvosym}
\usepackage{eurosym}

% Style-Aufhübschung
\usepackage{soul, color}	% Kapitälchen, Unterstrichen, Durchgestrichen usw. im Text
%\usepackage{titleref}

% Mathe usw.
\usepackage{amssymb}
\usepackage{amsthm}
\ifthenelse{\equal{\customDocumentClass}{beamer}}{}{
\usepackage[fleqn,intlimits]{amsmath}	% fleqn: align-Umgebung rechtsbündig; intlimits: Integralgrenzen immer ober-/unterhalb
}
%\usepackage{mathtools} % u.a. schönere underbraces
\usepackage{xcolor}
\usepackage{esint}	% Schönere Integrale, \oiint vorhanden
\everymath=\expandafter{\the\everymath\displaystyle}	% Mathe Inhalte werden weniger verkleinert
\usepackage{wasysym}	% mehr Symbole, bspw \lightning
% Auch arcus-Hyperbolicus-Funktionen
\DeclareMathOperator{\arccot}{arccot}
\DeclareMathOperator{\arccosh}{arccosh}
\DeclareMathOperator{\arcsinh}{arcsinh}
\DeclareMathOperator{\arctanh}{arctanh}
\DeclareMathOperator{\arccoth}{arccoth} 
%\renewcommand{\int}{\int\limits}
%\usepackage{xfrac}	% mehr fracs: sfrac{}{}
\let\oldemptyset\emptyset	% schöneres emptyset
\let\emptyset\varnothing
%\RequirePackage{mathabx}	% mehr Symbole
\mathchardef\mhyphen="2D	% Hyphen in Math

% tikz usw.
\usepackage{tikz}
\usepackage{pgfplots}
\pgfplotsset{compat=1.11}	% Umgeht Fehlermeldung
\usetikzlibrary{graphs}
%\usetikzlibrary{through}	% ???
\usetikzlibrary{arrows}
\usetikzlibrary{arrows.meta}	% Pfeile verändern / vergrößern: \draw[-{>[scale=1.5]}] (-3,5) -> (-3,3);
\usetikzlibrary{automata,positioning} % Zeilenumbruch im Node node[align=center] {Text\\nächste Zeile} automata für Graphen
\usetikzlibrary{matrix}
\usetikzlibrary{patterns}	% Schraffierte Füllung
\usetikzlibrary{shapes.geometric}	% Polygon usw.
\tikzstyle{reverseclip}=[insert path={	% Inverser Clip \clip
	(current page.north east) --
	(current page.south east) --
	(current page.south west) --
	(current page.north west) --
	(current page.north east)}
% Nutzen: 
%\begin{tikzpicture}[remember picture]
%\begin{scope}
%\begin{pgfinterruptboundingbox}
%\draw [clip] DIE FLÄCHE, IN DER OBJEKT NICHT ERSCHEINEN SOLL [reverseclip];
%\end{pgfinterruptboundingbox}
%\draw DAS OBJEKT;
%\end{scope}
%\end{tikzpicture}
]	% Achtung: dafür muss doppelt kompliert werden!
\usepackage{graphpap}	% Grid für Graphen
\tikzset{every state/.style={inner sep=2pt, minimum size=2em}}
\usetikzlibrary{mindmap, backgrounds}
%\usepackage{tikz-uml}	% braucht Dateien: http://perso.ensta-paristech.fr/~kielbasi/tikzuml/

% Tabular
\usepackage{longtable}	% Große Tabellen über mehrere Seiten
\usepackage{multirow}	% Multirow/-column: \multirow{2[Anzahl der Zeilen]}{*[Format]}{Test[Inhalt]} oder \multicolumn{7[Anzahl der Reihen]}{|c|[Format]}{Test2[Inhalt]}
\renewcommand{\arraystretch}{1.3} % Tabellenlinien nicht zu dicht
\usepackage{colortbl}
\arrayrulecolor{gray}	% heller Tabellenlinien
\usepackage{array}	% für folgende 3 Zeilen (für Spalten fester breite mit entsprechender Ausrichtung):
\newcolumntype{L}[1]{>{\raggedright\let\newline\\\arraybackslash\hspace{0pt}}m{\dimexpr#1\columnwidth-2\tabcolsep-1.5\arrayrulewidth}}
\newcolumntype{C}[1]{>{\centering\let\newline\\\arraybackslash\hspace{0pt}}m{\dimexpr#1\columnwidth-2\tabcolsep-1.5\arrayrulewidth}}
\newcolumntype{R}[1]{>{\raggedleft\let\newline\\\arraybackslash\hspace{0pt}}m{\dimexpr#1\columnwidth-2\tabcolsep-1.5\arrayrulewidth}}
\usepackage{caption}	% Um auch unbeschriftete Captions mit \caption* zu machen

% Nützliches
\usepackage{verbatim}	% u.a. zum auskommentieren via \begin{comment} \end{comment}
\usepackage{tabto}	% Tabs: /tab zum nächsten Tab oder /tabto{.5 \CurrentLineWidth} zur Stelle in der Linie
\NumTabs{6}	% Anzahl von Tabs pro Zeile zum springen
\usepackage{listings} % Source-Code mit Tabs
\usepackage{lstautogobble} 
\ifthenelse{\equal{\customDocumentClass}{beamer}}{}{
\usepackage{enumitem}	% Anpassung der enumerates
%\setlist[enumerate,1]{label=(\arabic*)}	% global andere Enum-Items
\renewcommand{\labelitemiii}{$\scriptscriptstyle ^\blacklozenge$} % global andere 3. Item-Aufzählungszeichen
}
\newenvironment{anumerate}{\begin{enumerate}[label=(\alph*)]}{\end{enumerate}} % Alphabetische Aufzählung
\usepackage{letltxmacro} % neue Definiton von Grundbefehlen
% Nutzen:
%\LetLtxMacro{\oldemph}{\emph}
%\renewcommand{\emph}[1]{\oldemph{#1}}
\RequirePackage{xpatch}	% ua. Konkatenieren von Strings/Variablen (etoolbox)


% Einrichtung von lst
\lstset{
basicstyle=\ttfamily, 
%mathescape=true, 
%escapeinside=^^, 
autogobble, 
tabsize=2,
basicstyle=\footnotesize\sffamily\color{black},
frame=single,
rulecolor=\color{lightgray},
numbers=left,
numbersep=5pt,
numberstyle=\tiny\color{gray},
commentstyle=\color{gray},
keywordstyle=\color{green},
stringstyle=\color{orange},
morecomment=[l][\color{magenta}]{\#}
showspaces=false,
showstringspaces=false,
breaklines=true,
literate=%
    {Ö}{{\"O}}1
    {Ä}{{\"A}}1
    {Ü}{{\"U}}1
    {ß}{{\ss}}1
    {ü}{{\"u}}1
    {ä}{{\"a}}1
    {ö}{{\"o}}1
    {~}{{\textasciitilde}}1
}
\usepackage{scrhack} % Fehler umgehen
\def\ContinueLineNumber{\lstset{firstnumber=last}} % vor lstlisting. Zum wechsel zum nicht-kontinuierlichen muss wieder \StartLineAt1 eingegeben werden
\def\StartLineAt#1{\lstset{firstnumber=#1}} % vor lstlisting \StartLineAt30 eingeben, um bei Zeile 30 zu starten
\let\numberLineAt\StartLineAt

% BibTeX
\usepackage[backend=bibtex8, bibencoding=ascii,
%style=authortitle, citestyle=authortitle-ibid,
%doi=false,
%isbn=false,
%url=false
]{biblatex}	% BibTeX
\usepackage{makeidx}
%\makeglossary
%\makeindex

% Grafiken
\usepackage{graphicx}
\usepackage{epstopdf}	% eps-Vektorgrafiken einfügen
%\epstopdfsetup{outdir=\customDir}

% pdf-Setup
\usepackage{pdfpages}
\ifthenelse{\equal{\customDocumentClass}{beamer}}{}{
\usepackage[bookmarks,%
bookmarksopen=false,% Klappt die Bookmarks in Acrobat aus
colorlinks=true,%
linkcolor=black,%
citecolor=red,%
urlcolor=green,%
]{hyperref}
}

%-- Unterscheidung des Stils
\newcommand{\customLogo}{}
\newcommand{\customPreamble}{}
\ifthenelse{\equal{\customDesign}{htw}}{
	% HTW Corporate Design: Arial (Helvetica)
	\usepackage{helvet}
	\renewcommand{\familydefault}{\sfdefault}
	\renewcommand{\customLogo}{HTW-Logo}
	\renewcommand{\customPreamble}{HTW Dresden}
}{
% \renewcommand{\customLogo}{HTW-Logo.eps}
}

% Nach Dokumentenbeginn ausführen:
\AtBeginDocument{
	% Autor und Titel für pdf-Eigenschaften festlegen, falls noch nicht geschehen
	\providecommand{\pdfAuthor}{John Doe}
	\ifdefempty{\customAuthor} {} {\renewcommand{\pdfAuthor}{\customAuthor}}
	\providecommand{\pdfTitle}{}
	\providecommand{\pdfTitleA}{}
	\providecommand{\pdfTitleB}{}
	\providecommand{\pdfTitleC}{}	
	\ifdefempty{\pdfTitle}{
		\ifdefempty{\customPreamble} {} {\renewcommand{\pdfTitleA}{\customPreamble{} | }}
		\ifdefempty{\customTitle} {\renewcommand{\pdfTitleB}{No Title}} {\renewcommand{\pdfTitleB}{\customTitle}}
		\ifdefempty{\customSubtitle} {} {\renewcommand{\pdfTitleC}{ - \customSubtitle}}
	}{}
	
	\newcommand{\customLogoLocation}{\customDir .LaTeX_master/\customLogo}
	\hypersetup{
		pdfauthor={\pdfAuthor},
		pdftitle={\pdfTitleA\pdfTitleB\pdfTitleC},
	}
	\ifthenelse{\equal{\customDocumentClass}{beamer}}{
		\title{\customTitle}
		\author{\customAuthor}
	}{
		\automark[section]{section}
		\automark*[subsection]{subsection}
		\pagestyle{scrheadings}
		\ifthenelse{\equal{\customDocumentClass}{report} \OR \equal{\customDocumentClass}{scrreprt}}{
		\renewcommand*{\chapterpagestyle}{scrheadings}
		}{}
		%\renewcommand*{\titlepagestyle}{scrheadings}
		\ihead{\includegraphics[height=1.7em]{\customLogoLocation}}
		%\ohead{\truncate{5cm}{\customTitle}}
		\ohead{\customTitle}
		\cfoot{\pagemark}
		\ofoot{\customSignature}
		% Titelseite
		\title{
		\includegraphics[width=0.35\textwidth]{\customDir .LaTeX_master/\customLogo}\\\vspace{0.5em}
		\Huge\textbf{\customTitle}
		\ifdefempty{\customSubtitle} {} {\\\vspace*{0.7em}\Large \customSubtitle}
		\\\vspace*{5em}}
		\author{
		\ifdefempty{\customNoteA} {} {\customNoteA \vspace*{1em}}\\ 
		\ifdefempty{\customAuthor} {} {\customTitleAuthor}
		\ifdefempty{\customNoteB}{}{\vspace*{1em}\\\customNoteB}
		}
		
		\ifthenelse{\equal{\customDocumentClass}{cheatsheet}}{
			\pagestyle{empty}
			\setlist{nolistsep}
	%		\usepackage{parskip}	% Aufzählung Abstand
	%		\setlength{\parskip}{0em}
			\lstset{
	    belowcaptionskip=0pt,
	    belowskip=0pt,
	    aboveskip=0pt,
			tabsize=2,
			frame=none,
			numbers=none,
			showspaces=false,
			showstringspaces=false,
			breaklines=true,
			}
		}{}
	}
}

% Unterabschnitte
%\newtheorem{example}{Beispiel}%[section]
%\newtheorem{definition}{Definition}[section]
%\newtheorem{discussion}{Diskussion}[section]
%\newtheorem{remark}{Bemerkung}[section]
%\newtheorem{proof}{Beweis}[section]
%\newtheorem{notation}{Schreibweise}[section]
\RequirePackage{xcolor}
\RequirePackage{amsmath}

% Horizontale Linie:
\newcommand{\HRule}[1][\medskipamount]{\par
  \vspace*{\dimexpr-\parskip-\baselineskip+#1}
  \noindent\rule[0.2ex]{\linewidth}{0.2mm}\par
  \vspace*{\dimexpr-\parskip-.5\baselineskip+#1}}
% Gestrichelte horizontale Linie:
\RequirePackage{dashrule}
\newcommand{\HDRule}[1][\medskipamount]{\par
  \vspace*{\dimexpr-\parskip-\baselineskip+#1}
  \noindent\hdashrule[0.2ex]{\linewidth}{0.2mm}{1mm} \par
  \vspace*{\dimexpr-\parskip-.5\baselineskip+#1}}
% Mathe in Anführungszeichen:
\newsavebox{\mathbox}\newsavebox{\mathquote}
\makeatletter
\newcommand{\mq}[1]{% \mathquotes{<stuff>}
  \savebox{\mathquote}{\text{"}}% Save quotes
  \savebox{\mathbox}{$\displaystyle #1$}% Save <stuff>
  \raisebox{\dimexpr\ht\mathbox-\ht\mathquote\relax}{"}#1\raisebox{\dimexpr\ht\mathbox-\ht\mathquote\relax}{''}
}
\makeatother

% Paragraph mit Zähler (Section-Weise)
\newcounter{cparagraphC}
\newcommand{\cparagraph}[1]{
\stepcounter{cparagraphC}
\paragraph{\thesection{}-\thecparagraphC{} #1}
%\addcontentsline{toc}{subsubsection}{\thesection{}-\thecparagraphC{} #1}
\label{\thesection-\thecparagraphC}
}
\makeatletter
\@addtoreset{cparagraphC}{section}
\makeatother


% (Vorlesungs-)Folien einbinden:
% Folien von einer Datei skaliert
\newcommand{\slide}[2][\customSlideScale]{\slides[#1]{}{#2}}
\newcommand{\slideTrim}[6][\customSlideScale]{\slides[#1 , clip,  trim = #5cm #4cm #6cm #3cm]{}{#2}}
% Folien von mehreren nummerierten Dateien skaliert
\newcommand{\slides}[3][\customSlideScale]{\begin{center}
\includegraphics[page=#3, scale=#1]{\customSlidePath #2.pdf}
\end{center}}

% \emph{} anders definieren
\makeatletter
\DeclareRobustCommand{\em}{%
  \@nomath\em \if b\expandafter\@car\f@series\@nil
  \normalfont \else \scshape \fi}
\makeatother

% unwichtiges
\newcommand{\unimptnt}[1]{{\transparent{0.5}#1}}

% alph. enumerate
\newenvironment{anumerate}{\begin{enumerate}[label=(\alph*)]}{\end{enumerate}} % Alphabetische Aufzählung

%% EINFACHE BEFEHLE

% Abkürzungen Mathe
\newcommand{\EE}{\mathbb{E}}
\newcommand{\QQ}{\mathbb{Q}}
\newcommand{\RR}{\mathbb{R}}
\newcommand{\CC}{\mathbb{C}}
\newcommand{\NN}{\mathbb{N}}
\newcommand{\ZZ}{\mathbb{Z}}
\newcommand{\PP}{\mathbb{P}}
\renewcommand{\SS}{\mathbb{S}}
\newcommand{\cA}{\mathcal{A}}
\newcommand{\cB}{\mathcal{B}}
\newcommand{\cC}{\mathcal{C}}
\newcommand{\cD}{\mathcal{D}}
\newcommand{\cE}{\mathcal{E}}
\newcommand{\cF}{\mathcal{F}}
\newcommand{\cG}{\mathcal{G}}
\newcommand{\cH}{\mathcal{H}}
\newcommand{\cI}{\mathcal{I}}
\newcommand{\cJ}{\mathcal{J}}
\newcommand{\cM}{\mathcal{M}}
\newcommand{\cN}{\mathcal{N}}
\newcommand{\cP}{\mathcal{P}}
\newcommand{\cR}{\mathcal{R}}
\newcommand{\cS}{\mathcal{S}}
\newcommand{\cZ}{\mathcal{Z}}
\newcommand{\cL}{\mathcal{L}}
\newcommand{\cT}{\mathcal{T}}
\newcommand{\cU}{\mathcal{U}}
\newcommand{\cV}{\mathcal{V}}
\renewcommand{\phi}{\varphi}
\renewcommand{\epsilon}{\varepsilon}

% Verschiedene als Mathe-Operatoren
\DeclareMathOperator{\arccot}{arccot}
\DeclareMathOperator{\arccosh}{arccosh}
\DeclareMathOperator{\arcsinh}{arcsinh}
\DeclareMathOperator{\arctanh}{arctanh}
\DeclareMathOperator{\arccoth}{arccoth} 
\DeclareMathOperator{\var}{Var} % Varianz 
\DeclareMathOperator{\cov}{Cov} % Co-Varianz 

% Farbdefinitionen
\definecolor{red}{RGB}{180,0,0}
\definecolor{green}{RGB}{75,160,0}
\definecolor{blue}{RGB}{0,75,200}
\definecolor{orange}{RGB}{255,128,0}
\definecolor{yellow}{RGB}{255,245,0}
\definecolor{purple}{RGB}{75,0,160}
\definecolor{cyan}{RGB}{0,160,160}
\definecolor{brown}{RGB}{120,60,10}

\definecolor{itteny}{RGB}{244,229,0}
\definecolor{ittenyo}{RGB}{253,198,11}
\definecolor{itteno}{RGB}{241,142,28}
\definecolor{ittenor}{RGB}{234,98,31}
\definecolor{ittenr}{RGB}{227,35,34}
\definecolor{ittenrp}{RGB}{196,3,125}
\definecolor{ittenp}{RGB}{109,57,139}
\definecolor{ittenpb}{RGB}{68,78,153}
\definecolor{ittenb}{RGB}{42,113,176}
\definecolor{ittenbg}{RGB}{6,150,187}
\definecolor{itteng}{RGB}{0,142,91}
\definecolor{ittengy}{RGB}{140,187,38}

\definecolor{htworange}{RGB}{249,155,28}

% Textfarbe ändern
\newcommand{\tred}[1]{\textcolor{red}{#1}}
\newcommand{\tgreen}[1]{\textcolor{green}{#1}}
\newcommand{\tblue}[1]{\textcolor{blue}{#1}}
\newcommand{\torange}[1]{\textcolor{orange}{#1}}
\newcommand{\tyellow}[1]{\textcolor{yellow}{#1}}
\newcommand{\tpurple}[1]{\textcolor{purple}{#1}}
\newcommand{\tcyan}[1]{\textcolor{cyan}{#1}}
\newcommand{\tbrown}[1]{\textcolor{brown}{#1}}

% Umstellen der Tabellen Definition
\newcommand{\mpb}[1][.3]{\begin{minipage}{#1\textwidth}\vspace*{3pt}}
\newcommand{\mpe}{\vspace*{3pt}\end{minipage}}

\newcommand{\resultul}[1]{\underline{\underline{#1}}}
\newcommand{\parskp}{$ $\\}	% new line after paragraph
\newcommand{\corr}{\;\widehat{=}\;}
\newcommand{\mdeg}{^{\circ}}

\newcommand{\nok}[2]{\binom{#1}{#2}}	% n über k BESSER: \binom{n}{k}
\newcommand{\mtr}[1]{\begin{pmatrix}#1\end{pmatrix}}	% Matrix
\newcommand{\dtr}[1]{\begin{vmatrix}#1\end{vmatrix}}	% Determinante (Betragsmatrix)
\renewcommand{\vec}[1]{\underline{#1}}	% Vektorschreibweise
\newcommand{\imptnt}[1]{\colorbox{red!30}{#1}}	% Wichtiges
\newcommand{\intd}[1]{\,\mathrm{d}#1}
\newcommand{\diffd}[1]{\mathrm{d}#1}
% für Module-Rechnung: \pmod{}
\newcommand{\unit}[1]{\,\mathrm{#1}}

%\bibliography{\customDir _Literatur/HTW_Literatur.bib}

\begin{document}

%\selectlanguage{english}
\maketitle
\newpage
\tableofcontents
\newpage

\chapter*{Hinweise}

Zugelassene Hilfsmittel Klausur: Spickzettel A-4 Blatt, doppelseitig (, man-page c++.com)\\
VarArg nicht Prüfungsinhalt, da kein Praktikum dazu\\
Beleg abzugeben bis spätestens: 28.02. per Mail an beck@informatik.htw-dresden.de (Mail-Text: Studiengruppe + s-Nummer, Anhang: Quelldateien, Readme oder link zu Datei(en) in publichtml)

\chapter{Einführung}

Bilde Durchschnitt aus folgender Notenübersicht:\\
\begin{tabular}{c c}
Index & Note\\
\hline
0&3\\
1&4\\
2&1\\
3&3\\
4&3\\
5&5\\
6&3\\
7&4\\
8&0\\
9&-\\
\end{tabular}

\section{Algorithmus}
\begin{enumerate}
\item Lösche Akku $\rightarrow$  2.
\item Lösche Counter $\rightarrow$  3.
\item Gibt es eine Zahl an Stelle Count? 
\begin{itemize}
\item Ja: $\rightarrow$  4.
\item Nein: $\rightarrow$  6.
\end{itemize}
\item Addiere markierte Zahl zu Akku $\rightarrow$  5.
\item Addiere 1 zu Counter $\rightarrow$  3.
\item Dividiere Wert in Akku durch Wert in Counter und speichere Akku $\rightarrow $ 7.
\item Ergebnis: Ausgabe des Akku $\rightarrow$ ENDE
\end{enumerate}

\section{Programmablaufplan (PAP)}

\begin{tikzpicture}[scale=0.3]
\draw (-3,6) ellipse (3 and 1) node{Start};
\draw[-latex] (-3,5) -> (-3,3);
\draw  (-7,3) rectangle (1,1) node[pos =.5]{Akk:=0};
\draw[-latex] (-3,1) -> (-3,-1);
\draw  (-7,-1) rectangle (1,-3) node[pos =.5]{Count:=0};
\draw[-latex] (-3,-3) -- (-3,-5);
\draw (-3,-5) -- (-10,-11) -- (-3,-17) -- (4,-11) -- cycle;
\draw  (-3,-11) node[align=center] {Gibt es eine\\ gültige Note an der\\ Stelle Count?};

\draw[-latex] (-3,-17) -- (-3,-19);
\draw (-2,-18) node{ja};
\draw  (-13,-19) rectangle (8,-21) node[pos =.5]{Akku := Akku + Noten[Count]};
\draw[-latex] (-3,-21) -- (-3,-23);
\draw (-10,-23) rectangle (5,-25) node[pos =.5]{Count := Count +1};
\draw[-latex] (-3,-25) -- (-3,-27) -- (10,-27) -- (10, -4) -- (-3,-4);

\draw[-latex] (-10,-11) -- (-19,-11) -- (-19,-13);
\draw (-19,-13) -- (-24,-17) -- (-19,-21) -- (-14,-17) -- cycle;
\draw  (-19,-17) node[align=center] {Count > 0?};
\draw[-latex] (-14,-17) -- (-12,-17);
\draw (-13,-16) node{nein};
\draw (-9,-17) ellipse (3 and 1) node {Fehler};
\draw[-latex] (-19,-21) -- (-19,-23);
\draw (-18,-22) node{ja};
\draw  (-27,-23) rectangle (-11,-25) node[pos =.5]{Akku := Akku/Count};
\draw[-latex] (-19,-25) -- (-19,-27);
\draw (-26,-27) -- (-10,-27) -- (-11,-29) -- (-27,-29) -- cycle;
\draw (-19,-28) node{Ausgabe Akku};
\draw[-latex] (-19,-29) -- (-19,-31);
\draw (-19,-32) ellipse (3 and 1) node{Erfolg};

\end{tikzpicture} 

\section[Struktogramm]{Struktogramm / Nassi-Shneiderman-Diagramm}

\begin{tikzpicture} [scale = 0.3]
\draw  (0,0) rectangle (28,2);
\draw (0,1) node[anchor=west]{Akku = 0}; 
\draw  (0,-6) rectangle (28,0);
\draw (0,-1) node[anchor=west]{while (Noten[Count]!=0)};
\draw  (4,-4) rectangle (28,-2);
\draw (4,-3) node[anchor=west]{Akku = Akku + Noten[Count]}; 
\draw  (4,-6) rectangle (28,-4);
\draw (4,-5) node[anchor=west]{Count :=  Count + 1}; 
\draw  (0,-8) rectangle (28,-6);
\draw (0,-7) node[anchor=west]{if (Count>0)}; 
\draw  (0,-12) rectangle (28,-8);
\draw (0,-11) node[anchor=west]{ja}; 
\draw (28,-11) node[anchor=east]{nein};
\draw (0,-8) -- (14,-12) -- (28,-8);
\draw (0,-12) rectangle (14,-16);
\draw (0,-14) node[anchor=west, align=left]{Akku = Akku/Count\\ Ausgabe Akku}; 
\draw (14,-12) rectangle (28,-16);
\draw (14,-14) node[anchor=west, align=left]{Fehler}; 
\end{tikzpicture}

\section{Quelltext in C}

\begin{lstlisting}[language=C]
#include <stdio.h>
#include <stdlib.h>

int Noten []={5,2,3,4,5,5,2,3,4,5,0}; //38/10

int main(){
	int Akku=0, Count=0;
	while (Noten[Count]!=0){
		Akku = Akku+Noten[Count];
		Count = Count+1;
	}
	if(Count>0){
		Akku = Akku/Count;
		printf("Durchschnitt: %d\n",Akku);
	} else
		printf("Fehler - Division durch 0\n");
	return 0;
}
\end{lstlisting}

Compilieren durch: 
\begin{lstlisting}
gcc SOURCE.c -o DESTINATION
\end{lstlisting}

Ergebnis:\\
„3“ … aber: 38/10 = 3,8. Integer im Source-Code $\Rightarrow$ abgerundet

Lösung:
\begin{lstlisting}[language=C]
// Ergebnis mit Runden (innerhalb der if(Count>0)-Klammer)
Akku = Akku*10/Count;
printf("Durchschnitt: %d.%d\n",Akku/10,Akku%10);
\end{lstlisting}

Alternativ:
\begin{lstlisting}[language=C]
// anstatt int Akku=0, Count=0;
double Akku=0;
int Count =0;
\end{lstlisting}

\chapter{gcc}

gcc Ablauf für eine „hello.c“ Datei.

\begin{enumerate}
\item Pre-Prozessor (hello.c $\rightarrow$ hello.e $\Rightarrow$ gcc -E hello.c > hello.e) \\
Alle Zeile im Quelltext mit \emph{\#} werden hier interpretiert.
\item Compiler (hello.e $\rightarrow$ hello.o $\Rightarrow$ gcc -c hello.c)
\item Linker (hello.o $\rightarrow$ a.out / hello.exe | gcc hello.c $\rightarrow$ a.out $\Rightarrow$ gcc -o hello hello.c [oder auch gcc hello.c -o hello])\\
Bindet Objekt-Datei (xxx.o) mit Librarys zusammen.
\end{enumerate}

\chapter{Grundlagen von C}

\begin{itemize}
\item Großbuchstaben, Kleinbuchstaben, Ziffern, Sonderzeichen
\item Ersatzsymbolfolgen (brauchen besondere Compileranweisung)
\item white spaces: 
\begin{itemize}
\item Leerzeichen
\item Zeilenende(\textbackslash n)
\item Tabulator(\textbackslash t)
\item vertikaler Tab(\textbackslash v)
\item Seitenvorschub(\textbackslash f)
\item Bell(\textbackslash a)
\item Backspace(\textbackslash b)
\item Carriage return(\textbackslash r)
\end{itemize}
\item Schlüsselwörter:\\
auto, default, float, long, sizeof, union, break, do, for, register, static, …
\end{itemize}

\section{Eingebaute Datentypen}
\begin{tabular}{l | l l l}
\emph{Natürliche Zahlen} & Bytes & Min & Max\\
\hline
unsigned char&1&0&255\\
unsigned short \textcolor{gray}{int}&2&0&65535\\
unsigned \textcolor{gray}{int}&4&0&4294967295\\
unsigned long \textcolor{gray}{int}&4 / 8&0&4294967295\\
unsigned long long \textcolor{gray}{int}&8&0&1.8...*1019
\end{tabular}\medskip\\
\begin{tabular}{l | l l l}
\emph{Ganze Zahlen}&Bytes&Min&Max\\
\hline
signed char&1&-128&127\\
\textcolor{gray}{signed} short \textcolor{gray}{int}&2&-32768&32767\\
\textcolor{gray}{signed} int&4&-2147483648&2147483647\\
\textcolor{gray}{signed} long \textcolor{gray}{int}&4 / 8&-21474836480&2147483647\\
\textcolor{gray}{signed} long long \textcolor{gray}{int}&8&-9.2*E18&9.2*E18\\
\end{tabular}\medskip\\
\begin{tabular}{l | l l l l l}
\emph{Gebrochene Zahlen}&Bytes&VZ&Exp.&Mant&Wertebereich\\
\hline
float&4&1&8&23&1.2E-38 \quad 3.4E+38\\
double&8&1&11&52&2.3E-308 \quad 1.7E+308\\
Long double&10&1&&&3.4E-4932 \quad 1.1E+4932\\
\end{tabular}\medskip\\
Was grau markiert ist, kann weggelassen auch werden $\Rightarrow$ „unsigned int i;“ = „unsigned i;“

\paragraph{Bsp.:} $ $
\begin{lstlisting}[language=C]
unsigned int i;	// Variablen-Definition
i = 12;		// Wertzuweisung

printf("Wert von i: %d - Adresse von i: %p\n", i, &i);
// Hinweis: 
// %d - Dezimalwert, 
// %p - Adresswert, 
// &i - Adresse von Variable
\end{lstlisting}
Erstellung einer Variablen (int i;): \emph{uninitialisierte Variable / Variablen-Definition}\\
Wertbelegung einer Variable während Definition einer Variablen (int i=0;): \emph{Initialisierung}\\
Wertbelegung zu späterem Zeitpunkt (i=2;): \emph{Wertzuweisung}

\section{Operationen}
Geordnet nach Priorität:\\
\begin{tabular}{l l l l l l | l}
&&&&&&Abarbeitung\\
\hline
( )&[ ]&->&.&&&li nach re\\
!&~&-&(…)& cast&&\\
++&--&*&sizeof&&&re nach li\\
*&/&\%&&&&li nach re\\
+&-&&&&&li nach re\\
<<&>>&&&&&li nach re\\
<&<=&>&>=&&&li nach re\\
==&!=&&&&&li nach re\\
\& &&&&&&li nach re\\
$\hat{}$&&&&&&li nach re\\
|&&&&&&li nach re\\
\&\& &&&&&&li nach re\\
||&&&&&&li nach re\\
?: &&&&&&li nach re\\
=&*=&/=&\%=&-=&=&re nach li\\
$\hat{}$=&|=& < < = & > > = &&& re nach li\\
,&&&&&&\\
\end{tabular}


\section{Ausdrücke}
Programmiersprachliche Konstruktion zur Berechnung von Werten.
\subsection{Assoziativität}
(siehe Operatoren: Gewichtung der Operatoren von oben nach unten)\medskip\\
Unäre Operatoren (bspw. $-$ (negativ-Zeichen), $++$ (Inkrementierung) oder Klammern(cast))\\
Binäre Operatoren (bspw. $+$, $-$ (Rechenzeichen), $<=$ usw.)
\begin{lstlisting}[language=C]
int i;
long d;

i=(int)d;	// cast: Typwandlung

i++; // Postfixoperator (wird im Rahmen eines groesseren 
			Ausdrucks als letztes ausgefuehrt:)
i=1;
j=6;
k=j+i++; // k=7, i=2
++i; // Praefixoperator (wird im Rahmen eines groesseren 
			Ausdrucks als erstes ausgefuehrt:)
i=1;
j=6;
k=j+ ++i; // k=8, i=2

// Vorsicht! negativ-Bsp, wie ++ nicht zu verwenden ist:
i=2;
printf("%d\n", i++ + ++i);	\\ i= 6
printf("%d\n", i);	\\ i=4
\end{lstlisting}
Bei Division mit ganzen Zahlen wird der Rest abgeschnitten (nicht gerundet)!\medskip\\
Kurzschlussverfahren von Aneinanderkettung von Bedingungen ( i<0 || i<6) $\Rightarrow$ wenn die erste Prüfung wahr ist, wird der Test weiterer Bedingungen abgebrochen (bei \&\& wenn das erste falsch ist).\bigskip\\
\&\& im Vergleich zu \& (\& ist eine Bit-weise Operation): \\
01101100 \& 00001111 = 00001100 bzw. \\
01101100 | 11110000 = 11111100 \smallskip\\
Andere Zeichen: \\
$\wedge$ = XOR\\
$\sim$ = Bit-weise Negation\\
<\! < = shift (nach links) (bsp. i=4; i= i <\! < 2; $\Rightarrow$ i wird 16:\\
00000100 <\! < 2 $\Rightarrow$ 00010000\\
Achtung: bei negativen Zahlen (also Typ signed) bleibt bei Shift an der ersten Stelle das entsprechende Vorzeichenbit.\\
Bsp. für Abarbeitungsreihenfolge der Operatoren:\\
i*= 3+1		// i*(3+1)
\section{Anweisungen}
\begin{itemize}
\item Berechnungen
\item Alternative
\item Iteration
\item Sequenz
\end{itemize}
\subsection[Ausdrucksanweisung]{Ausdrucksanweisung (Expressionstatement)}
Eine Ausdrucksanweisung besteht aus einem Ausdruck gefolgt von einem Semikolon:
\begin{lstlisting}
<expr_stmnt>:: <expr> ';' .
\end{lstlisting}

Bsp.:
\begin{lstlisting}
printf("%d\n", i);
\end{lstlisting}
Zu Ausdrucksanweisungen gehören:
\begin{itemize}
\item Berechnungen
\item Aufrufe von Funktionen
\end{itemize}
\paragraph{Block}
Konstruktion, die Anweisungen kapselt -- nach außen einzelne Anweilungen enthält
\begin{itemize}
\item Vereinbarungen
\item Anweisungen
\end{itemize}
\begin{lstlisting}
<block>:: '{' { <statement> } '}' .
\end{lstlisting}

Bsp.:
\begin{lstlisting}[language=C]
#include <stdio.h>
#include <stdlib.h>

char vbuf[128];	// Vereinbarung

int main() {
	int i;	// Vereinbarung
	double x;	// Vereinbarung
	fgets(vbuf, 128, stdin);	// Anweisungen ...
	x=atof(vbuf);
	i = 1;
	x=x*10+i;
	printf("x: %lf\n",x);
}
\end{lstlisting}

\subsection[Alternativanweisung]{Alternativanweisung (if-statement)}

\begin{lstlisting}
<if-stmnt>:: 'if' '(' <condition> ')' 
				<statement>
			['else' <statement>] .
\end{lstlisting}

Bsp.:
\begin{lstlisting}[language=C]
#include <stdio.h>
#include <stdlib.h>

char vbuf[128];

int main() {
	double x;
	fgets(vbuf, 128, stdin);
	x=atof(vbuf);
	printf("x: %lf\n",x);
	if (x>1) printf("Groesser als 1\n");
	else printf("Kleiner als 1\n");	// optional
	puts("Hier geht es weiter");	
	// " " Strings (Zeichenketten), einzelnes Zeichen: '*'
}
\end{lstlisting}

\subsection{Leeranweisung}
\begin{lstlisting}
<empty_stmnt>:: ';'
\end{lstlisting}

\subsection[Iteration]{Iteration (Schleife/Loop)}
\paragraph{abweisende Schleife (kopfgesteuert)} while-Schleife
\begin{lstlisting}
<while_statement>:: 'while' '(' <condition> ')' <statement> .
\end{lstlisting}

Beispiel: $e^x$\\
e hoch x = 1 + x/1! + x*x /2! + x*x*x /3! …
\begin{enumerate} [label= \arabic* .]
\item  Summand: $x^0=1$
\item  Summand: $x^1/1! = x$
\item  Summand: $(x^1/1!)*x/2 = x^2/2!$
\item  Summand: $(x^2/2!) + x/3 = x^3/3!$
\end{enumerate}
Vereinfachung der Rechnung (für den Rechner) $\Rightarrow$ Nutzung des vorhergehenden Summanden.
\begin{lstlisting}[language=C]
#include <stdio.h>
#include <stdlib.h>

char vbuf[128];

int main() {
	int i=1;
	double x, y=1.0, summand = 1.0;
	printf("Eingabe von x: ");
	fgets(vbuf, 128, stdin);
	x=atof(vbuf);
	while (summand>0,00005){
		summand = summand *x/i;
		y += summand;
		printf("Summand %d: %lf\n", i, summand);
		i++;
	}
	printf("e^%lf: %lf\n", x, y);
	return 0;
}
\end{lstlisting}

\paragraph{Nicht abweisende Schleife (fußgesteuert)} do-while-Schleife
\begin{lstlisting}
<do_stmnt>:: 'do' <statement> 'while' '(' <condition> ')' ';' .
\end{lstlisting}

Bsp.:
\begin{lstlisting}[language=C]
#include <stdio.h>
#include <stdlib.h>

char vbuf[128];

int main() {
	int i=1;
	double x, y=1.0, summand = 1.0;
	printf("Eingabe von x: ");
	fgets(vbuf, 128, stdin);
	x=atof(vbuf);
	do{
		summand = summand *x/i;
		y += summand;
		printf("Summand %d: %lf\n", i, summand);
		i++;
	} while (summand>0,00005);
	printf("e^%lf: %lf\n", x, y);
	return 0;
}
\end{lstlisting}

\paragraph{for-Schleife} $ $
\begin{lstlisting}
<for-stmnt>:: 'for' '(' <expr>';' <expr>';' <expr> ')' <statement> .
\end{lstlisting}

Bsp.:
\begin{lstlisting}[language=C]
#include <stdio.h>
#include <stdlib.h>

char vbuf[128];

int main() {
	int i=1;
	double x, y=1.0, summand = 1.0;
	printf("Eingabe von x: ");
	fgets(vbuf, 128, stdin);
	x=atof(vbuf);
	for (i=1; 	// Schleifeninitialisierung
	summand > 0.0005; 	// Abbruchbedingung / Condition
	i++){	// Iterationsausdruck
		summand = summand *x/i;
		y += summand;
		printf("Summand %d: %lf\n", i, summand);
	}
	printf("e^%lf: %lf\n", x, y);
	return 0;
}
\end{lstlisting}

Alternativ-Bsp der for-Schleife mit Komma-Operator:
\begin{lstlisting} [language=C]
int main() {
	int i=1;
	double x, y, summand;
	printf("Eingabe von x: ");
	fgets(vbuf, 128, stdin);
	x=atof(vbuf);
	for (i=1, y=1.0, summand=1.0;
	summand > 0.0005;
	summand*=x/i, y+=summand,
		 printf("Summand %d: %lf\n", i, summand), i++){
	}
	printf("e^%lf: %lf\n", x, y);
	return 0;
}
\end{lstlisting}

\paragraph{Verlassen der Schleife} break
\begin{lstlisting}[language=C]
#include <stdio.h>
#include <stdlib.h>

char vbuf[128];

int main() {
	int i=1;
	double x, y=1.0, summand = 1.0;
	printf("Eingabe von x: ");
	fgets(vbuf, 128, stdin);
	x=atof(vbuf);
	while (1){
		summand = summand *x/i;
		y += summand;
		printf("Summand %d: %lf\n", i, summand);
		if (summand<0,00005) break;
		i++;
	}
	printf("e^%lf: %lf\n", x, y);
	return 0;
}
\end{lstlisting}
\emph{break} bezieht sich auf die (von innen nach außen) nächste zu findende Schleife. Also auf die Schleife, in deren \emph{statement} sie vorkommt.

\paragraph{Neuberechnung der Bedingung} continue\\
Verlässt den Schleifenkörper (der eingebettete Anweisung) und prüft die Bedingung erneut.
\begin{lstlisting}[language=C]
#include <stdio.h>
#include <stdlib.h>

char vbuf[128];

int main() {
	int i=1;
	double x, y=1.0, summand = 1.0;
	printf("Eingabe von x: ");
	fgets(vbuf, 128, stdin);
	x=atof(vbuf);
	while (summand>0,00005){
		summand = summand *x/i;
		y += summand;		
		i++;
		if (summand > 0.00005) continue;
		printf("Summand %d: %lf\n", i-1, summand);
	}
	printf("e^%lf: %lf\n", x, y);
	return 0;
}
\end{lstlisting}
Wenn Summand größer als 0.00005 ist, startet er die Schleife neu. Die printf() wird erst ausgeführt, wenn er kleiner ist (also das letzte mal).

\paragraph{Fallunterscheidung} switch-Anweisung
\begin{lstlisting}[language=C]
switch (i){ // i ist ganzzahliger Ausdruck
	case 1:
		... break;
	case 2:
		... break;
	default:
		...
}
\end{lstlisting}
\begin{lstlisting}[language=C]
#include <stdio.h>
#include <stdlib.h>

char buf[128];

int main(){
	int wota;
	printf("Wochentag (1...7): ");
	fgets(buf, 128, stdin); wota=atoi(buf);
	switch (wota){
		case 1: puts("Montag");			break;
		case 2: puts("Dienstag");		break;
		case 3: puts("Mittwoch");		break;
		case 4: puts("Donnerstag");	break;
		case 5: puts("Freitag");		break;
		case 6: puts("Samstag");		break;
		case 7: puts("Sonntag");		break;
		
		default: puts("Die Woch hat nur 7 Tage!");
	}
	return 0;
}
\end{lstlisting}

\section{Zusammenfassendes Beispiel}

\begin{lstlisting} [language=C]
#include <stdio.h>
#include <stdlib.h>

char buf [128];

int main(){
	int result=0;
	char operator=0;
	int value;
	while (operator!=toupper('q')){
		printf("Eingabe Operator: ");
		fgets(buf, 128, stdin);
		operator = buf[0];
		printf("Eingabe Zahl: ");
		fgets(buf, 128, stdin);
		value = atoi(buf);
		switch (operator) {
			case '+':	// Erinnerung: kein "+" - nur '+' für einzelne Zeichenketten
				result += value;
				break;
			case '-':
				result -= value;
				break;
			case '*':
				result *= value;
				break;
			case '/':
				if (value) // bzw. value!=0 - aber !=0 kann in C weggelassen werden
					result /= value;
				else
					puts("Division durch 0 ist nicht erlaubt.");
				break;
			case '%':
				if (value) // bzw. value!=0 - aber !=0 kann in C weggelassen werden
					result %= value;
				else
					puts("Division durch 0 ist nicht erlaubt.");
				break;
			case 'q':
				break;		
			default: 
				printf("unerlaubte Operation %c\n", operator);
		}
		printf("result: %d\n", result);
	}
}
\end{lstlisting}

\section{Zeichenketten}
\begin{lstlisting} [language=C]
#include <stdio.h>
#include <stdlib.h>
#include <string.h>

char buf [128];

int main(){
	printf("Eingabe Zeichenkette: ");
	fgets(buf, 128,stdin);
	printf("Len von Str: %d\n", strlen(buf));
	buf[strlen(buf)-1]=0;
	puts(buf);
	while (buf[i]!=0)
		printf("%c", buf[i++]);
	printf("\n");
	return 0;
}
\end{lstlisting}
Bei der Eingabe „Max“ wird bei puts() sowohl die Eingabe der Zeichenkette, als auch die Eingabe der Eingabetaste (neue Zeile) ausgegeben.\\
\begin{tabular}{|l | l |l |l |l |l |l |l }
\hline
M & a & x & $\backslash$ n & $\emptyset$ &\; &\;&\;\\
\hline
\end{tabular}\\
mit $\emptyset$: binäre, terminierende Null ( 0000 0000 )\\
Hinweis: der Buffer muss immer noch Platz für „$\backslash$ n“ und „$\emptyset$“ haben, d.h. man hat in einem Buffer der Größe von 128 nur Platz für 126 zeichen.\\
mit \emph{buf[strlen(buf)-1]=0;} wird die Eingabetaste „$\backslash$ n“ raus gelöscht.\bigskip\\
Daraus ergibt sich eine Verbesserung für den Taschenrechner:
\begin{lstlisting} [language=C]
...
	printf("Eingabe Operator /Operand: ");
	fgets(buf, 128, stdin);
	operator = buf[0];
	value = atoi(buf+1);	// Buffer ab der Stelle 0+1: 1
...
\end{lstlisting}

\section{Funktionen}

= Unterprogramme, zur Wiederholung von Codepassagen und zur besseren Strukturierung.
\begin{lstlisting}
<function> :: <func_head><funcbody>.
<func_head> :: [<return_type>] name '('<param_lst> |void|')'.
<param_list> :: [<type_name> name {','<type_name> name}]
<func_body> :: <block>
\end{lstlisting}
Bsp.:\\
\begin{tabular}{r l}
(func\_head) & $\underset{\text{(return\_type)}}{\text{int}}$ main()\\
 & \{ \\
(func\_body) & return 0;\\
& \} \\
\end{tabular}\\
Wenn kein return\_type gewählt wurde, dann default: \emph{int}.\\
Wenn kein return\_type gebraucht wird, gibt man \emph{void} an.
\begin{lstlisting}[language=C]
long fakult (int x) {	// int x: Parameter
	long f=1;
	int i;
	for (i=1; i<=x; i++){
		f*=i;
	}
	return f;
}

char vBuf[128];

int main(){
	double x,y;
	printf("Eingabe x: ");
	fgets(buf,128,stdin); x= atof(vBuf);
	y =fakult(x);
	printf("y: %ld \n", y);
	return 0;
}
\end{lstlisting}

\subsection{Gültigkeit}
\begin{itemize}
\item Bereich im C-Quelltext, an dem ein Bezeichner sichtbar ist.
\item Lebensdaur: Zeit von Erzeugung bis zur Vernichtung
\end{itemize}
\begin{lstlisting}[language = C]
static int count; // default wert 0
count++;	// behält jedes Mal ihren Wert, im Gegensatz zu int count!
\end{lstlisting}
\paragraph{Speicherklassen:}\parskp
\emph{auto} (automatische Variable): wird vom Stack erzeugt (Kellerspeicher)\\
lokale Varibalen\\
\emph{extern}: Variable, die in einem anderen Kontext vereinbart ist\\
\emph{static}: leben bis zum Programmende, global-statische Variabln werden nicht exportiert, immer initialiesiert, default 0\\
\emph{register}: Variablen werden nach Möglichkeit in ein Prozessorregister gelegt (schnell)\\
\emph{volatile}: Variabeln werden immer im Speicher abgelegt
\begin{lstlisting}[language=C]
long fakult(int x);	//Funktionsdeklaration
					// (prototyp)
\end{lstlisting}

\section{Header-File}
enthält die Funktionsköpfe
\begin{lstlisting}[language=C]
#include "fe.h" // wie bspw. stdio.h kann die eigene Datei in anderen
							//		 Quelltexten eingebunden werden
\end{lstlisting}

\section{Pointer}
\begin{lstlisting}[language=C]
#include <stdio.h>
#include <stdlib.h>

int main(){
	int i=99, *pi=&i;
	// &: Adressoperator
	printf("i: %d &i: %p \n", i, &i);
	// folgendes gibt nicht 99, sondern die Adresse von i (und die eigene) aus.
	// erst über *pi wird der Wert des Pointers ausgegeben (-> Dereferenzieren)
	printf("pi: %lx &pi: %p \n *pi: %d", pi, &pi, *pi);
	// Äquivalent dazu: (Arrays und Pointer sind in C sehr eng miteinander verwandt)
	printf("pi: %lx &pi: %p \n pi[0]: %d", pi, &pi, pi[0]);
	
	return 0;
}
\end{lstlisting}

\paragraph{Beispiel}

Beispiel anhand Eingangsbeispiel der Vorlesung für das Berechnen eines Durchschnitts.

\begin{lstlisting}[language=C]
#include <stdio.h>
#include <stdlib.h>

int Noten []={5,2,3,4,5,5,2,3,4,5,0}; //38/10

// double CalcMean(int Noten[]){
// folgendes präziser, da eigentlich kein Array, sondern bloß ein Pointer übergeben wird, 
// der Pointer ist eine neue Variable, die auf die Adresse des ersten Elements
// des Arrays zeigt (gilt für Parameter (wie hier) und return-Werte)
// int n für die Weitergabe der Array-Länge
double CalcMean(int* Noten, int n){	
	double Akku=0.0;
	int Count=0;
	printf("Len von Noten in Funktion: %d\n", sizeof Noten);	// = 4
	// = 4, weil nur die Adresse des ersten Elements! Kein Information über 
	// die Länge des Arrays durch die Übergabe in die Funktion!!!
	while (Count < n){
		// Obwohl Noten nur Pointer ist, können wir trotzdem Noten[Count] verwenden
		Akku = Akku+Noten[Count];
		// Alternativ: (Pointer-Dereferenzierung und Pointer-Arithmetik)
		// * hier keine Operation, sondern der Dereferenzierungs-Stern
		// Akku = Akku + *(Noten+Count);
		// auch Alternativ: Pointer-Variable wird weiter gestellt (diese ist nicht global)
		// Akku = Akku + *Noten++;
		// Folgende Zeile würde das globale Array ändern!!! (s.u.)		
		// Noten[Count] = 0;
		Count = Count+1;
	}
	if(Count>0)
		// Ergebnis mit Runden
		Akku = Akku/Count;
	return Akku;
}

int main(){
	double mean = 0.0;
	
	// Die tatsächliche Länge des Arrays
	printf("Len von Noten in Main: %d\n", sizeof Noten);	// = 44
	
	// ACHTUNG: Wenn wir das Array an die Funktion übergeben, wird nicht das Array selbst
	// übergeben, sondern die Adresse des Arrays (bzw. des ersten Elements)! Wird also 
	// innerhalb der Funktion das Array geändert, wird auch das globale Array geändert
	mean = CalcMean(Noten, sizeof(Noten)/sizeof(int));
	printf("Durchschnitt: %lf\n", mean);
	// sizeof(...): Operator, der die größe eines Elements bestimmt.
	// Durch das teilen durch sizeof(int) erhält man die Anzahl der Elemente 
	for (i=0; i<sizeof(Noten)/sizeof(int); i++)
		printf("%d ", Noten[i]);
	puts("");
	return 0;
}
\end{lstlisting}

\section{Rekursion}

Beispiel der Fakultät mit Rekursion:
\begin{lstlisting}[language=C]
#include <stdio.h>
#include <stdlib.h>

long fakultr(int x){
	long f;
	if (x>1) 
		f = x*fakultr(x-1);
	else 
		f=1;
	return f;
}

// Bsp. Abarbeitung:
// - u: 120
//   u/10: 12 >0
// - u: 12 
//   u/10: 1 >0
// - u: 1
//   u/10: 0 = 0
// Rekursiv werden dann die ASCII-Codes ausgegeben (ASCII-Wert von 0 + Zahl)
void printu(unsigned int u){
	if (u/10>0)
		print(u/10);
	putchar(u%10+'0');
}
	
int main(){
	long f;
	int i = 5;
	f = fakultr(i);
	printu((unsigned int) f);
	puts("");
}
\end{lstlisting}
\begin{itemize}
\item Rechtsrekursion (erst etwas rechnen, dann in die Rekursion gehen $\rightarrow$ fakultr)
\item Linksrekursion (erst Rekursiv aufrufen, dann etwas ausführen $\rightarrow$ printu)
\end{itemize}
Eine Links- oder Rechtsrekursion lässt sich auch Iterativ darstellen.
\begin{itemize}
\item Zentralrekursion
\end{itemize}
Eine Zentralrekursion lässt sich nicht Iterativ darstellen.\medskip\\
Problem Rekursion: Es lässt sich nicht vorhersehen, wie viel Speicher benötigt wird. Da ist die Schleife leichter überschaubar.

\section{Benutzerdefinierte Datentypen}

\subsection{Struct}
Struktur: alle Elemente liegen hintereinander (nicht zwangsläufig unmittelbar hintereinander) im Speicher \medskip\\
Alle komplieren:\\
gcc *.c (dafür braucht es die Header Datei st1.h, die alle Funktionen enthält [außer main]) $\Rightarrow$ alle Dateien werden in eine kompiliert\\
st1.h:
\begin{lstlisting}[language=C]
typedef unsigned int uint; // wie unten beispiel für synonym definition

typedef	// Wenn typedef davor steht, muss unterhalb ein Typ angegeben werden
struct{ //tStudent{ wenn typedef angegeben, kann der Typname davor weggelassen werden.
	char name[28+1];
	int matrNr:15;		// hier bspw. uint, anstatt unsigned int (wenn benötigt)
	int belnote:3;		// :x gibt die Bitlänge des Integer-Wertes ein 
						// (falls Ressourcen knapp) Bezeichnung: Bitfelder
	float klNote;
	// Struktur innerhalb der Struktur:
	struct {
		int jj,stGang, stgr;
	} stGr;
} tStud; // <- hier der Typ, synonym zu tStud. Jetzt kann überall anstatt 
		// "struct tStudent" nur "tStud" geschrieben werden

void putStud(tStud s);	// hier kann nun die Verkürzug angewandt werden 
		//-> aber auch in Dokumenten, in denen dieses included wird.
void putStudp(tStud s);
void getStud1(tStud * pStud);
tStud getStud2();
\end{lstlisting}
st1.c:
\begin{lstlisting}[language=C]
#include <stdio.h>
#include <stlib.h>
#include "st1.h"

struct tStudent stud ={"Hans Huckebeil", 12121, 0, 0.0,{2015,41,1}};
tStud ProjGr[5] ={{"Hans Huckebeil", 12121},
					{"Peter Lehmann", 12122},
					{"Anna Bauer", 12123}};

int main(){
	ProjGr[3]=getStud2();
	ProjGr[4]=getStud2();
	
	int i;
	for (i=0; i<5; i++){
		putStud(ProjGr[i]);
	}
	for (i=0; i<5; i++){
		puts(ProjGr[i].name);
	}
	// größe des Studenten zum Prüfen von Bitfeldern
	printf("Size Student: %d \n",sizeof(tStud));
	// ohne Bitfelder: 44
	// mit Bitfelder: 36
	// größe einer Struktur immer mit SIZEOF ermitteln! Nicht selbst ausrechnen
	// Compiler füllt Lücken ein (siehe 2. Übung)
	printf("%s %d/%d/%d \n",stud.name, stud.stGr.jj, stud.stGr.stGang, stud.stGr.stgr);

	//putStud(stud);
	//getStud1(&stud);
	//putStudp(&stud);
	return 1;
}
\end{lstlisting}
st1f.c:
\begin{lstlisting}[language=C]
#include <stdio.h>
#include <stlib.h>
#include <string.h>
#include "st1.h"

static char buf[128]; 

void putStd(struct tStudent s){
	printf("%-28s %u \n", s.name,s.matrNr);
}

void putStudp(tStud *s){
	//printf("%-28s %u \n", (*s).name,(*s).matrNr); alternativ dazu:
	printf("%-28s %u \n", s->name,s->matrNr);
}

void getStud1(tStud * pStud){
	printf("Eingabe Name: ");
	fgets(vbuf,128,stdin); 
	buf[strlen(buf)-1]=0;	// wegschmeißen des "\n" der Eingabe
	// Arrays können nicht zugewiesen werden, deshalb muss 
	// über Funktion strcpy() Byte für Byte eingetragen werden:
	strcpy(pStud->name,buf);
	printf("Eingabe Matrikelnr.: "); 
	fgets(vbuf,128,stdin); 
	pStud->matrNr=atoi(buf);
}

tStud getStud2(){
	tStud s;
	printf("Eingabe Name: ");
	fgets(vbuf,128,stdin); 
	buf[strlen(buf)-1]=0;
	strcpy(s.name,buf);
	printf("Eingabe Matrikelnr.: "); 
	fgets(vbuf,128,stdin); 
	s.matrNr=atoi(buf);
	return s;
}
\end{lstlisting}
\subsection{Union}
Alle Elemente eines Union-Datentyps liegen übereinander und liegen auf einer Adresse. \medskip\\
tu1.c
\begin{lstlisting}[language=C]
#include <stdio.h>
#include <stlib.h>

typedef union{
	unsigned int i;
	unsigned char arr[5];
} tbsp;

tbsp x;

// Argumente der main: Nutzen durch Programmzeilen Parameter
// argc: Anzahl der Parameter
// argv[i]: Der entsprechende Parameter
// Aufruf: ./tu1.c Max
// argc = 2, argv[0]=./tu1.c , argv[1]=Max
int main(int argc, char* argv[]){
	int i;
	for (i=0; i<argc; i++){
		puts (argv[i]);
	}
	
	if (argc==2){
		x.i = strtod(argv[1]);	// strtod alternativ zu atoi
		for (i=0; i<4; i++){
			printf("%02x ",x.arr[i]);	// so liegt die Zahl im Speicher
		}
		puts("");
	}
	return 0;
}

\end{lstlisting}

Sinnvoll, wenn Unterstrukturen gleiche und auch ungleiche Eigenschaften haben. Beispielsweise Diplom- und Bachelor-Studenten.

\subsection{typedef}
Für Details s.o. \\
Zusätzlicher Trick für den Umgang mit Arrays, bzw. sizeof() und typedef.
\begin{lstlisting}[language=C]
typedef int tIntArr[6];

void dispArr(tIntArr x){
	int i;
	printf("length (Bytte): %d length (Elements): %d\n",sizeof(IntArr),
		sizeof(IntArr)/sizeof(int));
	// da Referenz auf typedef und nicht auf x, funktioniert sizeof ohne Probleme
	for (i=0; i<sizeof(IntArr)/sizeof(int); i++)	
		printf("%d ",x[i]);
	puts("");
}

int main (){
	tIntArr arr={2,4,6,8,10,12}
	dispArr(arr);
	return 0;
}
\end{lstlisting}

\subsection{enum}

\begin{lstlisting}[language=C]
#include <stdio.h>
#include <stlib.h>

enum tWoTa{
	Montag=1, // bei Montag=1, fängt's bei 1 an zu zählen (anstatt 0)
	Dienstag, 
	Mittwoch, 
	Donnerstag,
	Freitag,
	Samstag=Freitag+1+0x10,	// verändert das Wochenende
	Sonntag};
	
enum tWoTa w=1;
	
int main(){
	printf("%d %x \n",Montag, Montag);	// gibt standarmäßig 0 aus; Sonntag wäre 6 usw.
	printf("%d %x \n",Dienstag, Dienstag);
	printf("%d %x \n",Mittwoch, Mittwoch);
	printf("%d %x \n",Donnerstag, Donnerstag);
	printf("%d %x \n",Freitag, Freitag);
	printf("%d %x \n",Samstag, Samstag);
	printf("%d %x \n",Sonntag, Sonntag);
	w=99;	
	printf("%d %x \n",w, w);
	return 0;
}
\end{lstlisting}
\section{Memory-Allocation}
Aus vorhergehendem Beispiel:\\
st1.c:
\begin{lstlisting}[language=C]
#include <stdio.h>
#include <stlib.h>
#include "st1.h"

int anz = 0;
char vbuf[128];
char weiter ='y';	// Ja

tStud s;

int main(){
	int i;
	tStud *ps = NULL, *psx;
	while (weiter == 'y'){
		s = getStud2();
		if(ps==NULL){
			ps=malloc(sizeof(tStud));	// Malloc Speicher aus dem heap
			if (ps) {puts("malloc hat nicht geklappt."); exit(-1);}
		} else {
			psx=realloc(ps,(anz+1)*sizeof(tStud));
			// realloc guckt nach, ob dahinter noch genug Speicher frei ist und
			// fügt das neue alloc hinten dran. Wenn nicht, dann wird ein neuer Speicherbereich
			// gefunden, der groß genug ist für beide ist.
			if (psx){
				ps=psx;
			} else { puts("realloc hat nicht geklappt."); exit(-1);}
		}
		
		*(ps+anz) = s;
		anz++;
		for (i=0; i<anz; i++) {
			putStudp(ps+i);
			printf("i: %d, ps+i: %p, sizeof: %d %02x\n",i, ps+i, sizeof(tStud),sizeof(tStud));
		}
		printf("next (y/n): ");
		fgets(vbuf,128,stdin); 
		weiter=vbuf[0];
	}
	// am Ende des Programms oder wenn die Daten nicht mehr gebraucht werden
	// muss der Speicher wieder freigegeben werden
	free (ps);
	// free (psx); darf nicht freigegeben werden, da er nur ein Zeiger auf den ps ist.
	return 0;
}
\end{lstlisting}
Verwendung von malloc/realloc:\\
Speicher nach Bedarf aus dem heap.\\
malloc hat als Ausgabe void* (generischer Pointer) [siehe "man malloc"]. Dieser ist nicht derefernzierbar und zuweisungskompatibel zu jedem getypten Pointer. Man kann mit ihm ebenfalls nicht rechnen (keine Arithmetik).\\
malloc: Speicher für Variable frei machen\\
realloc: freigemachten Speicher erweitern\\
free: Speicher wieder freigeben\bigskip\\
Es kann bspw. auch Speicher in getStud2 (z.B. für den Namen, also *name anstatt name mit entsprechendem alloc in der Fkt.) allociert werden, um Platz zu sparen (oder dynamisch mehr freizugeben). Dann müsste der Speicher entsprechend in einer Schleife auch wieder freigegeben werden.

\chapter{Dateiarbeit in C}
\begin{lstlisting}[language=C]
FILE * pf;
pf=fopen("myFile.txt","rt"); // ("Dateiname inkl. Pfad", "Modus")
\end{lstlisting}
\begin{tabular}{l | l}
b/t& Texdatei ([t]) / Binärdatei(b)\\
\hline 
r&Zum Lesen öffen\\
w&Zum Scheiben (überschreiben) öffnen, ggf. erzeugen\\
a&Zum Schreiben am Dateiende öffnen, ggf. erzeugen\\
r+&Zum Lesen und Schreiben öffnen (ändern)\\
w+&Zum Lesen und Schreiben öffen (überschreiben), ggf. erzeugen\\
a+&Zum Anfügen, Lesen, Erzeugen öffnen, ggf. erzeugen\\
\end{tabular}
\paragraph{Als Binärdatei speichern}
aus tStud Beispiel(st1.c) (jeweils an den entsprechenden Stellen eingefügt):
\begin{lstlisting}[language=C]
#include "st1io.h"
...
FILE *pf;

int main(){
	int i;
	tStud* ps = NULL, *psx;
	pf = fopen("Studs.bin","rb");
	if(pf){
		// Dateigröße ermitteln:
		fseek(pf,0,SEEK_END);
		anz=ftell(pf)/sizeof(tStud);
		rewind(pf);
		
		// Daten lesen
		for (i=0; i<anz; i++){
			if(ps==NULL){
				ps=malloc(sizeof(tStud));
				if (ps==NULL){puts("malloc hat nicht geklappt"); exit(-1);}
				*ps=readStud(pf);
			} else {	
				psx=realloc(ps,(anz+1)*sizeof(tStud));
				if (psx){
					ps=psx;
					*(ps+i)=readStud(pf);
				} else { puts("realloc hat nicht geklappt."); exit(-1);}
			}
		}
		fclose(pf);
	}
...
	// Daten speichern
	pf=fopen("Studs.bin","wb");
	if(pf==NULL){puts("fopen (write) hat nicht geklappt");exit(-1);
	for (i=0; i<anz; i++){
		writeStud(pf,ps+i);
	}
	fclose(pf);
...
\end{lstlisting}

st1io.h
\begin{lstlisting}[language=C]
#include <stdio.h>
#include "st1f.h"
#include "st1io.h"

tStud readStud(FILE* f){
	tStud s;
	fread(&s, sizeof(tStud),1,f);
	return s;
}

void write Stud(File* f, tStud* pStud){
	fwrite(pStud, sizeof(tStud),1,f);
}
\end{lstlisting}
ACHTUNG: Arbeiten mit derartigen Binärdateien ist sehr Platform-Abhängig (angucken der Binärdatei unter linux mit "hexdump -C ...")

\paragraph{Als Textdatei speichern} (alternativ und ergänzend zum obigen Beispiel)\\
st1io.h
\begin{lstlisting}[language=C]
int getAnz(FILE* pf);
tStud readStud(FILE* pf);
void writeStud(FILE* pf, tStud* ps);
\end{lstlisting}

st1io.c
\begin{lstlisting}[language=C]
#include "st1f.c"

// Zeilen zählen und durch 4 teilen
int getAnz(FILE *pf){
	char buf[128];
	int n=0;
	while (fgets(buf, 128, pf)) n++;\\
	fseek(pf, 0, SEEK_SET);
	return n/4;
}

tStud readStud(FILE* pf){
	tStud s={};
	char buf[128];
	if(fgets(buf, 128, pf)){
		buf[strlen(buf)-1]=0;
		s.name = malloc(strlen(buf+1));
		if (s.name) strcpy(s.name,buf);
		else fprintf(stderr, "malloc faild in readStud\n");
		fscanf(pf,"%d\n%d\n%f\n",&s.matrNr,&s.belNote,&s.klNote);
	}
	return s;
}

void writeStud(FILE *pf, tStud* ps){
	pfrintf(pf,"%s\n%d\n%d\n%f\n",ps->name, ...
}


\end{lstlisting}

st1.c
\begin{lstlisting}[language=C]
...
	// Einlesen aus Datei
	pf=fopen("Studs.txt", "rt");
	if (pf){
		anz=getAnz(pf);		
...
	// Schreiben
	pf=fopen("Studs.txt", "wt");
		
\end{lstlisting}

\paragraph{Als CSV-Datei speichern}

Ähnlich wie bei .txt, bloß trennt man die Datensätze durch Semicolon und nicht durch neue Zeile.\\
Sinnvolle Funktion zur Zerlegung der Datensätze: 
\begin{lstlisting}[language=C]
strtok(buf,";\n");
\end{lstlisting}

\chapter{Funktionspointer}

Bisher wurde typedef für Variablen verwendet:
\begin{lstlisting}[language=C]
typedef struct{
	int sk1;
	int sk2;
} tStruct;
tStruct s={1,9999};
\end{lstlisting}
Nun kommt der Funktionspointer dazu:
\begin{lstlisting}[language=C]
typedef void f(void);
f* pf;	// Das ist der Funktionspointer
\end{lstlisting}
oder auch:
\begin{lstlisting}[language=C]
typedef void (*tpf)(void);
tpf pf;	// Das ist auch ein Funktionspointer
\end{lstlisting}
Anwendungsbeispiel:
\begin{lstlisting}[language=C]
void fxyz(void){
	puts("xyz");
}
pf = fxyz;

fxyz();
pf();	// diese beiden Funktiosaufrufe rufen die gleiche Funktion auf - ein mal dirket 
			//und ein mal als Pointer
\end{lstlisting}
\paragraph{Beispiel:} $ $
\begin{lstlisting}[language=C]
#include <stdio.h>

typedef void (*tpf)(int i);
tpf pFunc;

void printDec(int i);
void printHex(int i);

typedef struct{
	tpf func;
	int x;
}tstruct;

tstruct tdec={printDec,55};
tstruct thex={printHex,55};

void printDec(int i){
	printf("%d\n",i);
}

void printHex(int i){
	printf("%x\n",i);
}

// alternativ: printx(void (*func)(int i), int z){...}
void printx(tpf func, int z){
	func(z);
}
// Bezeichnung der Technik: Übergabe einer Callback Funktion
// Verwendung bei Programmierung mit graphischen Oberflöchen

tpf choice(char c){
	if (c=='h')	return printHex;
	else 				return printDec;
}

int main(int argc, char*argv[]){
	int x=9999;
	tpf aFctns[]={printDec,printHex};
	if (argc==2) x=atoi(argv[1]);
	
	pFunc=printDec;
	puts("Dezimal:");
	pFunc(x);
	pFunc=printHex;
	puts("Hexadezimal:");
	pFunc(x);
	
	puts("Dezimal aus Array:");
	aFctns[0](x);
	puts("Hexadezimal aus Array:");
	aFctns[1](x);
	
	puts("Dezimal aus Funktionsparameter:");
	printx(printDec,x);
	puts("Hexadezimal aus Funktionsparameter:");
	printx(printHex,x);
	
	puts("aus Struct:");
	tdec.func{tdec.x};
	thex.func{thex.x};	
	
	puts("aus Returnwert:");
	choice('h')(x);		// choice('h')()=printHex()
	choice('d')(x);		
		
	return 0;
}
\end{lstlisting}

Verwendungszweck beispielsweise im Compilerbau im Lexer oder ähnliches.

\chapter{Graphische Oberflächen}
\section{libforms}
FOLIE
\section{GTK}
FOLIE
\section{CGI}
FOLIE\\
Bsp. ascii.cgi

\chapter{Preprozessor}

\section{Umwandlung von 3-Zeichenfolgen}
(siehe Grundelemente): Bspw. ??> o.ä.\\
Braucht extra Befehl für gcc.

\section{Zeilenverlängerung}
Falls eine Zeichenkette im Editor mit Zeilenumbruch getrennt werden soll (bspw. ein langer String). Kann auch mitten im C-Code sein.\\
Das Backslash muss dafür das letzte Zeichen auf der Quelltextzeile sein.
\begin{lstlisting} [language=C]
#include <stdio.h>

int ma\
in(){
	puts{Spass\
	beim Programmieren\
	in c");
	return 0;
}
\end{lstlisting}

\section{include}

Der Preprozessor ersetzt die Include-Zeile mit dem Quelltext des einzubindenden Elements.
\begin{lstlisting}[language=C]
#include <...>	// für Systemheaderfiles (zu finden unter usr/include)
#include "..."	// für Applikationsheaderfiles (eigene)
\end{lstlisting}
Das Ergebnis des Procompilers kann mit \emph{gcc -E progr.c > progr.e} kompiliert werden.\\
Mehrfaches Einbinden u.ä. kann unterbunden werden durch:
\begin{lstlisting}[language=C]
#ifdefn _H_DEBUG_
#define _H_DEBUG_

#include <stdio.h>

#endif
\end{lstlisting}

\section{Symboldefinitionen}
Vorteile:
\begin{itemize}
\item Quelltext wird lesbarer (durch gute zweckgebundene Bezeichnung)
\item Symboldefinition kann im gesamten Quelltext mit einem Mal geändert werden.
\end{itemize}
\begin{lstlisting}[language = C]
// Allgemein:
#define SYMBOL Tokensequenz		// Symbol wird üblicher Weise groß geschrieben,
															// muss aber nicht (nur zur besseren Wiedererkennung).

// Beispiel:
#define N 10
...
int inArray[N];	// Preprozessor ersetzt N mit 10

#define LEN 30 + 1
// KLAMMERN SETZTEN! ->
// #define LEN (30 + 1)
...
char name[LEN*3];	// Achtung: LEN wird vorm Ausrechnen ersetzt. 
									// Also: LEN*3 entspricht 30+1*3 und nicht (30+1)*3
strcpy(name, "Max");
puts(name);
printf("Len name: %d, Len content: %d\n", sizeof name, strlen(name));

#define NAME1 Klaus
...
puts(NAME1);
\end{lstlisting}

Darauf ist zu achten:
\begin{itemize}
\item Tokensequenz (bei Zahlen) am besten Klammern.
\item kein Semikolon!
\item define-Konstruktion muss auf einer Zeile stehen.
\end{itemize}

\section{Parameterbehaftete Macros}

\begin{lstlisting} [language=C]
#define SYMBOL(<parameterlist>)	Ersatztokenfolge
// Wichtig: Runde Klammer muss unmittelbar hinter SYMBOL stehen
#define SYMBOL(x) "str1" #x "str2"
// Bewirkt Verkettung der Strings mit dem Parameter
#define SYMBOL(x,y) x##y
// Ein neues Token ensteht im C-Quelltext aus x und y
\end{lstlisting}
Beispiel:
\begin{lstlisting}[language=C]
#include <stdio.h>
#include <string.h>
#include <math.h>
#define LEN (30+1)
#define PYTHAGORAS(a,b) sqrt(a*a + b*b)
// Klammern wieder wichtig:
// #define PYTHAGORAS(a,b) sqrt((a)*(a) + (b)*(b))
#define STR1(x) "Max " #x " Moritz"
// Raute vor dem Parameter bewirkt das Zusammenfassen der umschlossenen Strings.
// (wird zu "Max " "und" " Moritz")
#define STR2(x,y) x##y
#define MYPUTS puts

int main(){
	double x=3;
	double y=4;
	printf("c:  %lf\n",PYTHAGORAS(x,y));	
	printf("c:  %lf\n",PYTHAGORAS(x+1,y+1));	
	// wenn nicht geklammert: PYTHAGORAS(x+1,y+1) = sqrt(x+1*x+1+y+1*y+1)
	// also sqrt(2x+1+2y+1) -> Falsch!
	printf("c:  %lf\n",PYTHAGORAS(x++,y++));
	// auf keinen Fall machen, da es schlecht überschaubar wird, wann inkrementiert wird!
	printf("c:  %lf\n",PYTHAGORAS(x,funktion(y)));
	// auch gefährlich, weil der Macro mit funktion(y)*function(y) ersetzt wird
	// und damit die funktion (unerwünscht) doppelt ausgeführt wird
	
	char name[LEN*3];
	strcpy(name, STR1(und));
	puts(name);

	STR2(MY,PUTS)("Test");	// ergibt MYPUTS("Test"), was dann mit puts() ersetzt wird.
	
	return 0;
}
\end{lstlisting}
Erinnerung:\\
bei \emph{\#\. include math.h} muss mit \emph{gcc progr.c -lm} kompiliert werden!\\
Darauf ist zu achten:
\begin{itemize}
\item Parameter im Macro klammern
\item keine Seiteneffekte programmieren (Inkrementierung, Funktionsaufruf usw.)
\end{itemize}
Liste von vordefinierten Symbolen zu finden durch \emph{gcc -E -dM - </dev/null}.\\
Bspw.:
\begin{lstlisting} [language=C]
__FILE__ // Quelltext-Dateiname
__LINE__ // Zeilennummer, in der das Symbol verwendet wurde
__DATE__ // Datum, zu dem das Programm kompiliert wurde
__TIME__ // Zeit, zu der das Programm kompliert wurde
__STDC__ // C-Version
\end{lstlisting}

\section{Bedingte Übersetzung}

Soll ein platformübergreifendes Programm geschrieben werden (mit einem Quelltext), gibt es Tücken (bspw. 32 oder 64 bit). Dann können bestimmte Fälle mit einer bedingten Übersetzung abgefangen werden.
\begin{lstlisting}[language=C]
#if <const-expr>
#if defined <symbol>
#ifdef <symbol>
#if !defined <symbol>
#ifndef <symbol>
#else 
#elif <const-expr>
#endif
\end{lstlisting}

\begin{lstlisting}[language=C]
#define MYPUTS puts
...
#ifdef MYPUTS
	MYPUTS("Text");
#else
	printf("Text2");
#endif
\end{lstlisting}
Symbole können mit der Commandlineoption \emph{-D} auch definiert werden.\\
\emph{gcc progr.c -D MYPUTS=puts} (entspricht der Zeile \emph{\#\. define MYPUTS puts}). \medskip\\
Gutes Anwendungsbeispiel für bedingte Übersetzung:
\begin{lstlisting} [language=C]
#ifdef DEBUG
	printf("Debuginformation");
#endif
\end{lstlisting}
Gibt Debug-Information nur bei \emph{gcc progr.c -DDEBUG} aus.

\chapter{Funktionen mit variabler Argumentliste}

vgl.: printf() mit beliebig vielen Argumenten abhängig von \% d usw. im ersten String.
\begin{enumerate}
\item wenigstens ein fester Parameter
\item dann folgt, … \\
(Es können also weitere Parameter folgen. Typ und Anzahl der Parameter ist unbekannt.)
\end{enumerate}

Macros zum Umgang mit variabler Argumentliste:
\begin{lstlisting}[language=C]
#include <stdarg.h>	// Voraussetzung für variable Argumentlisten

va_start(ap, la);	
// ap: Argument Pointer ( va_list ap; )
// la: Last Argument (letzter Parameter mit Typ und Namen vor ...)
printf("Name: %s, MatrNr: %d \n", Stud.name, Stud.MatrNr); 
\end{lstlisting}
\begin{tikzpicture}[scale=.4]
\draw  (0,4) rectangle (6,2)node[pos=.5]{MatrNr.};
\draw  (0,0) rectangle (6,2)node[pos=.5]{Name};
\draw  (6,-2) rectangle (0,0);
\draw  (12,1) rectangle (18,-1)node[pos=.5]{Max Otto $\emptyset$};
\draw [-latex] plot[smooth, tension=.7] coordinates {(5,1) (7.5,1) (10,0) (12.5,0)};
\draw [-latex] (7,-1) node[right]{ap} -- (5,-1);
\draw [-latex] plot[smooth, tension=.7] coordinates {(7,-0.5) (7.25,0) (7,0.5)};
\draw [-latex, dashed] plot[smooth, tension=.7] coordinates {(7.5,0) (10,-1.5) (13.5,-4.5)} node[right, align=left]{ap wandert weiter\\ bei $va\_arg()$};
\draw [-latex] (2,-1) -- (-6,-1) -- (-6,0.5) node[above]{printf("Name...};
\fill (2.5,-1) circle[radius=.2];
\end{tikzpicture}
\ContinueLineNumber
\begin{lstlisting}[language=C]
x=va_arg(ap, type);
// x: Zielvariable für den Wert des Arguments 
//		(muss vom Typ des tatsächlichen Parameters sein)
// ap wird um sizeof(type) erhöht
// dieser Wert wird x zugewiesen
\end{lstlisting}

\ContinueLineNumber
\begin{lstlisting}[language=C]
va_end(ap);
\end{lstlisting}
\StartLineAt1

\paragraph{Beispiel:}
myprintf.c
\begin{lstlisting}[language=C]
//#include <stdio.h>	// Kann auskommentiert werden, da entsprechende Funktionen
										// in recout.h nachprogrammiert wurden.
#include <stdarg.h>
#include <recout.h>	// Andere my...-Funktionen in Arg.tgz
										// als Systemheader-file, mit Befehlen aus librecout.c rein kopiert
#define xprintf(x) myprintf x

void myprintf(const char* fmt, ...){ // fmt bspw. ("Programm: %s\n", argv[0])
	va_list ap;
	char		*p;
	char		*pstr;
	char		cval;
	int			ival;
	double	dval;
	
	va_start(ap,fmt);
	
	for (	p=fmt; // p auf Anfang des Formatsteuerstrings
				*p;
				p++){
		if (*p!='%'){
			myputc(*p);
		} else {
			switch(*++p){
				case 's':
					pstr=va_arg(ap,char*);
					myputs(pstr);
					break;
				case 'd':
					ival=va_arg(ap,int);
					myputd(ival);
					break;
				case 'x':
					ival=va_arg(ap, int);
					myputx(ival);
					break;
				case 'd':
					dval=va_arg(ap, double);
					myputdouble(dval);
					break;
				case 'c':
					cval=va_arg(ap, int);	// char wird zu int geändert
					myputc(cval);
					break;
				case 'p':
					pstr=va_arg(ap, char*);
					myputx((long)pstr);	// muss gecastet werden
					break;
				default:
					myputc(*p);
					break;
			}
		}
	}
}

int main(int argc, char* argv[]){
	int i;
	if (argc>1)
		i = myatoi(argv[1]);
	else
		i = -1;
		
	double d=13.37;
	
	myprintf("Programm: %s\n int Value: %d\n", argv[0], argc);
	myprintf("i: %d, %x, DoubleVal: %f, Char: %c, 
						Adresse argv[0]: %p",i,i,d, argv[0][0], argv[0]);
						
	xprintf( ("Test xprintf: %s\n", argv[0]) );
	return 0;
}
\end{lstlisting}

Zum Kompilieren: Anlinken der Bibliothek: \emph{gcc myprintf.c -lrecout} (librecout.a wird zu lrecout). Include übernimmt nur Prototypen -- die komplierte Library muss so manuell übergeben werden (stdio-Library wird beim include automatisch angehängt).

In librecout.c:
\begin{lstlisting}[language=C]
// gcc -c librecout.c
// ar rvs librecout.a librecout.c
// sudo cp librecout.a /usr/local/lib/librecout.a
// sudo cp recout.h /usr/local/include
\end{lstlisting}

\chapter{CGI}
http://141.56.64.101/cgi-bin/cam.cgi\\
CGI Programm mit dem ein Text (über Telefon-Baustein) an ein entferntes anderes Telefon-Baustein gesendet wird (von dem das dan interpretiert wird)\bigskip\\
\emph{smslv.html} in \emph{cgi-bin}
\begin{lstlisting}[language=html]
<html>
<head>
	<title> sms cgi 27.01.2016 </title>
</head>
<body>
	<h3>snd sms from rasperry</h3>
	<form name "smsForm" action="smslv.cgi" METHODE="POST" id="smsform">
	<table cellpadding="4" cellspacing="0">
		<tr>
			<td style="width:100px"><label>to PhoneNr</label></td>			
			<td style="width:600px"><input type="text" name="smsToPhone"></input></td>
		</tr>
		<tr>
			<td style="width:100px"><label>Text</label></td>
			<td style="width:600px"><textarea rows="2" cols="80" 
						name="smsText" id="smsForm"></textarea></td>		
		</tr>
		<tr>
			<td></td>
			<td><input type="submit" value="Senden"></input></td>
		</tr>
	</table>
	</form>
</body>
</html>
\end{lstlisting}
\emph{smslv.c} in \emph{cgi-bin}
\begin{lstlisting}[language=C]
#include <stdlib.h>
#include <stdio.h>
#include <string.h>

FILE *f, *fsms;
char param[256];
char phone[20];
char text[256];
char buf[256];
char *p, *q;

#ifdef DEBUG
#define xprintf(x) printf x
#else 
#define xprintf(x)
#endif

int readText(char *p, char *q){
	int j;
	for (j=0; *p!=0; p++, j++)
	switch (*p){
		case '+': q[j]=' '; break;
		case '%': p++; q[j]=(char)strtol(p,&p,16); break;
		default: q[j]=*p; break;
	}
	q[j]=0;
	return j-1;
}

int main(){
	
	puts("Content-Type: text/html\n\n");
	fgets(param,256,stdin);	// html-Formulardaten landen in stdin
	if (strlen(param,)>2){
		xprintf(("Query: %s<br>", param))
		q=phone;
		// Telefonnummer holen:
		p=strstr(param,"smsToPhone=")+strlen("smsToPhone="); // strstr: suche im String
		// +strlen(): nicht auf den Anfang des Textes zeigend,
		// sondern auf die erste Zahl der Nummer (also nach dem Text)
		while(*p && *p!='&'){
			*q++ = *p++
		}
		xprintf(("Phone: %s<br>", phone));
		// Text holen:
		p=strstr(param,"smsText=")+strlen("smsText=");
		readText(p,text);
		xprintf(("Text: %s<br>", text);
		
		fsms=fopen("/var/spool/gnokii/sms.txt","wt");
		if (fsms){
			fprintf("%s\n%s", phone, text);
			fclose(fsms);
		} else{
			printf("Could not open sms.txt");
		}
	}
	f=fopen("smslv.html","rt");
	if (f) {
		while(fgets(buf,256,f))
			puts(buf);
	} else{
		puts("Could not open smslv.html");
	}
	return 0;
}
\end{lstlisting}

\end{document}