\begin{lstlisting}[language={[Sharp]C}]
// The topmost class in the .NET univeres: System.Object
namespace System{
	public class Object{
		public Object();	// Konstruktor, wird auch von Konstruktoren abgeleiteter Klassen gerufen
		public virtual Boolean Equals(Object obj);	// aktuelles Objekt und Objekt gleich?
		public static Boolean Equals(Object o1, Object o2); // Objekte o1 und o2 gleich?
		public virtual int GetHashCode();	// Eindeutiger Wert, override seitens abgeleiteter Klasse
		public Type GetType();	// liefert exakten Typ des Objektes zur Laufzeit
		pubilc virtual String ToString();	// gibt String zurück, der das aktuelle Objekt darstellt
		protected virtual void Finalize();	// gibt einem Objekt Gelegenheit zum Versuch, Ressourcen freizugeben, bevor Objekt vom Garbage Collector freigegeben wird
		protected Object MemberwiseClone();	// erstellt eine flache Kopie des aktuellen Objektes
		public static bool ReferenceEquals(object objA, object objB);	// stellt fest, ob die angegebenen Objekte identischen Speicherplatz belegen, d.h. nicht nur gleiche Werte besitzen. Aus ReferenceEquals(o1,o2) == true folgt immer Equals(o1,o2) == true bzw. o1.Equals(o2) == true. Umgekehrt gilt die Aussage jedoch nicht.
	}
}
\end{lstlisting}


%\begin{center}
%\includegraphics[scale=.25]{Vorlesung/a2GrundlagenC-img001.png} 
%\end{center}

\section{Intrinsische Datentypen}
{\begin{otherlanguage}{english}\footnotesize%
\begin{longtable}{L{.12} L{.125} L{.18} L{.28} L{.25}}
C\# Shorthand & CLS \newline Compliant? & System Type & Range & Meaning \\\hline
\lstinline$bool$ & yes & \lstinline$System.Boolean$ & \lstinline$true$ oder \lstinline$false$ & logical value \\
\lstinline$sbyte$ & no & \lstinline$System.SByte$ & $-128$ … $127$ & signed 8-bit number \\
\lstinline$byte$ & yes & \lstinline$System.Byte$ & $0$ … $255$ & unsigned 8-bit number \\
\lstinline$short$ & yes & \lstinline$System.Int16$ & $-32\,768$ … $32\,767$ & signed 16-bit number \\
\lstinline$ushort$ & no & \lstinline$System.UInt16$ & $0$ … $65\,535$ & unsigned 16-bit number \\
\lstinline$int$ & yes & \lstinline$System.Int32$ & $-2\,147\,483\,648$ … $2\,147\,483\,647$ & signed 32-bit number \\
\lstinline$uint$ & no & \lstinline$System.UInt32$ & $0$ … $4\;294\;967\;295$ & unsigned 32-bit number \\
\lstinline$long$ & yes & \lstinline$System.Int64$ & $-9\;223\;372\;036\;854\;775\;808$ … $9\;223\;372\;036\;854\;775\;807$ & signed 64-bit number \\
\lstinline$ulong$ & no & \lstinline$System.UInt64$ & $0$ … $18\;446\;744\;073\;\;709\;551\;615$ & unsigned 64-bit number \\
\lstinline$char$ & yes & \lstinline$System.Char$ & \lstinline$U+0000$ … \lstinline$U+ffff$ & single 16-bit Unicode character \\
\lstinline$float$ & yes & \lstinline$System.Single$ & $-3,4\cdot 10^{38}$ … $+3,4 \cdot 10^{38}$ & 32-bit floating-point number \\
\lstinline$double$ & yes & \lstinline$System.Double$ & $\pm 5,0 \cdot 10^{-324}$ … $\pm 1,7 \cdot 10^{308}$ & 64-bit floating-point number \\
\lstinline$decimal$ & yes & \lstinline$System.Decimal$ & $(-7,0 \times 10^{28}$ … $7,9\times 10^{28})$ / $10^{0 \text{ … }28}$ & \\
\lstinline$string$ & yes & \lstinline$System.String$ & Limited by system memory & Represents a set of Unicode characters \\
\lstinline$Object$ & yes & \lstinline$System.Object$ & Can store any data type in an object variable & The base class of all types in the .NET universe \\
\end{longtable}
\end{otherlanguage}}
%\begin{center}
%\includegraphics[width=.9\textwidth]{Vorlesung/a2GrundlagenC-img002.png}
%\end{center}
%\begin{center}
%\includegraphics[width=.9\textwidth]{Vorlesung/a2GrundlagenC-img003.png}
%\end{center}
\begin{center}
\includegraphics[width=.7\textwidth]{Vorlesung/a2GrundlagenC-img004.png}
\end{center}

\section{Boxing / Unboxing}

\emph{boxing}: Konvertierung (Einschachteln) eines Wertetyps in ein object (implizit)\\
\emph{unboxing}: Extraktion eines Wertetyps aus einem object (explizit)

\subsection*{Beispiel Boxing}

Boxing:
\begin{lstlisting}[language={[Sharp]C}]
Console.WriteLine("12.Equals(23) = \{0\}", 12.Equals(23));
Console.WriteLine("12.ToString() = \{0\}", 12.ToString());
Console.WriteLine("12.GetType() = \{0\}", 12.GetType());
\end{lstlisting}
Ausgabe:
\begin{lstlisting}[language={[Sharp]C}]
12.Equals(23) =
12.ToString() =
12.GetType() = 
\end{lstlisting}
Unboxing:
\begin{lstlisting}[language={[Sharp]C}]
object o = 123;
int i = (int) o;
\end{lstlisting}

\subsection{Member der intrinsischen Datentypen}

\begin{lstlisting}[language={[Sharp]C}]
int.MaxValue	// liefert größten int
int.MinValue	// liefert kleinsten int
double.Epsilon	// liefert kleinste darstellbare Zahl

char myChar = {}'a';
char.isDigit(myChar)	// liefert True wenn myChar eine Zahl ist
char.isLetter(myChar)	// liefert True wenn myChar eine Zahl ist
char.isWhiteSpace({}'Hallo du',5)	// steht an 5-ter Stelle ein Leerzeichen?

// Konvertierung von Strings:
bool b = bool.Parse(inputString);	// liefert True wenn inputString = "True"
double d = double.Parse(inputString);	// convertiert inputString in double
int i = int.Parse(inputString);	// convertiert inputString in int
\end{lstlisting}

\subsection{Arbeiten mit Strings}
\begin{otherlanguage}{english}
\begin{longtable}{L{.2} L{.7}}
String Member & Meaning\\\hline
\lstinline$Length$ & This property returns the length of the current string.\\
\lstinline$Compare()$ & This static method compares two strings.\\
\lstinline$Contains()$ & This method determines whether a string contains a specific substring.\\
\lstinline$Equals()$ & This method tests whether two string objects contain identical character data.\\
\lstinline$Format()$ & This static method formats a string using other primitives (e.g., numerical data, other strings) and the \lstinline${0}$ notation.\\
\lstinline$Insert()$ & This method inserts a string within a given string.\\
\lstinline$PadLeft()$\newline \lstinline$PadRight()$ & These methods ar used to pad a string with some characters\\
\lstinline$Remove()$\newline \lstinline$Replace()$ & Use these methods to receive a copy of a string with modifications (characters removed or replaced)\\
\lstinline$Split()$ & This method returns a String array containing the substrings in this instance that are delimited by elements of a specified char array or string array.\\
\lstinline$Trim()$ & This method removes all occurrences of a set of specified characters from the beginning and end of the current string.\\
\lstinline$ToUpper()$\newline \lstinline$ToLower()$ & These methods create a copy of the current string in uppercase or lowercase format, respectively.
\end{longtable}
\end{otherlanguage}

%\begin{center}
%\includegraphics[width=.9\textwidth]{Vorlesung/a2GrundlagenC-img005.png}
%\includegraphics[width=.9\textwidth]{Vorlesung/a2GrundlagenC-img006.png}
%\end{center}

\subsubsection*{Beispiel}
\begin{lstlisting}[language={[Sharp]C}]
string s = Console.ReadLine();	// z.B. "Hallo Welt"
s = s.Replace("Hallo", "Tschüss");
Console.WriteLine("Modified string: \{0\}", s);	// Tschüss Welt
s = s + " ...";	// string concatenation
Console.WriteLine("Modified string: \{0\}", s);	// Tschüss Welt ... 
\end{lstlisting}

\subsection{Textformatierung}
%\includegraphics[scale=.35]{Vorlesung/a2GrundlagenC-img007.png} 
\lstinline$Console.WriteLine("i = {0:F8}", i);$
\subsubsection*{Allgemein}
\lstinline${index[,breite][::formatierung]}$
\begin{itemize}
\item \lstinline$index$:\\
Laufende Nummer des Ausdrucks (von 0 an 0(1)…)
\item \lstinline$breite$:\\
Anzahl Zeichen für Ergebnis der Formatierung:
\begin{itemize}
\item $>0$: Rechtsbündig
\item $<0$: Linksbündig
\end{itemize}
wenn fehlt: kürzeste Ganzzahl (Länge)
\item \lstinline$formatierung$:\\
\begin{tabular}{l l l}
Zeichen & Beschreibung & Defaultergebnis\\\hline
\lstinline$C,c$ & Währung & \euro{}, \$ xx,xx.xx\\
\lstinline$D,d$ & dezimal & [-] xxxx \\
\lstinline$E,e$ & exponentiell & [-] x,xxxxE$\pm$xxx \\
\lstinline$F,f$ & Festkomma & [-] xxxx,xx \\
\lstinline$G,g[E,e]$ & exponentiell & variabel \\
\lstinline$N,n$ & Nummer mit Trennzeichen & [-] xx,xxx,xx \\
\lstinline$X,x$ & hexa, ganz & variabel \\
\end{tabular}
\end{itemize}
\subsection{Escape Characters}
\begin{otherlanguage}{english}
\begin{longtable}{L{.2} L{.7}}
Character & Meaning \\\hline
\lstinline$\'$ & Inserts a single quote into a string literal.\\
\lstinline$\"$ & Inserts a double quote into a string literal.\\
\lstinline$\\$ & Inserst a backslash into a string literal. This can be quite helpful when defining a file or network paths.\\
\lstinline$\a$ & Triggers a system alert (beep). For console programs, this can be an audio clue to the user.\\
\lstinline$\n$ & Inserts a new line (on Windows platforms).\\
\lstinline$\r$ & Inserts a carriage return.\\
\lstinline$\t$ & Inserts a horizontal tab into the string literal.
\end{longtable}
\end{otherlanguage}
%\begin{center}
%\includegraphics[width=.9\textwidth]{Vorlesung/a2GrundlagenC-img008.png}
%\end{center}

\subsection{Widening / Narrowing}

Im folgenden Code nimmt die Add Methode nur Argumente vom Typ int, bekommt aber short übergeben. Was passiert? Welcher Teil des Codes könnte Probleme geben?
%%% TODO
\begin{lstlisting}[language={[Sharp]C}]
public static void Main(string[] args){
	short a = 10, b = 10;	
	short c = Add(a, b);
	Console.WriteLine("{0} + {1} = {2}", a, b, c);
}

public static int Add(int a, int b){
	return a + b;
}
\end{lstlisting}[language={[Sharp]C}]

\subsection{Schlüsselwort var}
Typen werden automatisch aus der anfänglichen Zuweisungen geschlussfolgert.\\
$\to$ implizite Typisierung

\subsubsection*{Beispiel}
\begin{lstlisting}[language={[Sharp]C}]
var myInt = 0;
var myBool = true;
var myString = "Time, marches on ... ";
Console.WriteLine("myInt is a: {0}", myInt.GetType().Name);
Console.WriteLine("myBool is a: {0}", myBool.GetType().Name);
Console.WriteLine("myString is a: {0}", myString.GetType().Name);
\end{lstlisting}

~

\section{Schleifen}

Wie in C\# gibt es die folgenden Schleifen:
\begin{itemize}
\item \lstinline$do {...} while (...)$
\item \lstinline$while (...) {...}$
\item \lstinline$for (...; ...; ...) {...}$
\end{itemize}
Zusätzlich gibt es die foreach Schleife:
\begin{itemize}
\item \lstinline$foreach$ (Typ element in Menge)
\item jedes Element einer Menge wird genau 1 mal durchlaufen
\item Die Menge kann alles sein was das Ienumerable Interface implementiert (Array, Liste, …)
\item entspricht einem Iterator in C++
\item durchlaufene Elemente können nicht verändert werden
\item Beispiel:
\end{itemize}
\begin{lstlisting}[language={[Sharp]C}]
string[] carTypes = {"Ford", "BMW", "Trabbi", "Honda"};
foreach (string c in carTypes) 
Console.WriteLine(c);
\end{lstlisting}

%\includegraphics{Vorlesung/a2GrundlagenC-img009.png} 

\section{Garbage Collector}

\begin{itemize}
\item Objekte benötigen Speicher im Heap (Speicherallokation mit new)
\item werden Objekte nicht mehr benötigt sollte der Speicher wieder freigegeben werden
\item Bsp: 
\begin{lstlisting}[language={[Sharp]C}]
Circle kreis1 = new Circle();
Circle kreis2 = new Circle();
[...]
kreis1 = null2;
\end{lstlisting}
\item Speicherfreigabe ist in C\# nicht explizit möglich, sondern geschieht durch den Garbage Collector
\end{itemize}

\subsubsection*{Funktionsweise}
\begin{center}
\includegraphics[scale=.7]{Vorlesung/a2GrundlagenC-img010.png}
\end{center}
Garbage Collector wird ausgelöst:
\begin{enumerate}
\item wenn Programm wenig ausgelastet
\item wenn Speicherressurcen knapp
\item bei Programmende
\item manuell mit \lstinline$GC.collect();$
\end{enumerate}