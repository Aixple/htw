\section{Einführung}
\subsection{Erhöhung der Rechenleistung: Ansätze}
\slides{07}{1}

\subsection{Superskalar Definition}
\slides{07}{2}
\subsection{Parallele Strukturen bei Einprozessorsystemen}
\slides{07}{3}

\section{Superpipeline-Architektur (Superpipelining)}
\slides{07}{4}

\subsection{Diskussion}
\slides{07}{5}

\section{VLIW-Architektur}
\slides{07}{6}
\slides{07}{7}

\unimptnt{
\subsection{EPIC-Mikroarchitekturen}
\slides{07}{8}
}

\subsection{Probleme}
\slides{07}{9}

\section{Superskalar-Architektur}

\subsection{Architektur}
\slides{07}{10}

\subsubsection*{Gemeinsame Pipeline}
\slides{07}{11}

\subsubsection*{Superskalare Prozessor-Pipeline}
\slides{07}{12}
\unimptnt{
\slides{07}{13}
}

\subsubsection{Superskalarität: Scheduler}
\slides{07}{14}

\unimptnt{
\subsubsection{Superskalar-Architektur}
\slides{07}{15}
}

\subsubsection[Superpipeline <-> Supersklare Architektur]{Superpipeline $\leftrightarrow$ Supersklare Architektur}
\slides{07}{16}

\subsubsection{Superpipeline \& Superskalar kombiniert}
\slides{07}{17}

\subsubsection{Zusammenfassung}
\slides{07}{18}

\subsection{Befehlszuteilung}
\slides{07}{19}

\subsubsection{statisch (In-Order-Execution)}
\slides{07}{20}
\slides{07}{21}

\subsubsection{dynamisch (Out-Of-Order-Execution)}
\slides{07}{22}
\slides{07}{23}
\unimptnt{
\subsubsection*{Teilaufgaben}
\slides{07}{24}
\slides{07}{25}
}

\subsubsection{dynamisch: Scoreboard}
\slides{07}{26}

\unimptnt{
\subsubsection*{Ablaufüberwachung}
\slides{07}{27}
}

\subsubsection{dynamisch: Tomasolu-Methode}
\slides{07}{28}