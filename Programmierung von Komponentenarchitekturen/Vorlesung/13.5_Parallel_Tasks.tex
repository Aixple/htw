\begin{itemize}
\item neu in .NET 4.0
\item Im Zentrum steht die neue Klasse Task
\end{itemize}

\subsection{Tasks vs Threads}

\begin{itemize}
\item zu viele Threads sind schlecht:
\begin{itemize}
\item OS ist beschäftigt mit Scheduling.
\item jeder Thread bindet Speicher (im MB Bereich).
\end{itemize}
\item Tasks können Threads erzeugen (max 1 Thread pro Task), wenn sie es für notwendig erachten, Concurrency Runtime sorgt für optimale Lastverteilung.
\item Typische Probleme beim gleichzeitigen Zugriff auf Ressourcen (Race Conditions, Deadlock, …) bleiben bestehen, müssen umgangen werden.
\end{itemize}

\subsection{Laufzeitvergleich Task / Thread}

Aufgabe: 64 Listen mit 2\,000\,000 Elementen sortieren

\begin{itemize}
\item 64 Threads: 14 sec (CPU Auslastung 60 \%)
\item 64 Tasks: 3 sec (CPU Auslastung 99 \%)
\end{itemize}
%%% TODO

Neue thread-sichere Datenstrukturen:

\begin{itemize}
\item \lstinline`ConcurrentStack<T>`
\item \lstinline`ConcurrentQueue<T>`
\item \lstinline`BlockingCollection<T>`
\end{itemize}

%%% TODO

\subsection{Die Klasse Task}

\begin{itemize}
\item Anlegen und Verwendung ähnlich wie Thread
\item wichtigste Zustände eines Task:
\end{itemize}
\begin{tabular}{l l l}
Task definieren &
$\to$ &
\lstinline`Created`\\
Task starten   &
$\to$ &
\lstinline`WaitingToRun`\\
.NET startet den Task &
$\to$ &
\lstinline`Running`\\
Task ohne Fehler abgearbeitet &
$\to$  &
\lstinline`RanToCompletion`\\
\end{tabular}

\subsubsection{Task anlegen}

\begin{itemize}
\item mit lambda Functionen
\begin{lstlisting}[language={[Sharp]C}]
Task t = new Task ( ()=> { /* ... */ } );
Task t = new Task ( (o) => { /* ... use o here */ }, /* value for o */ );
\end{lstlisting}
\item ohne lambda Funktion
\begin{lstlisting}[language={[Sharp]C}]
Task t = new Task( action );

Action<object> delegate = new Action<object>(DoWork);
Task t = new Task ( delegate, objectValue );

void DoWork( object o ){ /* ... */ }
\end{lstlisting}
\end{itemize}

\subsubsection{Task starten}

\lstinline`Task.Start();`

\subsubsection{Task.Factory}

\lstinline`Task.Factory.StartNew( gleiche Parameter wie Konstruktor von Task ); `\\
(startet den Task auch gleich, funktioniert ohne Konstruktor)

\subsubsection{Rückgabewert eines Tasks}

\begin{lstlisting}[language={[Sharp]C}]
// <int> ist der Typ des Rückgabewerts
Task<int> t = Task.Factory.StartNew(() => {
	Thread.Sleep(2000); 
	return 13; 
}); 
int i = t.Result;
// Code wartet, bis Task fertig ist und Zugriff auf Result möglich ist.
\end{lstlisting}

\subsubsection{Auf einen Task warten}
\lstinline`t1.Wait();`

\subsubsection{Auf mehrere Tasks warten}
\lstinline`Task.WaitAll(Task[]);` \lstinline`Task.WaitAny(Task[]);`\\
z.B. warten auf einen Task: \lstinline`Task.WaitAny(new Task[] {t1, t2 } );`

\subsubsection{Parallele Schleifen}

\begin{itemize}
\item Die TPL stellt die Methoden \lstinline`Parallel.For` und \lstinline`Parallel.ForEach` zur Verfügung.
\item intelligenter Aufteilungs-Algorithmus zerlegt Datenmenge, bzw. Schleifen-Interationen und erstellt selbstständig Tasks.
\end{itemize}
Beispiel:
\begin{lstlisting}[language={[Sharp]C}]
// wie for (int i=0; i<5; i++):
Parallel.For(0,  5,  i => {
	Console.WriteLine("Durchlauf mit i = {0}", i);
	Thread.Sleep(1000);
});

Parallel.ForEach(new int[] {1, 2, 3, 4, 5}, item => {
	Console.WriteLine("Durchlauf mit i = {0}", item);
	Thread.Sleep(1000);
});
\end{lstlisting}
Wie könnte man die folgende Schleife parallelisieren ? 
\begin{lstlisting}[language={[Sharp]C}]
double result = 0;
for (int i = 0; i < vec1.GetLength(0) ; i++) {
	result += vec1[i] * vec2[i];
}
\end{lstlisting}
Lösung:
\begin{lstlisting}[language={[Sharp]C}]
Parallel.For(0, vec.GetLength(0), i=> {
	result += vec1[i] * vec2[i];
});
\end{lstlisting}
Problem wieder ähnlich wie zuvor, dass Variable gelockt werden muss. Dafür gibt es dann besser lokale Schleifenvariablen: 
\subsubsection{Lokale Schleifenvariablen}

Jeder einzelne Thread hat eine eigene Instanz der Variable

\begin{itemize}
\item keine Sperre nötig
\item am Ende können die lokalen Variablen zusammengeführt werden
\end{itemize}

\begin{lstlisting}[language={[Sharp]C}]
// <TLocal>: Typ der lokalen Schleifenvariable
Parallel.For<TLocal> (
	Int32, // Startwert
	Int32, // Endwert
	Func<TLocal>,	// Initialisierung der lokalen Schleifenvariable
	Func<Int32, ParallelLoopState, TLocal, TLocal>,	// Iterationsvorschrift
	Action<TLocal> // Abschluss, nutze lokale Variable um globale Variable zu updaten
) 
\end{lstlisting}
Beispiel: 
\begin{lstlisting}[language={[Sharp]C}]
Object myLock = new Object();
Parallel.For<double>(0, n, () => {return 0; }, 
	(i, pls, localSum) =>  {
		localSum += vec1[i] * vec2[i];
		return localSum;
	},
	x =>  {
		lock (myLock) result += x;
	}
);
\end{lstlisting}

\subsubsection{Schleife abbrechen}
Nutzt das \lstinline`ParallelLoopState` Objekt das übergeben wird

\begin{itemize}
\item \lstinline`pls.Break();`\\
Abbruch nachdem alle Iterationen bis zur aktuellen durchgeführt wruden
\item \lstinline`pls.Stop();`\\
sofortiger Abbruch
\end{itemize}

\subsubsection{Parallelisierung von GUI Aktivitäten}

\begin{itemize}
\item Die Controls der GUI gehört zu genau einem Thread, andere Threads haben keinen Zugriff auf GUI Aktivitäten
\item wird ein Teil eines Programms auf einen anderen Thread ausgelagert, so müssen darin enthaltene GUI-Aktivitäten zurück an den GUI-Thread gegeben werden. 
\item Dies geschieht mit der Methode Invoke, die auf einem GUI Control aufgerufen wird
\end{itemize}
Beispiel:
\begin{lstlisting}[language={[Sharp]C}]
Task.Factory.StartNew( () => {
	// aufwendige Berechnung hier
	this.Invoke(new Action(() => {
		label.Location = new Point(200,200);
	}));
}
\end{lstlisting}
