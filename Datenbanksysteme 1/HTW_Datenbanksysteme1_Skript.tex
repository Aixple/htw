% Header aus der Vorlage
\documentclass[a4paper,11pt, footheight=26pt
%,twoside
]{scrreprt}
\usepackage[head=23pt]{geometry}	% head=23pt umgeht Fehlerwarnung, dafür größeres "top" in geometry
\geometry{a4paper, top=30mm, bottom=22mm,headsep=10mm, footskip=12mm
, left=20mm, right=20mm
%, inner=27mm, outer=13mm
}

% Zeile 2 (,twoside) und 7 (inner=...) für eine Druckversion (doppelseitig) ent-kommentieren (Rand für Hefter)

\setcounter{secnumdepth}{3}	% zählt auch subsubsection
\setcounter{tocdepth}{3}	% Inhaltsverzeichnis bis in subsubsection

% Input inkl. Umlaute, Silbentrennung
\usepackage[T1]{fontenc}
\usepackage[utf8]{inputenc}
\usepackage[ngerman]{babel}
\usepackage{csquotes}	% Anführungszeichen
\usepackage{eurosym}

% HTW Corporate Design: Arial (Helvetica)
\usepackage{helvet}
\renewcommand{\familydefault}{\sfdefault}

% Style-Aufhübschung
\usepackage{soul, color}	% Kapitälchen, Unterstrichen, Durchgestrichen usw. im Text
\usepackage{scrlayer-scrpage}	% Kopf-/Fußzeile
%\usepackage{titleref}
\usepackage[perpage]{footmisc}	% Fußnotenzählung Seitenweit, nicht Dokumentenweit
\renewcommand*{\thefootnote}{\fnsymbol{footnote}}	% Fußnoten-Symbole anstatt Zahlen
\renewcommand*{\titlepagestyle}{empty} % Keine Seitennummer auf Titelseite

% Mathe usw.
\usepackage{amssymb}
\usepackage[fleqn]{amsmath}	% fleqn: align-Umgebung rechtsbündig
\usepackage{xcolor}
\usepackage{esint}	% Schönere Integrale, \oiint vorhanden
\everymath=\expandafter{\the\everymath\displaystyle}	% Mathe Inhalte werden weniger verkleinert
\usepackage{wasysym}	% mehr Symbole, bspw \lightning
% Auch arcus-Hyperbolicus-Funktionen
\DeclareMathOperator{\arccot}{arccot}
\DeclareMathOperator{\arccosh}{arccosh}
\DeclareMathOperator{\arcsinh}{arcsinh}
\DeclareMathOperator{\arctanh}{arctanh}
\DeclareMathOperator{\arccoth}{arccoth} 
% Mathe in Anführungszeichen:
\newsavebox{\mathbox}\newsavebox{\mathquote}
\makeatletter
\newcommand{\mq}[1]{% \mathquotes{<stuff>}
  \savebox{\mathquote}{\text{"}}% Save quotes
  \savebox{\mathbox}{$\displaystyle #1$}% Save <stuff>
  \raisebox{\dimexpr\ht\mathbox-\ht\mathquote\relax}{"}#1\raisebox{\dimexpr\ht\mathbox-\ht\mathquote\relax}{''}
}
\makeatother

% tikz usw.
\usepackage{tikz}
\usepackage{pgfplots}
\pgfplotsset{compat=1.11}	% Umgeht Fehlermeldung
\usetikzlibrary{graphs}
%\usetikzlibrary{through}	% ???
\usetikzlibrary{arrows}
\usetikzlibrary{arrows.meta}	% Pfeile verändern / vergrößern: \draw[-{>[scale=1.5]}] (-3,5) -> (-3,3);
\usetikzlibrary{automata,positioning} % Zeilenumbruch im Node node[align=center] {Text\\nächste Zeile} automata für Graphen
\usetikzlibrary{matrix}
\usetikzlibrary{patterns}	% Schraffierte Füllung
\tikzstyle{reverseclip}=[insert path={	% Inverser Clip \clip
	(current page.north east) --
	(current page.south east) --
	(current page.south west) --
	(current page.north west) --
	(current page.north east)}
% Nutzen: 
%\begin{tikzpicture}[remember picture]
%\begin{scope}
%\begin{pgfinterruptboundingbox}
%\draw [clip] DIE FLÄCHE, IN DER OBJEKT NICHT ERSCHEINEN SOLL [reverseclip];
%\end{pgfinterruptboundingbox}
%\draw DAS OBJEKT;
%\end{scope}
%\end{tikzpicture}
]	% Achtung: dafür muss doppelt kompliert werden!
\usepackage{graphpap}	% Grid für Graphen
\tikzset{every state/.style={inner sep=2pt, minimum size=2em}}

% Tabular
\usepackage{longtable}	% Große Tabellen über mehrere Seiten
\usepackage{multirow}	% Multirow/-column: \multirow{2[Anzahl der Zeilen]}{*[Format]}{Test[Inhalt]} oder \multicolumn{7[Anzahl der Reihen]}{|c|[Format]}{Test2[Inhalt]}
\renewcommand{\arraystretch}{1.3} % Tabellenlinien nicht zu dicht
\usepackage{colortbl}
\arrayrulecolor{gray}	% heller Tabellenlinien
\usepackage{array}	% für folgende 3 Zeilen (für Spalten fester breite mit entsprechender Ausrichtung):
\newcolumntype{L}[1]{>{\raggedright\let\newline\\\arraybackslash\hspace{0pt}}m{\dimexpr#1\columnwidth-2\tabcolsep-1.5\arrayrulewidth}}
\newcolumntype{C}[1]{>{\centering\let\newline\\\arraybackslash\hspace{0pt}}m{\dimexpr#1\columnwidth-2\tabcolsep-1.5\arrayrulewidth}}
\newcolumntype{R}[1]{>{\raggedleft\let\newline\\\arraybackslash\hspace{0pt}}m{\dimexpr#1\columnwidth-2\tabcolsep-1.5\arrayrulewidth}}

% Nützliches
\usepackage{verbatim}	% u.a. zum auskommentieren via \begin{comment} \end{comment}
\usepackage{tabto}	% Tabs: /tab zum nächsten Tab oder /tabto{.5 \CurrentLineWidth} zur Stelle in der Linie
\NumTabs{6}	% Anzahl von Tabs pro Zeile zum springen
\usepackage{listings} % Source-Code mit Tabs
\usepackage{lstautogobble} 
\usepackage{enumitem}	% Anpassung der enumerates
\setlist[enumerate,1]{label=\arabic*.)}	% global andere Enum-Items
\newenvironment{anumerate}{\begin{enumerate}[label=\alph*.)]}{\end{enumerate}} % Alphabetische Aufzählung
\renewcommand{\labelitemiii}{$\scriptscriptstyle ^\blacklozenge$} % global andere 3. Item-Aufzählungszeichen
\usepackage{letltxmacro} % neue Definiton von Grundbefehlen
% Nutzen:
%\LetLtxMacro{\oldemph}{\emph}
%\renewcommand{\emph}[1]{\oldemph{#1}}

% Einrichtung von lst
\lstset{
basicstyle=\ttfamily, 
mathescape=true, 
%escapeinside=^^, 
autogobble, 
tabsize=2,
basicstyle=\footnotesize\sffamily\color{black},
frame=single,
rulecolor=\color{lightgray},
numbers=left,
numbersep=5pt,
numberstyle=\tiny\color{gray},
commentstyle=\color{gray},
keywordstyle=\color{green},
stringstyle=\color{orange},
morecomment=[l][\color{magenta}]{\#}
%showspaces=false,
showstringspaces=false,
breaklines=true,
literate=%
    {Ö}{{\"O}}1
    {Ä}{{\"A}}1
    {Ü}{{\"U}}1
    {ß}{{\ss}}1
    {ü}{{\"u}}1
    {ä}{{\"a}}1
    {ö}{{\"o}}1
    {~}{{\textasciitilde}}1
}
\usepackage{scrhack} % Fehler umgehen
\def\ContinueLineNumber{\lstset{firstnumber=last}} % vor lstlisting. Zum wechsel zum nicht-kontinuierlichen muss wieder \StartLineAt1 eingegeben werden
\def\StartLineAt#1{\lstset{firstnumber=#1}} % vor lstlisting \StartLineAt30 eingeben, um bei Zeile 30 zu starten
\let\numberLineAt\StartLineAt

% BibTeX
\usepackage[backend=bibtex, bibencoding=ascii]{biblatex}	% BibTeX
\usepackage{makeidx}
%\makeglossary
%\makeindex

% Grafiken
\usepackage{graphicx}
\usepackage{epstopdf}	% eps-Vektorgrafiken einfügen

% pdf-Setup
\usepackage{pdfpages}
\usepackage[bookmarks,%
bookmarksopen=false,% Klappt die Bookmarks in Acrobat aus
colorlinks=true,%
linkcolor=black,%
citecolor=red,%
urlcolor=green,%
]{hyperref}

% Titel, Autor usw. werden vor dem Anfang des Dokuments in einem Rutsch definiert…
\newcommand{\DTitel}[1]{\newcommand{\Dokumententitel}{#1}}
\newcommand{\DUntertitel}[1]{\newcommand{\Dokumentenuntertitel}{#1}}
\newcommand{\DAutor}[1]{\newcommand{\Dokumentenautor}{#1}}
\newcommand{\DNotiz}[1]{\newcommand{\Dokumentennotiz}{#1}}
\newcommand{\DSign}[1]{\newcommand{\Dokumentensignatur}{#1}}
\DSign{\footnotesize{\textcolor{darkgray}{Mitschrift von\\ \Dokumentenautor}}}
\newcommand{\Autorformat}[1]{\textcolor{darkgray}{Mitschrift von #1}}
\newcommand{\workingdir}{../}	% Arbeitsordner (in Abhängigkeit vom Master) Standard: LateX_master Ordner liegt im Eltern-Ordner
% … Deswegen folgendes erst Nach Dokumentenbeginn ausführen:
\AtBeginDocument{
	\hypersetup{
		pdfauthor={\Dokumentenautor},
		pdftitle={HTW Dresden | \Dokumententitel - \Dokumentenuntertitel},
	}
	\automark[section]{section}
	\automark*[subsection]{subsection}
	\pagestyle{scrheadings}
	\ihead{\includegraphics[height=1.7em]{\workingdir LaTeX_master/HTW-Logo.eps}}
	\ohead{\Dokumententitel}
	\cfoot{\pagemark}
	\ofoot{\Dokumentensignatur}
	% Titelseite
	\title{\includegraphics[width=0.35\textwidth]{\workingdir LaTeX_master/HTW-Logo.eps}\\\vspace{0.5em}
	\Huge\textbf{\Dokumententitel} \\\vspace*{0,5cm}
	\Large \Dokumentenuntertitel \\\vspace*{4cm}}
	\author{\Autorformat{\Dokumentenautor} \vspace*{1cm}\\\Dokumentennotiz}
}

%% EINFACHE BEFEHLE

% Abkürzungen Mathe
\newcommand{\EE}{\mathbb{E}}
\newcommand{\QQ}{\mathbb{Q}}
\newcommand{\RR}{\mathbb{R}}
\newcommand{\CC}{\mathbb{C}}
\newcommand{\NN}{\mathbb{N}}
\newcommand{\ZZ}{\mathbb{Z}}
\newcommand{\PP}{\mathbb{P}}
\renewcommand{\SS}{\mathbb{S}}
\newcommand{\cA}{\mathcal{A}}
\newcommand{\cB}{\mathcal{B}}
\newcommand{\cC}{\mathcal{C}}
\newcommand{\cD}{\mathcal{D}}
\newcommand{\cE}{\mathcal{E}}
\newcommand{\cF}{\mathcal{F}}
\newcommand{\cG}{\mathcal{G}}
\newcommand{\cH}{\mathcal{H}}
\newcommand{\cI}{\mathcal{I}}
\newcommand{\cJ}{\mathcal{J}}
\newcommand{\cM}{\mathcal{M}}
\newcommand{\cN}{\mathcal{N}}
\newcommand{\cP}{\mathcal{P}}
\newcommand{\cR}{\mathcal{R}}
\newcommand{\cS}{\mathcal{S}}
\newcommand{\cZ}{\mathcal{Z}}
\newcommand{\cL}{\mathcal{L}}
\newcommand{\cT}{\mathcal{T}}
\newcommand{\cU}{\mathcal{U}}
\newcommand{\cV}{\mathcal{V}}
\renewcommand{\phi}{\varphi}
\renewcommand{\epsilon}{\varepsilon}

% Farbdefinitionen
\definecolor{red}{RGB}{180,0,0}
\definecolor{green}{RGB}{75,160,0}
\definecolor{blue}{RGB}{0,75,200}
\definecolor{orange}{RGB}{255,128,0}
\definecolor{yellow}{RGB}{255,245,0}
\definecolor{purple}{RGB}{75,0,160}
\definecolor{cyan}{RGB}{0,160,160}
\definecolor{brown}{RGB}{120,60,10}

\definecolor{itteny}{RGB}{244,229,0}
\definecolor{ittenyo}{RGB}{253,198,11}
\definecolor{itteno}{RGB}{241,142,28}
\definecolor{ittenor}{RGB}{234,98,31}
\definecolor{ittenr}{RGB}{227,35,34}
\definecolor{ittenrp}{RGB}{196,3,125}
\definecolor{ittenp}{RGB}{109,57,139}
\definecolor{ittenpb}{RGB}{68,78,153}
\definecolor{ittenb}{RGB}{42,113,176}
\definecolor{ittenbg}{RGB}{6,150,187}
\definecolor{itteng}{RGB}{0,142,91}
\definecolor{ittengy}{RGB}{140,187,38}

% Textfarbe ändern
\newcommand{\tred}[1]{\textcolor{red}{#1}}
\newcommand{\tgreen}[1]{\textcolor{green}{#1}}
\newcommand{\tblue}[1]{\textcolor{blue}{#1}}
\newcommand{\torange}[1]{\textcolor{orange}{#1}}
\newcommand{\tyellow}[1]{\textcolor{yellow}{#1}}
\newcommand{\tpurple}[1]{\textcolor{purple}{#1}}
\newcommand{\tcyan}[1]{\textcolor{cyan}{#1}}
\newcommand{\tbrown}[1]{\textcolor{brown}{#1}}

% Umstellen der Tabellen Definition
\newcommand{\mpb}[1][.3]{\begin{minipage}{#1\textwidth}\vspace*{3pt}}
\newcommand{\mpe}{\vspace*{3pt}\end{minipage}}

\newcommand{\resultul}[1]{\underline{\underline{#1}}}
\newcommand{\parskp}{$ $\\}	% new line after paragraph
\newcommand{\corr}{\;\widehat{=}\;}
\newcommand{\mdeg}{^{\circ}}

\newcommand{\nok}[2]{\begin{pmatrix}#1\\#2\end{pmatrix}}	% n über k BESSER: \binom{n}{k}
\newcommand{\mtr}[1]{\begin{pmatrix}#1\end{pmatrix}}	% Matrix
\newcommand{\dtr}[1]{\begin{vmatrix}#1\end{vmatrix}}	% Determinante (Betragsmatrix)
\renewcommand{\vec}[1]{\underline{#1}}	% Vektorschreibweise
\newcommand{\imptnt}[1]{\colorbox{red!30}{#1}}	% Wichtiges
\newcommand{\intd}[1]{\,\mathrm{d}#1}

%\bibliography{../Literatur/HTW_Literatur.bib}

% Definition von Titel, Autor usw.
\DTitel{Datenbanksysteme I}
\DUntertitel{Vorlesungsskript}
\DAutor{Falk-Jonatan Strube}
\DNotiz{Vorlesung von Dr. Axel Toll}

\newcommand{\folie}[2]{\begin{center}
\includegraphics[page=#2]{Vorlesung/oneperpage/Kap#1.pdf}
\end{center}}
\begin{document}
\maketitle
\newpage
\tableofcontents
\newpage

\chapter*{Prüfungsmodalitäten}
\paragraph{PVL} unbenoteter Beleg als Voraussetzung zur Prüfung
\begin{enumerate}
\item Access-Beleg (in Papier-Form abzugeben bis 27.05.2016)
\item Abnahme der SQL-Praktikums-Aufgaben (Abnahme während Praktikumszeit)
\end{enumerate}

\paragraph{SP} schriftliche Prüfung, 90min\\
keine eigenen Unterlagen zugelassen. Nur zuvor ausgegeben Referenzen.

\chapter[Datenbank als System und Modell]{Betriebliche Informations- und Kommunikationssysteme - Unternehmensmodell - Datenbank}

\section{Daten als Unternehmensressource}
\subsection{Daten und Informationen}
Redundante Daten bergen Gefahr von Inkonsistenz $\Rightarrow$ Ziel: Schaffen von Datenbank mit folgenden Eigenschaften:
\begin{itemize}
\item ohne Inkonsistenzen (redundanzarm)
\item Zugriffsschutz
\item Mehrfachzugriff
\item Backup-Möglichkeiten (mit Widerspruchsfreier Wiederherstellung)
\end{itemize}

\folie{1}{1}
\begin{tabular}{r | c c}
& Daten & Informationen\\
\hline
Zweck & zweckneutral & zweckgebunden\\
Verarbeitung & maschinell & Interpretation durch Menschen\\
Speicherform & vergegenständlicht & an Menschen gebunden\\
\end{tabular}
\paragraph{Betriebliche Produktionsfaktoren}
\begin{itemize}
\item klassische Faktoren
\begin{itemize}
\item Betriebsmittel
\item Werkstoffe
\item Arbeitskraft
\end{itemize}
\item Daten + Informationen
\end{itemize}

\folie{1}{2}
Große Datenbestände $\Rightarrow$ Maßnahmen zur Datenorganisation\bigskip\\
Eine mögliche Organisationsform (logisches Konzept): Ablage in Relationen (=Tabelle)\bigskip\\
Eine Zeile in dieser Tabelle nennt man \emph{Datensatz} (Tupel, Record, …).\\
Eine Spalte nennt man \emph{Datenfeld}.

\subsection{Klassifikation von Daten}
\paragraph{Mögliche Kriterien} für Datenfeld
\begin{itemize}
\item Zeichenart
\begin{itemize}
\item ganze Zahl $\Rightarrow$ für Aufzählungen
\item reelle zahl $\Rightarrow$ numerische Berechnungen
\item Währung $\Rightarrow$ finanztechnische Berechnungen
\item Datum $\Rightarrow$ kalendarische Berechnungen/Werte
\item Text $\Rightarrow$ Beschreibung
\item Bitmuster $\Rightarrow$ Video, Bilder, …
\end{itemize}
\item Erscheinungsform
\begin{itemize}
\item sprachlich
\item bildlich
\item schriftlich
\end{itemize}
\item Stellung im Verarbeitungsprozess (E - V - A)
\begin{itemize}
\item Eingabe
\item Verarbeitung
\item Ausgabe
\end{itemize}
\item Verarbeitbarkeit mittels IT\\
(Umwandlung in digitale Daten: analog $\rightarrow$ diskret $\rightarrow$ digital)
\item Verwendungszweck\\
\begin{tabular}{
p{\dimexpr0.2\columnwidth-2\tabcolsep-1.5\arrayrulewidth} | >{\raggedright}
p{\dimexpr0.4\columnwidth-2\tabcolsep-1.5\arrayrulewidth} | >{\raggedright}
p{\dimexpr0.3\columnwidth-2\tabcolsep-1.5\arrayrulewidth}}
& Charakterisierung & Beispiel\tabularnewline
\hline
Stammdaten & selten zu verändern (über längeren Zeitraum in Struktur und Inhalt konstant) & Personalstammdaten (Name, Adresse)\tabularnewline
Änderungsdaten & Aktualisierung der Stammdaten & Änderung der Adresse\tabularnewline
Bestandsdaten & Periodische Änderung des wertes (Inhalt) von Feldern, Datenstruktur besteht über längeren Zeitraum konstant & Lagerbestände, Kassenbestände\tabularnewline
Bewegungsdaten & Daten zur Aktualisierung des Wertes von Bestandsdaten & Lagerzugänge und -abgänge\tabularnewline
Archivdaten & vergangenheitsbezogene Daten die über langeren Zeitraum aufbewahrt werden & Rechnungen, Buchungen der vergangenen 5 Jahre\tabularnewline
Transferdaten & Daten, die von einem anderen Programm erzeugt wurden und an ein anderes transferiert werden & Verkauf von Kundenadresson\tabularnewline
Vormerkdaten & Daten, die solange existieren, bis ein genau definiertes Ereignis eintritt & Reservierung einer Materialmenge im Lager
\end{tabular}
\end{itemize}

\subsection{Datenverschlüsselung}
Gemeint ist nicht die Codierung und Decodierung von Daten, sondern das Zuweisen von Schlüsseln zu Datensätzen.
\folie{1}{3}
\paragraph{Identifizierender Schlüssel} \parskp
kennzeichnet Objekteindeutig\\
Bsp.:
\begin{itemize}
\item Personal-Nr.
\item Material-Nr.
\end{itemize}
\paragraph{Klassifiziernder Schlüssel} \parskp
ordnet Objekt einer Klasse zu\\
Bsp.:
\begin{itemize}
\item Länderkennung: D, C, CH, …
\item Geschlecht: M, W
\end{itemize}
\paragraph{Hierarchischer Verbundschlüssel} \parskp
identifizierender Teil hängt vom klassifizierenden Teil ab\\
Bsp.:
\begin{itemize}
\item Autokennzeichen: $\underbrace{\text{DD}}_{\text{klass.}} \underbrace{\text{XY 715}}_{\text{ident.}}$
\end{itemize}
\paragraph{Parallelschlüssel} \parskp
zwei unabhängige Schlüsselteile\\
Bsp.:
\begin{itemize}
\item Flugnummer $\underbrace{\text{LH 283}}_{\text{Flugnr.}} \underbrace{\text{AB3}}_{\text{Flugzeug}}$
\end{itemize}
\paragraph{spezielle Schlüssel in Datenbanksystemen}
\begin{itemize}
\item \emph{Primärschlüssel} (primary key PK): Datenfeld oder die Kombination aus Datenfeldern, die den Datensatz in der Tabelle eindeutig identifizieren.\\
Bsp. Vereinsdatenbank: \\
Primärschlüssel als einzelnes Datenfeld (Mitgliedertabelle): Migtlieds-ID\\ 
Primärschlüssel als eine Kombination von Datendfeldern (Betragstabelle): ID mit Jahr (für Vereinsbeitrag abhängig von Jahr)
\item \emph{Fremdschlüssel} (foreign key FK): Datenfeld, oder Kombination aus Datenfeldern, der (die) auf den PK einer anderen Tabelle zeigt.\\
Bsp.: Mitglieds-ID in Tabelle mit Datenfelder-Primärschlüssel kommt aus der ersten Tabelle
\item \emph{Referentielle Integrität}: Jeder Wert eines FK muss gleich dem Wert des PK sein, auf den der FK zeigt.\\
Bsp.: Neuer Eintrag in Beitragstabelle kann nur neue Einträge bekommen, die Mitglieder aus Mitgliedertabelle enthält. Anders herum kann aus der Mitgliedertabelle kein Mitglied gelöscht werden, das noch in der Beitragstabelle genutzt wird.
\end{itemize}
\folie{1}{4}

\subsection{Speicher- und Zugriffsformen}
\begin{itemize}
\item \emph{sequentielle Speicherung} (fortlaufend)\\
Bsp.: Bandlaufwerk\\
\begin{tabular}{| c | c | c | c}
101 & 102 & 103 & ...\\
\hline
\end{tabular}
\item \emph{verkettete Speicherung}\\
Bsp.: verkette Listen (vgl. Programmierung I)
\item \emph{indexverkettete Speicherung}\\
Trennung: Datenspeicherung und „Weg“ zu den Daten
\begin{itemize}
\item Indexdatei (sortiert nach entsprechendem Index)
\begin{itemize}
\item Primärindex zeigt auf physische Adresse
\item Sekundärindex zeigt auf Primärindex
\end{itemize}
\item Hauptdatei
\end{itemize}
\folie{1}{5}
\end{itemize}
Unterschied Primärschlüssel-Primärindex:
\begin{itemize}
\item Primärschlüssel dient dem Identifizieren
\item Primärindex zum schnellen Suchen
\end{itemize}

\section[Datenmodelle als Abbild]{Datenmodelle als informationelles Abbild der Unternehmensrealität}
\folie{1}{6}
Informationssystem
\begin{itemize}
\item \emph{Funktionsmodell} (was soll das System leisten: Produktion, Lager, Beschaffung, …) $\Rightarrow$ Kernfrage: „Was will ich machen“\\
Strukturen, Abläufe\\
Technik: Programm-Ablauf-Plan (PAP), Ereignisorientierte Prozessketten (EPK), …
\item \emph{Datenmodell}\\
Daten und deren logische Struktur\\
Technik: Entity-Relationship-Modell (ERM)
\end{itemize}
\folie{1}{7}
\folie{1}{8}
\folie{1}{9}
\subparagraph{Bsp.:} \parskp
ABB9 (1-3)

\section[Datenbanksysteme als Grundlage]{Datenbanksysteme als technologische Grundlage der Datenverwaltung}
\folie{1}{10}
ABB10\\
Datenbasis: Tabellen mit Metadaten\\
Datenbankbetriebssystem (DBMS): Software, die mit Datenbasis kommuniziert

\chapter[Datenbanksystem]{Grundlagen und Architektur eines Datenbanksystems (DBS)}

\section[Konventioneller / Datenbankorientierter Ansatz]{Defekte des konventionellen Ansatzes der Datenverwaltung / Zielstellung des datenbankorientierten Ansatzes}
\paragraph{konventionell} \parskp
ABB 11
\paragraph{konventionelle Datenorganisation}\parskp
\emph{Merkmale}
\begin{itemize}
\item Datenspeicherung je Anwendung
\item Datenspeicherung auf physischem Niveau
\end{itemize}
\emph{Nachteile}
\begin{itemize}
\item mangelnde Passfähigkeit (Zugriffskonflikte usw.)
\item Redundanz
\item Konsistenzprobleme
\item mangelnde Flexibilität
\item Daten-Programm-Abhängigkeit (kurz: Datenabhängigkeit)
\end{itemize}
\folie{2}{1}
\folie{2}{2}
\folie{2}{3}
\paragraph{Zielsetzung des Datenbankeinsatzes}
\folie{2}{4}
\begin{enumerate}
\item Bsp. für gewollte Redundanz: Sekundärindex
\item Datensicherheit:
\begin{itemize}
\item physisch, falls bspw. der Server abbrennt
\item logisch, dass bspw. alle Daten den richtigen Typ haben
\end{itemize}
\end{enumerate}

\section{Architektur von Datenbanksystemen}

\subsection{Grundlegende Begriffe}
Am Beispiel der Objekte der Datenmodellierung mittels ERM\\
\begin{tabular}{>{\raggedright}
p{\dimexpr0.2\columnwidth-2\tabcolsep-1.5\arrayrulewidth} | >{\raggedright}
p{\dimexpr0.4\columnwidth-2\tabcolsep-1.5\arrayrulewidth} | >{\raggedright}
p{\dimexpr0.3\columnwidth-2\tabcolsep-1.5\arrayrulewidth}
}
Begriff & Erklärung & Beispiel\tabularnewline
\hline 
Entity & Objekt der realen Welt & Max Meier, Arbeitsaufgabe Reportgenerator\tabularnewline
Entity-Typ & Objektklasse (-Menge), enthält Elemente mit struktureller Ähnlichkeit & Mitarbeiter, Arbeitsaufgabe, Abteilung \tabularnewline
Merkmale / Attribut / Prädikat & Beschreibungen eines Entity-Typs & Name, Vorname, Gehalt\tabularnewline
Wert & Ausprägung des Merkmals je Entity, aus einem bestimmten Wertevorrat (Domain) & „Meier“, „Max“, 3800,-\tabularnewline
Beziehung, Set & Logischer Zusammenhang zwischen Entity-Typen & Mitarbeiter -- \underline{arbeitet an} -- Arbeitsaufgabe \tabularnewline
Beziehungstyp, Settyp & Art der Beziehung (mögliche Anzahl an Entitäten, die in Beziehung treten) & $n:1$ Mitarbeiter -- \underline{gehört zu} -- Abteilung ABB50\tabularnewline
\end{tabular}

\subsection{3-Ebenen-Architektur}
gemäß ANSI x3/SPARC (1975)
\begin{itemize}
\item Architekturebene
\begin{itemize}
\item externe Ebene
\item konzeptionelle Ebene
\item interne Ebene
\end{itemize}
\item Modell
\begin{itemize}
\item externes Modell
\item konzeptionelles Modell
\item internes Modell
\end{itemize}
\item Schema (konkrete Ausprägung des Modells)
\begin{itemize}
\item externes Schema
\item konzeptionelles Schema
\item internes Schema
\end{itemize}
\end{itemize}

\subsubsection{Konzeptionelle Ebene}
\paragraph{Gegenstand:} logisches Modell des gesamten Systems
\paragraph{Beschreibungselemente:}
\begin{itemize}
\item Entity-Typen
\item Beziehungen
\item Attribute
\item Wertevorrate (bspw. Einschränkung von Alter: nur Zahlen zwischen 1 und 100)
\item Integritätsbedingung (bspw. NOT NULL, vgl. Wertevorrat)
\end{itemize}
\subsubsection{Externe Ebene}
\paragraph{Gegenstand:} Beschreibung \emph{ausgewählter} Elemente der konzeptionellen Ebene aus Sicht des jeweiligen Endbenutzers
\folie{2}{5}
\paragraph{Element:} Sicht (View)
\subsubsection{Interne Ebene}
\paragraph{Gegenstand:} Form/Art der Ablage der Elemente der konzeptionellen Ebene im physischen Speicher
\paragraph{Element:} Index
\folie{2}{7}

\section{Aufgbau und Arbeitsweise von DBMS}
5 Grundfunktionen eines DBMS
\folie{2}{8}
\subsection{Zugriffsvermittlung}
\folie{2}{9}
\subsection{Unterstützung Datenbeschreibung-Entwicklung}
\folie{2}{10}
\subsection{Integritätssicherung}
\folie{2}{11}
Bsp. operationale Integrität:\\
Gehaltserhöhungen sowohl für Organisatoren (O) und Programmierer (P) um \euro{50,-}.\\
Gehaltserhöhung darf nicht doppelt erfolgen $\Rightarrow$ Sperren von Gehalt, solange ein Nutzer das Gehalt ändert (bei Gefahr bezgl. Deadlock, muss das System das Problem erkennen und entsprechend auflösen).
\subsection{Zugriffsschutz}
\folie{2}{12}
\subsection{Dienstprogrammfunktionen}
\folie{2}{13}

\section{Datenorganisation}
\begin{itemize}
\item logische Datenorganisation (DO)
\begin{itemize}
\item externe Ebene
\item konzeptionelle Ebene
\end{itemize}
\item physische DO
\begin{itemize}
\item interne Ebene
\end{itemize}
\end{itemize}
\paragraph{klassische Datermodelle (logisch)}
\begin{itemize}
\item hierarchisch DM (graphisches DM)
\item Netzwerk DM (graphisches DM)
\item relationales DM (behandelt in DBS I+II)
\end{itemize}
\paragraph{weitere DM}
\begin{itemize}
\item objektorientiertes DM (DBS II)
\item objektrelationales DM (DBS II)
\item XML-DM / NoSQL DM … (DBS III)
\end{itemize}
\folie{2}{14}

\begin{tabular}{
>{\raggedright} p{\dimexpr0.3\columnwidth-2\tabcolsep-1.5\arrayrulewidth} |
>{\raggedright} p{\dimexpr0.22\columnwidth-2\tabcolsep-1.5\arrayrulewidth} |
>{\raggedright} p{\dimexpr0.22\columnwidth-2\tabcolsep-1.5\arrayrulewidth} |
>{\raggedright} p{\dimexpr0.22\columnwidth-2\tabcolsep-1.5\arrayrulewidth} 
}
& Hierarchisches DM & Netzwerk DM & relationales DM\tabularnewline
& ABB 51 & ABB 52 & ABB 53 \tabularnewline
\hline
Einstiegspunkt & ein Entity-Typ & mehrere Entity & beliebig\tabularnewline
strukturelle Beschräknung & Hierarchie & keine & keine \tabularnewline
Zeitpunkt des Aufbau der Beziehung & zur Entwicklungszeit & zur Entwicklungszeit & zur Laufzeit \tabularnewline
Performance & + & + & \Large{\textbf{--}}\tabularnewline
Flexibilität bzgl. Änderung & -- & -- & \Large{\textbf{+}}
\end{tabular}
\chapter{Relationales Datenmodell}
\section{Terminologie im Relationenmodell}
\folie{3}{1}

\paragraph{Bsp.:} \parskp
Entitytyp: 
\begin{itemize}
\item Zeugnis
\end{itemize}
Attribute:
\begin{itemize}
\item $A_1$ Fach
\item $A_2$ Note
\end{itemize}
Wertebereiche:
\begin{itemize}
\item $W_1 \; \{\text{Ma, Ph}\}$
\item $W_2 \{\text{1 ,2 ,3 , 4 , 5}\}$
\end{itemize}
$n=2$, d.h. 2-stellige Relation ableitbar (Grad = degree = 2)\\
$PM = W_1 * W_2 = W_1 \times W_2$\\
\begin{tabular}{c | c}
Fach & Note\\
\hline
Ma & 1\\
Ma & 2\\
Ma & 3\\
Ma & 4\\
Ma & 5\\
Pd & 1\\
Pd & 2\\
Pd & 3\\
Pd & 4\\
Pd & 5\\
\end{tabular}\\
Teilmenge 1  = Relation 1:\\
\begin{tabular}{c | c}
Fach & Note \\
\hline
Ma & 1\\
Ph & 2\\
\end{tabular} \tgreen{gültig}\\
Teilmenge 2  = Relation 2:\\
\begin{tabular}{c | c}
Fach & Note \\
\hline
Ma & 1\\
Ph & 1\\
Ph & 4\\
\end{tabular} \tgreen{gültige Relation} (unabhängig von der semantischen Sinnhaftigkeit)
\folie{3}{2}
\paragraph{Weitere Kernaussagen zum relationalen Modell:}
\begin{itemize}
\item Darstellung der Relation als Tabelle
\item Identifikation der Relation über Namen
\item Anzahl an Attributen (Spalten) ist fest (degree)
\item Anzahl der Tupel (Zeilen) ist variabel (Mächtigkeit)
\item Wertebereiche der Attribute = Domain
\item Im Kreuzungspunkt von Attribut und Tupel stehen \emph{atomare} Werte
\end{itemize}
\section{Definition und Manipulation im relationalen Datenmodell}
\subsection{Datendefinition}
$\Rightarrow$ Definition von Relationen
\folie{3}{3}
\folie{3}{4}
\subsection{Datenmanipulation / Relationenalgebra}
Relationenalgebra nach: Codd\\
Grundidee:\\
Operationen auf Relationen\\
$\Rightarrow$ Ergebnis ist wieder eine \emph{Relation}\\
D.h. mengenweise Arbeit \emph{nicht} satzweise.
\subsubsection{Mengenoperationen} $\cup\; \cap\; \setminus\; \times$\\
ABB57
\folie{3_Beispiele}{1, scale=0.8}
\paragraph{Vereinigung} $\cup$\\
ABB58 orange\\
UNION
\folie{3_Beispiele}{2, scale=0.8}
\paragraph{Durchschnitt} $\cap$\\
ABB58 grün\\
INTERSECTION
\folie{3_Beispiele}{3, scale=0.8}
\paragraph{Differenz} $\setminus$\\
$R_1\setminus R_2$
ABB 58 lila\\
Bedingung für $\cup, \cap, \setminus$ (\emph{Vereinigungsverträglichkeit}):
\begin{itemize}
\item Anzahl an Attributen ist gleich
\item unzugeordnete Attribute besitzen gleiche Domain (Domainverträglichkeit)
\end{itemize}
$R_1\cup R_2 = R_2 \cup R_1$\\
$R_1 \cap R_2 = R_2 \cap R_1$\\
$R_1 \setminus R_2 \not = R_2 \setminus R_1$\\
DIFFERENCE
\folie{3_Beispiele}{4, scale=0.8}
\paragraph{Kartesissches Produkt} $\times$\\
$R_1\times R_2$\\
Ergebnisrelation enthält: 
\begin{itemize}
\item alle Attribute aus $R_1$ und $R_2$.
\item alle Kombinationen an Tupeln aus $R_1$ und $R_2$.
\end{itemize}
ABB 59
\folie{3_Beispiele}{7, scale=0.8}
\subsubsection{Relationale Operationen}
\paragraph{Projektion} Spaltenauswahl\\
PROJ\\
ABB 60 grün
\folie{3_Beispiele}{5, scale=0.8}
\paragraph{Selektion} Tupelauswahl (laut Bedingung)\\
REST\\
ABB 60 orange
\folie{3_Beispiele}{6, scale=0.8}
\paragraph{Verbund} Verbindung zwischen zwei Relationen bezüglich der Gleichheit der Attributwerte in einer Verbindungsspalte\\
JOIN\\
intern:
\begin{enumerate}
\item Kartesisches Produkt der Relation
\item auf Ergebnisrelation Selektion nach Gleichheit der Werte in der/den Verbindungsspalten
\end{enumerate}
Merkmale des JOIN:
\begin{itemize}
\item Attribute über die den JOIN ausgeführt wird, müssen
\begin{itemize}
\item \emph{keine} Schlüsselspalten sein
\item gleiche Domain besitzen
\item \emph{nicht} die gleichen Namen besitzen
\end{itemize}
Jede Relation ist mit jeder Relation via JOIN verbindbar (auch mit sich selbst).
\end{itemize}
\folie{3_Beispiele}{8, scale=0.8}

\section{Normalformenlehre}
Ziele der Normalisierung:
\begin{itemize}
\item Vermeidung unerwünschter Abhängigkeiten beim Ändern, Löschen und Einfügen
\item Reduzierung der Umbildung von Relationen bei Einführung neuer Attribute
\item Erhöhung der Transparenz und Aussagekraft für den Nutzer (Trennung der unterschiedlichen Konzepte der realen Welt)
\item Gewährung der Korrektheit der Datenbakn (zu jedem Zeitpunkt)
\end{itemize}
Vorteile der Normalisierung:
\begin{itemize}
\item Sicherung von relativ einfachen, überschaubaren und einfach handhabbaren Relationen
\item Beseitigung von Update-/Insert- und Delete-Anomalien
\item Einfachere Überprüfung von Konsistenzbedingungen
\end{itemize}
Nachteile:
\begin{itemize}
\item größere Redundanz (Schlüsselredundanz)
\item höherer Aufwand bei komplexen Auswertungen
\end{itemize}
\paragraph{Codd} (1970)\\
Normalform (NF):
$\underbrace{\text{1. NF} \Rightarrow \text{2. NF} \Rightarrow \text{3. NF}}_{\text{praktisch relevant}}\Rightarrow\text{4. NF}\Rightarrow\text{5. NF}$
\subsection{1. Normalform}
\folie{3}{5}
\begin{itemize}[label=$\Rightarrow$]
\item Relation
\begin{itemize}
\item atomare Werte
\item PS erweitern
\end{itemize}
\end{itemize}
\subsection{2. Normalform}
\folie{3}{6}
Abhängigkeiten:\\
\begin{tabular}{r | l | l}
PS & Nichtschlüssel-Attribute & 2. NF\\
\hline
\underline{Mitnr}, \underline{Projnr} & Anteil & MiPro\\
\underline{Mitnr} & Name, Beruf, Gehalt, Abtnr, Abtbez & Mitarbeiter\\
\underline{Projnr} & Projbez & Projekt
\end{tabular}
\begin{itemize}[label=$\Rightarrow$]
\item Zerlegung
\begin{itemize}
\item volle funktionale Abhängigkeit
\end{itemize}
\end{itemize}
\subsection{3. Normalform}
\folie{3}{7}
für Mitarbeiter (M): (x$\to$y: von x kann man auf y schließen)\\
M.Mitnr $\to$ M.Abtnr\\
M.Abtnr $\not \to$ M.Mitnr\\
M.Abtnr $\to$ M.Abtbez\\
Also:\\
M.Mitnr $\to$ M.Abtnr $\to$ M.Abtbez\\
aber:\\
M.Abtnr $\not \to$ M.Mitnr
\begin{itemize}[label=$\Rightarrow$]
\item weitere Zerlegung
\end{itemize}
\underline{Abtnr} $\to$ weitere Tabelle Abteilung mit PS=\underline{Abtnr}.
\folie{3}{8}
\folie{3_Beispiele}{9, scale=0.8}
\folie{3_Beispiele}{10, scale=0.8}

\section{Vergleich relationaler DBMS}
\folie{3}{9}
\begin{align*}
\text{NULL} &= \text{missing value (kein Wert)}\\
&\not = \text{' '}\\
&\not = \emptyset
\end{align*}
\folie{3}{10}



%\newpage
%\printbibliography
\end{document}