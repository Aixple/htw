% Header aus der Vorlage
\documentclass[a4paper,11pt, footheight=26pt
%,twoside
]{scrreprt}
\usepackage[head=23pt]{geometry}	% head=23pt umgeht Fehlerwarnung, dafür größeres "top" in geometry
\geometry{a4paper, top=30mm, bottom=22mm,headsep=10mm, footskip=12mm
, left=20mm, right=20mm
%, inner=27mm, outer=13mm
}

% Zeile 2 (,twoside) und 7 (inner=...) für eine Druckversion (doppelseitig) ent-kommentieren (Rand für Hefter)

\setcounter{secnumdepth}{3}	% zählt auch subsubsection
\setcounter{tocdepth}{3}	% Inhaltsverzeichnis bis in subsubsection

% Input inkl. Umlaute, Silbentrennung
\usepackage[T1]{fontenc}
\usepackage[utf8]{inputenc}
\usepackage[ngerman]{babel}
\usepackage{csquotes}	% Anführungszeichen
\usepackage{eurosym}

% HTW Corporate Design: Arial (Helvetica)
\usepackage{helvet}
\renewcommand{\familydefault}{\sfdefault}

% Style-Aufhübschung
\usepackage{soul, color}	% Kapitälchen, Unterstrichen, Durchgestrichen usw. im Text
\usepackage{scrlayer-scrpage}	% Kopf-/Fußzeile
%\usepackage{titleref}
\usepackage[perpage]{footmisc}	% Fußnotenzählung Seitenweit, nicht Dokumentenweit
\renewcommand*{\thefootnote}{\fnsymbol{footnote}}	% Fußnoten-Symbole anstatt Zahlen
\renewcommand*{\titlepagestyle}{empty} % Keine Seitennummer auf Titelseite

% Mathe usw.
\usepackage{amssymb}
\usepackage[fleqn]{amsmath}	% fleqn: align-Umgebung rechtsbündig
\usepackage{xcolor}
\usepackage{esint}	% Schönere Integrale, \oiint vorhanden
\everymath=\expandafter{\the\everymath\displaystyle}	% Mathe Inhalte werden weniger verkleinert
\usepackage{wasysym}	% mehr Symbole, bspw \lightning
% Auch arcus-Hyperbolicus-Funktionen
\DeclareMathOperator{\arccot}{arccot}
\DeclareMathOperator{\arccosh}{arccosh}
\DeclareMathOperator{\arcsinh}{arcsinh}
\DeclareMathOperator{\arctanh}{arctanh}
\DeclareMathOperator{\arccoth}{arccoth} 
% Mathe in Anführungszeichen:
\newsavebox{\mathbox}\newsavebox{\mathquote}
\makeatletter
\newcommand{\mq}[1]{% \mathquotes{<stuff>}
  \savebox{\mathquote}{\text{"}}% Save quotes
  \savebox{\mathbox}{$\displaystyle #1$}% Save <stuff>
  \raisebox{\dimexpr\ht\mathbox-\ht\mathquote\relax}{"}#1\raisebox{\dimexpr\ht\mathbox-\ht\mathquote\relax}{''}
}
\makeatother

% tikz usw.
\usepackage{tikz}
\usepackage{pgfplots}
\pgfplotsset{compat=1.11}	% Umgeht Fehlermeldung
\usetikzlibrary{graphs}
%\usetikzlibrary{through}	% ???
\usetikzlibrary{arrows}
\usetikzlibrary{arrows.meta}	% Pfeile verändern / vergrößern: \draw[-{>[scale=1.5]}] (-3,5) -> (-3,3);
\usetikzlibrary{automata,positioning} % Zeilenumbruch im Node node[align=center] {Text\\nächste Zeile} automata für Graphen
\usetikzlibrary{matrix}
\usetikzlibrary{patterns}	% Schraffierte Füllung
\tikzstyle{reverseclip}=[insert path={	% Inverser Clip \clip
	(current page.north east) --
	(current page.south east) --
	(current page.south west) --
	(current page.north west) --
	(current page.north east)}
% Nutzen: 
%\begin{tikzpicture}[remember picture]
%\begin{scope}
%\begin{pgfinterruptboundingbox}
%\draw [clip] DIE FLÄCHE, IN DER OBJEKT NICHT ERSCHEINEN SOLL [reverseclip];
%\end{pgfinterruptboundingbox}
%\draw DAS OBJEKT;
%\end{scope}
%\end{tikzpicture}
]	% Achtung: dafür muss doppelt kompliert werden!
\usepackage{graphpap}	% Grid für Graphen
\tikzset{every state/.style={inner sep=2pt, minimum size=2em}}

% Tabular
\usepackage{longtable}	% Große Tabellen über mehrere Seiten
\usepackage{multirow}	% Multirow/-column: \multirow{2[Anzahl der Zeilen]}{*[Format]}{Test[Inhalt]} oder \multicolumn{7[Anzahl der Reihen]}{|c|[Format]}{Test2[Inhalt]}
\renewcommand{\arraystretch}{1.3} % Tabellenlinien nicht zu dicht
\usepackage{colortbl}
\arrayrulecolor{gray}	% heller Tabellenlinien
\usepackage{array}	% für folgende 3 Zeilen (für Spalten fester breite mit entsprechender Ausrichtung):
\newcolumntype{L}[1]{>{\raggedright\let\newline\\\arraybackslash\hspace{0pt}}m{\dimexpr#1\columnwidth-2\tabcolsep-1.5\arrayrulewidth}}
\newcolumntype{C}[1]{>{\centering\let\newline\\\arraybackslash\hspace{0pt}}m{\dimexpr#1\columnwidth-2\tabcolsep-1.5\arrayrulewidth}}
\newcolumntype{R}[1]{>{\raggedleft\let\newline\\\arraybackslash\hspace{0pt}}m{\dimexpr#1\columnwidth-2\tabcolsep-1.5\arrayrulewidth}}

% Nützliches
\usepackage{verbatim}	% u.a. zum auskommentieren via \begin{comment} \end{comment}
\usepackage{tabto}	% Tabs: /tab zum nächsten Tab oder /tabto{.5 \CurrentLineWidth} zur Stelle in der Linie
\NumTabs{6}	% Anzahl von Tabs pro Zeile zum springen
\usepackage{listings} % Source-Code mit Tabs
\usepackage{lstautogobble} 
\usepackage{enumitem}	% Anpassung der enumerates
\setlist[enumerate,1]{label=\arabic*.)}	% global andere Enum-Items
\newenvironment{anumerate}{\begin{enumerate}[label=\alph*.)]}{\end{enumerate}} % Alphabetische Aufzählung
\renewcommand{\labelitemiii}{$\scriptscriptstyle ^\blacklozenge$} % global andere 3. Item-Aufzählungszeichen
\usepackage{letltxmacro} % neue Definiton von Grundbefehlen
% Nutzen:
%\LetLtxMacro{\oldemph}{\emph}
%\renewcommand{\emph}[1]{\oldemph{#1}}

% Einrichtung von lst
\lstset{
basicstyle=\ttfamily, 
mathescape=true, 
%escapeinside=^^, 
autogobble, 
tabsize=2,
basicstyle=\footnotesize\sffamily\color{black},
frame=single,
rulecolor=\color{lightgray},
numbers=left,
numbersep=5pt,
numberstyle=\tiny\color{gray},
commentstyle=\color{gray},
keywordstyle=\color{green},
stringstyle=\color{orange},
morecomment=[l][\color{magenta}]{\#}
%showspaces=false,
showstringspaces=false,
breaklines=true,
literate=%
    {Ö}{{\"O}}1
    {Ä}{{\"A}}1
    {Ü}{{\"U}}1
    {ß}{{\ss}}1
    {ü}{{\"u}}1
    {ä}{{\"a}}1
    {ö}{{\"o}}1
    {~}{{\textasciitilde}}1
}
\usepackage{scrhack} % Fehler umgehen
\def\ContinueLineNumber{\lstset{firstnumber=last}} % vor lstlisting. Zum wechsel zum nicht-kontinuierlichen muss wieder \StartLineAt1 eingegeben werden
\def\StartLineAt#1{\lstset{firstnumber=#1}} % vor lstlisting \StartLineAt30 eingeben, um bei Zeile 30 zu starten
\let\numberLineAt\StartLineAt

% BibTeX
\usepackage[backend=bibtex, bibencoding=ascii]{biblatex}	% BibTeX
\usepackage{makeidx}
%\makeglossary
%\makeindex

% Grafiken
\usepackage{graphicx}
\usepackage{epstopdf}	% eps-Vektorgrafiken einfügen

% pdf-Setup
\usepackage{pdfpages}
\usepackage[bookmarks,%
bookmarksopen=false,% Klappt die Bookmarks in Acrobat aus
colorlinks=true,%
linkcolor=black,%
citecolor=red,%
urlcolor=green,%
]{hyperref}

% Titel, Autor usw. werden vor dem Anfang des Dokuments in einem Rutsch definiert…
\newcommand{\DTitel}[1]{\newcommand{\Dokumententitel}{#1}}
\newcommand{\DUntertitel}[1]{\newcommand{\Dokumentenuntertitel}{#1}}
\newcommand{\DAutor}[1]{\newcommand{\Dokumentenautor}{#1}}
\newcommand{\DNotiz}[1]{\newcommand{\Dokumentennotiz}{#1}}
\newcommand{\DSign}[1]{\newcommand{\Dokumentensignatur}{#1}}
\DSign{\footnotesize{\textcolor{darkgray}{Mitschrift von\\ \Dokumentenautor}}}
\newcommand{\Autorformat}[1]{\textcolor{darkgray}{Mitschrift von #1}}
\newcommand{\workingdir}{../}	% Arbeitsordner (in Abhängigkeit vom Master) Standard: LateX_master Ordner liegt im Eltern-Ordner
% … Deswegen folgendes erst Nach Dokumentenbeginn ausführen:
\AtBeginDocument{
	\hypersetup{
		pdfauthor={\Dokumentenautor},
		pdftitle={HTW Dresden | \Dokumententitel - \Dokumentenuntertitel},
	}
	\automark[section]{section}
	\automark*[subsection]{subsection}
	\pagestyle{scrheadings}
	\ihead{\includegraphics[height=1.7em]{\workingdir LaTeX_master/HTW-Logo.eps}}
	\ohead{\Dokumententitel}
	\cfoot{\pagemark}
	\ofoot{\Dokumentensignatur}
	% Titelseite
	\title{\includegraphics[width=0.35\textwidth]{\workingdir LaTeX_master/HTW-Logo.eps}\\\vspace{0.5em}
	\Huge\textbf{\Dokumententitel} \\\vspace*{0,5cm}
	\Large \Dokumentenuntertitel \\\vspace*{4cm}}
	\author{\Autorformat{\Dokumentenautor} \vspace*{1cm}\\\Dokumentennotiz}
}

%% EINFACHE BEFEHLE

% Abkürzungen Mathe
\newcommand{\EE}{\mathbb{E}}
\newcommand{\QQ}{\mathbb{Q}}
\newcommand{\RR}{\mathbb{R}}
\newcommand{\CC}{\mathbb{C}}
\newcommand{\NN}{\mathbb{N}}
\newcommand{\ZZ}{\mathbb{Z}}
\newcommand{\PP}{\mathbb{P}}
\renewcommand{\SS}{\mathbb{S}}
\newcommand{\cA}{\mathcal{A}}
\newcommand{\cB}{\mathcal{B}}
\newcommand{\cC}{\mathcal{C}}
\newcommand{\cD}{\mathcal{D}}
\newcommand{\cE}{\mathcal{E}}
\newcommand{\cF}{\mathcal{F}}
\newcommand{\cG}{\mathcal{G}}
\newcommand{\cH}{\mathcal{H}}
\newcommand{\cI}{\mathcal{I}}
\newcommand{\cJ}{\mathcal{J}}
\newcommand{\cM}{\mathcal{M}}
\newcommand{\cN}{\mathcal{N}}
\newcommand{\cP}{\mathcal{P}}
\newcommand{\cR}{\mathcal{R}}
\newcommand{\cS}{\mathcal{S}}
\newcommand{\cZ}{\mathcal{Z}}
\newcommand{\cL}{\mathcal{L}}
\newcommand{\cT}{\mathcal{T}}
\newcommand{\cU}{\mathcal{U}}
\newcommand{\cV}{\mathcal{V}}
\renewcommand{\phi}{\varphi}
\renewcommand{\epsilon}{\varepsilon}

% Farbdefinitionen
\definecolor{red}{RGB}{180,0,0}
\definecolor{green}{RGB}{75,160,0}
\definecolor{blue}{RGB}{0,75,200}
\definecolor{orange}{RGB}{255,128,0}
\definecolor{yellow}{RGB}{255,245,0}
\definecolor{purple}{RGB}{75,0,160}
\definecolor{cyan}{RGB}{0,160,160}
\definecolor{brown}{RGB}{120,60,10}

\definecolor{itteny}{RGB}{244,229,0}
\definecolor{ittenyo}{RGB}{253,198,11}
\definecolor{itteno}{RGB}{241,142,28}
\definecolor{ittenor}{RGB}{234,98,31}
\definecolor{ittenr}{RGB}{227,35,34}
\definecolor{ittenrp}{RGB}{196,3,125}
\definecolor{ittenp}{RGB}{109,57,139}
\definecolor{ittenpb}{RGB}{68,78,153}
\definecolor{ittenb}{RGB}{42,113,176}
\definecolor{ittenbg}{RGB}{6,150,187}
\definecolor{itteng}{RGB}{0,142,91}
\definecolor{ittengy}{RGB}{140,187,38}

% Textfarbe ändern
\newcommand{\tred}[1]{\textcolor{red}{#1}}
\newcommand{\tgreen}[1]{\textcolor{green}{#1}}
\newcommand{\tblue}[1]{\textcolor{blue}{#1}}
\newcommand{\torange}[1]{\textcolor{orange}{#1}}
\newcommand{\tyellow}[1]{\textcolor{yellow}{#1}}
\newcommand{\tpurple}[1]{\textcolor{purple}{#1}}
\newcommand{\tcyan}[1]{\textcolor{cyan}{#1}}
\newcommand{\tbrown}[1]{\textcolor{brown}{#1}}

% Umstellen der Tabellen Definition
\newcommand{\mpb}[1][.3]{\begin{minipage}{#1\textwidth}\vspace*{3pt}}
\newcommand{\mpe}{\vspace*{3pt}\end{minipage}}

\newcommand{\resultul}[1]{\underline{\underline{#1}}}
\newcommand{\parskp}{$ $\\}	% new line after paragraph
\newcommand{\corr}{\;\widehat{=}\;}
\newcommand{\mdeg}{^{\circ}}

\newcommand{\nok}[2]{\begin{pmatrix}#1\\#2\end{pmatrix}}	% n über k BESSER: \binom{n}{k}
\newcommand{\mtr}[1]{\begin{pmatrix}#1\end{pmatrix}}	% Matrix
\newcommand{\dtr}[1]{\begin{vmatrix}#1\end{vmatrix}}	% Determinante (Betragsmatrix)
\renewcommand{\vec}[1]{\underline{#1}}	% Vektorschreibweise
\newcommand{\imptnt}[1]{\colorbox{red!30}{#1}}	% Wichtiges
\newcommand{\intd}[1]{\,\mathrm{d}#1}

% Definition von Titel, Autor usw.
\DTitel{Mathematik I}
\DUntertitel{Vorlesungsskript}
\DAutor{Falk-Jonatan Strube}
\DNotiz{Vorlesung von Herrn Meinhold}

\begin{document}

\pagenumbering{Roman}

\maketitle
\newpage
\tableofcontents
\newpage

\pagenumbering{arabic}

\part{Elementare Grundlagen}

\section{Aussagen und Grundzüge der Logik}
%\subsection{Aussagen, Wahrheitswert}

\paragraph{Aussage:} (im weiteren Sinne) Sprachlich sinnvoller, konsatierender Satz. In diesem Abschnitt werden nur zweiwertige Aussagen betrachtet, d.h. Aussagen, die entwoder wahr oder falsch sind.

\subparagraph{Bsp. 1:} 
\begin{itemize}
\item[(1)] Es gibt unendlich viele Primzahlen (wahr)
\item[(2)] Es gibt unendlich viele Primzahlzwillinge, z.B. (3,5), (5,7), (11,13), (17,19) usw. (Wahrheitswert nicth bekannt!)
\item[(3)] $5+7=13$ (falsch)
\item[(4)] Wie spät ist es? (keine Aussage)
\item[(5)] Diese Aussage ist falsch! (keine Aussage, paradox)
\item[(6)] Am 30.06.2016 wird es in Dresden regnen.
\end{itemize}

(1)--(3) sind zweiwertige Aussagen, (4) und (5) sind keine Aussagen, (6) ist keine zweiwertige Aussage (Wahrscheinlichkeit, d.h. Zahl zwischen 0 und 1 angebbar).

\subparagraph{Bezeichnungen:} \parskp
\begin{tabular}{l l}
p, q, r, &… Aussagen,\\
0 &… falsche Aussage, \\
1 &… wahre Aussage\\
\end{tabular}

\paragraph{Wahrheitswert:} \parskp
$w(p)=\begin{cases}
1 & \text{(falls p wahr)} \\
0 & \text{(fallls p falsch)}
\end{cases}$
    
$p \equiv q$ (p \emph{identisch} q) … p und q haben denselben Wahrheitswert
    
\subsection{Aussagesverschiebung}

\begin{enumerate}
\item \emph{Negation} $\overline{p}$ („nicht p“) [oft auch $p!$ bzw. $\neg p$]\\
\begin{tabular}{c | c}
$p$ & $\overline{p}$\\
\hline
0 & 1 \\
1 & 0 \\
\end{tabular}

\item \emph{Konjunktion} $p \wedge q$ („p und q“)
\item \emph{Disjunktion} $ p \vee q $ („p oder q“) [Alternative -- nicht ausschließendes Oder!]\\
\begin{tabular}{c | c |c |c}
$p$ & $q$ & $p \wedge q$ & $p \vee q$\\
\hline
1 & 1 &1&1\\
1 & 0 &0&1\\
0 & 1 &0&1\\
0 & 0 &0&0\\
\end{tabular}

\item \emph{Implikation} $(p \Rightarrow q ):\equiv \overline{p} \vee q$ („aus p folgt q“, „wenn p, dann q“)\\
\begin{tabular}{c | c |c|c}
$p$ & $q$ & $\overline{p}$ & $p\Rightarrow q$\\
\hline
1&1&0&1 \\
1&0&0&0\\
1&1&1&1\\
1&0&1&1\\
\end{tabular}\\
Begriffe: $p \Rightarrow q$ (p: \emph{Prämisse}, q: \emph{Konklusion})\\
Eine Implikation ist genau dann falsch, wenn die Prämisse richtig und die Konklusion falsch ist!
\paragraph{Bsp. 2:}
\begin{itemize}
\item $-1=1$ (falsch) $\Rightarrow$ $1=1$ (wahr) [durch Quadrieren]
\item $-1=1$ (falsch) $\Rightarrow$ $0=2$ (falsch) [Addition von 1]
\end{itemize}
Aus einer falschen Aussage lassen sich durch richtiges Schließen sowohl falsche als auch richtige Aussagen gewinnen.

Andere Sprechweisen: „p ist \emph{hinreichend} für q“, „q ist \emph{notwendig} für p“
\item \emph{Äquivalenz} $(p\Leftrightarrow q):\equiv (p\Rightarrow q) \wedge (q \Rightarrow p)$ („p äquivalent q“, „p ist notwendig und hinreichend für q“, „p genau dann wenn q“)\\
(ist genau dann wahr, wenn p und q den selben Wahrheitswert besitzen)
\end{enumerate}

\subsection{Logische Gesetze (Tautologien)}
Eine Tautologie $t$ ist eine Aussagenverbindung, die unabhängig vom Wahrheitswert der einzelnen Aussagen stets wahr ist (d.h. $t\equiv 1$).
\paragraph{Bsp. 3:}\parskp
Einige wichtige Tautologien
\begin{enumerate}
\item $p\Leftrightarrow \overline{\overline{p}}$ \tab \tab(Negation der Negation)
\item $p \vee \overline{p}$\tab \tab (Satz vom ausgeschlossenem Dritten)
\item a) $\overline{p\wedge q} \equiv \overline{p} \vee \overline{q}$\\
b) $\overline{p\vee q} \equiv \overline{p} \wedge \overline{q}$ \tab(de Morgansche Regeln)
\item $(p\Rightarrow q) \equiv (\overline{q} \Rightarrow \overline{p})$ \tab(Kontrapositionsgesetz)
\item $p\wedge (p\Rightarrow q)) \Rightarrow q$ \tab(direkter Beweis)
\item $p\wedge (\overline{q} \Rightarrow \overline{p})) \Rightarrow q$ \tab(indirekter Beweis)
\end{enumerate}
Beweise mittels Wahrheitstafeln (vgl. Übung 1).\\
Bemerkung zu 1., 3., 4.: Eine Äquivalenz ist genau dann eine Tautologie, wenn beide Seiten identisch sind, z.B. $p\equiv \overline{\overline{p}}$.

\paragraph{Beweistechniken:} \parskp
Zu beweisen ist $q$.
\begin{enumerate}
\item Direkter Beweis:
\begin{itemize}
\item Nachweis von $p$ (Voraussetzung)
\item Richtiger Schluss $p\Rightarrow q$\\
Dann $q$ wahr (Behauptung)
\end{itemize}
\item Indirekter Beweis: Annahme von $\overline{q}$ auf Wiederspruch führen (auf unterschiedliche Weise möglich, vgl. folgendes Bsp).
\end{enumerate}

\subparagraph{Bsp. 4:}\parskp
$q = $„$\sqrt{2}$ ist irrational“ (keine rationale Zahl)

Beweis indirekt: \\
Es gelte $\overline{q}$, d.h. $\sqrt{2}$ ist rational, dann gelten folgende Schlüsse: $\sqrt{2} = \frac{m}{n}$ mit teilerfremden natürlichen Zahlen $m$ und $n$.\\
$\Rightarrow 2=\frac{m^2}{n^2} \Rightarrow 2 \cdot n^2 = m^2 \Rightarrow 2|m^2$\\
$\Rightarrow \boxed{2|m}$ (2 ist Teiler von m)\\
$\Rightarrow 4|m^2$ (mit $m^2=2n^2$)\\
$\Rightarrow 4|2n^2 \Rightarrow 2|n^2 \Rightarrow \boxed{2|n}$\\
Widerspruch: Da $m$ und $n$ teilerfremd sind.	\#

\paragraph{Weitere Gesetze}

\begin{itemize}
\item $p\wedge q \equiv q \wedge p$\\$p\vee q \equiv q \vee p$ \tab \tab \tab(Kommutativgesetze)
\item $(p\wedge q)\wedge r \equiv p\wedge (q\wedge r)$\\$(p\vee q)\vee r \equiv p\vee (q \vee r)$ \tab \tab(Assoziativgesetze)
\item $(p\wedge q)\vee r \equiv (p \vee r) \wedge (q \vee r)$\\$(p\vee q)\wedge r \equiv (p\wedge r) \vee (q \wedge r)$ \tab(Distributivgesetze)
\item $p\wedge 1 \equiv p$, $p\vee 1 \equiv 1$, $p\wedge p \equiv p$\\$p\wedge 0 \equiv 0$, $p\vee 0 \equiv p$, $p\wedge p \equiv p$
\item $p \vee (p\wedge q ) \equiv p$ \tab \tab(Absorptionsgesetz)
\end{itemize}

\subsection{Aussagefunktionen, Quantoren, Prädikatenlogik} \label{subsec:Aussagefunktionen}
$X$ sei eine Menge (Gesamtheit von Objekten $x$ mit einem gemeinsamen Merkmal, vgl.  Abschnitt \ref{sec:Mengen})\\
$x\in X$ … $x$ ist Element von $X$. Die Objekte haben Eigenschaften (\emph{Prädikate})

\paragraph{Aussagefunktion} (auch Aussageform) $p(x)$:
Jedem $x\in X$ ist eine Aussage $p(x)$ zugeordnet. Dabei steht $x$ für ein Objekt, $p$ für ein Prädikat.
\subparagraph{Bsp. 5:}\parskp
$X$ … Menge der positiven natürlichen Zahlen (1, 2, 3, …)\\
$p(x):=$„$x$ ist eine Primzahl“\\
$p(5)$ … wahr, $p(10)$ … falsch

\paragraph{Quantoren:} \parskp
Betrachtet werden folgende Aussagen:
\begin{enumerate}
\item „Für alle $x$ (aus $X$) gilt $p(x)$“ $\equiv$ $\boxed{\forall x\; p(x)}$ (\emph{universeller Quantor} / Allquantor)
\item „Es existiert (wenigstens) ein $x$, für welches $p(x)$ gilt“ $\equiv$ $\boxed{\exists x \; p(x)}$ (\emph{existenzieller Quantor})
\end{enumerate}
Zur Schreibweise: 
\begin{itemize}
\item Bei Anwendungen (außerhalb der reinen Logik) wird oft die Grundmenke $X$ mit angegeben: \\
$\forall x \in X\; p(x)$ usw.
\item Falls sich Quantoren auf eine Teilmenge $M$ von $X$ beziehen sollen, dann können folgende Schreibweisen verwendet werden:\\
$a = \forall x \in M \; p(x)$, $b=\exists x\in M \; p(x)$.
\item Die Schreibweisen in der formalen Logik sind dann:\\
$a = \forall x \;(x \in M \Rightarrow p(x))$
\end{itemize}

\paragraph{Rechenregeln:}\parskp
$\boxed{\overline{\forall x \; p(x)} \equiv \exists x \; \overline{p(x)}}$\\
$\boxed{\overline{\exists x \; p(x)} \equiv \forall x \; \overline{p(x)}}$

\paragraph{Mehrstellige Aussagefunktionen}

\begin{itemize}
\item $p(x_1, x_2,\: ..., x_n)$,\quad $x_1 \in X_1, x_2 \in X_2,\: ... , x_n \in X_n$\\
Die Grundmengen $X_i$ können, müssen aber nicht für jede Stelle gleich sein.
\item Wird ein Quantor auf eine n-stellige Aussagefunktion angewandt, so entsteht eine (n-1)-stellige Aussagefunktion (eine 0-stellige Aussagefunktion ist eine Aussage)\\
z.B.: $\exists y \; p(x,y,z)=: q(x,z)$, die Variable $y$ wird durch den Quantor $\exists$ gebunden ($y$… gebundene Variable). Wichtig ist der Platz, nicht der Name der Variable.\\
$x, z$ … freie Variable, können durch weitere Quantoren gebunden werden.
\end{itemize}

\subparagraph{Bsp. 6:} \parskp
Ein Dorf bestehe aus 2 Teilen (Ober- und Unterdorf). Es sei $M$ die Menge aller Bewohner des Dorfes. $M_1$ bzw. $M_2$ seien die Teilmengen von $M$, die dem Ober- bzw. Unterdorf entsprechen. 

Wir betrachten folgende zweistellige Aussagefunktionen:\\
$k(x,y)$… Person $x$ (aus $M$) kennt Person $y$ (aus $M$)
\begin{enumerate} [label=\alph*)]
\item $a(x):= \forall y \; k(x,y)$… Person $x$ kennt jeden ($\Rightarrow$ „Für alle $y$ gilt: $x$ kennt $y$“)\\
$b(y):= \exists x \; k(x,y)$ … es gibt jemanden, der $y$ kennt\\
$c := \forall x \forall y \; k(x,y)$ … jeder kennt jeden\\
$d := \forall y \exists x \: k(x,y)$ … jeder wird von wenigstens einer Person gekannt\\
$e := \exists x \forall y \; k(x,y)$ … es gibt mindestens eine Person, die alle Personen kennt\\
Man beachte: \begin{itemize}
\item $d$ und $e$ sind nicht das Gleiche: Die Reihenfolge unterschiedlicher Quantoren muss beachtet werden. Bei $d$ kann für jedes $y$ ein anderes $x$ mit $k(x,y)$ existieren. Diese Abhängigkeit von $y$ wird manchmal in Anwendungen durch $\forall y \: \exists x(y) \; k(x,y)$ ausgedrückt.
\item Es gilt aber $e \Rightarrow d$ (stets wahr: Tautologie). Der Wahrheitsgehalt von z.B. $c, d, e$ kann dagegen nicht mit logischen Mitteln bestimmt werden.
\end{itemize}
\item Negation der Aussagen  bzw. Aussageformen aus a).\\
$\overline{a(x)}\equiv \exists y \; \overline{k(x,y)}$… $x$ kennt wenigstens eine Person nicht\\
$\overline{b(x)} \equiv \forall x \; \overline{k(x,y)}$ … keiner kennt $y$\\
$\overline{c} \equiv \exists x \; \overline{\forall y \; k(x,y)}\equiv \exists x \; \exists y \; \overline{k(x,y)}$ … es gibt jemanden der wenigstens eine Person nicht kennt (jemanden, der nicht alle kennt)\\
$\overline{d} \equiv \exists y \; \forall x \; \overline{k(x,y)}$… es gibt jemanden, der von keiner Person gekannt wird
$\overline{e}\equiv \forall x \; \exists y \; \overline{k(x,y)}$ … jeder kennt wenigstens eine Person nicht.
\item Folgende Aussagen sind mit Hilfe von Quantoren auszudrücken:\\
$f$… jeder aus dem Oberdorf kennt wenigstens eine Person aus dem Unterdorf.\\
$g$… es gibt jemanden im Unterdorf, der alle Personen des Oberdorfs kennt.
\begin{align*}f&=\forall x \in M_1 \; \exists y \in M_2 \; k(x,y)\\
&=\forall x \; (x \in M_1 \Rightarrow \exists y \; (y \in M_2 \wedge k(x,y)))\end{align*}
\begin{align*}
g&= \exists x \in M_2 \; \forall y \in M_1 \; k(x,y)\\
&= \exists x \; (x\in M_2 \wedge \forall y \; (y \in M_1 \Rightarrow k(x,y)))
\end{align*}
\end{enumerate}

\section{Mengen}\label{sec:Mengen}
%\subsection{Begriffe}

\paragraph{Menge:} Zusammenfassung gewisser wohl unterscheidbarer Objekte (Elemente) mit einem gemeinsamen Merkmal zu einem Ganzen.

\subparagraph{Diskussion:} Naiver Mengenbegriff führt zu Widerpsrüchen. z.B. Menge $X$ aller Mengen, die sich nicht selbst als Element enthalten.

$X=\{A | A \; Menge, A\not\in A\}$\\
$X\in X$? Wenn $X\in X \Rightarrow X\not\in X$ und $X\not\in X\Rightarrow X \in X$ (Widerspruch!).

Diese Widersprüche können umgangen werden, wenn nur Teilmengen einer sogenannten Grundmenge betrachtet werden.

\subparagraph{Bezeichungen:}
\begin{itemize}
\item meist große Buchstaben für Mengen: $A, B, ..., M, ...,X$
\item $\boxed{x\in M}$… $x$ ist Element von $M$
\item $\boxed{x\not\in M}$… $x$ ist kein Element von $M$
\end{itemize}

\subparagraph{Schreibweise:} \parskp
$M=\{\underset{\text{Elemente}}{...}\}$ oder $M = \{ x | p(x)\}$\\
mit $p(x)=$ Aussage, die genau für die Elemente $x$ aus $M$ wahr ist.

\paragraph{Wichtige Grundmengen:}
\begin{itemize}
\item $\mathbb{N}$ … Menge der natürlichen Zahlen $\{0,1,2,3,...\}$
\item $\mathbb{N}^*=\mathbb{N}\setminus \{0\} = \{1,2,3,...\}$
\item $\mathbb{Z}$ … Menge der ganzen Zahlen $\{...,-3,-2,-1,0,1,2,3,...\}$
\item $\mathbb{Q}$ … Menge der rationaln Zahlen $\{ x | x = \frac{m}{n}, m\in \mathbb{Z}, n \in \mathbb{Z}, n\not= 0\}$
\item $\mathbb{R}$ … Menge der reelen Zahlen
\item $\mathbb{C}$ … Menge der komplexen Zahlen $\{ z | z=x+ i \cdot  y,\quad x, y \in \mathbb{R}, i^2=-1\}$
\end{itemize}

\subparagraph{Bsp. 1:} \parskp
$M_1$ … Menge der Primzahlen kleiner 10, $M_1=\{2,3,5,7\}$\\
$M_2$ … Menge der reelen Zahlen zwischen 0 und 1 $M_2=\{x \in \mathbb{R}| 0<x<1\} =: \underset{\text{Intervallschreibweise}}{(0,1)}$

\paragraph{Def. 1:} (Intervallschreibweisen) \\
Es seien $a$ und $b$ reele Zahlen mit $a<b$:\\
$[a,b]:=\{ x \in \mathbb{R} | a \le x \le b\}$ … abgeschlossenes Intervall\\
$(a,b):= \{ x \in \mathbb{R} | a < x < b\}$ … offenes Intervall\\
$[a,b):= \{ x \in \mathbb{R}  | a \le x < b\}$\\
$(-\infty , a ) := \{ x \in \mathbb{R} | - \infty < x < a \} = \{x \in \mathbb{R} | x < a\}$\\
usw.

\subparagraph{Leere Menge:} z.B. $\{ x \in \mathbb{R} | x =x+1\}= \{x \in \mathbb{R} | x^2+1=0\}$ enthält kein Element.\\
Bezeichnung: $\emptyset$ oder $\{\}$

\subsection{Mengenverknüpfungen}

\paragraph{Def. 2:} \parskp
$\boxed{M_1 = M_2} : \equiv \boxed{\forall x \; (x\in M_1 \Leftrightarrow x \in M_2}$ (\emph{Gleichheit})
\paragraph{Def. 3:} \parskp
$\boxed{M_1 \subseteq M_2} : \equiv \boxed{\forall x \; (x\in M_1 \Rightarrow x \in M_2}$ (\emph{Inkulsion}) „$M_1$ ist Teilmenge von $M_2$“

\subparagraph{Diskussion:} \parskp
Ist $M_1 \subseteq M_2$ aber $M_1\not = M_2$ so kann man schreiben $M_1\subset M_2$ (echte Teilmenge).

\paragraph{Def. 4:}
\begin{enumerate}
\item \mpb 
$A\cap B := \{x | x \in A \wedge x \in B\}$\\
\emph{Durchschnitt} von $A$ und $B$
\mpe \mpb \begin{tikzpicture} [scale =0.25]
\node at (-7,3) {A};
\node at (1,3) {B};
\draw  (-5,3) ellipse (3 and 3);
\draw  (-1,3) ellipse (3 and 3);
\begin{scope} % lokal halten der Wirkung von \clip
\clip (-5,3) ellipse (3 and 3);
\draw[pattern= north east lines, pattern color=gray] (-1,3) ellipse (3 and 3);
\end{scope}
\end{tikzpicture} 
\mpe
\item \mpb
$A \cup B := \{ x | x\in A \vee x \in B\}$\\
\emph{Vereinigung} von $A$ und $B$
\mpe \mpb \begin{tikzpicture} [scale =0.25]
\draw [pattern= north east lines, pattern color=gray]  (-5,3) ellipse (3 and 3) (-1,3) ellipse (3 and 3);
\node at (-7,3) {A};
\node at (1,3) {B};
\end{tikzpicture}
\mpe
\item \mpb
$A \setminus B := \{ x | x \in A \wedge x \not\in B\}$\\
\emph{Differenz} „$A$ minus $B$“
\mpe \mpb \begin{tikzpicture} [scale =0.25, remember picture]
\draw  (-5,3) ellipse (3 and 3);
\draw  (-1,3) ellipse (3 and 3);
\begin{scope}
\begin{pgfinterruptboundingbox} % To make sure our clipping path does not mess up the placement of the picture
\draw [clip] (-5,3) ellipse (3 and 3) [reverseclip];
\end{pgfinterruptboundingbox}
\draw[pattern= north east lines, pattern color=gray] (-1,3) ellipse (3 and 3);
\end{scope}
\node at (-7,3) {A};
\node at (1,3) {B};
\end{tikzpicture} 
\mpe
\item \mpb[0.4] Bei Vorliegen einer Grundmenge $E$:\\
$\overline{A} := E \setminus A$\\
\emph{Komplimentärmenge} von $A$
\mpe \mpb \begin{tikzpicture} [scale =0.25, remember picture]
\draw  (-9,-1) rectangle (3,7);
\draw  (-5,3) ellipse (3 and 3);
\begin{scope}
\begin{pgfinterruptboundingbox} % To make sure our clipping path does not mess up the placement of the picture
\draw [clip]  (-5,3) ellipse (3 and 3) [reverseclip];
\end{pgfinterruptboundingbox}
\draw[pattern= north east lines, pattern color=gray](-9,-1) rectangle (3,7);
\end{scope}
\node at (-7,3) {A};
\node at (1,3) {E};
\end{tikzpicture}
\mpe
\end{enumerate}
\subparagraph{Diskussion:} (ausgewählte Rechenregeln)
\begin{enumerate}
\item $\cup$ und $\cap$ sind kommutativ und assoziativ\\
z.B. gilt $A \cup B = B\cup A$, $(A\cap B)\cap C = A \cap (B \cap C) = A \cap B \cap C$
\item Allg. $I$ … Indexmenge, z.B. $\{1,2,...,n\}$, $\mathbb{N}$, $\mathbb{Z}$, $\mathbb{R}$ dann:\\
$\bigcup\limits_{i\in I}A_i := \{x| \exists i \in I \quad x \in A_i\}$\\
$\bigcap\limits_{i\in I}A_i := \{x| \forall i \in I \quad x \in A_i\}$
\end{enumerate}

\subsection{Relationen}

\subsubsection{Grundbegriffe}

\paragraph{Def. 5:} \parskp
Die Menge $M_1 \times M_2 := \{ (x_1,x_2) | x_1 \in M_1 \wedge x_2 \in M_2\}$ heißt \emph{kartesisches Produkt} der Mengen $M_1$ und $M_2$ (= Menge aller geordneten Paare)

\subparagraph{Bsp. 2:} \parskp
$\mathbb{R}$ … Menge der reelen Zahlen, veranschaulicht durch die Zahlengerade\\
$\mathbb{R}^2 := \mathbb{R} \times \mathbb{R} = \{ (x,y) | x \in \mathbb{R} \wedge y \in \mathbb{R}\}$ … x-y-Ebene

\paragraph{Def. 6:} \parskp
Eine Teilmenge $T \subseteq M_1 \times M_2$ heißt \emph{(binäre) Relation}.

\subparagraph{Diskussion:}
\begin{enumerate}
\item Verallgemeinerung: $M_1 \times M_2 \times ... \times M_n= \{ (x_1, x_2, ..., x_n) | x_1 \in M_1, ..., x_n \in M_n\}$
(= Menge geordneter n-Tupel)\\
Eine Teilmenge $T\subseteq M_1 \times M_2 \times ... \times M_n$ heißt \emph{n-stellige Relation}.
\item \emph{Jede} Teilmenge von $M_1\times M_2$ ist eine Relation, also auch die Grenfälle $\emptyset$ (gesamt leere Menge) und $M_1\times M_2$ (vollständige Menge). Wichtig sind aber im allgemeinen die echten Teilmengen, die die verschiedensten Beziehungen zwischen den Elementen von $M_1$ und $M_2$ ausdrücken.
\end{enumerate}

\paragraph{Def. 7:}\label{Def. 7} (Eigenschaften binärer Relationen in $M_1\times M_2$)

Eine Relation $T\subseteq M_1\times M_2$ heißt:
\begin{enumerate}[label=\alph*)]
\item \emph{linksvollständig (linkstotal)}, wenn für jedes $x_1 \in M_1$ (wenigstens) ein $x_2 \in M_2$ existiert mit $(x_1, x_2) \in T$.
\item \emph{recthvollständig (rechtstotal}, wenn für jedes $x_2 \in M_2$ (wenigstens) ein $x_1 \in M_1$ existiert mit $(x_1,x_2) \in T$.
\item \emph{rechteindeutig}, wenn für jedes $x_1 \in M_1$ höchstens ein $x_2 \in M_2$ existiert mit $(x_1, x_2) \in T$.
\item \emph{linkseindeutig}, wenn für jedes $x_2 \in M_2$ höchstens ein $x_1 \in M_1$ existiert mit $(x_1, x_2) \in T$.
\end{enumerate}

\subparagraph{Bsp. 3:} \parskp
Es seien $S$ bzw. $L$ folgende Mengen von Städten bzw. Ländern:\\
$S=\{Berlin, \; Dresden, \; K\ddot{o}ln, \; Paris, \; Ram, \; Neapel, \; Oslo\}$\\
$L=\{D(eutschland), \; F(rankreich), \; B(elgien), \; I(talien), \; P(olen), \; N(orwegen)\}$\\
Die Relation $T\subseteq S \times L$ soll darstellen, welche Stadt in welchem Land liegt.\\
Man gebe $T$ elementweise an und stelle die Relation graphisch dar!\\
Welche der Eigenschaften aus Def. 7 treffen zu?
\begin{itemize}
\item $T=\{(Berlin, D), (Dresden, D), (K\ddot{o}ln, D), (Paris, F), (Rom, I), (Neapel, I), (Oslo, N)\}$
\item graphische Darstellung: \\
\begin{tikzpicture} [scale=0.25]

\draw  (0,2) rectangle (10,0) node[pos =.5]{Berlin};
\draw  (0,-1) rectangle (10,-3) node[pos =.5]{Dresden};
\draw  (0,-4) rectangle (10,-6) node[pos =.5]{Köln};
\draw  (0,-7) rectangle (10,-9) node[pos =.5]{Paris};
\draw  (0,-10) rectangle (10,-12) node[pos =.5]{Rom};
\draw  (0,-13) rectangle (10,-15) node[pos =.5]{Neapel};
\draw  (0,-16) rectangle (10,-18) node[pos =.5]{Oslo};


\draw  (18,2) rectangle (28,0) node[pos =.5]{D};
\draw  (18,-1) rectangle (28,-3) node[pos =.5]{F};
\draw  (18,-4) rectangle (28,-6) node[pos =.5]{B};
\draw  (18,-7) rectangle (28,-9) node[pos =.5]{P};
\draw  (18,-10) rectangle (28,-12) node[pos =.5]{I};
\draw  (18,-13) rectangle (28,-15) node[pos =.5]{N};

\node at (5,3) {S};
\node at (23,3) {L};
\draw[-latex] (10,1) -- (18,1);
\draw[-latex] (10,-2) -- (18,1);
\draw[-latex] (10,-5) -- (18,1);
\draw[-latex] (10,-8) -- (18,-2);
\draw[-latex] (10,-11) -- (18,-11);
\draw[-latex] (10,-14) -- (18,-11);
\draw[-latex] (10,-17) -- (18,-14);
\end{tikzpicture}\\
$(x,y) \in T: x\rightarrow y $ (gerichteter Graph)
\item Eigenschaften:\\
linksvollständig \\
nicht rechtsvollständig\\
rechtseindeutig\\
nicht linkseindeutig\\
(solche Relationen nennt man auch „Funktionen“, eindeutige Zuordnung [von Stadt $\rightarrow$ Land])
\end{itemize}

\paragraph{Def. 8:} (Eigenschaften binärer Relationen in $M\times M$)\\
Eine Relation $T\subseteq M\times M$ (Sprechweise auch „Relation auf M“) heißt…
\begin{enumerate} [label=\alph*)]
\item \emph{reflexiv}, wenn $(x,x) \in T$ für alle $x \in M$,
\item \emph{symmetrisch}, wenn $(x,y) \in T \Rightarrow (y,x) \in T$,
\item \emph{antisymmetrisch}, wenn $((x,y) \in T \wedge (y,x) \in T) \Rightarrow x=y$,
\item \emph{asymmetrisch}, wenn $(x,y) \in T \Rightarrow (y,x) \not\in T$, 
\item \emph{transitiv}, wenn $((x,y) \in T \wedge (y,z) \in T) \Rightarrow (x,z) \in T$
\end{enumerate}
… jeweils für \emph{alle} $x,y,z \in M$ gilt.

\subparagraph{Bsp. 4:} \parskp
Welche Eigenschaften aus Def. 8 besitzen folgende Relationen?\\
Es sei $P$ eine Menge von Personen.
\begin{enumerate} [label=\alph*)]
\item Eine Person $x\in P$ sei jünger als $y\in P$, wenn ihr Geburtstag später als der von $y$ ist.\\
$\curvearrowright J \subseteq P \times P$ mit $J=\{ (x,y) | x \text{ ist jünger als } y\}$.\\
$J$ ist offensichtlich asymmetrisch (damit auch antisymmetrisch [Die Prämisse der Implikation $((x,y) \in J \wedge (y,x) \in J) \Rightarrow x=y$ ist stets falsch, damit die Implikation stets wahr]) und transitiv. Eine solche Relation nennt man auch \emph{strikte Ordnungsrelation} (vgl. Abschnitt \ref{subsec:Ordnungsrelationen}).
\item Zwei Personon $x\in P$ und $y\in P$ heißen gleichaltrig, wenn $x$ und $y$ das gleiche Geburts\emph{jahr} besitzen.\\
$\curvearrowright G \subseteq P\times P$ mit $G= \{ (x,y) | x \text{ und } y \text{ sind gleichaltrig}\}$.\\
$G$ ist offensichtlich reflexiv, symmetrisch und transitiv.\\
Derartige Relationen nennt man \emph{Äquivalenzrelationen}, vgl. Abschnitt \ref{subsec:Äquivalenzrelationen}. Sie teilen $P$ in disjunkte sogenannte Äquivalenzklassen auf ($x$ äquivalent $y$ heißt, $x$ und $y$ besitzen gleiches Geburtsjahr).
\end{enumerate}
Graphische Darstellung von Relationen $T$ in $M\times M$ (auf $M$). Möglichkeiten:
\begin{enumerate}
\item Elemente von $M$ nur einmal darstellen, Pfeildarstellun wie bisher, bei $(x,x) \in T$ eine Schlinge zeichnen.\\
\begin{tikzpicture} [scale=0.35]
\draw (0,0) circle (1) node{x};
\draw [-latex] plot[smooth, tension=1.5] coordinates {(0.2,1) (0.5,2) (-0.5,2) (-0.2,1)};
\draw[-latex] (1,0) -- (3,0);
\draw (4,0) circle (1) node{y};
\draw[-latex] (5,0) -- (7,0);
\draw (8,0) circle (1) node{z};
\draw[-latex] (1,0) -- (3,0);
\draw [-latex] plot[smooth, tension=1] coordinates {(8,-1) (6,-2) (4,-1)};
\end{tikzpicture}\\
(gerichteter Graph)
\item 
\begin{tikzpicture}[scale=0.5]
% Achsen zeichnen
\draw[thick] (0,0) -- (2,0);
\draw[thick] (0,0) -- (0,2);
\draw[dashed] (1,0) -- (1,2);
\draw[dashed] (2,0) -- (2,2);
\draw[dashed] (0,1) -- (2,1);
\draw[dashed] (0,2) -- (2,2);

\fill[black] (0,0) circle (0.1) node[below left]{x};
\fill[black] (1,0) circle (0.1) node[below]{y};
\fill[black] (2,0) circle (0.1) node[below]{z};
\fill[black] (0,1) circle (0.1) node[left]{y};
\fill[black] (0,2) circle (0.1) node[left]{z};

\draw (0,1) circle (0.2);
\draw (0,0) circle (0.2);
\draw (1,2) circle (0.2);
\draw (2,1) circle (0.2);
\end{tikzpicture}\\
(Koordinatensystem)\\
Diese Variante ist auch bei Relationen in $M_1 \times M_2$ möglich.
\end{enumerate}

\subparagraph{Diskussion:}
\begin{enumerate}
\item Die Eigenschaften Reflexivität, Symmetrie und Transitivität lassen sich beim gerichteten Graphen leicht nachprüfen.\\
\emph{Reflexivität:} Bei jedem Element ist eine Schlinge.\\
\emph{Symmetrie:} Jeder Pfeil $x\rightarrow y \; (y \not = x)$ besitzt „umkehrpfeil“ ($x\leftarrow y$). \\
Antisymmetrie: Schlinge möglich, aber keine Umkehrpfeile.\\
Asymmetrie: weder Schlingen noch Umkehrpfeile.\\
\emph{Transitivität:} Falls ein Pfeil $x\rightarrow y $ eine „Fortsetzung“ $y\rightarrow z$ besitzt, so verläuft auch ein Pfeil von $x$ nach $z$.
\item Auch die Darsteellung von Koordinatensystem lassen sich die Eigenschaften Reflexivität und Symmetrie sofort überprüfen.\\
\emph{Reflexivität:} Die Diagonale $I_M = \{(x,x) | x \in M\}$ gehört zu $T$ ($I_M$ heißt auch \emph{Identitätsrelation}, diese Relation ist eine spezielle Funktion, identische Funktion $y=f(x))=x$, $x\in M$ später als Funktion auch mit $i_M$ bezeichnet)\\
\emph{Symmetrie:} $T$ ist spiegelsymmetrisch bzgl. $I_M$ \\
\begin{tikzpicture}[scale=0.5]
% Achsen zeichnen
\draw[thick] (0,0) -- (4,0);
\draw[thick] (0,0) -- (0,4);
\foreach \x in {0,1,2,3,4}
\draw[dashed] (\x,0) -- (\x,4);
\foreach \y in {0,1,2,3,4}
\draw[dashed] (0,\y) -- (4,\y);
%Achsen beschriften
\fill[black] (0,0) circle (0.1) node[below left]{$x_1$};
\foreach[count=\x] \i in {2,3,4,5}
\fill[black] (\x,0) circle (0.1) node[below]{$x_\i$};
\foreach[count=\y] \i in {2,3,4,5}
\fill[black] (0,\y) circle (0.1) node[left]{$x_\i$};

\draw (0,0) circle (0.2);
\draw (1,1) circle (0.2);
\draw (2,2) circle (0.2);
\draw (3,3) circle (0.2);
\draw (4,4) circle (0.2);
\draw (1,3) circle (0.2);
\draw (2,1) circle (0.2);
\draw (4,2) circle (0.2);
\draw (3,0) circle (0.2);

\fill[red] (0,0) circle (0.1);
\fill[red] (1,1) circle (0.1);
\fill[red] (2,2) circle (0.1);
\fill[red] (3,3) circle (0.1);
\fill[red] (4,4) circle (0.1) node[above right]{$I_M$};
\end{tikzpicture}
ist reflexiv aber nicht symmetrisch\\
\begin{tikzpicture}[scale=0.5]
% Achsen zeichnen
\draw[thick] (0,0) -- (4,0);
\draw[thick] (0,0) -- (0,4);
\foreach \x in {0,1,2,3,4}
\draw[dashed] (\x,0) -- (\x,4);
\foreach \y in {0,1,2,3,4}
\draw[dashed] (0,\y) -- (4,\y);
%Achsen beschriften
\fill[black] (0,0) circle (0.1) node[below left]{$x_1$};
\foreach[count=\x] \i in {2,3,4,5}
\fill[black] (\x,0) circle (0.1) node[below]{$x_\i$};
\foreach[count=\y] \i in {2,3,4,5}
\fill[black] (0,\y) circle (0.1) node[left]{$x_\i$};

\draw (0,0) circle (0.2);
\draw (1,1) circle (0.2);
\draw (3,3) circle (0.2);
\draw (1,3) circle (0.2);
\draw (3,1) circle (0.2);
\draw (0,4) circle (0.2);
\draw (4,0) circle (0.2);
\draw (1,4) circle (0.2);
\draw (4,1) circle (0.2);

\end{tikzpicture}
ist symmetrisch aber nicht reflexiv
\end{enumerate}
\emph{Alternative Schreibweisen:} Es sei $T \subseteq M_1 \times M_2$ eine binäre Relation. \\
Anstelle $\boxed{ (x,y) \in T }$ kann man schreiben:
\begin{itemize}
\item $x T y$ ($x$ steht in Relation $T$ zu $y$), für viele wichtige Relationen gibt es spezielle Zeichen, z.B. $x<y$, $x=y$, $g||h$ oder $A \subseteq B$ usw.
\item Aussageformen (vgl. Prädikatenlogik): $\boxed{T(x,y)}$ (auch mit mehreren Variablen möglich)
\end{itemize}

\subsubsection{Operationen auf Relationen}

Da Relationen spezielle Mengen sind, gibt es Operationen wie $\cup$, $cap$ usw. auch hier. Weitere für Relationen wichtige Operationen in den folgenden Definitionen:

\paragraph{Def. 9:} \parskp
Es sein $T$ eine Relation in $U\times V$.\\
Die Menge $proj_1(T)=\{x\in U | \exists y \in V, (x,y) \in T\}$ heißt \emph{Projektion} von $T$ auf $u$ (1. Faktor des kartesischen Produkts).\\
Analog ist $proj_2(T)=\{y\in V | \exists x \in U, (x,y) \in T\}$ die Projektion auf den 2. Faktor.

\emph{Veranschaulichung:}\\
\begin{tikzpicture} [scale = 0.25]
%Grenze
\draw  (0,0) rectangle (10,8);
\draw (10,0) node [below left] {$u$};
\draw (0,8) node [below left] {$v$};
%Gebilde
\fill[pattern=north west lines,pattern color=gray] (3,2) to [out=-90, in=180] (4,1) to [out=0, in=-160] (7,3) to [out=20,in=-90] (8,4) to [out=90,in=0] (6,6) to [out=180, in=80] (5,4) to [out=-100, in=90] (3,2);
\draw (3,2) to [out=-90, in=180] (4,1) to [out=0, in=-160] (7,3) to [out=20,in=-90] (8,4) to [out=90,in=0] (6,6) to [out=180, in=80] (5,4) to [out=-100, in=90] (3,2);
\draw (6.5,4) node{$T$};

%Projektion
\draw[dashed, red] (3,2) -- (3,0);
\draw[dashed, red] (8,4) -- (8,0);
\draw[red] (3,-0.1)--(5.5, -0.1) node[below]{$proj_1(T)$}--(8,-0.1);

\draw[dashed, orange] (4,1) -- (10,1);
\draw[dashed, orange] (6,6) -- (10,6);
\draw[orange] (10.1,1)--(10.1, 3) node[right]{$proj_2(T)$}--(10.1,6);
\end{tikzpicture}

\subparagraph{Bsp. 5:} \parskp
Es sei $S=\{1,2,3,4,5\}$ eine Menge von Studenten und $F=\{a,b,c,d,e,f\}$ eine Menge von Fächern.\\
Es sei $P\subseteq S \times F$ die Relation, die angibt, welcher Student in welchem Fach eine Nach- bzw. Wiederholungsprüfung im bevorstehenden Prüfungsabschnitt hat.\\
Die Studenten $1$ und $3$ haben keine Prüfung ausstehen, Student $2$ muss die Prüfungen in $a$. $d$ und $e$, $4$ in $b$ und $f$ sowie 5 in $b$, $d$, $e$ und $f$ ablegen.
\begin{enumerate}[label=\alph*)]
\item Man gebe die Relation $P$ elementweise an und stelle sie in einem Koordinatensystem dar.
\item Man ermittle die Projektionen $P$ auf $S$ bzw. $F$ und kennzeichne diese in der Skizze.
\end{enumerate}
Lösung:
\begin{enumerate}[label=\alph*)]
\item $P=\{(2,a), (2,d), (2,e), (4,b), (4,f),(5,b), (5,d), (5,e), (5,f)\}$\\
\begin{tikzpicture} [scale=0.6]
% Bereich
\draw (0,0) rectangle (4,5);
\draw (0,5) node [above]{Fach};
\draw (4,0) node[right]{Student};
\foreach [count=\i] \x in {0,1,2,3,4}
\draw (\x,-.1) -- (\x,.1) node[below=5pt] {$\i$};
\draw (-.1,0) -- (.1,0) node[left=5pt] {$a$};
\draw (-.1,1) -- (.1,1) node[left=5pt] {$b$};
\draw (-.1,2) -- (.1,2) node[left=5pt] {$c$};
\draw (-.1,3) -- (.1,3) node[left=5pt] {$d$};
\draw (-.1,4) -- (.1,4) node[left=5pt] {$e$};
\draw (-.1,5) -- (.1,5) node[left=5pt] {$f$};

%Punkte
\draw (1,0) circle (0.2);
\draw (1,3) circle (0.2);
\draw (1,4) circle (0.2);

\draw (3,1) circle (0.2);
\draw (3,5) circle (0.2);

\draw (4,1) circle (0.2);
\draw (4,3) circle (0.2);
\draw (4,4) circle (0.2);
\draw (4,5) circle (0.2);

%Projektion
\draw[orange, latex-] (4.2,2.5) -- (5.2,2.5);
\draw[orange] (-0.6,0) circle (0.3);
\draw[orange] (-0.6,1) circle (0.3);
\draw[orange] (-0.6,3) circle (0.3);
\draw[orange] (-0.6,4) circle (0.3);
\draw[orange] (-0.6,5) circle (0.3);
\draw[orange] (-0.6, 2.5) node[left]{$proj_2(P)$};

\draw[red, latex-] (2,5.2) -- (2,6.2);
\draw[red] (1,-0.6) circle (0.3);
\draw[red] (3,-0.6) circle (0.3);
\draw[red] (4,-0.6) circle (0.3);
\draw[red] (2, -0.7) node[below]{$proj_1(P)$};
\end{tikzpicture}
\item 
$proj_1(P)=\{2,4,5\}\subseteq S$\\
(= Menge der Studenten, die wenigsten eine N/W-Prüfung haben.)\\
$proj_2(P)=\{a,b,d,e,f\}\subseteq F$\\
(= Menge der Fächer, in denen Student(en) eine N/W-Prüfung haben.)
\end{enumerate}

\paragraph{Def. 10:} \parskp
Es sei $T \subseteq M_1 \times M_2$ eine binäre Relation. \\
Die Relation $T^{-1}:=\{(y,x)|(x,y)\in T\} \subseteq M_2\times M_1$ heißt \emph{inverse Relation} (bzw. kurz: \emph{Inverse}) von $T$.

\subparagraph{Bsp. 6:} (vgl. Bsp. 5)\\
$P^{-1}=\{(a,2), (b,4), (b,5), (d,2), (d,5), (e,2), (e,5), (f,4), (f,5)\}$\\
Besonders wichtig ist die folgende Operation:

\paragraph{Def. 11:} \parskp
Es seien $T_1 \subseteq M_1\times M_2$ und $T_2 \subseteq M_2 \times M_3$ binäre Relationen.\\
Als \emph{Komposition} (oder auch \emph{Verkettung}) $T_1 \circ T_2$ („$T_2$ nach $T_1$“) wird die Relation $T_1 \circ T_2 := \{ (x,z) \in M_1 \times M_3 | \exists y \in M_2 \quad (x,y) \in T_1 \wedge (y,z) \in T_2\}$ in $M_1 \times M_3$ bezeichnet.

\subparagraph{Bsp. 7:} \parskp
Es sei $M$ die Menge aller Menschen, die zu einem bestimmten Zeitpunkt leben. Weiter seien $S=\{(x,y)| x \text{ ist Mutter von }y\} \subseteq M\times M$ und $T=\{(y,z)| y \text{ ist verheiratet mit }z\}\subseteq M\times M$.

Dann bedeutet $(x,z)\in S\circ T$: Es gibt ein $y$, sodass $x$ die Mutter von $y$ ist ($(x,y)\in S$) und $y$ mit $z$ verheiratet ($(y,z)\in T$) ist, d.h. „$x$ ist die Schwiegermutter von $z$“.

\subparagraph{Diskussion:} Wichtige Eigenschaft der Komposition $\circ$:
\begin{itemize}
\item Die Operation $\circ$ ist \emph{assoziativ}, d.h. seien $T_1\subseteq A\times B$, $T_2\subseteq B\times C$ und $T_3 \subseteq C\times D$, dann gilt:\\
$(\underbrace{T_1 \circ T_2}_{\subseteq A\times C})\underset{\subseteq C \times D}{\circ T_3} = \underset{\subseteq A \times B}{T_1} \circ (\underbrace{T_2 \circ T_3}_{\subseteq B \times D}) = T_1 \circ T_2 \circ T_3 \subseteq A \times D$
\end{itemize}

\paragraph{Def. 12:} \parskp
Es sei $T$ eine Relation in $M \times M$ (auf $M$).\\
Als \emph{transitive Hülle} $T^+$ von $T$ bezeichnet man die kleinste Relation, die $T$ enthält und transitiv ist.
\subparagraph{Satz 1:} Es gilt: $\boxed{T^+ = T \cup (T\circ T) \cup (T\circ T \circ T) \cup ...}$
\subparagraph{Bemerkung:} \parskp
Bezeichnung für $\underbrace{T\circ T \circ ...\circ T}_{\text{n-mal}}$ auch $T^n$\\
(Nicht verwechseln mit Mengenprodukt $\underbrace{T\times ... \times T }_{\text{n-mal}}$ bzw. Funktionen mit n-ten Potenz $f^n$!)

Damit ist $\boxed{T^+= \bigcup^{\infty}_{j=1}T^j}$

\subparagraph{Beweis:} 
\begin{enumerate}
\item $T^+$ ist transitiv, denn sei $(x,y)\in T^+$ und $(y,z)\in T^+$, dann existieren natürliche Zahlen $j_1, j_2 \geq 1$ mit $(x,y)\in T^{j_1}$ und $(y,z)\in T^{j_2}$, \\
d.h. $y$ wird in $j_1$ Schritten von $x$ aus erreicht und $z$ in $j_2$ Schritten von $y$ aus erreicht. Also wird $z$ in $j_1 + j_2 $ Schritten von $x$ aus erreicht, \\
d.h. $(x,z) \in T^{j_1+j_2} \subseteq T^+$
\item Es sei $T\subseteq S$ für eine transitive Relation $S$.\\
$\Rightarrow T\circ T \subseteq S \circ S \subset S$ und für beliebiges $j\geq 1$:\\
$T^j \subseteq S^j \subseteq S$ und somit:\\
$T^+ = \bigcup^{\infty}_{j=1}T^j \subseteq S$, \\
d.h. $T^+$ ist tatsächlich die kleinste transitive Relation, die $T$ enthält.
\end{enumerate}
\subparagraph{Diskussion:}
\begin{enumerate}
\item Analog zur transitiven Hülle einer Relation $T$ in $M\times M$ (auf $M$) werden die reflexive Hülle bzw. die symmetrische Hülle von $T$ als die jeweils kleinsten Relationen die T enthalten und reflexiv bzw. symmetrisch sind definiert.\\
Die Ermittlung gestaltet sich etwas „einfacher“ als bei der transitiven Hülle:\\
\emph{Reflexive Hülle} von $T$: $\boxed{T\cup I_M}$ (dabei ist $I_M=\{(x,x)|x\in M\}$ [Diagonale / Identitätsrelation])\\
\emph{Symmetrische Hülle} von $T$: $\boxed{T \cup T^{-1}}$
\item Von Bedeutung ist auch die \emph{reflexiv-transitive} Hüllo ven $T$:\\
$\boxed{T^*=T^+\cup I_M}$ (dabei $T^+$ … transitive Hülle von $T$)
\end{enumerate}

\subparagraph{Bsp. 8:} \parskp
Gegeben sei die Menge $M=\{a,b,c,d,e,f\}$ \\
sowie die Relation $T=\{(a,b), (b,c), (c,e), (b,d), (d,e), (e,f)\}$.
\begin{enumerate}[label=\alph*)]
\item \emph{Transitive Hülle}: Zur Ermittlung der Komposition $S\circ T$: \\
Für jedes Element $(x,y)\in S$ alle Fortsetzungen $(y,z) \in T$ suchen $\curvearrowright (x,z)$ als Element von $S\circ T$ notieren, falls es noch nicht vorkommt.\\
Bspw.: 
\begin{itemize}
\item $(a,b)$, Fortsetzungen wären $(b,c), (b,d)$ $\curvearrowright$ Elemente $(a,c)$ und $(a,d)$ notieren.
\item $(b,c)$, Fortsetzung $(c,e)$ $\curvearrowright$ $(b,e)$ notieren
\item usw.
\end{itemize}
$\Rightarrow T\circ T=\{(a,c),(a,d),(b,e),(c,f),(d,f)\}=T^2$\\
$T^3=T \circ (T \circ T)=\{(a,e), (b,f)\}$ (ausgehend von $T$ in $T\circ T$ nach Fortsetzung suchen)\\
$T^4=T\circ T^3 = \{(a,f)\}$\\
\begin{tikzpicture}[scale=.5]
% Menge T
\draw (0,0) node{a};
\draw[-latex] (0.3,0.3) -- (1.7,1.7);
\draw (2,2) node{b};
\draw[-latex] (2.3,2) -- (3.7,2);
\draw (4,2) node{c};
\draw[-latex] (4.3,2) -- (5.7,2);
\draw (4,0) node{d};
\draw[-latex] (2,1.7) -- (3.7,0.3);
\draw (6,2) node{e};
\draw[-latex] (4.3,0.3) -- (5.7,1.7);
\draw (8,2) node{f};
\draw[-latex] (6.3,2) -- (7.7,2);


\draw [-latex, red] plot[smooth, tension=0.9] coordinates {(0,0.3) (1.3,2.7) (3.7,2.3)};
\draw [-latex, red] plot[smooth, tension=0.9] coordinates {(0.3,0) (2,-0.5) (3.7,0)};
\draw [-latex, red] plot[smooth, tension=0.9] coordinates {(2.3,2.3) (4,2.8) (5.7,2.3)};
\draw [-latex, red] plot[smooth, tension=0.9] coordinates {(4.3,2.3) (6,2.8) (7.7,2.3)};
\draw [-latex, red] plot[smooth, tension=0.9] coordinates {(4.3,0) (6,0.5) (8,1.7)};

\draw [-latex, green] plot[smooth, tension=0.9] coordinates {(0,0.3) (2.3,3) (5.8,2.5)};
\draw [-latex, green] plot[smooth, tension=0.9] coordinates {(2.3,1.7) (5,1.3) (7.7,1.7)};

\draw [-latex, orange] plot[smooth, tension=0.9] coordinates {(0.3,-0.3) (5,-0.7) (8.1,1.7)};
\end{tikzpicture}\\
$\Rightarrow T^+=T \cup \underbrace{(T\circ T)}_{\text{\textcolor{red}{2 Schritte}}} \cup \underbrace{(T \circ T \circ T)}_{\text{\textcolor{green}{3 Schritte}}} \cup \underbrace{(T\circ T \circ T \circ T}_{\text{\textcolor{orange}{4 Schritte}}}=T \cup T^2 \cup T^3 \cup T^4$ \medskip\\
(Formel bricht im endlichen Fall nach endlich vielen Schritten ab.)
\item \emph{Reflexive Hülle}: $T\cup \{(a,a), (b,b), (c,c), (d,d), (e,e), (f,f)\}$\\
\begin{tikzpicture}[scale=.5]
% Menge T
\draw (0,0) node{a};
\draw[-latex] (0.3,0.3) -- (1.7,1.7);
\draw (2,2) node{b};
\draw[-latex] (2.3,2) -- (3.7,2);
\draw (4,2) node{c};
\draw[-latex] (4.3,2) -- (5.7,2);
\draw (4,0) node{d};
\draw[-latex] (2,1.7) -- (3.7,0.3);
\draw (6,2) node{e};
\draw[-latex] (4.3,0.3) -- (5.7,1.7);
\draw (8,2) node{f};
\draw[-latex] (6.3,2) -- (7.7,2);

\draw [-latex, orange] plot[smooth, tension=1.5] coordinates {(0.2,0.2) (0.5,1) (-0.5,1) (-0.2,0.2)};
\draw [-latex, orange] plot[smooth, tension=1.5] coordinates {(2.2,2.2) (2.5,3) (1.5,3) (1.8,2.2)}; 
\draw [-latex, orange] plot[smooth, tension=1.5] coordinates {(4.2,2.2) (4.5,3) (3.5,3) (3.8,2.2)};\draw [-latex, orange] plot[smooth, tension=1.5] coordinates {(4.2,0.2) (4.5,1) (3.5,1) (3.8,0.2)};
\draw [-latex, orange] plot[smooth, tension=1.5] coordinates {(6.2,2.2) (6.5,3) (5.5,3) (5.8,2.2)};
\draw [-latex, orange] plot[smooth, tension=1.5] coordinates {(8.2,2.2) (8.5,3) (7.5,3) (7.8,2.2)};
\end{tikzpicture}
\item \emph{Symmetrische Hülle}: $T\cup T^{-1}=T\cup \{ (b,a), (c,b), (e,c), (d,b), (e,d), (f,e)\}$\\
\begin{tikzpicture}[scale=.5]
% Menge T
\draw (0,0) node{a};
\draw[-latex] (0.3,0.3) -- (1.7,1.7);
\draw (2,2) node{b};
\draw[-latex] (2.3,2) -- (3.7,2);
\draw (4,2) node{c};
\draw[-latex] (4.3,2) -- (5.7,2);
\draw (4,0) node{d};
\draw[-latex] (2,1.7) -- (3.7,0.3);
\draw (6,2) node{e};
\draw[-latex] (4.3,0.3) -- (5.7,1.7);
\draw (8,2) node{f};
\draw[-latex] (6.3,2) -- (7.7,2);

\draw[latex-, orange] (0.2,0.5) -- (1.6,2);
\draw[latex-, orange] (2.2,2.2) -- (3.6,2.2);
\draw[latex-, orange] (4.2,2.2) -- (5.6,2.2);
\draw[latex-, orange] (1.9,1.5) -- (3.3,0.3);
\draw[latex-, orange] (4.6,0.3) -- (5.9,1.5);
\draw[latex-, orange] (6.2,2.2) -- (7.6,2.2);
\end{tikzpicture}
\end{enumerate}
Zur Überprüfung der Eigenschaften aus Def. 8 ist folgender Satz nützlich:
\subparagraph{Satz 2:} \parskp
Es sei $T \subseteq M \times M$ eine binäre Relation. Dann gilt:
\begin{enumerate} [label=\alph*)]
\item $T$ ist reflexiv $\Leftrightarrow I_M \subseteq T$  ($I_M$ … Identitätsrelation)
\item $T$ ist symmetrisch $\Leftrightarrow T^{-1}\subseteq T \quad [\Leftrightarrow T^{-1} = T] $
\item $T$ ist antisymmetrisch $\Leftrightarrow T \cap T^{-1} \subseteq I_M$
\item $T$ ist asymmetrisch $\Leftrightarrow T \cap T^{-1} = \emptyset$
\item $T$ ist transitiv $\Leftrightarrow T \circ T \subseteq T$
\end{enumerate}

\subparagraph{Disskussion:}
\begin{enumerate}
\item Beweise ergeben sich unmittelbar aus Def. 8, vgl. Übungsaufgabe 1.24 (für b) und e))
\item Aus c) und d) ergibt sich z.B.\\
$\boxed{T \text{ asymmetrisch}}\Rightarrow \boxed{T \text{ antisymmetrisch}}$ (da $\emptyset$ Teilmenge jeder Menge ist)
\end{enumerate}

\subsubsection{Äquivalenzrelationen} \label{subsec:Äquivalenzrelationen}
\paragraph{Def. 13:} \parskp
Eine Relation $T \subseteq M \times M$ heißt \emph{Äquivalenzrelation} auf $M$, wenn sie reflexiv, symmetrisch und transitiv ist.
\subparagraph{Diskussion:}
\begin{enumerate}
\item Durch eine Äuivalenzrelation wird $M$ vollständig in paarweise elementfremde (disjunkte) \emph{Äquivalenklassen} zerlegt. Die Menge aller Äquivalenzklassen von $M$ bezüglich $T$ heißt \emph{Quotientenmenge} $M/T$.\\
Aufgrund der 3. Eigenschaft aus Def. 13 erhält eine Äquivalenzklasse alle Elemente, die untereinander erreichbar sind (=äquivalent) und nur diese.
\item Äquivalenzklassen enthalten alle Elemente, die bezüglich einer bestimmten Eigenschaft nicht unterscheidbar sind, z.B. Bsp. 4 mit $M=P$ (Menge von Personen), Äquivalenzrelation $G\subseteq P \times P$ mit $G=\{ (x,y) | x \text{ und } y \text{ haben gleiches Geburtsjahr}\}$, Äquivalenzklassen sind die Jahrgänge.
\item Anstelle der Schreibweise $(x,y) \in T$, $xTy$ oder $T(x,y)$ verwendet man bei beliebigen Äquivalenzrelationen auf $x\sim y$. Bei vielen speziellen Äquivalenzrelationen spezielle Symbole, sie folgendes Beispiel.
\end{enumerate}

\subparagraph{Bsp. 9:}
\begin{enumerate}[label=\alph*)]
\item $M$ sei eine beliebige Menge $T_1=I_M=\{ (x,y) \in M\times M | x=y\}$ (Identitätsrelation) ist eine Äquivalenzrelation.\\
Äquivalent heißt hier gleich! \\
Äquivalenzklassen sind sämtliche einelementige Teilmengen $\{x\}, x\in M$. $T_1$ heißt die feinste Zerlegung von $M$ die möglich ist. Die größte Zerlegung liefert die Relation $T_2=M\times M$, die trivialer Weise ebenfalls eine Äquivalenzrelation ist mit nur einer Äquivalenzklasse $M$. Für die Anwendungen sind natürlich Relationen wichtig, die eine feinere Zerlegung liefern.
\item $M = \mathbb{Z}$ (ganze Zahlen), $m\in \mathbb{N}^*$, $T\subseteq \mathbb{Z}\times \mathbb{Z}$ mit
\begin{itemize}
\item $(x,y) \in T : \equiv$ „$x$ und $y$ lassen bei Division durch $m$ den gleichen Rest“
\item Bezeichunung $x \equiv y (mod\; m)$ … $x$ kongruent $y$ (modulo $m$), z.B. $29 \equiv 8 (mod\; 7)$
\item $T$ ist eine Äquivalenzrelation auf $\mathbb{Z}$, Äquivalenzklassen:\\
Restklassen modulo $m$ (siehe Übungsaufgabe 1.19)
\end{itemize}
\item $M$ … Menge aller Geraden einer Ebene, $T\subseteq M \times M$ mit
\begin{itemize}
\item $(x,y) \in T :\equiv$ „$x$ ist zu $y$ parallel“, Bezeichnung: $x||y$\\
$\curvearrowright T$ ist Äquivalenzrelation auf $M$ (siehe Übungsaufgabe 1.21.)
\end{itemize}
\end{enumerate}

\subsubsection{Ordnungsrelationen} \label{subsec:Ordnungsrelationen}
\paragraph{Def. 14:}
\begin{enumerate}[label=\alph*)]
\item Eine Relation $T\subseteq M\times M$ heißt Ordnungsrelation auf $M$, wenn sie reflexiv, antisymmetrisch und transitiv ist.
\item Eine Ordnungsrelation heißt \emph{vollstandig} oder \emph{linear}, wenn für alle $x,y \in M$ gilt $(x,y)\in T \vee (y,x) \in T$.
\end{enumerate}

\paragraph{Def. 15:} \parskp
Eine Relation $T\subseteq M \times M$ heißt \emph{strikte Ordnungsrelation} auf $M$, wenn sie asymmetrisch und transitiv ist. Eine strikte Ordnungsrelation heißt vollständig, wenn für alle $x,y \in M$ mit $x\not = y$ gilt $(x,y) \in T \vee (y,x) \in T$.

\subparagraph{Bsp. 10:}
\begin{enumerate} [label=\alph*)]
\item $M = \mathbb{R}$, $T\subseteq \mathbb{R} \times \mathbb{R}$ mit $\boxed{(x,y) \in T :\equiv x \leq y}$ ist eine vollständige Ordnungsrelation auf $\mathbb{R}$.
\item Die Relation „$<$“ ist eine (vollständige) strikte Ordnungsrelation.
\item $E$ sei eiine Menge, $M$ sei die \emph{Menge aller Teilmengen von $E$}, d.h. $M$ ist die Potenzmenge $M=\mathcal{P}(E)$ von $E$.\\
$T \subseteq M \times M$ mit $\boxed{(A,B)\in T :\equiv A \subseteq B}$ ist eine Ordnungsrelation auf $\mathcal{P}(E)=M$ (Inklusion).
\end{enumerate}

\subparagraph{Diskussion:}
\begin{enumerate}
\item In der Literatur wird manchmal die Relation im Sinne von Def. 14 als Halbordnung und nur eine vollständige als Ordnung als Ordnungsrelation bezeichnet.
\item Zu jeder Ordnung $T_1$ (auf $M$) gehört eine strikte Ordnung $T_2$ und umgekehrt: $T_2=T_1\backslash I_M$ bzw. $T_1=T_2 \cup I_M$ ($T_1$ eist die reflexive Hülle von $T_2$), z.B. ($\leq$, $<$) oder ($\subseteq$, $\subset$).
\item Die Symbole $\leq$ (bzw. $<$) können anstelle der Paarschreibweise auch bei beliebigen Ordnungen verwendet werden, falls keine anderen Zeichen dafür üblich sind.
\end{enumerate}

\paragraph{Def. 16:} \parskp
$T$ sei eine Ordnungsrelation auf eine Menge $M$. Weiter sei $A$ eine Teilmenge von $M$.
\begin{enumerate} [label=\alph*)]
\item Ein Element $a \in M$ heißt obere Schranke von $A$, wenn gilt:\\
$\forall x \in A \quad x \leq a \quad (x \leq a$ d.h. $(x,a) \in T$, vgl. 3.) der vorhergehenden Diskussion)
\item Es sei $B$ die Menge der oberen Schranken von $A$, diese sei nicht leer. Falls es eine \emph{kleinste obere Schranke $s$} von $A$ gibt, d.h. $\exists s \in B \quad \forall b \in B \quad s \leq b$, so heißt $s$ das \emph{Supremum von $A$}, $\boxed{s = sup \, A}$
\item Gilt $s \in A$, so heißt $s$ das Maximum von $A$: $s = max \, A = sup \, A$
\item Ein Element $m \in A$ heißt maximal, wenn es kein größeres Element in $A$ gibt, d.h. $\forall x \in A \; (m \leq x \Rightarrow m=x)$
\end{enumerate}

\subparagraph{Diskussion:} 
\begin{enumerate}
\item Die Begriffe aus Def. 16 lassen sich auf strikte Ordnungen $S$ übertragen, indem anstelle von $S$ die reflexive Hülle $T=S \cup I_M$ verwendet wird.
\item Bei Ordnungsrelationen $T$ (auch für strikte Ordnungen) auf endlichen Mengen $M$ kann ein vereinfachter Graph, das sogenannte \emph{HASSE-Diagramm}, betrachtet werden.\\
\begin{tikzpicture}[scale = 0.5]
\draw[-latex] (0,0) node[left] {a} -- (2,0) node[right]{b};
\end{tikzpicture}
$(a\not = b)$ bedeutet $(a,b) \in T$ und es gibt kein Zwischenglied $c \not = a$ und $c\not = b$ mit $(a,c) \in T \wedge (c,b) \in T$ ($a$ ist unmittelbarer Vorgänger von $b$ bzw. $b$ Nachfolger von $a$).\\
Diesem Diagramm entspricht eine Teilrelation $U \subseteq T$, deren transitiv-reflexive Hülle (bzw. transitive Hülle bei strikten Ordnungen) $T$ ist.
\item Veranschaulichung von Def. 16 mit einem HASSE-Diagramm einer nicht vollständingen Ordnung (nicht linear)\\
\begin{tikzpicture}[scale=.5]
% Menge T
\draw (0,0) node{a};
\draw[-latex] (0.3,0) -- (1.7,0);
\draw (2,0) node{b};
\draw[-latex] (2.3,0.1) -- (3.7,1);
\draw[-latex] (2.3,-0.1) -- (3.7,-1);
\draw (4,1) node{c};
\draw (4,-1) node{d};
\draw[-latex] (4.3,-1) -- (5.9,0.7);
\draw[-latex] (4.3,1) -- (5.7,1);
\draw (6,1) node{e};
\draw[-latex] (6.3,1) -- (7.7,1);
\draw[-latex] (6.3,0.7) -- (7.7,-1);
\draw (8,1) node{f};
\draw (8,-1) node{g};

\draw (-1,-2) rectangle (9,2) node[right] {$M$};
\draw[gray] (-0.5,-1.5)node[above right]{$A$} rectangle (4.5,1.5);

\end{tikzpicture}\\
z.B. Arbeitsgänge, die in einer bestimmten Reihenfolge durchgeführt werden müssen, $A$ bspw. Teilarbeiten einer Zweigfirma\\
\emph{obere Schranken:} $e,f,g$\\
\emph{$sup\, A$}$=e$\\
\emph{Maximum von $A$:} existiert nicht, da $e \not \in A$\\
\emph{maximale Elemente von $A$:} $c,d$
\item Bei nichtlinearen Ordnungen müssene obere Schranken, Supremum und Maximum nicht existieren, es kann mehrere maximale Elemente $A\subseteq M$ geben.\\
Bei linearen Ordungen auf \emph{endlichen} Mengen gibt es genau ein maximales Element $=max\, A =max\, B$
\item Analog zur Def. 16 werden die Begriffe \emph{untere Schranken} $a$ von $A$ ($\forall x \in A \quad a \leq x$), \emph{größte untere Schranke} (\emph{Infinum}) $s$ von $A$ ($B\not = \emptyset$ … Menge der unteren Schranken, $\exists s \in B \quad \forall a \in B \forall a \leq s$), \emph{Minimum von $A$} ($min \, A = inf\, A = s$ falls $s \in A$) und minimales Element $m$ von $A$ ($\forall x \in A \quad (x\leq m \Rightarrow x = m)$) definiert.
\subparagraph{Bsp. 11:} \parskp
Eine bestimmte Arbeitsaufgabe besteht aus mehreren Arbeitsgängen. \\
Es sei $A=\{1,2,3$,$4,5,6\}$ die Menge der Arbeitsgänge. Die Arbeitsgänge $\{2,3,5\}=:S$ werden von einer Subfirma durchgeführt. Für die Reihenfolge gilt: 1 muss vor 2, 2 vor 3 und 5, 3 vor 4 sowie 5 vor 6 durchgeführt werden.
\begin{enumerate}  [label=\alph*)]
\item Man beschreibe diese Forderungen durch eine Relation $U\subseteq A\times A$ und stelle sie graphisch dar (HASSE-Diagramm).
\item Man ermittle die transitive Hülle $U^+$ von $U$.
\item Man gebe (falls vorhanden) obere Schranken, Supremum, Maximum, max. Elemente sowie untere Schranken, Infinum, Minimum, min. Elemente von $S$ an.
\end{enumerate}
Lösung:
\begin{enumerate}   [label=\alph*)]
\item $U=\{(1,2),(2,3),(2,5),(3,4),(5,6\}$\\
\begin{tikzpicture}[scale=.5]
% Menge T
\draw (0,0) node{1};
\draw[-latex] (0.3,0) -- (1.7,0);
\draw (2,0) node{2};
\draw[-latex] (2.3,0.1) -- (3.7,1);
\draw[-latex] (2.3,-0.1) -- (3.7,-1);
\draw (4,1) node{3};
\draw[-latex] (4.3,1) -- (5.7,1);
\draw (6,1) node{4};
\draw (4,-1) node{5};
\draw[-latex] (4.3,-1) -- (5.7,-1);
\draw (6,-1) node{6};

\draw[gray] (1.5,-1.5)  node[left]{$S$} rectangle (4.5,1.5);
\end{tikzpicture}
\item $U\circ U=\{(1,3), (1,5), (2,4), (2,6)\}$\\
$U\circ (U \circ U)=\{ (1,4), (1,6)\}$\\
$U^4=\emptyset$
\end{enumerate}
\end{enumerate}

\subsubsection{Funktionen}

\paragraph{Def. 17:} \parskp
Eine Relation $f\subseteq x\times y$ heißt \emph{Funktion (Abbildung)} von $X$ in $Y$, wenn sie linksvollständig und rechtseindeutig ist.

\subparagraph{Diskussion:}
\begin{itemize}
\item Gemäß Def. 7 a+c aus Kapitel \ref{Def. 7} bedeutet linksvollständig \emph{und} rechtseindeutig, dass zu \emph{jedem} $x\in X$ \emph{genau ein} $y \in Y$ mit $(x,y) \in f$ existiert, also \emph{eindeutige Zuordnung}:\\
$\boxed{x\longmapsto y=: f(x)}$\\
\emph{Schreibweise:}
$f: X \rightarrow Y$ (manchmal $f|X\rightarrow Y$\\
$y=f(x)$ heißt auch \emph{Bild} von $x$, $x$ \emph{ein} Urbild von $y$ (muss nicht eindeutig sein).
\item $X=Db(f)$… \emph{Definitionsbereich}, \\
$Wb(f)=\{y \in Y | \exists x \in x \quad (x,y) \in f\} \subseteq Y$ … Wertebereich\\
Schreibweise auch $f(X):= Wb (f)$ (Menge aller Bilder).\\
\begin{tikzpicture} [scale=0.5]
\draw (0,0) rectangle (8,6);
\draw plot[smooth, tension=0.9] coordinates {(0,1) (2,3.5) (8,3)};

\draw[purple, dashed] (0,3.7) -- (3,3.7);
\draw[purple, very thick] (0,0) -- (0,3.7);
\draw[purple] (0,2) node[left]{$Wb(f)$};
\draw[orange, very thick] (0,0) -- (8,0);
\draw[orange] (4,0) node[below]{$x=Db(f)$};
\end{tikzpicture}\\
$f: \mathbb{R} \rightarrow \{0,1\}$
\end{itemize}

\paragraph{Def. 18:}
\begin{enumerate} [label=\alph*)]
\item Eine Abbildung $f$ heißt \emph{surjektiv} (Auch Abbildung auf $Y$), 
\item Eine Funktion $f$ heißt \emph{injektiv}, wenn zu jedem $y \in Wb(f)$ genau ein $x \in Db(f)$ existiert mit $(x,y) \in f$:\\
$\underset{\in Wb(f)}{y} \longmapsto x =:\underset{\in Db(f)} {f^{-1}(y)}$ \tab („$f$ oben -1“)\\
Die dadurch erklärte Abbildung $f^{-1}: Wb(f) \rightarrow Db (f)$ heißt \emph{Umkehrfunktion} von $f$, vgl. auch Kap \ref{subsec:Aussagefunktionen}.
\item Eine injektive \emph{und} surjektive Abb. von $X$ auf $Y$ heißt \emph{bijektiv}.
\item Gebräuchlich sind auch die Begriffe \emph{Surjektion}, \emph{Injektion} und \emph{Bijektion}!
\end{enumerate}

\subparagraph{Bsp. 12:} \parskp
Gegeben sind die Mengen $X=\{a,b,c\}$ und $Y=\{1,2,3,4\}$ sowie folgende Relation in $X\times Y$:
\begin{enumerate} [label=\alph*)]
\item $T_1$: \begin{tikzpicture}[scale=0.3]
\draw (0,0) node{$a$};
\draw (0,-2) node{$b$};
\draw (0,-4) node{$c$};
\draw (0,-8) node{$(X)$};

\draw (6,0) node{$1$};
\draw (6,-2) node{$2$};
\draw (6,-4) node{$3$};
\draw (6,-6) node{$4$};
\draw (6,-8) node{$(Y)$};

\draw[-latex] (0.5,0) -- (5.5,0);
\draw[-latex] (0.5,-2) -- (5.5,-6);
\draw[-latex] (0.5,-4) -- (5.5,-2);
\end{tikzpicture} \\
$T_1$ ist eine Funktion $f(=T_1): f: X\rightarrow Y \quad (1)$ diese ist injektiv, $Db(f) = X= \{a,b,c\}$, $Wb(f) = \{1,2,4\}=: W$, $f: X\rightarrow W \quad (2)$ ist surjektiv, also sogar bijektiv.\\
Als Relation sind $(1)$ und $(2)$ nicht zu unterscheiden, aber als Funktion.
\item $T_2$: \begin{tikzpicture}[scale=0.3]
\draw (0,0) node{$a$};
\draw (0,-2) node{$b$};
\draw (0,-4) node{$c$};
\draw (0,-8) node{$(X)$};

\draw (6,0) node{$1$};
\draw (6,-2) node{$2$};
\draw (6,-4) node{$3$};
\draw (6,-6) node{$4$};
\draw (6,-8) node{$(Y)$};

\draw[-latex] (0.5,0) -- (5.5,0);
\draw[-latex] (0.5,-2) -- (5.5,-4);
\end{tikzpicture} \\$T_2$ ist keine Funktion, nicht linksvollständig. Betrachtet man $D=\{a,b\}\subset X$, so ist durch $T_2$ eine Funktion $f:D\rightarrow Y$ beschrieben, die Funktion ist injektiv und kann mit $W:= f(D)=\{1,3\}$ zu einer bijektiven Abbildung $f: D\rightarrow W$ umgewandelt werden.
\item $T_3$: \begin{tikzpicture}[scale=0.3]
\draw (0,0) node{$a$};
\draw (0,-2) node{$b$};
\draw (0,-4) node{$c$};
\draw (0,-8) node{$(X)$};

\draw (6,0) node{$1$};
\draw (6,-2) node{$2$};
\draw (6,-4) node{$3$};
\draw (6,-6) node{$4$};
\draw (6,-8) node{$(Y)$};

\draw[-latex] (0.5,0.3) -- (5.5,0.3);
\draw[-latex] (0.5,0) -- (5.5,-2);
\draw[-latex] (0.5,-2) -- (5.5,-4);
\draw[-latex] (0.5,-4) -- (5.5,-6);
\end{tikzpicture} \\$T_3$ ist keine Funktion, da nicht rechtseindeutig.
\end{enumerate}

\subparagraph{Bsp. 13:}
\begin{enumerate} [label=\alph*)]
\item $f:[0,\infty) \rightarrow \mathbb{R}$ mit „$x \rightarrow y = f(x) = \sqrt{x}$ ist eine \emph{Funktion einer reelen Veränderlichen} (injektiv).\\
\begin{tikzpicture} [scale =0.5]
\draw[-latex] (-.5, 0) -- (8,0) node[right]{$x$};
\draw[-latex] (0, -.5) -- (0, 4) node[above]{$y$};
\draw[red, domain=0:7] plot[smooth] (\x, {sqrt(\x)});
\end{tikzpicture}\\
$Wb(f)=[0,\infty)$\\
$Db(f)=[0,\infty)$
\item $f: \mathbb{R}\times \mathbb{R}\rightarrow \mathbb{R}$ mit $(x,y) \longmapsto x^2 +y^2=f(x,y) =: z$ \emph{Funktion zweier reeller Veränderlicher}.\\
\begin{tikzpicture}[scale =0.5]
\draw[->,thick] (-0.5,0) -- (4,0) node[right] {$y$};
\draw[->,thick] (0,-0.5) -- (0,4) node[above] {$z$};
\draw[->,thick] (.5,.5) -- (-2,-2) node[left] {$x$};

\draw[red, domain=-2:2] plot[smooth] (\x,{\x*\x});
\draw[red, dashed] (0,1) ellipse (1 and 0.3);
\draw[red, dashed] (0,2) ellipse (1.4 and 0.42);
\draw[red, dashed] (0,3) ellipse (1.75 and 0.525);
\end{tikzpicture}
(Paraboloid)\\
$Db(f) = \mathbb{R} \times \mathbb{R}$ (x-y-Ebene)\\
$Wb(f)=[,\infty)$
\item $f: \mathbb{N} \rightarrow \mathbb{R}$ mit $n\longmapsto f(n) = \frac{n}{n+1}$ ist eine (reelle) Zahlenfolge. $f(0)=1, f(1)=\frac{1}{2}, f(2)=\frac{2}{3}, ...$\\
Bezeichnung meist mit Index: $a_n=f(n) \curvearrowright ZF(a_n) \quad n \in \mathbb{N}$
\end{enumerate}

\paragraph{Def. 19:} \parskp
Es seien $g:=X\rightarrow U$ mit $x\longmapsto u=g(x)$ und $f: U\rightarrow Y$ mit $u\longmapsto y = f(u)$ zwei Abbildungen. Dann stellt man die Zuordnung $x\longmapsto y = f(\underset{u}{g(x)})$ eine Abbildung von $X$ in $Y$ dar, eine sogenannte \emph{mittelbare Funktion (Komposition / Verkettung)}.
\emph{Bezeichnung:} $g \circ f: X \rightarrow Y$ mit $y=(g \circ f) (x) = f(g(x))$

\subparagraph{Diskussion:}
\begin{enumerate}
\item \begin{tikzpicture}[scale=0.3]
\draw (-1,1.5) node {$X$};
\draw (0,-2) ellipse (2 and 3);
\draw (0,0) node{\textbullet};
\draw (0,-2) node{\textbullet};
\draw (0,-4) node{\textbullet};

\draw (5,1.5) node {$U$};
\draw (6,-3) ellipse (2 and 4);
\draw (6,0) node{\textbullet};
\draw (6,-2) node{\textbullet};
\draw (6,-4) node{\textbullet};
\draw (6,-6) node{\textbullet};

\draw (11,1) node {$Y$};
\draw (12,-2) ellipse (2 and 3);
\draw (12,0) node{\textbullet};
\draw (12,-2) node{\textbullet};
\draw (12,-4) node{\textbullet};

\draw[-latex] (0.5,0) -- (5.5,0);
\draw[-latex] (0.5,-2) -- (5.5,-2);
\draw[-latex] (0.5,-4) -- (5.5,-6);

\draw[-latex, orange] (6.5,0) -- (11.5,0);
\draw[-latex, orange] (6.5,-1) -- (11.5,-4);
\draw[-latex, orange] (6.5,-4) -- (11.5,-.2);
\draw[-latex, orange] (6.5,-6) -- (11.5,-2);
\end{tikzpicture} \\
$x \longmapsto u=g(x) \quad u \textcolor{orange}{\longmapsto} f(u) = f(g(x))$\\
Paarschreibweise: $(x,u) \in g \qquad (u,y) \in f \curvearrowright (x,y) \in g \circ f$
\item $g$ wird zuerst angewendet, dann $f$. Wie bei beliebigen Relationen die die Schreibweise $g\circ f$
\item In der Literatur findet man oft die Schreibweise $f\circ g$ angelehnt an die Schreibweise $f(g(x))$. Die Reihenfolge der Berechnung ast aber von innen nach außen, erst innere Funktion $g$, dann die äußere $f$.
\end{enumerate}

\paragraph{Satz 3:} \parskp
Es sei $f: X\rightarrow Y$ eine \emph{Bijektion}, d.h. es existiert die Umkehrfunktion $f^{-1}: Y\rightarrow X$, weiter sei $i_A$ für eine beliebige Menge $A$ die identische Abbildung (Identitätsrelation): $i_A: A\rightarrow A$ mit $i_A(x)=x$ für alle $x\in A$.\\
Es gilt dann: \\
$f \circ f^{-1}= id_X$, d.h. $(f\circ f^{-1})(x)=f^{-1}(f(x))= x (\forall x \in X)$ und \\
$f^{-1}\circ f = id_Y$, d.h. $ (f^{-1}\circ f)(y))= f(f^{-1}(y))=y (\forall y \in Y)$\\
(Funktion und Umkehrfunktion nacheinander angewandt heben sich auf).

\paragraph{Satz 4:} \parskp
Es seien $g=X\rightarrow U$ und $h: U\rightarrow Y$ zwei Bijektionen. Dann ist die Komposition $f:= g \circ h : X\rightarrow Y$ ebenfalls eine Bijektion und es gilt:\\
$\boxed{f^{-1}=(g\circ h)^{-1}=h^{-1}\circ g^{-1}}$

\subsection{Gleichmächtigkeit, Kardinalzahlen}

Es sei eine hinreichend mufassend Grundmenge, die alle für eine mathematische Theorie relevante Objekte (Zahlen, Funktionen, usw.) enthält. $M$ sei die Potenzmenge von $E$(d.h. $M$ ist die Menge aller Teilmengen von $E$, $M=\mathcal{P}(E)$).

\paragraph{Def. 20:} \parskp
Zwei Mengen $A$ und $B$ ($A \subseteq E, B\subseteq E$ bzw. $A \in M, B \in M$) heißen \emph{gleichmächtig} (Bezeichnung $A\sim B$), wenn eine bijektive Abbildung von $A$ auf $B$ (damit auch $B$ auf $A$) existiert.

\subparagraph{Diskussion:}
\begin{enumerate}
\item Offensichtlich ist die Relation $T\subseteq M \times M$ mit $(A,B) \in T : \equiv A \sim B$ eine Äquivalenzrelation auf $M$.
\item Äquivalenzklassen sind Mengen gleichmächtiger Teilmengen von $E$. Diese Äquivalenzklassen nennt man \emph{Kardinalzahlen}.
\item Bei endlichen Mengen bedeutet Gleichmächtigkeit: \\
Gleiche Anzahl von Elementen \\
$A = \{a, b,c\}, B=\{X,Y,Z\}$ \\
(Abbildung bspw. $a \rightarrow X \quad b\rightarrow Y \quad c\rightarrow Z$)\\
Bezeichnung: $card A = |A| = 3 \quad (=|B|)$\\
\emph{Natürliche Zahlen sind die Kardinalzahlen endlicher Mengen.}
\item Die Anschauung versagt bei unendlichen Mengen.\\
\begin{tikzpicture} [scale = 0.4]
\draw (0,0) -- (10,0) node[below left]{B};
\draw (0,0) -- (10,5) node[above left]{A};

\draw[green, latex-] (2,0) -- (2,1);
\draw[green, latex-] (4,0) -- (4,2);
\draw[green, latex-] (6,0) -- (6,3);
\draw[green, latex-] (8,0) -- (8,4);
\end{tikzpicture}\\
Die Strecken $A$ und $B$ sind gleichmächtig, obwohl $A$ länger als $B$ ist.
\end{enumerate}

\paragraph{Def. 21:} \parskp
Eine Menge heißt \emph{abzählbar unendlich}, wenn sie mit der Menge $\mathbb{N}=\{1,2,3,4,...\}$ der natürlichen Zahlen gleichmächtig ist.

\subparagraph{Diskussion:}

\begin{enumerate}
\item $M$ ist abzählbar unendlich heißt, es existiert eine \emph{Zählvorschrift}, bei der jedes Element von $M$ nach endlich vielen Schritten erreicht wird.
\item Die Menge $\mathbb{Z}$ der ganzen Zahlen ist abzählbar unendlich. \\
Andordnen nach steigendem Betrag:\\
$\mathbb{Z}=\{0,-1,1,-2,2,-3,3, ...\}$
\item $\mathbb{Q}^+$ … Menge der pos. rationalen Zahlen\\
\begin{tikzpicture} [scale = 1.25]
\draw (-1,0) node{$\mathbb{Q}^+:$};

\draw (0,0) node{$\frac{1}{1}$,};
\draw[green] (-0.2,0.5) rectangle (0.2,-0.5);
\draw (1,0) node{$\frac{1}{2}$,};
\draw (2,0) node{$\frac{1}{3}$,};
\draw (3,0) node{…};

\draw[green, -latex] (.75,-.25) -- (.25,-.75);

\draw (0,-1) node{$\frac{2}{1}$,};
\draw (1,-1) node{$\frac{2}{2}$,};
\draw (2,-1) node{$\frac{2}{3}$,};
\draw (3,-1) node{…};

\draw[green, -latex] (1.75,-.25) -- (1.25,-.75);
\draw[green, -latex] (.75,-1.25) -- (.25,-1.75);

\draw (0,-2) node{$\frac{3}{1}$,};
\draw (1,-2) node{$\frac{3}{2}$,};
\draw (2,-2) node{$\frac{3}{3}$,};
\draw (3,-2) node{…};

\draw[green, -latex] (2.75,-.25) -- (2.25,-.75);
\draw[green, -latex] (1.75,-1.25) -- (1.25,-1.75);
\draw[green, -latex] (.75,-2.25) -- (.25,-2.75);

\draw[green, -latex] (2.75,-1.25) -- (2.25,-1.75);
\draw[green, -latex] (1.75,-2.25) -- (1.25,-2.75);

\draw[green, -latex] (2.75,-2.25) -- (2.25,-2.75);

\draw (0,-3) node{…};
\draw (1,-3) node{…};
\draw (2,-3) node{…};
\draw (3,-3) node{…};
\end{tikzpicture}\\
Zählvorschrift:
\begin{enumerate}
\item (aufsteigend) Ordnen nach Summen von Zöhler und Nenner
\item Zahlen mit gleicher Summe der Größe nach aufsteigend anordnen.
\item Bereits enthaltene Zählen (=kürzbare Brücke) weglassen.
\end{enumerate}
$\mathbb{Q}^+=\left\lbrace\frac{1}{1},\frac{1}{2}, \frac{2}{1}, \frac{1}{3}\textcolor{red}{, \frac{3}{2}}, \frac{3}{1}, ... \right\rbrace$\\
analog zu $\mathbb{Z}$: Die Menge $\mathbb{Q}$ aller rationalen Zahlen (also $\mathbb{Q}^-$ zusammen mit $\mathbb{Q}^+$) ist abzählbar unendlich.
\item Es gibt Mengen, die mächtiger sind als die Menge der natürlichen Zahlen: \emph{überabzählbare Mengen} ($B$ heißt \emph{mächtiger} als $A$, wenn se eine injektive Abbildung $f: A \rightarrow B$ gibt, aber keine bijektive. Schreibweise: $|A|<|B|$).
\end{enumerate}
z. B. gilt:
\subparagraph{Satz 5:} Die Menge $M=\{x \in \mathbb{R} | 0<x<1\}=(0,1)$ ist überabzählbar.\\
Beweis: (\emph{CANTORsches Diagonalverfahren})\\
Indirekt, angenommen $M=(0,1)$ sei abzhälbar unendlich, d.h. $M=\{x_1, x_2, x_3, ...\}$.\\
Für die Zahlen $x_k$ wählen wir z.B. die eindeutige Darstellung als Dezimalbruch (9er Periode vermeiden). Also bspw. $0,39999...=0,3\overline{9}=0,4=0,40000...$)\\
$x_{\textcolor{green}{1}}=0,a_1^{\textcolor{green}{(1)}}a_2^{\textcolor{green}{(1)}}a_3^{\textcolor{green}{(1)}}...\\
x_2=0,a_1^{(2)}a_2^{(2)}a_3^{(2)}...\\
x_3=0,a_1^{(3)}a_2^{(3)}a_3^{(3)}...\\
...$\\
Es sei $z=0,b_1 b_2 b_3 ...$ mit 
$b_k=\begin{cases} 
1 & \text{falls } a_k^{(k)}\not = 1\\
2 & \text{falls } a_k^{(k)} = 1\\
\end{cases}$
für $k=1,2,3,...$\\
Damit unterscheiden sich $x_k$ und $z$ an der $k$-ten Stelle, d.h. $z \not = x_k$ für alle $k\geq 1$. $z$ ist also nicht in der Folge $x_1, x_2, x_3, ...$ enthalten, also \emph{$z\not \in M$}.\\
Andererseits ist $0<z<1$ also \emph{$z\in (0,1) = M$}. \lightning\, Widerspruch! \#

\subparagraph{Satz 6:} \parskp
Es sei $E$ eine Menge. Dann ist die Potenzmenge $M=\mathcal{P}(E)$ mächtiger als $E$.\\
Beweis: 
\begin{enumerate}
\item Die Abbildung $f: E \rightarrow M$ mit $f(x)=\{x\}$, die jedem $x \in E$ die einelementige Menge $\{x\}\in M$ zuordnet, ist injektiv.
\item Angenommen, es gäbe eine bijektive (damit auch surjektive) Abbildung $g: E \rightarrow M$. Es sei $A = \{x \in E| x \not \in g (x)\}\in M$ ($A$ Teilmenge von $E$). Da $g$ surjektiv ist, gibt ein $a \in E$ mit $g(a)=A$. Fallunterscheidung:
\begin{enumerate} %[label=\arab*.]
\item $a\in A= g(a) \Rightarrow a \not \in g(a)$ \lightning\, Widerspruch!
\item $a \not \in A =g(a) \Rightarrow a \in g(a)$ \lightning\, Widerspruch!
\end{enumerate}
Beide Fälle führen auf einen Widerspruch, es gibt keine surjektive und damit auch keine bijektive Abbildung von $E$ auf $\mathcal{P}(E)$. \#
\end{enumerate}

\subparagraph{Diskussion:}\parskp
Satz 6 zeigt, dass es unendlich viele unendliche Mächtigkeiten gibt. So gilt bspw. $|\mathbb{N}|<|\mathcal{P}(\mathbb{N})|<|\mathcal{P}(|\mathcal{P}(\mathbb{N})|)|$ usw. 

\subparagraph{Satz 7:} \parskp
Die Potenzmenge $\mathcal{P}(\mathbb{N})$ der Menge der natürlichen Zahlen ist gleichmächtig mit dem Intervall $(0,1)$, also überabzählbar (Beweis: siehe Übungsaufgabe 1.38).

\subsection{Prinzip der vollständigen Induktion}

Es sei $n_0 \in \mathbb{N}$. Zu beweisen ist: „Für alle natürlichen Zahlen $n\geq n_0$ gilt die Aussage $p(u)$.“ (Es sind also abzählbar unendlich viele Einzelaussagen zu beweisen!)

\paragraph{Satz 8:} 
\begin{enumerate}
\item Es sei $\boxed{p(n_0)}$ wahr (Induktionsanfang)
\item Für alle nat. Zahlen $n \geq n_0$ sei die Implikation $\boxed{p(n)\Rightarrow p(n+1)}$ wahr (Induktionsschluss)
\end{enumerate}
Dann gilt: $p(n)$ ist für alle nat. Zahlen $n\geq n_0$ wahr.\smallskip\\
Zum Beweis:
\begin{enumerate}
\item $\curvearrowright p(n_0)$ ist wahr, 
\item $p(n_0)\Rightarrow p(n_0+1) ... \curvearrowright$ Konkulsion wahr usw. (Dominoeffekt)
\end{enumerate}

\subparagraph{Bsp. 14:} \parskp
Zu beweisen ist $\sum^{n}_{k=1}k^2=\frac{n (n+1) (2n+1)}{6}$ für alle $n \in \mathbb{N}^*$
\begin{enumerate}
\item $p(1)$: $1^2=\frac{1\cdot2\cdot3}{6}=1$ (wahr) \qquad Induktiansanfang (IA)
\item Es gelte $p(n)$: $\sum^{n}_{k=1}k^2=\frac{n (n+1) (2n+1)}{6}$ \qquad Induktionsannahme/-voraussetzung (IV)
\item zu zeigen $p(n+1)$: $\sum^{n+1}_{k=1}k^2=\frac{n+1 (n+2) (2n+3)}{6}$ \qquad Induktionsschritt (IS)
\item \emph{Induktionsschluss:} $(p(n)\Rightarrow p(n+1))$
\begin{align*}
\sum^{n+1}_{k=1}k^2&=\sum^{n}_{k=1}k^2+ (n+1)^2 \\
&\overset{IA}{=} \frac{n\cdot (n+1)(2n+1)}{6}+(n+1)^2 \\
&= \frac{n+1}{6}\left(n(2n+1)+6(n+1)\right)\\
&= \frac{n+1}{6}(2n^2 + 7n + 6)\\
&= \frac{n+1}{6}(n+2)(2n+3) \qquad \#
\end{align*}
\end{enumerate}

\section{Zahlen}
%
\subsection{Gruppen, Ringe, Körper}
\begin{itemize}
\item Gegeben sei eine Menge $M$ und eine zweistellige Operation $\circ$ (d.h. Abb. von $M\times M$ in $M$)\\
Bezeichnung: $(M,\circ)$, analog $(M, \circ, *)$
\item Die Operation $\circ$ heißt \emph{kommutativ}, wenn $a \circ b= b \circ a $ für alle $a, b \in M$.
\item Die Operation $\circ$ heißt \emph{assoziativ}, wenn $(a \circ b) \circ c =a \circ (b \circ c)$ für alle  $a, b, c \in M$.
\end{itemize}

\paragraph{Def. 1:}\parskp
$(M,\circ)$ heißt \emph{Gruppe}, wenn gilt:
\begin{enumerate}
\item Die Operation $\circ$ ist assoziativ
\item Es gibt genau ein \emph{neutrales Element} $e\in M$ mit $a\circ e = e\circ a = a$ (für alle $a \in M$)
\item Es gibt zu jedem $a \in M$ genau ein \emph{inverses Element} $a^{-1}$ mit $a\circ a^{-1}=a^{-1}\circ a = e$
\item Eine Gruppe heißt \emph{ABELsch}, wenn zusätzlich folgendes gilt:\\
$\circ$ ist kommutativ
\end{enumerate}

\paragraph{Def. 2:}\parskp
$(M,\oplus,*)$ heißt \emph{Ring}, wenn gilt:
\begin{enumerate}
\item $(M, \oplus)$ ist eine ABELsche Gruppe.
\item Die Operation $*$ ist assoziativ.
\item Es gelten für beliebige $a,b,c\in M$:\\
$a*(b\oplus c)=(a*b)\oplus (a*c)\\
(a \oplus b) *c = (a*c) \oplus (b*c)$ \qquad (Distributivgesetze)
\item Ein Ring heiß \emph{kommutativer Ring}, wenn gilt:\\
$*$ ist kommutativ
\end{enumerate}

\paragraph{Def. 3:}\parskp
$(M, \oplus, *)$ heißt \emph{Körper}, wenn gilt:
\begin{enumerate}
\item $(M,\oplus, *)$ ist ein Ring \\
(mit dem neutralen Element $E_0$ für die Operation $\oplus$)
\item $(M\setminus \{E_0\}, *)$ ist eine ABELsche Gruppe\\
(mit dem neutralen Element $E_1$ für die Operation $*$)
\end{enumerate}

\subsection{Zahlentheorie}
\begin{itemize}
\item Eine natürliche Zahl $p>1$, die nurch durch $1$ und sich selbst teilbar ist heißt \emph{Primzahl}.
\item Jede natürliche Zahl $n>1$ ist entweder eine Primzahl, oder sie lässt sich als Produkt von Primzahlen schreiben.\\
Diese sogenannte \emph{Primfaktorzerlegung} ist bis auf die Reihenfolge der Faktoren eindeutig.
\end{itemize}

\paragraph{Def. 4:}\parskp
Zwei natürliche zahlen aus $\mathbb{N}^*$ heißen \emph{teilerfremd}, wenn sie außer $1$ keine gemeinsamen teiler besitzen.
\begin{itemize}
\item Es sei $a \in \mathbb{Z}$ und $m\in\mathbb{N}^*$. Dann gibt es eine eindeutige Darstellung der Gestalt $\boxed{a=q\cdot m + r}$ mit $\boxed{0\leq r < m}$ und $q \in \mathbb{Z}$.\\
Bezeichnung: $m$ … \emph{Modul} \qquad $r$ … (kleinste nichtnegative) \emph{Rest modulo m} ($r \equiv mod (a,m)$)
\item Zur Erinnerung: $a$ und $b$ seien ganze Zahlen, $m\in \mathbb{n}^*$, dann $a\equiv b (mod\; m)$ [$a$ kongruent $b\; modulo\; m$]\\
$\Leftrightarrow$ $a$ und $b$ haben den gleicher Rest $modulo\; m$\\
$\Leftrightarrow$ $a-b$ ist durch $m$ teilbar (d.h. $\exists k \in \mathbb{Z} \quad a-b = k\cdot m$)
\paragraph{Satz 1:} \parskp
Es sei $a \equiv b (mod\; m)$, $c\equiv d (mod\; m)$, dann gilt: $a+c \equiv b+d (mod \;m)$ und $a\cdot c \equiv b \cdot d (mod\; m)$ (d.h. in Summen und Produktenn darf jede Zahl durch einen beliebigen Vertreter der gleichen Restklasse ersetzt werden).
\subparagraph{Bsp. 1:} 
\begin{enumerate}[label=\alph*)]
\item $307+598 \equiv 1+(-2)\equiv -1 \equiv 5 (mod \; 6)$
\item $307\cdot 598 \equiv 1 \cdot (-2) \equiv -2 \equiv 4 (mod \; 6)$
\item $598^6 \equiv (-2)^6\equiv 64 \equiv 4 (mod \; 6)$
\end{enumerate}
\item Man wählt aus jeder Restklasse den kleinsten nichtnegativen Vertreter \\
$\curvearrowright$ Menge von Resten $modulo \; m$: $\mathbb{Z}_m:= \{0,1,...,m-1\}$ \\
$\curvearrowright$ „modulare Arithmetik“: Operation $\oplus$ und $\odot$ für Zahlen aus $\mathbb{Z}_m$ erklärbar, in dem für das Ergebnis jeweils der kleinste nichtnegative Rest $modulo \; m$ gewählt wird (vgl. Satz 1)\\
z.B. $\mathbb{Z}_7 = \{ 0,1,..., 6\}$, \quad $5\oplus 4=2$, da $5+4\equiv 9\equiv 2 (mod \; 7)$ \quad $5\odot 6=2$, da $5\cdot 6\equiv 30 \equiv 2 (mod \;7)$\smallskip\\
Falls keine Verwechselung zu befürchten ist, wird die übliche Schreibweise $+$ und $\cdot$ anstelle von $\oplus$ und $\odot$ verwendet.
\end{itemize}

\paragraph{Def. 5:} \parskp
Wenn es zu $c \in \mathbb{Z}_m$ eine Zahl $d\in \mathbb{Z}_m$ gibt, mit $c\cdot d \equiv 1 (mod \; m)$ (bzw. $c\odot d \equiv 1$), so heißt $d$ die \emph{(multiplikative) modulare Inverse} zu $c$ in $\mathbb{Z}_m$.\\
Bezeichnung: $d=c^{-1}$
\subparagraph{Bsp. 2:} \parskp
$c=3 \in \mathbb{Z}_7$, wegen $3 \cdot 5 \equiv 1 (mod \; 7)$ ist (in $\mathbb{Z}_7$) $3^{-1}=5$.
\subparagraph{Satz 2:} Zu $a \in \mathbb{Z}_m, a \not = 0$, gibt es genau dann eine modulare Inverse in $\mathbb{Z}_m$, wenn $a$ und $m$ teilerfremd sind ($ggT(a,m)=1$).
\subparagraph{Satz 3:} Es sei $p$ eine Primzahl. Dann ist $(\mathbb{Z}_m, \oplus, \odot)$ ein Körper.\\
Bemerkung: Falls $m$ keine Primzahl ist, so ist $(\mathbb{Z}_m, \oplus, \odot)$ ein kommutativer Ring.
\subparagraph*{EUKLIDischer Algorithmus}
\begin{itemize}
\item Verfahren zur Ermittlung des größten gemeinsamen Teilers $t$ zweier positiver natürlicher Zahlen, $t=ggT(a,b)$.
\item In erweiterter Form bietet der Algorithmus eine Möglichkeit zur Bestimmung  der modularen Inversen von $a$ zum Modul $m$ (mit $a<m$ und $a,m$ teilerfremd).
\end{itemize}

\subparagraph{Satz 4:} (EUKLIDischer Algorithmus)\\
Es seien $a,b \in \mathbb{N}^*, a>b$. Man bildet die endliche Folge \\
$r_0:= b, \; r_1=mod \;(a,b), \;r_2 = mod \; (r_0,r_1),...,\;r_n = mod \; (r_{n-2}, r_{n-1})$, Abbruch falls $r_n=0$.\\
In diesem Fall gist $\boxed{ggT(a,b)=r_{n-1}}$ (letzter nicht verschwindender Rest).\\
Bezeichnung: $j$-te Division ... $\boxed{r_{j-2}:r_{j-1}= q_j\text{ Rest } r_j}$ ($j=1,...,n$) (dabei $r_1:=a$).

\subparagraph{Satz 5:}  (erweiterter EUKLIDischer Algorithmus)\\
Zusätzlich zur Folge $(r_n)$ aus Satz 4 bilde man die Folgen \\
$x_0=0,\; x_1=1, \; x_2 = x_0 - q_2x_1, ...,\; x_j=x_{j-2}-q_jx_{j-1}\quad (j\leq n-1)$ und\\
$y_0=1, \; y_1=-q_1, \; y_2=y_0- q_2 y_1, ... ,\; y_j=y_{j-2}-q_jy_{j-1}\quad (j \leq n-1)$\\
Dann gilt für alle $j=0, ..., \; n-1$: $\boxed{r_j=x_j\cdot a + y_j \cdot b}$\\
Insbesondere gilt $\boxed{ggT(a,b) = x_{n-1}\cdot a + y_{n-1}\cdot b}$
\subparagraph{Diskussion:}
\begin{enumerate}
\item Der Sinn der erweiterten EUKLIDischen Algorithmus besteht darin, in jedem Schrit den \emph{Divisionsrest $r$ als linearkombination von $a$ und $b$ mit ganzzahligen Koeffizienten $x$ und $y$} darzustellen: $r=x\cdot a + y\cdot b$\\
Der Mechanismus wird am besten im Rechenschema des nachfolgenden Bsp. 4 deutlich.
\item  Sind $c$ und $m$ teilerfremd, $1\leq c < m$, d.h. $ggT(m,c)=1$, so erhält man mit dem erweiterten EUKLIDischen Algorithmus ($a=m, b=c$) eine Darstellung in der Form $\boxed{1=x\cdot m+ y\cdot c}$.\\
$\curvearrowright y\cdot c \equiv 1 (mod\; m)$ und damit $c^{-1}\equiv y (mod\; m)$ (für die modulare Inverse muss eventuell noch der in $\mathbb{Z}_m$ liegende, zu $y$ kongruente, Wert gebildet werden!).
\end{enumerate}

\subparagraph{Bsp. 3:} \parskp
Man ermittle den größten gemeinsamen Teiler $t$ sowie das kleinste gemeinsame Vielfache $v$ der Zahlen $132$ und $84$.
\begin{itemize}
\item Es genügt der „einfache“ Algorithmus:
\begin{align*}
132:84 &= 1\text{ Rest } 48\\
84:48 &= 1\text{ Rest } 36\\
48:36 &= 1\text{ Rest } 12 \quad \curvearrowright t = ggT (132,84) = \resultul{12}\\
36:12 &= 3\text{ Rest } \boxed{0} \curvearrowright \text{ Ende.}\\
\end{align*}
\item $v=\frac{a \cdot b}{t}=\frac{132\cdot 84}{12}=\resultul{924}=kgV (132,84)$
\end{itemize}

\subparagraph{Bsp. 4:} \parskp
Man ermittle die modulare Inverse von $\overbrace{11}^b$ zum Modul $\overbrace{25}^a$. \medskip\\
\begin{tabular}{r l | l | r l}
				&				&						&	$11$&$=0 \cdot 25 + 1 \cdot 11$ (1)\\
$25:11 $&$= 2 \text{ Rest }  3$	&	$3=25-2 \cdot 11$	&	$3 $&$= 1 \cdot 25 - 2 \cdot 11$ (2)\\
$11: 3 $&$= 3 \text{ Rest } 2$		&	$2= 11-3 \cdot 3$	&	$2 $&$= -3 \cdot 25 + 7 \cdot 11$ (3)\\
$3:2 $&$= 1 \text{ Rest } 1$		&	$1 = 3-2$			&	$\boxed{1} $&$= 4 \cdot 25 - 9 \cdot 11$\\
$2 : 1 $&$= 2 \text{ Rest }  0$	&						& &\\
\end{tabular} \medskip\\
$\curvearrowright (-9) \cdot 11 \equiv 1 (mod \; 25)\\
\curvearrowright 11^{-1} \equiv -9 \equiv 16 (mod \; 25)$, die Inverse von $11$ in $\mathbb{Z}_{25}$ ist $16$.\\
Zu den Schritten:
\begin{itemize}
\item [(1)] $b=0 \cdot a + 1 \cdot b$
\item [(2)] mittleres Feld als Linearkombination
\item [(3)] ab hier Rechnung links spaltenweise durchführen, dabei Faktoren $a$ und $b$ beibehalten. 
\end{itemize}
\subsubsection*{EULERsche $\varphi$-Funktion, Satz von EULER}
\paragraph{Def. 6:} \parskp
Es sei $n \in \mathbb{N}^*$. Dann \emph{EULERsche $\varphi$-Funktion}:

$\varphi(n):=$ Anzahl der zu $n$ teilerfremden Elemente aus $\{1,2,...,n\}$.
Eigenschaften der $\varphi$-Funktion:
\begin{itemize}
\item Es sei $p$ eine Primzahl, dann ist $\boxed{\varphi(p)=p-1}$, $\boxed{\varphi(p^k)=p^{k-1}(p-1)} \; (k \in \mathbb{N}^*$
\item Falls $ggT (m, n) =1$, so gilt $\varphi (m\cdot n) = \varphi (m) \cdot \varphi (n)$.
\item Speziell: $n=p\cdot q $ ($p,q$ Primzahlen), dann $\boxed{\varphi(n)=(p-1)\cdot (q-1)}$ (1).
\end{itemize}
\subparagraph{Satz 6:} (Satz von EULER)\\
Es sei $ggT(a,n)=1$, dann gilt:\\
$\boxed{a^{\varphi(n)}\equiv 1 (mod \; n)}$ (2).

\subsubsection*{RSA-Verschlüsselung}
\begin{itemize}
\item Die Formeln (1) und (2) [siehe oberhalb] bilden die Grundlage für die sogenannte RSA-Ver\-schlüsselung (RIVES, SHAMIR, ADLEMAN - 1978)
\item Schlüsselerzeugung:
\begin{enumerate}
\item Man wählt (in der Praxis sehr große) Primzahlen $d$ und $q$.
\item $n:= p \cdot q$, $m:= \varphi (n) \overset{(1)}{=} (p-1)(q-1)$
\item $e$ wird so gewählt, dass $ggT(e,m)=1$
\item $d:= e^{-1} (mod \; m)$ (modulare Inverse)
\item $(n,e)$ … öffentlicher Schlüssel\\
$(n,d)$ … geheimer Schlüssel (geheim ist nur $d$)\\
$p$, $q$ und $m$ werden nicht mehr benötigt, bleiben aber geheim!
\end{enumerate}
\item Verschlüsselung: \\
Klartext $a$ teilerfremd zu $n$ verschlüsseln mit $e$, d.h. $b:\equiv a^e (mod \; n)$ bilden ($b$ … Geheimtext)
\item Entschlüsselung: \\
Der Empfänger und Besitzer des geheimen Schlüssels bildet $b^d (mod \; m)$ und erhält $b^d\equiv a (mod \; n)$ denn $b^d\equiv (a^e)^d \equiv a^{ed}\equiv a ^{1+k\cdot m}\equiv a\cdot \underset{\equiv 1 \text{ wegen Satz 5}}{\left( a^{\varphi(n)}\right)}^k\equiv a (mod \; n)$.
\item Praktische Durchführung vgl Übungsaufgabe 2.4
\end{itemize}

\subsection{Reelle Zahlen}

$\mathbb{R}$ … Menge der reellen Zahlen.\\
Auf $\mathbb{R}$ existiert eine algebraische Struktur und eine Ordnungsstruktur.

\subsubsection{Algebraische Struktur}
$(\mathbb{R}, + ,\cdot)$ mit den arithmetischen Operationen $+$ (Addition) und $\cdot$ (Multiplikation) ist ein Körper.
\paragraph{Def. 7:}
\begin{enumerate}[label=\alph*.)]
\item $\boxed{0! := 1, \; n!=n\cdot (n-1)!}$ mit $n\in \mathbb{N}^*$ \emph{Fakultät} (rekursive Funktion)
\item Sei $\alpha \in \mathbb{R}, k\in \mathbb{N}^*$, dann sei $\boxed{\nok{\alpha}{0}:=1, \nok{\alpha}{k}:= \frac{\alpha}{k}\nok{\alpha-1}{k-1}}$ Binominalkoeffizient $\alpha$ über $k$.\\
d.h. $\boxed{\nok{\alpha}{k}= \frac{\alpha(\alpha-1)(\alpha-2)...(\alpha-k-1)}{k!}}$
\end{enumerate}

\subparagraph{Diskussion:}
\begin{enumerate}
\item Für $k, n \in \mathbb{N}, \; 0 \leq k \leq n$ gilt $\boxed{\nok{n}{k}=\nok{n}{n-k}=\frac{n!}{k!(n-k)!}}$
\item Für $k \in \mathbb{N}, \alpha \in \mathbb{R}$ gilt $\boxed{\nok{\alpha}{k}+\nok{\alpha}{k+1}=\nok{\alpha+1}{k+1}}$
\item \emph{Binomischer Lehrsatz}: $(a+b)^k=\sum_{k=0}^n \nok{n}{k} a^{n-k} \cdot b^k=a^n+n\cdot a^{n-1} \cdot b + \nok{n}{2}a^{n-2}\cdot b^2+...+b^n$\\
\emph{Pascalsches Dreieck}:
\begin{tabular}{c c c c c c c c c c c r}
   &   &   &   &   & 1 &   &   &   &   &   & \textcolor{red}{$n=0$} \\
   &   &   &   & 1 &   & 1 &   &   &   &   & \textcolor{red}{$1$} \\
   &   &   & 1 &   & 2 &   & 1 &   &   &   & \textcolor{red}{$2$} \\
   &   & 1 &   & 3 &   & 3 &   & 1 &   &   & \textcolor{red}{$3$} \\
   & 1 &   & 4 &   & 6 &   & 4 &   & 1 &   & \textcolor{red}{$4$} \\
 1 &   & 5 &   & 10&   & 10&   & 5 &   & 1 & \textcolor{red}{$5$} \\

\end{tabular}
(Spalte $\widehat{=}k$ in $\nok{n}{k}$)
\end{enumerate}

\paragraph{Stellenwertsysteme:}
\begin{itemize}
\item Es sei $k>1$ eine natürliche Zahl (die sogenannte Basis)
\item $x = (\underbrace{x_p x_{p-1}...x_1x_0}_{\text{Vorkomma}},\;\underbrace{x_{-1}x_{-2}...x_{-q}}_{\text{Nachkomma}})_b\\
:= \underbrace{x_p\cdot b^p+x_{p-1}\cdot b^{p-1}+...+x_1\cdot b^1+x_0\cdot b^0}_{\text{Vorkomma}}+\underbrace{x_{-1}\cdot b^{-1}+x_{-2}\cdot b^{-2}+...+x_{-q}\cdot b^{-q}}_{\text{Nachkomma}}$\\
heißt Darstellung zur Basis b (*).
\end{itemize}

\subparagraph{Bsp. 5:}
\begin{itemize}
\item $b=2$ … \emph{Dual- oder Binärsystem} (Ziffern $\{0,1\}$)
\item $b=3$ … Trialsystem
\item $b=10$ … \emph{Dezimalsystem}
\item $b=16$ … \emph{Hexadezimalsystem} (Ziffern $\{0,1,2,...,9,\underset{10}{A},\underset{11}{B},\underset{12}{C},\underset{13}{D},\underset{14}{E},\underset{15}{F}\}$
\end{itemize}
z.B. $(47)_{10}=(101111)_2=(1202)_3=(57)_8=(2F)_{16}$
\paragraph{Übergang von einem Ziffernsystem zu einem anderen}\parskp
z.B. $p=3,\; q=2$
\begin{align*}
x&=x_3b^3+x_2b^2+x_1b^1+x_0+x_{-1}b^{-1}+x_{-2}b^{-2}\\
&= (x_3b^2+x_2x^1+x_1)b+x_0+(x_{-1}+x_{-2}b^{-1})b^{-1}\\
&=((x_3b^1+x_2)b+x_1)b+x_0+(x_{-1}+x_{-2}b^{-1})b^{-1}
\end{align*}
Grundlage: fortgesetztes Klammern:
\begin{align*}
x&=((....((x_p b+x_{p-1})b+x_{p-2})b+...+x_2)b+x_1)b+x_0+\\
&\qquad((...(x_{-q}b^{-1}+x_{-(q-1)})b^{-1}+...+x_{-2})b^{-1}+x_{-1})b^{-1}
\end{align*} (**)
\subparagraph{Bsp. 6:} Übergang Dezimalsystem $\rightarrow$ anderes System
\begin{itemize}
\item ganzer Teil: fortgesetzte Division durch $b$ und Restabspaltung liefert $b$-Ziffern in der Reihenfolge $x_0,x_1,x_2,...$
\item gebrochener Teil: fortgesetzte Multiplikation mit $b$ und Abspaltung des ganzzahligen Anteils liefert $b$-Ziffern in der Reihenfolge $x_{-1},x_{-2},...$
\end{itemize}
z.B. Dezimalsystem $\rightarrow$ Hexadezimalsystem ($b=16$)\\
$x=435,9$\\
ganzer Teil:\\
\begin{tabular}{r c l l}
$435:16=$ & $27$ & Rest $3$ &  $\rightarrow x_0$\\
$27:16=$ & $1$ & Rest $11$ &  $\rightarrow x_1$\\
$1:16=$ & $\boxed{0}$ & Rest $1$ &  $\rightarrow x_2$\\
\end{tabular}\\
gebrochener Teil: \\
\begin{tabular}{r c l l}
$0,9 \cdot 16=$ & $0.4$ & $+14$ & $\rightarrow x_{-1}$\\
$0,4 \cdot 16=$ & $\boxed{0.4}$ & $+6$ & $\rightarrow x_{-2}$ (Periode, da gleicher „Nachkommarest“)\\
\end{tabular}\\
$\curvearrowright x=(1B3,E\overline{6})_{16}$

\subparagraph{Bsp. 7:} Übergang anderer Systeme $\rightarrow$ Dezimalsystem\\
Entweder direkte Auswertung von (*) (v.a. beim Dualsystem $\rightarrow$ Addition von 2er-Potenzen) \emph{oder} Klammern in (**) von innen nach außen berechnen (zweckmäßig HORNER-Schema).
\begin{itemize}
\item ganzer Teil: beginnend mit $x_p$\\
ABB1 
\item gebrochener Teil: beginnend mit $x_{-q}$\\
ABB 2
\end{itemize}
$x=(1E2,B8)_{16}$\\
ganzer Teil:\\
\begin{tabular}{c | c c c}
 & $1$  &  $14$  & $2$\\
 $16$  &   & $16$  & $480$\\
 \hline
  & $1$ & $30$ & $\boxed{482}$\\

\end{tabular}\\
gebrochener Teil:\\
\begin{tabular}{c | c c c}
 & $8$  &  $11$  & *\\
 $\frac{1}{16}=0,0625$  &   & $0,5$  & $\boxed{0,71875}$\\
 \hline
  & $8$ & $11,5$ & \\
\end{tabular}\\
$\curvearrowright x=\resultul{482,71875}$

\subparagraph{Bsp. 8:} Hexadezimalsystem $\leftrightarrow$ Dualsystem\\
4 Dualziffern entsprechen einer Hexadezimalziffer ($2^4=16$) $\curvearrowright$ 4er Gruppen von Dualziffern ab Komma bilden.
\begin{enumerate}[label=\alph*.)]
\item $(A8C,B2)_{16}=(1010 \; 1000 \; 1100, \; 1011 \; 1011 \; 001(0))_2$
\item $((0)110\; 1110, \; 101(0))_2 = (6E,A)_{16}$
\end{enumerate}

\subsubsection{Zahlendarstellung im Computer}

\begin{enumerate}
\item Ganze Binärzahlen in Zweierkomplementdarstellung ($n$ Bit, meist $n=8,16,32,64$)
\begin{itemize}
\item Bsp.: $n=8 \qquad (100)_{^10}=(64)_{16}$\\
\begin{tabular}{|c | c | c |c | c | c |c | c |}
\hline
\textcolor{red}{0} & 1 &1&0&0&1&0&0\\
\hline
$2^7$&$2^6$&$2^5$&$2^4$&$2^3$&$2^2$&$2^1$&$2^0$\\
\hline
\end{tabular}\\
\textcolor{red}{MSB}: most siginficant bit\\
(LSB: least significant bit)
\item Um auch negative Zahlen darstellen zu können, wird das MSB als Vorzeichen reserviert. Negative Zahlen $-a$ ($1\leq a \leq 2^{n-1}$) werden im sogenannten Zweierkomplement $\overline{a}:=2^n-a$ dargestellt ($\curvearrowright \overline{a}\geq 2^{n-1} \curvearrowright MSB =1$)
\item Nightnegtavie Zahlen $0\leq a \leq 2^{n-1}-1$ werden unverändert dargestellt ($MSB=0$)
\item Damit Darstellung ganzer Zahlen von $-2^{n-1}$ bis $2^{n-1}-1$ (Anzahl $2^n$) möglich, $n=8: -128 bis 127$.
\item Umwandlung negativer Zahlen $\rightarrow$ Zweierkomplement
\subparagraph{Bsp. 9:} $n=8$, umzuwandeln sei $-100$ (dezimal)\\
zwei Möglichkeiten:
\begin{enumerate}[label=\arabic*.)]
\item  (für die Handrechnung): $\overline{100}=\underbrace{2^8}_{=256}-100=\underbrace{156}_{(9C)_{16}}=$\begin{tabular}{|c|c c c|c c c c|}
\hline
\textcolor{red}{$1$}&$0$&$0$&$1$&$1$&$1$&$0$&$0$\\
\hline
\end{tabular}\\
\emph{Bemerkung}: Das Zweierkomplement der positiven Zahl $100$ ist die positive Zahl $156=\overline{100}$, diese wird wegen $MSB=1$ als negative Zahl $-100$ interpretiert.
\item (am schnellsten): Rechts (beim LSB) beginnend alle Ziffern bis einschließlich der ersten $1$ übernehmen (unverändert lassen), für alle höherwertigen Ziffern $Z$ das \emph{Einerkomplement} $1-z$ bilden:\\
$(100)_{10}=$\begin{tabular}{|c|c c c|c c c c|}
\hline
$0$&$1$&$1$&$0$&$0$&$1$&$0$&$0$\\
\hline
\end{tabular}\\
\begin{tabular}{|c|c c c|c c c c|}
\hline
\textcolor{red}{$1$}&\textcolor{red}{$0$}&\textcolor{red}{$0$}&\textcolor{red}{$1$}&\textcolor{red}{$1$}&\textcolor{green}{$1$}&\textcolor{green}{$0$}&\textcolor{green}{$0$}\\
\hline
\end{tabular} $=(\overline{100})_{10}=(156)_2$\\
Rückumwandlung (Zahl mit $MSB=1 \rightarrow$ neg. Zahl) analog:\\
$\overline{156}=256-156=100\rightarrow \resultul{-100}$\\
Die Subtraktion wird damit auf die Addition des Zweierkomplements zurückgeführt.
\end{enumerate}
\subparagraph{Bsp. 10:} $a=64-100=64+(-100)$
\begin{align*}
64=2^6 &= 0100 \, 0000\\
-100 &= 1001 \, 1100 \quad +\\
Summe &= \textcolor{red}{1}101\, 1100 \quad \text{\textcolor{red}{Ergebins negativ}}\\
36 &= 0010\, 0100 \quad \text{dargestellst ist aber }\overline{z}=2^n-z
\end{align*}
$\curvearrowright$ Ergebnis: $a=-36=-z$\\
Ein \emph{Überlauf} (Ergebnis $\geq 2^{n-1}$ oder $<-2^{n-1}$) ensteht in folgenden Fällen ($\rightarrow$ ERROR!)
\begin{tabular}{r | c |c |c l}
 & $a$ & $b$ & $a+b$ & \\
 \hline
 MSB & 0 & 0& \textcolor{red}{1} & ($MSB = 0$ erwartet!)\\
 \hline 
 MSB & 1 & 1& \textcolor{red}{0} & ($MSB = 1$ erwartet!)\\
\end{tabular}\\
Bemerkung: Für die Handrechnung (z.B. $2-5=:a$) kleinere zahl von größerer Subtrahieren $a=-(5-2)$, dabei genügt es für $n$ die Binärstellenzahl des Minuenden $(5)_{10}=(101)_2$ also $n=3$  zu verwenden. Es wird dabei ausschließlich mit nicht-negativen Zahlen gerechnet ($0,1,...,2^n-1$):\\
$(5-2)_{10}=((5+\textcolor{red}{2^n}-2)-\textcolor{red}{2^n})_{10}=(5+\overline{2}-\textcolor{red}{2^n})_{10}$\\
$(2)_{10}=(010)_2 \curvearrowright \overline{2}=(110)_2$\\
$(5)_{10}=(101)_2$\\
\begin{tabular}{r l}
$5$:& $101$\\
$\overline{2}$:& $110 \quad +$\\
$\textcolor{red}{1}$& $011$
\end{tabular}\\
\textcolor{red}{vordere Stelle $2^n$ ignorieren}\\
$\curvearrowright 5-3=3=(011)_2 \curvearrowright \resultul{a=-3}$
\end{itemize}
\item Gleitkommasysteme\\
$\boxed{x=v\cdot m \cdot b^e}$ dabei
\begin{itemize}
\item $v=(-1)^V$ … \emph{Vorzeichen} $\begin{cases}
V= 0 \quad \text{positive Zahl}\\
V=1 \quad \text{negative Zahl}
\end{cases}$
\item $m$ … \emph{Mantisse}, Stellenzahl $p$, die Mantisse heißt \emph{normalisiert} falls sie folgende Gestalt besitzt:\\
 $m_1, m_2, ..., m_p$ oder $0, m_1, m_2, ..., m_p$ mit $m\not = 0$. Dabei sind  $m_1, m_2, ..., m_p$ die Ziffern zur Basis $b$.
\item $e$ … \emph{Exponent}, ganzzahlig $e_{min}\leq e \leq e_{max}$.
\end{itemize}
In jedem Gleitkommasystem sind nur endlich viele Zahlen darstellbar, die Menge der reellen zahlen ist aber überabzählbar (unendlich).\\
Gleitkommazahlen liegen auf der Zahlengeraden diskret verteilt (fester Exponent $\curvearrowright$ gleiche Abstände, wächst Exponent um $k$, so wachsen die Abstände auf $b^k$-fache!)\\
Veranschaulichung für $b=10$, Mantissenlänge $p=1$:\\
\begin{tikzpicture}[scale=0.9]
\draw (0,0) -- (11,0);
\draw (0,0) node[below]{0};
\draw (0,0.15)--(0,-0.15);
\draw (1,0.15)--(1,-0.15);
\draw (2,0.15)--(2,-0.15);
\draw (3,0.15)--(3,-0.15);
\draw (4,0.15)--(4,-0.15);
\draw (5,0.15)--(5,-0.15);
\draw (6,0.15)--(6,-0.15);
\draw (7,0.15)--(7,-0.15);
\draw (8,0.15)--(8,-0.15);
\draw (9,0.15)--(9,-0.15);
\draw (10,0.15)--(10,-0.15);
\draw (1,0) node[below]{\textcolor{red}{1}};
\draw (2,0) node[below]{\textcolor{red}{2}};
\draw (3,0) node[below]{\textcolor{red}{3}};
\draw (4,0) node[below]{\textcolor{red}{4}};
\draw (5,0) node[below]{\textcolor{red}{5}};
\draw (6,0) node[below]{\textcolor{red}{6}};
\draw (7,0) node[below]{\textcolor{red}{7}};
\draw (8,0) node[below]{\textcolor{red}{8}};
\draw (9,0) node[below]{\textcolor{red}{9}};

\draw[green] (0.1,0.15)--(0.1,-0.15);
\draw[green] (0.2,0.15)--(0.2,-0.15);
\draw[green] (0.3,0.15)--(0.3,-0.15);
\draw[green] (0.4,0.15)--(0.4,-0.15);
\draw[green] (0.5,0.15)--(0.5,-0.15);
\draw[green] (0.6,0.15)--(0.6,-0.15);
\draw[green] (0.7,0.15)--(0.7,-0.15);
\draw[green] (0.8,0.15)--(0.8,-0.15);
\draw[green] (0.9,0.15)--(0.9,-0.15);


\draw (10,0) node[below]{\textcolor{cyan}{$1\cdot 10^1$...}};
\end{tikzpicture}\\
\emph{Exponent} 0: $1\cdot 10^{\textcolor{red}{0}}, 2\cdot 10^{\textcolor{red}{0}}, ... , 9\cdot 10^{\textcolor{red}{0}}$\\
\emph{Exponent} -1: $1\cdot 10^{\textcolor{green}{-1}}, 2\cdot 10^{\textcolor{green}{-1}}, ... , 9\cdot 10^{\textcolor{green}{-1}}$\\
\emph{Exponent} 1: $1\cdot 10^{\textcolor{cyan}{1}}, 2\cdot 10^{\textcolor{cyan}{1}}, ... , 9\cdot 10^{\textcolor{cyan}{1}}$\\
\emph{Rundung}: Zahlen, die nicht in diesem „Raster“ enthalten sind, werden auf dei nächstgelegene darstellbare Gleitkommazahl gerundet. Falls die Zahl genau in der Mitte zwischen zwei darstellbaren Zahlen liegt, wird auf die gerade Zahl gerundet (Bsp. $3,75 \rightarrow 3,8$ oder $4,65 \rightarrow 4,6$ bei Rundung auf eine Stelle nach dem Komma).
\paragraph{Numerische Probleme beim Rechnen mit Gleitkommazahlen}
\begin{itemize}
\item Kommutativ-, Assoziativ- und Distributivgesetze gelten im allgemeinen nicht mehr. Ursachen sind bspw. Ziffernauslöschung bei der Subtraktion fast gleicher Zahlen, Addition oder Subtraktion von Zahlen unterschiedlicher Größenordnung.
\subparagraph{Bsp. 11:}
\begin{enumerate}[label=\arabic*.)]
\item Man berechne $(a+b)+c$ und $a+(b+c)$ in einem System mit 3-Stelliger Mantisse:\\
$a=3,73\cdot 10^6$, $b=-3,71\cdot 10^6$ und $c=6,42\cdot 10^3$
\begin{itemize}
\item $a+b= 0,02\cdot 10^6=2,00 \cdot 10^4$ (Normalisierung)\\
$c=0,642\cdot 10^4=0,64\cdot 10^4$ (Exponentenangleichung und Rundung)\\
$(a+b)+c=2,64\cdot 10^4=\resultul{26400}$
\item $c=0,00642\cdot 10^6=0,01\cdot 10^6$ (Exponentenangleichung und Rundung)\\
$b+c=-3,,70\cdot 10^6$\\
$a+(b+c)=0,03\cdot 10^6=3,00\cdot 10^4 = \resultul{30000}$
\item exakter Wert: $a+b+c = \resultul{26420}$
\end{itemize}
\item Aufgabe der numerischen Mathematik ist es, die unvermeidlichen Genauigkeitsverluste beim Rechnen mit Maschinenzahlen durch optimale Organisation (Reihenfolge) der Rechnung und Fehleranalyse in Grenzen zu halten.
\end{enumerate}
\end{itemize}
\item Gleitkommaformat IEEE 754 (single precision, 32 Bit)\\
$x=v\cdot m \cdot b^e=(-1)^{\torange{V}}\cdot 1, \tblue{m_2\,m_3...m_24}\cdot 2^{\tred{E}-B}$ ($b=2$, Binärsystenm)
\begin{itemize}
\item Vorzeichen $V=0\curvearrowright$ positiv, $V=1\curvearrowright$ negativ (1 Bit)
\item Mantisse $m_1$ im Binärsystem stets gleich $1$.\\
$\curvearrowright$ nur Abspeicherung von $\tblue{\boxed{M=m_2\,m_3...m_24}}$ (23 Bit)
\item Exponent: Abgespeichert wird $\tred{\boxed{E:=e+B}} $ \\
mit dem sogenannten \emph{Biaswert $B=127$} (Bias = Verzerrung) als nichtnegative 8-stellige Binärzahl, $e_{min}=-126 $ ($E=1$), $e_{max}=127$ ($E=254=(1111\;1110)_2)$, die Grenzfälle $E=(0000\;0000)_2$ und $E=(1111\;1111)_2$ sind für Sonderfälle ($0$, $\infty$, nichtdefinierte Werte) vorgesehen.
\end{itemize}
Abspeicherung in der Reihenfolge $\boxed{\torange{V}\tred{E}\tblue{M}}$ (32 Bit)
\subparagraph{Bsp. 12:} Umwandlung einer Dezimalzahl in das IEEE 754-Format (32-Bit), $x=435,9$ (vgl. Bsp. 6)
\begin{enumerate}[label=\arabic*.)]
\item Konvertierung in Dualzahl (unter Verwendung von Bsp. 6/Hexadez.)\\
$x=1B3,E\overline{6})_{16}=(1\; 1011\;0011,1100\;0110\;0110 ...)_2$
\item Normalisierte Gleitkommadarstellung, Mantisse mit 23 Stellen nach dem Komma, Kommaverschiebung um 8 Stellen.\\
$\curvearrowright x = 1, \underbrace{1011\;0011\;1110\;0110\;0110\;011}_{\tblue{M}}(0\; 0110...))_2\cdot 2^8$ \quad (Abrundung!)
\item Exponent $e=8\curvearrowright E=e+B=8+127=135=\underbrace{(1000\;0111)_2}_{E}$
\item Vorzeichenbit $V=0$ (da $x$ positiv)
\end{enumerate}
$\curvearrowright x: \quad \boxed{0 \; \tred{1000\;0111}\;\tblue{1011\;0011\;1110\;0110\;0110\;011}}$
\paragraph{Bsp. 13:} IEEE 754$\rightarrow$ Dezimalzahl\\
$\boxed{1\;\tred{1000\;0011}\;\tblue{0111\; 1100 \; 0000\;0000\;0000 \; 000}}$
\begin{enumerate}[label=\arabic*.)]
\item $E=(\tred{1000\;00111})_2=131\curvearrowright e=E-B=131-127=\resultul{4}$
\item $V=1\curvearrowright$ negativ, normalisierte Mantisse $1,\tblue{M}$\\
$\curvearrowright x=-(1,\tblue{011111})_2\cdot 2^4=-(10111,11)_2$\\
$\curvearrowright x = -23,75$
\end{enumerate}
Bemerkung: 
\begin{enumerate}[label=\arabic*.)]
\item Neben dem single-Format gibt es in IEEE 754 das double-Format (54 Bit, V=1Bit, E=11Bit, M=52 Bit, B=1023) sowie das erweiterbare Format
\item Zahlbereiche single: $1,401\cdot 10^{-45}...3,403\cdot 10^{38}$, double: $4,941\cdot 10^{-324}...1,798\cdot 10^{308}$
\end{enumerate}
\end{enumerate}
\subsubsection{Ordnungsstruktur}
\begin{itemize}
\item Durch $\leqq$ (auch $\leq$) ist auf $\mathbb{R}$ eine vollständige Ordnungsrelation erklärt.
\item Verträglichkeit mit der algebraischen Struktur (für alle $x,y,z\in \mathbb{R}$):
\begin{enumerate}[label=(\arabic*)]
\item $x\leq y \quad \Rightarrow \quad x+z \leqq y+z$
\item $(x\leq y) \wedge (z \geq 0 )\quad \Rightarrow \quad x \cdot z \leq y \cdot z$\\
$(x\leq y) \wedge (z \leq 0 )\quad \Rightarrow \quad x \cdot z \geq y \cdot z$
\end{enumerate}
Für die strikte Ordnung $<$ gilt:\\
$\boxed{(x<y)\wedge (z<0) \quad \Rightarrow \quad x \cdot > y \cdot z}$
\end{itemize}
\paragraph{Def. 8:} \parskp
Sei $x$ eine reele Zahl. Dann heißt $|x|:=\begin{cases}
x \quad \text{ für } x\geq 0\\
-x \quad x <0
\end{cases}$ der (absolute) Betrag von $x$.\\
ABB 5\\
Vorzeichenfunktion $sgn(x):= \begin{cases}
1 \quad x>0\\
0 \quad x=0\\
-1 \quad x<0
\end{cases}$
\subparagraph{Diskussion:} Es gilt:
\begin{enumerate}
\item $|a-b|$ = „Abstand der Zahlen $a$ und $b$ auf der Zahlengeraden“\\
ABB 6\\
(speziell: $|a|$ = „Abstand von $a$ zum Ursprung“)
\item $a = |a| \cdot sgn (a)$
\item $|a|=|-a|, |ab|=|a|\cdot |b|, \left|\frac{a}{b}\right|=\frac{|a|}{|b|}$ (falls $b\not = 0$)
\item $|a \pm b| \leq |a| + |b|$ (\emph{Dreiecksungleichung}) für alle $a,b \in \mathbb{R}$
\end{enumerate}
\paragraph{Lösung von Ungleichung}
\subparagraph{Bsp. 14:} (Ungleichung mit Beträgen)\\
Gesucht sei die Lösungmenge $L$ der reellen Zahlen, die die Ungleichung $|x-1|<3+\frac{1}{2}x$ (*) erfüllen.
\begin{itemize}
\item kritische Stelle(n): Nullstellen des Terms innerhalb der Betragsstriche d.h. $x=1 \curvearrowright$ Fallunterscheidung\\
ABB 7
\item 1. Fall: $x-1<0$ d.h. $x<1$\\
in (*): \\
$\tred{-}(x-1)<3+\frac{1}{2}x \Leftrightarrow \\
-\frac{3}{2}x<2 \Leftrightarrow \\
\resultul{x>-\frac{4}{3}}$\\
$\curvearrowright L_1=\left\lbrace x|(x<1) \wedge x>\frac{4}{3}\right\rbrace = \left(-\frac{4}{3},1\right)$
\item 2. Fall $x-1 \geq 0$ d.h. $x\geq 1$\\
in (*):\\
$x-1< 3+ \frac{1}{2}x \Leftrightarrow\\
\frac{1}{2}x<4 \Leftrightarrow\\
\resultul{x<8}$\\
$\curvearrowright L_2 = \{ x| (x\geq 1) \wedge (x <8)\}=[1,8)$
\item $\Rightarrow L = L_1 \cup L_2 = \resultul{\left(-\frac{4}{3},8\right)}$
\end{itemize}
\subparagraph{Bsp. 15:} (Ungleichung mit gebrochen rationalen Termen)\\
$\frac{x}{x+1}<1$ (*)
\begin{itemize}
\item kritische Stelle(n): Nenner-Nullstellen, hier: $x=-1$.\\
ABB 7 (äquiv. mit -1)
\item 1. Fall: $x<-1$\\
in (*):\\
$\Leftrightarrow x>x+1 \Leftrightarrow 0>1$ (Widerspruch)\\
$L_1=\emptyset$.
\item 2. Fall: $x > -1$\\
in (*):\\
$\Leftrightarrow x < x+1 \Leftrightarrow 0<1$ (wahre Aussage)\\
$L_2=\{x|x>-1\}=(-1,\infty)$
\item $\Rightarrow L = L_1 \cup L_2 = \resultul{(-1,\infty)}$
\end{itemize}

\subparagraph{Bsp. 16:} (quadratische Ungleichungen)\\
$x^2+3x<10 \Leftrightarrow$ \\
$\left( x+\frac{3}{3}\right)^2-\frac{9}{4}<10 \Leftrightarrow$ (vereinfacht durch quadratische Ergännzung)\\
$\left( x+\frac{3}{2}\right)^2<\frac{49}{4} \Leftrightarrow\\
\left|x+\frac{3}{2}\right|<\frac{7}{2} \Leftrightarrow$ (Äquivalenz siehe Übung) \\
$\frac{-7}{2}<x+\frac{3}{2}<\frac{7}{2} \Leftrightarrow\\
-5<x<2$\\
$L=\resultul{(-5,2)}$\medskip\\
Bemerkung:\\
In vielen Fällen ist auch ein graphischer Lösungsansatz möglich. Dabei sind geeignete Schnittpunkte ($\widehat{=}$Gleichung) exakt rechnerisch zu ermitteln, ausschließend Ungleichheitszeichen betrachten.

\paragraph{in Bsp. 16:} \parskp
$x^2+3x<10 \qquad \Leftrightarrow \qquad \underbrace{x_2+3x-10}_{=f(x)}\tred{<}0$\\
Nullstellen von $f(x)$: \\
$x^2+3x-10=0$\\
$x_{1,2}=-\frac{3}{2}\pm \sqrt{\frac{9}{4}+10}=\begin{cases}
-5\\
2
\end{cases}$\\
$\curvearrowright$ Grobskizze von $y=f(x)$ (Parabel, nach oben geöffnet)\\
ABB 81\\
$\tred{\curvearrowright} L=(-5,2)$

\paragraph{Schranken und Grenzen}
\begin{itemize}
\item Eine Menge $M\subseteq \mathbb{R}$ heißt nach oben beschränkt, wenn es eine obere Schranke gibt, vgl. \ref{sec:Mengen}. Man kann zeigen, dass es bei diesen Ordnungsrelationen ($\leq$) auf $\mathbb{R}$ dann auch eine kleinste obere Schranke (\emph{Supremum, $sup\;M$, $s=max \; M$} falls $s \in M$)
\item Analog: nach \emph{unten beschränkt, Infimum, Minimum}.
\item Falls $M$ \emph{nicht} nach oben beschränkt ist, d.h. es gilt:\\
$\exists a \in \mathbb{R}\quad \forall x \in M\quad x \leq a = \forall a \in \mathbb{R} \quad \exists x \in M \quad x >a$, dann Schreibweise $\boxed{sup\; M := \infty}$
\item Analog: in $\boxed{inf\; M:= - \infty}$ falls $M$ \emph{nicht} nach unten beschränkt.
\item $M$ heißt \emph{beschränkt}, falls $M$ nach oben und unten beschränkt ist.
\end{itemize} 

\subparagraph{Bsp. 17:} \parskp
$M=\left\lbrace 1+\frac{1}{n}\;|\;n\in \mathbb{N}^*\right\rbrace$
\begin{itemize}
\item \emph{obere Schranken}: $\pi$, $2300$, $7$, $2,01$ usw.\\
kleinste obere Schranke: $sup \; M = 2= max \; M$
\item \emph{untere Schranken}: $-31$, $0$, $0,99$ usw.\\
größte untere Schranke: $inf\; M =1$ ($1 \not \in M \curvearrowright min \; M$ existiert nicht)
\end{itemize}
ABB 82

\subsection{Komplexe Zahlen}

\paragraph{Motivation:} z.B. $\boxed{x^2+1=0}$ ($\Leftrightarrow x^2=-1$) im Bereich der reellen Zahlen nicht lösbar.
$\curvearrowright$ Zahlenbereichserweiterung

\subsubsection{Begriff, Rechenregeln}

Die Menge $\mathbb{C}$ der komplexen Zahlen ist eine Obermenge der Menge der reellen Zahlen mit folgenden Eigenschaften:
\begin{enumerate}
\item $\mathbb{C}$ enthält eine Zahl $i$ mit $\boxed{i^2=-1}$ (oft auch mit $j$ bezeichnet)
\item Jede komplexe Zahl $z$ lässt sich in der Form $\boxed{z=x+i\cdot y} \quad (x,y \in \mathbb{R})$ darstellen.\\
Dabei $x=Re(z)$ (Realteil) und $y=Im(z)$ (Imaginärteil)
\item Auf $\mathbb{C}$ werden die Operationen $+$ (Addition) und $\cdot$ (Multiplikation) wie folgt erklärt: \\
Sei $z_1=x_1+i\cdot y_1$, $z_2=x_2+i\cdot y_2$ Dann gilt: \\
$z_1+z_2:= (x_1+x_2)+ i (y_1+y_2)$\\
$z_1\cdot z_2 := (x_1 x_2-y_1 y_2)+i (x_1 y_2 + x_2 y_1)$\\
Die Menge $\mathbb{C}$ wird mit diesen Operationen zum \emph{Körper der komplexen Zahlen}. Die arithmetischen Operationen erfolgen unter Beachtung von $i^2=-1$ wie im Reellen.
\item Auf $\mathbb{C}$ gibt es keine natürliche Ordnungsrelation.\\
Veranschaulichung: \emph{GAUSSsche Zahlenebene}\\
Zahl $z \leftrightarrow$ Punkt $(x,y) \leftrightarrow$ Vektor $\overrightarrow{OP}$\\
ABB 83
\begin{itemize}
\item \emph{Betrag} von $z$: \\
$|z|=\sqrt{x^2+y^2}$
\item Hauptarrgument von $z$: orientierter Winkel zwischen positiver x-Achse und $\overrightarrow{OP}$ (gemessen auf kürzestem Wege)\\
$Arg(z):= \varphi \qquad (-\pi < \varphi \leq \pi )$
\item zu $z$ konjugiert komplexe Zahl $\overline{z}:$\\
$\overline{z} := x - i \cdot y$
\end{itemize}
\end{enumerate}

\paragraph{Diskussion:}
\begin{itemize}
\item Falls nicht notwendig kürzester Weg gewählt wird: Argument $arg(z)= Arg(z) + 2\pi k \quad (k \in \mathbb{Z})$\\
z.B. $z=1-i \quad : \quad Arg(z)=-45^{\circ}=-\frac{\pi}{4}$, ein (Neben-)argument bspw. $arg(z)=315^{\circ}$
\item Berechnung von $Arg(z) \; (z \not = 0)$: $cos \varphi = \frac{x}{|z|}$, $sin\varphi = \frac{y}{|z|}$\\
$\boxed{ Arg (z)= \begin{cases}
+arccos \frac{x}{|z|}\qquad \text{falls } y \leq 0\\
-arccos \frac{x}{|z|} \qquad \text{falls } y < 0
\end{cases}}$
\end{itemize}
\subparagraph{Bsp. 18:} \parskp
$z_1 = 3 + 4i$, $z_2=-12-5i$\\
ABB 84
\begin{enumerate}[label=\alph*.)]
\item Betrag und Hauptargument\\
$|z_1|=\sqrt{3^2+4^2}=5$\\
$\varphi_1 = Arg(z_1) = arccos \frac{3}{5} \approx 53,13 ^{\circ}$\\
$|z_2| = 13$\\
$\varphi_2=Arg(z_2) = - arccos -\frac{12}{13} \approx -157,38 ^{\circ}$
\item Arithmetische Operationen:\\
$z_1 + z_2 = -9 - i$\\
$z_1 - z_2 = 15 + 9 i$\\
$z_1 \cdot z_2 = -16 - 63 i$\\
$\frac{z_1}{z_2}=\frac{z_1 \cdot \overline{z_2}}{z_2 \cdot \overline{z_2}}=\frac{z_1 \cdot \overline{z_2}}{|z_2|^2}=-\frac{56}{169}-\frac{33}{169}i$
\end{enumerate}

\subsubsection{Darstellungsformen komplexer Zahlen}
\begin{itemize}
\item Trigonometrische Darstellung
\item EULERsche Form
\item exponentielle Darstellung
\end{itemize}
ABB 91\\
$\boxed{z=x+iy}$ \qquad (\emph{arithmetische Darstellung})\\
$x=|z|\cdot cos \varphi$\\
$y=|z|\cdot sin \varphi$\\
$\curvearrowright \boxed{z=|z|\cdot (cos\varphi + i\; sin \varphi)}$ \qquad (\emph{trigonometrische Darstellung})\\
(und $\varphi=arg\; z$, meist $\varphi = Arg \; z$)
\subparagraph{Diskussion:} \parskp
$z_1= |z_1|\cdot (cos \varphi_1 + i \; sin \varphi_1)$\\
$z_2= |z_2|\cdot (cos \varphi_2 + i \; sin \varphi_2)$\medskip\\
$z_1 \cdot z_2 = |z_1| \cdot |z_2| \cdot ( \underbrace{(cos\varphi_1 \; cos \varphi_2 - sin\varphi_1 \; sin \varphi_2)}_{cos(\varphi_1+\varphi_2)}+i\underbrace{(sin \varphi_1 \; cos \varphi_2 + sin \varphi_2 \; cos \varphi_1)}_{sin(\varphi_1+\varphi_2)})$\\
$\boxed{z_1 \cdot z_2 = |z_1| \cdot |z_2| \cdot(cos (\varphi_1+\varphi_2)+i\; sin (\varphi_1+\varphi_2))}$\\
Folgerung:\\
$\boxed{|z_1\cdot z_2| = |z_1| \cdot |z_2|}\\
\boxed{arg(z_1\cdot z_2)=arg(z_1)+arg(z_2)}$\\
Analog:\\
$\boxed{\left|\frac{z_1}{z_2}\right| = \frac{|z_1|}{|z_2|} \qquad (z_2 \not = 0)} \\
\boxed{arg\left(\frac{z_1}{z_2}\right) = arg(z_1) - arg (z_2)}$\\
Es ist also sinnvoll zu definieren:
\paragraph{Def. 10:} \parskp
$\boxed{e^{i\varphi}=cos\varphi+i\; sin \varphi}$ \qquad (\emph{EULERsche Form})
\subparagraph{Diskussion:}
\begin{enumerate}
\item \emph{Exponentielle Darstellung} von z: $\boxed{z=|z|\cdot e^{i\varphi}}$
\item Wegen der obigen Formeln bleiben für diese Darstellung die vom Reellen bekannnten Potenzgesetze gültig.\\
Insbesondere gilt die Formel von MOIVRE: \\
$\boxed{z^n=\left(|z|\cdot e^{i\varphi}\right)^n=|z|^n\cdot e^{i\varphi n}=|z|^n(cos(n\varphi)+i\; sin (n \varphi)}$
\end{enumerate}

\subparagraph{Bsp. 19:}
\begin{enumerate}
[label=\alph*.)]
\item $z_1=\underbrace{3+4i}_{\tgreen{arithmetisch}}=\underbrace{5\cdot (cos 53,13^{\circ}+i\; sin (53,13^{\circ})}_{\tgreen{trigonemetrisch}}=\underbrace{5\cdot e^{i\cdot 53,13^{\circ}}}_{\tgreen{exponentiell}}$\\
$z_2=-12-5i=13\cdot (cos(-157,38^{\circ}+i\; sin(-157,38^{\circ}))=13\cdot e^{-i\cdot 157,38^{\circ}}$
\item $z=-1+i$, gesucht: $z^{12}$\\
$|z|=\sqrt{2}$, $Arg(z)=arccos\left(-\frac{1}{\sqrt{2}}\right)=135^{\circ}=\frac{3}{4}\pi$\\
$z=\sqrt{2}\cdot e^{i\cdot \frac{3}{4}\pi}\Rightarrow z^12=\left(\sqrt{2}\cdot e^{i\frac{3}{4}\pi}\right)=2^6\cdot e^{i\frac{3}{4}\pi \cdot 12}=64 \cdot e ^{i\cdot 9 \pi}=64\cdot e^{i\pi}$\\
arithmetische Darstellung: $z^{12}=64\cdot (cos\pi+i\; sin\pi)=\resultul{-64}$
\end{enumerate}

\subsubsection{Spezielle Gleichungen}
Quadratische Gleichung: $z^2+p\cdot z+q=0 \qquad (p,q \in \mathbb{R})$

quadratische Ergänzung: $\left(z+\frac{p}{2}\right)^2=\frac{p^2}{4}-q$

1. Fall: $\frac{p}{2}-q \geq 0 \curvearrowright z_{1/2}=-\frac{p}{2}\pm \sqrt{\frac{p^2}{4}-q}$

2. Fall: $\frac{p^2}{4}-q < 0 \\
\Leftrightarrow \left(z+\frac{p}{2}\right)^2+\overbrace{\underbrace{q-\frac{p^2}{4}}_{>0}}^{\widehat=a^2}=0 \\
\Leftrightarrow \left(\left(z+\frac{p}{2}\right)+i\cdot a\right)\cdot \left(\left(p+\frac{p}{2}\right)-i\cdot a\right)=0\\
\Leftrightarrow \boxed{z_{1,2}=-\frac{p}{2}\pm i \sqrt{q-\frac{p^2}{4}}}$\\
praktisches Vorgehen:

Lösungsformel aus dem ersten Fall stets anwenden, im Fall 2 Formal $\sqrt{-1}=\pm i$.

\subparagraph{Bsp. 20:}\parskp
$z^2+28z+200=0$\\
$z_{1,2}=-14\pm \sqrt{-4}=\resultul{-14\pm 2i}$

\paragraph{Kreisteilungsgleichung:} \parskp
$\boxed{z^n=b}, b\in \mathbb{C}, n\in \mathbb{N}^*$\\
Lösung:
\begin{itemize}
\item $b$ exponentiell darstellen: $b=|b| \cdot e ^{i\beta} \quad , \beta = Arg(b)$
\item Gleichung besitzt $n$ Lösungen $\boxed{z_k=\sqrt[n]{|b|}\cdot e^{i\frac{\beta+k\cdot 360^{\circ}}{n}}}$ \qquad mit $k=0,1,...,n-1$
\end{itemize}
zum Beweis: \\
Ansatz: $z=r\cdot e^{i\varphi}$\\
$z^n=\torange{r^n}\cdot e^{i\tgreen{\varphi n}}=\torange{|b|} \cdot e^{i\tgreen{\beta}}$\\
Zwei Gleichungen stimmen überein, wenn jeweils der \torange{Betrag} gleich und \tgreen{Winkel} bis auf ein vielfaches von $\pi$ gleich ist.
\begin{enumerate}
\item $r^n=|b| \curvearrowright r = \sqrt[n]{|b|}$
\item $\varphi \cdot n = \beta + k \cdot 360^{\circ} \qquad (k \in \mathbb{Z})\\
\curvearrowright \varphi = \frac{\beta + k \cdot 360^{\circ}}{n}$\qquad (nur $n$ verschiedene Argumente)
\end{enumerate}
Beispiele:
\begin{enumerate}[label=\alph*.)]
\item $z^3=1=1\cdot e^{i\cdot 0}$\\
$z_k=1^{\frac{1}{3}}\cdot e^{i\frac{0+k\cdot 2 \pi}{3}} \qquad (k=0,1,2)$\\
$z_0=e^{i\cdot 0}=\resultul{1}\\
z_1= e^{\frac{2}{3}\pi}=cos(\frac{2}{3}\pi)+i\; sin (\frac{2}{3}\pi)=\resultul{-\frac{1}{2}+\frac{1}{2}\sqrt{3}i}\\
z_2=e^{i\frac{4}{3}\pi}=cos(\frac{4}{3}\pi)+i\; sin(\frac{4}{3}\pi)=\resultul{-\frac{1}{2}-\frac{1}{2}\sqrt{3}i}$\\
Allgemein: Lösungen der Gleichung $z^n=b$ teilen Kreis mit Radius $\sqrt[n]{|b|}$ um $0$ in $n$ gleiche Teile.\\
ABB 92
\item $z^4=-16=16\cdot e^{i\cdot \pi}$\\
$z_k=2\cdot e^{i\frac{\pi + k \cdot 2\pi}{4}} \qquad (k=0,1,2,3,4)$\\
$z_0=2\cdot e^{i\frac{\pi}{4}}=\resultul{\sqrt{2}+\sqrt{2}i}\\
z_1=2\cdot e^{i\frac{3\pi}{4}}=\resultul{-\sqrt{2}+\sqrt{2}i}\\
z_2=2\cdot e^{-i\frac{3\pi}{4}}=\resultul{-\sqrt{2}-\sqrt{2}i}\\
z_3=2\cdot e^{-i\frac{\pi}{4}}=\resultul{\sqrt{2}-\sqrt{2}i}$\\
ABB 93
\end{enumerate}
Anwendung: \\
Faktorisierung des Polynoms $p(x)=x^4+16$
\begin{align*}
x^4+16 &= (x-z_0)\cdot(x-z_1)\cdot(x-z_2)\cdot(x-z_3)\\
&= (x-\sqrt{2}-\sqrt{2}i)\cdot (x-\sqrt{2}+\sqrt{2}i)\cdot (x+\sqrt{2}-\sqrt{2}i) \cdot (x+\sqrt{2}+\sqrt{2}i)\\
&= (x^2-2\sqrt{2}x+4)(x^2+2\sqrt{2}x+4)
\end{align*}

\subsubsection{Anwendung im Wechselstromkreis}
\begin{enumerate}
\item Spule: \\
Stromstärke $I=I_m\cdot \tred{(}cos (\omega t) \tred{+ i \; sin(\omega t))}$\\
Spannung $U=\omega \cdot L\cdot I_m \cdot \tred{(}cos (\omega t +\frac{\pi}{2}) \tred{+ i \; sin(\omega t  +\frac{\pi}{2}))}$\\
\tred{(formale Ergänzung zu komplexer Größe)}\\
$\curvearrowright I=I_m \cdot e^{i\omega t}, \; U=I_m\cdot \omega \cdot L \cdot e^{i(\omega t+\frac{\pi}{2}}=\underbrace{I_m\cdot \omega \cdot L \cdot e^{i\omega t}}_{I\cdot \omega \cdot L}\cdot \underbrace{e^{i\frac{\pi}{2}}}_i$\\
$R=\frac{U}{I}\curvearrowright \boxed{R_L=\omega \cdot L \cdot i}$ (\emph{induktiver Widerstand})
\item Kondensator: \\
$\boxed{R_C=\frac{1}{\omega \cdot C \cdot i}}$ (\emph{kapazitiver Widerstand})
\end{enumerate}
Bezeichnung in E-Technik:\\
$Z:= R_ges = R+i\cdot X$\\
Wirkwiderstand $R$, Blindwiderstand $X$, Scheinwiderstand $|Z|$, Leitwert $Y=\frac{1}{Z}$

\subparagraph{Bsp. 22:} \parskp
ABB 94\\
$R=100 \Omega, C = 20 \mu F=20\cdot 10^{-6}\frac{As}{V}, L=1H=1 \frac{Vs}{A}, \omega = 2\pi \cdot \underbrace{50 \frac{1}{s}}_{f}$\\
Gesucht ist der Gesamtwiderstand $Z$.\\
$Z=R + R_C + R_L = R+i \omega L + \frac{1}{i \omega C}=R+i(\omega L - \frac{1}{\omega C})$\\
$Z=(\tblue{100}+\torange{155,04}i)\Omega = \tgreen{184,44} \cdot e^{i\cdot 57,17^{\circ}}\Omega$\\
$\curvearrowright$ \\
Wirkwiderstand: $\tblue{Re(Z)}=100 \Omega$\\
Blindwiderstand: $\torange{Im(Z)}=155,04 \Omega$\\
Scheinwiederstand: $\tgreen{|Z|}=184,44 \Omega$\\
Phasenverschiebung: $\tpurple{Arg(Z)}=57,17^{\circ}$

\section{Reellwertige Funktionen einer reellen Veränderlichen}
%\subsection{Elementare Funktionen (Teil 1)}
\subsubsection{Polynome}
\paragraph{Def. 1:} \parskp
$y=f(x)=a_n\; x^n+a_{n-1}\;x^{n-1}+...+a_2\; x^2+a_1\; x+a_0$ mit $(a_0, ..., a_n \in \mathbb{R}, x\in \mathbb{R})$ heißt ganze rationale Funktion oder \emph{Polynom} vom Grad $n$ (falls $a_n \not = 0$).\\
Zur Beschreibung der Funktionswerte zweckmäßig: HORNER-Schema (vgl. Stellenwertsysteme \ref{sec:Stellenwertsysteme})\\
\begin{tabular}{r | c c c c c c}
 & $a_n$ & $a_{n-1}$ & $ a_{n-2}$ & ... & $a_1$ & $a_0$\\
$x_0$ &  & $b_{n-1}\cdot x_0$ & $b_{n-2}\cdot x_0$ & ... & $b_1 \cdot x_0$ & $b_0\cdot x_0$ \\
\hline
 & $\underbrace{\boxed{b_{n-1}}}_{=a_n}$ & $\boxed{b_{n-2}}$ & $\boxed{b_{n-3}}$ & ... & $\boxed{b_0}$ & $f(x_0)=r_0$\\
\end{tabular}\\
Polynomdivision: $\frac{f(x)}{x-x_0}=b_{n-1}\;x^{n-1}+b_{n-2}\;x^{n-2}+...+b_1\; x + b_0 + \frac{r_0}{x-x_0}$

\subparagraph{Bsp. 1:} \parskp
$f(x)=x^5-2x^3+x^2-6$, \quad $x_0=3$, \quad ges: $f(x_0)$, \quad$\frac{f(x)}{x-x_0}$\\
\begin{tabular}{r| c c c c c c}
 & $1$ & $0$ & $-2$ & $1$ & $0$ & $-6$\\
$x_0=3$ &  & $3$ & $9$ & $21$ & $66$ & $198$ \\
\hline
 & $1$ & $3$ & $7$ & $22$ & $66$ & $192$\\
\end{tabular}\\
$f(x):(x-3)=x^4+3 x^3 + 7x^2+22x + 66 + \frac{192}{x-3}$

\paragraph{Satz 1:} \parskp
Es sei $f(x)=p_n(x) = a_n \; x^n + ... + a_0$ ein Polynom vom Grad $n$ (d.h. $a_n\not = 0$). Dann besitzt $f$ (in $\mathbb{C}$) genau $n$ Nullstellen $x_1,...,x_n$ und es gilt: $f(x) = a_n (x-x_1)\cdot (x-x_2)\cdot ... \cdot (x-x_n)$. (Zerlegung in Linearfaktoren)

\subparagraph{Diskussion:}
\begin{enumerate}
\item Falls in der Linearfaktorzerlegung der Faktor $(x-x_0)$ genau $k$-mal ($1\geq k \geq n$) vorkommt, so heißt $x_0$ \emph{$k$-fache Nullstelle} (Nullstelle der Vielfachheit $k$).
\item Nichtreelle Nullstellen sind möglich, sie treten stets paarweise als konjugiert komplexe Zahlen auf ($x_0, \overline{x_0})$). In diesem Falle Zusammenfassung der Linearfaktoren zu reellen quadratischen Faktoren möglich: $(x-x_0)\cdot (x-\overline{x_0})=x^2-(x_0 + \overline{x_0})x + x_0 \cdot \overline{x_0}=x^2-2\cdot Re(x_0) \cdot x + |x_0|^2$
\item Falls $a_0, a_1, ... , a_n$ ganze Zahlen sind, dann sind ganzzahlige Nullstellen Teiler von $a_0$ (falls vorhanden).
\item Allgemeine Methoden zur Nullstellenbestimmung später (Kap. 3 \ref{Kap.3})
\end{enumerate}

\subparagraph{Bsp. 2:} \parskp
$p(x) = x^4 + x^3 -5x^2 + x-6$, \quad ges: Nullstellen\\
Durch Probieren $\boxed{x_1=2}$\\
mit HORNER-Schema:\\
\begin{tabular}{r | c c c c c l}
 & 1 & 1 & -5 & 1  & -6 & \\
$x_1=2$ & & 2 & 6 & 2 & 6 &\\
\hline
 & 1 & 3& 1 & 3 &$\boxed{0}$ & $\curvearrowright p(x) = (x-2)\cdot \underbrace{(x^3+3x^2+x+3)}_{\text{durch Probieren }x_2=-3}$\\
$x_2=-3$ & & -3 & 0 & -3 & & \\
\hline 
 & 1 & 0 & 1 &\boxed{0}& & $\curvearrowright p(x) = (x-2)(x+3)\underbrace{(x^2+1)}_{x_{3,4}=\pm i}$\\ 
\end{tabular}\\
$\curvearrowright$ Zerlegung: $p(x)= (x-2)(x+3)(x-i)(x+i)$

\subsubsection{Gebrochen rationale Funktionen}

\paragraph{Def. 2:} \parskp
$y=f(x) = \frac{p(x)}{q(x)}=\frac{a_m\; x^m + a_{m-1}\; x^{m-1}+...+a_1 \; x+a_0}{b_n \; x^n+b_{n-1}\; x^{n-1} + ... + b_1 \; x + b_0}$ mit $a_m \not = 0$, $b_n \not = 0$ und $ Db(f)=\{x\in \mathbb{R}| q(x)\not = 0\}$ heißt gebrochenrationale Funktion. f heißt \emph{echt gebrochen}, falls $m<n$ und \emph{unechtgebrochen}, falls $m\geq n$.

\subparagraph{Diskussion:}
\begin{itemize}
\item Wir nehmen ohne Beschränkung der Allgemeinheit an, dass Zähler- und Nennerpolynom keine gemeinsamen Nullstellen besitzen (ansonsten: Kürzen gemeinsamer Linearfaktoren von Zähler und Nenner [unter Beachtung des Definitionsbereichs])
\item Die Nullstellen des Nennerpolynoms heißen \emph{Polstellen} der gebrochen rationalen Funktion (bei Polstelle $x_P$: $|f(x)|\rightarrow \infty$ für $x\rightarrow x_P$).
\item Die Nullstellen des Zählerpolynoms sind die Nullstellen von $f$.
\item Verhalten von $f(x)$ bei $k$-facher reeller Nullstelle oder Polstelle:\\
\fbox{Vorzeichenwechsel $\Leftrightarrow$ $k$ ungerade}
\item Polynomdivision $p(x): q(x) = \underbrace{s(x)}_{Polynom}+\underbrace{\frac{r(x)}{q(x)}}_{echt gebrochen}$\\
$y=a(x)$ ist die sogenannte Asymptote
\end{itemize}
\subparagraph{Bsp. 3:}\parskp
$y=\frac{x^3+2x^2}{x^2-x-2}=\frac{x^2(x+2)}{(x+1)(x-2)}=x+3+\frac{5x+6}{x^2-x-2}$\\
Daraus lassen sich leicht erste Werte der Kurvendiskussion ableiten:
\begin{itemize}
\item Nullstellen: $x_{1,2}=0\; \text{(doppelt)}, x_3=-2$
\item Polstellen: $x=-1,\, x=2$ (einfach $\Rightarrow$ Vorzeichenwechsel)
\item Asymptote: $y=x+3$\\
Schnittstellen und Asymptoten: $5x+6=0 \curvearrowright x=-1,2$
\end{itemize}
\subsubsection{Trigonometrische Funktion}
Übliche Definition der trigonometrische Funktionen (Kreisfunktionen wie $\sin() \cos()$ usw.):
\paragraph{Def. 3:} \parskp
Eine Funktion $y=f(x)$ heißt periodisch, wenn es eine Zahl $p>0$ gibt mit $f(x)=f(+p)$ (für alle $x \in Db(f)$). Die kleinste positive Zahl $p$ mit dieser Eigenschaft heißt Periode $f$.
\paragraph{Def. 4:} (Symmetrieeigenschaft)\\
Eine Funktion $y=f(x)$ heißt:
\begin{enumerate}
\item gerade (symmetrische zur y-Achse), wenn $f(-x)=f(x)$ für alle $x\in Db(f)$ gilt.
\item ungerade (punktsymmetrisch zum Ursprung), wenn $f(-x)=-f(x)$ für alle $x\in Db(f)$ gilt.
\end{enumerate}
\subparagraph{Diskussion:} \parskp
Einige Funktionen
\begin{tabular}{c c c c}
Funktion & $Db$ & Periode & Symmetrie\\
\hline
$\sin x$ & $\mathbb{R}$ & $2\pi$ & ungerade\\
$\cos x$ & $\mathbb{R}$ & $2\pi$ & gerade\\
$\tan x$ & $\mathbb{R}\setminus \left\lbrace\frac{\pi}{2}+k\pi | k\in \mathbb{Z}\right\rbrace$ & $\pi$ & ungerade\\
$\cot x$ & $\mathbb{R}\setminus \left\lbrace k\pi | k\in \mathbb{Z}\right\rbrace$ & $\pi$ & ungerade\\
\end{tabular}\\
Einige wichtige Formeln:
\begin{itemize}
\item $\sin^2 x+\cos^2 x = 1$
\item $\tan x = \frac{\sin x}{\cos x}$
\item $\cot x=\frac{1}{\tan x}$
\item $\sin 2x=2\sin x \cos x$
\item $\cos 2x=2 \cos^2x-1=1-2\sin^2x$
\end{itemize}
\subsubsection{Exponentialfunktion}
$y=f(x)=a\; (a>0, x\in \mathbb{R})$
\begin{itemize}
\item Wichtig: Potenzgesetze, z.B. $a^{x_1}\cdot a^{x_2}=a^{x_1+x_2}$ usw.
\item Besondere Bedeutung besitzt die Funktion $y=e^x$ ($x\in \mathbb{R}$ mit $e=\lim_{x\to \infty} \left( 1+\frac{1}{n}\right)^n=2,7182...$
\end{itemize}
\subsubsection{Hyperbelfunktion}
\paragraph{Def. 5:}Hyperbolicus
\begin{itemize}
\item $y=\cosh x := \frac{1}{2}(e^x+e^{-x}) \quad  (x \in \mathbb{R})$
\item $y=\sinh x := \frac{1}{2}(e^x-e^{-x}) \quad  (x \in \mathbb{R})$
\item $y=\tanh x := \frac{\sinh x}{\cosh x} \quad  (x \in \mathbb{R})$
\item $y=\coth x := \frac{1}{\tanh x} \quad  (x \not = 0)$
\end{itemize}
Wichtig: $\cosh^2 x-\sinh^2 x = 1$\\
Fourier: $y=\cosh x$ ist gerade, alle anderen Hyperbelfunktionen ungerade.
\subsection{Umkehrfunktionen}
\begin{itemize}
\item Zur Erinnerung: $y=f(x), x\in Db(f)$ heißt injektiv (umkehrbar eindeutig), wenn es zu jedem Bild $y \in Wb(f)$ genau ein Urbild $x\in Db(f)$ mit $y=f(x)$ gibt. D.h.:\\
$\underbrace{y}_{\in Wb(f)} \to \underbrace{x}_{\in Db(f)} =: f^{-1}(y)$\\
Die dadurch erklärte Funktion $f^{-1}$ („f oben -1“) ist die Umkehrfunktion von $f$.\\
Es gilt: $\boxed{Db\left(f^{-1}\right) = Wb(f)}, \; Wb\left( f^{-1}\right)= Db(f)$
\item Bilden der Umkehrfunktion zu $y=f(x)$, $x \in Db(f)$:
\begin{enumerate}
\item Auflösen der Funktionsgleichung nach $x$: $x=:f^{-1}(y)$ (falls dies eindeutig möglich ist, andernfalls existiert $f^{-1}$ nicht!)
\item Oft erfolgt anschließend eine Vertauschung von $x$ und $y$:\\
$y=f^{-1}(x)$, $x \in Db(f^{-1})=Wb(f)$.\\
Vertauschung entspricht geometrisch Spiegelung an der Geraden $x=y$, vgl. Bsp. 4.
\end{enumerate}
\end{itemize}
\subparagraph{Bsp. 4:} \parskp
$y=f(x)=\sqrt{x}+2 \qquad , x \in [0,\infty)$
\begin{enumerate}
\item Auflösen nach $x$: $y-2=\sqrt{x}\Rightarrow x=(y-2)^2=f^{-1}(y)$
\item Vertauschen von $x$ und $y$: $y=f^{-1}(x)=(x-2)^2, Db(f^{-1})=Wb(f)=[2,\infty)$\\
Abb 101\\
! $Db(f^{-1})$ nur $[2,\infty)$, obwohl $(x-2)^2$ für alle $x \in \mathbb{R}$ erklärt ist.
\end{enumerate}

\paragraph{Def. 6:} \parskp
Die reellwertige Fkt. $y=f(x)$ heißt
\begin{enumerate}[label=\alph*.)]
\item \emph{streng monoton wachsend}, falls $x_1<x_2 \Rightarrow f(x_1)<f(x_2)$ gilt.
\item \emph{monoton wachsend} (=nicht fallend), falls $x_1<x_2 \Rightarrow f(x_1) \leq f(x_2)$ gilt für alle $x_1, x_2 \in Db(f)$.
\item Analog: \emph{Streng monoton fallend} bzw \emph{monoton fallend} (=nicht wachsend).
\end{enumerate}

\paragraph{Satz 2:} \parskp
$f$ streng monoton $\Rightarrow$ $f$ ist injektiv (d.h. $f^{-1}$ existiert)

\subsection{Elementare Funktionen (Teil 2)}

\subsubsection{Wurzel- und Logarithmusfunktionen}
\paragraph{Def. 7:} \parskp
$y=x^{\frac{1}{n}}=\sqrt[n]{x} \qquad (x\in [0,\infty), n \in \mathbb{N}^*)$ ist die Umkehrfunktion zu $y=x^n \quad (x \in [0, \infty))$

\subparagraph{Diskussion: }
\begin{enumerate}
\item Im Bereich der reellen Zahlen ist $\sqrt[n]{x}$ nur für $x\geq 0 $ erklärt, der Funktionswert ist selber nicht negativ.
\item Lässt man in $x^{\frac{1}{3}}$ negative $x$ zu (etwa $\sqrt[3]{-8}=-2$), so ergeben sich Widersprüche:\\
z.B.: $\sqrt[3]{-8}=-2 \Rightarrow -2=-8^{\frac{1}{3}}=(-8)^{\frac{2}{6}}=\left((-8)^2\right)^{\frac{1}{6}}=64^{\frac{1}{6}}=2$
\item Es gilt $\sqrt{x^2}=|x|$ für alle $x \in \mathbb{R}$
\end{enumerate}

\paragraph{Def. 8:}\parskp
$y=log_a (x) \quad (a>0\wedge a \not = 1, x\in (0,\infty))$ ist Umkehrfunktion zu $y=a^x \quad (x \in \mathbb{R})$.\\
Speziell: \begin{itemize}
\item $lg(x):= log_{10}(x)$
\item $ln(x):= log_e (x)$
\end{itemize}
ABB 102
\subparagraph{Diskussion:}
\begin{enumerate}
\item Log-Gesetze:
\begin{align*}
log_a(x \cdot y) &= log_a x + log_a y\\
log_a\left(\frac{x}{y}\right) &= log_a x - log_a y\\
log_a(x^{\alpha})&=\alpha \; log_a x\\
log_c b&=\frac{log_a b}{log_a c}
\end{align*}
\item Es gilt $x=a^{log_a x}$ \qquad ($f(f^{-1}(x))=x \forall y \in Db(f^{-1})$)
\item Ferner gilt $a^x=e^{ln(a^x)}=e^{x\cdot ln\;a}$
\end{enumerate}
\subsubsection{Arcusfunktionen}
Vorbetrachtung: $y=f(x)=sin\; x (x \in \mathbb{R})$ ist nicht injektiv, also existiert keine Umkehrfunktion.\\
Aber: $y=sin(x)$, eingeschränkt auf z.B. $\left[-\frac{\pi}{2}, \frac{\pi}{2}\right]$ ist injektiv und damit umkehrbar.\\
ABB 103

\paragraph{Def. 9:} \parskp
Umkehrfunktionen\\
\renewcommand{\arraystretch}{1.6}
\begin{tabular}{c | c | c | c l}
 & Db & Wb & Umkehrfunktion von ...\\
\hline
$y=arcsin\; x$ & $[-1,1]$ & $\left[-\frac{\pi}{2}, \frac{\pi}{2}\right]$ & $y=sin\;x$ & $ -\frac{\pi}{2}\leq x \leq \frac{\pi}{2}$\\
$y=arccos\; x $ & $[-1,1]$ & $[0,\pi]$ & $y=cos\; x$ & $0\leq x \leq \pi$\\
$y=arctan\; x$ & $\mathbb{R}$ & $\left(-\frac{\pi}{2}, \frac{\pi}{2}\right)$ & $y=tan \; x $ & $-\frac{\pi}{2}<x<\frac{\pi}{2}$\\
$y=arccot\; x $ & $\mathbb{R}$ & $(0,\pi)$ & $y=cot\; x$ & $0< x <\pi$\\
\end{tabular}
\renewcommand{\arraystretch}{1.3}

\subparagraph{Bsp. 5:}\parskp
Gesucht sind alle Lösungen der folgenden Gleichung: $tan(2x)=y$\\
Es sei $2x \in \left( -\frac{\pi}{2}+k \cdot \pi, \frac{\pi}{2} + k \cdot \pi\right)$, mit $k \in \mathbb{Z}$.\\
$y=tan(2x)=tan(2x-k\cdot \pi)$ mit $2x-k\pi \in \left(-\frac{\pi}{2}, \frac{\pi}{2}\right)$\\
$\Rightarrow 2x - k \pi = arctan(y) \Rightarrow \resultul{x = \frac{arctan(y)+k\pi}{2}}$

\subsubsection{Areafunktionen}

\paragraph{Def. 10:} \parskp
Die Umkehrfunktionen der Hyperbelfunktionen\\
\begin{tabular}{c | c | c | c l}
 & Db & Wb & Umkehrfunktion von ...\\
\hline
$y=arsinh\; x$ & $\mathbb{R}$ & $\mathbb{R}$ & $y=sinh\;x$ & $ x \in \mathbb{R}$\\
$y=arcosh\; x $ & $[1,\infty)$ & $[0,\infty)$ & $y=cosh\; x$ & $x\geq 0$\\
$y=artanh\; x$ & $(-1,1)$ & $\mathbb{R}$ & $y=tanh \; x $ & $x \in \mathbb{R}$\\
$y=arcoth\; x $ & $\mathbb{R}\setminus[-1,1]$ & $\mathbb{R}\setminus\{0\}$ & $y=coth\; x$ & $x \not = 0$\\
\end{tabular}\\
Aus der Def. der Hyberbelfunktionen (Def. 5) folgt:\\
$arsinh\; x = ln (x+\sqrt{x^2+1})$\\
$artanh\; x=\frac{1}{2}ln\left(\frac{1+x}{1-x}\right)$\\
$arcosh\; x = ln (x+\sqrt{x^2+1}$\\
$arcoth\; x = \frac{1}{2}ln\left(\frac{x+1}{x-1}\right)$

\section{Lineare Algebra}
%\subsection{Vektorräume}

\paragraph{Begriff:}
\begin{enumerate}
\item Gegeben seien ein Körper $(K,+,\cdot)$, dessen Elemente \emph{Skalare} heißen (meist $(\mathbb{R}, + ,\cdot)$) und eine ABELsche Gruppe $(V,\oplus)$ ($V$... Menge, Elemente heißen Vektoren, $\oplus$... Vektoraddition).
\item Es gibt eine Abbildung $\odot$ von $K\times V$ in $V$ die jedem $x \in V$ und jedem $\lambda \in K$ ein Element $\lambda \odot x$ in $V$ mit folgenden Eigenschaften zuordnet.
\begin{itemize}
\item Distributivgesetze: \\
$(\lambda + \mu)\odot x = (\lambda \odot x) \oplus (\mu \odot x)$\\
$\lambda\odot(x\oplus y)=(\lambda \odot x) \oplus (\lambda \odot y)$
\item Assoziativgesetz:\\
$(\lambda \cdot \mu ) \odot x = \lambda \odot (\mu \odot x)$
\item Neutrales Element:\\
$1 \odot x = x $
\end{itemize}
(für alle $\lambda, \mu \in K$ und $x,y\in V$)
\end{enumerate}
Eine Menge $V$ mit den in 1.) und 2.) aufgeführten Operationen $\oplus$ und $\odot$ heißt \emph{Vektorraum} (VR) \emph{über $K$}.\\
Bemerkung: Schreibweise meist $+$ anstelle von $\oplus$ und $\cdot$ anstelle von $odot$ (ergibt sich aus Zusammenhang der Elemente).
\subparagraph{Bsp. 1:} \parskp
Skalarbereich $\mathbb{R}$.\\
Vektoren: Größen, die durch eine Zahlenangabe (Länge) und eine Richtung charakterisiert sind (z.B. Kräfte, Geschwindigkeiten, Translatimen).\\
\begin{tikzpicture}[scale=0.3]
\draw[-latex] (0,0) -- (3,3);
\draw (0,0) node{$\bullet$};
\draw (0,0) node[below left] {$P$};
\draw (1,1) node[above left] {$\underline{a}$};
\draw (3,3) node[above right] {$Q$};
\draw (3,3) node{$\bullet$};

\draw[-latex] (8,1) -- (11,4);
\draw (8,1) node{$\bullet$};
\draw (8,1) node[below left] {$R$};
\draw (9,2) node[above left] {$\underline{a}$};
\draw (11,4) node[above right] {$S$};
\draw (11,4) node{$\bullet$};
\end{tikzpicture}\\
Pfeile als Repräsentanten eines Vektors $\underline{a}$.\\
Bezeichnung: $\underline{a}=\overrightarrow{a}=\overrightarrow{PQ}=\overrightarrow{RS}$
\paragraph{Ortskurven:} Angeheftet in gemeinsamen Anfangspunkt $O$ (Ursprung).
\begin{itemize}
\item Vektoraddition: $\underline{a}+\underline{b}$ ABB 110
\item Multiplikation mit Skalar: $\lambda \cdot \underline{a}$:\\
$\lambda > 0$ ABB 111\\
$\lambda < 0$ ABB 112\\
Länge von $\lambda \cdot \underline{a}$ ist das $|\lambda|$-fache der Länge von $\underline{a}$.
\item Subtraktion: $\underline{a}-\underline{b}=\underline{a}+(-\underline{b})=\underline{a}+((-1)\cdot \underline{b})$ ABB 113
\item Nullvektor: $\underline{0}$ (Länge 0, keine Richtung)
\end{itemize}
\subparagraph{Bsp. 2:} \parskp
$K = \mathbb{R}, V = \left\lbrace \begin{pmatrix}
x_1\\ x_2 \\ ... \\ x_n
\end{pmatrix}, x_1, x_2, ..., x_n \in \mathbb{R}\right\rbrace$\\
Vektoraddition: $\begin{pmatrix}
x_1\\ x_2 \\ ... \\ x_n
\end{pmatrix} + \begin{pmatrix}
y_1\\ y_2 \\ ... \\ y_n
\end{pmatrix}=\begin{pmatrix}
x_1+y_1\\ x_2+y_2 \\ ... \\ x_n+y_3
\end{pmatrix}$\\
Multiplikation mit Skalar: $\lambda \cdot \begin{pmatrix}
x_1\\ x_2 \\ ... \\ x_n
\end{pmatrix} = \begin{pmatrix}
\lambda \cdot x_1\\ \lambda \cdot x_2 \\ ... \\  \lambda \cdot x_n
\end{pmatrix}$\medskip\\
$\curvearrowright$ $V$ Vektorraum über $\mathbb{R}$, Bezeichnung: $\mathbb{R}^n$, Nullvektor $\begin{pmatrix}
0\\ 0 \\ ... \\ 0
\end{pmatrix}$
\paragraph{Def. 1:} \parskp
Die Vektoren $\underline{a}_1, ..., \underline{a}_n$ heißen \emph{linear unabhängig}, wenn die Gleichung $\boxed{x_1\underline{a}_1+...+x_n\underline{a}_n=\underline{0}}$ nur die triviale Lösung $x_1=x_2 = .... =x_n = 0$ besitzt.
\subparagraph{Diskussion:}
\begin{enumerate}
\item $x_1\underline{a}_1+...+x_n\underline{a}_n$ heißt \emph{Linearkombination} (LK) der Vektoren $\underline{a}_1, ..., \underline{a}_n$.
\item Falls es eine darstellung der Gestalt wie in Def. 1 gibt, in der nicht alle $x_i$ gleich 0 sind, so heißen $\underline{a}_1, ..., \underline{a}_n$ \emph{linear unabhängig}.\\
In diesem Falle lässt sich (wenigstens) einer der Vektoren als LK der anderen darstellen.
\end{enumerate}

\paragraph{Def. 2:} \parskp
Es sei $V_1 \subseteq V$ eine nichtleere Teilmenge von $V$. Wir bezeichnen mit $L(V_1)$ die Menge \emph{aller} LK von jeweils endlich vielen Vektoren aus $V_1$. $L(V_1)$ ist die sogenante \emph{lineare Hülle} von $V_1$.
\subparagraph{Bemerkung:} \parskp
$L(V_1)$ ist selbst ein Vektorraum, nämlich der von $V_1$ aufgespannte Teilraum von $V$ (kleinster VR, welcher $V_1$ enthält).

\paragraph{Def. 3:}
\begin{itemize}
\item Ein Vektorraum $V$ heißt \emph{n-dimensional}, wenn es $n$ linear unabhängige Vektoren $\underline{a}_1, ..., \underline{a}_n$ gibt, die den gesamten Raum aufspannen ($L(\{\underline{a}_1, ..., \underline{a}_n\})=L(\underline{a}_1, ..., \underline{a}_n)=V$).
\item Die Menge der Vektoren $\underline{a}_1, ..., \underline{a}_n$ nennt man in diesem Falle eine Basis von $V$.
\end{itemize}
\subparagraph{Diskussion:}\parskp
In einem Vektorraum gibt es unterschiedliche Basen, jedoch ist die Anzahl der Vektoren, die eine Basis bilden, stets gleich (Dimension des VR).
\paragraph{Satz 1:}\parskp
Es sei $\underline{a}_1, ..., \underline{a}_n$ eine Basis des VRs $V$. Dann gibt es für \emph{jedes} $\underline{x}\in V$ eine \emph{eindeutige} Darstellung der Gestalt $\underline{x}=x_1\underline{a}_1, ..., x_n\underline{a}_n$.
\subparagraph{Bemerkung:}
\begin{itemize}
\item Die Koeffizienten $x_1,...x_n$ heißen \emph{Koordinaten} von $\underline{x}$ bezüglich der Basis $\underline{a}_1, ..., \underline{a}_n$.
\item Die Summanden $x_1\underline{a}_1, ..., x_n\underline{a}_n$ heißen \emph{Komponenten} von $\underline{x}$ bezüglich der Basis $\underline{a}_1, ..., \underline{a}_n$.
\end{itemize}
\subparagraph{Bsp. 3:} \parskp
Die Vektoren $\underline{e}_1=\begin{pmatrix}
1\\0\\...\\0
\end{pmatrix}, \underline{e}_2=\begin{pmatrix}
0\\1\\...\\0
\end{pmatrix}, ..., \underline{e}_N=\begin{pmatrix}
0\\0\\...\\1
\end{pmatrix}$des Raumes $\mathbb{R}^n$ bilden offensichtlich eine Basis von $\mathbb{R}^n$.\\
$\curvearrowright \underline{e}_1, ..., \underline{e}_n$ sind linear unabhängig. Ferner gilt für beliebiges $\underline{x}=\begin{pmatrix}
x_1 \\ ...\\ x_n
\end{pmatrix}\in \mathbb{R}^n$. $\underline{x}=x_1\cdot \underline{e}_1+...+x_n \cdot \underline{e}_n$.
\subparagraph{Bsp. 4:} \parskp
Zwei Vektoren $\underline{a}_1 \not = \underline{0}$ und $\underline{a}_2 \not = \underline{0}$ in einer Ebene bilden genau dann eine Basis, wenn sie nicht parallel sind.

\subsection{Matrizen}

\paragraph{Def. 4:} \parskp
Ein aus $m\cdot n$ Zahlen $a_{ij}\in \mathbb{R}$, welche in $m$ Zeilen und $n$ Spalten angeordnet sind, bestehendes Schema heißt \emph{Matrix vom Typ $(m,n)$}.\\
$\underline{A}=\begin{pmatrix}
a_{11} & a_{12} & a_{13}& ... &a_{1n}\\
a_{21} & a_{22} & a_{23}& ... &a_{2n}\\
... & ... & ... &... & ...\\
a_{m1} & a_{m2} & a_{m3} &...& a_{mn}\\
\end{pmatrix}=(a_{ij})_{\substack{i=1,...,m \text{ (Zeilenindex)} \\ j=1,...,n \text{ (Spaltenindex)}}}
$

\paragraph{Def. 5} Rechenoperationen
\begin{enumerate}
\item $\underline{A}=(a_{ij}), \underline{B}=(b_{ij})$ seien vom gleichen Typ $(m,n)$.\\
$\boxed{\underline{A}+\underline{B}:= (a_{ij}+b_{ij})}$ \emph{Addition von Matrizen}
\item Sei $\lambda \in \mathbb{R}$ und $\underline{A}=(a_{ij})$ vom Typ $(m,n)$.\\
$\boxed{\lambda\cdot \underline{A}=(\lambda\cdot a_{ij})}$ \emph{Multiplikation einer Matrix mit einem Skalar}
\item $\underline{A}=(a_{ij}$ sei vom Typ $(m,n)$\\
$\underline{B}=(b_{ij})$ sei vom Typ $(n,p)$\\
$\underline{A}$ und $\underline{B}$ heißen in dieser Reihenfolge \emph{verkettet} (Spaltenzahl von $\underline{A}$ = Zeilenzahl von $\underline{B}$).\\
$\boxed{\underline{A}\cdot \underline{B}= \left( \sum_{j=1}^{n} a_{ij}\cdot b_{jk}\right)_{\substack{i=1,...,m \\ k= 1,...,p}}}$ \emph{Matrizenmultiplikation}\\
Das Produkt ist also vom Typ $(m,p)$.
\end{enumerate}
\subparagraph{Diskussion:} \parskp
Zweckmäßig FALK-Schema zur Matrizenmultiplikation (vgl. folgendes Bsp. 5).

\paragraph{Def. 6} \parskp
Die  aus der $(m,n)$-Matrix $\underline{A}$ durch Vertauschung von Zeilen und Spalten entstehende $(n,m)$-Matrix heißt \emph{Transformierte} von $\underline{A}$. Bezeichnung: $\underline{A}^T$.

\subparagraph{Bsp. 5:} \parskp
$\underline{A}=\begin{pmatrix}
5 & -3\\
1 & 4\\
\end{pmatrix}, \underline{B}=\begin{pmatrix}
3 & 6 & 4\\
-2 & 0 & 1\\
\end{pmatrix}, \underline{C}=\begin{pmatrix}
-1 & 5\\
0 & 3\\
\end{pmatrix}$
\begin{enumerate}[label=\alph*.)]
\item $\underline{A}+\underline{B}$ existiert nicht (unterschiedliche Typen).
\item $\underline{A}+\underline{C}=\begin{pmatrix}
4 & 2\\
1 & 7\\
\end{pmatrix}$
\item $2\cdot \underline{A}=\begin{pmatrix}
10 & -6\\
2 & 8\\
\end{pmatrix}$
\item $\underline{B}^T=\begin{pmatrix}
3 & -2\\
6 & 0\\
4 & 1\\
\end{pmatrix}$
\item $\underline{B}\cdot \underline{A}$ existiert nicht ($(2,3)$ und $(2,2)$ nicht verkettet)
\item $\underline{A}\cdot \underline{B}=\begin{pmatrix}
21 & 30 & 17\\
-5 & 6 & 8\\
\end{pmatrix}$\\
(mit FALK-Schema:)
\begin{tabular}{c c | c c c}
 & & 3 & 6& 4\\
 & & -2 & 0 & 1\\
\hline
5 & -3 & 21 & 30 & 17\\
1 & 4 &  -5& 6 & 8\\
\end{tabular}\\
Bemerkung: Die Matrizenmultiplikation ist nicht kommutativ!
\end{enumerate}
\subparagraph{Diskussion:} (ausgewählte Rechenregeln)
\begin{enumerate}
\item Die Menge der Matrizen vom gleichen Typ bilden mit den Operationen Addition und Multiplikation mit einem Skalar einen Vektorraum.\\
Bsp:  $V=\{\text{Matrizen vom Typ }(2,2)\}$\\
Basis: $\begin{pmatrix}
1& 0\\
0 &0 \\
\end{pmatrix},\begin{pmatrix}
0& 1\\
0 &0 \\
\end{pmatrix},\begin{pmatrix}
0& 0\\
1 &0 \\
\end{pmatrix},\begin{pmatrix}
0& 0\\
0 &1 \\
\end{pmatrix}$
\item Falls die entsprechenden Typvoraussetzungen erfüllt sind, gelten: 
\begin{itemize}
\item $(\underline{A}\cdot \underline{B})\cdot \underline{C}=\underline{A}\cdot (\underline{B}\cdot \underline{C})$ (Assoziativgesetz)
\item $(\underline{A}\cdot \underline{B})+ \underline{C}=\underline{A}\cdot \underline{B}+ \underline{A}\cdot \underline{C}$\\
$(\underline{A}+ \underline{B})\cdot \underline{C}=\underline{A}\cdot \underline{C}+ \underline{B}\cdot \underline{C}$ (Distributivgesetze)
\item $(\lambda \cdot \underline{A})\cdot \underline{B}=\lambda \cdot (\underline{A}\cdot \underline{B})=\underline{A}\cdot (\lambda\cdot \underline{B})$
\item $(\lambda \cdot \underline{A})^T=\lambda \cdot \underline{A}^T \qquad \left(\underline{A}^T\right)^T=\underline{A}$
\item $(\underline{A}+\underline{B})^T=\underline{A}^T+\underline{B}^T
 \qquad \boxed{\left(\underline{A}\cdot \underline{B}\right)^T=\underline{B}\cdot \underline{A}^T}$
\end{itemize}
\item Achtung: Im Allgemeinen gilt $\underline{A} \cdot \underline{B} \not = \underline{B} \cdot \underline{A}$!
\item FALK-Schema bei fortgesetzter Multiplikation $\underline{A}\cdot \underline{B}\underline{C}$\\
\begin{tabular}{c | c | c}
 & $\underline{B}$ & $\underline{C}$\\
 \hline
$\underline{A}$ & $\underline{A}\cdot\underline{B}$ & $(\underline{A}\cdot\underline{B})\cdot \underline{C}$
\end{tabular} oder 
\begin{tabular}{c | c}
 & $\underline{C}$\\
\hline
$\underline{B}$ & $\underline{B}\cdot \underline{C}$\\
$\underline{A}$ & $\underline{A}\cdot (\underline{B}\cdot \underline{C})$\\
\end{tabular} (2 Varianten, gemäß Assoziativgesetz)
\end{enumerate}
\paragraph{Spezielle Matrizen}
\begin{enumerate}
\item \emph{Quadratische Matrizen}: Typ $(n,n)$\\
Eine quadratische Matrix $\underline{A}$ heißt
\begin{enumerate}
\item \emph{symmetrisch}, wenn $\underline{A}^T=\underline{A}$ gilt.
\item obere \emph{Dreiecksmatrix}, wenn $a_{ij}=0$ für $i>j$.\\
untere \emph{Dreiecksmatrix}, wenn $a_{ij}=0$ für $i<j$.
\item \emph{Diagonalmatrix}, wenn $a_ij=$ für $i\not = j$.
\item \emph{Einheitsmatrix} $\underline{E}$, wenn $a_{ij}=\begin{cases}
1 \quad \text{für }i=j\\
0 \quad \text{für } i \not = j\\
\end{cases}$ (spezielle Diagonalmatrix). \\
$\underline{E}=\begin{pmatrix}
1 & 0 & ... &0 \\
0 & 1 & ... &0 \\
... &... & ... & ...\\
0 & 0 & ... & 1 \\
\end{pmatrix}$
\end{enumerate}
\item \emph{Nullmatrix} $\underline{0}$ (sämtliche Elemente $0$, nicht notwendig quadratisch).
\item Matrizen vom Typ $(n,1)$ ($n$ Zeilen, eine Spalte) heißen (Spalten-)Vektoren.\\
$\underline{a}=\begin{pmatrix}
a_1\\
a_2\\
...\\
a_n
\end{pmatrix} \in \mathbb{R}^n$ (vgl. \ref{ 5.1})\\
Es ist $\underline{a}^T=(a_1| a _2 | ... | a_n)=\begin{pmatrix}
a_1 & a_2 & ... & a_n\\
\end{pmatrix}$ vom Typ $(1,n)$ (Zeilenvektor).
\end{enumerate}
\subparagraph{Diskussion:}
\begin{enumerate}
\item Die quadratischen Matrizen vom Typ $(n,n)$ bilden mit den Operationen Addition und Multiplikation von Matrizen einen (nicht kommutativen) Ring.
\item Für quadratische Matrizen $\underline{A}$ sind Potenzen bildbar: \\
$\boxed{\underline{A}^0=\underline{E} \quad \underline{A}^n=\underbrace{\underline{A}\cdot\underline{A}\cdot ... \cdot\underline{A}}_{n-\text{Faktoren}}, n \in \mathbb{N}}$
\item Falls die entsprechenden Typvoraussetzungen erfüllt sind, gelten:\\
$\underline{A}\cdot \underline{E}=\underline{A}\\
\underline{E}\cdot \underline{A}=\underline{A}\\
\underline{0}\cdot \underline{A} = \underline{0}\\
\underline{A} \cdot \underline{0} = \underline{0}\\
\underline{A} + \underline{0} = \underline{A}\\
\underline{0} + \underline{A} = \underline{A}$\\
(analog $0$ und $1$ bei den reellen Zahlen)
\item Sei $\underline{A}$ vom Typ $(m,n)$, $x \in \mathbb{R}^n$, d.h. vom Typ $(n,1)$.\\ Dann ist $\boxed{\underline{y}= \underline{A} \cdot \underline{x}}$ vom Typ $(m,1)$.\\
Durch die Zuordnung $\underline{x}\longmapsto \underline{A}\cdot \underline{x} = \underline{y}$ wird eine \emph{lineare Abbildung} von $\mathbb{R}^n$ in $\mathbb{R}^m$ beschrieben (Fkt. $f$ heißt linear, wenn gilt $f(x+y)=f(x)+f(y)$ und $f(\alpha \cdot x)=\alpha \cdot f(x)$ für alle $\alpha \in \mathbb{R}, \; x,y \in Db(f)$ gilt).
\end{enumerate}

\subsection{Determinanten}
\paragraph{Def. 7:} \parskp
Jeder $n$-reihigen quadratischen Matrix ist eindeutig eine Zahl $det\;\underline{A}$, die sogenannte \emph{Determinante} von $\underline{A}$, wie folgt zugeordnet.\\
$n=1$: $det\left((a_{11})\right):=a_{11}$\\
$n\geq 2$: $det\left(\begin{pmatrix}
a_{11} & ... & a_{1n}\\
...	&	& ...\\
a_{n1} &...&a_{nn}
\end{pmatrix}
\right):= a_{11}A_{11}+a_{12}A_{12}+...+a_{1n}A_{1n}$.\\
Dabei ist $A_{ij}=(-1)^{i+j} det\; U_{ij}$ die \emph{Adjunkte} des Elements $a_{ij}$.\\
$U_{ij}$ ist die $(n-1)$-reihige \emph{(Unter-)Matrix}, die durch Streichen der $i$-ten Zeile und der $j$-ten Spalte von $\underline{A}$ ernsteht.\\
Bezeichnung: $det(\underline{A})=det\left(\begin{pmatrix}
... & ...\\
... & ...\\
\end{pmatrix}\right)=\begin{vmatrix}
a_{11} & ... & a_{1n}\\
...	&	& ...\\
a_{n1} &...&a_{nn}
\end{vmatrix}$
\subparagraph{Bsp. 6:}
\begin{enumerate}[label=\alph*.)]
\item $n=2$: \\
$\begin{vmatrix}
a_{11} &  a_{12}\\
a_{21} &a_{22}
\end{vmatrix}\\
=a_{11}A_{11}+a_{12}A_{12}=a_{11}\cdot (-1)^{1+1}\cdot a_{22}+a_{12}\cdot(-1)^{1+2}\cdot a_{21}\\
=\resultul{a_{11}\cdot a_{22}-a_{12}\cdot a_{21}}$
\item $n=3$: \\
$\begin{vmatrix}
a_{11} &  a_{12} & a_{13}\\
a_{21} &a_{22} & a_{23}\\
a_{31} & a_{32} & a_{33}
\end{vmatrix}\\
=a_{11}A_{11}+a_{12}A_{12}+a_{13}A_{13}\\
=a_{11} \begin{vmatrix}
a_{22} & a_{23}\\
a_{32} & a_{33}\\
\end{vmatrix}-a_{12}\begin{vmatrix}
a_{21} & a_{23}\\
a_{31} & a_{33}
\end{vmatrix}+a_{13} \begin{vmatrix}
a_{21} & a_{22}\\
a_{31} & a_{32}
\end{vmatrix}\\
=a_{11}a_{22}a_{33}+a_{12} a_{23}a_{31}+a_{13} a_{21} a_{32}-(a_{13} a_{22} a_{31}+a_{11}a_{23}a_{32}+a_{12}+a_{21}+a_{33})
$\medskip\\
(Alternativ auch: Regel von SARRUS [diese gilt NUR für 3-reihige Determinanten] $\Rightarrow$ (Summe der Produkte der Diagonalen nach rechts unten)-(Summe der Produkte der Diagonalen nach links unten))
\end{enumerate}
\paragraph{Satz 2:}
\begin{enumerate}[label=\alph*.)]
\item $det(\underline{A}\cdot \underline{B}) = det (\underline{A})\cdot det(\underline{B})$
\item $det(\underline{A})=det(\underline{A}^T)$
\end{enumerate}
Wegen Satz 2b gelten für alle folgenden, für die Zeilen formulierten Eigenschaften auch sinngemäß für die Spalten.
\paragraph{Satz 3:} (Eigenschaften der Determinante)
\begin{enumerate}[label=(E\arabic*)]
\item $\underline{B}$ gehe aus $\underline{A}$ durch Vertauschen zweier Zeilen hervor, dann gilt $det(\underline{B})=-det(\underline{A})$.
\item Es gilt $det(\underline{A})=0$ falls zwei Zeilen elementweise proportional sind bzw. falls alle Elemente einer Zeile gleich $0$ sind. 
\item Es gilt $\begin{vmatrix}
a_{11} & ... & a_{1n}\\
... &  & ...\\
\lambda a_{i1} & ... & \lambda a_{in}\\
... & & ...\\
a_{n1} & ... & a_{nn}
\end{vmatrix}=\lambda\begin{vmatrix}
a_{11} & ... & a_{1n}\\
... &  & ...\\
a_{i1} & ... & a_{in}\\
... & & ...\\
a_{n1} & ... & a_{nn}
\end{vmatrix}$ (steht ein Faktor in einer Zeile einer Determinante, so kann er auch vorgezogen werden).
\item Der Wert einer Determinante ändert sich nicht, wenn das $\lambda$-fache einer Zeile elementweise zu einer anderen Zeile addiert wird.
\item $det(\underline{A})=\sum_{j=1}^{n}a_{ij}A_{ij}$ (Entwicklung nach $i$-ter Zeile, $(i=1,...,n)$)\\
$det(\underline{A})=\sum_{i=1}^{n}a_{ij}A_{ij}$ (Entwicklung nach $j$-ten Spalte, $(j=1,...,n)$)\\
$\rightarrow$ \emph{Entwicklungssatz}
\end{enumerate}
\paragraph{Bsp. 7:} \parskp
$\begin{vmatrix}
1 & 1 & 3 & -1\\
-2 & 2& -5& -4\\
-1 & 1 & -4 & 2\\
6 & 2 &-1 & 0
\end{vmatrix}$\\
3. Spalte = Arbeitsspalte (bleibt unverändert)\\
Um in der untersten Spalte mehr Nullen zu erzeugen (mit Regel E4):\\
$S_{1,neu}:=S_1+6\cdot S_3\\
S_{2,neu}:=S_2+2\cdot S_3$\\
$\Rightarrow$\\
$=\begin{vmatrix}
19 & 7 & 3 & -1\\
-32 & -8& -5& -4\\
-26 & -7 & -4 & 2\\
0 & 0 &-1 & 0
\end{vmatrix}$\\
Nun kann mit der letzen Zeile relativ einfach die Determinante berchnet werden:\\
$=(-1)\cdot (-1)^{4+3}\cdot \begin{vmatrix}
19 & 7 &-1\\
-32 & -8 & -4\\
-26 & -7 & 2\\
\end{vmatrix}$\\
Auf gleiche Weise werden nun wieder in Zeilen Nullen erzeugt:\\
$Z_{2,neu}:=Z_2-4Z_1\\
Z_{3,neu}:=Z_3+2Z_1$\\
$\Rightarrow$\\
$=\begin{vmatrix}
19 & 7 & 1\\
-108 & -36 & 0\\
12 & 7 &0
\end{vmatrix}\\
=(-1)\cdot (-1)^{1+3}\begin{vmatrix}
-108 & -36\\
12 & 7
\end{vmatrix}\\
\overset{E3}{=}(-1)\cdot (-36) \cdot \begin{vmatrix}
3 & 1 \\
12 & 7
\end{vmatrix}=36 \cdot 9 = \resultul{324}$\\
Prinzip: Nullen erzeugen mit (E4), dann mit Entwicklungssatz lösen (E5).
\paragraph{Anwendungen}
\begin{enumerate}
\item Vekotorrechnung in $\mathbb{R}^3$ (vgl. später, Abschnitt 5.5 \ref{5.5})
\item Gegeben sei ein lineares Gleichungssytem ($n$ Gleichungen, $n$ Unbekannte)\\
Matrixform $\boxed{\underline{A}\cdot \underline{x}=\underline{b}}$ mit $\underline{A}=(a_{ij}), \underline{x}=\begin{pmatrix}
x_1\\
...\\
x_n
\end{pmatrix}, \underline{b}=\begin{pmatrix}
b_1\\
...\\
b_n
\end{pmatrix}$. Diese Matrixform besitzt genau dann eine eindeutige Lösung $\underline{x}$, wenn $det(\underline{A})\not = 0$. \\
In diesem Falle gilt $\boxed{x_j=\frac{det(\underline{B}_j)}{det(\underline{A})}}\; (j=1,...,n)$. Wobei $\underline{B}_j$ aus $\underline{A}$ hervorgeht, indem man die $j$-te Spalte durch $\underline{b}$ ersetzt (\emph{CRAMERsche Regel}, theoretische Bedeutung, praktisches Vorgehen zur Lösung der Matrixform vgl. folgenden Abschnitt).
\end{enumerate}

\subsection{Lineare Gleichungssysteme, Rang einer Matrix, Inverse}
\subsubsection{Das Austauschverfahren}
Gegeben sei System von $m$ linearen Funktionen mit den unabhängigen Veränderlichen $x_1,...,x_n$ und den abhängigen Veränderlichen $y_1,...,y_n$.
\begin{align*}
y_1&= a_{11}x_{1}+a_{12}x_{2}+...+a_{1n}x_{n}+a_{10}\\
y_2&= a_{21}x_{1}+a{22}x_{2}+...+a{2n}x_{n}+a_{20}\\
...\\
y_m&=a_{m1}x_{1}+a{m2}x_{2}+...+a{mn}x_{n}+a_{m0}
\end{align*}

\subparagraph{Bsp. 8:}\parskp
Betrieb, in Abteilungen, $n$ Produkte $P_1,...,P_n$:\\
$a_{ij}$... Kosten pro Einheit von $P_j$ die in Abteilung $i$ entstehen.\\
$a_{i0}$... Fixkosten in Abteilung $i$.\\
$x_{j}$... produzierte Mengen von $P_j$.\\
$y_{i}$... Gesamtkosten in Abteilung $i$.\medskip\\
Matrix-Schreibweise: $\underline{y}=\underline{A}\,\underline{x}+\underline{a}$ mit $\underline{A}=(a_{ij})_{\substack{i=1,...,m\\j=1,...,n}},\; \underline{a}=\begin{pmatrix}
a_{01}\\
...\\
a_{0m}
\end{pmatrix}$\\
Tabellenform:\\
\begin{tabular}{c | c c c c c}
& $x_1$ & $x_2$ & ... & $x_n$ & 1\\
\hline
$y_1$ & $a_{11}$ & $a_{12}$ & ... & $a_{1n}$ & $a_{10}$\\
$y_2$ & $a_{21}$ & $a_{22}$ & ... & $a_{2n}$ & $a_{20}$\\
... & ... & & & & \\
$y_m$ & $a_{m1}$ & $a_{m2}$ & ... & $a_{mn}$ & $a_{m0}$\\
\end{tabular} bzw. 
\begin{tabular}{c | c c}
& $x^T$ & $1$\\
\hline
$\underline{y}$ & $\underline{A}$ & $\underline{a}$\\
\end{tabular}\\
Aufgaben:
\begin{enumerate}
\item $\underline{x}$ vorgegeben, $\underline{y}$ ist zu berechnen (klar!).
\item $\underline{y}$ vorgegeben, $\underline{x}$ zu berechnen (nicht immer lösbar, falls lösbar, nicht immer eindeutig lösbar).
\end{enumerate}
Lösungsprinzip:\\
Man tausche so oft wie möglich $y_r$ gegen $x_s$ aus, Austauschschritt AS ($y_r\leftrightarrow x_s$) $\rightarrow$ \emph{Austauschverfahren}.\\
Austauschschritt $y_r\leftrightarrow x_s$ bedeutet:
\begin{enumerate}
\item $r$-te Zeile $y_r=...$ nach $x_s$ auflösen $x_s=...$.
\item in allen anderen Zeilen $x_s$ durch die rechte Seite vom obigen $x_s$ ersetzen.\\
$\curvearrowright$ neue Tabelle
\end{enumerate}
FOLIEN IM NETZ (Neumann)\\
Praktisches Vorgehen:
\begin{enumerate}
\item Pivotelement (Pivot) kennzeichnen $\circ$
\item Austauschregeln Austauschregel (AR) 1 bis AR 4 abarbeiten\\
Dabei für AR 4 unter der alten Tabelle die neue Pivotzeile (PZ) als Kellerzeile notieren.
\end{enumerate}
ABB51\\
$a_{ij}^{*}=a_{ij}+a_{is}\cdot a_{rj}^{*}$ (Rechteckregel)
\subparagraph{Bsp. 8} (Fortsetzung)\\
$y_1=2x_1+3x_2+x_3+50$ (Kosten in Abt. 1)\\
$y_2=x_1+2x_3+40$ (Kosten in Abt. 2)\\
\begin{tabular}{c | c c c c}
$T_1$ & $x_1$ & $x_2$ & $x_3$ & $1$\\
\hline
$y_1$ & 2 & 3 & \fbox{1} & 50\\
$y_2$ & 1 & 0 & 2 & 40\\
\hline
K & -2 & -3 & * & -50
\end{tabular} \quad
\begin{tabular}{c | c c c c}
$T_2$ & $x_1$ & $x_2$ & $y_1$ & $1$\\
\hline
$x_3$ & -2 & -3 & 1 & -50\\
$y_2$ & \fbox{-3} & -6 & 2 & -60\\
\hline
K & * & -2 & $\tfrac{2}{3}$ & -20
\end{tabular}\quad
\begin{tabular}{c | c c c c}
$T_3$ & $y_2$ & $x_2$ & $y_1$ & $1$\\
\hline
$x_3$ & $\tfrac{2}{3}$ & 1 & $-\tfrac{1}{3}$ & -10\\
$x_1$ & $-\tfrac{1}{3}$ & -2 & $\tfrac{2}{3}$ & -20\\
\hline
\\
\end{tabular}\\
d.h.:\\
$x_3=\frac{2}{3}y_2 + x_2 - \frac{1}{3}y_1 - 10$\\
$x_1=-\frac{1}{3} y_2 - 2x_2 + \frac{2}{3}y_1-20$\\
$\curvearrowright$ bei vorgegebenen Kosten $y_1, y_2$ ist die Lösung $\underline{x}=\begin{pmatrix}
x_1\\x_2\\x_3
\end{pmatrix}$ nicht eindeutig bestimmbar.\\
 z.B. $y_1=600,\; y_2=300$:\\
$x_2=t$ (\emph{frei wählbar})\\
$x_3=\frac{2}{3}\cdot 300 + t - \frac{1}{3}\cdot 600 -10 = t-10\\
x_1=-\frac{1}{3}\cdot 300 - 2t + \frac{2}{3}\cdot 600 -20=280-2t\\
\Rightarrow \underline{x}=\begin{pmatrix}
280-2t\\
t\\
t-10
\end{pmatrix}$\\
hier für $x_i\geq 0$: $10\leq t \leq 140$.
\bigskip\\
Varianten des Austauschverfahrens (AV)
\begin{enumerate}
\item AVZ … Austauschverfahren mit \emph{Zeilentilgung}, d.h. neue PZ in neuer Tabelle weglassen.
\item AVS … Austauschverfahren mit \emph{Spaltentilgung}, d.h. neue Pivotspalte in neuer Tabelle weglassen (nur anwendbar, wenn Variable über der weggelassenen Spalte = Null ist, siehe folgender Abschnitt).
\item AVSZ … AVZ+AVS gleichzeitig.
\end{enumerate}

\subsubsection{Lineare Gleichungssysteme}
\begin{itemize}
\item Gegeben sei das lineare Gleichungssystem ($m$ Gleichungen, $n$ Unbekannte $x_1,...,x_n$)\\
$a_{11}x_{1}+a_{12}x_2+...+a_{1n}x_n=b_1\\
...\\
a_{m1}x_1+a_{m2}x_2+...+a_{mn}x_n=b_m$
\item Gleichungssystem heißt \emph{homogen}, falls $b_1=...=b_m=0$ gilt, sonst \emph{unhomogen}.
\item Matrixform $\underline{A}\,\underline{x}=\underline{b}$ mit $\underline{A}=(a_{ij})_{\substack{i=1,...,m\\ j=1,...,n}},\;\underline{x}=\begin{pmatrix}
x_1\\
...\\
x_n
\end{pmatrix},\; \underline{b}=\begin{pmatrix}
b_1\\
...\\
b_m
\end{pmatrix}$
\item Äquivalente Form: $\underline{y}=\underline{A}\,\underline{x}-\underline{b}\cdot 1$ mit $\underline{y}=\begin{pmatrix}
y_1\\
...\\
y_m
\end{pmatrix} = \underline{0}=\begin{pmatrix}
0\\
...\\
0
\end{pmatrix}$\\
Hilfsgrößen $y_1=y_2=...=y_m=0$
\item Tabellenform: \begin{tabular}{c | c c}
& $\underline{x}^T$ & 1\\
\hline
$\underline{y}$ & $\underline{A}$ & $\underline{b}$
\end{tabular}
\end{itemize}
Lösungsprinzip: \\
Austauschverfahren, Variante AVS (da $y_i=0$: Pivotspalte in neuer Tabelle weglassen!)

Fall 1:\\
Alle $y_i$ sind austauschbar $\Rightarrow$ Gleichungssystem ist lösbar, Lösung aus letzter Tabelle (TE) ablesbar.\\
z.B.: \begin{tabular}{ c | c c}
TE & $x_3$ & 1 \\
\hline
$x_1$ & 0 & 4\\
$x_2$ & 2 & -3
\end{tabular} \qquad $\curvearrowright x_1=4, \quad x_2=2x_3-3$ ($x_3$ frei wählbar)

Fall 2:\\
Wenigstens ein $y_i$ ist gegen kein $x_j$ austauschbar.\\
$\curvearrowright$ Tabelle: \begin{tabular}{c| c c}
& (evtl.) noch nicht ausgetauschte $x_j$ & 1\\
\hline
...&&\\
$y_i$&0…0…0&$\alpha$\\ 
...&&\\
\end{tabular}
$\curvearrowright y_i=\alpha$

Fall 2a:
$\alpha=0$\\
Zeile $y_i$ kann gestrichen werden ($0=0$).

Fall 2b:
$\alpha \not = 0 $\\
Gleichungssystem nicht lösbar (Widerspruch, da $y_i=0$)\medskip\\
Das Verfahren endet also im Fall 2b (unlösbar) oder mit einer Tabelle, in der kein $y_i$ mehr vorkommt (Fall 1 oder 2a).\\
\begin{tabular}{c | c c c c c}
TE & $x_{S1}$ & $x_{S2}$ & ... & $x_{Sq}$ & 1\\
\hline
$x_{r1}$ &...\\
$x_{r2}$ & ...\\
...\\
$x_{rp}$& ...
\end{tabular}\\
$x_{S...}$: NBV … \emph{Nichtbasisvariablen} (nicht ausgetauschte $x_i$)\\
$x_{r...}$: BV … \emph{Basisvariablen} (ausgetauschte $x_i$)
\begin{itemize}
\item Allgemeine Lösung ergibt sich aus Endtabelle: NBV beliebig vorgeben, BV daraus berechenbar.
\item Falls keine NBV vorhanden sind, ist die Lösung eindeutig.
\end{itemize}
\paragraph{Def. 8:}\parskp
Die Darstellung der Endtabelle heißt Basisdarstellung des lin. Gleichungssystems.\\
Bemerkung: Aus einer Basisdarstellung lassen sich weitere Basisdarstellungen durch Austausch $x_{ri}\leftrightarrow x_ {sj}$ gewinnen.
\subparagraph{Bsp. 9} \parskp
$3x_1+x_2 + 2x_3 = -2\\
-5x_1 - 3x_2 -2x_3=-2\\
x_1+3x_2 - 2x_3 =10$\\
\begin{tabular}{r | c c c c}
T1 & $x_1$ & $x_2$& $x_3$& $1$\\
\hline
$y_1$ & 3 & \fbox{1} & 2 & 2\\
$y_2$ & -5 & -3 & -2 & 2 \\
$y_3$ & 1 & 3 & -2 &10\\
\hline
K & -3 & *  & -2 & -2
\end{tabular} mit AVS:
\begin{tabular}{r | c c c c}
T2 & $x_1$ & $x_3$& $1$\\
\hline
$x_2$ & -3 & -2 & -2\\
$0$ & \fbox{4} & 4 & 8 \\
$0$ & -8 & -8 & -16 \\
\hline
K & * & -1  & -2 
\end{tabular} \quad 
\begin{tabular}{r | c c c c}
T3 & $x_3$& $1$\\
\hline
$x_2$ & 1 & 4\\
$x_1$ & 1 & -2 \\
$0$ & 0 & 0 \\
\hline
\\
\end{tabular} \\
(in T3 kann letzte 0-Zeile gestrichen werden)\\
$\curvearrowright$ T3 ist Endtabelle (BV: $x_1,x_2$, NBV: $x_3$)\\
allg. Lösung:\\
$x_2=x_3+x_4\\
x_1=-x_3-2$\\
$x_3 \in \mathbb{R}$ frei wählbar\\
andere Form: $x_3=t$ (Parameter), $\underline{x}=\begin{pmatrix}
x_1\\
x_2\\
x_3
\end{pmatrix}=\begin{pmatrix}
-t-2\\
t+4\\
t
\end{pmatrix}, t\in \mathbb{R}$
Bemerkung:
\begin{enumerate}
\item Bei homogenen System $\underline{A}\,\underline{x}=\underline{0}$ muss die 1-Spalte $\begin{pmatrix}
0\\
0\\
...\\
0
\end{pmatrix}$ nicht geschrieben werden (nur „gedacht“).
\item Die Methode AVS entspricht dem sogenannten \emph{Gauß-Jordan-Verfahren}.\\
Der \emph{Gauß-Algorithmus} (siehe folgendes Beispiel):
\begin{itemize}
\item AVSZ (Spalten- und Zeilentilgung)
\item weggelassene Zeilen merken ($\rightarrow$ Kellerzeilen)
\item Rückrechnung durchführen
\end{itemize}
\end{enumerate}
\subparagraph{Bsp. 10:}\parskp
$-x_1+2x_2+2x_3=4\\
2x_1+5x_2+2x_3=4\\
2x_1+x_2-4x_3=-3$\\
\begin{tabular}{r | c c c c}
$T_1$ & $x_1$ & $x_2$ & $x_2$ & $1$\\
\hline
0 & -1 & 2 & 2 & -4\\
0 & 2 & 5 & 2 &-4\\
0 & 2 &\fbox{1} & -1 & +3\\
\hline
$x_2$ & -2 & * & 4 & -3
\end{tabular}\quad
\begin{tabular}{r | c c c}
$T_2$ & $x_1$ & $x_3$ & 1\\
\hline
0 & \fbox{5} & 10 & -10\\
0 & -8 & 22 & -19\\
\hline
$x_1$& * & 2 & -2
\end{tabular}\quad
\begin{tabular}{r| c c}
$T_3$ & $x_3$ & 1\\
0 & \fbox{6} & -3\\
$x_3$ & * & $\tfrac{1}{2}$
\end{tabular}\\
Rückrechnung:\\
$T_3 \curvearrowright x_3=\resultul{\frac{1}{2}}\\
T_2 \curvearrowright x_1=2x_3-2=\resultul{-1}\\
T_1 \curvearrowright x_2=-2x_1 + 4x_3 -3 = \resultul{1}$\\
Lösung: $\underline{x}=\begin{pmatrix}
x_1\\
x_2\\
x_3
\end{pmatrix}= \begin{pmatrix}
-1\\
1\\
\frac{1}{2}
\end{pmatrix}$
Bemerkung:\\
$m$ Gleichungen, $n$ Unbekannte\\
$m\leq n \quad \curvearrowright$ AVS günstiger\\
$m\geq n \quad \curvearrowright$ Gauß oder AVS

\subsubsection{Weitere Anwendungen des Austauschverfahrens}
\begin{enumerate}
\item Lineare Unabhängigkeit von Vektoren $\underline{a}_1,..., \underline{a}_n \in \mathbb{R}^m$ überprüfen.\\
Ansatz: $\boxed{x_1\underline{a}_1+x_2 \underline{a}_2+...+x_n\underline{a}_n=\underline{0}}\Leftrightarrow \boxed{\underline{A}\,\underline{x}=\underline{0}}$ mit $\underline{A}=\begin{pmatrix}
\underline{a}_1\\
\underline{a}_2\\
...\\
\underline{a}_n
\end{pmatrix}$ (Spalten von $\underline{A}$ sind die (Spalten-)Vektoren $\underline{a}_1,...,\underline{a}_n$). Homogenes GLS mit AVS mit Starttabelle: \begin{tabular}{c | c}
& $\underline{x}^T$\\
\hline
$\underline{y}$& A
\end{tabular}
\begin{itemize}
\item Unabhängigkeit genau dann, wenn alle $x_i$ ausgetauscht werden können.
\item Allgemein: Die zu den ausgetauschten $x_i$, d.h. BV, gehörenden $a_i$ sind unabhängig. Sie bilden die Basis von $L(\underline{a}_1,...,\underline{a}_n$.
\end{itemize}
\item Rang einer Matrix $\underline{A}=\begin{pmatrix}
\underline{a}_1\\
...\\
\underline{a}_n
\end{pmatrix}$... $rang(\underline{A})$ (auch: $rank(\underline{A}), rk(\underline{A}),...$)\\
Def.: $\boxed{rang(\underline{A}):=dim\; L(\underline{a}_1,...,\underline{a}_n)}$ \\
(Dimension des von den Spaltenvektoren aufgespannten Teilraumes).\\
Berechnung: $rang(\underline{A})=$Anzahl der ausführbaren Austauschschritte im AVSZ mit \begin{tabular}{r | c}
& $x^T$\\
\hline
$\underline{y}$& A
\end{tabular} als Starttabelle (1-Spalte entfällt).\\
Bemerkung: Es gilt $rang(\underline{A}^T)=rang(\underline{A})$.
\item Berechnung der Determinante einer $(n,n)$-Matrix (vgl. Merkblatt „Lineare Algebra“)
\end{enumerate}
\subsubsection{Die Inverse einer (n,n)-Matrix}
\paragraph{Def. 9:} \parskp
Es sei $\underline{A}$ vom Typ $(n,n)$. Das Gleichungssystem $\underline{y}=\underline{A}\,\underline{x}$ sei für jedes $\underline{y}$ \emph{eindeutig} nach $\underline{x}$ auflösbar, d.h. $\underline{x}=\underline{B}\,\underline{y}$. Dann heißt die $(n,n)$-Matrix $\underline{B}$ \emph{Inverse} von $\underline{A}$. Bezeichnung: $\underline{A}^{-1}=\underline{B}$.\\
Falls $\underline{A}^{-1}$ existiert, so heißt $\underline{A}$ \emph{regulär}, sonst \emph{singulär}.\\
Bemerkung:
\begin{enumerate}
\item \fbox{$\underline{A}$ ist regulär} $\Leftrightarrow$ \fbox{$det\,\underline{A}\not = 0$}
\item $\underline{A}$ regulär, dann hat $\underline{A}\, \underline{x}=\underline{b}$ die eindeutige Lösung $\boxed{\underline{x}=\underline{A}^{-1}\underline{b}}$.
\end{enumerate}
Rechenregeln: Seien $\underline{A}$ und $\underline{B}$ regulär. Dann gilt:
\begin{itemize}
\item $\underline{A}\cdot \underline{A}^{-1}=\underline{E}$, $\underline{A}^{-1}\cdot \underline{A}= \underline{E}$
\item $\left(\underline{A}^{-1}\right)^{-1}=\underline{A}$
\item $\underline{A}\,\underline{B}=\underline{E}$
\item $(\underline{A}\,\underline{B})^{-1}=\underline{B}^{-1}\underline{A}^{-1}$
\item $\left(\underline{A}^T\right)^{-1}=\left(\underline{A}^{-1}\right)^T$
\end{itemize}
Verfahren zur Ermittlung der Inversen:
\begin{itemize}
\item vollständiges AV mit Starttabelle \begin{tabular}{r | c}
& $\underline{x}^T$\\
\hline
$y$ & $\underline{A}$
\end{tabular}\\
Fall 1: alle $x_i$ austauschbar $\curvearrowright$ $\underline{A}$ regulär.\\
Fall 2: nicht alle $x_i$ austauschbar $\curvearrowright$ $\underline{A}$ singulär. \\
im Fall 1: $\curvearrowright$ nach Ordnen der Zeilen und Spalten: $\underline{A}^{-1}$ aus TE ablesbar.
\item Probe: $\underline{A}\cdot \underline{A}^{-1}=\underline{E}$
\end{itemize}
\subparagraph{Bsp. 11:} \parskp
$\underline{A}=\begin{pmatrix}
1 & 2 & 1\\
1 & 0 & 2\\
1 & -1 & 1
\end{pmatrix}$ gesucht: $\underline{A}^{-1}$ (falls diese existiert).\\
Lösung:\\
\begin{tabular}{r | c c c}
$T_1$ & $x_1$ & $x_2$ & $x_3$\\
\hline
$y_1$& \fbox{1}& 2 & 1 \\
$y_2$ & 1 & 0 &2\\
$y_3$ & 1 & -1 & 1\\
\hline
K & * & -2 & -1
\end{tabular}\quad
\begin{tabular}{r | c c c}
$T_2$ & $y_1$& $x_2$ & $x_3$\\
\hline
$x_1$ & 1 & -2 & -1 \\
$y_2$ & 1 & -2 & \fbox{1}\\
$y_3$ & 1 & -3 & 0\\
\hline
K & -1 & 2 & *
\end{tabular}\quad
\begin{tabular}{r | c c c}
$T_3$ & $y_1$ & $x_2 $ & $y_2$\\
$x_1$ & 2 & -4 & -1\\
$x_3$ & -1 & -2 & 1 \\
$y_2$ & 1 & \fbox{-3} & 0\\
\hline
K & $\tfrac{1}{3}$ & * & 0\\
\end{tabular} \quad
\begin{tabular}{r | c c c}
$T_4$ & $y_1$ & $y_3$ & $y_2$\\
\hline
$x_1$ & $\tfrac{2}{3}$ & $\tfrac{4}{3}$ & -1\\
$x_3$ & $-\tfrac{1}{3}$ & $-\tfrac{2}{3}$ & 1\\
$x_2$ & $\tfrac{1}{3}$ & $-\tfrac{1}{3}$ & 0
\end{tabular}\\
$\curvearrowright \underline{A}^{-1}=\begin{pmatrix}
\frac{2}{3} & -1 & \frac{4}{3}\\
\frac{1}{3} & 0 & -\frac{1}{3}\\
-\frac{1}{3} & 1 & -\frac{2}{3}
\end{pmatrix}$\\
Probe: $\underline{A} \, \underline{A}^{-1}=\underline{E}=\underline{A}^{-1}\, \underline{A}$

\subsection{Vektorrechnung im Raum}
\subsubsection{Kartesische Basis}
Einige Begriffe:
\begin{enumerate}
\item \emph{Betrag} eines Vektors $\underline{a}$: Länge des Pfeils, der $\underline{a}$ repräsentiert.\\
Bezeichnung: $|\underline{a}|$
\item \emph{Einheitsvektor}: Vektor mit $|\underline{a}|=1$.
\item zu $|\underline{a}|\not =\underline{0}$ gehörender Einheitsvektor $\boxed{\underline{a}^0=\frac{1}{|\underline{a}|}\underline{a}}$
\item \emph{Kartesische Basis} $\{\underline{i}, \underline{j}, \underline{k}\}$ $\underline{i}, \underline{j}, \underline{k}$ besitzen Betrag 1, stehen $\bot$ aufeinander und bilden in dieser Reihenfolge ein Rechtssystem (Rechtsschraubregel: Rechtsschraube $\bot$ zu $\underline{i}$ und $\underline{j}$ halten, auf kürzestem Weg von $\underline{i}$ nach $\underline{j}$ drehen. $\curvearrowright$ Bewegung in Richtung $\underline{k}$).\\
ABB 52
\item \emph{Kartesisches Koordinatensystem}: 
\begin{itemize}
\item Fester Punkt $O$ als Ursprung
\item kartesische Basis $\{\underline{i}, \underline{j}, \underline{k}\}$ (jeweils linear unabhängig)
\end{itemize}
\end{enumerate}
Damit eineindeutige Zuordnung:\\
$\underset{\text{Punkt}}{P} \overset{1}{\underset{1}{\longleftrightarrow}}\underset{\text{Ortsvektor}}{\overrightarrow{OP}}=\underline{r}=x\cdot \underline{i}+y\cdot \underline{j}+z\cdot \underline{k}$\\
ABB 53\\
$\underline{r}=x\cdot \underline{i}+y\cdot \underline{j}+z\cdot \underline{k}=\begin{pmatrix}
x\\
y\\
z
\end{pmatrix}$ (Kurzschreibweise -- beide Schreibweisen gleichberechtigt)\\
Betrag eines Vektors $\underline{a}=\begin{pmatrix}
a_1\\
a_2\\
a_3
\end{pmatrix}$: 
$\boxed{|\underline{a}|=\sqrt{a_1^2+a_2^2+a_3^2}}$\\
Bemerkung: \\
Bezeichnung auch $\underline{e_1}=\underline{i}=\begin{pmatrix}
1\\
0\\
0
\end{pmatrix}, \underline{e_2}=\underline{j}=\begin{pmatrix}
0\\
1\\
0
\end{pmatrix}, \underline{e_3}=\underline{k}=\begin{pmatrix}
0\\
0\\
1
\end{pmatrix}$\\
$\underline{r}=\begin{pmatrix}
x\\
y\\
z
\end{pmatrix}=\begin{pmatrix}
x_1\\
x_2\\
x_3
\end{pmatrix}=\underline{x}=\overrightarrow{x}=\textbf{x}$

\subsubsection{Das Skalarprodukt}
\paragraph{Def. 10:} \parskp
Die Zahl $(\underline{a}, \underline{b}):=|\underline{a}|\cdot |\underline{b}|\cdot cos (\varphi)$ heißt \emph{Skalarprodukt} der Vektoren $\underline{a}$ und $\underline{b}$. Dabei ist $\varphi$ der Winkel zwischen den Vektoren $\underline{a}$ und $\underline{b}$.\medskip\\
Eigenschaften des Sklarproduktes:
\begin{enumerate}[label=\alph*.)]
\item $(\underline{a}, \underline{a})>0$ für $\underline{a}\not = \underline{0}$
\item $(\underline{a}, \underline{b})=(\underline{b}, \underline{a})$ (Symmetrie)
\item $(\lambda\, \underline{a}+\mu\, \underline{b},\; \underline{c}=\lambda\cdot (\underline{a}, \underline{c})+\mu(\underline{b}, \underline{c})$ (Linearität)
\end{enumerate}

\paragraph{Satz 4:} \parskp
Es sei $\underline{a}=\begin{pmatrix}
a_1\\
a_2\\
a_3
\end{pmatrix}, \underline{b}=\begin{pmatrix}
b_1\\
b_2\\
b_3
\end{pmatrix}$. Dann gilt $\boxed{(\underline{a}, \underline{b})=a_1b_1+a_2b_2+a_3b_3}$.\\
Folgerung: $\boxed{(\underline{a}, \underline{b})=\underline{a}^T \cdot \underline{b}=\underline{b}^T \cdot \underline{a}}$\\
Schreibweisen: $(\underline{a}, \underline{b})= \underline{a} \circ \underline{b}= ...$\medskip\\
Anwendungen:
\begin{enumerate}
\item Projektion $\underline{a}_{\underline{b}}$ von $\underline{a}$ auf $\underline{b}$: $\boxed{\underline{a}_{\underline{b}}=(\underline{a}, \underline{b}^0)\underline{b}^0=\frac{(\underline{a}, \underline{b})}{|\underline{b}|^2}\underline{b}}$\\
ABB 54\\
Herleitung:\\
$|\underline{a}_{\underline{b}}|=|\underline{a}|\cdot cos(\varphi)$\\
$\underline{a}_{\underline{b}}=|\underline{a}|\cdot cos(\varphi)\frac{\underline{b}}{|\underline{b}|}=|\underline{a}| \cdot |\underline{b}| \cdot cos (\varphi) \frac{\underline{b}}{|\underline{b}|^2}=(\underline{a}, \underline{b})\cdot \frac{1}{|\underline{b}|^2}\cdot \underline{b}$
\item Winkel $\varphi$ zwischen zwei Vektoren: $\boxed{cos(\varphi)=\frac{(\underline{a}, \underline{b})}{|\underline{a}|\cdot |\underline{b}|}}$
\subparagraph{Bsp. 12:} \parskp
$\underline{a}=\begin{pmatrix}
1\\
-2\\
3
\end{pmatrix}, \underline{b}=\begin{pmatrix}
0\\
-4\\
7
\end{pmatrix}$
\begin{enumerate} [label=\alph*.)]
\item $|\underline{a}| = \sqrt{1^2+(-2)^2+3^2}=\sqrt{14}, |\underline{b}|=\sqrt{0^2+(-4)^2+7^2}=\sqrt{65}$\\
$cos(\varphi)=\frac{(\underline{a}, \underline{b})}{|\underline{a}|\cdot |\underline{b}|}=\frac{1 \cdot 0 + (-2) \cdot (-4) + 3\cdot 7}{\sqrt{14} \cdot \sqrt{65}}=\frac{29}{\sqrt{14} \cdot \sqrt{65}}$\\
$\varphi=arccos\left(\frac{29}{\sqrt{14} \cdot \sqrt{65}}\right)\approx 15,92^{\circ}$
\item Projektion von $\underline{b}$ auf $\underline{a}$: $\underline{b}\underline{a}=\frac{(\underline{a}, \underline{b})}{|\underline{a}|^2}\underline{a}=\frac{29}{14}\begin{pmatrix}
1\\
-2\\
3
\end{pmatrix}=\frac{29}{14}\underline{e}_1-\frac{29}{7}\underline{e}_2+\frac{29\cdot 3}{14} \underline{e}_3$
\end{enumerate}
\item \emph{Orthogonalitätskriterium}:\\
$(\underline{a}, \underline{b})=0 \Leftrightarrow (\underbrace{|\underline{a}|=0}_{\underline{a}=} \vee \underbrace{|\underline{b}|=\underline{0}}_{\underline{b}=\underline{0}} \vee cos(\varphi)=0)$\\
Vereinbarung: $\underline{0}$ orthogonal zu jedem Vektor
$\curvearrowright \boxed{(\underline{a}, \underline{b})=0 \;\Leftrightarrow\; \underline{a} \bot \underline{b}}$

\subsubsection{Das vektorielle Produkt}
\paragraph{Def. 11:} \parskp
Das vektorielle Produkt $\underline{a} \times \underline{b}$ zweier Vektoren $(\underline{a}, \underline{b} \in \mathbb{R}^3)$ ist ein Vektor, der eindeutig festgelegt ist durch:
\begin{enumerate}[label=(\arabic*)]
\item $|\underline{a} \times \underline{b}| = |\underline{a}| \cdot |\underline{b}| \cdot sin(\varphi)$
\item $\underline{a} \times \underline{b}$ ist senkrecht zu $\underline{a}$ und senkrecht zu $\underline{b}$.
\item $\underline{a}, \underline{b}$ und $\underline{a}\times \underline{b}$ bilden in dieser Reihenfolge ein Rechtssystem.
\end{enumerate}
Eigenschaften des vektoriellen Produktes:
\begin{itemize}
\item $\underline{a} \times \underline{b} = -(\underline{b}\times \underline{a})$ (Anti-Kommutativgesetz)
\item $\underline{a} \times (\underline{a} + \underline{c})=\underline{a} \times \underline{b} + \underline{a} \times \underline{c}$ (Distributivgesetz)
\item $\lambda (\underline{a} \times \underline{b} )= (\lambda \underline{a}) \times \underline{b}= \underline{a} \times (\lambda \underline{b})$
\item Speziell: $\underline{a} \times \underline{a} = \underline{0}$
\item $\underline{e}_1 \times \underline{e}_2= \underline{e}_3,\; \underline{e}_2\times \underline{e}_3=\underline{e}_1$ usw.
\end{itemize}

\paragraph{Satz 5:} \parskp
Es sei $\underline{a}=\begin{pmatrix}
a_1\\
a_2\\
a_3
\end{pmatrix}$ und $\underline{b}=\begin{pmatrix}
b_1\\
b_2\\
b_3
\end{pmatrix}$, dann gilt:\\
$\underline{a} \times \underline{b} \underset{\text{Schema}}{\corr} \begin{vmatrix}
\underline{i}& a_1 & b_1\\
\underline{j} & a_2 & b_2\\
\underline{k} & a_3 & b_3
\end{vmatrix} \corr \begin{vmatrix}
a_2 & b_2\\
a_3 & b_3
\end{vmatrix} \underline{i} - \begin{vmatrix}
a_1 & b_1\\
a_3 & b_3
\end{vmatrix} \underline{j} + \begin{vmatrix}
a_1 & b_1 \\
a_2 & b_2
\end{vmatrix} \underline{k} \\
\underline{a} \times \underline{b}=(a_2b_3-a_3b_2)\underline{i}-(a_1b_3-a_3b_1)\underline{j}+(a_1b_2-a_2b_1)\underline{k}= \begin{pmatrix}
a_2b_3-a_3b_2\\
a_2b_1-a_1b_3\\
a_1b_2-a_2b_1
\end{pmatrix}$
\end{enumerate}

\subparagraph{Bsp. 13:}\parskp
$\underline{a}=\begin{pmatrix}
1\\
-2\\
3
\end{pmatrix}, \; \underline{b}= \begin{pmatrix}
0\\
-4\\
7
\end{pmatrix}\\
\underline{a}\times \underline{b} = \begin{vmatrix}
-2 & -4\\
3 & 7
\end{vmatrix}\underline{i}- \begin{vmatrix}
1 & 0 \\
3 & 7
\end{vmatrix} \underline{j}+ \begin{vmatrix}
1 & 0\\
-2 & -4
\end{vmatrix} \underline{k}= -2 \underline{i} -7 \underline{j} - 4 \underline{k}= \begin{pmatrix}
-2\\
-7\\
-4
\end{pmatrix}$\\
Kontrolle: $(\underline{a} \times \underline{b}, \underline{a})=0, \; (\underline{a} \times \underline{b}, \underline{b})=0 \; !$\bigskip\\
Anwendungen:
\begin{enumerate}
\item \emph{Flächeninhalt} des von $\underline{a}$ und $\underline{b}$ aufgespannte \emph{Parallelogramms}: $\boxed{F=|\underline{a} \times \underline{b}|}$\\
ABB 55\\
$sin(\alpha)=\frac{h}{|\underline{b}|}$\\
$F=|\underline{a}|\cdot h = |\underline{a}| \cdot |\underline{b}| \cdot sin(\alpha)=|\underline{a} \times \underline{b}|$
\item Flächeninhalt eines Dreiecks $\Delta P_1P_2P_3$: $\boxed{F= \frac{1}{2}\left|\overrightarrow{P_1P_2}\times \overrightarrow{P_1P_2}\right|}$ (halbes Parallelogramm)
\item \emph{Parallelitätskriterium}: $\underline{a}\times \underline{b}=\underline{0} \; \Leftrightarrow \; |\underline{a} \times \underline{b}|=0 \; \Leftrightarrow \; (|\underline{a}| =0 \vee |\underline{b}|=0 \vee sin(\varphi)=0)$\\
Vereinbarung: $\underline{0}\; ||$ zu jedem Vektor\\
$\curvearrowright \boxed{\underline{a}\times \underline{b}=\underline{0} \; \Leftrightarrow \; \underline{a} || \underline{b}}$
\end{enumerate}

\subsubsection{Das Spatprodukt}
\paragraph{Def. 12:} \parskp
Die Zahl $(\underline{a} \times \underline{b}, \underline{c})$ heißt Spatprodukt der Vektoren $\underline{a}, \underline{b}$ und $\underline{c}$.\\
Eigenschaften: 
$(\underline{a} \times \underline{b},\underline{c})=(\underline{b} \times \underline{c}, \underline{a})=(\underline{c}\times \underline{a}, \underline{b})$ (durch zyklisches Vertauschen)\\
Berechnung: $\boxed{(\underline{a} \times \underline{b}, \underline{c})= det(\underline{a} |\underline{b}|\underline{c})= \begin{vmatrix}
a_1 & b_1 & c_1\\
a_2 & b_2 & c_2\\
a_3 & b_3 & c_3
\end{vmatrix}}$\\
Anwendung: 
\begin{enumerate}
\item Volumen des von $\underline{a}, \underline{b}$ und $\underline{c}$ aufgespannten Spates (Parallelotop): $\boxed{V=|(\underline{a}\times \underline{b}, \underline{c})|}$\\
ABB 57\\
$V=F_{\text{Grundfläche}} \cdot h=|\underline{a}\times \underline{b}| \cdot |\underline{c}| \cdot |cos(\alpha)|=|(\underline{a}\times \underline{b}, \underline{c})|$\\
Bemerkung:\\
Spatprodukt $\begin{cases}
>0 & \text{... Rechtssystem}\\
<0 & \text{... Linkssystem}
\end{cases}$
\item \emph{Komplanaritätskriterium}:\\
Die Vektoren $\underline{a}, \underline{b}, \underline{c}$ sind komplanar, d.h. sie liegen in einer (in $O$ angehefteten) Ebene \\
$\Leftrightarrow \; (\underline{a} \times \underline{b}, \underline{c})=0 \\
\Leftrightarrow\;  \underline{a}, \underline{b}, \underline{c}$ sind linear abhängig.
\end{enumerate}

\subsubsection{Geraden- und Ebenengleichungen}
\begin{enumerate}
\item Parameterdarstellung einer Geraden $g$ durch $P_1$ und $P_2$:\\
$P$ … beliebiger Punkt von $g$\\
ABB 58\\
$\overrightarrow{OP}=\overrightarrow{OP_1}+t\cdot \overrightarrow{P_1P_2} \quad ( t \in \mathbb{R})$\\
$\boxed{\underline{r}=\underline{r}_1+t \cdot \underline{a}}$ (Punkt-Richtungs-Form)\\
$\boxed{\underline{r}=\underline{r}_1 + t \cdot (\underline{r}_2-\underline{r}_1)} \quad( t \in \mathbb{R})$ (Zwei-Punkte-Form)
\subparagraph{Bsp.:} \parskp
Gerade durch die Punkte $P_1=(1,2,-1), P_2=(0,1,4)$\\
$g: \; \begin{pmatrix}
x\\
y\\
z
\end{pmatrix}=\begin{pmatrix}
1\\
2\\
-1
\end{pmatrix}+t\begin{pmatrix}
-1\\
-1\\
5
\end{pmatrix}\quad (t \in \mathbb{R})$
\item Parameterdarstellung einer Ebene $\varepsilon$ durch 3 Punkte $P_1, P_2 ,P_3$, die nicht auf einer Geraden liegen.\\
ABB 59 \\
$P$ … beliebiger Punkt von $\varepsilon$\\
$\overrightarrow{OP}=\overrightarrow{OP_1}+ u \cdot \overrightarrow{P_1P_2}+v\cdot \overrightarrow{P_1P_3} \quad (u,v \in \mathbb{R})$\\
$\boxed{\underline{r}=\underline{r}_1+u\cdot \underline{a}+v\cdot \underline{b}} \quad (u,v \in \mathbb{R})$\\
$\boxed{\underline{r}=\underline{r}_1+u\cdot (\underline{r}_2-\underline{r}_1)+v(\underline{r}_3-\underline{r}_1)}$
\item Parameterfreie Ebenengleichung\\
ABB 60\\
Normalenvektor $\underline{n}$ ($\underline{n}\not = 0,\;\underline{n}\bot \varepsilon$):\\
$\underline{n}=\begin{pmatrix}
a\\
b\\
c
\end{pmatrix},\; \underline{n} \bot \overrightarrow{P_0P}$\\
Dabei sei $P(x,y,z)$ ein beliebiger Punkt in $\varepsilon$ und $P_0(x_0,y_0,z_0)$ ein fester Punkt in $\varepsilon$ mit Orthogonalitätskriterium $(\underline{n}, \overrightarrow{P_0P})=0$ bzw. $\boxed{(\underline{n}, \underline{r}-\underline{r}_0)=0}$.\\
Ausführlich: $\left(\begin{pmatrix}
a\\
b\\
c
\end{pmatrix}, \begin{pmatrix}
x-x_0\\
y-y_0\\
z-z_0
\end{pmatrix}\right)=0$, d.h. $a \cdot (x-x_0)+b(y-y_0)+c(z-z_0)=0$\\
Allgemeine Form: $\boxed{ax+by+cz+d=0}$ mit $d=-ax_0-by_0-cz_0$.
\subparagraph{Bsp. 15:} \parskp
Ebene durch $P_1(1,0,0), P_2(3,1,5), P_3(-2,0,2)$
\begin{itemize}
\item P.d. (Parameterdarstellung) $\begin{pmatrix}
x\\
y\\
z
\end{pmatrix} = \begin{pmatrix}
1\\
0\\
0
\end{pmatrix} + u \underbrace{\begin{pmatrix}
2\\
1\\
5
\end{pmatrix}}_{\underline{a}}+v \underbrace{\begin{pmatrix}
-3\\
0\\
2
\end{pmatrix}}_{\underline{b}} \quad (u,v \in \mathbb{R})$
\item Ein Normalenvektor ist bpsw. $\underline{u}=\underline{a} \times \underline{b}=\begin{pmatrix}
2\\
-19\\
3
\end{pmatrix}$\\
$\curvearrowright$ Parameterfreie Darstellung: $2x-19y+3z+d=0$\\
$d$ berechnen: Einsetzen von $x=1, y=z=0 \; (P_1)$ liefert $2\cdot 1 +d =0 \; \Rightarrow\; d=-2$\\
$\curvearrowright \boxed{2x+19y+3z-2=0}$
\end{itemize}
\end{enumerate}

\subsubsection{Einige geometrische Grundaufgaben}
\begin{enumerate}
\item Schnitt von Gerade und Ebene
\subparagraph{Bsp. 16:} \parskp
Gegeben:\\
Ebene $\varepsilon$: $2x-4y+z+3=0$\\
Gerade $g$: $\begin{pmatrix}
x\\
y\\
z
\end{pmatrix} = \begin{pmatrix}
3\\
0\\
1
\end{pmatrix}+t\begin{pmatrix}
-1\\
1\\
-2
\end{pmatrix}$\\
Gesucht:
\begin{enumerate}[label=\alph*.)]
\item Schnittpunkt (Spurpunkt) $S(x_S,y_S,z_S)$
\item Schnittwinkel
\end{enumerate}
\begin{enumerate}[label=zu \alph*.)]
\item $g: x=3-t, \; y=t, \; z=1-2t$ einsetzen in Ebenengleichung: $2(3-t)-4\cdot t+1-2t+3=0 \Rightarrow -8t+10=0 \Rightarrow t=\frac{5}{4}$\\
$t=\frac{5}{4}$ in Geradengleichung einsetzen: $x_S=3-\frac{5}{4}=\frac{7}{4}, y_S=\frac{5}{4}, z_S=1-2\frac{5}{4}=-\frac{3}{2} \\
\curvearrowright \resultul{S\left(\frac{7}{4},\frac{5}{4},-\frac{3}{2}\right)}$
\item Schnittwinkel:\\
ABB 61\\
$\beta= \measuredangle (\underline{n}, \underline{a})$ (Richtungsvektor von $g$)\\
$\alpha = | 90^\circ - \beta|$\\
$\underline{n}=\begin{pmatrix}
2\\
-4\\
1
\end{pmatrix}, \underline{a}=\begin{pmatrix}
-1\\
1\\
-2
\end{pmatrix}, \beta = arrcos\left(\frac{(\underline{n}, \underline{a})}{|\underline{n}| \cdot |\underline{a}|}\right)\approx 135,45^\circ$\\
$\curvearrowright \alpha = | 90^\circ - \beta | \approx 45,45^\circ$
\end{enumerate}
\item Schnitt zweier Ebenen: 2 Gleichungen, 3 Unbekannte
\subparagraph{Bsp. 17:} \parskp
Schnitt der Ebenen $\varepsilon_1:\; x+y+z-1=0$ und $\varepsilon_2:\; x-2y+3z+4=0$.\\
Austauschverfahren:\\
\begin{tabular}{r | c c c c}
$T_1$ & x & y & z & 1\\
\hline
0 & \fbox{1} & 1 & 1 &-1\\
0 & 1 & -2 &3 & 4\\
\hline
$K$ & * & -1 & -1 & 1
\end{tabular}\quad 
\begin{tabular}{r | c c c c}
$T_2$ & x  & z & 1\\
\hline
x & -1 & -1 &1\\
0 & -3 &\fbox{2} & 5\\
\hline
$K$ & $\tfrac{3}{2}$ & * & $-\tfrac{5}{2}$
\end{tabular}\quad
\begin{tabular}{r | c c c c}
$T_3$ & y & 1\\
\hline
x & $-\tfrac{5}{2}$ & $\tfrac{7}{2}$\\
z & $\tfrac{3}{2}$ & $-\tfrac{5}{2}$
\end{tabular}\\
$y$ … NBV, $y=t$ (beliebig), $x=-\frac{5}{2}t+\frac{7}{2}$, $z=\frac{3}{2}t-\frac{5}{2}$\\
also $\begin{pmatrix}
x\\
y\\
z
\end{pmatrix} = \begin{pmatrix}
-\frac{5}{2}t+\frac{7}{2}\\
t\\
\frac{3}{2}t - \frac{5}{2}
\end{pmatrix}$, oder $\begin{pmatrix}
x\\
y\\
z
\end{pmatrix} = t \begin{pmatrix}
-\frac{5}{2}\\
1\\
\frac{3}{2}
\end{pmatrix}+\begin{pmatrix}
\frac{7}{2}\\
0\\
-\frac{5}{2}
\end{pmatrix} \quad (t \in \mathbb{R})$
\item Abstand $d(P_1,\varepsilon)$ eines Punktens $P_1$ in einer Ebene $\varepsilon$.\\
$\varepsilon$: $ax+by+cz+d=0$, Punkt $P_1(x_1,y_1,z_1)$\\
$\boxed{d(P_1,\varepsilon)=\frac{|ax_1+by_1+cz_1+d|}{\sqrt{a^2+b^2+c^2}}}$\\
ABB 91\\
$\underline{n}=\begin{pmatrix}
a\\
b\\
c
\end{pmatrix}, P_0\in \varepsilon$, d.h. $ax_0+by_0+cz_0+d=0$\\
$d(P_1,\varepsilon)=\left|\overrightarrow{P_0P_1}_{\overrightarrow{n}}\right|=\frac{\left|\left(\overrightarrow{P_0P_1,\underline{n}}\right)\right|}{|\underline{n}|}$\\
$\overrightarrow{P_0P_1}=\begin{pmatrix}
x_1-y_0\\
y_1-y_0\\
z_1-z_0
\end{pmatrix}$\\
$\Rightarrow d(P_1,\varepsilon)=\frac{|a(x_1-x_0+b(y_1-y_0)+c(z_1-z_0)|}{\sqrt{a^2+b^2+c^2}}=\frac{|ax_1+by_1+cz_1|}{\sqrt{a^2+b^2+c^2}}$
\subparagraph{Bsp. 18:} \parskp
Abstand von $P_1(2,-9,-16)$ von der Ebene $\varepsilon: \; 3x-7y+8z+26=0$.\\
$d(P_1, \varepsilon)=\frac{|3\cdot 2 - 7 \cdot (-9)+8 \cdot (-16) +26 |}{\sqrt{3^2+(-7)^2+8}}=\frac{|-33|}{\sqrt{122}}=\resultul{\frac{33}{\sqrt{122}}}$\\
Bemerkung: Gerade $g$ in der x-y-Ebene, Gleichung $ax+by+c=0$, NV: $\underline{n}=\begin{pmatrix}
a\\
b
\end{pmatrix}$, Abstand eines Punktes $P_1(x_1,y_1)$ von $g$:\\
$\boxed{d(P_1,g)}=\frac{|ax_1+by_1+c|}{\sqrt{a^2+b^2}}$
\item Abstand $d(Q,g)$ eines Punktes $Q$ von einer Geraden $g$ (in $\mathbb{R}^3$).\\
$g: \underline{r}=\underbrace{\overrightarrow{OP_1}}_{\underline{r}_1}+t\underline{a} \quad t \in \mathbb{R}$ (Parameterdarstellung)\\
ABB 92\\
($d$ ist Höhe $\overline{LQ}$ des von $\underline{a}$ und $\overrightarrow{P_1Q}$ aufgespannten Parallelogramms)\\
Lotfußpunkt: $\boxed{\overrightarrow{OL}=\overrightarrow{OP_1}+\overrightarrow{P_1Q}_{\underline{a}}}$
\subparagraph{Bsp. 19:} \parskp
$g: \begin{pmatrix}
x\\
y\\
z
\end{pmatrix}=\begin{pmatrix}
2\\
3\\
1
\end{pmatrix}+t \begin{pmatrix}
1\\
0\\
-1
\end{pmatrix} \quad t \in \mathbb{R}, Q(1,1,1)$
\begin{enumerate}
\item Abstand $d(Q,g)$:\\
$\underline{a}=\begin{pmatrix}
1\\
0\\
-1
\end{pmatrix} , \underline{r}_1=\begin{pmatrix}
2\\
3\\
1
\end{pmatrix}, \overrightarrow{OQ}=\begin{pmatrix}
1\\
1\\
1
\end{pmatrix}, \overrightarrow{P_1Q}=\begin{pmatrix}
-1\\
-2\\
0
\end{pmatrix}$\\
$\overrightarrow{P_1Q} \times \underline{a} = 2 \underline{i} - \underline{j} + 2 \underline{k}= \begin{pmatrix}
2\\
-1\\
2
\end{pmatrix}$\\
$\left| \overrightarrow{P_1Q} \times \underline{a}\right| = \sqrt{2^2+(-1)^2+2^2}= \sqrt{9} = 3$\\
$d(Q,g)= \frac{3}{\sqrt{2}}= \resultul{\frac{3}{2} \sqrt{2}}$
\item Lotfußpunkt: $\overrightarrow{P_1Q}_{\underline{a}}=\frac{\left(\overrightarrow{P_1Q}, \underline{a}\right)}{|\underline{a}|^2}\cdot \underline{a}=-\frac{1}{2}\begin{pmatrix}
1\\
0\\
-1
\end{pmatrix}$\\
$\Rightarrow \overrightarrow{OL}=\begin{pmatrix}
2\\
3\\
1
\end{pmatrix}-\frac{1}{2}\begin{pmatrix}
1\\
0\\
-1
\end{pmatrix}=\begin{pmatrix}
\tfrac{3}{2}\\
3\\
\tfrac{3}{2}
\end{pmatrix} \;\Rightarrow\;L\left(\frac{3}{2},3,\frac{3}{2}\right)$
\end{enumerate}
\item Abstand $d(g_1,g_2)$ zweier nicht paralleler Geraden $g_1$ und $g_2$.\\
$g_1: \; \underline{r}=\underline{r}_1 + s \cdot \underline{a}_1$\\
$g_2: \; \underline{r}=\underline{r}_2 + t \cdot \underline{a}_2 \quad (s,t \in \mathbb{R})$\\
ABB 93\\
$d=\left| \overrightarrow{P_1P_2}_{\underline{a}_1 \times \underline{a}_2}\right|=d(g_1,g_2)=\frac{|(\underline{r}_2-\underline{r}_1, \underline{a}_1 \times \underline{a}_2)|}{|\underline{a}_1 \times \underline{a}_2|}$\\
Bemerkung: Lotfußpunkte $L_1$ und $L_2$ aus Bedingungen $\overrightarrow{L_1L_2} \bot \underline{a}_1$ und $\overrightarrow{L_1L_2}\bot \underline{a}_2$ ermittelbar.
\end{enumerate}

\subsection{Eigenwerte und Eigenvektoren}
%Motivation:\\
%Zum Beispiel $A$ vom Typ $(2,2)$ durch Abbildung $f: \mathbb{R}^2\rightarrow \mathbb{R}^2$:\\ $f(\underline{x})=f\left( \begin{pmatrix}
%x\\
%y\\
%z
%\end{pmatrix}\right)=\underline{A}_{\underline{x}}$\\
%Falls $f(\underline{x})=\lambda\cdot \underline{x}$, dann $\lambda =$Eigenwert und $\underline{x} =$Eigenvektor. Dann also $\boxed{\lambda\cdot \underline{x}=\underline{A} \underline{x}}$, woraus sich das Gleichungssystem $\underline{A} \underline{x}-\lambda \cdot \underline{x}=\underline{0}$ bzw. $(\underline{A}-\lambda E) \underline{x}=\underline{0}$ ergibt.
Es sei $\underline{A}$ eine $(n,n)$-Matrix.
\paragraph{Def. 13:} \parskp
Die Zahl $\lambda \in \mathbb{C}$ heißt \emph{Eigenwert} (EW) der quadratischen Matrix $\underline{A}$, falls die Gleichung $\boxed{\underline{A}\,\underline{x}=\lambda \underline{x}}$ nichttriviale Lösungsvektoren $\underline{x}$ besitzt. Diese heißen dann \emph{Eigenvektoren} (EV) von $\underline{A}$ zum Eigenwert $\lambda$.
\subparagraph{Diskussion:}
\begin{enumerate}
\item $\underline{A}\,\underline{x} = \lambda \underline{x} \Leftrightarrow (\underline{A}-\lambda\underline{E})\underline{x}=\underline{0}$\\
D.h. nichttriviale Lösungen existieren genau dann, wenn $\boxed{det(\underline{A}-\lambda\underline{E})=0}$ (\emph{charakteristische Gleichung}) gilt.\\
Vorgehensweise zur Ermittlung von EW und EV:
\begin{itemize}
\item charakt. Gleichung lösen ($n$ i.a. komlpexe Lösungen $\lambda_1, ..., \lambda_n$)
\item Gleichungssystem $(\underline{A}-\lambda_i\underline{E})\underline{x}=\underline{0}$ für $i=1,...,n$ lösen.
\end{itemize}
Im folgenden werden nur symmetrische $(n,n)$-Matrizen $\underline{S}$ betrachtet, d.h. $\underline{S}^T=\underline{S}$.
\paragraph{Satz 6:} \parskp
Es sei $\underline{S}$ eine symmetrische $(n,n)$-Matrix. Dann gilt:
\begin{enumerate}[label=(\arabic*)]
\item Alle Eigenwerte von $\underline{S}$ sind reell.
\item Zu verschidenen EW $\lambda_1$ bzw. $\lambda_2$ ($\lambda_1 \not = \lambda_2$) gehörende EV $\underline{v}_1$ bzw. $\underline{v}_2$ sind orthogonal (vgl. Disskussion).
\item Es gibt eine Basis des Raumes $\mathbb{R}^n$, die aus $n$ paarweise orthonormierten EV $\underline{v}_1, ... , \underline{v}_n$ von $\underline{S}$ besteht.
\item Es sei $\underline{V}=(\underline{v}_1|...|\underline{v}_n)$ eine Matrix, deren Spaltenvektoren $n$ paarweise orthonomierte EV von $\underline{S}$ sind. Dann gilt:
\begin{itemize}
\item $\underline{V} \cdot \underline{V}^T=\underline{V}^T\cdot \underline{V} = \underline{E}$ (d.h. $\underline{V}^{-1}=\underline{V}^T$, $\underline{V}$ ist sogenannte orthogonale Matrix)
\item $\underline{V}^T\cdot \underline{S}\cdot \underline{V}=\begin{pmatrix}
\lambda_1 & 0 & 0\\
0 & ... & 0\\
0 & 0 & \lambda_n
\end{pmatrix}=\Lambda \; \curvearrowright \boxed{\underline{S}=\underline{V} \cdot\Lambda\cdot \underline{V}^T}$
\item Es gilt $\underline{S}^{-1}=\underline{V}\cdot\Lambda^{-1}\cdot\underline{V}^T$ mit $\begin{pmatrix}
\tfrac{1}{\lambda_1} & 0 & 0\\
0 & ... & 0\\
0 & 0 & \tfrac{1}{\lambda_n}
\end{pmatrix}=\Lambda^{-1}$\\
$\underline{S}^n=\underline{V}\cdot\Lambda^n\cdot\underline{V}^T$
\end{itemize}
\end{enumerate}
\emph{Betrag (Norm)} eines Vektors $|\underline{a}|=\sqrt{\sum_{i=1}^n a_i^2}$ paarweise orthonormiert bedeutet:\\
 $(\underline{v}_i, \underline{v}_j)=\begin{cases}
1 & \text{für }i=j\\
0 & \text{für }i \not = j
\end{cases}$.
\item Veranschaulichung im Fall $n=2$:\\
Die symmetrische Matrix $\underline{A}$ habe die Eigenwerte $\lambda_1$ und $\lambda_2$ und orthonomierte EV $\underline{v}_1$ und $\underline{v}_2$, $V=(\underline{v}_1| \underline{v}_2)$. Es gilt $\underline{A} \cdot \underline{v}_1=\lambda_1 \underline{v}_1$, $\underline{A} \cdot \underline{v}_2=\lambda_2 \underline{v}_2$.\\
\fbox{D.h $\underline{A}$ bewirkt eine Skalierung mit den Faktoren $\lambda_1$ bzw. $\lambda_2$ in Richtung $\underline{v}_1$ bzw. $\underline{v}_2$.}\\
\begin{tikzpicture} [scale =.5]
\draw[-latex] (-5,0) -- (5,0) node [right] {x};
\draw[-latex] (0,-.5) -- (0,5) node[right] {y};

\draw[-latex, green] (0,0) -- (3,3) node[below right] {$\underline{v}_1$};
\draw[-latex, orange] (0,0.1) -- (5,5.1) node[below right] {$\lambda_1\,\underline{v}_1$};

\draw[-latex, green] (0,0) -- (-3,3) node[below left] {$\underline{v}_2$};
\draw[-latex, orange] (0,0.1) -- (-5,5.1) node[below left] {$\lambda_2\,\underline{v}_2$};
\end{tikzpicture}
\end{enumerate}

\paragraph{Def. 14: }
Es sei $\underline{S}$ eine reelle symmetrische Matrix vom Typ $(n,n)$. Die Funktion $y=Q(\underline{x}):=\underline{x}^T\, \underline{S}\, \underline{x}$ ($\underline{x}\in \mathbb{R}^n,\; y \in \mathbb{R}$) heißt \emph{quadratische Form}.
\subparagraph{Diskussion:} 
\begin{enumerate}
\item Im Falle $n=2$ stellt $Q(\underline{x})=const)$ (bzw. $Q(\underline{x})+\underline{a}^T\underline{x=const}$) eine Kurve 2. Ordnung dar. Deren Gestalt kann durch die sogennante Hauptachsentransformation ermittelt werden.
\item Ausführliche Schreibweise $\underline{x}=\begin{pmatrix}
x\\
y
\end{pmatrix},\; \underline{S}=\begin{pmatrix}
s_{11} & s_{12}\\
s_{21} & s_{22}
\end{pmatrix}$ (mit $s_{12}=s_{21}$).\\
$\boxed{Q(x,y)=\begin{pmatrix}
x & y
\end{pmatrix} \begin{pmatrix}
s_{11} & s_{12}\\
s_{21} & s_{22}
\end{pmatrix} \begin{pmatrix}
x\\
y
\end{pmatrix}=s_{11}x^2+2s_{12}xy+s_{22}x^2}$
\item Es seien $\lambda_1$ und $\lambda_2$ die EV von $\underline{S}$ und $\underline{v}_1$ bzw. $\underline{v}_2$ orthonormierte EV. Für einen beliebigen Vektor $\underline{x}=\begin{pmatrix}
x\\
y
\end{pmatrix}\in \mathbb{R}^2$ seien $x^*$ und $y^*$ die Koordinaten bzgl. der Basis $\underline{v}_1, \underline{v}_2$:\\
$\boxed{\underline{x}=x^*\underline{v}_1+y^*\underline{v}_2 = (\underline{v}_1 | \underline{v}_2) \begin{pmatrix}
x^*\\
y^*
\end{pmatrix}=\underline{V}\,\underline{x}^*}$\\
Dann gilt:\\
$\boxed{Q(x,y)=\lambda_1x^{*2}+\lambda_2x^{*2}}$ (Darstellung bzgl. der sog. Hauptachsen)\\
Dann: $Q(x,y)=\underline{x}^T\, \underline{S}\, \underline{x}=\left(\underline{V}\, \underline{x}^*\right)^T \, \underline{S} \, \left( \underline{V} \, \underline{x}^*\right)=\underline{x}^{*T} \underbrace{\underline{V}^T\, \underline{S} \, \underline{V}}_{\Lambda}\underline{x}^*=\begin{pmatrix}
x^*&
y^*
\end{pmatrix} \begin{pmatrix}
\lambda_1 & 0 \\
0 & \lambda_2
\end{pmatrix}\begin{pmatrix}
x^*\\
y^*
\end{pmatrix}\\
\Leftrightarrow Q(x,y)= \lambda_1 x^{*2}+\lambda_2y^{*2}$
\end{enumerate}
\subparagraph{Bsp. 20:} \parskp
$Q(x,,y)=13x^2-32xy+37y^2=45$\\
Welche Kurve ist das?
\begin{itemize}
\item Matrix $\underline{S}$ (vgl. Gleichung aus 2.) aus obiger Diskussion):\\
$\underline{S}=\begin{pmatrix}
13 & -16\\
-16 & 37
\end{pmatrix}$
\item charakteristische Gleichung: \\
$det(\underline{S}-\lambda\underline{E})\overset{!}{=}0 \; \Leftrightarrow \; \begin{vmatrix}
13-\lambda & -16\\
-16 & 37-\lambda
\end{vmatrix}=\lambda^2+50\lambda+225\overset{!}{=}0$\\
$\resultul{\lambda_1=5}, \; \resultul{\lambda_2 = 45}$ (Eigenwerte)
\item EV zu $\lambda_1=5$ ($(\underline{S}-\lambda\underline{E})\underline{x}\overset{!}{=}\underline{x}$):
\begin{align*}
8x-16y &= 0\\
-16x+32y&=0
\end{align*}
$\curvearrowright x=2y \curvearrowright\resultul{\begin{pmatrix}
x\\
y
\end{pmatrix}= \begin{pmatrix}
2t\\
t
\end{pmatrix}=t \begin{pmatrix}
2\\
1
\end{pmatrix}} \quad (t \in \mathbb{R}, t\not = 0)$
\item EV zu $\lambda_2=45$:\\
$\resultul{\begin{pmatrix}
x\\
y
\end{pmatrix}=u\cdot \begin{pmatrix}
1\\
-2
\end{pmatrix}}\quad (u \in \mathbb{R}, u\not = 0)$
\item orthonomierte EV:\\
z.B. $\underline{v}_1=\frac{1}{\sqrt{5}}\begin{pmatrix}
2\\
1
\end{pmatrix}, \; \underline{v}_2=\frac{1}{\sqrt{5}}\begin{pmatrix}
1\\
-2
\end{pmatrix}$ (Rechtssystem!)
\end{itemize}
Mit Gleichung aus 3.) aus obiger Diskussion:\\
$Q(x,y)=\lambda_1x^{*2}+\lambda_2 x^{*2}=5x^{*2}+45y^{*2}=45$\\
$\Leftrightarrow \boxed{\frac{x^{*2}}{9}+\frac{y^{*2}}{1}=1}$ (Ellipse mit Halbachsen $a=3, \; b=1$)
\newpage
\part{Folgen, Reihen, Grenzwerte}
\setcounter{section}{0}
%\renewcommand*{\theHsection}{chX.\the\value{section}}
\section{Zahlenfolgen}
\subsection{Grenzwerte von Zahlenfolgen}
\paragraph{Def. 1:} \parskp
Es sei $n_0 \in \mathbb{N}$. Eine Funktion $f$ mit $Db(f)=\{u\in \mathbb{N}|n\geq n_0\}$ und $Wb(f) \subset \mathbb{R}$ heißt reelle Zahlenfolge.\\
Schreibweise: \\
$a_n=f(n)$ \qquad ($n \in Db(f)$)\\
$\left(a_n\right)_{n\geq n_0}=\left(a_{n_0}, a_{n_0+1}, a_{n_0+2}, ...\right)$\\
oft $n_0=0$ oder $n_0=1$.
\subparagraph{Bsp. 1:} 
\begin{enumerate}[label=\alph*.)]
\item $a_n=(-1)^n\cdot n \quad (n \in \mathbb{N})\\
(a_n)=(0,-1,2,-3,4,...)$
\item $a_0=-1,\; a_n=n\cdot a_{n-1} \quad (n \in \mathbb{N}^*)$ \quad (rekursive Def.)\\
$(a_n)=(-1,-1,-2,-6,-24,...), \; a_n = -n!$
\item $a_n=\frac{3}{10}+\frac{3}{10^2}+...+\frac{3}{10^n} \quad (n\in \mathbb{N}^*)$\\
$(a_n)=(0.3, 0.33, 0.333, ... )$
\item $a_n=1+(-1)^n\frac{1}{n^2} \quad (n \in \mathbb{N}^*)$\\
$(a_n)=\left( \frac{5}{4}, \frac{8}{9}, \frac{17}{16}, \frac{24}{25},...\right)$
\end{enumerate}
\paragraph{Def. 2:} 
\begin{itemize}
\item $(a_n)$ heißt \emph{konvergent}, wenn es eine Zahl $a \in \mathbb{R}$ gibt mit folgender Eigenschaft:\\
Zu jedem $\varepsilon > 0$ existiert eine natürliche Zahl $n_0(\varepsilon)$, sodass für alle $n \geq n_0(\varepsilon)$ gilt: $|a_n-a|< \varepsilon$.
\item Die Zahl $a$ heißt \emph{Grenzwert} von $(a_n)$.\\
Schreibweisen:\\
$\boxed{a=\underset{n\rightarrow \infty}{lim}(a_n)}$ oder $\boxed{a_n \underset{\color{gray}n\rightarrow\infty}{\longrightarrow} a}$
\item $(a_n)$ heißt \emph{divergent}, falls $(a_n)$ nicht konvergent ist.
\end{itemize}
\subparagraph{Diskussion}
\begin{enumerate}
\item Für $\varepsilon>0$ heißt $U_{\varepsilon}(a):=(a-\varepsilon, a+\varepsilon)$ (offenes Intervall) \emph{$\varepsilon$-Umgebung von $a$}.\\
\begin{tikzpicture}
\draw [-latex] (-5,0) -- (5,0);
\draw (-3,0) node{(};
\draw (-3,0) node[below]{$a-\varepsilon$};
\draw (3,0) node{)};
\draw (3,0) node[below]{$a+\varepsilon$};
\draw (0,0.1) -- (0,-0.1);
\draw (0,0) node[below] {$a$};
\draw [red, thick] (-3,0)--(3,0);
\draw [red] (-1,0) node [above] {$U_{\varepsilon}(a)$};
\end{tikzpicture}\\
$\left( \underset{n\rightarrow \infty}{lim}a_n=a\right)\equiv \left( \forall\varepsilon>0 \; \exists n_0(\varepsilon) \; \forall n\geq n_0(\varepsilon) \; a_n \in U_{\varepsilon}(a)\right)$\\
d.h. für jedes (noch so kleine) $\varepsilon$, liegen ab einem bestimmten (von $\varepsilon$ abhängigen) Index $n_0(\varepsilon)$ alle Glieder $a_n(n\geq n_0(\varepsilon))$ in $U_{\varepsilon}(a)$.
\item Im Bsp. 1 sind:\\
konvergente Folgen: \\
c.) mit $\underset{n\rightarrow\infty}{a_n}=\frac{1}{3}$\\
d.) mit $\underset{n\rightarrow\infty}{a_n}=1$\\
divergente Folgen: a.) und b.)
\item Ist $\underset{n\rightarrow\infty}{a_n}=0$, so heißt $(a_n)$ \emph{Nullfolge}.
\end{enumerate}
\paragraph{Def. 3:} \parskp
$(a_n)$ heißt:
\begin{itemize}
\item \emph{streng monoton wachsend}, falls für jedes $n$ gilt: $a_n<a_{n+1}$.
\item \emph{monoton wachsend}, falls für jedes $n$ gilt: $a_n\leq a_{n+1}$.
\item \emph{streng monoton fallend}, falls für jedes $n$ gilt: $a_n>a_{n+1}$.
\item \emph{monoton fallend}, falls für jedes $n$ gilt: $a_n\geq a_{n+1}$.
\end{itemize}
\paragraph{Def. 4:} \parskp
$(a_n)$ heißt beschränkt, wenn es eine Konstante $C>0$ gibt mit $|a_n|\leq C$ für alle $n$.






\end{document} 