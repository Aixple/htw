\section{Microsoft .NET Platform}
\lecdate{24.03.2017}

\begin{center}
\href{https://worldcompressed.blogspot.com/2016/06/download-microsoft-net-framework-all.html}{ \includegraphics[scale=.3]{Vorlesung/a1Introduction-img001.png} }
\end{center}


\begin{itemize}
\item 2002 eingeführt
\item Hauptprogrammierplattform für Windows
\end{itemize}

\subsection{.NET Versionen und Komponenten}
\begin{center}
\href{https://de.wikipedia.org/wiki/.NET_Framework}{ \includegraphics[scale=.6]{Vorlesung/a1Introduction-img002.png} }
\includegraphics[width=.7\textwidth]{Vorlesung/a1Introduction-img003.png} 
\end{center}

\subsection{Features}
\begin{itemize}
\item Komponentenbasiert: \\
Unterschiedliche Komponenten aus unterschiedlichen Programmiersprachen:
\begin{itemize}
\item[]
\begin{itemize}
\item C\#, Visual Basic, F\#, JavaScript, C++ (in Visual Studio)
\item Ruby, Python, COBOL, Pascal (externe Compiler)
\item \href{https://en.wikipedia.org/wiki/List_of_CLI_languages}{Liste von CLI Sprachen}
\end{itemize}
\item Vererbung, Exception Handling, Debugging (über mehrere Programmiersprachen hinweg)
\end{itemize}
\item Versionskompatibilität: mehrere Versionen der gleichen .dll können nebeneinander existieren
\item .NET Distributionen:
\begin{itemize}
\item Microsoft, included in Visual Studio
\item Andere Distributionen (\href{http://www.mono-project.com}{www.mono-project.com}, \href{http://www.dotgnu.org}{www.dotgnu.org}) stellen Compiler und Runtime Environment für Linux Android, IOS, … 
\item .NET Core (plattformunabhängig, schlank)
\end{itemize}
\item Erstellte Anwendungen sind Plattform-unabhängig, und laufen unter verschiedenen Betriebssystemen (Windows, MacOS, Linux, Android, … ) 
\end{itemize}

\paragraph{Warum mehrere Programmiersprachen?}

\begin{itemize}
\item Nutzung von Vorteilen der unterschiedlichen Sprachen
\end{itemize}

\paragraph{Wie tauschen sich die unterschiedlichen Programmiersprachen untereinander aus?}

\begin{itemize}
\item über CLI
\end{itemize}

\subsection{Compilierung in .NET}

\begin{itemize}
\item Erzeugt *.dll Bibliotheken und *.exe Dateien $\to$ heißen Assembly
\item Diese Dateien haben bis auf ihre Endung nichts mit normalen .exe, .dll Dateien zu tun
\item Dateien enthalten Code in einer Intermediate Language (IL) + Metadaten + Manifest
\item Enthält keine Plattform-spezifischen Informationen $\to$ plattformunabhängig
\item Metadaten beschreiben Typen, bspw. von wem eine Klasse erbt, welche Methoden sie hat, Version, …
\end{itemize}
\begin{center}
\includegraphics[width=.7\textwidth]{Vorlesung/a1Introduction-img004.png} 
\end{center}

\subsubsection{C\# Code}

\begin{lstlisting}[language={[Sharp]C}]
class Calc{
	public int Add(int x, int y){
		return x + y;
	}
}
\end{lstlisting}
%\begin{center}
%\includegraphics[scale=.5]{Vorlesung/a1Introduction-img005.png}
%\end{center}

~

\subsubsection{Visual Basic Code}

\begin{lstlisting}[language={[Visual]Basic}]
Class Calc
	Public Function Add(ByVal x As Integer, ByVal y As Integer) As Integer
		Return x + y
	End Function
End Class
\end{lstlisting}

%\begin{center}
%\includegraphics{Vorlesung/a1Introduction-img006.png} 
%\end{center}

\subsubsection{Entsprechender IL Code}

\url{https://en.wikipedia.org/wiki/List_of_CIL_instructions}

\begin{lstlisting}[language=]
.method public hidebysig instance int32 Add(int32 x, int32 y) cil managed{
// Code size 9 (0x9)
.maxstack 2
.lacals init (int32 V_0)
IL_0000:	nop
IL_0001:	ldarg.1
IL_0002:	ldarg.2
IL_0003:	add
IL_0004:	stloc.0
IL_0005:	br.s IL_0007
IL_0007:	ldloc.0
IL_0008:	ret
}	//end of method Calc::Add
\end{lstlisting}

%\begin{center}
%\includegraphics[scale=.5]{Vorlesung/a1Introduction-img007.png}
%\end{center} 

\subsection{Ausführung des Programms}

\begin{itemize}
\item ruft just-in time compiler (JIT, jitter) auf der den code compiliert und ausführt
\item Methoden werden nur einmal kompiliert und im Speicher gelagert, ($\to$ kein Interpreter) so dass mehrfacher Aufruf der gleichen Methode nur eine einzige Kompilierung benötigt. 
\end{itemize}
~

\subsection{Teile von .NET}
\begin{itemize}
\item Common Language Runtime (CLR)
\begin{itemize}
\item Findet, lädt und managed .Net Objekte
\item Nutzt Metadaten der Assembly um Typen zu finden
\item Memory management, thread coordination, …
\item Compiliert nötigen IL Code und führt ihn aus
\end{itemize}
\item Common Type System (CTS)
\begin{itemize}
\item Spezifiziert Datentypen und Programmkonstrukte und deren Interaktion
\end{itemize}
\item Common Language Specification (CLS)
\begin{itemize}
\item Definiert eine Teilmenge von CTS, die über alle .NET Sprachen geteilt ist
\item Denn nicht jede .NET Sprache unterstützt alle features und Datentypen der CTS
\item Programmierung nur mit CLS features erzeugt Code Bibliotheken die garantiert mit jeder anderen .NET Sprache kommunizieren können
\item Festgelegt in Regeln
\begin{itemize}
\item Erste Regel: Alle Regeln beziehen sich nur auf die Teile eines Typs, der außerhalb eines Assemblys sichtbar sind
\end{itemize}
\item C\# Compiler kann prüfen, ob die Regeln eingehalten wurden, also ob der Code der CLS entspricht
\end{itemize}
\item Base Class Library
\begin{itemize}
\item Existiert für jede .NET Sprache
\item Beinhaltet Behandlung von I/O, Threads, Grafik, Hardware-Schnittstellen
\end{itemize}
\end{itemize}

\begin{center}
\includegraphics[width=.9\textwidth]{Vorlesung/a1Introduction-img008.png}
\end{center} 


\paragraph{Wie behält man die Übersicht bei vielen verschiedenen Klassen?}
$\to$ Namespaces und/oder Vererbung

\subsection{Namespaces}
\begin{itemize}
\item Zur Organisation aller Typen der .NET Platform über alle Sprachen hinweg
\item Namespaces können subspaces (Unterräume) enthalten
\begin{itemize}
\item System, System.IO, … 
\end{itemize}
\item Jedes Assembly kann beliebig viele Namespaces enthalten 
\item Bsp: System namespace enthält System.Int32, System.String, …
\item Zugriff auf Typen durch
\begin{itemize}
\item \lstinline$Namespace.Subnamespace.Subnamespace.Typname$
\item using keyword: \lstinline$using Namespace.Subnamespace.Subnamespace Typname$
\end{itemize}
\end{itemize}

\section{Die Sprache C\#}
%\href{http://www.pskills.org/csharp.jsp}{\includegraphics{Vorlesung/a1Introduction-img009.png}}
\begin{itemize}
\item Syntax sehr ähnlich zu Java
\item Vereint Konzepte von Java, Visual Basic, C++
\item Idee: die Vorteile aller Sprachen vereinigen
\begin{itemize}
\item Syntaktisch sauber wie Java
\item Einfach wie Visual Basic
\item Flexibel und Mächtig wie C++
\end{itemize}
\end{itemize}

\subsection{Features}
\begin{itemize}
\item Keine Pointer nötig
\item Automatisches Speichermanagement (Garbage Collector)
\item Operatorüberladung
\item Attribut-basierte Programmierung
\item Generische Programmierung
\item Lambda Operatoren
\item Einfaches Multithreading mit Async/Await Keywords
\end{itemize}

\begin{center}
\href{https://visualstudiomagazine.com/articles/salary-surveys/salary-survey.aspx}{ \includegraphics[scale=.4]{Vorlesung/a1Introduction-img010.jpg}}
\end{center}

%Aus \url{https://www.google.de/search?q=c%23+statistics&source=lnms&tbm=isch&sa=X&ved=0ahUKEwjtiJ30ocnSAhWsIsAKHRvXAKAQ_AUIBigB&biw=1920&bih=918}

~

\subsubsection*{Most popular coding languages}

\begin{center}
\href{https://blogs.msdn.microsoft.com/zxue/2016/10/24/how-many-developers-use-c-vs-c-vs-other-programming-languages/}{\includegraphics[width=.7\textwidth]{Vorlesung/a1Introduction-img011.png}}
\end{center}

%Aus \url{https://www.google.de/search?q=c%23+f%C3%BCr+windows+anwendungen+statistik&source=lnms&tbm=isch&sa=X&ved=0ahUKEwjsr9fgusnSAhXCFsAKHRSBC1wQ_AUICSgC&biw=1920&bih=918} 

\subsection{.NET Core}
.NET Core ist eine \href{https://de.wikipedia.org/wiki/Free/Libre_Open_Source_Software}{freie und quelloffene }\href{https://de.wikipedia.org/wiki/Plattform_(Computer)}{Software-Plattform} (innerhalb) der \href{https://de.wikipedia.org/wiki/.NET}{.NET}{}-Plattform
\begin{itemize}
\item 2016 veröffentlicht
\item Plattform-unabhängig!
\item Abgespeckte Version von .NET
\item Fast alles was wir hier lernen kann direkt in .NET Core verwendet werden
\end{itemize}

\section{Gliederung des Kurses}
\begin{itemize}
\item Grundlagen C\#
\item Vererbung, Polymorphie, Schnittstellen
\item Generische Programmierung
\item Fehlerbehandlung Exceptions
\item Delegates und Ereignisse
\item Windows Forms
\item Parallele Programmierung
\item Test-basierte Entwicklung
\item Agile / emergente Entwicklung
\end{itemize}

%\href{http://www.enterprise-application-development.org/research/partners/companies.html}{ \includegraphics{Vorlesung/a1Introduction-img012.jpg} }
\endinput
