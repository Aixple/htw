\newcommand{\customDir}{../}
\RequirePackage{ifthen,xifthen}

% Input inkl. Umlaute, Silbentrennung
\RequirePackage[T1]{fontenc}
\RequirePackage[utf8]{inputenc}

% Arbeitsordner (in Abhängigkeit vom Master) Standard: .LateX_master Ordner liegt im Eltern-Ordner
\providecommand{\customDir}{../}
\newcommand{\setCustomDir}[1]{\renewcommand{\customDir}{#1}}
%%% alle Optionen:
% Doppelseitig (mit Rand an der Innenseite)
\newboolean{twosided}
\setboolean{twosided}{false}
% Eigene Dokument-Klasse (alle KOMA möglich; cheatsheet für Spicker [3 Spalten pro Seite, alles kleiner])
\newcommand{\customDocumentClass}{scrreprt}
\newcommand{\setCustomDocumentClass}[1]{\renewcommand{\customDocumentClass}{#1}}
% Unterscheidung verschiedener Designs: htw, fjs
\newcommand{\customDesign}{htw}
\newcommand{\setCustomDesign}[1]{\renewcommand{\customDesign}{#1}}
% Dokumenten Metadaten
\newcommand{\customTitle}{}
\newcommand{\setCustomTitle}[1]{\renewcommand{\customTitle}{#1}}
\newcommand{\customSubtitle}{}
\newcommand{\setCustomSubtitle}[1]{\renewcommand{\customSubtitle}{#1}}
\newcommand{\customAuthor}{}
\newcommand{\setCustomAuthor}[1]{\renewcommand{\customAuthor}{#1}}
%	Notiz auf der Titelseite (A: vor Autor, B: nach Autor)
\newcommand{\customNoteA}{}
\newcommand{\setCustomNoteA}[1]{\renewcommand{\customNoteA}{#1}}
\newcommand{\customNoteB}{}
\newcommand{\setCustomNoteB}[1]{\renewcommand{\customNoteB}{#1}}
% Format der Signatur in Fußzeile:
\newcommand{\customSignature}{\ifthenelse{\equal{\customAuthor}{}} {} {\footnotesize{\textcolor{darkgray}{Mitschrift von\\ \customAuthor}}}}
\newcommand{\setCustomSignature}[1]{\renewcommand{\customSignature}{#1}}
% Format des Autors auf dem Titelblatt:
\newcommand{\customTitleAuthor}{\textcolor{darkgray}{Mitschrift von \customAuthor}}
\newcommand{\setCustomTitleAuthor}[1]{\renewcommand{\customTitleAuthor}{#1}}
% Standard Sprache
\newcommand{\customDefaultLanguage}[1]{}
\newcommand{\setCustomDefaultLanguage}[1]{\renewcommand{\customDefaultLanguage}{#1}}
% Folien-Pfad (inkl. Dateiname ohne Endung und ggf. ohne Nummerierung)
\newcommand{\customSlidePath}{}
\newcommand{\setCustomSlidePath}[1]{\renewcommand{\customSlidePath}{#1}}
% Folien Eigenschaften
\newcommand{\customSlideScale}{0.5}
\newcommand{\setCustomSlideScale}[1]{\renewcommand{\customSlideScale}{#1}}
\newcommand{\customSlideHeight}{9.63cm}
\newcommand{\setCustomSlideHeight}[1]{\renewcommand{\customSlideHeight}{#1}}
\newcommand{\customSlideWidth}{12.8cm}
\newcommand{\setCustomSlideWidth}[1]{\renewcommand{\customSlideWidth}{#1}}

%\setboolean{twosided}{true}
%\setCustomDocumentClass{scrartcl}
%\setCustomDesign{htw}
\setCustomSlidePath{Vorlesung/oneperpage/DBII_Kap}
\setCustomSlideScale{1}

\setCustomTitle{Datenbanksysteme II}
\setCustomSubtitle{Vorlesungsskript}
\setCustomAuthor{Falk-Jonatan Strube}
%\setCustomNoteA{TitlepageNoteBeforeAuthor}
\setCustomNoteB{Vorlesung von Prof. Dr.-Ing. Toll}

%\setcustomSignature{\footnotesize{\textcolor{darkgray}{Mitschrift von\\ \customAuthor}}	% Formatierung der Signatur in der Fußzeile
%\setcustomTitleAuthor{\textcolor{darkgray}{Mitschrift von #1}}	% Formatierung des Autors auf dem Titelblatt

%-- Prüfen, ob Beamer
\ifthenelse{\equal{\customDocumentClass}{beamer}}{
%%% TODO: andere Layouts für Beamer außer HTW
	\documentclass[ignorenonframetext, 11pt, table]{beamer}
	
	\usenavigationsymbolstemplate{}
	\setbeamercolor{author in head/foot}{fg=black}
	\setbeamercolor{title}{fg=black}
	\setbeamercolor{bibliography entry author}{fg=htworange!70}
	%\setbeamercolor{bibliography entry title}{fg=blue} 
	\setbeamercolor{bibliography entry location}{fg=htworange!60} 
	\setbeamercolor{bibliography entry note}{fg=htworange!60}  
	
	\setbeamertemplate{itemize item}{\color{black}$\bullet$}
	\setbeamertemplate{itemize subitem}{\color{black}--}
	\setbeamertemplate{itemize subsubitem}{\color{black}$\bullet$}
	\makeatother
	\setbeamertemplate{footline}
	{
	\leavevmode
	\def\arraystretch{1.2}
	\arrayrulecolor{gray}
	\begin{tabular}{ p{0.167\textwidth} | p{0.491\textwidth} | p{0.089\textwidth} | p{0.103\textwidth}}
	\hline
	\strut\insertshortauthor & \insertshorttitle & Slide \insertframenumber{}% / \inserttotalframenumber{}
	 & May 4, 2016\\
	\end{tabular}
	}
	\setbeamertemplate{headline}
	{
	\leavevmode
	\setlength{\arrayrulewidth}{1pt}
	\hspace*{2em}	
	\begin{tabular}{p{0.63\textwidth}}
	\rule{0pt}{3em}\normalsize{\textbf{\insertsection\strut}}\\
	\arrayrulecolor{htworange}
	\hline
	\end{tabular}
	\begin{tabular}{l}
	\rule{0pt}{4em}\includegraphics[width=3.25cm]{\customDir .LaTeX_master/HTW_GESAMTLOGO_CMYK.eps}\\
	\end{tabular}
	}
	\makeatletter	
}{	
	%-- Für Spicker einiges anders:
	\ifthenelse{\equal{\customDocumentClass}{cheatsheet}}{
		\documentclass[a4paper,10pt,landscape]{scrartcl}
		\usepackage{geometry}
		\geometry{top=2mm, bottom=2mm, headsep=0mm, footskip=0mm, left=2mm, right=2mm}
		
		% Für Spicker \spsection für Section, zur Strukturierung \HRule oder \HDRule Linie einsetzen
		\usepackage{multicol}
		\newcommand{\spsection}[1]{\textbf{#1}}	% Platzsparende "section" für Spicker
	}{	%-- Ende Spicker-Unterscheidung-if
		%-- Unterscheidung Doppelseitig
		\ifthenelse{\boolean{twosided}}{
			\documentclass[a4paper,11pt, footheight=26pt,twoside]{\customDocumentClass}
			\usepackage[head=23pt]{geometry}	% head=23pt umgeht Fehlerwarnung, dafür größeres "top" in geometry
			\geometry{top=30mm, bottom=22mm, headsep=10mm, footskip=12mm, inner=27mm, outer=13mm}
		}{
			\documentclass[a4paper,11pt, footheight=26pt]{\customDocumentClass}
			\usepackage[head=23pt]{geometry}	% head=23pt umgeht Fehlerwarnung, dafür größeres "top" in geometry
			\geometry{top=30mm, bottom=22mm, headsep=10mm, footskip=12mm, left=20mm, right=20mm}
		}
		%-- Nummerierung bis Subsubsection für Report
		\ifthenelse{\equal{\customDocumentClass}{report} \OR \equal{\customDocumentClass}{scrreprt}}{
			\setcounter{secnumdepth}{3}	% zählt auch subsubsection
			\setcounter{tocdepth}{3}	% Inhaltsverzeichnis bis in subsubsection
		}{}
	}%-- Ende Spicker-Unterscheidung-else
	
	\usepackage{scrlayer-scrpage}	% Kopf-/Fußzeile
	\renewcommand*{\thefootnote}{\fnsymbol{footnote}}	% Fußnoten-Symbole anstatt Zahlen
	\renewcommand*{\titlepagestyle}{empty} % Keine Seitennummer auf Titelseite
	\usepackage[perpage]{footmisc}	% Fußnotenzählung Seitenweit, nicht Dokumentenweit
}

% Input inkl. Umlaute, Silbentrennung
\RequirePackage[T1]{fontenc}
\RequirePackage[utf8]{inputenc}
\usepackage[english,ngerman]{babel}
\usepackage{csquotes}	% Anführungszeichen
\RequirePackage{marvosym}
\usepackage{eurosym}

% Style-Aufhübschung
\usepackage{soul, color}	% Kapitälchen, Unterstrichen, Durchgestrichen usw. im Text
%\usepackage{titleref}

% Mathe usw.
\usepackage{amssymb}
\usepackage{amsthm}
\ifthenelse{\equal{\customDocumentClass}{beamer}}{}{
\usepackage[fleqn,intlimits]{amsmath}	% fleqn: align-Umgebung rechtsbündig; intlimits: Integralgrenzen immer ober-/unterhalb
}
%\usepackage{mathtools} % u.a. schönere underbraces
\usepackage{xcolor}
\usepackage{esint}	% Schönere Integrale, \oiint vorhanden
\everymath=\expandafter{\the\everymath\displaystyle}	% Mathe Inhalte werden weniger verkleinert
\usepackage{wasysym}	% mehr Symbole, bspw \lightning
% Auch arcus-Hyperbolicus-Funktionen
\DeclareMathOperator{\arccot}{arccot}
\DeclareMathOperator{\arccosh}{arccosh}
\DeclareMathOperator{\arcsinh}{arcsinh}
\DeclareMathOperator{\arctanh}{arctanh}
\DeclareMathOperator{\arccoth}{arccoth} 
%\renewcommand{\int}{\int\limits}
%\usepackage{xfrac}	% mehr fracs: sfrac{}{}
\let\oldemptyset\emptyset	% schöneres emptyset
\let\emptyset\varnothing
%\RequirePackage{mathabx}	% mehr Symbole
\mathchardef\mhyphen="2D	% Hyphen in Math

% tikz usw.
\usepackage{tikz}
\usepackage{pgfplots}
\pgfplotsset{compat=1.11}	% Umgeht Fehlermeldung
\usetikzlibrary{graphs}
%\usetikzlibrary{through}	% ???
\usetikzlibrary{arrows}
\usetikzlibrary{arrows.meta}	% Pfeile verändern / vergrößern: \draw[-{>[scale=1.5]}] (-3,5) -> (-3,3);
\usetikzlibrary{automata,positioning} % Zeilenumbruch im Node node[align=center] {Text\\nächste Zeile} automata für Graphen
\usetikzlibrary{matrix}
\usetikzlibrary{patterns}	% Schraffierte Füllung
\usetikzlibrary{shapes.geometric}	% Polygon usw.
\tikzstyle{reverseclip}=[insert path={	% Inverser Clip \clip
	(current page.north east) --
	(current page.south east) --
	(current page.south west) --
	(current page.north west) --
	(current page.north east)}
% Nutzen: 
%\begin{tikzpicture}[remember picture]
%\begin{scope}
%\begin{pgfinterruptboundingbox}
%\draw [clip] DIE FLÄCHE, IN DER OBJEKT NICHT ERSCHEINEN SOLL [reverseclip];
%\end{pgfinterruptboundingbox}
%\draw DAS OBJEKT;
%\end{scope}
%\end{tikzpicture}
]	% Achtung: dafür muss doppelt kompliert werden!
\usepackage{graphpap}	% Grid für Graphen
\tikzset{every state/.style={inner sep=2pt, minimum size=2em}}
\usetikzlibrary{mindmap, backgrounds}
%\usepackage{tikz-uml}	% braucht Dateien: http://perso.ensta-paristech.fr/~kielbasi/tikzuml/

% Tabular
\usepackage{longtable}	% Große Tabellen über mehrere Seiten
\usepackage{multirow}	% Multirow/-column: \multirow{2[Anzahl der Zeilen]}{*[Format]}{Test[Inhalt]} oder \multicolumn{7[Anzahl der Reihen]}{|c|[Format]}{Test2[Inhalt]}
\renewcommand{\arraystretch}{1.3} % Tabellenlinien nicht zu dicht
\usepackage{colortbl}
\arrayrulecolor{gray}	% heller Tabellenlinien
\usepackage{array}	% für folgende 3 Zeilen (für Spalten fester breite mit entsprechender Ausrichtung):
\newcolumntype{L}[1]{>{\raggedright\let\newline\\\arraybackslash\hspace{0pt}}m{\dimexpr#1\columnwidth-2\tabcolsep-1.5\arrayrulewidth}}
\newcolumntype{C}[1]{>{\centering\let\newline\\\arraybackslash\hspace{0pt}}m{\dimexpr#1\columnwidth-2\tabcolsep-1.5\arrayrulewidth}}
\newcolumntype{R}[1]{>{\raggedleft\let\newline\\\arraybackslash\hspace{0pt}}m{\dimexpr#1\columnwidth-2\tabcolsep-1.5\arrayrulewidth}}
\usepackage{caption}	% Um auch unbeschriftete Captions mit \caption* zu machen

% Nützliches
\usepackage{verbatim}	% u.a. zum auskommentieren via \begin{comment} \end{comment}
\usepackage{tabto}	% Tabs: /tab zum nächsten Tab oder /tabto{.5 \CurrentLineWidth} zur Stelle in der Linie
\NumTabs{6}	% Anzahl von Tabs pro Zeile zum springen
\usepackage{listings} % Source-Code mit Tabs
\usepackage{lstautogobble} 
\ifthenelse{\equal{\customDocumentClass}{beamer}}{}{
\usepackage{enumitem}	% Anpassung der enumerates
%\setlist[enumerate,1]{label=(\arabic*)}	% global andere Enum-Items
\renewcommand{\labelitemiii}{$\scriptscriptstyle ^\blacklozenge$} % global andere 3. Item-Aufzählungszeichen
}
\newenvironment{anumerate}{\begin{enumerate}[label=(\alph*)]}{\end{enumerate}} % Alphabetische Aufzählung
\usepackage{letltxmacro} % neue Definiton von Grundbefehlen
% Nutzen:
%\LetLtxMacro{\oldemph}{\emph}
%\renewcommand{\emph}[1]{\oldemph{#1}}
\RequirePackage{xpatch}	% ua. Konkatenieren von Strings/Variablen (etoolbox)


% Einrichtung von lst
\lstset{
basicstyle=\ttfamily, 
%mathescape=true, 
%escapeinside=^^, 
autogobble, 
tabsize=2,
basicstyle=\footnotesize\sffamily\color{black},
frame=single,
rulecolor=\color{lightgray},
numbers=left,
numbersep=5pt,
numberstyle=\tiny\color{gray},
commentstyle=\color{gray},
keywordstyle=\color{green},
stringstyle=\color{orange},
morecomment=[l][\color{magenta}]{\#}
showspaces=false,
showstringspaces=false,
breaklines=true,
literate=%
    {Ö}{{\"O}}1
    {Ä}{{\"A}}1
    {Ü}{{\"U}}1
    {ß}{{\ss}}1
    {ü}{{\"u}}1
    {ä}{{\"a}}1
    {ö}{{\"o}}1
    {~}{{\textasciitilde}}1
}
\usepackage{scrhack} % Fehler umgehen
\def\ContinueLineNumber{\lstset{firstnumber=last}} % vor lstlisting. Zum wechsel zum nicht-kontinuierlichen muss wieder \StartLineAt1 eingegeben werden
\def\StartLineAt#1{\lstset{firstnumber=#1}} % vor lstlisting \StartLineAt30 eingeben, um bei Zeile 30 zu starten
\let\numberLineAt\StartLineAt

% BibTeX
\usepackage[backend=bibtex8, bibencoding=ascii,
%style=authortitle, citestyle=authortitle-ibid,
%doi=false,
%isbn=false,
%url=false
]{biblatex}	% BibTeX
\usepackage{makeidx}
%\makeglossary
%\makeindex

% Grafiken
\usepackage{graphicx}
\usepackage{epstopdf}	% eps-Vektorgrafiken einfügen
%\epstopdfsetup{outdir=\customDir}

% pdf-Setup
\usepackage{pdfpages}
\ifthenelse{\equal{\customDocumentClass}{beamer}}{}{
\usepackage[bookmarks,%
bookmarksopen=false,% Klappt die Bookmarks in Acrobat aus
colorlinks=true,%
linkcolor=black,%
citecolor=red,%
urlcolor=green,%
]{hyperref}
}

%-- Unterscheidung des Stils
\newcommand{\customLogo}{}
\newcommand{\customPreamble}{}
\ifthenelse{\equal{\customDesign}{htw}}{
	% HTW Corporate Design: Arial (Helvetica)
	\usepackage{helvet}
	\renewcommand{\familydefault}{\sfdefault}
	\renewcommand{\customLogo}{HTW-Logo}
	\renewcommand{\customPreamble}{HTW Dresden}
}{
% \renewcommand{\customLogo}{HTW-Logo.eps}
}

% Nach Dokumentenbeginn ausführen:
\AtBeginDocument{
	% Autor und Titel für pdf-Eigenschaften festlegen, falls noch nicht geschehen
	\providecommand{\pdfAuthor}{John Doe}
	\ifdefempty{\customAuthor} {} {\renewcommand{\pdfAuthor}{\customAuthor}}
	\providecommand{\pdfTitle}{}
	\providecommand{\pdfTitleA}{}
	\providecommand{\pdfTitleB}{}
	\providecommand{\pdfTitleC}{}	
	\ifdefempty{\pdfTitle}{
		\ifdefempty{\customPreamble} {} {\renewcommand{\pdfTitleA}{\customPreamble{} | }}
		\ifdefempty{\customTitle} {\renewcommand{\pdfTitleB}{No Title}} {\renewcommand{\pdfTitleB}{\customTitle}}
		\ifdefempty{\customSubtitle} {} {\renewcommand{\pdfTitleC}{ - \customSubtitle}}
	}{}
	
	\newcommand{\customLogoLocation}{\customDir .LaTeX_master/\customLogo}
	\hypersetup{
		pdfauthor={\pdfAuthor},
		pdftitle={\pdfTitleA\pdfTitleB\pdfTitleC},
	}
	\ifthenelse{\equal{\customDocumentClass}{beamer}}{
		\title{\customTitle}
		\author{\customAuthor}
	}{
		\automark[section]{section}
		\automark*[subsection]{subsection}
		\pagestyle{scrheadings}
		\ifthenelse{\equal{\customDocumentClass}{report} \OR \equal{\customDocumentClass}{scrreprt}}{
		\renewcommand*{\chapterpagestyle}{scrheadings}
		}{}
		%\renewcommand*{\titlepagestyle}{scrheadings}
		\ihead{\includegraphics[height=1.7em]{\customLogoLocation}}
		%\ohead{\truncate{5cm}{\customTitle}}
		\ohead{\customTitle}
		\cfoot{\pagemark}
		\ofoot{\customSignature}
		% Titelseite
		\title{
		\includegraphics[width=0.35\textwidth]{\customDir .LaTeX_master/\customLogo}\\\vspace{0.5em}
		\Huge\textbf{\customTitle}
		\ifdefempty{\customSubtitle} {} {\\\vspace*{0.7em}\Large \customSubtitle}
		\\\vspace*{5em}}
		\author{
		\ifdefempty{\customNoteA} {} {\customNoteA \vspace*{1em}}\\ 
		\ifdefempty{\customAuthor} {} {\customTitleAuthor}
		\ifdefempty{\customNoteB}{}{\vspace*{1em}\\\customNoteB}
		}
		
		\ifthenelse{\equal{\customDocumentClass}{cheatsheet}}{
			\pagestyle{empty}
			\setlist{nolistsep}
	%		\usepackage{parskip}	% Aufzählung Abstand
	%		\setlength{\parskip}{0em}
			\lstset{
	    belowcaptionskip=0pt,
	    belowskip=0pt,
	    aboveskip=0pt,
			tabsize=2,
			frame=none,
			numbers=none,
			showspaces=false,
			showstringspaces=false,
			breaklines=true,
			}
		}{}
	}
}

% Unterabschnitte
%\newtheorem{example}{Beispiel}%[section]
%\newtheorem{definition}{Definition}[section]
%\newtheorem{discussion}{Diskussion}[section]
%\newtheorem{remark}{Bemerkung}[section]
%\newtheorem{proof}{Beweis}[section]
%\newtheorem{notation}{Schreibweise}[section]
\RequirePackage{xcolor}
\RequirePackage{amsmath}

% Horizontale Linie:
\newcommand{\HRule}[1][\medskipamount]{\par
  \vspace*{\dimexpr-\parskip-\baselineskip+#1}
  \noindent\rule[0.2ex]{\linewidth}{0.2mm}\par
  \vspace*{\dimexpr-\parskip-.5\baselineskip+#1}}
% Gestrichelte horizontale Linie:
\RequirePackage{dashrule}
\newcommand{\HDRule}[1][\medskipamount]{\par
  \vspace*{\dimexpr-\parskip-\baselineskip+#1}
  \noindent\hdashrule[0.2ex]{\linewidth}{0.2mm}{1mm} \par
  \vspace*{\dimexpr-\parskip-.5\baselineskip+#1}}
% Mathe in Anführungszeichen:
\newsavebox{\mathbox}\newsavebox{\mathquote}
\makeatletter
\newcommand{\mq}[1]{% \mathquotes{<stuff>}
  \savebox{\mathquote}{\text{"}}% Save quotes
  \savebox{\mathbox}{$\displaystyle #1$}% Save <stuff>
  \raisebox{\dimexpr\ht\mathbox-\ht\mathquote\relax}{"}#1\raisebox{\dimexpr\ht\mathbox-\ht\mathquote\relax}{''}
}
\makeatother

% Paragraph mit Zähler (Section-Weise)
\newcounter{cparagraphC}
\newcommand{\cparagraph}[1]{
\stepcounter{cparagraphC}
\paragraph{\thesection{}-\thecparagraphC{} #1}
%\addcontentsline{toc}{subsubsection}{\thesection{}-\thecparagraphC{} #1}
\label{\thesection-\thecparagraphC}
}
\makeatletter
\@addtoreset{cparagraphC}{section}
\makeatother


% (Vorlesungs-)Folien einbinden:
% Folien von einer Datei skaliert
\newcommand{\slide}[2][\customSlideScale]{\slides[#1]{}{#2}}
\newcommand{\slideTrim}[6][\customSlideScale]{\slides[#1 , clip,  trim = #5cm #4cm #6cm #3cm]{}{#2}}
% Folien von mehreren nummerierten Dateien skaliert
\newcommand{\slides}[3][\customSlideScale]{\begin{center}
\includegraphics[page=#3, scale=#1]{\customSlidePath #2.pdf}
\end{center}}

% \emph{} anders definieren
\makeatletter
\DeclareRobustCommand{\em}{%
  \@nomath\em \if b\expandafter\@car\f@series\@nil
  \normalfont \else \scshape \fi}
\makeatother

% unwichtiges
\newcommand{\unimptnt}[1]{{\transparent{0.5}#1}}

% alph. enumerate
\newenvironment{anumerate}{\begin{enumerate}[label=(\alph*)]}{\end{enumerate}} % Alphabetische Aufzählung

%% EINFACHE BEFEHLE

% Abkürzungen Mathe
\newcommand{\EE}{\mathbb{E}}
\newcommand{\QQ}{\mathbb{Q}}
\newcommand{\RR}{\mathbb{R}}
\newcommand{\CC}{\mathbb{C}}
\newcommand{\NN}{\mathbb{N}}
\newcommand{\ZZ}{\mathbb{Z}}
\newcommand{\PP}{\mathbb{P}}
\renewcommand{\SS}{\mathbb{S}}
\newcommand{\cA}{\mathcal{A}}
\newcommand{\cB}{\mathcal{B}}
\newcommand{\cC}{\mathcal{C}}
\newcommand{\cD}{\mathcal{D}}
\newcommand{\cE}{\mathcal{E}}
\newcommand{\cF}{\mathcal{F}}
\newcommand{\cG}{\mathcal{G}}
\newcommand{\cH}{\mathcal{H}}
\newcommand{\cI}{\mathcal{I}}
\newcommand{\cJ}{\mathcal{J}}
\newcommand{\cM}{\mathcal{M}}
\newcommand{\cN}{\mathcal{N}}
\newcommand{\cP}{\mathcal{P}}
\newcommand{\cR}{\mathcal{R}}
\newcommand{\cS}{\mathcal{S}}
\newcommand{\cZ}{\mathcal{Z}}
\newcommand{\cL}{\mathcal{L}}
\newcommand{\cT}{\mathcal{T}}
\newcommand{\cU}{\mathcal{U}}
\newcommand{\cV}{\mathcal{V}}
\renewcommand{\phi}{\varphi}
\renewcommand{\epsilon}{\varepsilon}

% Verschiedene als Mathe-Operatoren
\DeclareMathOperator{\arccot}{arccot}
\DeclareMathOperator{\arccosh}{arccosh}
\DeclareMathOperator{\arcsinh}{arcsinh}
\DeclareMathOperator{\arctanh}{arctanh}
\DeclareMathOperator{\arccoth}{arccoth} 
\DeclareMathOperator{\var}{Var} % Varianz 
\DeclareMathOperator{\cov}{Cov} % Co-Varianz 

% Farbdefinitionen
\definecolor{red}{RGB}{180,0,0}
\definecolor{green}{RGB}{75,160,0}
\definecolor{blue}{RGB}{0,75,200}
\definecolor{orange}{RGB}{255,128,0}
\definecolor{yellow}{RGB}{255,245,0}
\definecolor{purple}{RGB}{75,0,160}
\definecolor{cyan}{RGB}{0,160,160}
\definecolor{brown}{RGB}{120,60,10}

\definecolor{itteny}{RGB}{244,229,0}
\definecolor{ittenyo}{RGB}{253,198,11}
\definecolor{itteno}{RGB}{241,142,28}
\definecolor{ittenor}{RGB}{234,98,31}
\definecolor{ittenr}{RGB}{227,35,34}
\definecolor{ittenrp}{RGB}{196,3,125}
\definecolor{ittenp}{RGB}{109,57,139}
\definecolor{ittenpb}{RGB}{68,78,153}
\definecolor{ittenb}{RGB}{42,113,176}
\definecolor{ittenbg}{RGB}{6,150,187}
\definecolor{itteng}{RGB}{0,142,91}
\definecolor{ittengy}{RGB}{140,187,38}

\definecolor{htworange}{RGB}{249,155,28}

% Textfarbe ändern
\newcommand{\tred}[1]{\textcolor{red}{#1}}
\newcommand{\tgreen}[1]{\textcolor{green}{#1}}
\newcommand{\tblue}[1]{\textcolor{blue}{#1}}
\newcommand{\torange}[1]{\textcolor{orange}{#1}}
\newcommand{\tyellow}[1]{\textcolor{yellow}{#1}}
\newcommand{\tpurple}[1]{\textcolor{purple}{#1}}
\newcommand{\tcyan}[1]{\textcolor{cyan}{#1}}
\newcommand{\tbrown}[1]{\textcolor{brown}{#1}}

% Umstellen der Tabellen Definition
\newcommand{\mpb}[1][.3]{\begin{minipage}{#1\textwidth}\vspace*{3pt}}
\newcommand{\mpe}{\vspace*{3pt}\end{minipage}}

\newcommand{\resultul}[1]{\underline{\underline{#1}}}
\newcommand{\parskp}{$ $\\}	% new line after paragraph
\newcommand{\corr}{\;\widehat{=}\;}
\newcommand{\mdeg}{^{\circ}}

\newcommand{\nok}[2]{\binom{#1}{#2}}	% n über k BESSER: \binom{n}{k}
\newcommand{\mtr}[1]{\begin{pmatrix}#1\end{pmatrix}}	% Matrix
\newcommand{\dtr}[1]{\begin{vmatrix}#1\end{vmatrix}}	% Determinante (Betragsmatrix)
\renewcommand{\vec}[1]{\underline{#1}}	% Vektorschreibweise
\newcommand{\imptnt}[1]{\colorbox{red!30}{#1}}	% Wichtiges
\newcommand{\intd}[1]{\,\mathrm{d}#1}
\newcommand{\diffd}[1]{\mathrm{d}#1}
% für Module-Rechnung: \pmod{}
\newcommand{\unit}[1]{\,\mathrm{#1}}

%\bibliography{\customDir _Literatur/HTW_Literatur.bib}

\begin{document}

%\selectlanguage{english}
\maketitle
\newpage
\tableofcontents
\newpage
\chapter*{Vorbemerkungen}
\section*{Prüfung}
SP: 90 min

\section*{PVL}
Beleg (Abnahme im Praktikum)
\begin{itemize}
\item MS SQL
\item Oracle
\end{itemize}

\chapter{Erweiterung von Datenbanksprachen}

\section{Übersicht der Datenbank-Schnittstellen}
Bsp.: Abfrage aller Namen von Kunden des Ortes Dresden\\
\begin{tabular}{ l | l l}
& Programm (3. Gl) z.B. C & SQL\\
\hline
Pseudocode & \mpb[0.4]
\begin{lstlisting}[language=C]
open (Kunden)
while (not end of file) {
	read (Kunde)
	if (Kunde.Ort == 'Dresden')
		print('Name ='+Kunde.Name)
}
close (Kunden)
\end{lstlisting}
\mpe& \mpb[0.34]
\begin{lstlisting}[language=SQL]
SELECT Name
FROM Kunde
WHERE Ort='Dresden'
\end{lstlisting}
\mpe\\
& prozedural (WIE?) & deskriptiv (WAS?)\\
\hline
Vorteile& \mpb[0.4]
\begin{itemize}[leftmargin=*]
\item math. Operationen gut möglich
\item Eingabe/Ausgabe gut möglich
\item Steuerung des Programmablaufs (Algorithmus)
\end{itemize}
\mpe & \mpb[0.34]
\begin{itemize}[leftmargin=*]
\item Zugriff über Datenmengen
\item Fkt. des DBMS:
\begin{itemize}[leftmargin=*]
\item Zugriffsschutz
\item Integritätssicherung
\item …
\end{itemize}
\end{itemize}
\mpe\\
\hline
Nachteil & \mpb[0.4]
\begin{itemize}[leftmargin=*]
\item Zugriff nur satzorientiert
\item Vorteile des DBMS kaum möglich
\end{itemize} 
\mpe & \mpb[0.34]
\begin{itemize}[leftmargin=*]
\item Vorteile des Programms kaum möglich
\end{itemize}
\mpe
\end{tabular}
\subsection{Probleme von Datenbank-Schnittstellen}
\slides{1}{1}
\subsection{Zielstellung der Erweiterungen}
\begin{itemize}
\item Verbindung der Möglichkeiten einer prozeduralen Sprache mit den Möglichkeiten von SQL
\item satzorientierte Arbeit mit Ergebnismengen aus SQL („liefert“ Menge)
\item Nutzung der Verarbeitungslogik
\end{itemize}
Lösung der Probleme…
\begin{itemize}
\item der Datenübertragung
\item der Anpassung der unterschiedlichen Verarbeitungslogiken (Wie $\Leftrightarrow$ Was)
\end{itemize}
über:
\subsection{Aufgaben von Schnittstellen zwischen Datenbank-Server und Client-Anwendung}
\slides{1}{2}
\paragraph{Bsp.:}
\begin{itemize}
\item Variablendeklaration in C
\item $\vdots$
\item SQLCHAR myName[21];
\item $\vdots$
\item // Abfrage: SELECT Name FROM Kunde
\item $\vdots$
\item // Binden: Spalte1 an Variable myName
\item SQLBindCol(…,1,SQL\_ C\_ CHAR,myName,…);
\end{itemize}
Die SQL-Verarbeitung geschieht über ODBC
\subsection{Auswahl an Schnittstellen zwischen Datenbank-Server und Client-Anwendung}
\slides{1}{3}
\subsection{Arten an Verbindungen zur Datenbank}
\begin{itemize}
\item \emph{eigenständig} (Client des DBMS):\\
MS SQL Management Studio, Oracle Developer, …
\item eingebundene
\begin{itemize}
\item eingebettet in Wirtssprache (HLI: host level interface) \\
(Erweiterung des Sprachumfangs um SQL-Befehle + Precompiler):\\
eSQL in C, eSQL in Java, …
\item CLI (call level interface)\\
(Erweiterung der Sprache um Funktionen zum Zugriff auf DBMS):\\
ODBC, JDBC, …
\item objektorientiert\\
(Sprachumfang um Klassen erweitern, OR-Mapping):\\
ASP.NET, …
\end{itemize}
\end{itemize}

Bsp. Ablauf eSQL:
\begin{itemize}
\item Editor:\\
bsp.cpp	(eSQL-Befehle)
\item Precompiler:\\
bsp.c (Funktionsaufrufe an das DBMS)
\item C-Compiler (lib/obj/…)\\
bsp.out\\
$\Rightarrow$ ausführbar
\end{itemize}

\subsection{Architekturverfahren für Datenbank-Schnittstellen}
\slides{1}{4}

\section{Anweisungen von SQL in anderen Sprachen (ODBC)}
\subsection{Call-Level Interface (CLI)}
\slides{1}{5}
\subsection{Call-Level Interface Dialekte}
\slides{1}{6}
\subsection{Architektur der ODBC-Schnittstelle}
\slides{1}{7}
\subsection{Merkmale von ODBC}
\slides{1}{8}
\subsection{Aufbau einer Datenbank-Verbindung über ODBC}
\slides{1}{9}
Bsp. Aufbau:
\begin{lstlisting}[language=C]
// Headerdatei sqltypes.h
// ...
typedef void* SQLHANDLE;
typedef SQLHANDLE SQLHENN;
typedef SQLHANDLE SQLHDBC;
typedef SQLHANDLE SQLHSTMC;
// ...
\end{lstlisting}
$\Rightarrow$ Aufbau des C-Programms (Ablauf):
\begin{itemize}
\item Handle anlegen: (ODBC: SQLAllocHandle)
\begin{itemize}
\item Umgebung
\item DB-Verbindung
\item Abfrage
\end{itemize}
\item Umgebungsattribute setzen (ODBC: SQLSetEnvAttr)
\item Verbindungsattribute setzen (ODBC: SQLSetCommentAttr)
\item Vererbung zur DB aufbauen (ODBC: SQLDriverConnect / SQLConnect)
\item SQL-Abfrage an DB senden (ODBC: SQLExecDirect)
\item Spalten der Abfrage binden (ODBC: SQLBindCol)\\
(je Spalte im Abfrageergebnis eine Bindung)
\item Im Zyklus Datensatzweise lesen und ausgeben/verarbeiten (ODBC: SQLFetch)
\item Verbindung zur DB beenden (ODBC: SQLDisconnect)
\item Handle freigeben: (ODBC: SQLFreeHandle)
\begin{itemize}
\item Umgebung
\item DB-Verbindung
\item Abfrage
\end{itemize}
\end{itemize}
\subsection{Beschreibung der wichtigsten CLI-Funktionen in ODBC}
\setCustomSlideScale{0.8}
\slides{1B}{1}
\slides{1B}{2}
\setCustomSlideScale{1}
Achtung: mit konstanten Bezeichnungen anstatt mit numerischen Werten arbeiten:
\begin{lstlisting}[language=C]
// Headerdatei sql.h
// ...
#define SQL_HANDLE_ENV 1
#define SQL_HANDLE_DBC 2
#define SQL_HANDLE_STMT 3
\end{lstlisting}

\subsection{Minimalbeispiel ODBC}
\setCustomSlideScale{0.7}
\slides{1B}{3}
\slides{1B}{4}
\setCustomSlideScale{1}
Achtung: Variable muss ein Zeichen größer sein als SQL-Eintrag ($\to$ Nullterminiert)
\subsection{Programmausschnitt ODBC}
\slides{1}{10}

\subsection{Ergebnistypen ODBC}
\slides{1}{11}
\newpage\paragraph{Beispiel Datentypen}
\setCustomSlideScale{0.7}
\slides{1C}{1}
$\Rightarrow$ FETCH liest immer einen Datensatz (Zeilenweise)\\
Achtung: Zeilen retcode braucht als 4. Argument ein @…

\chapter{Datensicherheit/Zugriffsschutz}
\section{Übersicht Datensicherheit/Transaktionen}
\setCustomSlideScale{0.5}
\slides{2}{1}
\subsection{Datensicherheit in Datenbanksystemen}
\begin{tabular}{l | l | l}
& höhere Gewalt & Ausfall technischer Mittel\\
\hline 
Gefahrenbereich & \mpb \begin{itemize}[leftmargin=*]
\item Brand, Wasser
\item Stromausfall
\item Blitzschlag
\item Staub
\end{itemize}\mpe & \mpb \begin{itemize}[leftmargin=*]
\item HDD defekt
\item SW gelöscht
\end{itemize}\mpe \\
\hline 
Maßnahmen & \mpb \begin{itemize}[leftmargin=*]
\item Backup / Spiegelung
\item Brandschutzmaßnahmen
\item Maßnahmen nach Bundes Datenschutzgesetz (BDSG)
\end{itemize} \mpe & \mpb \begin{itemize}[leftmargin=*]
\item Backup / Spiegelung
\item sichere HW/SW
\end{itemize} \mpe
\end{tabular}\\
\begin{tabular}{l | l | l}
unbewusste menschliche Fehler & parallele Datennutzung & Computerkriminalität\\
\hline 
\mpb \begin{itemize}[leftmargin=*]
\item vergessene Handlung (Eintrag nicht komplett ausgefüllt)
\item versehentliches Löschen
\end{itemize} \mpe & \mpb \begin{itemize}[leftmargin=*]
\item gleichzeitiges schreiben\newline von Daten
\end{itemize}\mpe & \mpb \begin{itemize}[leftmargin=*]
\item Spam, Hacker
\item Viren, Trojaner
\end{itemize}\mpe \\
\hline
\mpb \begin{itemize}[leftmargin=*]
\item ReadOnly-Felder
\item Pflichfelder
\item Abhängigkeiten
\end{itemize} \mpe & \mpb \begin{itemize}[leftmargin=*]
\item interne Serialisierung (Warteschlange)
\end{itemize} \mpe & \mpb \begin{itemize}[leftmargin=*]
\item Maßnahmen Datenschutz
\item Kryptographie
\item sichere Passwörter
\item Zertifikate
\end{itemize} \mpe
\end{tabular}\\
Diese lassen sich in Kategorien einteilen:
\begin{itemize}
\item physische Integrität
\begin{itemize}
\item höhere Gewalt
\item Ausfall technischer Mittel
\end{itemize}
\item semantische Integrität
\begin{itemize}
\item unbewusste menschliche Fehler
\end{itemize}
\item operationale Integrität
\begin{itemize}
\item parallele Datennutzung
\end{itemize}
\item Zugriffsschutz
\begin{itemize}
\item Computerkriminalität
\end{itemize}
\end{itemize}
\subsubsection{Klassifikation der Gefahrenbereiche}
\slides{2}{2}
\paragraph{Schadensursachen} \parskp
\begin{tabular}{l l}
Bedienfehler & $25\%$\\
Blitzschlag & $17\%$\\
Kurzschluss & $16\%$\\
Diebstahl & $10\%$\\
Wasser & $8\%$\\
Feuer & $6\%$\\
Sturm & $1\%$\\
Sonstige & $17\%$
\end{tabular}
\subsubsection{Sicherheitskonzepte}
\slides{2}{3}
\subsubsection*{Beispiel einer Schlüsselsicherung mit Prüfziffer}
\slides{2}{4}
\subsubsection{Überblick zur Integritätssicherung}
\slides{2}{5}
\subsection{Übersicht Transaktionsverarbeitung}
\subsubsection*{Was ist eine Transaktion?}
\slides{2}{6}
\subsubsection{Eigenschaften einer Transaktion - ACID}
(potentielle Prüfungsfrage!!!)
\slides{2}{7}
\subsubsection*{prinzipieller Ablauf}
\begin{tabular}{L{0.3} c L{0.5}}
\emph{Programm} & & \emph{DBMS}\\
\hline 
BEGIN TRANSACTION &$\to$& Maßnahmen für pot. Rücksetzen \newline (Before Image / Undo Information)\\
Folge an DML-Befehlen & $\swarrow$ & \\
&$\searrow$ & Ausführen der Befehle (Integritätsprüfung)\\
Ende der Transaktion \newline (COMMIT/ROLLBACK) & $\swarrow$ & \\
&$\overset{\text{COMMIT}}{\searrow}$ & Maßnahmen zur Gewährleistung der Wiederholung der Transaktion \newline (After Image / Redo Information)\\
\hfill (alternativ) & $\overset{\text{ROLLBACK}}{\searrow}$ & Lade Before Image\\
Weiterer Programmablauf & $\swarrow$ & 
\end{tabular}\\
Before/After Image: gespeichert als Log, nicht direkt in der Datenbank

\paragraph{Beispiel} 2 Kontennr. in Tabelle Kunde\\
\begin{tabular}{l|l|l}
Ktnr & Inhaber & Wert\\
\hline
1001 & Meyer & $22000,\mhyphen$\\
1211 & Lehmann & $1100,\mhyphen$
\end{tabular}\\
Umbuchung: Meyer $\to$ Lehmann ($1000,\mhyphen$) $\Rightarrow$ 2 Updates:
\begin{lstlisting}[language=SQL]
BEGIN TRANSACTION
	UPDATE Kunde SET Wert = Wert - 1000
		WHERE Ktnr = 1001
	-- ohne Transaktion würde hier eine Inkonsistenz bei Störung auftreten
	UPDATE Kunde SET Wert = Wert + 1000
		WHERE Ktnr = 1211
COMMIT
\end{lstlisting}

\paragraph{Beispiel} Bestellwert Kontrolle: Maximale Summe des Bestellwertes je Kunde = $10000,\mhyphen$
\begin{lstlisting}[language=SQL]
BEGIN TRANSACTION
	UPDATE Kauf SET Menge = Menge * 2
		WHERE Kunr = 123
	IF (SELECT SUM(Menge * vPreis) FROM Kauf WHERE Kunr = 123) > 10000
		BEGIN
			PRINT 'Bestellwert > 10000 unzulässig'
			ROLLBACK
		END
	ELSE
		COMMIT
\end{lstlisting}

\paragraph{Hinweise}
\begin{itemize}
\item in Transaktion \emph{nicht}:
\begin{itemize}
\item DB, Index anlegen
\item DB, Tabelle ändern
\item Objekt löschen
\item SELECT INTO
\item Zugriffsrechte vergeben
\end{itemize}
\item Aufrufe von stored proc können \emph{nicht} rückgängig gemacht werden.
\end{itemize}

\section{Semantische Integrität}
\slides{2}{8}
\slides{2}{9}
\subsection{Möglichkeiten zur Sicherung der semantischen Integrität}
\slides{2}{10}
Trigger:\\
z.B. für Insert/Update/Delete (sodass dieser Trigger immer dann ausgeführt wird, wenn eine der Aktionen ausgeführt wird)
\slides{2}{11}
\subsection{Spalten- und Tabellen-bezogene Integrität}
\slides{2}{12}

Integritätsbedingungen (=constraints)
\begin{itemize}
\item Spalten-bezogen (column-constraint)
\item Tabellen-bezogen (table-constraint)
\end{itemize}
\begin{lstlisting}[language=SQL]
CREATE TABLE Kunde(
	Kunr INT,
	Name CHAR(10) NOT NULL,
	...,
	CONSTRAINT PK_Kunde PRIMARY KEY (Kunr) -- wobei "CONSTRAINT PK_Kunde" optional ist
	CONSTRAINT DF_Kunde_Ort DEFAULT 'Dresden' FOR Ort -- Default
	CONSTRAINT CH_Kunde_GebDat CHECK (GebDat < GETDATE()) -- Check
)

CREATE TABLE Artikel ( ... )

CREATE TABLE Kauf (
	Kunr INT,
	Artnr SMALLINT,
	Menge INT NOT NULL,
	vPreis DECIMAL(8,2) NOT NULL,
	CONSTRAINT PK_Kauf PRIMARY KEY (Kunr, Artnr), -- table constraint
	CONSTRAINT FK_Kauf_Kunde FOREIGN KEY (Kunr) REFERENCES Kunde(Kunr),
	CONSTRAINT FK_Kauf_Artikel FOREIGN KEY (Artnr) REFERENCES Artikel (Artnr)
)
\end{lstlisting}
\subsection{DEFAULTS und CHECK-Einschränkungen}
\slides{2}{13}
Bsp.: siehe oberhalb
\subsection{DOMAINS / Nutzerdefinierte Datentypen und ASSERTIONS}
\slides{2}{14}
Bsp. 1:
\begin{lstlisting}[language=SQL]
CREATE DOMAIN Note INT 
	CHECK (Note BETWEEN 1 AND 5)

CREATE TABLE Zeugnis (
	...
	Fach CHAR(20),
	Zensur Note,	-- definierter Datentyp kann bspw. so wieder verwendet werden.
	...
)

CREATE TABLE Pruefung (
	...
	Matnr CHAR(5),
	Modul CHAR(10),
	Zensur Note,
	...
)
\end{lstlisting}
Bsp. 2:
\begin{lstlisting}[language=SQL]
CREATE DOMAIN PLZ AS CHAR(5)
	CONSTRAINT DO_PLZ CHECK (CONVERT(INT,PLZ) > 1000 AND CONVERT (INT,PLZ)>10000) 
	DEFAULT '01069'
\end{lstlisting}
Löschen:
\begin{lstlisting}[language=SQL]
DROP DOMAIN <domainname> {RESTRICT|CASCADE} -- bspw. DROP DOMAIN PLZ
	-- RESTRICT: kann nicht weg gelöscht werden, wenn der Constraint noch in einer Tabelle in Verwendung ist
	-- CASCADE: ersetzt an der Stelle, in der Constraint in Verwendung ist, durch Datentyp um Constraint löschen zu können	
\end{lstlisting}
Bsp. Bibliothek-Datenbank:\\
Tabellen: Vorgemerkte\_Buecher, Ausleihbare\_Buecher\\
Bedingung: nicht mehr als $10\%$ der ausleihbaren Bücher vormerken.
\begin{lstlisting}[language=SQL]
CREATE ASSERTION AS_Vormerkungen
	CHECK ((SELECT COUNT(*) FROM Vorgemerkte_Buecher) / (SELECT COUNT(*) FROM Ausleihbare_Buecher) < 0.1) DEFERRED
	-- DEFERRED: am Ende der Transaktion
\end{lstlisting}
Anmerkung: Assertion nicht allzu relevant.

\subsection{Referentielles Integritätsproblem}
\slides{2}{15}

\subsection{Foreign key Klausel}
\slides{2}{16}

\subsection{Beachtung zyklischer Abhängigkeit}
\slides{2}{17}
Bsp.:

\begin{tabular}{l l}
Abteilung &Mitarbeiter\\
Abtnr& Abtltr \\
\hline
1001 & 13\\
1002 & 15
\end{tabular}
\begin{tabular}{l l}
Mitnr & Abtnr\\
12 & 1001\\
13 & 1002\\
14 & 1002\\
15 & 1001\\
\end{tabular}\\
Problem: Tabelle 1 referenziert die 2. (Abtltr) und die 1. die 2. (Abtnr)\\
Vorgehen: Erste Referenz im erstellen der Tabelle normal definieren, die zweite Referenz erst im Nachhinein mit ALTER TABLE einfügen

\subsection{Benutzerdefinierte Funktionen}
\slides{2}{18}
\begin{lstlisting}[language=SQL]
SELECT <Spaltenliste>	-- Skalarwertfunktion
FROM <Tabellenliste>	-- Tabellenwertfunktion
WHERE ...
\end{lstlisting}
\subsubsection{Skalarwertfunktion}
\slides{2}{19}
Anwendung für diese Funktion:
\begin{lstlisting}[language=SQL]
SELECT Kunr, Name, Kredit,
	dbo.Gebuehrensatz(Kredit) AS Gebuehrensatz,
	Kredit*dbo.Gebuehrensatz(Kredit)/100 AS Kreditgebuehr
FROM Kunde
\end{lstlisting}
\subsubsection*{Details zum Beispiel}
\slides{2B}{1}

\subsubsection{Tabellenwertfunktion}
\slides{2}{20}
wichtiger Unterschied zwischen „View“ und „Tabellenwertfunktion“: View können keine Parameter übergeben werden.
\subsubsection*{Details zum Beispiel}
\slides{2B}{2}
\subsubsection*{Weiteres Beispiel}
\begin{lstlisting}[language=SQL]
declare @ort char(20)
set @ort = 'Dresden'

SELECT a.*, b.Artnr, b.Menge, b.VPreis
FROM ErmittleKundenImOrt(@ort) a
	JOIN Kauf b ON a.Kunr = b.Kunr
\end{lstlisting}

\subsection{Trigger}
\slides{2}{21}
Trigger ist mit Tabelle verknüpft (für Aktionen wie Insert, Update, Delete, …).\\
Wenn nun ein Insert, Update, Delete, … auf die Tabelle ausgeführt wird, wird der entsprechende Trigger ausgeführt.

Trigger können nur vom Besitzer der Tabelle erstellt werden.

Beispiel für einfachen Trigger:
\begin{lstlisting}[language=SQL]
CREATE TRIGGER delKunde
 ON Kunde
 FOR DELETE
 AS 
 	print 'Sie haben Kunden gelöscht'
RETURN
\end{lstlisting}
Zugriff auf das Transaktionlog:\\
t-sql: virtuelle Tabellen (deleted, inserted)
\subsubsection*{Ablauf der Datenmanipulation}
Tabelle liegt auf der Festplatte. Tabelle wird zum Bearbeiten in den Hauptspeicher des Rechners kopiert. Transaktion ist auf separater Festplatte mit Transaktionlog (Protokolldatei) abgelegt.

Beispiel DELETE:\\
\begin{tabular}{l L{.25} c L{.3} c L{.3}}
&DB && Speicher && Log\\\hline
1.&DELETE Kunde & $\searrow$ &\\
2.&&&Kunden-Seite anlegen & $\searrow$ & \\
3.&&&&$\swarrow$&deleted = before image = Undo Information\\
4.&&$\swarrow$&Kunden-Seite löschen, dann bekannt geben nach außen, dass gelöscht &&\\
5.&Löschung schreiben (verzögert) &&
\end{tabular}

Beispiel INSERT:\\
\begin{tabular}{l L{.25} c L{.3} c L{.3}}
&DB && Speicher && Log\\\hline
1.&INSERT INTO Kauf & $\searrow$ &\\
2.&&&Kauf-Seite anlegen & $\searrow$ & \\
3.&&$\swarrow$&&$\leftarrow$&inserted=after image= Redo Information, dann bekannt geben nach außen, dass angelegt\\
4.&Löschung schreiben (verzögert) &&
\end{tabular}\\
Fall UPDATE: ist bloß Kombination aus DELETE+INSERT\\
Mit dem Trigger erhält man Zugriff auf das letzte Stück des Transaktionlog (inserted/deleted), welches die Transaktion betrifft, die gerade bearbeitet wird.
\subsubsection{Zusammenhang zwischen Programm und DBMS}
\slides{2}{22}
\subsubsection*{Trigger zur Abbildung der ref. Integrität}
Löschweitergabe: kaskadierendes Löschen\\
Kunde löschen in Tabelle Kunde $\Rightarrow$ Löschen seiner Einkäufe in Tabelle Kauf
\begin{lstlisting}[language=SQL]
CREATE TRIGGER Kund_loesch
	ON Kunde
	FOR DELETE
	AS
		DELETE Kauf
		WHERE Kunr IN (SELECT Kunr FROM deleted)
RETURN
\end{lstlisting}
Achtung: Trigger sollten für mehrere Datensätze funktionieren, nicht nur für einen (Deswegen Kunr IN … und nicht Kunr = … -- im Fall von mehreren und nicht nur einem Datensatz). Trigger sollte möglichst performant sein!

\subsubsection*{Zeitpunkt des Triggeraufrufs}
Bsp.:
\begin{lstlisting}[language=SQL]
DELETE FROM Kunde
WHERE Ort = 'Dresden'
\end{lstlisting}
Server arbeitet mit AUTOCOMMIT:\\
\begin{tabular}{l | l}
ohne Trigger & mit Trigger\\\hline
Beginn der Transaktion & Beginn der Transaktion\\
deleted schreiben (Tabelle) & deleted schreiben (Tabelle)\\
& DELETE-Trigger aufrufen\\
Transaktion beendet & Transaktion beendet ODER Trigger setzt zurück $\uparrow$ (Rollback)
\end{tabular}

\subsubsection*{Beispiel}
Löschen in Kunde unterbinden, wenn in Kauf noch Datensätze für den Kunden vorhanden sind (RESTRICT):
\begin{lstlisting}[language=SQL]
CREATE TRIGGER delKund1
	ON Kunde FOR DELETE
	AS
		IF (SELECT COUNT(*) FROM Kauf k JOIN delete d ON k.Kunr = d.Kunr) > 0
		BEGIN
			print 'Abbruch, da noch Daten in Kauf'
			SELECT ... -- Datensätze, die referenziellen Integrität widersprechen auch mit print ausgeben
			ROLLBACK TRANSACTION -- Trigger steckt mitten in Transaktion, deswegen kann hier ein Rollback der Transaktion veranlasst werden, obwohl sie hier nicht explizit begonnen wurde
		END
RETURN
\end{lstlisting}
\subsubsection*{Beispiel}
Umsatzsumme je Kunde in Kauf max $10.000,-$\euro{}.
\begin{lstlisting}[language=SQL]
CREATE TRIGGER itr_Kauf ON Kauf
FOR insert, update
	AS
		IF EXISTS (
			SELECT SUM(Menge*vPreis)
			FROM Kauf
			GROUP BY Kunr
			HAVING SUM(Menge*vPreis) > 10000)
		BEGIN
			print 'Kundenumsatz größer 10.000 unzulässig.'
			ROLLBACK TRANSACTION
			-- wenn hier RETURN steht, dann können nach dem END weitere Bedingungen (IF...) angehängt werden.
		END
RETURN
\end{lstlisting}
\subsubsection*{Weiter Beispiele}
\slides{2}{23}
\begin{lstlisting}[language=SQL]
CREATE TRIGGER MittelAenderung
ON Projekt
FOR UPDATE AS IF UPDATE (Mittel) -- aus Performancegründen nur triggern, wenn Mittel geändert
BEGIN
	INSERT INTO Mittel_protokoll
		SELECT i.Pronr, user_name(), GETDATE(), d.Mittel, i.Mittel
		FROM inserted i JOIN deleted d ON i.Kunr = d.Kunr AND i.Artnr = d.Artnr
END
\end{lstlisting}
\slides{2}{24}

\section{Operationale Integrität}
„Gefahrenbereich“ Mehrbenutzerbetrieb.
\subsection{Anomalien im Mehrbenutzerbetrieb}
\subsubsection{Lost Update}
\slides{2}{25}
\subsubsection{Dirty Read}
\slides{2}{26}
\subsubsection{Non repeatable Read}
\slides{2}{27}
\subsubsection{Phantomproblem}
\slides{2}{28}
\subsection{Sicherung der operationalen Integrität}
\slides{2}{29}
\subsubsection{Sperrverfahren}
\slides{2}{30}
BOT… begin of transaction\\
EOT… end of transaction\\
\begin{tikzpicture}
\node [red] at (2,3) {$\bigcirc$: gefährdete Bereiche};
\end{tikzpicture}
\subsubsection*{Einfaches Zweiphasen-Sperrprotokoll}
\begin{tikzpicture}[scale=.7]
\draw[-latex] (-4,0) -- (2,0) node[below right]{Zeit};
\draw[-latex] (-4,0) -- (-4,3) node[left, align= right]{Anzahl\\Sperren};
\draw (-3,0) node[below]{BOT} -- (-2,2) -- (0,2) -- (1,0) node[below]{EOT};
\node [red] at (-2.5,1) {$\bigcirc$};
\node [red] at (.5,1) {$\bigcirc$};
\end{tikzpicture}
\subsubsection*{Zweiphasen-Sperrprotokoll mit Preclaiming}
\begin{tikzpicture}[scale=.7]
\draw[-latex] (-4,0) -- (2,0) node[below right]{Zeit};
\draw[-latex] (-4,0) -- (-4,3) node[left, align= right]{Anzahl\\Sperren};
\draw (-3,0) node[below]{BOT} -- (-3,2) -- (0,2) -- (1,0) node[below]{EOT};
\node [red] at (.5,1) {$\bigcirc$};
\end{tikzpicture}
\subsubsection*{Striktes Zweiphasen-Protokoll}
\begin{tikzpicture}[scale=.7]
\draw[-latex] (-4,0) -- (2,0) node[below right]{Zeit};
\draw[-latex] (-4,0) -- (-4,3) node[left, align= right]{Anzahl\\Sperren};
\draw (-3,0) node[below]{BOT} -- (-2,2) -- (1,2) -- (1,0) node[below]{EOT};
\node [red] at (-2.5,1) {$\bigcirc$};
\end{tikzpicture}
\subsubsection*{Striktes Zweiphasen-Protokoll mit Preclaiming}
\begin{tikzpicture}[scale=.7]
\draw[-latex] (-4,0) -- (2,0) node[below right]{Zeit};
\draw[-latex] (-4,0) -- (-4,3) node[left, align= right]{Anzahl\\Sperren};
\draw (-3,0) node[below]{BOT} -- (-3,2) -- (1,2) -- (1,0) node[below]{EOT};
\end{tikzpicture}
\subsubsection{Statische und dynamische Sperrverfahren}
\slides{2}{31}
\subsubsection{Anforderungen an Sperrmodi}
\slides{2}{32}
Sperrgranulate:
\begin{itemize}
\item logische Objekte:\\
Tupel, Tabelle
\item physische Objekte:\\
Seite (Block auf Festplatte), Segmente 
\end{itemize}
Sperrverwaltung über Lockmanager.\\
$+$ … erlauben, $-$ … ablehnen (Warteschlange)\\
\begin{tabular}{c | c | c | c}
Sperrwunsch & aktuell keine Sperre & aktuell S-Sperre & aktuell X-Sperre\\\hline
S & $+$ & $+$ & $-$\\
X & $+$ & $-$ & $-$
\end{tabular}
\subsubsection{Stufen der Isolation paralleler Transaktionen}
\slides{2}{33}
SQL:
\begin{lstlisting}[language=SQL]
SET TRANSACTION ISOLATION LEVEL ...
\end{lstlisting}
Mögliche Anomalien nach Ebene:\\
\begin{tabular}{c | c | c | c | c}
Ebene & Lost Update & Dirty Read & Non repeat. Read & Phantom\\\hline
0 & X & X & X  & X\\
1 & & & X & X\\
2 & & & & X\\
3 &&&&\\
\end{tabular}\\
Also: Nur mit Ebene-3 können alle Anomalien abgedeckt sein. Aber: tiefere Ebenen kosten mehr Leistung!

\subsubsection{Optimistische Verfahren}
\slides{2}{34}
Grundgedanke:
\begin{itemize}
\item Transaktionen nutzen oft verschiedene Daten
\item Konflikte selten
\item kein Overhead für Sperrverwaltung
\end{itemize}

\subsubsection{Zeitstempelverfahren}
\slides{2}{35}

Letzten Endes werden Kombinationen verschiedener Verfahren verwendet:\\
Jeder Nutzer bekommt einen Zeitstempel, im Backend läuft allerdings ein pessimistisches Verfahren.
\section{Physische Integrität}
\slides{2}{36}
\begin{itemize}
\item abgeschlossene Transaktion $\Rightarrow$ Ergebnis herstellen\\
nicht abgeschlossene Transaktion $\Rightarrow$ Rücksetzen
\item Verfahren des Sicherns $=$ Logging, Backup\\
Verfahren des Wiederherstellens $=$ Recovering
\end{itemize}
\subsection{Protokollierung}
\begin{itemize}
\item Abspeicherzeitpunkt (Beginn/Ende der Transaktion)
\item Protokollgranulat (Tupel/Befehle)
\item Niveau (logisch: Tupel/Befehl oder physisch: Seite)
\end{itemize}
\subsubsection*{Eintragungen in der Protokolldatei}
\slides{2}{37}
\subsection{Einbringstrategien}
\slides{2}{38}
\subsection{Prinzip der physischen Protokollierung}
\slides{2}{39}
\slides{2}{40}
\subsection{Datensicherung und Wiederherstellung am Beispiel MS SQL-Server}
Log (*.ldf Datei)\\
$\uparrow$\\
Daten- und Log-Seiten im Hauptspeicher \\
$\updownarrow$ \\
Datenbank (*.mdf Datei)
\subsubsection*{Archivsicherung}
\begin{itemize}
\item Datenbank: Vertrieb.bak
\begin{lstlisting}[language=SQL]
BACKUP DATABASE Vetrieb TO DISK = 'E:\Backups\Vertrieb.bak'
\end{lstlisting}
\item Log: V\_log1.bak
\begin{lstlisting}[language=SQL]
BACKUP LOG Vertrieb TO DISK = 'E:\Backups\V_log1.bak'
\end{lstlisting}
Sobald Log gesichert wird, wird *.ldf Datei geleert!
\end{itemize}
\subsection{Wiederherstellungsmodi}
\slides{2}{41}
\begin{itemize}
\item vollständig (Standard)
\begin{lstlisting}[language=SQL]
ALTER DATABASE SET RECOVERY FULL
\end{lstlisting}
Nachteil: \\
$-$ Log-File erst mit Log-Sicherung abgeschnitten\\
Vorteil:\\
$+$ Wiederherstellung bis letzte Log-Sicherung möglich (bei Festplatten-Fehler)
\begin{lstlisting}[language=SQL]
RESTOR DATABASE Vertrieb FROM DISK = 'E:\Backups\Vertrieb.bak'
\end{lstlisting}
Zum Wiederherstellen bei einem Ausfall (bspw. um 9:30) wird die gesicherte Datenbank hergestellt (die bspw. um 6:00 erstellt wurde) und alle danach vorhandenen Logs (die bspw. um 7:00, 8:00 und 9:00 erstellt wurden) angewandt, um wieder zum verlorenen Zustand zurück zu kommen. Achtung: Logs sind inkrementell! Ist ein Log kaputt, können folgende Logs nicht mehr zur Wiederherstellung genutzt werden.\\
Anhand der Logs kann auch bei falscher Manipulation bis zu einem bestimmten Zeitpunkt wiederhergestellt werden (mit dem Backup von 9:00 Zustand von genau 8:20 herstellbar).
\item einfach
\begin{lstlisting}[language=SQL]
ALTER DATABASE SET RECOVERY SIMPLE
\end{lstlisting}
Nachteil:\\
$-$ Wiederherstellung nur bis letztem DB-Backup
Vorteil:\\
$+$ Log-Datei „wächst nicht“ (Wird nur für Transaktionen benutzt)\\
Zum Wiederherstellen bei einem Ausfall (bspw. um 9:30) kann nur die gesicherte Datenbank (von bspw. 6:00) wieder hergestellt werden, es gibt keine Logs dazwischen, die einen neueren Zustand wiederherstellen können!
\end{itemize}
\subsubsection{Zusammenhang zwischen Protokoll- und Recoveryverfahren}
\slides{2}{42}
\subsubsection{Weitere Recoveryverfahren}
\slides{2}{43}
\subsubsection{Übertragung Datenbankpuffer-Platte}
\slides{2}{44}
\subsection{Hochverfügbarkeit}
\slides{2}{45}
\slides{2}{46}
\subsubsection{Spiegelung}
\begin{enumerate}
\item Eintragung in Datenbank 1
\item REDO ins Log der DB 1
\item Spiegelung der Eintragung in Datenbank 2
\item REDO der Spiegelung ins Log der DB 2
\item Bestätigung an DB 1
\end{enumerate}

\subsection{DB-Serverstrukturen}
am Beispiel MS SQL / Sybase\\
Datenbanken (DB):
\begin{itemize}
\item Benutzerdatenbanken (siehe Praktikum)
\begin{itemize}
\item Benutzertabellen (z.B. Mitarbeiter, Projekt)
\item Systemtabellen sys.… (Speicherung der Metadaten) + Systemsicht auf Systemtabellen
\end{itemize}
\item Systemdatenbanken
\begin{itemize}
\item master
\begin{itemize}
\item Verwaltung der Benutzer-DB
\item Berechtigungen auf dem Server
\item[] $\vdots$
\end{itemize}
\item model
\begin{itemize}
\item Prototyp einer Benutzer-DB
\end{itemize}
\item tempdb
\begin{itemize}
\item Ablage nicht dauerhafter Objekte
\end{itemize}
\end{itemize}
\end{itemize}
\emph{Wichtig}: Metadaten liegen in Systemtabellen sys.***!

\section{Zugriffsschutz/Datenschutz}
\slides{2}{47}
\subsection{Probleme beim Datenschutz}
\slides{2}{48}
\begin{tabular}{L{0.19} | L{0.4} L{0.4}}
& Datenschutz & Zugriffsschutz \\\hline
Bereich & Bundesweit & Informationssystem\\
Gegenstand & personenbezogene Daten & Daten aller Fachgebiete\\
Mittel & juristisch/organisatorisch & organisatorisch, SW/HW-Schutzfunktion
\end{tabular}

\subsection{Access- und Identity Management (IAM)}
\slides{2}{49}

\subsection{Authentifizierung}
\slides{2}{50}
Beispiel: Privilegien vergeben (Tabelle Mitarbeiter)\\
\begin{tabular}{l | c | c | c | c}
& MitNr & Name & Qualif. & Gehalt \\\hline
Meier (Produktplaner) & lesen & lesen & lesen & X\\
Schmidt (Geschäftsführung) & lesen & lesen & lesen & lesen\\
Lehmann (HR/Personal) & schreiben & schreiben & schreiben & schreiben
\end{tabular}

\subsection{Verwaltung von Nutzer und Gruppen in SQL}
\slides{2}{51}
\subsection{Aufgabenkomplexe Autorisierung}
\slides{2}{52}

\subsection{Aufgaben der Zugriffskontrolle}
\slides{2}{53}

\subsection{Komponenten der Autorisierung}
\slides{2}{54}

\subsection{Zugriff auf den Server (MS SQL Server)}
\slides{2}{55}
\subsubsection*{Anweisungen zum Zugriffsschutz}
Zugriffsschutz im MS SQL Server\\
sa … Systemadministrator\\
dbo … DataBaseOwner
\begin{lstlisting}[language=SQL]
-- sa:
CREATE LOGIN MeierU WITH PASSWORD '123'
CREATE LOGIN Schmidt WITH PASSWORD '234'
CREATE LOGIN Richter WITH PASSWORD '345'

CREATE USER MeierU FOR LOGIN MeierU
GRANT CREATE DATABASE TO MeierU
-- Logout sa
-- MeierU:
CREATE DATABASE Test
USE Test	-- in Datenbank Test springen
CREATE USER Schmidt FOR LOGIN Schmidt
GRANT CREATE TABLE TO Schmidt
-- Logout MeierU
-- Schmidt:
USE Test
CREATE TABLE Adresse (Ort, ...
\end{lstlisting}
Login: Zugriff auf Datenbankserver (ohne Berechtigungen an lesen/schreiben/…)\\
User: Nutzer der Datenbank (Zugriffsberechtigung nach Vorgabe)
\subsection{Zugriff auf Datenbank (MS SQL Server)}
\slides{2}{56}
\subsubsection{Zugriff auf Befehle kontrollieren (Systemprivilegien)}
\slides{2}{57}
\subsubsection{Zugriff auf Objekte kontrollieren (Objektprivilegien)}
\slides{2}{58}
\subsubsection{Varianten der Umsetzung von Zugriffsschutz}
Beispiel: Tabelle Mitarbeiter\\
public: Mitnr, Name, Vorname, Anschrift\\
geschützt: Gehalt
\begin{enumerate}
\item Direkter Schutz
\begin{lstlisting}[language=SQL]
GRANT SELECT ON Mitarbeiter(Mitr, Name, Vorname, Anschrift) TO MeierU
\end{lstlisting}
\item Indirekter Schutz (über VIEW)
\begin{lstlisting}[language=SQL]
CREATE VIEW MitAdresse AS 
	SELECT Mitnr, Name, Vorname, Adresse
	FROM Mitarbeiter
GRANT SELECT ON MitAdresse TO MeierU
\end{lstlisting}
\item Indirekter Schutz (über stored procedure)
\begin{lstlisting}[language=SQL]
CREATE PROC MitProc AS
	SELECT Mitnr, Name, Vorname, Adresse
	FROM Mitarbeiter
GRANT EXEC MitProc TO MeierU
\end{lstlisting}
\end{enumerate}

\subsubsection{SQL-Injection abwehren}
Beispiel: e-Business-Application\\
\begin{tabular}{l | l | l}
	ID & Name & PW\\\hline
	1 & Meier & abc\\
	2 & Schulze & xyz
\end{tabular}
\begin{lstlisting}[language=SQL]
IF (SELECT COUNT (*))
	FROM User
	WHERE Name='Meier'
	AND PW='abc') > 1
\end{lstlisting}
$\Rightarrow$ Wenn ein Ergebnis zurückkommt ist das Passwort richtig geraten.
\begin{lstlisting}[language=SQL]
string sql= "SELEC COUNT(*) FROM user WHERE Name='"
	+strName
	+"' AND PW='"
	+strPW+"'";
\end{lstlisting}
Mittels einer SQL Injection kann der SQL Code so erweitert werden, dass eine Anmeldung immer erfolgreich ist.
\begin{lstlisting}[language=SQL]
strName = A'OR'x'='x
strPW = A'OR'x'='x

sql = SELECT COUNT(*)
	FROM user
	WHERE Name='A'OR'x'='x'
	AND PW = 'A'OR'x'='x'
\end{lstlisting}
Lösung: stored procedure einsetzen:
\begin{lstlisting}[language=SQL]
CREATE PROC logintest (@Name char(10), @Pw char(10))
BEGIN
	IF EXISTS (
		SELECT COUNT(*)
		FROM user
		WHERE Name=@Name
		AND PW=@pw		
	)
		RETURN 1
	ELSE
		RETURN 0
END
\end{lstlisting}
Aufruf der stored procedure:
\begin{lstlisting}[language=SQL]
string sql= 'exec dlogintest('=
	+ strName
	+ "','"
	+ strPw
	+ "')";
\end{lstlisting}
Aufruf der fehlerhaften stored procedure vie SQL-Injektion
\begin{lstlisting}[language=SQL]
exec logintest('A'OR'x'='x','A'OR'x'='x')
\end{lstlisting}
\chapter{Logische Datenmodelle}
\setCustomSlideScale{1}
Datenmodellierung ist ein zweistufiger Abstraktionsprozess:
\slides{3}{15}
\section{Logische und physische Datenorganisation / Überblick über Datenmodelle}
\subsection{Zusammenhang zwischen semantischen und logischen Datenmodellen}
\slides{3}{16}
\subsection{Semantische Datenmodelle}
\slides{3}{17}
ERM:\\
\begin{tikzpicture}[scale=.7]
\draw  (-2.5,3.5) rectangle (3,1.5) node[midway]{Abteilung};
\draw[rotate=45]  (0.5,0.5) rectangle (-0.5,-0.5)node [midway] (v2) {hat};
\draw  (-2.5,-1.5) rectangle (3,-3.5)node[midway]{Mitarbeiter};
\draw[rotate=45]  (-3,-3) node (v1) {} rectangle  (-5,-5)node [midway] {bearbeitet};
\draw  (-2.5,-8) rectangle (3,-10)node[midway]{Projekt};
\draw (0,-3.5) -- (0,-4.25)node[right, midway]{m};
\draw (0,-7.1) -- (0,-8)node[right, midway]{n};
\draw (0,1.5) -- (0,0.7)node[right, midway]{1};
\draw (0,-0.7) -- (0,-1.5) node[right, midway]{n};
\end{tikzpicture}
\subsection{Logische Datenmodelle}
\slides{3}{18}
\slides{3}{19}
\section{Klassische logische Datenmodelle}
\subsection{Hierarchisches Modell}
\slides{3}{20}
Speicherung der Beziehung in den Datensätzen\\
Vorteil: hohe Abfrageperformance\\
Nachteil: eingeschränkte Datenunabhängigkeit („unflexibel“)
\slides{3}{21}
\subsection{Netwerkdatenmodell}
\slides{3}{22}
Speicherung der Beziehung in den Datensätzen\\
Vorteil: hohe Abfrageperformance\\
Nachteil: eingeschränkte Datenunabhängigkeit („unflexibel“)
\slides{3}{23}
\subsection{Relationales Datenmodell}
\subsubsection{Relationenbegriff}
\slides{3}{24}
Beispiel:\\
Entity-Typ: Zeugnis\\
Attribute:\\
$A_1$ Fach $W_1\{Ma,Ph\}$\\
$A_2$ Note $W_2\{1,2,3,4,5\}$\\
$PM=W_1*W_2$
\subsubsection{Charakteristika}
\slides{3}{25}
Abfrage der Beziehungen zur Laufzeit\\
Vorteil: Datenunabhängig, flexibel (Erweiterung, Abfrage)\\
Nachteil: Performance schlechter als hierarchisch oder Netzwerk-DM

\subsubsection{Grenzen des Relationalen Datenmodells: Modellierung Polyeder}
\slides{3}{26}
Abfrage aller Punkte mit \lstinline|F.fid<3|
\begin{lstlisting}[language=SQL]
SELECT F.fid, P.x, P.y, P.z
FROM Punkt P, KP_Rel, KP, FK_Rel FK, Flaeche F
WHERE F.fid<3
	AND P.pid = KP.pid
	AND KP.kid = K.kid
	AND K.kid = FK.kid
	AND FK.fid = F.fid
\end{lstlisting}
$\Rightarrow$ NF$^2$-Modell wäre schneller (keine 1. Normalform)
\slides{3}{27}

% Vorlesung

\section{Objektorientiertes Datenmodell}
\begin{tabular}{ll}
\textbf{Konventionelles Programm}&\textbf{Objektorientiertes Programm}\\
\mpb[.5]
\begin{tikzpicture}

\draw  (-4.5,3.5) rectangle (-0.5,2.5) node [midway]{Funktion};
\draw  (-4.5,1.5) rectangle (-0.5,0.5) node [midway]{Funktion};
\draw  (-4.5,-1.5) rectangle (-0.5,-0.5) node [midway]{Funktion};
\draw[-latex] (-2.5,2.5) -- (-2.5,1.5);

\draw [-latex](-2.5,0.5) -- (-2.5,-0.5);

\draw  (2.5,3) ellipse (1 and 0.5)node{Datenmodell};
\draw (1.5,3) -- (1.5,1.5);
\draw (3.5,3) -- (3.5,1.5);

\draw[-latex, gray] (-0.5,3) -- (1.5,2);

\draw[-latex] (-0.5,-1) node (v1) {} -- (1,-1) -- (1,1) -- (-0.5,1);

\draw[-latex, gray] (-0.5,-1) -- (1.5,2);
\end{tikzpicture}
\mpe&
\mpb
\begin{itemize}
    \item Objekt:
    \begin{itemize}
        \item Daten
        \item Funktion
        \item Funktion
        \item Funktion
    \end{itemize}
    
    \item Objekt:
    \begin{itemize}
        \item Daten
        \item Funktion
    \end{itemize}
    
    \item Objekt:
    \begin{itemize}
        \item Daten
        \item Funktion
        \item Funktion
    \end{itemize}
\end{itemize}
\mpe
\end{tabular}
\slides{3}{28}
\begin{tikzpicture}

\node at (-2,2) {KFZ};
\draw  (-0.5,3) rectangle (2,2)node[midway]{Hersteller};
\draw  (-0.5,2) rectangle (2,1)node[midway]{Baujahr};
\draw  (-0.5,1) rectangle (2,0)node[midway]{KFz-KZ};
\draw [-latex](-1.5,2) node (v1) {} -- (-0.5,2.5);

\draw [-latex](v1) -- (-0.5,1.5);

\draw [-latex](v1) -- (-0.5,0.5);
\node at (-2,-1.5) {LKW};
\node at (3,-1.5) {PKW};
\draw (-2,1.5) node (v2) {} -- (-2,-1);

\draw (v2) -- (-0.5,-0.5) -- (3,-1);
\draw  (-4.5,-1) rectangle (-2.5,-1.5)node[midway]{Ladeflaeche};
\draw  (-4.5,-1.5) rectangle (-2.5,-2)node[midway]{Nutzlast};
\draw[-latex] (-2,-1.7) -- (-2.5,-1.8);

\draw [-latex](-2,-1.3) -- (-2.5,-1.2);

\node at (-2.5,-4) {Pickup};
\draw (-2,-2) -- (-2.5,-3.5);

\draw (3,-2) -- (-2,-3.5);
\draw  (4,-1) rectangle (6.5,-1.5)node[midway]{Sitzplaetze};
\draw[-latex] (3.5,-1.5) -- (4,-1.3);
\end{tikzpicture}
\subsection{Charakterisierung der Objektorientierung}
\slides{3}{29}
\slides{3}{30}
\subsection{Bedeutung des Objektorientierten Datenmodells}
\slides{3}{31}
\subsection{Objektorientierten und Objektrelationales Datenmodell}
\slides{3}{32}
\subsection{Objektrelationales Datenmodell}
\slides{3}{33}
\subsubsection[NF2-Ausprägung (1 Tupel)]{NF$^2$-Ausprägung (1 Tupel)}
\slides{3}{34}
\section{DBMS-spezifische Erweiterungen vom Standard-SQL (Oracle)}

\subsection{Datendefinition in Oracle}
\slides{3}{1}
\subsubsection{Oracle SQL Datentypen}
\slides{3}{2}
\subsection{Arbeit mit Systemtabellen des Data Dictonary}
\slides{3}{3}
\subsection{Sequenz}
\slides{3}{4}
Kein SELECT ohne FROM.\\
SELECT in Oracle:
\begin{lstlisting}[language=SQL]
SELECT sysdate FROM dual;
--Output: 16.12.2016 09:41

SELECT user FROM dual;
--Output: sxxxxx

SELECT Name || ',' || Vorname 
       || ' (' || SUBSTR(Beruf,1,3) 
       || ')' AS Kundenname
FROM Kunde

-- Output:
-- Meier, Uwe (Ing)
-- Adler, Sabine (Inf)
\end{lstlisting}
\subsection{Spezifika bei Abfragen}
\slides{3}{5}
\subsubsection{Hierarchische Abfragen}
\slides{3}{6}
\subsection{Ausgewählte SQL-Funktionen in Oracle}
\slides{3}{7}
\subsection{PL/SQL-Statements in Proceduren und Triggern}
\slides{3}{8}
Variablendeklaration:
\begin{lstlisting}[language=SQL]
DECLARE v_name VARCHAR2(10) := 'Otto';
        v_ort VARCHAR2(20);
        v_alt NUMBER(3);
\end{lstlisting}
Wertzuweisung:
\begin{lstlisting}[language=SQL]
v_alt := 27;
v_ort := 'Dresden';
\end{lstlisting}
\subsection{Stored Procedures}
\slides{3}{9}
\subsection{Cursor in PL/SQL}
\slides{3}{10}
\begin{lstlisting}[language=SQL]
CREATE OR REPLACE PROCEDURE p1(K_wert IN INTEGER)
IS
    CURSOR C1 IS
        SELECT Kunr, Name
        FROM Kunde
        WHERE Kredit > K_wert;
    FOR satz IN C1 LOOP
        DBMS_OUTPUT.PUTLINE('Kunr:' || satz.Kunr);
        DBMS_OUTPUT.PUTLINE('Name:' || satz.Name);
    END LOOP;
    END;
/
\end{lstlisting}

2. Variante CURSOR-Nutzung:
\begin{lstlisting}[language=SQL]
...
CURSOR C1 IS 
	SELECT Kunr, Name, Vorname
	FROM Kunde
	WHERE Kredit > K_wert;
FOR satz IN C1 LOOP
	DBMS_OUTPUT.PUTLINE('Kunr: '|| satz.Kunr);
	DBMS_OUTPUT.PUTLINE('Name: '|| satz.Name);
	DBMS_OUTPUT.PUTLINE('Vorname: '|| satz.Vorname);
END LOOP;
...
\end{lstlisting}
\subsection{Trigger in PL/SQL}
\slides{3}{11}
\subsubsection{Trigger-Beispiele}
\slides{3}{12}
\subsubsection*{Trigger zur Sicherung referentieller Integrität}
(DS in Kauf einfügen oder ändern $\Rightarrow$ prüfen, ob Kunde vorhanden ist)
\begin{lstlisting}[language=SQL]
CREATE OR REPLACE TRIGGER CheckKunde
	AFTER INSERT OR UPDATE ON Kauf
	FOR EACH ROW
		DECLARE i int;
						nicht_moeglich EXCEPTION;
		BEGIN 
			SELECT COUNT(*) into i
			FROM Kunde k
			WHERE k.Kunr = :NEW.Kunr;
			IF i=0 THEN
				RAISE nicht_moeglich;
			END IF
			-- ggf. weitere Prüfungen
			
			EXCEPTION
				WHEN nicht_moeglich
					raise_application_error(-2100,'Kunr nicht vorhanden');
				-- ggf. weitere Exceptions
			
			END;
		/
\end{lstlisting}

\slides{3}{13}
\paragraph{Zeilen} :NEW :OLD nur im Row-Trigger
\begin{tabular}{l | l | l}
Operation & :OLD & :NEW\\\hline
INSERT & NULL & einzufügende Tupel\\
UPDATE & Tupel vor Update & Tupel nach Update\\
DELETE & zu löschender Tupel & NULL
\end{tabular}

Trigger:
\begin{itemize}
\item Beim Einfügen in Artikel\\
$\Rightarrow$ ArtNr als nächsten Wert einer Sequenz setzen
\item Beim Ändern\\
$\Rightarrow$ keine Änderung der ArtNr (primary key) zulassen
\end{itemize}
\begin{enumerate}
\item Sequenz anlegen:
\begin{lstlisting}[language=SQL]
create sequence sq_artikel
	increment by 1
	start with 10000;
\end{lstlisting}
\item Trigger anlegen:
\begin{lstlisting}[language=SQL]
create or replace trigger tr_artikel
	befor insert or update on Artikel
	for each row
	begin
		if inserting then		-- nur für Insert
			select sq_artikel.nextval
				into :NEW.Artnr from dual
		end if;
		if updating then		-- nur für Update
			:NEW.Artnr := :OLD.Artnr
		end if;
	exception
		when others then
			raise_application_error(-2100, sqlerrm);
	end;
\end{lstlisting}
\end{enumerate}
\slides{3}{14}

\subsection{Unterschiedliche Strategien beim Umgang mit dem Transaktion-Log in Triggern}
\begin{tabular}{L{0.18} | l | l}
& MS SQL-Server/Sybase & Oracle\\\hline
Ansatz & Tabellen-orientiert & Datensatz-orientiert\\
Umsetzung & „magic“ \emph{tables}: deleted, inserted & \emph{tupel} (einzelne Datensätze!): :OLD, :NEW\\
Verwendung & in jedem Trigger & nur im ROW-Trigger\\
Transaktion zurücksetzen & ROLLBACK TRAN & RAISE\_APPLICATION\_ERROR(…)
\end{tabular}\bigskip\\
Beispiel:
\begin{lstlisting}[language=SQL]
UPDATE Kauf SET vPreis = vPreis * 1.1 WHERE Kunr=123
\end{lstlisting}
\emph{MS-SQL}\\
table deleted:\\
\begin{tabular}{c | c | c | c}
Kunr & Artnr & Menge & vPreis\\\hline
123 & 1234 & 10 & 360\\
123 & 5555 & 10 & 320\\
123 & 2222 & 10 & 3600\\
\end{tabular}\\
table inserted:\\
\begin{tabular}{c | c | c | c}
Kunr & Artnr & Menge & vPreis\\\hline
123 & 1234 & 10 & 396\\
123 & 5555 & 10 & 352\\
123 & 2222 & 10 & 3960\\
\end{tabular}\bigskip\\
\emph{Oracle}\\
1. row:\\
\begin{tabular}{c | c | c | c | c}
&Kunr & Artnr & Menge & vPreis\\\hline
:OLD & 123 & 1234 & 10 & 360\\
:NEW & 123 & 1234 & 10 & 396
\end{tabular}\\
2. row:\\
\begin{tabular}{c | c | c | c | c}
&Kunr & Artnr & Menge & vPreis\\\hline
:OLD & 123 & 5555 & 10 & 320\\
:NEW & 123 & 5555 & 10 & 352
\end{tabular}\\
3. row:\\
\begin{tabular}{c | c | c | c | c}
&Kunr & Artnr & Menge & vPreis\\\hline
:OLD & 123 & 2222 & 10 & 3600\\
:NEW & 123 & 2222 & 10 & 3960
\end{tabular}
\section{Objektrelationale Erweiterungen im RDM (Oracle)}
\subsection{Objektrelationale Features in Oracle}
\slides{3}{35}
\subsection{Komplexbeispiel für relationale Features}
\slides{3}{36}
\subsection{Abstrakte Datentypen - Objekt}
\slides{3}{37}
\slides{3}{38}
\subsection{Objektviews}
\slides{3}{39}
Überschreibt den eigentlichen INSERT-Befehl.
\slides{3}{40}
\subsection{Methoden}
\slides{3}{41}
\slides{3}{42}
\subsection{Verschachtelte Tabellen (Nested Tables)}
\slides{3}{43}
\slides{3}{44}
\slides{3}{45}
\subsection{Variable Arrays (VARRAYs)}
\slides{3}{46}
\subsection{Large Objects (LOBs)}
\slides{3}{47}
\slides{3}{48}
\subsection{Evolution von Datenmodellen}
\slides{3}{49}
\subsubsection{Aufgabenverteilung im Wandel der Zeit}
\slides{3}{50}

\chapter{Physische Datenorganisation}
\section{Übersicht/Abgrenzung}
2 Aufgabenstellungen:
\begin{itemize}
\item Bildung und Strukturierung internen Sätze
\begin{itemize}
\item variable oder feste Satzlänge
\item feste Formate oder Kennzeichenformate
\item Datentypen
\item Blockung, Segmentierung\\
z.B.: 1 Black = 2 kByte = 2048 Byte
\end{itemize}
\item Festlegung der Zugriffspfade
\begin{itemize}
\item Pfade der Daten (Kapitel 4.2)
\item Pfade innerhalb der Daten (Kapitel 4.3)
\end{itemize}
\end{itemize}
\slides{4}{1}
Datensatz (Mitnr, Name, Vorname, Alter)
\begin{itemize}
\item feste Länge ohne Kennzeichen:\\
\begin{tabular}{c | c | c | c}
 \_103 & Meier\_\,\_\,\_\,\_ & Paul \_\,\_ & \_37
\end{tabular}
\item feste Länge mit Kennzeichen\\
\begin{tabular}{c | c | c | c | c | c }
\tblue{I} & \_103 & \tblue{N} & Meier\_\,\_\,\_\,\_& \tblue{A} & \_37
\end{tabular}
\item variable Länge ohne Kennzeichen\\
\begin{tabular}{c | c | c | c | c | c | c  | c}
\tred{L} & 103 & \tred{L} & Meier & \tred{L} & Paul & \tred{L} & 37
\end{tabular}
\item variable Länge mit Kennzeichen\\
\begin{tabular}{c | c | c | c | c | c | c | c | c }
\tblue{I} & \tred{L} & 103 & \tblue{N} & \tred{L} & Meier & \tblue{A} & \tred{L} & 37
\end{tabular}
\end{itemize}
\section{Zugriffspfade außerhalb der Datensätze/Indexierung}
\subsection{Grundprinzip der Indexierung}
\slides{4}{2}
\subsubsection{Geordneter Binärbaum}
\slides{4}{3}
\subsubsection{Geordneter, balancierter Binärbaum}
\slides{4}{4}
Binärbaum:
\begin{itemize}
\item binäre Suche möglich
\item einfaches Einfügen/Ändern
\item alle Knoten enthalten Daten
\item gut für Datenhaltung im Hauptspeicher
\end{itemize}
aber DBS $\Rightarrow$ Datenhaltung auf Festplatte $\to$ Datenhaltung an Blöcken (Seiten) orientieren
\subsection{B*-Bäume}
\slides{4}{5}
\subsubsection*{Beispielstruktur eines B*-Baums}
\slides{4}{6}
\subsubsection*{Bsp.: Datenmenge im B*-Baum}
\begin{tabular}{l l}
DS-Länge: & $80 \unit{Byte}$\\
Schlüssel + Zeiger: & $10 \unit{Byte}$\\
Seitengröße: & $2048 \unit{Byte}$\\
Anzahl DS/Seite: & $\tfrac{2048}{80} = 25$\\
durchschnittliche Anzahl DS/Seite: & $\tfrac{5}{6} \cdot 25 = \boxed{20}$\\
Anzahl Schlüssel/Seite: & $\tfrac{2048}{10} = 204$\\
durschnittliche Anzahl Schlüssel/Seite & $\tfrac{5}{6} \cdot 204 = \boxed{170}$
\end{tabular}\\
\begin{tabular}{c r}
Ebenen-Anzahl & DS-Anzahl\\
1 & $20$\\
2 & $170 \cdot 20 = 3\,400$\\
3 & $170 \cdot 170 \cdot 20 = 578\,000$\\
4 & $170 \cdot 170 \cdot 170  \cdot 20 = 98\,260\,000$
\end{tabular}
\subsubsection{B*-Baum-Transformation (Splitting)}
\slides{4}{7}

Struktur der DB aus physischer Sicht:
\begin{itemize}
\item Verzeichnis der Seiten
\begin{itemize}
\item Freie Seiten
\end{itemize}
\item je Index
\begin{itemize}
\item Wurzelseite
\item Index-Schlüssel-Seiten
\item Datenseiten
\end{itemize}
\end{itemize}
Zugriff von „oben“ nach „unten“: Wurzelseite $\to$ Schlüssel-Seiten $\to$ Datenseite\bigskip\\
B*-Baum:
\begin{itemize}
\item HDD-Speicherung
\item optimiert für Einzel-Tupel-Zugriffe
\end{itemize}
\subsubsection{Ermittlung der benötigten Kapazität (Anzahl der Seiten)}
\slides{4}{8}
Beispiel:\\
\begin{tabular}{l r}
Tupelanzahl & $10\,000$\\
Spaltenanzahl & $10$\\
DS-Länge & $85$
\end{tabular}
\begin{align*}
\text{Seitenanzahl:}&=\frac{10\,000 \cdot (5+10+85)}{(2\,048-90)\cdot \left(1-\frac{10}{10\,000}\right)}\\
&= \frac{1\,000\,000}{1\,958\cdot 0,9}= \frac{1\,000\,000}{1\,762,2}\\
&=567,5 = 568 \unit{Seiten}
\end{align*}

\emph{Begriffswege}
\begin{itemize}
    \item getrennt von Datensätzen $\Rightarrow$ \emph{Index}
    \item im Datensatz $\Rightarrow$ \emph{Verkettung}
\end{itemize}

\section{Zugriffspfade außerhalb der Datensätze/Verkettung}
\subsection{Adressverkettung}
\slides{4}{9}
\subsubsection{Owner-Member}
\slides{4}{10}
\subsubsection{Owner1-Member-Owner2}
\slides{4}{11}

\subsection{Anfragebeispiel auf relationales DBMS}
\slides{4}{12}

\section{Speicherung auf Basis von  Bitmap-Indizes}
Problemstellung: Index für Abfrage nutzen?\\
Strategie klassischer DBMS:\\
Index: B$^*$-Baum\\
1. Abfrage (ja, Index nutzen, da Einzel-DS lesen)
\begin{lstlisting}[language=SQL]
SELECT Name, Vorname
FROM Kunde
WHERE Kunr=345
\end{lstlisting}
2. Abfrage (nein, Index nicht nutzen, da alle DS lesen)
\begin{lstlisting}[language=SQL]
SELECT AVG(Kredit), Ort
FROM Kunde
GROUP BY (Ort)
\end{lstlisting}
\subsection{Vertikale Positionierung und Bitmap Index}
\slides{4}{13}
\subsection{Beispiel für I/O-Reduzierung}
\slides{4}{14}

\subsection{Beispiel Index für relationales DBMS}
\slides{4}{15}

\subsection{Fast Projection}
\slides{4}{16}
\subsubsection{FFP und FFFP}
\slides{4}{17}
\subsection{Low Fast}
\slides{4}{18}

\subsection{High Group}
\slides{4}{19}

\subsection{High Non Group}
\slides{4}{20}

\subsection{Beispiel für den Einsatz der Indexarten}
\slides{4}{21}

\chapter{Projektierung von relationalen Datenbanksystemen (Betriebliche Datenmodellierung und Datenbankanwendungen)}
Datenbank-Anwendungssystem:
\begin{itemize}
\item DBS
\begin{itemize}
\item DBMS
\begin{itemize}
\item Datenbasis
\end{itemize}
\end{itemize}
\item Nutzer/Anwender-Programme
\end{itemize}
\section{Lebenszyklus von Datenbankanwendungen}
\begin{enumerate}
\item Systemdefinition
\item Entwurf
\item Implementierung
\item Dateneingabe und -analyse
\item Test und Validierung
\item Nutzung
\item Verbesserung und Wartung / Anpassung
\end{enumerate}
\section{Überblick über den Entwurfsprozess von DBS}
\begin{itemize}
\item Anforderungsanalyse
\item Konzeptioneller Entwurf
\item Logischer Entwurf
\item Implementierungsentwurf
\item Physischer Entwurf
\item Wartung
\end{itemize}
\subsection{Diskussion ausgewählter Problemfelder}
\begin{tabular}{l | l | l | l | l | l | l}
& Analyse & Entwurf & Implem. & Test & Installation & Nutzung\\\hline
Fehlerkosten & 0,2 & 0,5 & 1 & 2 & 5 & 10-20\\
Fehlerentdeckung & bei Nutzung & im Test & in Impl. & & \\
Aufwand (soll) & hoch & hoch & hoch & sehr hoch & hoch & gering\\
Aufwand (oft) &  sehr gering & gering & mittel & hoch & sehr hoch & extrem hoch\\

\end{tabular}
\subsection{Mitarbeiteranzahl und Bearbeitungsdauer}
Eigentlich: mit steigender Anzahl der Mitarbeiter sollte die Bearbeitungsdauer linear sinken.\\
Aber: Kommunikationswege heben die Dauer wieder. Zusammen mit den Kommunikationswegen schaffen zu viele Mitarbeiter wieder mehr Aufwand als dass sie die Dauer senken $\to$ Badewannenkurve. Die optimale Anzahl an Mitarbeitern ist wichtig.
\subsection{Software-Entwicklung}
\begin{itemize}
\item Coding-Guides
\begin{itemize}
\item Namens-Konventionen
\item Kommentierung
\end{itemize}
\item Software-Tests
\begin{itemize}
\item automatisiert (Unit-Tests)
\end{itemize}
\item formale Methoden
\begin{itemize}
\item Versionsverwaltung
\item Ticketsysteme
\item Netzplantechnik
\item Ressourcenplanung
\item Terminplanung
\end{itemize}
\item soziale Kompetenz
\end{itemize}
\section{Phasen des Entwurfsprozesses}
\setCustomSlideScale{.7}
\slides{5_1}{1}
\subsection{Anforderungsanalyse}
\subsection{Konzeptueller Entwurf}
\subsection{Logischer Entwurf}
\subsection{Implementierungsentwurf}
\subsection{Physischer Entwurf}
\subsection{Nutzung/Wartung}

\chapter{Überblick zu aktuellen Entwicklungen auf dem Gebiet DBS}

%\newpage
%\printbibliography

\end{document}