%\slides{STTI_intro_2017}{7}
\section{Einführung in IoT-Protokolle}
\lecdate{10.04.2017}
\slideScale{.65}
\subsection{Das Szenario}
\slides{IoT/STTI_SoSe2017_CloudMQTT}{2}
\subsection{Schaltzentrale IoT-Hub}
\slides{IoT/STTI_SoSe2017_CloudMQTT}{3}
\subsubsection{Dashboards}
\slides{IoT/STTI_SoSe2017_CloudMQTT}{4}
\subsubsection{Einfache Regeln}
\slides{IoT/STTI_SoSe2017_CloudMQTT}{5}

\subsection{Business Rules via iLog}
\slides{IoT/STTI_SoSe2017_CloudMQTT}{6}

\subsection{IoT-Baukasten -- Node RED}
\slides{IoT/STTI_SoSe2017_CloudMQTT}{7}

\section{Enterprise Integration Pattern}
\slides{IoT/STTI_SoSe2017_CloudMQTT}{8}

Übertragungsoptionen:
\begin{itemize}
\item Über Verbindung (bspw. http)\\
Nachteil: beide Gerät müssen an sein für Kommunkation
\item Über Datei\\
Nachteil: Struktur der Datei muss bekannt sein\\
Vorteil: beide Geräte müssen nicht an sein (bspw. über USB-Stick)
\item Über Datenbank\\
Nachteil: zusätzlicher Verwaltungsaufwand nötig\\
Vorteil: klare Struktur
\end{itemize}

\subsection{Anwendungsintegration}

\subsubsection{Point-to-Point (P2P)}
\slides{IoT/STTI_SoSe2017_CloudMQTT}{9}

\subsubsection{Message Bus}
\slides{IoT/STTI_SoSe2017_CloudMQTT}{10}

\subsubsection{Publish-Subscribe}
\slides{IoT/STTI_SoSe2017_CloudMQTT}{11}

\subsubsection{P2P versus Publish-Subscribe}
\slides{IoT/STTI_SoSe2017_CloudMQTT}{12}

\subsubsection{Durable Subscriber}
\slides{IoT/STTI_SoSe2017_CloudMQTT}{13}
(Kombination aus Publish-Subscribe und P2P)

\section{Messaging mit Java = JMS}
\slides{IoT/STTI_SoSe2017_CloudMQTT}{14}

\section{MQTT -- sprachunabhängig und leichtgewichtig}
\slides{IoT/STTI_SoSe2017_CloudMQTT}{15}
(MQTT: Message Queue Telemetry Transport)
\subsection{Features}
\slides{IoT/STTI_SoSe2017_CloudMQTT}{16}
\subsection{Quality of Service}
\slides{IoT/STTI_SoSe2017_CloudMQTT}{17}
(QoS: Quality of Service)

\subsection{Populäre Broker und Clients}
\slides{IoT/STTI_SoSe2017_CloudMQTT}{18}

\subsection{Java Anwendung}
\subsubsection{Publisher}
\slides{IoT/STTI_SoSe2017_CloudMQTT}{20}
\subsubsection{Subscriber}
\slides{IoT/STTI_SoSe2017_CloudMQTT}{21}
\slides{IoT/STTI_SoSe2017_CloudMQTT}{22}

\section{Aufgaben}
\slides{IoT/STTI_SoSe2017_CloudMQTT}{25}
\begin{enumerate}
\item Siehe Java-files
\item Siehe Java-files
\item Siehe Java-files
\item \begin{itemize}
\item QoS 0: Temperatur (nicht schlimm, wenn eine Nachricht verloren geht)
\item QoS 1: Stückzahl Zähler (Mehrfachempfang von Nachricht ist nicht relevant)
\item QoS 2: Überweisung (Nachricht muss \emph{genau} ein Mal empfangen werden)
\end{itemize}
\item „Persistent Sessions“ setzt das „Durable Subscriber Pattern“ um:\\
Bei häufigen Verbindungsabbrüchen von Clients kann der Broker eine persistente Session vorhalten. Wenn ein Client sich erneut verbindet schickt der Broker alle verpassten Nachrichten für seine Subscriptions an den Client.\\
Dafür muss \lstinline$cleanSessions$ auf \lstinline$false$ gesetzt werden.
\end{enumerate}











