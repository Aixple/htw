\newcommand{\customDir}{../}
\RequirePackage{ifthen,xifthen}

% Input inkl. Umlaute, Silbentrennung
\RequirePackage[T1]{fontenc}
\RequirePackage[utf8]{inputenc}

% Arbeitsordner (in Abhängigkeit vom Master) Standard: .LateX_master Ordner liegt im Eltern-Ordner
\providecommand{\customDir}{../}
\newcommand{\setCustomDir}[1]{\renewcommand{\customDir}{#1}}
%%% alle Optionen:
% Doppelseitig (mit Rand an der Innenseite)
\newboolean{twosided}
\setboolean{twosided}{false}
% Eigene Dokument-Klasse (alle KOMA möglich; cheatsheet für Spicker [3 Spalten pro Seite, alles kleiner])
\newcommand{\customDocumentClass}{scrreprt}
\newcommand{\setCustomDocumentClass}[1]{\renewcommand{\customDocumentClass}{#1}}
% Unterscheidung verschiedener Designs: htw, fjs
\newcommand{\customDesign}{htw}
\newcommand{\setCustomDesign}[1]{\renewcommand{\customDesign}{#1}}
% Dokumenten Metadaten
\newcommand{\customTitle}{}
\newcommand{\setCustomTitle}[1]{\renewcommand{\customTitle}{#1}}
\newcommand{\customSubtitle}{}
\newcommand{\setCustomSubtitle}[1]{\renewcommand{\customSubtitle}{#1}}
\newcommand{\customAuthor}{}
\newcommand{\setCustomAuthor}[1]{\renewcommand{\customAuthor}{#1}}
%	Notiz auf der Titelseite (A: vor Autor, B: nach Autor)
\newcommand{\customNoteA}{}
\newcommand{\setCustomNoteA}[1]{\renewcommand{\customNoteA}{#1}}
\newcommand{\customNoteB}{}
\newcommand{\setCustomNoteB}[1]{\renewcommand{\customNoteB}{#1}}
% Format der Signatur in Fußzeile:
\newcommand{\customSignature}{\ifthenelse{\equal{\customAuthor}{}} {} {\footnotesize{\textcolor{darkgray}{Mitschrift von\\ \customAuthor}}}}
\newcommand{\setCustomSignature}[1]{\renewcommand{\customSignature}{#1}}
% Format des Autors auf dem Titelblatt:
\newcommand{\customTitleAuthor}{\textcolor{darkgray}{Mitschrift von \customAuthor}}
\newcommand{\setCustomTitleAuthor}[1]{\renewcommand{\customTitleAuthor}{#1}}
% Standard Sprache
\newcommand{\customDefaultLanguage}[1]{}
\newcommand{\setCustomDefaultLanguage}[1]{\renewcommand{\customDefaultLanguage}{#1}}
% Folien-Pfad (inkl. Dateiname ohne Endung und ggf. ohne Nummerierung)
\newcommand{\customSlidePath}{}
\newcommand{\setCustomSlidePath}[1]{\renewcommand{\customSlidePath}{#1}}
% Folien Eigenschaften
\newcommand{\customSlideScale}{0.5}
\newcommand{\setCustomSlideScale}[1]{\renewcommand{\customSlideScale}{#1}}
\newcommand{\customSlideHeight}{9.63cm}
\newcommand{\setCustomSlideHeight}[1]{\renewcommand{\customSlideHeight}{#1}}
\newcommand{\customSlideWidth}{12.8cm}
\newcommand{\setCustomSlideWidth}[1]{\renewcommand{\customSlideWidth}{#1}}

%\setboolean{twosided}{true}
%\setCustomDocumentClass{scrartcl}
%\setCustomDesign{htw}
\setCustomTitle{Betriebssysteme I}
\setCustomSubtitle{Vorlesungs- und Praktikumsnotizen}
\setCustomAuthor{Falk-Jonatan Strube}
%\setCustomNoteA{Titlepage Note before Author}
\setCustomNoteB{Vorlesung und Praktikum von \\Prof. Dr.-Ing. Baumgartl}

%\setcustomSignature{\footnotesize{\textcolor{darkgray}{Mitschrift von\\ \customAuthor}}	% Formatierung der Signatur in der Fußzeile
%\setcustomTitleAuthor{\textcolor{darkgray}{Mitschrift von #1}}	% Formatierung des Autors auf dem Titelblatt

%-- Prüfen, ob Beamer
\ifthenelse{\equal{\customDocumentClass}{beamer}}{
%%% TODO: andere Layouts für Beamer außer HTW
	\documentclass[ignorenonframetext, 11pt, table]{beamer}
	
	\usenavigationsymbolstemplate{}
	\setbeamercolor{author in head/foot}{fg=black}
	\setbeamercolor{title}{fg=black}
	\setbeamercolor{bibliography entry author}{fg=htworange!70}
	%\setbeamercolor{bibliography entry title}{fg=blue} 
	\setbeamercolor{bibliography entry location}{fg=htworange!60} 
	\setbeamercolor{bibliography entry note}{fg=htworange!60}  
	
	\setbeamertemplate{itemize item}{\color{black}$\bullet$}
	\setbeamertemplate{itemize subitem}{\color{black}--}
	\setbeamertemplate{itemize subsubitem}{\color{black}$\bullet$}
	\makeatother
	\setbeamertemplate{footline}
	{
	\leavevmode
	\def\arraystretch{1.2}
	\arrayrulecolor{gray}
	\begin{tabular}{ p{0.167\textwidth} | p{0.491\textwidth} | p{0.089\textwidth} | p{0.103\textwidth}}
	\hline
	\strut\insertshortauthor & \insertshorttitle & Slide \insertframenumber{}% / \inserttotalframenumber{}
	 & May 4, 2016\\
	\end{tabular}
	}
	\setbeamertemplate{headline}
	{
	\leavevmode
	\setlength{\arrayrulewidth}{1pt}
	\hspace*{2em}	
	\begin{tabular}{p{0.63\textwidth}}
	\rule{0pt}{3em}\normalsize{\textbf{\insertsection\strut}}\\
	\arrayrulecolor{htworange}
	\hline
	\end{tabular}
	\begin{tabular}{l}
	\rule{0pt}{4em}\includegraphics[width=3.25cm]{\customDir .LaTeX_master/HTW_GESAMTLOGO_CMYK.eps}\\
	\end{tabular}
	}
	\makeatletter	
}{	
	%-- Für Spicker einiges anders:
	\ifthenelse{\equal{\customDocumentClass}{cheatsheet}}{
		\documentclass[a4paper,10pt,landscape]{scrartcl}
		\usepackage{geometry}
		\geometry{top=2mm, bottom=2mm, headsep=0mm, footskip=0mm, left=2mm, right=2mm}
		
		% Für Spicker \spsection für Section, zur Strukturierung \HRule oder \HDRule Linie einsetzen
		\usepackage{multicol}
		\newcommand{\spsection}[1]{\textbf{#1}}	% Platzsparende "section" für Spicker
	}{	%-- Ende Spicker-Unterscheidung-if
		%-- Unterscheidung Doppelseitig
		\ifthenelse{\boolean{twosided}}{
			\documentclass[a4paper,11pt, footheight=26pt,twoside]{\customDocumentClass}
			\usepackage[head=23pt]{geometry}	% head=23pt umgeht Fehlerwarnung, dafür größeres "top" in geometry
			\geometry{top=30mm, bottom=22mm, headsep=10mm, footskip=12mm, inner=27mm, outer=13mm}
		}{
			\documentclass[a4paper,11pt, footheight=26pt]{\customDocumentClass}
			\usepackage[head=23pt]{geometry}	% head=23pt umgeht Fehlerwarnung, dafür größeres "top" in geometry
			\geometry{top=30mm, bottom=22mm, headsep=10mm, footskip=12mm, left=20mm, right=20mm}
		}
		%-- Nummerierung bis Subsubsection für Report
		\ifthenelse{\equal{\customDocumentClass}{report} \OR \equal{\customDocumentClass}{scrreprt}}{
			\setcounter{secnumdepth}{3}	% zählt auch subsubsection
			\setcounter{tocdepth}{3}	% Inhaltsverzeichnis bis in subsubsection
		}{}
	}%-- Ende Spicker-Unterscheidung-else
	
	\usepackage{scrlayer-scrpage}	% Kopf-/Fußzeile
	\renewcommand*{\thefootnote}{\fnsymbol{footnote}}	% Fußnoten-Symbole anstatt Zahlen
	\renewcommand*{\titlepagestyle}{empty} % Keine Seitennummer auf Titelseite
	\usepackage[perpage]{footmisc}	% Fußnotenzählung Seitenweit, nicht Dokumentenweit
}

% Input inkl. Umlaute, Silbentrennung
\RequirePackage[T1]{fontenc}
\RequirePackage[utf8]{inputenc}
\usepackage[english,ngerman]{babel}
\usepackage{csquotes}	% Anführungszeichen
\RequirePackage{marvosym}
\usepackage{eurosym}

% Style-Aufhübschung
\usepackage{soul, color}	% Kapitälchen, Unterstrichen, Durchgestrichen usw. im Text
%\usepackage{titleref}

% Mathe usw.
\usepackage{amssymb}
\usepackage{amsthm}
\ifthenelse{\equal{\customDocumentClass}{beamer}}{}{
\usepackage[fleqn,intlimits]{amsmath}	% fleqn: align-Umgebung rechtsbündig; intlimits: Integralgrenzen immer ober-/unterhalb
}
%\usepackage{mathtools} % u.a. schönere underbraces
\usepackage{xcolor}
\usepackage{esint}	% Schönere Integrale, \oiint vorhanden
\everymath=\expandafter{\the\everymath\displaystyle}	% Mathe Inhalte werden weniger verkleinert
\usepackage{wasysym}	% mehr Symbole, bspw \lightning
% Auch arcus-Hyperbolicus-Funktionen
\DeclareMathOperator{\arccot}{arccot}
\DeclareMathOperator{\arccosh}{arccosh}
\DeclareMathOperator{\arcsinh}{arcsinh}
\DeclareMathOperator{\arctanh}{arctanh}
\DeclareMathOperator{\arccoth}{arccoth} 
%\renewcommand{\int}{\int\limits}
%\usepackage{xfrac}	% mehr fracs: sfrac{}{}
\let\oldemptyset\emptyset	% schöneres emptyset
\let\emptyset\varnothing
%\RequirePackage{mathabx}	% mehr Symbole
\mathchardef\mhyphen="2D	% Hyphen in Math

% tikz usw.
\usepackage{tikz}
\usepackage{pgfplots}
\pgfplotsset{compat=1.11}	% Umgeht Fehlermeldung
\usetikzlibrary{graphs}
%\usetikzlibrary{through}	% ???
\usetikzlibrary{arrows}
\usetikzlibrary{arrows.meta}	% Pfeile verändern / vergrößern: \draw[-{>[scale=1.5]}] (-3,5) -> (-3,3);
\usetikzlibrary{automata,positioning} % Zeilenumbruch im Node node[align=center] {Text\\nächste Zeile} automata für Graphen
\usetikzlibrary{matrix}
\usetikzlibrary{patterns}	% Schraffierte Füllung
\usetikzlibrary{shapes.geometric}	% Polygon usw.
\tikzstyle{reverseclip}=[insert path={	% Inverser Clip \clip
	(current page.north east) --
	(current page.south east) --
	(current page.south west) --
	(current page.north west) --
	(current page.north east)}
% Nutzen: 
%\begin{tikzpicture}[remember picture]
%\begin{scope}
%\begin{pgfinterruptboundingbox}
%\draw [clip] DIE FLÄCHE, IN DER OBJEKT NICHT ERSCHEINEN SOLL [reverseclip];
%\end{pgfinterruptboundingbox}
%\draw DAS OBJEKT;
%\end{scope}
%\end{tikzpicture}
]	% Achtung: dafür muss doppelt kompliert werden!
\usepackage{graphpap}	% Grid für Graphen
\tikzset{every state/.style={inner sep=2pt, minimum size=2em}}
\usetikzlibrary{mindmap, backgrounds}
%\usepackage{tikz-uml}	% braucht Dateien: http://perso.ensta-paristech.fr/~kielbasi/tikzuml/

% Tabular
\usepackage{longtable}	% Große Tabellen über mehrere Seiten
\usepackage{multirow}	% Multirow/-column: \multirow{2[Anzahl der Zeilen]}{*[Format]}{Test[Inhalt]} oder \multicolumn{7[Anzahl der Reihen]}{|c|[Format]}{Test2[Inhalt]}
\renewcommand{\arraystretch}{1.3} % Tabellenlinien nicht zu dicht
\usepackage{colortbl}
\arrayrulecolor{gray}	% heller Tabellenlinien
\usepackage{array}	% für folgende 3 Zeilen (für Spalten fester breite mit entsprechender Ausrichtung):
\newcolumntype{L}[1]{>{\raggedright\let\newline\\\arraybackslash\hspace{0pt}}m{\dimexpr#1\columnwidth-2\tabcolsep-1.5\arrayrulewidth}}
\newcolumntype{C}[1]{>{\centering\let\newline\\\arraybackslash\hspace{0pt}}m{\dimexpr#1\columnwidth-2\tabcolsep-1.5\arrayrulewidth}}
\newcolumntype{R}[1]{>{\raggedleft\let\newline\\\arraybackslash\hspace{0pt}}m{\dimexpr#1\columnwidth-2\tabcolsep-1.5\arrayrulewidth}}
\usepackage{caption}	% Um auch unbeschriftete Captions mit \caption* zu machen

% Nützliches
\usepackage{verbatim}	% u.a. zum auskommentieren via \begin{comment} \end{comment}
\usepackage{tabto}	% Tabs: /tab zum nächsten Tab oder /tabto{.5 \CurrentLineWidth} zur Stelle in der Linie
\NumTabs{6}	% Anzahl von Tabs pro Zeile zum springen
\usepackage{listings} % Source-Code mit Tabs
\usepackage{lstautogobble} 
\ifthenelse{\equal{\customDocumentClass}{beamer}}{}{
\usepackage{enumitem}	% Anpassung der enumerates
%\setlist[enumerate,1]{label=(\arabic*)}	% global andere Enum-Items
\renewcommand{\labelitemiii}{$\scriptscriptstyle ^\blacklozenge$} % global andere 3. Item-Aufzählungszeichen
}
\newenvironment{anumerate}{\begin{enumerate}[label=(\alph*)]}{\end{enumerate}} % Alphabetische Aufzählung
\usepackage{letltxmacro} % neue Definiton von Grundbefehlen
% Nutzen:
%\LetLtxMacro{\oldemph}{\emph}
%\renewcommand{\emph}[1]{\oldemph{#1}}
\RequirePackage{xpatch}	% ua. Konkatenieren von Strings/Variablen (etoolbox)


% Einrichtung von lst
\lstset{
basicstyle=\ttfamily, 
%mathescape=true, 
%escapeinside=^^, 
autogobble, 
tabsize=2,
basicstyle=\footnotesize\sffamily\color{black},
frame=single,
rulecolor=\color{lightgray},
numbers=left,
numbersep=5pt,
numberstyle=\tiny\color{gray},
commentstyle=\color{gray},
keywordstyle=\color{green},
stringstyle=\color{orange},
morecomment=[l][\color{magenta}]{\#}
showspaces=false,
showstringspaces=false,
breaklines=true,
literate=%
    {Ö}{{\"O}}1
    {Ä}{{\"A}}1
    {Ü}{{\"U}}1
    {ß}{{\ss}}1
    {ü}{{\"u}}1
    {ä}{{\"a}}1
    {ö}{{\"o}}1
    {~}{{\textasciitilde}}1
}
\usepackage{scrhack} % Fehler umgehen
\def\ContinueLineNumber{\lstset{firstnumber=last}} % vor lstlisting. Zum wechsel zum nicht-kontinuierlichen muss wieder \StartLineAt1 eingegeben werden
\def\StartLineAt#1{\lstset{firstnumber=#1}} % vor lstlisting \StartLineAt30 eingeben, um bei Zeile 30 zu starten
\let\numberLineAt\StartLineAt

% BibTeX
\usepackage[backend=bibtex8, bibencoding=ascii,
%style=authortitle, citestyle=authortitle-ibid,
%doi=false,
%isbn=false,
%url=false
]{biblatex}	% BibTeX
\usepackage{makeidx}
%\makeglossary
%\makeindex

% Grafiken
\usepackage{graphicx}
\usepackage{epstopdf}	% eps-Vektorgrafiken einfügen
%\epstopdfsetup{outdir=\customDir}

% pdf-Setup
\usepackage{pdfpages}
\ifthenelse{\equal{\customDocumentClass}{beamer}}{}{
\usepackage[bookmarks,%
bookmarksopen=false,% Klappt die Bookmarks in Acrobat aus
colorlinks=true,%
linkcolor=black,%
citecolor=red,%
urlcolor=green,%
]{hyperref}
}

%-- Unterscheidung des Stils
\newcommand{\customLogo}{}
\newcommand{\customPreamble}{}
\ifthenelse{\equal{\customDesign}{htw}}{
	% HTW Corporate Design: Arial (Helvetica)
	\usepackage{helvet}
	\renewcommand{\familydefault}{\sfdefault}
	\renewcommand{\customLogo}{HTW-Logo}
	\renewcommand{\customPreamble}{HTW Dresden}
}{
% \renewcommand{\customLogo}{HTW-Logo.eps}
}

% Nach Dokumentenbeginn ausführen:
\AtBeginDocument{
	% Autor und Titel für pdf-Eigenschaften festlegen, falls noch nicht geschehen
	\providecommand{\pdfAuthor}{John Doe}
	\ifdefempty{\customAuthor} {} {\renewcommand{\pdfAuthor}{\customAuthor}}
	\providecommand{\pdfTitle}{}
	\providecommand{\pdfTitleA}{}
	\providecommand{\pdfTitleB}{}
	\providecommand{\pdfTitleC}{}	
	\ifdefempty{\pdfTitle}{
		\ifdefempty{\customPreamble} {} {\renewcommand{\pdfTitleA}{\customPreamble{} | }}
		\ifdefempty{\customTitle} {\renewcommand{\pdfTitleB}{No Title}} {\renewcommand{\pdfTitleB}{\customTitle}}
		\ifdefempty{\customSubtitle} {} {\renewcommand{\pdfTitleC}{ - \customSubtitle}}
	}{}
	
	\newcommand{\customLogoLocation}{\customDir .LaTeX_master/\customLogo}
	\hypersetup{
		pdfauthor={\pdfAuthor},
		pdftitle={\pdfTitleA\pdfTitleB\pdfTitleC},
	}
	\ifthenelse{\equal{\customDocumentClass}{beamer}}{
		\title{\customTitle}
		\author{\customAuthor}
	}{
		\automark[section]{section}
		\automark*[subsection]{subsection}
		\pagestyle{scrheadings}
		\ifthenelse{\equal{\customDocumentClass}{report} \OR \equal{\customDocumentClass}{scrreprt}}{
		\renewcommand*{\chapterpagestyle}{scrheadings}
		}{}
		%\renewcommand*{\titlepagestyle}{scrheadings}
		\ihead{\includegraphics[height=1.7em]{\customLogoLocation}}
		%\ohead{\truncate{5cm}{\customTitle}}
		\ohead{\customTitle}
		\cfoot{\pagemark}
		\ofoot{\customSignature}
		% Titelseite
		\title{
		\includegraphics[width=0.35\textwidth]{\customDir .LaTeX_master/\customLogo}\\\vspace{0.5em}
		\Huge\textbf{\customTitle}
		\ifdefempty{\customSubtitle} {} {\\\vspace*{0.7em}\Large \customSubtitle}
		\\\vspace*{5em}}
		\author{
		\ifdefempty{\customNoteA} {} {\customNoteA \vspace*{1em}}\\ 
		\ifdefempty{\customAuthor} {} {\customTitleAuthor}
		\ifdefempty{\customNoteB}{}{\vspace*{1em}\\\customNoteB}
		}
		
		\ifthenelse{\equal{\customDocumentClass}{cheatsheet}}{
			\pagestyle{empty}
			\setlist{nolistsep}
	%		\usepackage{parskip}	% Aufzählung Abstand
	%		\setlength{\parskip}{0em}
			\lstset{
	    belowcaptionskip=0pt,
	    belowskip=0pt,
	    aboveskip=0pt,
			tabsize=2,
			frame=none,
			numbers=none,
			showspaces=false,
			showstringspaces=false,
			breaklines=true,
			}
		}{}
	}
}

% Unterabschnitte
%\newtheorem{example}{Beispiel}%[section]
%\newtheorem{definition}{Definition}[section]
%\newtheorem{discussion}{Diskussion}[section]
%\newtheorem{remark}{Bemerkung}[section]
%\newtheorem{proof}{Beweis}[section]
%\newtheorem{notation}{Schreibweise}[section]
\RequirePackage{xcolor}
\RequirePackage{amsmath}

% Horizontale Linie:
\newcommand{\HRule}[1][\medskipamount]{\par
  \vspace*{\dimexpr-\parskip-\baselineskip+#1}
  \noindent\rule[0.2ex]{\linewidth}{0.2mm}\par
  \vspace*{\dimexpr-\parskip-.5\baselineskip+#1}}
% Gestrichelte horizontale Linie:
\RequirePackage{dashrule}
\newcommand{\HDRule}[1][\medskipamount]{\par
  \vspace*{\dimexpr-\parskip-\baselineskip+#1}
  \noindent\hdashrule[0.2ex]{\linewidth}{0.2mm}{1mm} \par
  \vspace*{\dimexpr-\parskip-.5\baselineskip+#1}}
% Mathe in Anführungszeichen:
\newsavebox{\mathbox}\newsavebox{\mathquote}
\makeatletter
\newcommand{\mq}[1]{% \mathquotes{<stuff>}
  \savebox{\mathquote}{\text{"}}% Save quotes
  \savebox{\mathbox}{$\displaystyle #1$}% Save <stuff>
  \raisebox{\dimexpr\ht\mathbox-\ht\mathquote\relax}{"}#1\raisebox{\dimexpr\ht\mathbox-\ht\mathquote\relax}{''}
}
\makeatother

% Paragraph mit Zähler (Section-Weise)
\newcounter{cparagraphC}
\newcommand{\cparagraph}[1]{
\stepcounter{cparagraphC}
\paragraph{\thesection{}-\thecparagraphC{} #1}
%\addcontentsline{toc}{subsubsection}{\thesection{}-\thecparagraphC{} #1}
\label{\thesection-\thecparagraphC}
}
\makeatletter
\@addtoreset{cparagraphC}{section}
\makeatother


% (Vorlesungs-)Folien einbinden:
% Folien von einer Datei skaliert
\newcommand{\slide}[2][\customSlideScale]{\slides[#1]{}{#2}}
\newcommand{\slideTrim}[6][\customSlideScale]{\slides[#1 , clip,  trim = #5cm #4cm #6cm #3cm]{}{#2}}
% Folien von mehreren nummerierten Dateien skaliert
\newcommand{\slides}[3][\customSlideScale]{\begin{center}
\includegraphics[page=#3, scale=#1]{\customSlidePath #2.pdf}
\end{center}}

% \emph{} anders definieren
\makeatletter
\DeclareRobustCommand{\em}{%
  \@nomath\em \if b\expandafter\@car\f@series\@nil
  \normalfont \else \scshape \fi}
\makeatother

% unwichtiges
\newcommand{\unimptnt}[1]{{\transparent{0.5}#1}}

% alph. enumerate
\newenvironment{anumerate}{\begin{enumerate}[label=(\alph*)]}{\end{enumerate}} % Alphabetische Aufzählung

%% EINFACHE BEFEHLE

% Abkürzungen Mathe
\newcommand{\EE}{\mathbb{E}}
\newcommand{\QQ}{\mathbb{Q}}
\newcommand{\RR}{\mathbb{R}}
\newcommand{\CC}{\mathbb{C}}
\newcommand{\NN}{\mathbb{N}}
\newcommand{\ZZ}{\mathbb{Z}}
\newcommand{\PP}{\mathbb{P}}
\renewcommand{\SS}{\mathbb{S}}
\newcommand{\cA}{\mathcal{A}}
\newcommand{\cB}{\mathcal{B}}
\newcommand{\cC}{\mathcal{C}}
\newcommand{\cD}{\mathcal{D}}
\newcommand{\cE}{\mathcal{E}}
\newcommand{\cF}{\mathcal{F}}
\newcommand{\cG}{\mathcal{G}}
\newcommand{\cH}{\mathcal{H}}
\newcommand{\cI}{\mathcal{I}}
\newcommand{\cJ}{\mathcal{J}}
\newcommand{\cM}{\mathcal{M}}
\newcommand{\cN}{\mathcal{N}}
\newcommand{\cP}{\mathcal{P}}
\newcommand{\cR}{\mathcal{R}}
\newcommand{\cS}{\mathcal{S}}
\newcommand{\cZ}{\mathcal{Z}}
\newcommand{\cL}{\mathcal{L}}
\newcommand{\cT}{\mathcal{T}}
\newcommand{\cU}{\mathcal{U}}
\newcommand{\cV}{\mathcal{V}}
\renewcommand{\phi}{\varphi}
\renewcommand{\epsilon}{\varepsilon}

% Verschiedene als Mathe-Operatoren
\DeclareMathOperator{\arccot}{arccot}
\DeclareMathOperator{\arccosh}{arccosh}
\DeclareMathOperator{\arcsinh}{arcsinh}
\DeclareMathOperator{\arctanh}{arctanh}
\DeclareMathOperator{\arccoth}{arccoth} 
\DeclareMathOperator{\var}{Var} % Varianz 
\DeclareMathOperator{\cov}{Cov} % Co-Varianz 

% Farbdefinitionen
\definecolor{red}{RGB}{180,0,0}
\definecolor{green}{RGB}{75,160,0}
\definecolor{blue}{RGB}{0,75,200}
\definecolor{orange}{RGB}{255,128,0}
\definecolor{yellow}{RGB}{255,245,0}
\definecolor{purple}{RGB}{75,0,160}
\definecolor{cyan}{RGB}{0,160,160}
\definecolor{brown}{RGB}{120,60,10}

\definecolor{itteny}{RGB}{244,229,0}
\definecolor{ittenyo}{RGB}{253,198,11}
\definecolor{itteno}{RGB}{241,142,28}
\definecolor{ittenor}{RGB}{234,98,31}
\definecolor{ittenr}{RGB}{227,35,34}
\definecolor{ittenrp}{RGB}{196,3,125}
\definecolor{ittenp}{RGB}{109,57,139}
\definecolor{ittenpb}{RGB}{68,78,153}
\definecolor{ittenb}{RGB}{42,113,176}
\definecolor{ittenbg}{RGB}{6,150,187}
\definecolor{itteng}{RGB}{0,142,91}
\definecolor{ittengy}{RGB}{140,187,38}

\definecolor{htworange}{RGB}{249,155,28}

% Textfarbe ändern
\newcommand{\tred}[1]{\textcolor{red}{#1}}
\newcommand{\tgreen}[1]{\textcolor{green}{#1}}
\newcommand{\tblue}[1]{\textcolor{blue}{#1}}
\newcommand{\torange}[1]{\textcolor{orange}{#1}}
\newcommand{\tyellow}[1]{\textcolor{yellow}{#1}}
\newcommand{\tpurple}[1]{\textcolor{purple}{#1}}
\newcommand{\tcyan}[1]{\textcolor{cyan}{#1}}
\newcommand{\tbrown}[1]{\textcolor{brown}{#1}}

% Umstellen der Tabellen Definition
\newcommand{\mpb}[1][.3]{\begin{minipage}{#1\textwidth}\vspace*{3pt}}
\newcommand{\mpe}{\vspace*{3pt}\end{minipage}}

\newcommand{\resultul}[1]{\underline{\underline{#1}}}
\newcommand{\parskp}{$ $\\}	% new line after paragraph
\newcommand{\corr}{\;\widehat{=}\;}
\newcommand{\mdeg}{^{\circ}}

\newcommand{\nok}[2]{\binom{#1}{#2}}	% n über k BESSER: \binom{n}{k}
\newcommand{\mtr}[1]{\begin{pmatrix}#1\end{pmatrix}}	% Matrix
\newcommand{\dtr}[1]{\begin{vmatrix}#1\end{vmatrix}}	% Determinante (Betragsmatrix)
\renewcommand{\vec}[1]{\underline{#1}}	% Vektorschreibweise
\newcommand{\imptnt}[1]{\colorbox{red!30}{#1}}	% Wichtiges
\newcommand{\intd}[1]{\,\mathrm{d}#1}
\newcommand{\diffd}[1]{\mathrm{d}#1}
% für Module-Rechnung: \pmod{}
\newcommand{\unit}[1]{\,\mathrm{#1}}

%\bibliography{\customDir _Literatur/HTW_Literatur.bib}

\begin{document}

%\selectlanguage{english}
\maketitle
\newpage
\tableofcontents
\newpage

\chapter*{Hinweise}
Bei der Klausur sind keine Hilfsmittel zugelassen.

Die Praktikumsaufgaben, die mit * gekennzeichnet sind, sind nicht Voraussetzung für die Prüfung.

\chapter{Einführung}

\paragraph{Unix} ist eine Betriebssystem-Familie.

\paragraph{Lizensierung:} open source VS closed source

GNU: „vererbendes“ open source
\section{Klassifikation BS}
\begin{itemize}
\item Nutzer: single/multi
\item Tasking: single/multi
\item Kommunikation: autonom (batch)/interaktiv
\item Verteilung: lokal/verteilt
\item Architektur und Zweck: Server/eingebettet/Echtzeit/PC/…
\end{itemize}

\section{Modellierung BS}
\begin{itemize}
\item \imptnt{monolithisch:} jede Routine/Funktion/… darf auf alles zugreifen (kein Information Hiding)
\item \imptnt{geschichtet:} Kommunikation nur zwischen benachbarten Schichten.

Bsp.: \begin{tabular}{|c|}
\hline
App\\
\hline
…\\
\hline
Treiber\\
\hline
…\\
\hline
Hardware\\
\hline
\end{tabular}
\item \imptnt{client-server:} Server arbeitet Wünsche von Clienten ab ($\Rightarrow$ Speicherverwaltung innerhalb BS)
\end{itemize}


\paragraph{Zweck eines BS:}
\begin{itemize}
\item Bereitstellen von Diensten + Abstraktionen (Prozess, Datei, Treiber, …)
\item Ressourcenverwaltung
\item Ablaufkoordination
\item Schutz + Sicherheit
\end{itemize}

\chapter{Shell}

\section{Wichtige Shell-Befehle}
\begin{longtable}[l]{ l | l } 
Befehl & Wirkung\\
\hline
\imptnt{ls} & Dateien in Ordner auflisten (-l: detailliert)\\
\imptnt{cd} & in Ordner wechseln\\
\imptnt{cp} & Datei kopieren\\
scp & Datei im Netzwerk kopieren\\
\imptnt{mv} & Datei bewegen\\
\imptnt{rm} dir & Datei löschen (-r: rekursiv durch Ordner)\\
\imptnt{mkdir} & Ordner anlegen\\
rmdir & leeren Ordner löschen\\
\imptnt{chmod} & Nutzerrechte ändern (u/g/o +/- r/w/x)\\
chown & Eigentümer ändern\\
\imptnt{less} & Dateiausgabe (seitenweise)\\
\imptnt{cat} file & Dateiausgabe im Terminal\\
w & Anzeige eingeloggter Nutzer und deren Prozesse\\
\imptnt{grep} & Durchsuche Datei nach Zeichenkette, gebe passende Zeilen aus \\
&(-o: gebe passende Worte aus (auch bei doppeltem Vorkommen in Zeile) \\
\imptnt{wc} -l-w-c & Word Count (-l Zeilen, -w Worte, -c Bytes)\\
\imptnt{cut} file & Zeilen beschneiden\\
& -d ' ': Worttrenner (hier: Leerzeichen)\\
& -f 1: Welche Worte anzeigen: -f 1 $\to$ 1. Wort; -f -3 $\to$ ab 3. Wort\\
\imptnt{uniq} & nur einzigartige Zeilen ausgeben\\
\imptnt{sort} & Zeilen alph. Sortieren (-n: numerisch; -r: in umgekehrter Reihenfolge\\
\imptnt{find} -name *test* & Suche Datei (-name: nach Name)\\
\imptnt{man} & Handbuch über alle Befehle\\
ps & aktuelle Information über Prozess \\
&(-A: alle Prozesse; r: Prozesse die bereit sind; X: Inhalt stackpointer, …;\\
& f: Verwandschaftsverhältnisse [besser: pstree]; -l: langes Format)\\
kill & Signal senden\\
bg & Schickt Programm in den Hintergrund\\
top & Anzeige der rechenintensivsten Programme\\
mount & Datenträger einbinden\\
du & Anzeige Platzbedarf von Datei(en) (-s: Summe aller)\\
df & Anzeige Belegung Dateisystem\\
ln & Verweis erstellen\\
shred & sicheres Löschen\\
stat & Anzeige von Dateiattributen\\
fdisk & Partitionierung\\
mkfs & Dateisystem anlegen\\
fsck & Prüfung+Reperatur Dateisystem\\
hdparm & Detailinformationen Massenspeicher\\
pgrep & Suche nach Prozess anhand von regexp.\\
nice & Setze Prozess-Priorität
\end{longtable}

\section[Weiterleitung]{\imptnt{Weiterleitung}}
\begin{tabular}{l l}
ls > ls.txt & Ausgabe von ls in Datei ls.txt (neu erstellt)\\
ls >> ls.txt & Ausgabe von ls in Datei ls.txt (ergänzend, hängt an)\\
foo < ls.txt & Datei ls in Eingabe von Prozess foo\\
& (wenn ls.txt mehrere Zeilen enthält, sind das mehrere Eingaben für bspw. read())\\
foo | bar & Ausgabe von foo in Eingabe von bar (siehe Pipe)
\end{tabular}\\
Ausgabemöglichkeiten:\\
\begin{tabular}{l l l}
stdin & 1 &\\
stdout & 1 &\\
stderr & 2 & find xyz 2>/dev/null (Fehler ins nichts umleiten) 
\end{tabular}

\section{Schleifen/Anweisungen}
\begin{lstlisting}[language=bash]
if [ ... ] 
then 
	... 
fi

while [ ... ] 
do 
	... 
done

for x in ...	(bspw. |\$|1/*)
do 
 ...
done

for (( x=0; |\$|x < |\$|y; x++ ))
do
 ...
done
\end{lstlisting}
\paragraph{Bedingungen:} $ $
\begin{lstlisting}[language=bash]
-lt, -le, -eq, -ne, -ge, -gt  (für numerische Vergleiche)
|\$\#|					Anzahl Parameter
-f ...			Prüft, ob ... Datei ist
"|\$|1" != ""	Prüft, ob Variable/Parameter leer ist
|\$|?					Rückgabewert letzter Funktion/Anweisung
|\$|*					gibt alle Parameter aus
|\$|RANDOM		Zufallszahl

return			max zwischen 0...255

x = '...' $\Leftrightarrow$ x = |\$|(...)			|\$|(...) weißt stdout einer Fkt einer Variablen zu
let x = |\$|a + |\$|b $\Leftrightarrow$ (( x = |\$|a + |\$|b ))
\end{lstlisting}

\section[Regex]{\imptnt{Regex}}
\begin{tabular}{c l}
. & beliebiges Zeichen\\
* & beliebig viele des vorhergehenden Zeichens\\
\textbackslash ? & 0 oder 1 des vorhergehenden Zeichens\\
\textbackslash + & 1 oder mehr des vorhergehenden Zeichens\\
$[$xyz$]$ & eines der in eckigen Klammern stehendes Zeichen\\
$[$\^{}xyz$]$ & alle außer eines der Klammer-zeichen\\
\^{} & Zeilenanfang\\
\$ & Zeilenende\\
$[$0-9$]$ & Bereich\\
$[[$:alpha:$]]$ & $\Leftrightarrow$ $[$a-z A-Z$]$\\
$[[$:digit:$]]$ & $[[$0-9$]]$\\
$[[$:alnum:$]]$ & $\Leftrightarrow$ alpha+digit\\
$[[$:upper:$]]$ & Großbuchstaben\\
$[[$:lower:$]]$ & Kleinbuchstaben\\
$[[$:space:$]]$ & Leerzeichen\\
\textbackslash . & Escape-Char \textbackslash $\Rightarrow$ . wird normal ausgegeben\\
\textbackslash < \textbackslash > & Wortanfang / -ende\\
\textbackslash \{ m,n \textbackslash \} & mind m, höchstens des vorhergehenden Zeichen\\
\textbackslash \{ m \textbackslash \} & m mal\\
\textbackslash ( xyz \textbackslash ) & xyz wird als Zeichen behandelt (zusammenfassen von Ausdrücken)\\
x \textbackslash | y & x oder y\\
\end{tabular}

\section{Brace Expansion}
\{a,b,c\}xyz $\to$ Permutation von a,b,c mit xyz: axyz, bxyz, cxyz\\
\{a..c\}\{x..z\} = \{a,b,c\}\{x,y,z\} $\to$ alle a,b,c mit x,y,z: ax, ay, az, bx, by, bz, cx, cy, cz\\
Auch geschachtelt: \{ \{a..z\}$\underset{oder}{,}$\{A..Z\} \} $\to$ alle Buchstaben: Kleinbuchstaben oder Großbuchstaben


\paragraph{Sorten von Dateien}
\begin{enumerate}
\item gewöhnliche Datei
\item Verzeichnis
\item Spezialdateien
\begin{enumerate} [label=$\to$]
\item Links
\item Geräte
\item …
\end{enumerate}
\end{enumerate}

\chapter{Dateisystem}
Abstraktion zur Strukturierung von Informationen\\
$\rightarrow$ Abhängig von: Geschwindigkeit des Mediums, Informationsmenge, Fehlertoleranz\\
Aufbau einer Festplatte:
\begin{itemize}
\item hat mehrere Platten in einem Zylinder
\item Platte hat Spuren (verschiedene Plattenradien) und Sektoren („Kuchenstücke“ der Platte)
\end{itemize}
\paragraph{Datei} (Grundeigenschaften)
\begin{itemize}
\item Eigentümer
\item Zugriffsrechte
\item Sichtbarkeit
\item Dateiname
\item Zeitstempel
\item Größe
\item Dateipositions-Zeiger
\item Typ ($\Rightarrow$ \imptnt{Magic Word} an Beginn der Datei)
\end{itemize}
\begin{tabular}{l | c | l | l}
\label{tab:syscalls}
Dateifktn C & & Systemruf Unix & … über Verzeichnis\\
\hline 
fopen() & & open() & opendir()\\
fclose() & & close() & closedir()\\
fread() & beim Verzeichnis: sequentiell über alle & read() & readdir()\\
fwrite() & schreiben & write()\\
fprintf() & schreiben (formatiert) & \\
feof() & Test auf Dateiende &\\
ferror() & Test auf Fehler &\\
fseek() & Versetzen Pos.-Zeiger & lseek()\\
ftell() & Abfragen Pos.-Zeiger &\\
flock() & Sperren der Datei &\\
& Verweis anlegen & link() & symlink() (Softlink)\\
& umbenennen & rename()\\
& Datei in Hauptspeicher einbleden & mmap()\\
& Verz. anlegen && mkdir()\\
& Löschen&& rmdir()\\
& Suche nach Einträgen&& scandir()\\
& Zurücksetzen Eintragszeiger & & rewinddir()
\end{tabular}
\paragraph{Rechteverwaltung} verschieden Möglichkeiten:\\
\imptnt{Zugriffsmatrix}\\
\begin{tabular}{l | l l }
& DateiA & DateiB\\
\hline
NutzerA & Rechte & Rechte\\
NutzerB & Rechte & 
\end{tabular}\\
Daraus die Projektionen:
\begin{itemize}
\item \imptnt{Access Control List (ACL)}:
\begin{itemize}
\item DateiA: NutzerA(Rechte), NutzerB(Rechte)
\item DateiB: NutzerA(Rechte)
\end{itemize}
\item \imptnt{Capability List}:
\begin{itemize}
\item NutzerA: DateiA(Rechte), DateiB(Rechte)
\item NutzerB: DateiA(Rechte)
\end{itemize}
\end{itemize}
\subparagraph{Darstellung in Unix} \parskp
\begin{tabular}{c | c | c l}
Owner & Group & Others & \quad All Users\\
u & g & o & \quad u+g+o$\Rightarrow$a\\
rwx & rwx & rwx
\end{tabular}\\
Beispiel in Bash:
\begin{lstlisting}[language=bash]
chmod u+rwx g+r-wx o-rwx FILE
\end{lstlisting}
Darstellung ls -l:\\
$\underset{\text{Typ}}{-}\;\underset{\text{Rechte (u g o)}}{rwxrwxrwx}\;\underset{\text{Anz. Verz.}}{1}\;\underset{\text{owner}}{nutzerA}\;\underset{\text{group}}{gruppeC}\;\underset{\text{size}}{1234}\;\underset{\text{last changed}}{2016-...07:42}\; \underset{\text{file name}}{FILE}$

\chapter{Ressource}
\begin{tabular}{l | l}
Schnittstelle & Protokoll\\
\hline
statischer Aspekt d. Kommunikation & dynamischer Aspekt d. Kommunikation\\
synchron/asynchron\\
Hardware/Software
\end{tabular}
\paragraph{Def. Ressource}
Wird von Aktivitäten genutzt.
\begin{itemize}
\item existiert in allen Schichten des Systems (Datei, Code, Festplatte (Hardware), …)
\item hat Zustand (Dateiinhalt, Registerinhalt, …)
\end{itemize}
\section{Ressource entziehen}
Voraussetzung Entziehbarkeit:
\begin{itemize}
\item Ressourcen-Zustand ist \imptnt{vollständig auslesbar}
\item Ressourcen-Zustand \imptnt{kann beliebig manipuliert werden}
\end{itemize}
Bsp.:\\
\begin{tabular}{l | l}
entziehbar & nicht entziehbar\\
\hline
CPU & CPU-Cache\\
Hauptspeicherblock & Drucker\\
& Netzwerkkarte
\end{tabular}
\paragraph{Vorgang} ist für Aktivität transparent
\begin{enumerate}
\item Aktivität anhalten
\item Ressourcenzustand sichern
\item (entzogene Ressource anderweitig verwenden)
\item Zustand restaurieren
\item Aktivität fortsetzen
\end{enumerate}
\paragraph{Exlusivität} 
\begin{itemize}
\item maximal von einer Aktivität geilchzeitig verwendbar (wird von BS durchgesetzt durch Zuteilung)
\end{itemize}
\section{Klassifikation}
\begin{tabular}{l | l}
\imptnt{entziehbar} & \imptnt{nicht entziehbar}\\
(Prozessor, Speicher) & (verbrauchbare Betriebsmittel, ext. Hardware)\\
\hline
\imptnt{gleichzeitig nutzbar} & \imptnt{exlusiv nutzbar}\\
(Code, Datei, Speicher) & (Prozessor, Drucker, Signal)\\
\hline
\imptnt{wiederverwendbar} & \imptnt{verbrauchbar}\\
(CPU, Datei, Speicher) & (Signal, Nachricht, Interrupt)\\
\hline
\imptnt{physisch} & \imptnt{logisch/virtuell}\\
(Hardware: CPU, RAM, …) & (Datei, Signal, CPU)
\end{tabular}
\section{Ressourcen-Transformation}
(in Ebenen)\\
\begin{tabular}{l | l}
Applikation & $\uparrow$ Byte der Datei\\
Dateisystem & $\uparrow$ Logischer Block\\
Treiber & $\uparrow$ physischer Sektor\\
Hardware & 
\end{tabular}

\section{Kernel}
\imptnt{Kernel-Modus} (Gegensatz: User-Modus)\\
Beschreibt einen previligierten Zugriff auf Hardware (bei entsprechender Unterstützung der CPU)\\
Nutzer hat nur eingeschränkte Rechte, Kernel exklusive:
\begin{itemize}
\item \imptnt{erstellen eines neuen Prozesses}
\item \imptnt{Treiber laden/entfernen}
\item $\Rightarrow$ \imptnt{Diensterbringung des BS}
\end{itemize}
Damit der Nutzer trotzdem auf diese Systemfunktionen zugreifen kann gibt es die:
\subsection{Systemrufe}
Anweisungen, mit denen ein User Kernel-Dienste nutze kann (siehe Tabelle im Kapitel \ref{tab:syscalls} (Dateisysteme))

\chapter{Aktivität}
\section{Prozesseigenschaften}
\begin{itemize}
\item hat Lebenszyklus: Erzeugung $\to$ Abarbeitung $\to $ Beendigung
\item benötigt Ressourcen beim Start (Speicher, PID, Code)
\item benötigt dynamisch Ressourcen beim Abarbeiten
\item hat eigenen virtuellen Prozessor und Speicher (Adressraum)
\item besitzt immer einen Vaterprozess (und ggf. einen Kindprozess)
\begin{itemize}
\item \imptnt{nur Prozess kann Prozess erzeugen} (bspw. durch Doppelklick auf Icon im GUI, über die Shell usw.)
\end{itemize}
\end{itemize}
\paragraph{Mögliche Prozesszustände}
\begin{itemize}
\item \imptnt{aktiv} (arbeitet)
\item \imptnt{bereit} (kann arbeiten)
\item \imptnt{wartend} (wartet auf Ressource zum Arbeiten)
\end{itemize}
Übergänge:
\begin{itemize}
\item \imptnt{aktiv $\to$ bereit} (Prozess wird verdrängt (bspw. durch Scheduling))
\item \imptnt{aktiv $\to$ wartend} (Prozess benötigt Ressource)
\item \imptnt{wartend $\to$ bereit} (Prozess hat alle benötigten Ressourcen)
\item \imptnt{bereit $\to $ aktiv} (Prozess beginnt (wieder) zu arbeiten)
\end{itemize}
Jeder Prozess starte zuerst im bereit-Zustand.
\paragraph{Adressraum Prozess} \parskp
\begin{tabular}{| c |}
\hline 
Umgebung, Argumente\\
\hline 
Stack\\
$\downarrow$\\
…\\
$\uparrow$\\
Heap\\
\hline
uninitialisierte Daten\\
\hline
initialisierte Daten\\
\hline
Text (Code)\\
\hline
\end{tabular}
\section{Prozess beenden}
\begin{itemize}
\item \imptnt{durch sich selbst}:
\begin{itemize}
\item Verlassen der main (durch letzte main-Zeile oder return)
\item exit() an beliebiger Stelle
\end{itemize}
\item Fremdbeedigung:
\begin{itemize}
\item Signal ($\Rightarrow$ kill)
\item fataler Fehler (bspw. 0-Division)
\end{itemize}
\end{itemize}

\section[Fork]{\imptnt{Fork}}
\begin{lstlisting}[language=C]
pid_t fork(void);
\end{lstlisting}
\begin{itemize}
\item klont aufrufenden Prozess
\begin{itemize}[label=$\to$]
\item nur unterschiedliche PID, PPID
\item geforkter Prozess \imptnt{setzt an gleicher Stelle fort}, wie aufrufender Prozess (also nach dem fork()-Aufruf)
\item Prozess-Abarbeitungsreihenfolge \imptnt{nicht determiniert} (Vater kann vor Sohn weiter abgearbeitet werden oder anders herum)
\item return-value von fork():\\
Vater:
\begin{itemize}
\item -1 (Fehler)
\item PID vom Sohn (Erfolg)
\end{itemize}
Sohn:
\begin{itemize}
\item 0
\end{itemize}
\item Variablen sind jeweils Privat (übernehmen jeweils die Werte wie vor dem fork, Änderungen im Vater/Sohn finden aber nur dort statt)
\end{itemize}
\end{itemize}
\begin{lstlisting}[language=C]
ret = fork();
if ( ret == -1 ) {
	perror("fork"); exit(EXIT_FAILURE);
}
if ( ret == 0 ) {
	// Sohn
} else {
	// Vater
	puts(ret);
}
\end{lstlisting}
\section[Exec]{\imptnt{Exec}}
\begin{lstlisting}[language=C]
int execl("Programmpfad", "Argumente...");
\end{lstlisting}
\begin{itemize}
\item ersetzt aktuellen Programmcode durch entsprechende Binärdatei aus Pfad
\begin{itemize} [label=$\to$]
\item springt diesen Code sofort an (kehrt nur im Fehlerfall zurück, bspw. bei falscher Pfadangabe oder mangelden Zugriffsrechten)
\item \imptnt{Rückkehr in Ausgangsprozess danach unmöglich!}
\end{itemize}
\item erzeugt keinen neuen Prozess (muss bspw. im Sohn nach fork() ausgeführt werden, wenn der Prozess (Vater) erhalten werden soll)!
\end{itemize}
\section{System}
\begin{lstlisting}[language=C]
int system("Kommando");
\end{lstlisting}
\begin{itemize}
\item kombiniert fork+exec und führt Kommando aus
\begin{itemize}[label=$\to$]
\item kehrt erst zurück, wenn Kommando abgearbeitet ist
\end{itemize}
\end{itemize}
\section[Wait]{\imptnt{Wait}}
\begin{lstlisting}[language=C]
pid_t wait (int *status);
\end{lstlisting}
\begin{itemize}
\item bringt aufrufenden Prozess \imptnt{in Wartezustand}, falls ein Kind vorhanden ist. Wartezustand wird \imptnt{beendet, sobald \underline{ein} (irgendein) Kind-Prozess beendet} wird.
\item Status kann u.a. erwarteter Rückkehrcode des Sohns sein.
\item return: -1 bei Fehler, Sohn-PID sonst
\end{itemize}
\chapter{Kommunikation}
IPC (inter process communication)\\
bspw. über: Datei, Pipe, Signal, shared memory, …\\
Unterscheidung bzgl. Anforderung:
\begin{itemize}
\item Teilnehmerzahl: 1:1, 1:m, n:1, n:m
\item Richtung: uni-/bidirektional
\item lokale/entfernte Kommunikation
\item direkte/indirekte Kommunikation
\end{itemize}
\section{Übertragungsarten}
\paragraph{\imptnt{Synchron}}
\begin{itemize}
\item wartet, bis Sende-/Empfangvorgang abgeschlossen ist
\end{itemize}
\paragraph{\imptnt{Asynchron}}
\begin{itemize}
\item Senden/Empfangen ohne „Bestätigung“
\begin{itemize}[label=$\to$]
\item unabhängig davon, ob etwas (oder wie viel bereits) gesendet oder empfangen wurde wird weiter gearbeitet
\item braucht Zwischenspeicher
\item gut, da kein Deadlock wegen Warten entstehen kann
\end{itemize}
\end{itemize}
\paragraph{\imptnt{Verbindungsorientiert}}
\begin{itemize}
\item Verbindungsabbau
\item Übertragung
\item Verbindungsabbau
\end{itemize}
bspw. Telefon, Pipe
\paragraph{\imptnt{Verbindungslos}}
\begin{itemize}
\item nur Übertragung
\end{itemize}
bpsw. Telegramm, Signal
\paragraph{Verbindungsarten}
\begin{tabular}{l l}
$1-1$ & unicast\\
$1-x_i$ & multicast\\
$1-all$ & broadcast (bpsw. in Subnetz)
\end{tabular}
\section[Übertragung über Datei]{Übertragung \imptnt{über Datei}}
Schlecht, da:
\begin{itemize}
\item doppelter Zugriff auf langsame HDD (nur gut, wenn Dateisystem im RAM)
\item überlappender Zugriff muss mit lockf() geregelt wender $\Rightarrow$ fehleranfällig
\end{itemize}

\section[Übertragung über Pipe]{Übertragung über \imptnt{Pipe}}
\begin{lstlisting}[language=C]
int pipe ( int filedes[2] );
\end{lstlisting}
\begin{itemize}
\item [2]: 2 Diskriptoren: 0 $\corr$ read, 1 $\corr$ write
\end{itemize}
\paragraph{Vorgehen}
\begin{enumerate}
\item pipe(x[2])
\item fork()
\item Prozesse schließen jeweils einen Diskriptor:
\begin{itemize}
\item Sohn: close(x[1]); (write geschlossen, also lesend)
\item Vater: close(x[0]); (read geschlossen, also schreibend)
\end{itemize}
\item Datenübertragung:
\begin{itemize}
\item Sohn: read(x[0], intoVar, length); (Sohn wartet hier, bis Vater etwas schickt)
\item Vater: write(x[1], text, lenght);
\end{itemize}
\item beide Prozesse rufen close() (des noch nicht geschlossenen Diskriptors) auf $\Rightarrow$ Pipe geschlossen
\end{enumerate}
\begin{lstlisting}[language=C]
int pipe[2]
ret = fork();
if ( ret == 0 ) {
	close(pipe[1]);
	x = read(pipe[0], var, 80);
	close(pipe[0]);
} else {
	close(pipe[0]);
	write(pipe[1], text, 80);
	close(pipe[1]);
}
\end{lstlisting}
Eine Pipe ist:
\begin{itemize}
\item unidirektional (sonst 2 Pipes nötig)
\item keine persistente Ressource (verschwindet nach close() aller Teilnehmer)
\item nur zwischen verwandten Prozessen möglich!
\end{itemize}
Pipe in Shell:
\begin{lstlisting}[language=bash]
du | sort -n -r | less
\end{lstlisting}
$\Rightarrow$stdout von du in stdin von sort

\subsection{popen}
\begin{lstlisting}[language=C]
FILE *popen (Kommando, Pipe-Typ [w/r]);
\end{lstlisting}
\begin{itemize}
\item legt Pipe an, forkt Prozess, der dann Kommando ausführt
\begin{itemize}
\item je nach Typ:
w: stdin vom Kommando zeigt auf Pipe $\Rightarrow$ Kommando liest
r: stdout vom Kommando zeigt auf Pipe $\Rightarrow$ Kommando schreibt
\end{itemize}
\item pclose() $\Rightarrow$ wieder schließen
\end{itemize}

\section[Signale]{\imptnt{Signale}}
\begin{itemize}
\item Informationsübertragung (in der Form von Anweisungen) ohne nötige Vererbung
\end{itemize}
\paragraph{Ablauf}
\begin{enumerate}
\item Sendeprozess generiert Signal
\item Signal zustellen (durch System)
\item Aufruf Signalhandler (falls vorhanden)
\item Aufruf default-Aktion (falls kein handler implementiert)
\end{enumerate}
Signal bpsw. SIGINT mit default-Aktion: Abbruch (auch durch Strg+C zu erreichen)
\subsection{Signalhandler}
\paragraph{Bash} $ $
\begin{lstlisting}[language=bash]
trap handleIt SIGINT
\end{lstlisting}
\paragraph{C} $ $
\begin{lstlisting}[language=C]
sig_t ret;

void handler (int c){
	// bspw.: default signal reaktivieren:
	ret = signal(SIGINT, SIG_DFL);
}

main (){
	ret = signal (SIGINT, (sig_t) &handler);
	if (ret == SIG_ERR){
		perror("signal"); exit(EXIT_FAILURE);
	}
}
\end{lstlisting}

\subsection[kill]{\imptnt{kill}}
Signal senden
\begin{lstlisting}[language=C]
int kill (pid_t pid, int sig);
\end{lstlisting}
\begin{itemize}
\item sendet Signal (falls Nutzer Berechtigung dazu hat)
\item falls pid=-1 $\Rightarrow$ senden an alle
\end{itemize}
\paragraph{Zustellung}
\begin{enumerate}
\item nicht abfangbare Signale ausführen
\item abfangbare Signale an handler geben, sonst default-Aktion ausführen
\end{enumerate}
Anweisung bzgl. Signalen:
\begin{itemize}
\item pause() - wartet auf Signal
\item alarm(x) - wartet x, dann wird Signal SIGALRM zugestellt
\end{itemize}
\paragraph{Einordnung}
\begin{itemize}
\item unzuverlässig
\item keine Nutzdaten-Übertragung
\item keine Priorisierung
\item keine Speicherung (Warteschlange)
\end{itemize}

\section{Shared Memory}
\begin{itemize}
\item gemeinsam genutzer Speicher
\item Segmente bleiben \imptnt{persistent}
\end{itemize}

\section{Message Passing}
\paragraph{Prinzip}
\begin{enumerate}
\item Sender trägt Nachricht in Puffer ein
\item Sender sendet: send()
\item System transportiert Nachricht
\item Empfänger empfängt: recieve()\\
und schreibt Daten in Puffer
\end{enumerate}
\begin{itemize}
\item bei nicht gemeinsamen Speicher nützlich
\item synchron und asynchron möglich
\end{itemize}

\chapter{Prozessorzuteilung}
Zielgrößen Prozess(or)-Verteilung:
\begin{itemize}
\item Durschnittliche Reaktionszeit Prozesse
\item Durschnittliche Verweilzeit Prozesse
\item maximale CPU-Auslastung
\item maximale Anzahl an Datenströmen
\item garantierte maximale Reaktionszeit vorhanden?
\item Fairness? n Prozesse $\to$ $\tfrac{1}{n}$ Prozessorzeit
\item Ausschluss des Verhungerns
\end{itemize}

\section[Off-Line Scheduling]{\imptnt{Off-Line Scheduling}}
\begin{itemize}
\item Ermittlung Abarbeitungsreihenfolge und Prozessorzuteilung vor Laufzeit
\begin{itemize}[label=$\to$]
\item -- inflexibel\\
+ hohe Auslastung möglich
\item alle Randbedingungen müssen bekannt sein
\end{itemize}
\end{itemize}
\imptnt{Präzendenzgraph} zum ermitteln (topologisch sortierter gerichteter Graph)\\
\begin{tikzpicture}[scale=1]
\draw  (0,0) node{3} ellipse (.3 and .3);
\node at (-.3,0.3) [left] {$p_{11}$};
\draw  (0,-2) node{5} ellipse (.3 and .3);
\node [left] at (-0.3,-1.7) {$p_{21}$};
\draw  (2,-1) node{2} ellipse (.3 and .3);
\node [right] at (2.3,-0.7) {$p_{12}$};
\draw [-latex] (0.4,-0.2) -- (1.6,-0.8);
\draw [-latex] (0.4,-1.8) -- (1.6,-1.2);
\end{tikzpicture}\\
$\Rightarrow$ \imptnt{Gantt-Diagramm}:\\
\begin{tikzpicture}[scale=.4]
\draw [-latex] (0,0) -- (8,0);
\draw  (0,1) rectangle (5,0) node [pos=.5]{$p_{21}$};
\draw  (0,2) rectangle (3,1) node [pos=.5]{$p_{11}$};
\draw  (5,2) rectangle (7,1) node [pos=.5]{$p_{12}$};
\draw (7,-.3) node[below]{$t$} -- (7,0.3);
\node at (0,1) [above left] {$P_1$};
\node at (0,1) [below left] {$P_2$};
\end{tikzpicture}
\section{On-Line Scheduling}
\begin{itemize}
\item Auswahl der Abarbeitungs-Reihenfolge zu Laufzeit
\begin{itemize}[label=$\to$]
\item + flexibel\\
-- keine Zeit für lange Auswahlverfahren $\Rightarrow$ Kompromiss bei Optimierung
\end{itemize}
\end{itemize}

\subsection[Zeitgesteuertes Scheduling]{\imptnt{Zeitgesteuertes Scheduling}}
\begin{itemize}
\item Abläufe periodisch
\item keine Interrupts
\end{itemize}
\paragraph{\imptnt{Time Division Multiple Access (TDMA)}}
\begin{itemize}
\item innerhalb jeder Periode wird Periodenlänge zwischen $n$ Teilnehmern aufgeteilt. Jeder hat $\tfrac{1}{n}$ Zeit (auch wenn nicht direkt genutzt wird). \\
$\Rightarrow$ keine Kollision möglich
\end{itemize}

\subsection[Ereignisgesteuertes Scheduling]{\imptnt{Ereignisgesteuertes Scheduling}}
\begin{itemize}
\item reagiert auf Einflüsse von außen
\item keine Garantie von Ausführungszeiten möglich, da Interrupts unvorhersehbar
\begin{itemize}[label = $\to$]
\item interaktive System (GUI, …)
\end{itemize}
\end{itemize}

\paragraph{\imptnt{Interrupt}} passiert asynchron zum Programmablauf. Bspw. durch IO-Geräte, BS, Programm, …

\subsection{Schedulingzeitpunkt}
\begin{itemize}
\item \imptnt{präemtptives Multiptasking}:\\
Unterbrechung jederzeit durch BS möglich (Prozess blockiert, bereit, fertig) [idR. an bestimmten „Preemption Points“)
\item \imptnt{kooperatives Multitasking}:\\
Freiwilliges Unterbrechen durch Prozess, bpsw. durch Systemaufrufe
\item oder wenn Aktivität komplett (run-to-completion)
\end{itemize}

\subsection[Priorisierung]{\imptnt{Priorisierung}}
Prozesse besitzen unterschiedliche Wichtigkeiten (unfair)
\begin{itemize}
\item \imptnt{statische Priorität}
\begin{itemize}
\item Priorität konstant
\item[+] einfacher Scheduler
\item[+] einfache Analyse
\item[--] nicht flexibel
\end{itemize}
Bsp.: \imptnt{Fixed External Priorities (FEP)}\\
jeder Prozess erhält vor LZ einen Prioritäts-Paramter zugeordnet\\
$\to$ zur LZ wird immer der höchste gewählt
\item \imptnt{dynamische Priorität}
\begin{itemize}
\item[--] periodische Neuberechnung der Abarbeitungsreinfolge $\Rightarrow$ Aufwand
\item[+] flexibel
\item[--] schwer zu analysieren
\end{itemize}
Bsp.: \imptnt{Implizite Prioritäten}\\
Priorität basiert auf bspw.:
\begin{itemize}
\item Joblänge
\item verbleibende Abarbeitungszeit
\item Zeit der letzten Aktualisierung
\item Deadline
\end{itemize}
\end{itemize}

\paragraph{Uni-/Multiprozessor} zusätzliche Probleme
\begin{itemize}
\item wo wird Prozess abgearbeitet (am besten unbeschäftigter Prozessor oder Prozessor auf dem Prozess zuvor lief)
\end{itemize}

\subsection[Round-Robin]{\imptnt{Round-Robin}}
Kenngrößen:\\
\imptnt{$t_q$} \quad Prozesszeit (Quantum)\\
\imptnt{$t_{cs}$} \quad Umschaltzeit

Je größer die Prozesszeit ist, desto träger kann das System werden (lange Reaktionszeit). Bei kurzer Prozesszeit ist die Umschaltzeit im vergleich relativ groß $\Rightarrow$ ineffizient (aber schnelle Reaktionszeit)

\subsection[FIFO/FCFS]{\imptnt{FIFO/FCFS}}
fair, leicht zu analysieren ($\to$ Warteschlange)
\begin{itemize}
\item Prozesse werden in Reihenfolge des Eintreffens vollständig abgearbeitet
\end{itemize}
\subsection[Shortest Job Next (SJN)]{\imptnt{Shortest Job Next (SJN)}}
\begin{itemize}
\item kleine Prozesse werden immer vor großen abgearbeitet\\
$\Rightarrow$ kann zum \imptnt{verhungern} führen, unfair!
\item Ausführzeit muss bekannt sein!
\end{itemize}

\section{Unix}
\begin{itemize}
\item Unterscheidet fürs Scheduling \imptnt{2 Arten von Prozessen}:
\begin{enumerate}
\item interaktive ($\to$ nutzen Zeit nicht aus (wartend))
\item rechnende
\end{enumerate}
\item \imptnt{bevorzugt interaktive}
\end{itemize}

\paragraph{Linux}
\begin{itemize}
\item Prioritäten + Zeitscheiben (Prioritäten von 19 (niedrigste) über 0 (normal) bis -20 (höchste))
\end{itemize}

\section[Priority Boost]{\imptnt{Priority Boost}}
Priorität wird durch langes Warten erhöht und durch Abarbeiten schrittweise verringert.
\end{document}
