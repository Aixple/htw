% Header aus der Vorlage
\documentclass[a4paper,11pt, footheight=26pt
%,twoside
]{scrreprt}
\usepackage[head=23pt]{geometry}	% head=23pt umgeht Fehlerwarnung, dafür größeres "top" in geometry
\geometry{a4paper, top=30mm, bottom=22mm,headsep=10mm, footskip=12mm
, left=20mm, right=20mm
%, inner=27mm, outer=13mm
}

% Zeile 2 (,twoside) und 7 (inner=...) für eine Druckversion (doppelseitig) ent-kommentieren (Rand für Hefter)

\setcounter{secnumdepth}{3}	% zählt auch subsubsection
\setcounter{tocdepth}{3}	% Inhaltsverzeichnis bis in subsubsection

% Input inkl. Umlaute, Silbentrennung
\usepackage[T1]{fontenc}
\usepackage[utf8]{inputenc}
\usepackage[ngerman]{babel}
\usepackage{csquotes}	% Anführungszeichen
\usepackage{eurosym}

% HTW Corporate Design: Arial (Helvetica)
\usepackage{helvet}
\renewcommand{\familydefault}{\sfdefault}

% Style-Aufhübschung
\usepackage{soul, color}	% Kapitälchen, Unterstrichen, Durchgestrichen usw. im Text
\usepackage{scrlayer-scrpage}	% Kopf-/Fußzeile
%\usepackage{titleref}
\usepackage[perpage]{footmisc}	% Fußnotenzählung Seitenweit, nicht Dokumentenweit
\renewcommand*{\thefootnote}{\fnsymbol{footnote}}	% Fußnoten-Symbole anstatt Zahlen
\renewcommand*{\titlepagestyle}{empty} % Keine Seitennummer auf Titelseite

% Mathe usw.
\usepackage{amssymb}
\usepackage[fleqn]{amsmath}	% fleqn: align-Umgebung rechtsbündig
\usepackage{xcolor}
\usepackage{esint}	% Schönere Integrale, \oiint vorhanden
\everymath=\expandafter{\the\everymath\displaystyle}	% Mathe Inhalte werden weniger verkleinert
\usepackage{wasysym}	% mehr Symbole, bspw \lightning
% Auch arcus-Hyperbolicus-Funktionen
\DeclareMathOperator{\arccot}{arccot}
\DeclareMathOperator{\arccosh}{arccosh}
\DeclareMathOperator{\arcsinh}{arcsinh}
\DeclareMathOperator{\arctanh}{arctanh}
\DeclareMathOperator{\arccoth}{arccoth} 
% Mathe in Anführungszeichen:
\newsavebox{\mathbox}\newsavebox{\mathquote}
\makeatletter
\newcommand{\mq}[1]{% \mathquotes{<stuff>}
  \savebox{\mathquote}{\text{"}}% Save quotes
  \savebox{\mathbox}{$\displaystyle #1$}% Save <stuff>
  \raisebox{\dimexpr\ht\mathbox-\ht\mathquote\relax}{"}#1\raisebox{\dimexpr\ht\mathbox-\ht\mathquote\relax}{''}
}
\makeatother

% tikz usw.
\usepackage{tikz}
\usepackage{pgfplots}
\pgfplotsset{compat=1.11}	% Umgeht Fehlermeldung
\usetikzlibrary{graphs}
%\usetikzlibrary{through}	% ???
\usetikzlibrary{arrows}
\usetikzlibrary{arrows.meta}	% Pfeile verändern / vergrößern: \draw[-{>[scale=1.5]}] (-3,5) -> (-3,3);
\usetikzlibrary{automata,positioning} % Zeilenumbruch im Node node[align=center] {Text\\nächste Zeile} automata für Graphen
\usetikzlibrary{matrix}
\usetikzlibrary{patterns}	% Schraffierte Füllung
\tikzstyle{reverseclip}=[insert path={	% Inverser Clip \clip
	(current page.north east) --
	(current page.south east) --
	(current page.south west) --
	(current page.north west) --
	(current page.north east)}
% Nutzen: 
%\begin{tikzpicture}[remember picture]
%\begin{scope}
%\begin{pgfinterruptboundingbox}
%\draw [clip] DIE FLÄCHE, IN DER OBJEKT NICHT ERSCHEINEN SOLL [reverseclip];
%\end{pgfinterruptboundingbox}
%\draw DAS OBJEKT;
%\end{scope}
%\end{tikzpicture}
]	% Achtung: dafür muss doppelt kompliert werden!
\usepackage{graphpap}	% Grid für Graphen
\tikzset{every state/.style={inner sep=2pt, minimum size=2em}}

% Tabular
\usepackage{longtable}	% Große Tabellen über mehrere Seiten
\usepackage{multirow}	% Multirow/-column: \multirow{2[Anzahl der Zeilen]}{*[Format]}{Test[Inhalt]} oder \multicolumn{7[Anzahl der Reihen]}{|c|[Format]}{Test2[Inhalt]}
\renewcommand{\arraystretch}{1.3} % Tabellenlinien nicht zu dicht
\usepackage{colortbl}
\arrayrulecolor{gray}	% heller Tabellenlinien
\usepackage{array}	% für folgende 3 Zeilen (für Spalten fester breite mit entsprechender Ausrichtung):
\newcolumntype{L}[1]{>{\raggedright\let\newline\\\arraybackslash\hspace{0pt}}m{\dimexpr#1\columnwidth-2\tabcolsep-1.5\arrayrulewidth}}
\newcolumntype{C}[1]{>{\centering\let\newline\\\arraybackslash\hspace{0pt}}m{\dimexpr#1\columnwidth-2\tabcolsep-1.5\arrayrulewidth}}
\newcolumntype{R}[1]{>{\raggedleft\let\newline\\\arraybackslash\hspace{0pt}}m{\dimexpr#1\columnwidth-2\tabcolsep-1.5\arrayrulewidth}}

% Nützliches
\usepackage{verbatim}	% u.a. zum auskommentieren via \begin{comment} \end{comment}
\usepackage{tabto}	% Tabs: /tab zum nächsten Tab oder /tabto{.5 \CurrentLineWidth} zur Stelle in der Linie
\NumTabs{6}	% Anzahl von Tabs pro Zeile zum springen
\usepackage{listings} % Source-Code mit Tabs
\usepackage{lstautogobble} 
\usepackage{enumitem}	% Anpassung der enumerates
\setlist[enumerate,1]{label=\arabic*.)}	% global andere Enum-Items
\newenvironment{anumerate}{\begin{enumerate}[label=\alph*.)]}{\end{enumerate}} % Alphabetische Aufzählung
\renewcommand{\labelitemiii}{$\scriptscriptstyle ^\blacklozenge$} % global andere 3. Item-Aufzählungszeichen
\usepackage{letltxmacro} % neue Definiton von Grundbefehlen
% Nutzen:
%\LetLtxMacro{\oldemph}{\emph}
%\renewcommand{\emph}[1]{\oldemph{#1}}

% Einrichtung von lst
\lstset{
basicstyle=\ttfamily, 
mathescape=true, 
%escapeinside=^^, 
autogobble, 
tabsize=2,
basicstyle=\footnotesize\sffamily\color{black},
frame=single,
rulecolor=\color{lightgray},
numbers=left,
numbersep=5pt,
numberstyle=\tiny\color{gray},
commentstyle=\color{gray},
keywordstyle=\color{green},
stringstyle=\color{orange},
morecomment=[l][\color{magenta}]{\#}
%showspaces=false,
showstringspaces=false,
breaklines=true,
literate=%
    {Ö}{{\"O}}1
    {Ä}{{\"A}}1
    {Ü}{{\"U}}1
    {ß}{{\ss}}1
    {ü}{{\"u}}1
    {ä}{{\"a}}1
    {ö}{{\"o}}1
    {~}{{\textasciitilde}}1
}
\usepackage{scrhack} % Fehler umgehen
\def\ContinueLineNumber{\lstset{firstnumber=last}} % vor lstlisting. Zum wechsel zum nicht-kontinuierlichen muss wieder \StartLineAt1 eingegeben werden
\def\StartLineAt#1{\lstset{firstnumber=#1}} % vor lstlisting \StartLineAt30 eingeben, um bei Zeile 30 zu starten
\let\numberLineAt\StartLineAt

% BibTeX
\usepackage[backend=bibtex, bibencoding=ascii]{biblatex}	% BibTeX
\usepackage{makeidx}
%\makeglossary
%\makeindex

% Grafiken
\usepackage{graphicx}
\usepackage{epstopdf}	% eps-Vektorgrafiken einfügen

% pdf-Setup
\usepackage{pdfpages}
\usepackage[bookmarks,%
bookmarksopen=false,% Klappt die Bookmarks in Acrobat aus
colorlinks=true,%
linkcolor=black,%
citecolor=red,%
urlcolor=green,%
]{hyperref}

% Titel, Autor usw. werden vor dem Anfang des Dokuments in einem Rutsch definiert…
\newcommand{\DTitel}[1]{\newcommand{\Dokumententitel}{#1}}
\newcommand{\DUntertitel}[1]{\newcommand{\Dokumentenuntertitel}{#1}}
\newcommand{\DAutor}[1]{\newcommand{\Dokumentenautor}{#1}}
\newcommand{\DNotiz}[1]{\newcommand{\Dokumentennotiz}{#1}}
\newcommand{\DSign}[1]{\newcommand{\Dokumentensignatur}{#1}}
\DSign{\footnotesize{\textcolor{darkgray}{Mitschrift von\\ \Dokumentenautor}}}
\newcommand{\Autorformat}[1]{\textcolor{darkgray}{Mitschrift von #1}}
\newcommand{\workingdir}{../}	% Arbeitsordner (in Abhängigkeit vom Master) Standard: LateX_master Ordner liegt im Eltern-Ordner
% … Deswegen folgendes erst Nach Dokumentenbeginn ausführen:
\AtBeginDocument{
	\hypersetup{
		pdfauthor={\Dokumentenautor},
		pdftitle={HTW Dresden | \Dokumententitel - \Dokumentenuntertitel},
	}
	\automark[section]{section}
	\automark*[subsection]{subsection}
	\pagestyle{scrheadings}
	\ihead{\includegraphics[height=1.7em]{\workingdir LaTeX_master/HTW-Logo.eps}}
	\ohead{\Dokumententitel}
	\cfoot{\pagemark}
	\ofoot{\Dokumentensignatur}
	% Titelseite
	\title{\includegraphics[width=0.35\textwidth]{\workingdir LaTeX_master/HTW-Logo.eps}\\\vspace{0.5em}
	\Huge\textbf{\Dokumententitel} \\\vspace*{0,5cm}
	\Large \Dokumentenuntertitel \\\vspace*{4cm}}
	\author{\Autorformat{\Dokumentenautor} \vspace*{1cm}\\\Dokumentennotiz}
}

%% EINFACHE BEFEHLE

% Abkürzungen Mathe
\newcommand{\EE}{\mathbb{E}}
\newcommand{\QQ}{\mathbb{Q}}
\newcommand{\RR}{\mathbb{R}}
\newcommand{\CC}{\mathbb{C}}
\newcommand{\NN}{\mathbb{N}}
\newcommand{\ZZ}{\mathbb{Z}}
\newcommand{\PP}{\mathbb{P}}
\renewcommand{\SS}{\mathbb{S}}
\newcommand{\cA}{\mathcal{A}}
\newcommand{\cB}{\mathcal{B}}
\newcommand{\cC}{\mathcal{C}}
\newcommand{\cD}{\mathcal{D}}
\newcommand{\cE}{\mathcal{E}}
\newcommand{\cF}{\mathcal{F}}
\newcommand{\cG}{\mathcal{G}}
\newcommand{\cH}{\mathcal{H}}
\newcommand{\cI}{\mathcal{I}}
\newcommand{\cJ}{\mathcal{J}}
\newcommand{\cM}{\mathcal{M}}
\newcommand{\cN}{\mathcal{N}}
\newcommand{\cP}{\mathcal{P}}
\newcommand{\cR}{\mathcal{R}}
\newcommand{\cS}{\mathcal{S}}
\newcommand{\cZ}{\mathcal{Z}}
\newcommand{\cL}{\mathcal{L}}
\newcommand{\cT}{\mathcal{T}}
\newcommand{\cU}{\mathcal{U}}
\newcommand{\cV}{\mathcal{V}}
\renewcommand{\phi}{\varphi}
\renewcommand{\epsilon}{\varepsilon}

% Farbdefinitionen
\definecolor{red}{RGB}{180,0,0}
\definecolor{green}{RGB}{75,160,0}
\definecolor{blue}{RGB}{0,75,200}
\definecolor{orange}{RGB}{255,128,0}
\definecolor{yellow}{RGB}{255,245,0}
\definecolor{purple}{RGB}{75,0,160}
\definecolor{cyan}{RGB}{0,160,160}
\definecolor{brown}{RGB}{120,60,10}

\definecolor{itteny}{RGB}{244,229,0}
\definecolor{ittenyo}{RGB}{253,198,11}
\definecolor{itteno}{RGB}{241,142,28}
\definecolor{ittenor}{RGB}{234,98,31}
\definecolor{ittenr}{RGB}{227,35,34}
\definecolor{ittenrp}{RGB}{196,3,125}
\definecolor{ittenp}{RGB}{109,57,139}
\definecolor{ittenpb}{RGB}{68,78,153}
\definecolor{ittenb}{RGB}{42,113,176}
\definecolor{ittenbg}{RGB}{6,150,187}
\definecolor{itteng}{RGB}{0,142,91}
\definecolor{ittengy}{RGB}{140,187,38}

% Textfarbe ändern
\newcommand{\tred}[1]{\textcolor{red}{#1}}
\newcommand{\tgreen}[1]{\textcolor{green}{#1}}
\newcommand{\tblue}[1]{\textcolor{blue}{#1}}
\newcommand{\torange}[1]{\textcolor{orange}{#1}}
\newcommand{\tyellow}[1]{\textcolor{yellow}{#1}}
\newcommand{\tpurple}[1]{\textcolor{purple}{#1}}
\newcommand{\tcyan}[1]{\textcolor{cyan}{#1}}
\newcommand{\tbrown}[1]{\textcolor{brown}{#1}}

% Umstellen der Tabellen Definition
\newcommand{\mpb}[1][.3]{\begin{minipage}{#1\textwidth}\vspace*{3pt}}
\newcommand{\mpe}{\vspace*{3pt}\end{minipage}}

\newcommand{\resultul}[1]{\underline{\underline{#1}}}
\newcommand{\parskp}{$ $\\}	% new line after paragraph
\newcommand{\corr}{\;\widehat{=}\;}
\newcommand{\mdeg}{^{\circ}}

\newcommand{\nok}[2]{\begin{pmatrix}#1\\#2\end{pmatrix}}	% n über k BESSER: \binom{n}{k}
\newcommand{\mtr}[1]{\begin{pmatrix}#1\end{pmatrix}}	% Matrix
\newcommand{\dtr}[1]{\begin{vmatrix}#1\end{vmatrix}}	% Determinante (Betragsmatrix)
\renewcommand{\vec}[1]{\underline{#1}}	% Vektorschreibweise
\newcommand{\imptnt}[1]{\colorbox{red!30}{#1}}	% Wichtiges
\newcommand{\intd}[1]{\,\mathrm{d}#1}

% Definition von Titel, Autor usw.
\DTitel{Mathematik I}
\DUntertitel{Vorlesungsskript}
\DAutor{Falk Jonatan Strube}
\DNotiz{Vorlesung von Herrn Meinhold}

\begin{document}

\pagenumbering{Roman}

\maketitle
\newpage
\tableofcontents
\newpage

\pagenumbering{arabic}

\part{Elementare Grundlagen}

\section{Aussagen und Grundzüge der Logik}
%\subsection{Aussagen, Wahrheitswert}

\paragraph{Aussage:} (im weiteren Sinne) Sprachlich sinnvoller, konsatierender Satz. In diesem Abschnitt werden nur zweiwertige Aussagen betrachtet, d.h. Aussagen, die entwoder wahr oder falsch sind.

\subparagraph{Bsp. 1:} 
\begin{itemize}
\item[(1)] Es gibt unendlich viele Primzahlen (wahr)
\item[(2)] Es gibt unendlich viele Primzahlzwillinge, z.B. (3,5), (5,7), (11,13), (17,19) usw. (Wahrheitswert nicth bekannt!)
\item[(3)] $5+7=13$ (falsch)
\item[(4)] Wie spät ist es? (keine Aussage)
\item[(5)] Diese Aussage ist falsch! (keine Aussage, paradox)
\item[(6)] Am 30.06.2016 wird es in Dresden regnen.
\end{itemize}

(1)--(3) sind zweiwertige Aussagen, (4) und (5) sind keine Aussagen, (6) ist keine zweiwertige Aussage (Wahrscheinlichkeit, d.h. Zahl zwischen 0 und 1 angebbar).

\subparagraph{Bezeichnungen:} \parskp
\begin{tabular}{l l}
p, q, r, &… Aussagen,\\
0 &… falsche Aussage, \\
1 &… wahre Aussage\\
\end{tabular}

\paragraph{Wahrheitswert:} \parskp
$w(p)=\begin{cases}
1 & \text{(falls p wahr)} \\
0 & \text{(fallls p falsch)}
\end{cases}$
    
$p \equiv q$ (p \emph{identisch} q) … p und q haben denselben Wahrheitswert
    
\subsection{Aussagesverschiebung}

\begin{enumerate}
\item \emph{Negation} $\overline{p}$ („nicht p“) [oft auch $p!$ bzw. $\neg p$]\\
\begin{tabular}{c | c}
$p$ & $\overline{p}$\\
\hline
0 & 1 \\
1 & 0 \\
\end{tabular}

\item \emph{Konjunktion} $p \wedge q$ („p und q“)
\item \emph{Disjunktion} $ p \vee q $ („p oder q“) [Alternative -- nicht ausschließendes Oder!]\\
\begin{tabular}{c | c |c |c}
$p$ & $q$ & $p \wedge q$ & $p \vee q$\\
\hline
1 & 1 &1&1\\
1 & 0 &0&1\\
0 & 1 &0&1\\
0 & 0 &0&0\\
\end{tabular}

\item \emph{Implikation} $(p \Rightarrow q ):\equiv \overline{p} \vee q$ („aus p folgt q“, „wenn p, dann q“)\\
\begin{tabular}{c | c |c|c}
$p$ & $q$ & $\overline{p}$ & $p\Rightarrow q$\\
\hline
1&1&0&1 \\
1&0&0&0\\
1&1&1&1\\
1&0&1&1\\
\end{tabular}\\
Begriffe: $p \Rightarrow q$ (p: \emph{Prämisse}, q: \emph{Konklusion})\\
Eine Implikation ist genau dann falsch, wenn die Prämisse richtig und die Konklusion falsch ist!
\paragraph{Bsp. 2:}
\begin{itemize}
\item $-1=1$ (falsch) $\Rightarrow$ $1=1$ (wahr) [durch Quadrieren]
\item $-1=1$ (falsch) $\Rightarrow$ $0=2$ (falsch) [Addition von 1]
\end{itemize}
Aus einer falschen Aussage lassen sich durch richtiges Schließen sowohl falsche als auch richtige Aussagen gewinnen.

Andere Sprechweisen: „p ist \emph{hinreichend} für q“, „q ist \emph{notwendig} für p“
\item \emph{Äquivalenz} $(p\Leftrightarrow q):\equiv (p\Rightarrow q) \wedge (q \Rightarrow p)$ („p äquivalent q“, „p ist notwendig und hinreichend für q“, „p genau dann wenn q“)\\
(ist genau dann wahr, wenn p und q den selben Wahrheitswert besitzen)
\end{enumerate}

\subsection{Logische Gesetze (Tautologien)}
Eine Tautologie $t$ ist eine Aussagenverbindung, die unabhängig vom Wahrheitswert der einzelnen Aussagen stets wahr ist (d.h. $t\equiv 1$).
\paragraph{Bsp. 3:}\parskp
Einige wichtige Tautologien
\begin{enumerate}
\item $p\Leftrightarrow \overline{\overline{p}}$ \tab \tab(Negation der Negation)
\item $p \vee \overline{p}$\tab \tab (Satz vom ausgeschlossenem Dritten)
\item a) $\overline{p\wedge q} \equiv \overline{p} \vee \overline{q}$\\
b) $\overline{p\vee q} \equiv \overline{p} \wedge \overline{q}$ \tab(de Morgansche Regeln)
\item $(p\Rightarrow q) \equiv (\overline{q} \Rightarrow \overline{p})$ \tab(Kontrapositionsgesetz)
\item $p\wedge (p\Rightarrow q)) \Rightarrow q$ \tab(direkter Beweis)
\item $p\wedge (\overline{q} \Rightarrow \overline{p})) \Rightarrow q$ \tab(indirekter Beweis)
\end{enumerate}
Beweise mittels Wahrheitstafeln (vgl. Übung 1).\\
Bemerkung zu 1., 3., 4.: Eine Äquivalenz ist genau dann eine Tautologie, wenn beide Seiten identisch sind, z.B. $p\equiv \overline{\overline{p}}$.

\paragraph{Beweistechniken:} \parskp
Zu beweisen ist $q$.
\begin{enumerate}
\item Direkter Beweis:
\begin{itemize}
\item Nachweis von $p$ (Voraussetzung)
\item Richtiger Schluss $p\Rightarrow q$\\
Dann $q$ wahr (Behauptung)
\end{itemize}
\item Indirekter Beweis: Annahme von $\overline{q}$ auf Wiederspruch führen (auf unterschiedliche Weise möglich, vgl. folgendes Bsp).
\end{enumerate}

\subparagraph{Bsp. 4:}\parskp
$q = $„$\sqrt{2}$ ist irrational“ (keine rationale Zahl)

Beweis indirekt: \\
Es gelte $\overline{q}$, d.h. $\sqrt{2}$ ist rational, dann gelten folgende Schlüsse: $\sqrt{2} = \frac{m}{n}$ mit teilerfremden natürlichen Zahlen $m$ und $n$.\\
$\Rightarrow 2=\frac{m^2}{n^2} \Rightarrow 2 \cdot n^2 = m^2 \Rightarrow 2|m^2$\\
$\Rightarrow \boxed{2|m}$ (2 ist Teiler von m)\\
$\Rightarrow 4|m^2$ (mit $m^2=2n^2$)\\
$\Rightarrow 4|2n^2 \Rightarrow 2|n^2 \Rightarrow \boxed{2|n}$\\
Widerspruch: Da $m$ und $n$ teilerfremd sind.	\#

\paragraph{Weitere Gesetze}

\begin{itemize}
\item $p\wedge q \equiv q \wedge p$\\$p\vee q \equiv q \vee p$ \tab \tab \tab(Kommutativgesetze)
\item $(p\wedge q)\wedge r \equiv p\wedge (q\wedge r)$\\$(p\vee q)\vee r \equiv p\vee (q \vee r)$ \tab \tab(Assoziativgesetze)
\item $(p\wedge q)\vee r \equiv (p \vee r) \wedge (q \vee r)$\\$(p\vee q)\wedge r \equiv (p\wedge r) \vee (q \wedge r)$ \tab(Distributivgesetze)
\item $p\wedge 1 \equiv p$, $p\vee 1 \equiv 1$, $p\wedge p \equiv p$\\$p\wedge 0 \equiv 0$, $p\vee 0 \equiv p$, $p\wedge p \equiv p$
\item $p \vee (p\wedge q ) \equiv p$ \tab \tab(Absorptionsgesetz)
\end{itemize}

\subsection{Aussagefunktionen, Quantoren, Prädikatenlogik} \label{subsec:Aussagefunktionen}
$X$ sei eine Menge (Gesamtheit von Objekten $x$ mit einem gemeinsamen Merkmal, vgl.  Abschnitt \ref{sec:Mengen})\\
$x\in X$ … $x$ ist Element von $X$. Die Objekte haben Eigenschaften (\emph{Prädikate})

\paragraph{Aussagefunktion} (auch Aussageform) $p(x)$:
Jedem $x\in X$ ist eine Aussage $p(x)$ zugeordnet. Dabei steht $x$ für ein Objekt, $p$ für ein Prädikat.
\subparagraph{Bsp. 5:}\parskp
$X$ … Menge der positiven natürlichen Zahlen (1, 2, 3, …)\\
$p(x):=$„$x$ ist eine Primzahl“\\
$p(5)$ … wahr, $p(10)$ … falsch

\paragraph{Quantoren:} \parskp
Betrachtet werden folgende Aussagen:
\begin{enumerate}
\item „Für alle $x$ (aus $X$) gilt $p(x)$“ $\equiv$ $\boxed{\forall x\; p(x)}$ (\emph{universeller Quantor} / Allquantor)
\item „Es existiert (wenigstens) ein $x$, für welches $p(x)$ gilt“ $\equiv$ $\boxed{\exists x \; p(x)}$ (\emph{existenzieller Quantor})
\end{enumerate}
Zur Schreibweise: 
\begin{itemize}
\item Bei Anwendungen (außerhalb der reinen Logik) wird oft die Grundmenke $X$ mit angegeben: \\
$\forall x \in X\; p(x)$ usw.
\item Falls sich Quantoren auf eine Teilmenge $M$ von $X$ beziehen sollen, dann können folgende Schreibweisen verwendet werden:\\
$a = \forall x \in M \; p(x)$, $b=\exists x\in M \; p(x)$.
\item Die Schreibweisen in der formalen Logik sind dann:\\
$a = \forall x \;(x \in M \Rightarrow p(x))$
\end{itemize}

\paragraph{Rechenregeln:}\parskp
$\boxed{\overline{\forall x \; p(x)} \equiv \exists x \; \overline{p(x)}}$\\
$\boxed{\overline{\exists x \; p(x)} \equiv \forall x \; \overline{p(x)}}$

\paragraph{Mehrstellige Aussagefunktionen}

\begin{itemize}
\item $p(x_1, x_2,\: ..., x_n)$,\quad $x_1 \in X_1, x_2 \in X_2,\: ... , x_n \in X_n$\\
Die Grundmengen $X_i$ können, müssen aber nicht für jede Stelle gleich sein.
\item Wird ein Quantor auf eine n-stellige Aussagefunktion angewandt, so entsteht eine (n-1)-stellige Aussagefunktion (eine 0-stellige Aussagefunktion ist eine Aussage)\\
z.B.: $\exists y \; p(x,y,z)=: q(x,z)$, die Variable $y$ wird durch den Quantor $\exists$ gebunden ($y$… gebundene Variable). Wichtig ist der Platz, nicht der Name der Variable.\\
$x, z$ … freie Variable, können durch weitere Quantoren gebunden werden.
\end{itemize}

\subparagraph{Bsp. 6:} \parskp
Ein Dorf bestehe aus 2 Teilen (Ober- und Unterdorf). Es sei $M$ die Menge aller Bewohner des Dorfes. $M_1$ bzw. $M_2$ seien die Teilmengen von $M$, die dem Ober- bzw. Unterdorf entsprechen. 

Wir betrachten folgende zweistellige Aussagefunktionen:\\
$k(x,y)$… Person $x$ (aus $M$) kennt Person $y$ (aus $M$)
\begin{enumerate} [label=\alph*)]
\item $a(x):= \forall y \; k(x,y)$… Person $x$ kennt jeden ($\Rightarrow$ „Für alle $y$ gilt: $x$ kennt $y$“)\\
$b(y):= \exists x \; k(x,y)$ … es gibt jemanden, der $y$ kennt\\
$c := \forall x \forall y \; k(x,y)$ … jeder kennt jeden\\
$d := \forall y \exists x \: k(x,y)$ … jeder wird von wenigstens einer Person gekannt\\
$e := \exists x \forall y \; k(x,y)$ … es gibt mindestens eine Person, die alle Personen kennt\\
Man beachte: \begin{itemize}
\item $d$ und $e$ sind nicht das Gleiche: Die Reihenfolge unterschiedlicher Quantoren muss beachtet werden. Bei $d$ kann für jedes $y$ ein anderes $x$ mit $k(x,y)$ existieren. Diese Abhängigkeit von $y$ wird manchmal in Anwendungen durch $\forall y \: \exists x(y) \; k(x,y)$ ausgedrückt.
\item Es gilt aber $e \Rightarrow d$ (stets wahr: Tautologie). Der Wahrheitsgehalt von z.B. $c, d, e$ kann dagegen nicht mit logischen Mitteln bestimmt werden.
\end{itemize}
\item Negation der Aussagen  bzw. Aussageformen aus a).\\
$\overline{a(x)}\equiv \exists y \; \overline{k(x,y)}$… $x$ kennt wenigstens eine Person nicht\\
$\overline{b(x)} \equiv \forall x \; \overline{k(x,y)}$ … keiner kennt $y$\\
$\overline{c} \equiv \exists x \; \overline{\forall y \; k(x,y)}\equiv \exists x \; \exists y \; \overline{k(x,y)}$ … es gibt jemanden der wenigstens eine Person nicht kennt (jemanden, der nicht alle kennt)\\
$\overline{d} \equiv \exists y \; \forall x \; \overline{k(x,y)}$… es gibt jemanden, der von keiner Person gekannt wird
$\overline{e}\equiv \forall x \; \exists y \; \overline{k(x,y)}$ … jeder kennt wenigstens eine Person nicht.
\item Folgende Aussagen sind mit Hilfe von Quantoren auszudrücken:\\
$f$… jeder aus dem Oberdorf kennt wenigstens eine Person aus dem Unterdorf.\\
$g$… es gibt jemanden im Unterdorf, der alle Personen des Oberdorfs kennt.
\begin{align*}f&=\forall x \in M_1 \; \exists y \in M_2 \; k(x,y)\\
&=\forall x \; (x \in M_1 \Rightarrow \exists y \; (y \in M_2 \wedge k(x,y)))\end{align*}
\begin{align*}
g&= \exists x \in M_2 \; \forall y \in M_1 \; k(x,y)\\
&= \exists x \; (x\in M_2 \wedge \forall y \; (y \in M_1 \Rightarrow k(x,y)))
\end{align*}
\end{enumerate}

\section{Mengen}\label{sec:Mengen}

\subsection{Begriffe}

\paragraph{Menge:} Zusammenfassung gewisser wohl unterscheidbarer Objekte (Elemente) mit einem gemeinsamen Merkmal zu einem Ganzen.

\subparagraph{Diskussion:} Naiver Mengenbegriff führt zu Widerpsrüchen. z.B. Menge $X$ aller Mengen, die sich nicht selbst als Element enthalten.

$X=\{A | A \; Menge, A\not\in A\}$\\
$X\in X$? Wenn $X\in X \Rightarrow X\not\in X$ und $X\not\in X\Rightarrow X \in X$ (Widerspruch!).

Diese Widersprüche können umgangen werden, wenn nur Teilmengen einer sogenannten Grundmenge betrachtet werden.

\subparagraph{Bezeichungen:}
\begin{itemize}
\item meist große Buchstaben für Mengen: $A, B, ..., M, ...,X$
\item $\boxed{x\in M}$… $x$ ist Element von $M$
\item $\boxed{x\not\in M}$… $x$ ist kein Element von $M$
\end{itemize}

\subparagraph{Schreibweise:} \parskp
$M=\{\underset{\text{Elemente}}{...}\}$ oder $M = \{ x | p(x)\}$\\
mit $p(x)=$ Aussage, die genau für die Elemente $x$ aus $M$ wahr ist.

\paragraph{Wichtige Grundmengen:}
\begin{itemize}
\item $\mathbb{N}$ … Menge der natürlichen Zahlen $\{0,1,2,3,...\}$
\item $\mathbb{N}^*=\mathbb{N}\setminus \{0\} = \{1,2,3,...\}$
\item $\mathbb{Z}$ … Menge der ganzen Zahlen $\{...,-3,-2,-1,0,1,2,3,...\}$
\item $\mathbb{Q}$ … Menge der rationaln Zahlen $\{ x | x = \frac{m}{n}, m\in \mathbb{Z}, n \in \mathbb{Z}, n\not= 0\}$
\item $\mathbb{R}$ … Menge der reelen Zahlen
\item $\mathbb{C}$ … Menge der komplexen Zahlen $\{ z | z=x+ i \cdot  y,\quad x, y \in \mathbb{R}, i^2=-1\}$
\end{itemize}

\subparagraph{Bsp. 1:} \parskp
$M_1$ … Menge der Primzahlen kleiner 10, $M_1=\{2,3,5,7\}$\\
$M_2$ … Menge der reelen Zahlen zwischen 0 und 1 $M_2=\{x \in \mathbb{R}| 0<x<1\} =: \underset{\text{Intervallschreibweise}}{(0,1)}$

\paragraph{Def. 1:} (Intervallschreibweisen) \\
Es seien $a$ und $b$ reele Zahlen mit $a<b$:\\
$[a,b]:=\{ x \in \mathbb{R} | a \le x \le b\}$ … abgeschlossenes Intervall\\
$(a,b):= \{ x \in \mathbb{R} | a < x < b\}$ … offenes Intervall\\
$[a,b):= \{ x \in \mathbb{R}  | a \le x < b\}$\\
$(-\infty , a ) := \{ x \in \mathbb{R} | - \infty < x < a \} = \{x \in \mathbb{R} | x < a\}$\\
usw.

\subparagraph{Leere Menge:} z.B. $\{ x \in \mathbb{R} | x =x+1\}= \{x \in \mathbb{R} | x^2+1=0\}$ enthält kein Element.\\
Bezeichnung: $\emptyset$ oder $\{\}$

\subsection{Mengenverknüpfungen}

\paragraph{Def. 2:} \parskp
$\boxed{M_1 = M_2} : \equiv \boxed{\forall x \; (x\in M_1 \Leftrightarrow x \in M_2}$ (\emph{Gleichheit})
\paragraph{Def. 3:} \parskp
$\boxed{M_1 \subseteq M_2} : \equiv \boxed{\forall x \; (x\in M_1 \Rightarrow x \in M_2}$ (\emph{Inkulsion}) „$M_1$ ist Teilmenge von $M_2$“

\subparagraph{Diskussion:} \parskp
Ist $M_1 \subseteq M_2$ aber $M_1\not = M_2$ so kann man schreiben $M_1\subset M_2$ (echte Teilmenge).

\paragraph{Def. 4:}
\begin{enumerate}
\item \mpb 
$A\cap B := \{x | x \in A \wedge x \in B\}$\\
\emph{Durchschnitt} von $A$ und $B$
\mpe \mpb \begin{tikzpicture} [scale =0.25]
\node at (-7,3) {A};
\node at (1,3) {B};
\draw  (-5,3) ellipse (3 and 3);
\draw  (-1,3) ellipse (3 and 3);
\begin{scope} % lokal halten der Wirkung von \clip
\clip (-5,3) ellipse (3 and 3);
\draw[pattern= north east lines, pattern color=gray] (-1,3) ellipse (3 and 3);
\end{scope}
\end{tikzpicture} 
\mpe
\item \mpb
$A \cup B := \{ x | x\in A \vee x \in B\}$\\
\emph{Vereinigung} von $A$ und $B$
\mpe \mpb \begin{tikzpicture} [scale =0.25]
\draw [pattern= north east lines, pattern color=gray]  (-5,3) ellipse (3 and 3) (-1,3) ellipse (3 and 3);
\node at (-7,3) {A};
\node at (1,3) {B};
\end{tikzpicture}
\mpe
\item \mpb
$A \setminus B := \{ x | x \in A \wedge x \not\in B\}$\\
\emph{Differenz} „$A$ minus $B$“
\mpe \mpb \begin{tikzpicture} [scale =0.25, remember picture]
\draw  (-5,3) ellipse (3 and 3);
\draw  (-1,3) ellipse (3 and 3);
\begin{scope}
\begin{pgfinterruptboundingbox} % To make sure our clipping path does not mess up the placement of the picture
\draw [clip] (-5,3) ellipse (3 and 3) [reverseclip];
\end{pgfinterruptboundingbox}
\draw[pattern= north east lines, pattern color=gray] (-1,3) ellipse (3 and 3);
\end{scope}
\node at (-7,3) {A};
\node at (1,3) {B};
\end{tikzpicture} 
\mpe
\item \mpb[0.4] Bei Vorliegen einer Grundmenge $E$:\\
$\overline{A} := E \setminus A$\\
\emph{Komplimentärmenge} von $A$
\mpe \mpb \begin{tikzpicture} [scale =0.25, remember picture]
\draw  (-9,-1) rectangle (3,7);
\draw  (-5,3) ellipse (3 and 3);
\begin{scope}
\begin{pgfinterruptboundingbox} % To make sure our clipping path does not mess up the placement of the picture
\draw [clip]  (-5,3) ellipse (3 and 3) [reverseclip];
\end{pgfinterruptboundingbox}
\draw[pattern= north east lines, pattern color=gray](-9,-1) rectangle (3,7);
\end{scope}
\node at (-7,3) {A};
\node at (1,3) {E};
\end{tikzpicture}
\mpe
\end{enumerate}
\subparagraph{Diskussion:} (ausgewählte Rechenregeln)
\begin{enumerate}
\item $\cup$ und $\cap$ sind kommutativ und assoziativ\\
z.B. gilt $A \cup B = B\cup A$, $(A\cap B)\cap C = A \cap (B \cap C) = A \cap B \cap C$
\item Allg. $I$ … Indexmenge, z.B. $\{1,2,...,n\}$, $\mathbb{N}$, $\mathbb{Z}$, $\mathbb{R}$ dann:\\
$\bigcup\limits_{i\in I}A_i := \{x| \exists i \in I \quad x \in A_i\}$\\
$\bigcap\limits_{i\in I}A_i := \{x| \forall i \in I \quad x \in A_i\}$
\end{enumerate}

\subsection{Relationen}

\subsubsection{Grundbegriffe}

\paragraph{Def. 5:} \parskp
Die Menge $M_1 \times M_2 := \{ (x_1,x_2) | x_1 \in M_1 \wedge x_2 \in M_2\}$ heißt \emph{kartesisches Produkt} der Mengen $M_1$ und $M_2$ (= Menge aller geordneten Paare)

\subparagraph{Bsp. 2:} \parskp
$\mathbb{R}$ … Menge der reelen Zahlen, veranschaulicht durch die Zahlengerade\\
$\mathbb{R}^2 := \mathbb{R} \times \mathbb{R} = \{ (x,y) | x \in \mathbb{R} \wedge y \in \mathbb{R}\}$ … x-y-Ebene

\paragraph{Def. 6:} \parskp
Eine Teilmenge $T \subseteq M_1 \times M_2$ heißt \emph{(binäre) Relation}.

\subparagraph{Diskussion:}
\begin{enumerate}
\item Verallgemeinerung: $M_1 \times M_2 \times ... \times M_n= \{ (x_1, x_2, ..., x_n) | x_1 \in M_1, ..., x_n \in M_n\}$
(= Menge geordneter n-Tupel)\\
Eine Teilmenge $T\subseteq M_1 \times M_2 \times ... \times M_n$ heißt \emph{n-stellige Relation}.
\item \emph{Jede} Teilmenge von $M_1\times M_2$ ist eine Relation, also auch die Grenfälle $\emptyset$ (gesamt leere Menge) und $M_1\times M_2$ (vollständige Menge). Wichtig sind aber im allgemeinen die echten Teilmengen, die die verschiedensten Beziehungen zwischen den Elementen von $M_1$ und $M_2$ ausdrücken.
\end{enumerate}

\paragraph{Def. 7:}\label{Def. 7} (Eigenschaften binärer Relationen in $M_1\times M_2$)

Eine Relation $T\subseteq M_1\times M_2$ heißt:
\begin{enumerate}[label=\alph*)]
\item \emph{linksvollständig (linkstotal)}, wenn für jedes $x_1 \in M_1$ (wenigstens) ein $x_2 \in M_2$ existiert mit $(x_1, x_2) \in T$.
\item \emph{recthvollständig (rechtstotal}, wenn für jedes $x_2 \in M_2$ (wenigstens) ein $x_1 \in M_1$ existiert mit $(x_1,x_2) \in T$.
\item \emph{rechteindeutig}, wenn für jedes $x_1 \in M_1$ höchstens ein $x_2 \in M_2$ existiert mit $(x_1, x_2) \in T$.
\item \emph{linkseindeutig}, wenn für jedes $x_2 \in M_2$ höchstens ein $x_1 \in M_1$ existiert mit $(x_1, x_2) \in T$.
\end{enumerate}

\subparagraph{Bsp. 3:} \parskp
Es seien $S$ bzw. $L$ folgende Mengen von Städten bzw. Ländern:\\
$S=\{Berlin, \; Dresden, \; K\ddot{o}ln, \; Paris, \; Ram, \; Neapel, \; Oslo\}$\\
$L=\{D(eutschland), \; F(rankreich), \; B(elgien), \; I(talien), \; P(olen), \; N(orwegen)\}$\\
Die Relation $T\subseteq S \times L$ soll darstellen, welche Stadt in welchem Land liegt.\\
Man gebe $T$ elementweise an und stelle die Relation graphisch dar!\\
Welche der Eigenschaften aus Def. 7 treffen zu?
\begin{itemize}
\item $T=\{(Berlin, D), (Dresden, D), (K\ddot{o}ln, D), (Paris, F), (Rom, I), (Neapel, I), (Oslo, N)\}$
\item graphische Darstellung: \\
\begin{tikzpicture} [scale=0.25]

\draw  (0,2) rectangle (10,0) node[pos =.5]{Berlin};
\draw  (0,-1) rectangle (10,-3) node[pos =.5]{Dresden};
\draw  (0,-4) rectangle (10,-6) node[pos =.5]{Köln};
\draw  (0,-7) rectangle (10,-9) node[pos =.5]{Paris};
\draw  (0,-10) rectangle (10,-12) node[pos =.5]{Rom};
\draw  (0,-13) rectangle (10,-15) node[pos =.5]{Neapel};
\draw  (0,-16) rectangle (10,-18) node[pos =.5]{Oslo};


\draw  (18,2) rectangle (28,0) node[pos =.5]{D};
\draw  (18,-1) rectangle (28,-3) node[pos =.5]{F};
\draw  (18,-4) rectangle (28,-6) node[pos =.5]{B};
\draw  (18,-7) rectangle (28,-9) node[pos =.5]{P};
\draw  (18,-10) rectangle (28,-12) node[pos =.5]{I};
\draw  (18,-13) rectangle (28,-15) node[pos =.5]{N};

\node at (5,3) {S};
\node at (23,3) {L};
\draw[-latex] (10,1) -- (18,1);
\draw[-latex] (10,-2) -- (18,1);
\draw[-latex] (10,-5) -- (18,1);
\draw[-latex] (10,-8) -- (18,-2);
\draw[-latex] (10,-11) -- (18,-11);
\draw[-latex] (10,-14) -- (18,-11);
\draw[-latex] (10,-17) -- (18,-14);
\end{tikzpicture}\\
$(x,y) \in T: x\rightarrow y $ (gerichteter Graph)
\item Eigenschaften:\\
linksvollständig \\
nicht rechtsvollständig\\
rechtseindeutig\\
nicht linkseindeutig\\
(solche Relationen nennt man auch „Funktionen“, eindeutige Zuordnung [von Stadt $\rightarrow$ Land])
\end{itemize}

\paragraph{Def. 8:} (Eigenschaften binärer Relationen in $M\times M$)\\
Eine Relation $T\subseteq M\times M$ (Sprechweise auch „Relation auf M“) heißt…
\begin{enumerate} [label=\alph*)]
\item \emph{reflexiv}, wenn $(x,x) \in T$ für alle $x \in M$,
\item \emph{symmetrisch}, wenn $(x,y) \in T \Rightarrow (y,x) \in T$,
\item \emph{antisymmetrisch}, wenn $((x,y) \in T \wedge (y,x) \in T) \Rightarrow x=y$,
\item \emph{asymmetrisch}, wenn $(x,y) \in T \Rightarrow (y,x) \not\in T$, 
\item \emph{transitiv}, wenn $((x,y) \in T \wedge (y,z) \in T) \Rightarrow (x,z) \in T$
\end{enumerate}
… jeweils für \emph{alle} $x,y,z \in M$ gilt.

\subparagraph{Bsp. 4:} \parskp
Welche Eigenschaften aus Def. 8 besitzen folgende Relationen?\\
Es sei $P$ eine Menge von Personen.
\begin{enumerate} [label=\alph*)]
\item Eine Person $x\in P$ sei jünger als $y\in P$, wenn ihr Geburtstag später als der von $y$ ist.\\
$\curvearrowright J \subseteq P \times P$ mit $J=\{ (x,y) | x \text{ ist jünger als } y\}$.\\
$J$ ist offensichtlich asymmetrisch (damit auch antisymmetrisch [Die Prämisse der Implikation $((x,y) \in J \wedge (y,x) \in J) \Rightarrow x=y$ ist stets falsch, damit die Implikation stets wahr]) und transitiv. Eine solche Relation nennt man auch \emph{strikte Ordnungsrelation} (vgl. Abschnitt \ref{subsec:Ordnungsrelationen}).
\item Zwei Personon $x\in P$ und $y\in P$ heißen gleichaltrig, wenn $x$ und $y$ das gleiche Geburts\emph{jahr} besitzen.\\
$\curvearrowright G \subseteq P\times P$ mit $G= \{ (x,y) | x \text{ und } y \text{ sind gleichaltrig}\}$.\\
$G$ ist offensichtlich reflexiv, symmetrisch und transitiv.\\
Derartige Relationen nennt man \emph{Äquivalenzrelationen}, vgl. Abschnitt \ref{subsec:Äquivalenzrelationen}. Sie teilen $P$ in disjunkte sogenannte Äquivalenzklassen auf ($x$ äquivalent $y$ heißt, $x$ und $y$ besitzen gleiches Geburtsjahr).
\end{enumerate}
Graphische Darstellung von Relationen $T$ in $M\times M$ (auf $M$). Möglichkeiten:
\begin{enumerate}
\item Elemente von $M$ nur einmal darstellen, Pfeildarstellun wie bisher, bei $(x,x) \in T$ eine Schlinge zeichnen.\\
\begin{tikzpicture} [scale=0.35]
\draw (0,0) circle (1) node{x};
\draw [-latex] plot[smooth, tension=1.5] coordinates {(0.2,1) (0.5,2) (-0.5,2) (-0.2,1)};
\draw[-latex] (1,0) -- (3,0);
\draw (4,0) circle (1) node{y};
\draw[-latex] (5,0) -- (7,0);
\draw (8,0) circle (1) node{z};
\draw[-latex] (1,0) -- (3,0);
\draw [-latex] plot[smooth, tension=1] coordinates {(8,-1) (6,-2) (4,-1)};
\end{tikzpicture}\\
(gerichteter Graph)
\item 
\begin{tikzpicture}[scale=0.5]
% Achsen zeichnen
\draw[thick] (0,0) -- (2,0);
\draw[thick] (0,0) -- (0,2);
\draw[dashed] (1,0) -- (1,2);
\draw[dashed] (2,0) -- (2,2);
\draw[dashed] (0,1) -- (2,1);
\draw[dashed] (0,2) -- (2,2);

\fill[black] (0,0) circle (0.1) node[below left]{x};
\fill[black] (1,0) circle (0.1) node[below]{y};
\fill[black] (2,0) circle (0.1) node[below]{z};
\fill[black] (0,1) circle (0.1) node[left]{y};
\fill[black] (0,2) circle (0.1) node[left]{z};

\draw (0,1) circle (0.2);
\draw (0,0) circle (0.2);
\draw (1,2) circle (0.2);
\draw (2,1) circle (0.2);
\end{tikzpicture}\\
(Koordinatensystem)\\
Diese Variante ist auch bei Relationen in $M_1 \times M_2$ möglich.
\end{enumerate}

\subparagraph{Diskussion:}
\begin{enumerate}
\item Die Eigenschaften Reflexivität, Symmetrie und Transitivität lassen sich beim gerichteten Graphen leicht nachprüfen.\\
\emph{Reflexivität:} Bei jedem Element ist eine Schlinge.\\
\emph{Symmetrie:} Jeder Pfeil $x\rightarrow y \; (y \not = x)$ besitzt „umkehrpfeil“ ($x\leftarrow y$). \\
Antisymmetrie: Schlinge möglich, aber keine Umkehrpfeile.\\
Asymmetrie: weder Schlingen noch Umkehrpfeile.\\
\emph{Transitivität:} Falls ein Pfeil $x\rightarrow y $ eine „Fortsetzung“ $y\rightarrow z$ besitzt, so verläuft auch ein Pfeil von $x$ nach $z$.
\item Auch die Darsteellung von Koordinatensystem lassen sich die Eigenschaften Reflexivität und Symmetrie sofort überprüfen.\\
\emph{Reflexivität:} Die Diagonale $I_M = \{(x,x) | x \in M\}$ gehört zu $T$ ($I_M$ heißt auch \emph{Identitätsrelation}, diese Relation ist eine spezielle Funktion, identische Funktion $y=f(x))=x$, $x\in M$ später als Funktion auch mit $i_M$ bezeichnet)\\
\emph{Symmetrie:} $T$ ist spiegelsymmetrisch bzgl. $I_M$ \\
\begin{tikzpicture}[scale=0.5]
% Achsen zeichnen
\draw[thick] (0,0) -- (4,0);
\draw[thick] (0,0) -- (0,4);
\foreach \x in {0,1,2,3,4}
\draw[dashed] (\x,0) -- (\x,4);
\foreach \y in {0,1,2,3,4}
\draw[dashed] (0,\y) -- (4,\y);
%Achsen beschriften
\fill[black] (0,0) circle (0.1) node[below left]{$x_1$};
\foreach[count=\x] \i in {2,3,4,5}
\fill[black] (\x,0) circle (0.1) node[below]{$x_\i$};
\foreach[count=\y] \i in {2,3,4,5}
\fill[black] (0,\y) circle (0.1) node[left]{$x_\i$};

\draw (0,0) circle (0.2);
\draw (1,1) circle (0.2);
\draw (2,2) circle (0.2);
\draw (3,3) circle (0.2);
\draw (4,4) circle (0.2);
\draw (1,3) circle (0.2);
\draw (2,1) circle (0.2);
\draw (4,2) circle (0.2);
\draw (3,0) circle (0.2);

\fill[red] (0,0) circle (0.1);
\fill[red] (1,1) circle (0.1);
\fill[red] (2,2) circle (0.1);
\fill[red] (3,3) circle (0.1);
\fill[red] (4,4) circle (0.1) node[above right]{$I_M$};
\end{tikzpicture}
ist reflexiv aber nicht symmetrisch\\
\begin{tikzpicture}[scale=0.5]
% Achsen zeichnen
\draw[thick] (0,0) -- (4,0);
\draw[thick] (0,0) -- (0,4);
\foreach \x in {0,1,2,3,4}
\draw[dashed] (\x,0) -- (\x,4);
\foreach \y in {0,1,2,3,4}
\draw[dashed] (0,\y) -- (4,\y);
%Achsen beschriften
\fill[black] (0,0) circle (0.1) node[below left]{$x_1$};
\foreach[count=\x] \i in {2,3,4,5}
\fill[black] (\x,0) circle (0.1) node[below]{$x_\i$};
\foreach[count=\y] \i in {2,3,4,5}
\fill[black] (0,\y) circle (0.1) node[left]{$x_\i$};

\draw (0,0) circle (0.2);
\draw (1,1) circle (0.2);
\draw (3,3) circle (0.2);
\draw (1,3) circle (0.2);
\draw (3,1) circle (0.2);
\draw (0,4) circle (0.2);
\draw (4,0) circle (0.2);
\draw (1,4) circle (0.2);
\draw (4,1) circle (0.2);

\end{tikzpicture}
ist symmetrisch aber nicht reflexiv
\end{enumerate}
\emph{Alternative Schreibweisen:} Es sei $T \subseteq M_1 \times M_2$ eine binäre Relation. \\
Anstelle $\boxed{ (x,y) \in T }$ kann man schreiben:
\begin{itemize}
\item $x T y$ ($x$ steht in Relation $T$ zu $y$), für viele wichtige Relationen gibt es spezielle Zeichen, z.B. $x<y$, $x=y$, $g||h$ oder $A \subseteq B$ usw.
\item Aussageformen (vgl. Prädikatenlogik): $\boxed{T(x,y)}$ (auch mit mehreren Variablen möglich)
\end{itemize}

\subsubsection{Operationen auf Relationen}

Da Relationen spezielle Mengen sind, gibt es Operationen wie $\cup$, $cap$ usw. auch hier. Weitere für Relationen wichtige Operationen in den folgenden Definitionen:

\paragraph{Def. 9:} \parskp
Es sein $T$ eine Relation in $U\times V$.\\
Die Menge $proj_1(T)=\{x\in U | \exists y \in V, (x,y) \in T\}$ heißt \emph{Projektion} von $T$ auf $u$ (1. Faktor des kartesischen Produkts).\\
Analog ist $proj_2(T)=\{y\in V | \exists x \in U, (x,y) \in T\}$ die Projektion auf den 2. Faktor.

\emph{Veranschaulichung:}\\
\begin{tikzpicture} [scale = 0.25]
%Grenze
\draw  (0,0) rectangle (10,8);
\draw (10,0) node [below left] {$u$};
\draw (0,8) node [below left] {$v$};
%Gebilde
\fill[pattern=north west lines,pattern color=gray] (3,2) to [out=-90, in=180] (4,1) to [out=0, in=-160] (7,3) to [out=20,in=-90] (8,4) to [out=90,in=0] (6,6) to [out=180, in=80] (5,4) to [out=-100, in=90] (3,2);
\draw (3,2) to [out=-90, in=180] (4,1) to [out=0, in=-160] (7,3) to [out=20,in=-90] (8,4) to [out=90,in=0] (6,6) to [out=180, in=80] (5,4) to [out=-100, in=90] (3,2);
\draw (6.5,4) node{$T$};

%Projektion
\draw[dashed, red] (3,2) -- (3,0);
\draw[dashed, red] (8,4) -- (8,0);
\draw[red] (3,-0.1)--(5.5, -0.1) node[below]{$proj_1(T)$}--(8,-0.1);

\draw[dashed, orange] (4,1) -- (10,1);
\draw[dashed, orange] (6,6) -- (10,6);
\draw[orange] (10.1,1)--(10.1, 3) node[right]{$proj_2(T)$}--(10.1,6);
\end{tikzpicture}

\subparagraph{Bsp. 5:} \parskp
Es sei $S=\{1,2,3,4,5\}$ eine Menge von Studenten und $F=\{a,b,c,d,e,f\}$ eine Menge von Fächern.\\
Es sei $P\subseteq S \times F$ die Relation, die angibt, welcher Student in welchem Fach eine Nach- bzw. Wiederholungsprüfung im bevorstehenden Prüfungsabschnitt hat.\\
Die Studenten $1$ und $3$ haben keine Prüfung ausstehen, Student $2$ muss die Prüfungen in $a$. $d$ und $e$, $4$ in $b$ und $f$ sowie 5 in $b$, $d$, $e$ und $f$ ablegen.
\begin{enumerate}[label=\alph*)]
\item Man gebe die Relation $P$ elementweise an und stelle sie in einem Koordinatensystem dar.
\item Man ermittle die Projektionen $P$ auf $S$ bzw. $F$ und kennzeichne diese in der Skizze.
\end{enumerate}
Lösung:
\begin{enumerate}[label=\alph*)]
\item $P=\{(2,a), (2,d), (2,e), (4,b), (4,f),(5,b), (5,d), (5,e), (5,f)\}$\\
\begin{tikzpicture} [scale=0.6]
% Bereich
\draw (0,0) rectangle (4,5);
\draw (0,5) node [above]{Fach};
\draw (4,0) node[right]{Student};
\foreach [count=\i] \x in {0,1,2,3,4}
\draw (\x,-.1) -- (\x,.1) node[below=5pt] {$\i$};
\draw (-.1,0) -- (.1,0) node[left=5pt] {$a$};
\draw (-.1,1) -- (.1,1) node[left=5pt] {$b$};
\draw (-.1,2) -- (.1,2) node[left=5pt] {$c$};
\draw (-.1,3) -- (.1,3) node[left=5pt] {$d$};
\draw (-.1,4) -- (.1,4) node[left=5pt] {$e$};
\draw (-.1,5) -- (.1,5) node[left=5pt] {$f$};

%Punkte
\draw (1,0) circle (0.2);
\draw (1,3) circle (0.2);
\draw (1,4) circle (0.2);

\draw (3,1) circle (0.2);
\draw (3,5) circle (0.2);

\draw (4,1) circle (0.2);
\draw (4,3) circle (0.2);
\draw (4,4) circle (0.2);
\draw (4,5) circle (0.2);

%Projektion
\draw[orange, latex-] (4.2,2.5) -- (5.2,2.5);
\draw[orange] (-0.6,0) circle (0.3);
\draw[orange] (-0.6,1) circle (0.3);
\draw[orange] (-0.6,3) circle (0.3);
\draw[orange] (-0.6,4) circle (0.3);
\draw[orange] (-0.6,5) circle (0.3);
\draw[orange] (-0.6, 2.5) node[left]{$proj_2(P)$};

\draw[red, latex-] (2,5.2) -- (2,6.2);
\draw[red] (1,-0.6) circle (0.3);
\draw[red] (3,-0.6) circle (0.3);
\draw[red] (4,-0.6) circle (0.3);
\draw[red] (2, -0.7) node[below]{$proj_1(P)$};
\end{tikzpicture}
\item 
$proj_1(P)=\{2,4,5\}\subseteq S$\\
(= Menge der Studenten, die wenigsten eine N/W-Prüfung haben.)\\
$proj_2(P)=\{a,b,d,e,f\}\subseteq F$\\
(= Menge der Fächer, in denen Student(en) eine N/W-Prüfung haben.)
\end{enumerate}

\paragraph{Def. 10:} \parskp
Es sei $T \subseteq M_1 \times M_2$ eine binäre Relation. \\
Die Relation $T^{-1}:=\{(y,x)|(x,y)\in T\} \subseteq M_2\times M_1$ heißt \emph{inverse Relation} (bzw. kurz: \emph{Inverse}) von $T$.

\subparagraph{Bsp. 6:} (vgl. Bsp. 5)\\
$P^{-1}=\{(a,2), (b,4), (b,5), (d,2), (d,5), (e,2), (e,5), (f,4), (f,5)\}$\\
Besonders wichtig ist die folgende Operation:

\paragraph{Def. 11:} \parskp
Es seien $T_1 \subseteq M_1\times M_2$ und $T_2 \subseteq M_2 \times M_3$ binäre Relationen.\\
Als \emph{Komposition} (oder auch \emph{Verkettung}) $T_1 \circ T_2$ („$T_2$ nach $T_1$“) wird die Relation $T_1 \circ T_2 := \{ (x,z) \in M_1 \times M_3 | \exists y \in M_2 \quad (x,y) \in T_1 \wedge (y,z) \in T_2\}$ in $M_1 \times M_3$ bezeichnet.

\subparagraph{Bsp. 7:} \parskp
Es sei $M$ die Menge aller Menschen, die zu einem bestimmten Zeitpunkt leben. Weiter seien $S=\{(x,y)| x \text{ ist Mutter von }y\} \subseteq M\times M$ und $T=\{(y,z)| y \text{ ist verheiratet mit }z\}\subseteq M\times M$.

Dann bedeutet $(x,z)\in S\circ T$: Es gibt ein $y$, sodass $x$ die Mutter von $y$ ist ($(x,y)\in S$) und $y$ mit $z$ verheiratet ($(y,z)\in T$) ist, d.h. „$x$ ist die Schwiegermutter von $z$“.

\subparagraph{Diskussion:} Wichtige Eigenschaft der Komposition $\circ$:
\begin{itemize}
\item Die Operation $\circ$ ist \emph{assoziativ}, d.h. seien $T_1\subseteq A\times B$, $T_2\subseteq B\times C$ und $T_3 \subseteq C\times D$, dann gilt:\\
$(\underbrace{T_1 \circ T_2}_{\subseteq A\times C})\underset{\subseteq C \times D}{\circ T_3} = \underset{\subseteq A \times B}{T_1} \circ (\underbrace{T_2 \circ T_3}_{\subseteq B \times D}) = T_1 \circ T_2 \circ T_3 \subseteq A \times D$
\end{itemize}

\paragraph{Def. 12:} \parskp
Es sei $T$ eine Relation in $M \times M$ (auf $M$).\\
Als \emph{transitive Hülle} $T^+$ von $T$ bezeichnet man die kleinste Relation, die $T$ enthält und transitiv ist.
\subparagraph{Satz 1:} Es gilt: $\boxed{T^+ = T \cup (T\circ T) \cup (T\circ T \circ T) \cup ...}$
\subparagraph{Bemerkung:} \parskp
Bezeichnung für $\underbrace{T\circ T \circ ...\circ T}_{\text{n-mal}}$ auch $T^n$\\
(Nicht verwechseln mit Mengenprodukt $\underbrace{T\times ... \times T }_{\text{n-mal}}$ bzw. Funktionen mit n-ten Potenz $f^n$!)

Damit ist $\boxed{T^+= \bigcup^{\infty}_{j=1}T^j}$

\subparagraph{Beweis:} 
\begin{enumerate}
\item $T^+$ ist transitiv, denn sei $(x,y)\in T^+$ und $(y,z)\in T^+$, dann existieren natürliche Zahlen $j_1, j_2 \geq 1$ mit $(x,y)\in T^{j_1}$ und $(y,z)\in T^{j_2}$, \\
d.h. $y$ wird in $j_1$ Schritten von $x$ aus erreicht und $z$ in $j_2$ Schritten von $y$ aus erreicht. Also wird $z$ in $j_1 + j_2 $ Schritten von $x$ aus erreicht, \\
d.h. $(x,z) \in T^{j_1+j_2} \subseteq T^+$
\item Es sei $T\subseteq S$ für eine transitive Relation $S$.\\
$\Rightarrow T\circ T \subseteq S \circ S \subset S$ und für beliebiges $j\geq 1$:\\
$T^j \subseteq S^j \subseteq S$ und somit:\\
$T^+ = \bigcup^{\infty}_{j=1}T^j \subseteq S$, \\
d.h. $T^+$ ist tatsächlich die kleinste transitive Relation, die $T$ enthält.
\end{enumerate}
\subparagraph{Diskussion:}
\begin{enumerate}
\item Analog zur transitiven Hülle einer Relation $T$ in $M\times M$ (auf $M$) werden die reflexive Hülle bzw. die symmetrische Hülle von $T$ als die jeweils kleinsten Relationen die T enthalten und reflexiv bzw. symmetrisch sind definiert.\\
Die Ermittlung gestaltet sich etwas „einfacher“ als bei der transitiven Hülle:\\
\emph{Reflexive Hülle} von $T$: $\boxed{T\cup I_M}$ (dabei ist $I_M=\{(x,x)|x\in M\}$ [Diagonale / Identitätsrelation])\\
\emph{Symmetrische Hülle} von $T$: $\boxed{T \cup T^{-1}}$
\item Von Bedeutung ist auch die \emph{reflexiv-transitive} Hüllo ven $T$:\\
$\boxed{T^*=T^+\cup I_M}$ (dabei $T^+$ … transitive Hülle von $T$)
\end{enumerate}

\subparagraph{Bsp. 8:} \parskp
Gegeben sei die Menge $M=\{a,b,c,d,e,f\}$ \\
sowie die Relation $T=\{(a,b), (b,c), (c,e), (b,d), (d,e), (e,f)\}$.
\begin{enumerate}[label=\alph*)]
\item \emph{Transitive Hülle}: Zur Ermittlung der Komposition $S\circ T$: \\
Für jedes Element $(x,y)\in S$ alle Fortsetzungen $(y,z) \in T$ suchen $\curvearrowright (x,z)$ als Element von $S\circ T$ notieren, falls es noch nicht vorkommt.\\
Bspw.: 
\begin{itemize}
\item $(a,b)$, Fortsetzungen wären $(b,c), (b,d)$ $\curvearrowright$ Elemente $(a,c)$ und $(a,d)$ notieren.
\item $(b,c)$, Fortsetzung $(c,e)$ $\curvearrowright$ $(b,e)$ notieren
\item usw.
\end{itemize}
$\Rightarrow T\circ T=\{(a,c),(a,d),(b,e),(c,f),(d,f)\}=T^2$\\
$T^3=T \circ (T \circ T)=\{(a,e), (b,f)\}$ (ausgehend von $T$ in $T\circ T$ nach Fortsetzung suchen)\\
$T^4=T\circ T^3 = \{(a,f)\}$\\
\begin{tikzpicture}[scale=.5]
% Menge T
\draw (0,0) node{a};
\draw[-latex] (0.3,0.3) -- (1.7,1.7);
\draw (2,2) node{b};
\draw[-latex] (2.3,2) -- (3.7,2);
\draw (4,2) node{c};
\draw[-latex] (4.3,2) -- (5.7,2);
\draw (4,0) node{d};
\draw[-latex] (2,1.7) -- (3.7,0.3);
\draw (6,2) node{e};
\draw[-latex] (4.3,0.3) -- (5.7,1.7);
\draw (8,2) node{f};
\draw[-latex] (6.3,2) -- (7.7,2);


\draw [-latex, red] plot[smooth, tension=0.9] coordinates {(0,0.3) (1.3,2.7) (3.7,2.3)};
\draw [-latex, red] plot[smooth, tension=0.9] coordinates {(0.3,0) (2,-0.5) (3.7,0)};
\draw [-latex, red] plot[smooth, tension=0.9] coordinates {(2.3,2.3) (4,2.8) (5.7,2.3)};
\draw [-latex, red] plot[smooth, tension=0.9] coordinates {(4.3,2.3) (6,2.8) (7.7,2.3)};
\draw [-latex, red] plot[smooth, tension=0.9] coordinates {(4.3,0) (6,0.5) (8,1.7)};

\draw [-latex, green] plot[smooth, tension=0.9] coordinates {(0,0.3) (2.3,3) (5.8,2.5)};
\draw [-latex, green] plot[smooth, tension=0.9] coordinates {(2.3,1.7) (5,1.3) (7.7,1.7)};

\draw [-latex, orange] plot[smooth, tension=0.9] coordinates {(0.3,-0.3) (5,-0.7) (8.1,1.7)};
\end{tikzpicture}\\
$\Rightarrow T^+=T \cup \underbrace{(T\circ T)}_{\text{\textcolor{red}{2 Schritte}}} \cup \underbrace{(T \circ T \circ T)}_{\text{\textcolor{green}{3 Schritte}}} \cup \underbrace{(T\circ T \circ T \circ T}_{\text{\textcolor{orange}{4 Schritte}}}=T \cup T^2 \cup T^3 \cup T^4$ \medskip\\
(Formel bricht im endlichen Fall nach endlich vielen Schritten ab.)
\item \emph{Reflexive Hülle}: $T\cup \{(a,a), (b,b), (c,c), (d,d), (e,e), (f,f)\}$\\
\begin{tikzpicture}[scale=.5]
% Menge T
\draw (0,0) node{a};
\draw[-latex] (0.3,0.3) -- (1.7,1.7);
\draw (2,2) node{b};
\draw[-latex] (2.3,2) -- (3.7,2);
\draw (4,2) node{c};
\draw[-latex] (4.3,2) -- (5.7,2);
\draw (4,0) node{d};
\draw[-latex] (2,1.7) -- (3.7,0.3);
\draw (6,2) node{e};
\draw[-latex] (4.3,0.3) -- (5.7,1.7);
\draw (8,2) node{f};
\draw[-latex] (6.3,2) -- (7.7,2);

\draw [-latex, orange] plot[smooth, tension=1.5] coordinates {(0.2,0.2) (0.5,1) (-0.5,1) (-0.2,0.2)};
\draw [-latex, orange] plot[smooth, tension=1.5] coordinates {(2.2,2.2) (2.5,3) (1.5,3) (1.8,2.2)}; 
\draw [-latex, orange] plot[smooth, tension=1.5] coordinates {(4.2,2.2) (4.5,3) (3.5,3) (3.8,2.2)};\draw [-latex, orange] plot[smooth, tension=1.5] coordinates {(4.2,0.2) (4.5,1) (3.5,1) (3.8,0.2)};
\draw [-latex, orange] plot[smooth, tension=1.5] coordinates {(6.2,2.2) (6.5,3) (5.5,3) (5.8,2.2)};
\draw [-latex, orange] plot[smooth, tension=1.5] coordinates {(8.2,2.2) (8.5,3) (7.5,3) (7.8,2.2)};
\end{tikzpicture}
\item \emph{Symmetrische Hülle}: $T\cup T^{-1}=T\cup \{ (b,a), (c,b), (e,c), (d,b), (e,d), (f,e)\}$\\
\begin{tikzpicture}[scale=.5]
% Menge T
\draw (0,0) node{a};
\draw[-latex] (0.3,0.3) -- (1.7,1.7);
\draw (2,2) node{b};
\draw[-latex] (2.3,2) -- (3.7,2);
\draw (4,2) node{c};
\draw[-latex] (4.3,2) -- (5.7,2);
\draw (4,0) node{d};
\draw[-latex] (2,1.7) -- (3.7,0.3);
\draw (6,2) node{e};
\draw[-latex] (4.3,0.3) -- (5.7,1.7);
\draw (8,2) node{f};
\draw[-latex] (6.3,2) -- (7.7,2);

\draw[latex-, orange] (0.2,0.5) -- (1.6,2);
\draw[latex-, orange] (2.2,2.2) -- (3.6,2.2);
\draw[latex-, orange] (4.2,2.2) -- (5.6,2.2);
\draw[latex-, orange] (1.9,1.5) -- (3.3,0.3);
\draw[latex-, orange] (4.6,0.3) -- (5.9,1.5);
\draw[latex-, orange] (6.2,2.2) -- (7.6,2.2);
\end{tikzpicture}
\end{enumerate}
Zur Überprüfung der Eigenschaften aus Def. 8 ist folgender Satz nützlich:
\subparagraph{Satz 2:} \parskp
Es sei $T \subseteq M \times M$ eine binäre Relation. Dann gilt:
\begin{enumerate} [label=\alph*)]
\item $T$ ist reflexiv $\Leftrightarrow I_M \subseteq T$  ($I_M$ … Identitätsrelation)
\item $T$ ist symmetrisch $\Leftrightarrow T^{-1}\subseteq T \quad [\Leftrightarrow T^{-1} = T] $
\item $T$ ist antisymmetrisch $\Leftrightarrow T \cap T^{-1} \subseteq I_M$
\item $T$ ist asymmetrisch $\Leftrightarrow T \cap T^{-1} = \emptyset$
\item $T$ ist transitiv $\Leftrightarrow T \circ T \subseteq T$
\end{enumerate}

\subparagraph{Disskussion:}
\begin{enumerate}
\item Beweise ergeben sich unmittelbar aus Def. 8, vgl. Übungsaufgabe 1.24 (für b) und e))
\item Aus c) und d) ergibt sich z.B.\\
$\boxed{T \text{ asymmetrisch}}\Rightarrow \boxed{T \text{ antisymmetrisch}}$ (da $\emptyset$ Teilmenge jeder Menge ist)
\end{enumerate}

\subsubsection{Äquivalenzrelationen} \label{subsec:Äquivalenzrelationen}
\paragraph{Def. 13:} \parskp
Eine Relation $T \subseteq M \times M$ heißt \emph{Äquivalenzrelation} auf $M$, wenn sie reflexiv, symmetrisch und transitiv ist.
\subparagraph{Diskussion:}
\begin{enumerate}
\item Durch eine Äuivalenzrelation wird $M$ vollständig in paarweise elementfremde (disjunkte) \emph{Äquivalenklassen} zerlegt. Die Menge aller Äquivalenzklassen von $M$ bezüglich $T$ heißt \emph{Quotientenmenge} $M/T$.\\
Aufgrund der 3. Eigenschaft aus Def. 13 erhält eine Äquivalenzklasse alle Elemente, die untereinander erreichbar sind (=äquivalent) und nur diese.
\item Äquivalenzklassen enthalten alle Elemente, die bezüglich einer bestimmten Eigenschaft nicht unterscheidbar sind, z.B. Bsp. 4 mit $M=P$ (Menge von Personen), Äquivalenzrelation $G\subseteq P \times P$ mit $G=\{ (x,y) | x \text{ und } y \text{ haben gleiches Geburtsjahr}\}$, Äquivalenzklassen sind die Jahrgänge.
\item Anstelle der Schreibweise $(x,y) \in T$, $xTy$ oder $T(x,y)$ verwendet man bei beliebigen Äquivalenzrelationen auf $x\sim y$. Bei vielen speziellen Äquivalenzrelationen spezielle Symbole, sie folgendes Beispiel.
\end{enumerate}

\subparagraph{Bsp. 9:}
\begin{enumerate}[label=\alph*)]
\item $M$ sei eine beliebige Menge $T_1=I_M=\{ (x,y) \in M\times M | x=y\}$ (Identitätsrelation) ist eine Äquivalenzrelation.\\
Äquivalent heißt hier gleich! \\
Äquivalenzklassen sind sämtliche einelementige Teilmengen $\{x\}, x\in M$. $T_1$ heißt die feinste Zerlegung von $M$ die möglich ist. Die größte Zerlegung liefert die Relation $T_2=M\times M$, die trivialer Weise ebenfalls eine Äquivalenzrelation ist mit nur einer Äquivalenzklasse $M$. Für die Anwendungen sind natürlich Relationen wichtig, die eine feinere Zerlegung liefern.
\item $M = \mathbb{Z}$ (ganze Zahlen), $m\in \mathbb{N}^*$, $T\subseteq \mathbb{Z}\times \mathbb{Z}$ mit
\begin{itemize}
\item $(x,y) \in T : \equiv$ „$x$ und $y$ lassen bei Division durch $m$ den gleichen Rest“
\item Bezeichunung $x \equiv y (mod\; m)$ … $x$ kongruent $y$ (modulo $m$), z.B. $29 \equiv 8 (mod\; 7)$
\item $T$ ist eine Äquivalenzrelation auf $\mathbb{Z}$, Äquivalenzklassen:\\
Restklassen modulo $m$ (siehe Übungsaufgabe 1.19)
\end{itemize}
\item $M$ … Menge aller Geraden einer Ebene, $T\subseteq M \times M$ mit
\begin{itemize}
\item $(x,y) \in T :\equiv$ „$x$ ist zu $y$ parallel“, Bezeichnung: $x||y$\\
$\curvearrowright T$ ist Äquivalenzrelation auf $M$ (siehe Übungsaufgabe 1.21.)
\end{itemize}
\end{enumerate}

\subsubsection{Ordnungsrelationen} \label{subsec:Ordnungsrelationen}
\paragraph{Def. 14:}
\begin{enumerate}[label=\alph*)]
\item Eine Relation $T\subseteq M\times M$ heißt Ordnungsrelation auf $M$, wenn sie reflexiv, antisymmetrisch und transitiv ist.
\item Eine Ordnungsrelation heißt \emph{vollstandig} oder \emph{linear}, wenn für alle $x,y \in M$ gilt $(x,y)\in T \vee (y,x) \in T$.
\end{enumerate}

\paragraph{Def. 15:} \parskp
Eine Relation $T\subseteq M \times M$ heißt \emph{strikte Ordnungsrelation} auf $M$, wenn sie asymmetrisch und transitiv ist. Eine strikte Ordnungsrelation heißt vollständig, wenn für alle $x,y \in M$ mit $x\not = y$ gilt $(x,y) \in T \vee (y,x) \in T$.

\subparagraph{Bsp. 10:}
\begin{enumerate} [label=\alph*)]
\item $M = \mathbb{R}$, $T\subseteq \mathbb{R} \times \mathbb{R}$ mit $\boxed{(x,y) \in T :\equiv x \leq y}$ ist eine vollständige Ordnungsrelation auf $\mathbb{R}$.
\item Die Relation „$<$“ ist eine (vollständige) strikte Ordnungsrelation.
\item $E$ sei eiine Menge, $M$ sei die \emph{Menge aller Teilmengen von $E$}, d.h. $M$ ist die Potenzmenge $M=\mathcal{P}(E)$ von $E$.\\
$T \subseteq M \times M$ mit $\boxed{(A,B)\in T :\equiv A \subseteq B}$ ist eine Ordnungsrelation auf $\mathcal{P}(E)=M$ (Inklusion).
\end{enumerate}

\subparagraph{Diskussion:}
\begin{enumerate}
\item In der Literatur wird manchmal die Relation im Sinne von Def. 14 als Halbordnung und nur eine vollständige als Ordnung als Ordnungsrelation bezeichnet.
\item Zu jeder Ordnung $T_1$ (auf $M$) gehört eine strikte Ordnung $T_2$ und umgekehrt: $T_2=T_1\backslash I_M$ bzw. $T_1=T_2 \cup I_M$ ($T_1$ eist die reflexive Hülle von $T_2$), z.B. ($\leq$, $<$) oder ($\subseteq$, $\subset$).
\item Die Symbole $\leq$ (bzw. $<$) können anstelle der Paarschreibweise auch bei beliebigen Ordnungen verwendet werden, falls keine anderen Zeichen dafür üblich sind.
\end{enumerate}

\paragraph{Def. 16:} \parskp
$T$ sei eine Ordnungsrelation auf eine Menge $M$. Weiter sei $A$ eine Teilmenge von $M$.
\begin{enumerate} [label=\alph*)]
\item Ein Element $a \in M$ heißt obere Schranke von $A$, wenn gilt:\\
$\forall x \in A \quad x \leq a \quad (x \leq a$ d.h. $(x,a) \in T$, vgl. 3.) der vorhergehenden Diskussion)
\item Es sei $B$ die Menge der oberen Schranken von $A$, diese sei nicht leer. Falls es eine \emph{kleinste obere Schranke $s$} von $A$ gibt, d.h. $\exists s \in B \quad \forall b \in B \quad s \leq b$, so heißt $s$ das \emph{Supremum von $A$}, $\boxed{s = sup \, A}$
\item Gilt $s \in A$, so heißt $s$ das Maximum von $A$: $s = max \, A = sup \, A$
\item Ein Element $m \in A$ heißt maximal, wenn es kein größeres Element in $A$ gibt, d.h. $\forall x \in A \; (m \leq x \Rightarrow m=x)$
\end{enumerate}

\subparagraph{Diskussion:} 
\begin{enumerate}
\item Die Begriffe aus Def. 16 lassen sich auf strikte Ordnungen $S$ übertragen, indem anstelle von $S$ die reflexive Hülle $T=S \cup I_M$ verwendet wird.
\item Bei Ordnungsrelationen $T$ (auch für strikte Ordnungen) auf endlichen Mengen $M$ kann ein vereinfachter Graph, das sogenannte \emph{HASSE-Diagramm}, betrachtet werden.\\
\begin{tikzpicture}[scale = 0.5]
\draw[-latex] (0,0) node[left] {a} -- (2,0) node[right]{b};
\end{tikzpicture}
$(a\not = b)$ bedeutet $(a,b) \in T$ und es gibt kein Zwischenglied $c \not = a$ und $c\not = b$ mit $(a,c) \in T \wedge (c,b) \in T$ ($a$ ist unmittelbarer Vorgänger von $b$ bzw. $b$ Nachfolger von $a$).\\
Diesem Diagramm entspricht eine Teilrelation $U \subseteq T$, deren transitiv-reflexive Hülle (bzw. transitive Hülle bei strikten Ordnungen) $T$ ist.
\item Veranschaulichung von Def. 16 mit einem HASSE-Diagramm einer nicht vollständingen Ordnung (nicht linear)\\
\begin{tikzpicture}[scale=.5]
% Menge T
\draw (0,0) node{a};
\draw[-latex] (0.3,0) -- (1.7,0);
\draw (2,0) node{b};
\draw[-latex] (2.3,0.1) -- (3.7,1);
\draw[-latex] (2.3,-0.1) -- (3.7,-1);
\draw (4,1) node{c};
\draw (4,-1) node{d};
\draw[-latex] (4.3,-1) -- (5.9,0.7);
\draw[-latex] (4.3,1) -- (5.7,1);
\draw (6,1) node{e};
\draw[-latex] (6.3,1) -- (7.7,1);
\draw[-latex] (6.3,0.7) -- (7.7,-1);
\draw (8,1) node{f};
\draw (8,-1) node{g};

\draw (-1,-2) rectangle (9,2) node[right] {$M$};
\draw[gray] (-0.5,-1.5)node[above right]{$A$} rectangle (4.5,1.5);

\end{tikzpicture}\\
z.B. Arbeitsgänge, die in einer bestimmten Reihenfolge durchgeführt werden müssen, $A$ bspw. Teilarbeiten einer Zweigfirma\\
\emph{obere Schranken:} $e,f,g$\\
\emph{$sup\, A$}$=e$\\
\emph{Maximum von $A$:} existiert nicht, da $e \not \in A$\\
\emph{maximale Elemente von $A$:} $c,d$
\item Bei nichtlinearen Ordnungen müssene obere Schranken, Supremum und Maximum nicht existieren, es kann mehrere maximale Elemente $A\subseteq M$ geben.\\
Bei linearen Ordungen auf \emph{endlichen} Mengen gibt es genau ein maximales Element $=max\, A =max\, B$
\item Analog zur Def. 16 werden die Begriffe \emph{untere Schranken} $a$ von $A$ ($\forall x \in A \quad a \leq x$), \emph{größte untere Schranke} (\emph{Infinum}) $s$ von $A$ ($B\not = \emptyset$ … Menge der unteren Schranken, $\exists s \in B \quad \forall a \in B \forall a \leq s$), \emph{Minimum von $A$} ($min \, A = inf\, A = s$ falls $s \in A$) und minimales Element $m$ von $A$ ($\forall x \in A \quad (x\leq m \Rightarrow x = m)$) definiert.
\subparagraph{Bsp. 11:} \parskp
Eine bestimmte Arbeitsaufgabe besteht aus mehreren Arbeitsgängen. \\
Es sei $A=\{1,2,3$,$4,5,6\}$ die Menge der Arbeitsgänge. Die Arbeitsgänge $\{2,3,5\}=:S$ werden von einer Subfirma durchgeführt. Für die Reihenfolge gilt: 1 muss vor 2, 2 vor 3 und 5, 3 vor 4 sowie 5 vor 6 durchgeführt werden.
\begin{enumerate}  [label=\alph*)]
\item Man beschreibe diese Forderungen durch eine Relation $U\subseteq A\times A$ und stelle sie graphisch dar (HASSE-Diagramm).
\item Man ermittle die transitive Hülle $U^+$ von $U$.
\item Man gebe (falls vorhanden) obere Schranken, Supremum, Maximum, max. Elemente sowie untere Schranken, Infinum, Minimum, min. Elemente von $S$ an.
\end{enumerate}
Lösung:
\begin{enumerate}   [label=\alph*)]
\item $U=\{(1,2),(2,3),(2,5),(3,4),(5,6\}$\\
\begin{tikzpicture}[scale=.5]
% Menge T
\draw (0,0) node{1};
\draw[-latex] (0.3,0) -- (1.7,0);
\draw (2,0) node{2};
\draw[-latex] (2.3,0.1) -- (3.7,1);
\draw[-latex] (2.3,-0.1) -- (3.7,-1);
\draw (4,1) node{3};
\draw[-latex] (4.3,1) -- (5.7,1);
\draw (6,1) node{4};
\draw (4,-1) node{5};
\draw[-latex] (4.3,-1) -- (5.7,-1);
\draw (6,-1) node{6};

\draw[gray] (1.5,-1.5)  node[left]{$S$} rectangle (4.5,1.5);
\end{tikzpicture}
\item $U\circ U=\{(1,3), (1,5), (2,4), (2,6)\}$\\
$U\circ (U \circ U)=\{ (1,4), (1,6)\}$\\
$U^4=\emptyset$
\end{enumerate}
\end{enumerate}

\subsubsection{Funktionen}

\paragraph{Def. 17:} \parskp
Eine Relation $f\subseteq x\times y$ heißt \emph{Funktion (Abbildung)} von $X$ in $Y$, wenn sie linksvollständig und rechtseindeutig ist.

\subparagraph{Diskussion:}
\begin{itemize}
\item Gemäß Def. 7 a+c aus Kapitel \ref{Def. 7} bedeutet linksvollständig \emph{und} rechtseindeutig, dass zu \emph{jedem} $x\in X$ \emph{genau ein} $y \in Y$ mit $(x,y) \in f$ existiert, also \emph{eindeutige Zuordnung}:\\
$\boxed{x\longmapsto y=: f(x)}$\\
\emph{Schreibweise:}
$f: X \rightarrow Y$ (manchmal $f|X\rightarrow Y$\\
$y=f(x)$ heißt auch \emph{Bild} von $x$, $x$ \emph{ein} Urbild von $y$ (muss nicht eindeutig sein).
\item $X=Db(f)$… \emph{Definitionsbereich}, \\
$Wb(f)=\{y \in Y | \exists x \in x \quad (x,y) \in f\} \subseteq Y$ … Wertebereich\\
Schreibweise auch $f(X):= Wb (f)$ (Menge aller Bilder).\\
\begin{tikzpicture} [scale=0.5]
\draw (0,0) rectangle (8,6);
\draw plot[smooth, tension=0.9] coordinates {(0,1) (2,3.5) (8,3)};

\draw[purple, dashed] (0,3.7) -- (3,3.7);
\draw[purple, very thick] (0,0) -- (0,3.7);
\draw[purple] (0,2) node[left]{$Wb(f)$};
\draw[orange, very thick] (0,0) -- (8,0);
\draw[orange] (4,0) node[below]{$x=Db(f)$};
\end{tikzpicture}\\
$f: \mathbb{R} \rightarrow \{0,1\}$
\end{itemize}

\paragraph{Def. 18:}
\begin{enumerate} [label=\alph*)]
\item Eine Abbildung $f$ heißt \emph{surjektiv} (Auch Abbildung auf $Y$), 
\item Eine Funktion $f$ heißt \emph{injektiv}, wenn zu jedem $y \in Wb(f)$ genau ein $x \in Db(f)$ existiert mit $(x,y) \in f$:\\
$\underset{\in Wb(f)}{y} \longmapsto x =:\underset{\in Db(f)} {f^{-1}(y)}$ \tab („$f$ oben -1“)\\
Die dadurch erklärte Abbildung $f^{-1}: Wb(f) \rightarrow Db (f)$ heißt \emph{Umkehrfunktion} von $f$, vgl. auch Kap \ref{subsec:Aussagefunktionen}.
\item Eine injektive \emph{und} surjektive Abb. von $X$ auf $Y$ heißt \emph{bijektiv}.
\item Gebräuchlich sind auch die Begriffe \emph{Surjektion}, \emph{Injektion} und \emph{Bijektion}!
\end{enumerate}

\subparagraph{Bsp. 12:} \parskp
Gegeben sind die Mengen $X=\{a,b,c\}$ und $Y=\{1,2,3,4\}$ sowie folgende Relation in $X\times Y$:
\begin{enumerate} [label=\alph*)]
\item $T_1$: \begin{tikzpicture}[scale=0.3]
\draw (0,0) node{$a$};
\draw (0,-2) node{$b$};
\draw (0,-4) node{$c$};
\draw (0,-8) node{$(X)$};

\draw (6,0) node{$1$};
\draw (6,-2) node{$2$};
\draw (6,-4) node{$3$};
\draw (6,-6) node{$4$};
\draw (6,-8) node{$(Y)$};

\draw[-latex] (0.5,0) -- (5.5,0);
\draw[-latex] (0.5,-2) -- (5.5,-6);
\draw[-latex] (0.5,-4) -- (5.5,-2);
\end{tikzpicture} \\
$T_1$ ist eine Funktion $f(=T_1): f: X\rightarrow Y \quad (1)$ diese ist injektiv, $Db(f) = X= \{a,b,c\}$, $Wb(f) = \{1,2,4\}=: W$, $f: X\rightarrow W \quad (2)$ ist surjektiv, also sogar bijektiv.\\
Als Relation sind $(1)$ und $(2)$ nicht zu unterscheiden, aber als Funktion.
\item $T_2$: \begin{tikzpicture}[scale=0.3]
\draw (0,0) node{$a$};
\draw (0,-2) node{$b$};
\draw (0,-4) node{$c$};
\draw (0,-8) node{$(X)$};

\draw (6,0) node{$1$};
\draw (6,-2) node{$2$};
\draw (6,-4) node{$3$};
\draw (6,-6) node{$4$};
\draw (6,-8) node{$(Y)$};

\draw[-latex] (0.5,0) -- (5.5,0);
\draw[-latex] (0.5,-2) -- (5.5,-4);
\end{tikzpicture} \\$T_2$ ist keine Funktion, nicht linksvollständig. Betrachtet man $D=\{a,b\}\subset X$, so ist durch $T_2$ eine Funktion $f:D\rightarrow Y$ beschrieben, die Funktion ist injektiv und kann mit $W:= f(D)=\{1,3\}$ zu einer bijektiven Abbildung $f: D\rightarrow W$ umgewandelt werden.
\item $T_3$: \begin{tikzpicture}[scale=0.3]
\draw (0,0) node{$a$};
\draw (0,-2) node{$b$};
\draw (0,-4) node{$c$};
\draw (0,-8) node{$(X)$};

\draw (6,0) node{$1$};
\draw (6,-2) node{$2$};
\draw (6,-4) node{$3$};
\draw (6,-6) node{$4$};
\draw (6,-8) node{$(Y)$};

\draw[-latex] (0.5,0.3) -- (5.5,0.3);
\draw[-latex] (0.5,0) -- (5.5,-2);
\draw[-latex] (0.5,-2) -- (5.5,-4);
\draw[-latex] (0.5,-4) -- (5.5,-6);
\end{tikzpicture} \\$T_3$ ist keine Funktion, da nicht rechtseindeutig.
\end{enumerate}

\subparagraph{Bsp. 13:}
\begin{enumerate} [label=\alph*)]
\item $f:[0,\infty) \rightarrow \mathbb{R}$ mit „$x \rightarrow y = f(x) = \sqrt{x}$ ist eine \emph{Funktion einer reelen Veränderlichen} (injektiv).\\
\begin{tikzpicture} [scale =0.5]
\draw[-latex] (-.5, 0) -- (8,0) node[right]{$x$};
\draw[-latex] (0, -.5) -- (0, 4) node[above]{$y$};
\draw[red, domain=0:7] plot[smooth] (\x, {sqrt(\x)});
\end{tikzpicture}\\
$Wb(f)=[0,\infty)$\\
$Db(f)=[0,\infty)$
\item $f: \mathbb{R}\times \mathbb{R}\rightarrow \mathbb{R}$ mit $(x,y) \longmapsto x^2 +y^2=f(x,y) =: z$ \emph{Funktion zweier reeller Veränderlicher}.\\
\begin{tikzpicture}[scale =0.5]
\draw[->,thick] (-0.5,0) -- (4,0) node[right] {$y$};
\draw[->,thick] (0,-0.5) -- (0,4) node[above] {$z$};
\draw[->,thick] (.5,.5) -- (-2,-2) node[left] {$x$};

\draw[red, domain=-2:2] plot[smooth] (\x,{\x*\x});
\draw[red, dashed] (0,1) ellipse (1 and 0.3);
\draw[red, dashed] (0,2) ellipse (1.4 and 0.42);
\draw[red, dashed] (0,3) ellipse (1.75 and 0.525);
\end{tikzpicture}
(Paraboloid)\\
$Db(f) = \mathbb{R} \times \mathbb{R}$ (x-y-Ebene)\\
$Wb(f)=[,\infty)$
\item $f: \mathbb{N} \rightarrow \mathbb{R}$ mit $n\longmapsto f(n) = \frac{n}{n+1}$ ist eine (reelle) Zahlenfolge. $f(0)=1, f(1)=\frac{1}{2}, f(2)=\frac{2}{3}, ...$\\
Bezeichnung meist mit Index: $a_n=f(n) \curvearrowright ZF(a_n) \quad n \in \mathbb{N}$
\end{enumerate}

\paragraph{Def. 19:} \parskp
Es seien $g:=X\rightarrow U$ mit $x\longmapsto u=g(x)$ und $f: U\rightarrow Y$ mit $u\longmapsto y = f(u)$ zwei Abbildungen. Dann stellt man die Zuordnung $x\longmapsto y = f(\underset{u}{g(x)})$ eine Abbildung von $X$ in $Y$ dar, eine sogenannte \emph{mittelbare Funktion (Komposition / Verkettung)}.
\emph{Bezeichnung:} $g \circ f: X \rightarrow Y$ mit $y=(g \circ f) (x) = f(g(x))$

\subparagraph{Diskussion:}
\begin{enumerate}
\item \begin{tikzpicture}[scale=0.3]
\draw (-1,1.5) node {$X$};
\draw (0,-2) ellipse (2 and 3);
\draw (0,0) node{\textbullet};
\draw (0,-2) node{\textbullet};
\draw (0,-4) node{\textbullet};

\draw (5,1.5) node {$U$};
\draw (6,-3) ellipse (2 and 4);
\draw (6,0) node{\textbullet};
\draw (6,-2) node{\textbullet};
\draw (6,-4) node{\textbullet};
\draw (6,-6) node{\textbullet};

\draw (11,1) node {$Y$};
\draw (12,-2) ellipse (2 and 3);
\draw (12,0) node{\textbullet};
\draw (12,-2) node{\textbullet};
\draw (12,-4) node{\textbullet};

\draw[-latex] (0.5,0) -- (5.5,0);
\draw[-latex] (0.5,-2) -- (5.5,-2);
\draw[-latex] (0.5,-4) -- (5.5,-6);

\draw[-latex, orange] (6.5,0) -- (11.5,0);
\draw[-latex, orange] (6.5,-1) -- (11.5,-4);
\draw[-latex, orange] (6.5,-4) -- (11.5,-.2);
\draw[-latex, orange] (6.5,-6) -- (11.5,-2);
\end{tikzpicture} \\
$x \longmapsto u=g(x) \quad u \textcolor{orange}{\longmapsto} f(u) = f(g(x))$\\
Paarschreibweise: $(x,u) \in g \qquad (u,y) \in f \curvearrowright (x,y) \in g \circ f$
\item $g$ wird zuerst angewendet, dann $f$. Wie bei beliebigen Relationen die die Schreibweise $g\circ f$
\item In der Literatur findet man oft die Schreibweise $f\circ g$ angelehnt an die Schreibweise $f(g(x))$. Die Reihenfolge der Berechnung ast aber von innen nach außen, erst innere Funktion $g$, dann die äußere $f$.
\end{enumerate}

\paragraph{Satz 3:} \parskp
Es sei $f: X\rightarrow Y$ eine \emph{Bijektion}, d.h. es existiert die Umkehrfunktion $f^{-1}: Y\rightarrow X$, weiter sei $i_A$ für eine beliebige Menge $A$ die identische Abbildung (Identitätsrelation): $i_A: A\rightarrow A$ mit $i_A(x)=x$ für alle $x\in A$.\\
Es gilt dann: \\
$f \circ f^{-1}= id_X$, d.h. $(f\circ f^{-1})(x)=f^{-1}(f(x))= x (\forall x \in X)$ und \\
$f^{-1}\circ f = id_Y$, d.h. $ (f^{-1}\circ f)(y))= f(f^{-1}(y))=y (\forall y \in Y)$\\
(Funktion und Umkehrfunktion nacheinander angewandt heben sich auf).

\paragraph{Satz 4:} \parskp
Es seien $g=X\rightarrow U$ und $h: U\rightarrow Y$ zwei Bijektionen. Dann ist die Komposition $f:= g \circ h : X\rightarrow Y$ ebenfalls eine Bijektion und es gilt:\\
$\boxed{f^{-1}=(g\circ h)^{-1}=h^{-1}\circ g^{-1}}$

\subsection{Gleichmächtigkeit, Kardinalzahlen}

Es sei eine hinreichend mufassend Grundmenge, die alle für eine mathematische Theorie relevante Objekte (Zahlen, Funktionen, usw.) enthält. $M$ sei die Potenzmenge von $E$(d.h. $M$ ist die Menge aller Teilmengen von $E$, $M=\mathcal{P}(E)$).

\paragraph{Def. 20:} \parskp
Zwei Mengen $A$ und $B$ ($A \subseteq E, B\subseteq E$ bzw. $A \in M, B \in M$) heißen \emph{gleichmächtig} (Bezeichnung $A\sim B$), wenn eine bijektive Abbildung von $A$ auf $B$ (damit auch $B$ auf $A$) existiert.

\subparagraph{Diskussion:}
\begin{enumerate}
\item Offensichtlich ist die Relation $T\subseteq M \times M$ mit $(A,B) \in T : \equiv A \sim B$ eine Äquivalenzrelation auf $M$.
\item Äquivalenzklassen sind Mengen gleichmächtiger Teilmengen von $E$. Diese Äquivalenzklassen nennt man \emph{Kardinalzahlen}.
\item Bei endlichen Mengen bedeutet Gleichmächtigkeit: \\
Gleiche Anzahl von Elementen \\
$A = \{a, b,c\}, B=\{X,Y,Z\}$ \\
(Abbildung bspw. $a \rightarrow X \quad b\rightarrow Y \quad c\rightarrow Z$)\\
Bezeichnung: $card A = |A| = 3 \quad (=|B|)$\\
\emph{Natürliche Zahlen sind die Kardinalzahlen endlicher Mengen.}
\item Die Anschauung versagt bei unendlichen Mengen.\\
\begin{tikzpicture} [scale = 0.4]
\draw (0,0) -- (10,0) node[below left]{B};
\draw (0,0) -- (10,5) node[above left]{A};

\draw[green, latex-] (2,0) -- (2,1);
\draw[green, latex-] (4,0) -- (4,2);
\draw[green, latex-] (6,0) -- (6,3);
\draw[green, latex-] (8,0) -- (8,4);
\end{tikzpicture}\\
Die Strecken $A$ und $B$ sind gleichmächtig, obwohl $A$ länger als $B$ ist.
\end{enumerate}

\paragraph{Def. 21:} \parskp
Eine Menge heißt \emph{abzählbar unendlich}, wenn sie mit der Menge $\mathbb{N}=\{1,2,3,4,...\}$ der natürlichen Zahlen gleichmächtig ist.

\subparagraph{Diskussion:}

\begin{enumerate}
\item $M$ ist abzählbar unendlich heißt, es existiert eine \emph{Zählvorschrift}, bei der jedes Element von $M$ nach endlich vielen Schritten erreicht wird.
\item Die Menge $\mathbb{Z}$ der ganzen Zahlen ist abzählbar unendlich. \\
Andordnen nach steigendem Betrag:\\
$\mathbb{Z}=\{0,-1,1,-2,2,-3,3, ...\}$
\item $\mathbb{Q}^+$ … Menge der pos. rationalen Zahlen\\
ABB 61\\
Zählvorschrift:
\begin{enumerate}
\item (aufsteigend) Ordnen nach Summen von Zöhler und Nenner
\item Zahlen mit gleicher Summe der Größe nach aufsteigend anordnen.
\item Bereits enthaltene Zählen (=kürzbare Brücke) weglassen.
\end{enumerate}
$\mathbb{Q}^+=\left\lbrace\frac{1}{1},\frac{1}{2}, \frac{2}{1}, \frac{1}{3}\textcolor{red}{, \frac{3}{2}}, \frac{3}{1}, ... \right\rbrace$\\
analog zu $\mathbb{Z}$: Die Menge $\mathbb{Q}$ aller rationalen Zahlen (also $\mathbb{Q}^-$ zusammen mit $\mathbb{Q}^+$) ist abzählbar unendlich.
\item Es gibt Mengen, die mächtiger sind als die Menge der natürlichen Zahlen: \emph{überabzählbare Mengen} ($B$ heißt \emph{mächtiger} als $A$, wenn se eine injektive Abbildung $f: A \rightarrow B$ gibt, aber keine bijektive. Schreibweise: $|A|<|B|$).\\
\end{enumerate}
z. B. gilt:
\subparagraph{Satz 5:} Die Menge $M=\{x \in \mathbb{R} | 0<x<1\}=(0,1)$ ist überabzählbar.\\
Beweis: (\emph{CANTORsches Diagonalverfahren})\\
Indirekt, angenommen $M=(0,1)$ sei abzhälbar unendlich, d.h. $M=\{x_1, x_2, x_3, ...\}$.\\
Für die Zahlen $x_k$ wählen wir z.B. die eindeutige Darstellung als Dezimalbruch (9er Periode vermeiden). Also bspw. $0,39999...=0,3\overline{9}=0,4=0,40000...$)\\
$x_{\textcolor{green}{1}}=0,a_1^{\textcolor{green}{(1)}}a_2^{\textcolor{green}{(1)}}a_3^{\textcolor{green}{(1)}}...\\
x_2=0,a_1^{(2)}a_2^{(2)}a_3^{(2)}...\\
x_3=0,a_1^{(3)}a_2^{(3)}a_3^{(3)}...\\
...$\\
Es sei $z=0,b_1 b_2 b_3 ...$ mit 
$b_k=\begin{cases} 
1 & \text{falls } a_k^{(k)}\not = 1\\
2 & \text{falls } a_k^{(k)} = 1\\
\end{cases}$
für $k=1,2,3,...$\\
Damit unterscheiden sich $x_k$ und $z$ an der $k$-ten Stelle, d.h. $z \not = x_k$ für alle $k\geq 1$. $z$ ist also nicht in der Folge $x_1, x_2, x_3, ...$ enthalten, also \emph{$z\not \in M$}.\\
Andererseits ist $0<z<1$ also \emph{$z\in (0,1) = M$}. \lightning\, Widerspruch! \#

\subparagraph{Satz 6:} \parskp
Es sei $E$ eine Menge. Dann ist die Potenzmenge $M=\mathcal{P}(E)$ mächtiger als $E$.\\
Beweis: 
\begin{enumerate}
\item Die Abbildung $f: E \rightarrow M$ mit $f(x)=\{x\}$, die jedem $x \in E$ die einelementige Menge $\{x\}\in M$ zuordnet, ist injektiv.
\item Angenommen, es gäbe eine bijektive (damit auch surjektive) Abbildung $g: E \rightarrow M$. Es sei $A = \{x \in E| x \not \in g (x)\}\in M$ ($A$ Teilmenge von $E$). Da $g$ surjektiv ist, gibt ein $a \in E$ mit $g(a)=A$. Fallunterscheidung:
\begin{enumerate} %[label=\arab*.]
\item $a\in A= g(a) \Rightarrow a \not \in g(a)$ \lightning\, Widerspruch!
\item $a \not \in A =g(a) \Rightarrow a \in g(a)$ \lightning\, Widerspruch!
\end{enumerate}
Beide Fälle führen auf einen Widerspruch, es gibt keine surjektive und damit auch keine bijektive Abbildung von $E$ auf $\mathcal{P}(E)$. \#
\end{enumerate}

\subparagraph{Diskussion:}\parskp
Satz 6 zeigt, dass es unendlich viele unendliche Mächtigkeiten gibt. So gilt bspw. $|\mathbb{N}|<|\mathcal{P}(\mathbb{N})|<|\mathcal{P}(|\mathcal{P}(\mathbb{N})|)|$ usw. 

\subparagraph{Satz 7:} \parskp
Die Potenzmenge $\mathcal{P}(\mathbb{N})$ der Menge der natürlichen Zahlen ist gleichmächtig mit dem Intervall $(0,1)$, also überabzählbar (Beweis: siehe Übungsaufgabe 1.38).

\end{document}