\lecdate{25.04.2017}
\slides{it1-14-email_print}{3}

\subsubsection*{Struktur}
\slides{it1-14-email_print}{5}

\subsubsection*{DNS}
\slides{it1-14-email_print}{7}

\section{SMTP}
\slides{it1-14-email_print}{4}
\slides{it1-14-email_print}{8}
\subsection{SMTP-Relay-Server}
\slides{it1-14-email_print}{6}
\subsection{Kommandos}
\slides{it1-14-email_print}{9}
\subsection{Antwortcodes}
\slides{it1-14-email_print}{10}
\subsection{Beispiel}
\subsubsection{Sitzung}
\slides{it1-14-email_print}{11}
\subsubsection{Email}
\slides{it1-14-email_print}{12}
\subsection{Extended SMTP (ESMTP)}
\slides{it1-14-email_print}{13}
\subsection{Selber ausprobieren}
\slides{it1-14-email_print}{14}
\subsection{Zusammenfassung}
\slides{it1-14-email_print}{15}

\section{Zugriffsprotokolle}
\slides{it1-14-email_print}{16}
\subsection{Transportverschlüsselung}
\slides{it1-14-email_print}{17}
\subsection{Vertraulichkeit bei Transport}
\slides{it1-14-email_print}{18}
\subsection{POP3 und IMAP}
\slides{it1-14-email_print}{19}
\subsubsection{POP3 Protokoll}
\slides{it1-14-email_print}{20}
\subsubsection{POP3 Befehle}
\slides{it1-14-email_print}{21}
\subsubsection{IMAP Befehle}
\slides{it1-14-email_print}{22}

\section{E-Mail Format}
\slides{it1-14-email_print}{23}
\subsection{Header}
\slides{it1-14-email_print}{24}
\slides{it1-14-email_print}{25}
\subsection{None Delivery Notification (NDN) -- Bounce}
\slides{it1-14-email_print}{26}
\subsection{Multipurpose Internet Mail Extension (MIME)}
\slides{it1-14-email_print}{27}
\subsubsection{Header}
\slides{it1-14-email_print}{28}
\subsubsection{Datentypen}
\slides{it1-14-email_print}{29}
\subsubsection{Codierschema}
\slides{it1-14-email_print}{30}
\subsubsection{Quoted-Printable}
\slides{it1-14-email_print}{31}
\subsubsection{Base64}
\slides{it1-14-email_print}{32}
\slides{it1-14-email_print}{33}
\subsection{Headercodierung}
\slides{it1-14-email_print}{34}
\subsection{Multipart}
\slides{it1-14-email_print}{35}
\subsubsection*{Beispiel}
\slides{it1-14-email_print}{36}

\section{Verschlüsselung}
\slides{it1-14-email_print}{37}
\subsection{Secure Multiparts for MIME}
\slides{it1-14-email_print}{38}

\subsection{OpenPGP}
\slides{it1-14-email_print}{39}
\subsubsection{Beispiele}
\subsubsection*{Signierte Mail}
\slides{it1-14-email_print}{40}
\subsubsection*{Verschlüsselte Mail}
\slides{it1-14-email_print}{41}
\subsubsection*{Inline-Verschlüsselung}
\slides{it1-14-email_print}{42}
\subsubsection*{Inline-Signierung}
\slides{it1-14-email_print}{43}
\subsubsection*{Inline mit Anhang}
\slides{it1-14-email_print}{44}
\subsubsection*{S/MIME Signatur}
\slides{it1-14-email_print}{45}

\section{Spam-Vermeidung}
\slides{it1-14-email_print}{46}
Graue Listen: Mail wird beim ersten Sendeversuch geblockt. Theoretisch werden Spam-Mails nicht erneut versendet, also kommen nur nicht-Spam-Mails letztendlich an.

\subsection{Reverse DNS}
\slides{it1-14-email_print}{47}
\subsection{Forward-confirmed Reverse DNS}
\slides{it1-14-email_print}{48}
\subsection{Prüfung der HELO Meldung}
\slides{it1-14-email_print}{49}
\subsection{Sender Policy Framework (SPF)}
\slides{it1-14-email_print}{50}
\subsubsection{Beispiel}
\slides{it1-14-email_print}{51}
\slides{it1-14-email_print}{52}
\subsection{Ursachen für Blacklisting}
\lecdate{02.05.2017}
\slides{it1-14-email_print}{53}
\subsection{Greylisting}
\slides{it1-14-email_print}{54}
\slides{it1-14-email_print}{55}

\subsection{DomainKeys Identified Mail (DKIM)}
\lecdate{02.05.2017}
\slides{it1-14-email_print}{56}
(Senderidentifikation)
\slides{it1-14-email_print}{57}
\subsubsection{Beispiel}
\slides{it1-14-email_print}{58}
\subsubsection{Diskussion}
\slides{it1-14-email_print}{59}

\section{Java-Mail-API}
\slides{it1-14-email_print}{60}

\section{Fragestullungen}
\slides{it1-14-email_print}{62}

\section{Zusammenfassung}
\slides{it1-14-email_print}{63}