\section{Grundlagen}
\slides{V4_Parser}{1}
\subsection{Kellerautomat}
\slides{V4_Parser}{2}
\slides{V4_Parser}{3}

\section{LL(1)-Grammatik}
\slides{V4_Parser}{4}
\slides{V4_Parser}{5}
Hinweis: In PL0 darf das \lstinline`statement` auch leer sein (da eckige Klammern).
\subsection{First/Follow}
\slides{V4_Parser}{6}
\subsection{Als Graph}
\subsubsection{programm}
\slides[.5]{V4_Parser}{7}
\subsubsection{block}
\slides[.5]{V4_Parser}{8}
\subsubsection{statement}
\slides[.5]{V4_Parser}{9}
\subsubsection{expr}
\lecdate{08.11.2017}
\slides[.5]{V4_Parser}{10}
1. Variante reicht für Parser, 3. Variante ist sinnvoller, wenn das Ergebnis im Compiler weiter genutzt werden soll.
\subsubsection{term}
\slides[.5]{V4_Parser}{11}
\subsubsection{factor}
\slides[.5]{V4_Parser}{12}
\subsubsection{condition}
\slides[.5]{V4_Parser}{13}
\subsubsection{Bewertung}
\slides{V4_Parser}{14}
\slides{V4_Parser}{15}
\subsubsection*{Alternative}
\slides{V4_Parser}{16}
\slides{V4_Parser}{17}

\section{Implementation von Graphen}
\slides{V4_Parser}{18}
\subsection{Struktur eines Bogens}
\slides{V4_Parser}{19}
\subsubsection*{Beispiel Factor}
\begin{lstlisting}[language=C]
typedef unsigned long ul;
tBog gFact[] = {
/* 0*/ {BgMo,{(ul)mcIdent	}, NULL, 5, 1},	/*(0)---ident--->(E)*/
/* 1*/ {BgMo,{(ul)mcNumb 	}, NULL, 5, 2},	/* +----number-->(E)*/
/* 2*/ {BgSy,{(ul)'(' 		}, NULL, 3, 0},	/*(+)----'('---->(3)*/
/* 3*/ {BgGr,{(ul)gExpr 	}, NULL, 4, 0},	/*(1)--express-->(4)*/
/* 4*/ {BgSy,{(ul)')' 		}, NULL, 5, 0},	/*(0)----')'---->(E)*/
/* 5*/ {BgEn,{(ul)0 			}, NULL, 0, 0} 	/*(E)---------(ENDE)*/
};
\end{lstlisting}
\slides[.5]{V4_Parser}{20}
Hinweis: $0$ in \lstinline`iAlt` bedeutet keine Alternative (vorher festgelegt, könnte auch $-1$ sein).
\subsection{Erstellen eines Graphen}
\slides[.5]{V4_Parser}{21}
\begin{lstlisting}[language=C]
tBog gFact[]={
/* 0*/ {BgMo,{(ul)mcIdent	}, NULL, 5, 1},
/* 1*/ {BgMo,{(ul)mcNumb	}, NULL, 5, 2},
/* 2*/ {BgSy,{(ul)'(' 		}, NULL, 3, 0},
/* 3*/ {BgGr,{(ul)gExpr 	}, NULL, 4, 0},
/* 4*/ {BgSy,{(ul)')' 		}, NULL, 5, 0},
/* 5*/ {BgEn,{(ul)0 			}, NULL, 0, 0}
};
\end{lstlisting}
\subsubsection*{Hinweise zur Umsetzung}
\slides{V4_Parser}{22}
\subsection{Implementation von Bögen in Graphen}
\slides[.5]{V4_Parser}{23}
\begin{lstlisting}[language=C]
int pars(tBog* pGraph){
	tBog* pBog=pGraph;
	int succ=0;
	if (Morph.MC==mcEmpty)Lex();
	while(1){
		switch(pBog->BgD){
			case BgNl:
				succ=1;
				break;
			case BgSy:
				succ=(Morph.Val.Symb==pBog->BgX.S);
				break;
			case BgMo:
				succ=(Morph.MC==(tMC)pBog->BgX.M);
				break;
			case BgGr:
				succ=pars(pBog->BgX.G);
				break;
			case BgEn:
				return 1; /* Ende erreichet Erfolg */
		}
		if (succ && pBog->fx!=NULL)
			succ=pBog->fx();
		if (!succ)/* Alternativbogen probieren */
			if (pBog->iAlt != 0)
				pBog=pGraph+pBog->iAlt;
			else
				return FAIL;
		else{/* Morphem formal akzeptiert (eaten) */
			if (pBog->BgD & BgSy || pBog->BgD & BgMo)
				Lex();
			pBog=pGraph+pBog->iNext;
		}
	}/* while */
}
\end{lstlisting}
\subsection{Parsebaum (optional)}
\slides{V4_Parser}{25}
\begin{lstlisting}[language=C]
typedef struct{
	int 		line;
	char 		Type[10];
	tList* 	pList;
	char * 	Descr;
}TreeItem;

TreeItem* crItem(	tList* pList, // neue Liste od. NULL
									char* Descr, // Description
									char* Type, // Bogentyp
									int line){ // Quellzeile
	TreeItem *ptmp=malloc(sizeof(TreeItem));
	strncpy(ptmp->Type,Type,9);
	ptmp->pList=pList;
	ptmp->Descr=Descr;
	ptmp->line=line;
	return ptmp;
}
\end{lstlisting}
Aufruf:
\begin{lstlisting}[language=C]
InsertHead(pTree,crItem((pl=CreateList()),"Prog","MetaSymb",0));
if (pars(gProg,pl)==1)
/* ... */

int pars(tBog* pGraph,tList* pList){
	int verarbMorph=0;
	tBog* pBog=pGraph;
	int succ;
	tList* pl=NULL;
	TreeItem *pItem=NULL;
	if (Morph.MC==mcEmpty)Lex();
	while(1){
		switch(pBog->BgD & (BgNl+BgSy+BgMo+BgGr+BgEn)){
			case BgNl:
				succ=1;
				break;
			case BgSy:
				succ=(Morph.Val.Symb==pBog->BgX.S);
				if (succ)
					InsertTail(pList,crItem(NULL,(crStr(Morph.vBuf)),"Symbol",Morph.PosLine));
				break;
			case BgMo:
				succ=(Morph.MC==(tMC)pBog->BgX.M);
				if (succ)
					InsertTail(pList,crItem(NULL,(crStr(Morph.vBuf)),"Morph",Morph.PosLine));
				break;
			case BgGr:
				pl=CreateList();
				pItem=crItem((pl),crStr(StrGr[pBog->BgX.G]),"MetaSymb",Morph.PosLine);
				InsertTail(pList,pItem);
				succ=pars(vGr[pBog->BgX.G],pl);
				if (succ!=OK){
					DeleteList(pl); 
					free(pItem->Descr);
					free(pItem); 
					GetLast(pList); 
					RemoveItem(pList);
				}
				break;
			case BgEn:
				return 1; /* Ende erreichet - Erfolg */
		}
		
		if (succ && pBog->fx!=NULL)
			succ=pBog->fx();
		if (!succ){/* Alternativbogen probieren */
			if (pBog->iAlt != 0)
				pBog=pGraph+pBog->iAlt;
			/* Graph nicht erfolgreich verlassen */
			else if (verarbMorph)
				return ERROR; /* Syntaxfehler */
			else{
				//printf("go back\n");
				while(pItem=GetFirst(pList)){
					free(pItem->Descr);
					free(pItem);
					RemoveItem(pList);
				}
				return FAIL; /* vielleicht gibt es noch Alternative */
			}
		} else { /* Morphem formal akzeptiert */
			if (pBog->BgD & BgSy || pBog->BgD & BgMo){
				verarbMorph=1;
				Lex();
			}
			pBog=pGraph+pBog->iNext;
		}
	}
}
\end{lstlisting}
\slides{V4_Parser}{27}

\subsection{Erweiterte Graphen mit Backtracking}
\slides[.5]{V4_Parser}{30}
\slides[.5]{V4_Parser}{31}
\slides[.5]{V4_Parser}{32}
\slides[.5]{V4_Parser}{33}
\begin{lstlisting}[language=C]
tBog gStmnt[]= { 
/* 0*/ {BgGr,{(unsigned long)iAssign	}, NULL, 7, 1},
/* 1*/ {BgGr,{(unsigned long)iCall 		}, NULL, 7, 2},
/* 2*/ {BgGr,{(unsigned long)iBegin 	}, NULL, 7, 3},
/* 3*/ {BgGr,{(unsigned long)iIf 			}, NULL, 7, 4},
/* 4*/ {BgGr,{(unsigned long)iWhile 	}, St4 , 7, 5},
/* 5*/ {BgGr,{(unsigned long)iInput 	}, NULL, 7, 6},
/* 6*/ {BgGr,{(unsigned long)iOutput	}, NULL, 7, 0},
/* 7*/ {BgEn,{(unsigned long)0 				}, NULL, 0, 0}
};
tBog gAssign[]={
/* 0*/ {BgMo,{(unsigned long)mcIdent	}, St0 , 1, 0},
/* 1*/ {BgSy,{(unsigned long)zErg 		}, NULL, 2, 0},
/* 2*/ {BgGr,{(unsigned long)iExpr 		}, St8 , 3, 0},
/* 3*/ {BgEn,{(unsigned long)0 				}, NULL, 0, 0}
};
\end{lstlisting}





