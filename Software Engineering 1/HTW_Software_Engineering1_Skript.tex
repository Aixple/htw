\newcommand{\customDir}{../}
\RequirePackage{ifthen,xifthen}

% Input inkl. Umlaute, Silbentrennung
\RequirePackage[T1]{fontenc}
\RequirePackage[utf8]{inputenc}

% Arbeitsordner (in Abhängigkeit vom Master) Standard: .LateX_master Ordner liegt im Eltern-Ordner
\providecommand{\customDir}{../}
\newcommand{\setCustomDir}[1]{\renewcommand{\customDir}{#1}}
%%% alle Optionen:
% Doppelseitig (mit Rand an der Innenseite)
\newboolean{twosided}
\setboolean{twosided}{false}
% Eigene Dokument-Klasse (alle KOMA möglich; cheatsheet für Spicker [3 Spalten pro Seite, alles kleiner])
\newcommand{\customDocumentClass}{scrreprt}
\newcommand{\setCustomDocumentClass}[1]{\renewcommand{\customDocumentClass}{#1}}
% Unterscheidung verschiedener Designs: htw, fjs
\newcommand{\customDesign}{htw}
\newcommand{\setCustomDesign}[1]{\renewcommand{\customDesign}{#1}}
% Dokumenten Metadaten
\newcommand{\customTitle}{}
\newcommand{\setCustomTitle}[1]{\renewcommand{\customTitle}{#1}}
\newcommand{\customSubtitle}{}
\newcommand{\setCustomSubtitle}[1]{\renewcommand{\customSubtitle}{#1}}
\newcommand{\customAuthor}{}
\newcommand{\setCustomAuthor}[1]{\renewcommand{\customAuthor}{#1}}
%	Notiz auf der Titelseite (A: vor Autor, B: nach Autor)
\newcommand{\customNoteA}{}
\newcommand{\setCustomNoteA}[1]{\renewcommand{\customNoteA}{#1}}
\newcommand{\customNoteB}{}
\newcommand{\setCustomNoteB}[1]{\renewcommand{\customNoteB}{#1}}
% Format der Signatur in Fußzeile:
\newcommand{\customSignature}{\ifthenelse{\equal{\customAuthor}{}} {} {\footnotesize{\textcolor{darkgray}{Mitschrift von\\ \customAuthor}}}}
\newcommand{\setCustomSignature}[1]{\renewcommand{\customSignature}{#1}}
% Format des Autors auf dem Titelblatt:
\newcommand{\customTitleAuthor}{\textcolor{darkgray}{Mitschrift von \customAuthor}}
\newcommand{\setCustomTitleAuthor}[1]{\renewcommand{\customTitleAuthor}{#1}}
% Standard Sprache
\newcommand{\customDefaultLanguage}[1]{}
\newcommand{\setCustomDefaultLanguage}[1]{\renewcommand{\customDefaultLanguage}{#1}}
% Folien-Pfad (inkl. Dateiname ohne Endung und ggf. ohne Nummerierung)
\newcommand{\customSlidePath}{}
\newcommand{\setCustomSlidePath}[1]{\renewcommand{\customSlidePath}{#1}}
% Folien Eigenschaften
\newcommand{\customSlideScale}{0.5}
\newcommand{\setCustomSlideScale}[1]{\renewcommand{\customSlideScale}{#1}}
\newcommand{\customSlideHeight}{9.63cm}
\newcommand{\setCustomSlideHeight}[1]{\renewcommand{\customSlideHeight}{#1}}
\newcommand{\customSlideWidth}{12.8cm}
\newcommand{\setCustomSlideWidth}[1]{\renewcommand{\customSlideWidth}{#1}}

%\setboolean{twosided}{true}
%\setCustomDocumentClass{scrartcl}
%\setCustomDesign{htw}
\setCustomSlidePath{Vorlesung/VO}
\setCustomSlideScale{1.5}

\setCustomTitle{Software Engineering 1}
\setCustomSubtitle{Vorlesungsskript}
\setCustomAuthor{Falk-Jonatan Strube}
%\setCustomNoteA{TitlepageNoteBeforeAuthor}
\setCustomNoteB{Vorlesung von Prof. Dr. Hauptmann}

%\setcustomSignature{\footnotesize{\textcolor{darkgray}{Mitschrift von\\ \customAuthor}}	% Formatierung der Signatur in der Fußzeile
%\setcustomTitleAuthor{\textcolor{darkgray}{Mitschrift von #1}}	% Formatierung des Autors auf dem Titelblatt

%-- Prüfen, ob Beamer
\ifthenelse{\equal{\customDocumentClass}{beamer}}{
%%% TODO: andere Layouts für Beamer außer HTW
	\documentclass[ignorenonframetext, 11pt, table]{beamer}
	
	\usenavigationsymbolstemplate{}
	\setbeamercolor{author in head/foot}{fg=black}
	\setbeamercolor{title}{fg=black}
	\setbeamercolor{bibliography entry author}{fg=htworange!70}
	%\setbeamercolor{bibliography entry title}{fg=blue} 
	\setbeamercolor{bibliography entry location}{fg=htworange!60} 
	\setbeamercolor{bibliography entry note}{fg=htworange!60}  
	
	\setbeamertemplate{itemize item}{\color{black}$\bullet$}
	\setbeamertemplate{itemize subitem}{\color{black}--}
	\setbeamertemplate{itemize subsubitem}{\color{black}$\bullet$}
	\makeatother
	\setbeamertemplate{footline}
	{
	\leavevmode
	\def\arraystretch{1.2}
	\arrayrulecolor{gray}
	\begin{tabular}{ p{0.167\textwidth} | p{0.491\textwidth} | p{0.089\textwidth} | p{0.103\textwidth}}
	\hline
	\strut\insertshortauthor & \insertshorttitle & Slide \insertframenumber{}% / \inserttotalframenumber{}
	 & May 4, 2016\\
	\end{tabular}
	}
	\setbeamertemplate{headline}
	{
	\leavevmode
	\setlength{\arrayrulewidth}{1pt}
	\hspace*{2em}	
	\begin{tabular}{p{0.63\textwidth}}
	\rule{0pt}{3em}\normalsize{\textbf{\insertsection\strut}}\\
	\arrayrulecolor{htworange}
	\hline
	\end{tabular}
	\begin{tabular}{l}
	\rule{0pt}{4em}\includegraphics[width=3.25cm]{\customDir .LaTeX_master/HTW_GESAMTLOGO_CMYK.eps}\\
	\end{tabular}
	}
	\makeatletter	
}{	
	%-- Für Spicker einiges anders:
	\ifthenelse{\equal{\customDocumentClass}{cheatsheet}}{
		\documentclass[a4paper,10pt,landscape]{scrartcl}
		\usepackage{geometry}
		\geometry{top=2mm, bottom=2mm, headsep=0mm, footskip=0mm, left=2mm, right=2mm}
		
		% Für Spicker \spsection für Section, zur Strukturierung \HRule oder \HDRule Linie einsetzen
		\usepackage{multicol}
		\newcommand{\spsection}[1]{\textbf{#1}}	% Platzsparende "section" für Spicker
	}{	%-- Ende Spicker-Unterscheidung-if
		%-- Unterscheidung Doppelseitig
		\ifthenelse{\boolean{twosided}}{
			\documentclass[a4paper,11pt, footheight=26pt,twoside]{\customDocumentClass}
			\usepackage[head=23pt]{geometry}	% head=23pt umgeht Fehlerwarnung, dafür größeres "top" in geometry
			\geometry{top=30mm, bottom=22mm, headsep=10mm, footskip=12mm, inner=27mm, outer=13mm}
		}{
			\documentclass[a4paper,11pt, footheight=26pt]{\customDocumentClass}
			\usepackage[head=23pt]{geometry}	% head=23pt umgeht Fehlerwarnung, dafür größeres "top" in geometry
			\geometry{top=30mm, bottom=22mm, headsep=10mm, footskip=12mm, left=20mm, right=20mm}
		}
		%-- Nummerierung bis Subsubsection für Report
		\ifthenelse{\equal{\customDocumentClass}{report} \OR \equal{\customDocumentClass}{scrreprt}}{
			\setcounter{secnumdepth}{3}	% zählt auch subsubsection
			\setcounter{tocdepth}{3}	% Inhaltsverzeichnis bis in subsubsection
		}{}
	}%-- Ende Spicker-Unterscheidung-else
	
	\usepackage{scrlayer-scrpage}	% Kopf-/Fußzeile
	\renewcommand*{\thefootnote}{\fnsymbol{footnote}}	% Fußnoten-Symbole anstatt Zahlen
	\renewcommand*{\titlepagestyle}{empty} % Keine Seitennummer auf Titelseite
	\usepackage[perpage]{footmisc}	% Fußnotenzählung Seitenweit, nicht Dokumentenweit
}

% Input inkl. Umlaute, Silbentrennung
\RequirePackage[T1]{fontenc}
\RequirePackage[utf8]{inputenc}
\usepackage[english,ngerman]{babel}
\usepackage{csquotes}	% Anführungszeichen
\RequirePackage{marvosym}
\usepackage{eurosym}

% Style-Aufhübschung
\usepackage{soul, color}	% Kapitälchen, Unterstrichen, Durchgestrichen usw. im Text
%\usepackage{titleref}

% Mathe usw.
\usepackage{amssymb}
\usepackage{amsthm}
\ifthenelse{\equal{\customDocumentClass}{beamer}}{}{
\usepackage[fleqn,intlimits]{amsmath}	% fleqn: align-Umgebung rechtsbündig; intlimits: Integralgrenzen immer ober-/unterhalb
}
%\usepackage{mathtools} % u.a. schönere underbraces
\usepackage{xcolor}
\usepackage{esint}	% Schönere Integrale, \oiint vorhanden
\everymath=\expandafter{\the\everymath\displaystyle}	% Mathe Inhalte werden weniger verkleinert
\usepackage{wasysym}	% mehr Symbole, bspw \lightning
% Auch arcus-Hyperbolicus-Funktionen
\DeclareMathOperator{\arccot}{arccot}
\DeclareMathOperator{\arccosh}{arccosh}
\DeclareMathOperator{\arcsinh}{arcsinh}
\DeclareMathOperator{\arctanh}{arctanh}
\DeclareMathOperator{\arccoth}{arccoth} 
%\renewcommand{\int}{\int\limits}
%\usepackage{xfrac}	% mehr fracs: sfrac{}{}
\let\oldemptyset\emptyset	% schöneres emptyset
\let\emptyset\varnothing
%\RequirePackage{mathabx}	% mehr Symbole
\mathchardef\mhyphen="2D	% Hyphen in Math

% tikz usw.
\usepackage{tikz}
\usepackage{pgfplots}
\pgfplotsset{compat=1.11}	% Umgeht Fehlermeldung
\usetikzlibrary{graphs}
%\usetikzlibrary{through}	% ???
\usetikzlibrary{arrows}
\usetikzlibrary{arrows.meta}	% Pfeile verändern / vergrößern: \draw[-{>[scale=1.5]}] (-3,5) -> (-3,3);
\usetikzlibrary{automata,positioning} % Zeilenumbruch im Node node[align=center] {Text\\nächste Zeile} automata für Graphen
\usetikzlibrary{matrix}
\usetikzlibrary{patterns}	% Schraffierte Füllung
\usetikzlibrary{shapes.geometric}	% Polygon usw.
\tikzstyle{reverseclip}=[insert path={	% Inverser Clip \clip
	(current page.north east) --
	(current page.south east) --
	(current page.south west) --
	(current page.north west) --
	(current page.north east)}
% Nutzen: 
%\begin{tikzpicture}[remember picture]
%\begin{scope}
%\begin{pgfinterruptboundingbox}
%\draw [clip] DIE FLÄCHE, IN DER OBJEKT NICHT ERSCHEINEN SOLL [reverseclip];
%\end{pgfinterruptboundingbox}
%\draw DAS OBJEKT;
%\end{scope}
%\end{tikzpicture}
]	% Achtung: dafür muss doppelt kompliert werden!
\usepackage{graphpap}	% Grid für Graphen
\tikzset{every state/.style={inner sep=2pt, minimum size=2em}}
\usetikzlibrary{mindmap, backgrounds}
%\usepackage{tikz-uml}	% braucht Dateien: http://perso.ensta-paristech.fr/~kielbasi/tikzuml/

% Tabular
\usepackage{longtable}	% Große Tabellen über mehrere Seiten
\usepackage{multirow}	% Multirow/-column: \multirow{2[Anzahl der Zeilen]}{*[Format]}{Test[Inhalt]} oder \multicolumn{7[Anzahl der Reihen]}{|c|[Format]}{Test2[Inhalt]}
\renewcommand{\arraystretch}{1.3} % Tabellenlinien nicht zu dicht
\usepackage{colortbl}
\arrayrulecolor{gray}	% heller Tabellenlinien
\usepackage{array}	% für folgende 3 Zeilen (für Spalten fester breite mit entsprechender Ausrichtung):
\newcolumntype{L}[1]{>{\raggedright\let\newline\\\arraybackslash\hspace{0pt}}m{\dimexpr#1\columnwidth-2\tabcolsep-1.5\arrayrulewidth}}
\newcolumntype{C}[1]{>{\centering\let\newline\\\arraybackslash\hspace{0pt}}m{\dimexpr#1\columnwidth-2\tabcolsep-1.5\arrayrulewidth}}
\newcolumntype{R}[1]{>{\raggedleft\let\newline\\\arraybackslash\hspace{0pt}}m{\dimexpr#1\columnwidth-2\tabcolsep-1.5\arrayrulewidth}}
\usepackage{caption}	% Um auch unbeschriftete Captions mit \caption* zu machen

% Nützliches
\usepackage{verbatim}	% u.a. zum auskommentieren via \begin{comment} \end{comment}
\usepackage{tabto}	% Tabs: /tab zum nächsten Tab oder /tabto{.5 \CurrentLineWidth} zur Stelle in der Linie
\NumTabs{6}	% Anzahl von Tabs pro Zeile zum springen
\usepackage{listings} % Source-Code mit Tabs
\usepackage{lstautogobble} 
\ifthenelse{\equal{\customDocumentClass}{beamer}}{}{
\usepackage{enumitem}	% Anpassung der enumerates
%\setlist[enumerate,1]{label=(\arabic*)}	% global andere Enum-Items
\renewcommand{\labelitemiii}{$\scriptscriptstyle ^\blacklozenge$} % global andere 3. Item-Aufzählungszeichen
}
\newenvironment{anumerate}{\begin{enumerate}[label=(\alph*)]}{\end{enumerate}} % Alphabetische Aufzählung
\usepackage{letltxmacro} % neue Definiton von Grundbefehlen
% Nutzen:
%\LetLtxMacro{\oldemph}{\emph}
%\renewcommand{\emph}[1]{\oldemph{#1}}
\RequirePackage{xpatch}	% ua. Konkatenieren von Strings/Variablen (etoolbox)


% Einrichtung von lst
\lstset{
basicstyle=\ttfamily, 
%mathescape=true, 
%escapeinside=^^, 
autogobble, 
tabsize=2,
basicstyle=\footnotesize\sffamily\color{black},
frame=single,
rulecolor=\color{lightgray},
numbers=left,
numbersep=5pt,
numberstyle=\tiny\color{gray},
commentstyle=\color{gray},
keywordstyle=\color{green},
stringstyle=\color{orange},
morecomment=[l][\color{magenta}]{\#}
showspaces=false,
showstringspaces=false,
breaklines=true,
literate=%
    {Ö}{{\"O}}1
    {Ä}{{\"A}}1
    {Ü}{{\"U}}1
    {ß}{{\ss}}1
    {ü}{{\"u}}1
    {ä}{{\"a}}1
    {ö}{{\"o}}1
    {~}{{\textasciitilde}}1
}
\usepackage{scrhack} % Fehler umgehen
\def\ContinueLineNumber{\lstset{firstnumber=last}} % vor lstlisting. Zum wechsel zum nicht-kontinuierlichen muss wieder \StartLineAt1 eingegeben werden
\def\StartLineAt#1{\lstset{firstnumber=#1}} % vor lstlisting \StartLineAt30 eingeben, um bei Zeile 30 zu starten
\let\numberLineAt\StartLineAt

% BibTeX
\usepackage[backend=bibtex8, bibencoding=ascii,
%style=authortitle, citestyle=authortitle-ibid,
%doi=false,
%isbn=false,
%url=false
]{biblatex}	% BibTeX
\usepackage{makeidx}
%\makeglossary
%\makeindex

% Grafiken
\usepackage{graphicx}
\usepackage{epstopdf}	% eps-Vektorgrafiken einfügen
%\epstopdfsetup{outdir=\customDir}

% pdf-Setup
\usepackage{pdfpages}
\ifthenelse{\equal{\customDocumentClass}{beamer}}{}{
\usepackage[bookmarks,%
bookmarksopen=false,% Klappt die Bookmarks in Acrobat aus
colorlinks=true,%
linkcolor=black,%
citecolor=red,%
urlcolor=green,%
]{hyperref}
}

%-- Unterscheidung des Stils
\newcommand{\customLogo}{}
\newcommand{\customPreamble}{}
\ifthenelse{\equal{\customDesign}{htw}}{
	% HTW Corporate Design: Arial (Helvetica)
	\usepackage{helvet}
	\renewcommand{\familydefault}{\sfdefault}
	\renewcommand{\customLogo}{HTW-Logo}
	\renewcommand{\customPreamble}{HTW Dresden}
}{
% \renewcommand{\customLogo}{HTW-Logo.eps}
}

% Nach Dokumentenbeginn ausführen:
\AtBeginDocument{
	% Autor und Titel für pdf-Eigenschaften festlegen, falls noch nicht geschehen
	\providecommand{\pdfAuthor}{John Doe}
	\ifdefempty{\customAuthor} {} {\renewcommand{\pdfAuthor}{\customAuthor}}
	\providecommand{\pdfTitle}{}
	\providecommand{\pdfTitleA}{}
	\providecommand{\pdfTitleB}{}
	\providecommand{\pdfTitleC}{}	
	\ifdefempty{\pdfTitle}{
		\ifdefempty{\customPreamble} {} {\renewcommand{\pdfTitleA}{\customPreamble{} | }}
		\ifdefempty{\customTitle} {\renewcommand{\pdfTitleB}{No Title}} {\renewcommand{\pdfTitleB}{\customTitle}}
		\ifdefempty{\customSubtitle} {} {\renewcommand{\pdfTitleC}{ - \customSubtitle}}
	}{}
	
	\newcommand{\customLogoLocation}{\customDir .LaTeX_master/\customLogo}
	\hypersetup{
		pdfauthor={\pdfAuthor},
		pdftitle={\pdfTitleA\pdfTitleB\pdfTitleC},
	}
	\ifthenelse{\equal{\customDocumentClass}{beamer}}{
		\title{\customTitle}
		\author{\customAuthor}
	}{
		\automark[section]{section}
		\automark*[subsection]{subsection}
		\pagestyle{scrheadings}
		\ifthenelse{\equal{\customDocumentClass}{report} \OR \equal{\customDocumentClass}{scrreprt}}{
		\renewcommand*{\chapterpagestyle}{scrheadings}
		}{}
		%\renewcommand*{\titlepagestyle}{scrheadings}
		\ihead{\includegraphics[height=1.7em]{\customLogoLocation}}
		%\ohead{\truncate{5cm}{\customTitle}}
		\ohead{\customTitle}
		\cfoot{\pagemark}
		\ofoot{\customSignature}
		% Titelseite
		\title{
		\includegraphics[width=0.35\textwidth]{\customDir .LaTeX_master/\customLogo}\\\vspace{0.5em}
		\Huge\textbf{\customTitle}
		\ifdefempty{\customSubtitle} {} {\\\vspace*{0.7em}\Large \customSubtitle}
		\\\vspace*{5em}}
		\author{
		\ifdefempty{\customNoteA} {} {\customNoteA \vspace*{1em}}\\ 
		\ifdefempty{\customAuthor} {} {\customTitleAuthor}
		\ifdefempty{\customNoteB}{}{\vspace*{1em}\\\customNoteB}
		}
		
		\ifthenelse{\equal{\customDocumentClass}{cheatsheet}}{
			\pagestyle{empty}
			\setlist{nolistsep}
	%		\usepackage{parskip}	% Aufzählung Abstand
	%		\setlength{\parskip}{0em}
			\lstset{
	    belowcaptionskip=0pt,
	    belowskip=0pt,
	    aboveskip=0pt,
			tabsize=2,
			frame=none,
			numbers=none,
			showspaces=false,
			showstringspaces=false,
			breaklines=true,
			}
		}{}
	}
}

% Unterabschnitte
%\newtheorem{example}{Beispiel}%[section]
%\newtheorem{definition}{Definition}[section]
%\newtheorem{discussion}{Diskussion}[section]
%\newtheorem{remark}{Bemerkung}[section]
%\newtheorem{proof}{Beweis}[section]
%\newtheorem{notation}{Schreibweise}[section]
\RequirePackage{xcolor}
\RequirePackage{amsmath}

% Horizontale Linie:
\newcommand{\HRule}[1][\medskipamount]{\par
  \vspace*{\dimexpr-\parskip-\baselineskip+#1}
  \noindent\rule[0.2ex]{\linewidth}{0.2mm}\par
  \vspace*{\dimexpr-\parskip-.5\baselineskip+#1}}
% Gestrichelte horizontale Linie:
\RequirePackage{dashrule}
\newcommand{\HDRule}[1][\medskipamount]{\par
  \vspace*{\dimexpr-\parskip-\baselineskip+#1}
  \noindent\hdashrule[0.2ex]{\linewidth}{0.2mm}{1mm} \par
  \vspace*{\dimexpr-\parskip-.5\baselineskip+#1}}
% Mathe in Anführungszeichen:
\newsavebox{\mathbox}\newsavebox{\mathquote}
\makeatletter
\newcommand{\mq}[1]{% \mathquotes{<stuff>}
  \savebox{\mathquote}{\text{"}}% Save quotes
  \savebox{\mathbox}{$\displaystyle #1$}% Save <stuff>
  \raisebox{\dimexpr\ht\mathbox-\ht\mathquote\relax}{"}#1\raisebox{\dimexpr\ht\mathbox-\ht\mathquote\relax}{''}
}
\makeatother

% Paragraph mit Zähler (Section-Weise)
\newcounter{cparagraphC}
\newcommand{\cparagraph}[1]{
\stepcounter{cparagraphC}
\paragraph{\thesection{}-\thecparagraphC{} #1}
%\addcontentsline{toc}{subsubsection}{\thesection{}-\thecparagraphC{} #1}
\label{\thesection-\thecparagraphC}
}
\makeatletter
\@addtoreset{cparagraphC}{section}
\makeatother


% (Vorlesungs-)Folien einbinden:
% Folien von einer Datei skaliert
\newcommand{\slide}[2][\customSlideScale]{\slides[#1]{}{#2}}
\newcommand{\slideTrim}[6][\customSlideScale]{\slides[#1 , clip,  trim = #5cm #4cm #6cm #3cm]{}{#2}}
% Folien von mehreren nummerierten Dateien skaliert
\newcommand{\slides}[3][\customSlideScale]{\begin{center}
\includegraphics[page=#3, scale=#1]{\customSlidePath #2.pdf}
\end{center}}

% \emph{} anders definieren
\makeatletter
\DeclareRobustCommand{\em}{%
  \@nomath\em \if b\expandafter\@car\f@series\@nil
  \normalfont \else \scshape \fi}
\makeatother

% unwichtiges
\newcommand{\unimptnt}[1]{{\transparent{0.5}#1}}

% alph. enumerate
\newenvironment{anumerate}{\begin{enumerate}[label=(\alph*)]}{\end{enumerate}} % Alphabetische Aufzählung

%% EINFACHE BEFEHLE

% Abkürzungen Mathe
\newcommand{\EE}{\mathbb{E}}
\newcommand{\QQ}{\mathbb{Q}}
\newcommand{\RR}{\mathbb{R}}
\newcommand{\CC}{\mathbb{C}}
\newcommand{\NN}{\mathbb{N}}
\newcommand{\ZZ}{\mathbb{Z}}
\newcommand{\PP}{\mathbb{P}}
\renewcommand{\SS}{\mathbb{S}}
\newcommand{\cA}{\mathcal{A}}
\newcommand{\cB}{\mathcal{B}}
\newcommand{\cC}{\mathcal{C}}
\newcommand{\cD}{\mathcal{D}}
\newcommand{\cE}{\mathcal{E}}
\newcommand{\cF}{\mathcal{F}}
\newcommand{\cG}{\mathcal{G}}
\newcommand{\cH}{\mathcal{H}}
\newcommand{\cI}{\mathcal{I}}
\newcommand{\cJ}{\mathcal{J}}
\newcommand{\cM}{\mathcal{M}}
\newcommand{\cN}{\mathcal{N}}
\newcommand{\cP}{\mathcal{P}}
\newcommand{\cR}{\mathcal{R}}
\newcommand{\cS}{\mathcal{S}}
\newcommand{\cZ}{\mathcal{Z}}
\newcommand{\cL}{\mathcal{L}}
\newcommand{\cT}{\mathcal{T}}
\newcommand{\cU}{\mathcal{U}}
\newcommand{\cV}{\mathcal{V}}
\renewcommand{\phi}{\varphi}
\renewcommand{\epsilon}{\varepsilon}

% Verschiedene als Mathe-Operatoren
\DeclareMathOperator{\arccot}{arccot}
\DeclareMathOperator{\arccosh}{arccosh}
\DeclareMathOperator{\arcsinh}{arcsinh}
\DeclareMathOperator{\arctanh}{arctanh}
\DeclareMathOperator{\arccoth}{arccoth} 
\DeclareMathOperator{\var}{Var} % Varianz 
\DeclareMathOperator{\cov}{Cov} % Co-Varianz 

% Farbdefinitionen
\definecolor{red}{RGB}{180,0,0}
\definecolor{green}{RGB}{75,160,0}
\definecolor{blue}{RGB}{0,75,200}
\definecolor{orange}{RGB}{255,128,0}
\definecolor{yellow}{RGB}{255,245,0}
\definecolor{purple}{RGB}{75,0,160}
\definecolor{cyan}{RGB}{0,160,160}
\definecolor{brown}{RGB}{120,60,10}

\definecolor{itteny}{RGB}{244,229,0}
\definecolor{ittenyo}{RGB}{253,198,11}
\definecolor{itteno}{RGB}{241,142,28}
\definecolor{ittenor}{RGB}{234,98,31}
\definecolor{ittenr}{RGB}{227,35,34}
\definecolor{ittenrp}{RGB}{196,3,125}
\definecolor{ittenp}{RGB}{109,57,139}
\definecolor{ittenpb}{RGB}{68,78,153}
\definecolor{ittenb}{RGB}{42,113,176}
\definecolor{ittenbg}{RGB}{6,150,187}
\definecolor{itteng}{RGB}{0,142,91}
\definecolor{ittengy}{RGB}{140,187,38}

\definecolor{htworange}{RGB}{249,155,28}

% Textfarbe ändern
\newcommand{\tred}[1]{\textcolor{red}{#1}}
\newcommand{\tgreen}[1]{\textcolor{green}{#1}}
\newcommand{\tblue}[1]{\textcolor{blue}{#1}}
\newcommand{\torange}[1]{\textcolor{orange}{#1}}
\newcommand{\tyellow}[1]{\textcolor{yellow}{#1}}
\newcommand{\tpurple}[1]{\textcolor{purple}{#1}}
\newcommand{\tcyan}[1]{\textcolor{cyan}{#1}}
\newcommand{\tbrown}[1]{\textcolor{brown}{#1}}

% Umstellen der Tabellen Definition
\newcommand{\mpb}[1][.3]{\begin{minipage}{#1\textwidth}\vspace*{3pt}}
\newcommand{\mpe}{\vspace*{3pt}\end{minipage}}

\newcommand{\resultul}[1]{\underline{\underline{#1}}}
\newcommand{\parskp}{$ $\\}	% new line after paragraph
\newcommand{\corr}{\;\widehat{=}\;}
\newcommand{\mdeg}{^{\circ}}

\newcommand{\nok}[2]{\binom{#1}{#2}}	% n über k BESSER: \binom{n}{k}
\newcommand{\mtr}[1]{\begin{pmatrix}#1\end{pmatrix}}	% Matrix
\newcommand{\dtr}[1]{\begin{vmatrix}#1\end{vmatrix}}	% Determinante (Betragsmatrix)
\renewcommand{\vec}[1]{\underline{#1}}	% Vektorschreibweise
\newcommand{\imptnt}[1]{\colorbox{red!30}{#1}}	% Wichtiges
\newcommand{\intd}[1]{\,\mathrm{d}#1}
\newcommand{\diffd}[1]{\mathrm{d}#1}
% für Module-Rechnung: \pmod{}
\newcommand{\unit}[1]{\,\mathrm{#1}}

%\bibliography{\customDir .Literatur/HTW_Literatur.bib}

\begin{document}

%\selectlanguage{english}
\maketitle
\newpage
\tableofcontents
\newpage

\chapter{Einführung}
\slidesScale{1}{3}

\section{Gliederung}
\slidesScale{1}{5}
\slidesScale{1}{6}

\section{Die ersten Phasen}
\slidesScale{1}{14}
\begin{itemize}
\item Wie werden Softwaresysteme entwickelt?
\item Was ist Software-Engineering?
\item Warum haben Analyse und Definition von Anforderungen an das SW-System große Bedeutung im Entwicklungsprozess?
\end{itemize}

\section{Vergleich Bauingenieur und Softwareentwicklung}
\slidesScale[2]{1}{16}
Begleitend:
\begin{itemize}
\item Projektleiter
\item technischer Autor
\item Qualitäts-Beauftragter
\end{itemize}
Zertifizierung für Software: IREB
\slidesScale{1}{17}

\section{Erfolgsquote von Software-Projekten gestern und heute}
\slidesScale{1}{19}
Also: Analyse ist Grundvoraussetzung für erfolgreiche Softwareentwicklung.

Veranschaulichung:
\slidesScale{1}{20}
Warum haben Analyse und Definition von Anforderungen an das SW‐System große Bedeutung
im Entwicklungsprozess?
\begin{itemize}
\item Auch heute noch werden nur die Hälfte aller SW‐Projekte wie geplant realisiert.
\item 60\% der Fehler im SW‐System resultieren aus Fehlern in der Analysephase, d.h. aus Fehlern bei der Definition der Anforderungen.
\item Die Behebung von Fehlern aus der Analysephase sind vergleichsweise sehr teuer.
\end{itemize}

\section{Definitionen Software Engineering}

\slidesScale{1}{22}
\slidesScale{1}{23}
\slidesScale{1}{24}
Also:

\paragraph{Software-Engineering ist}
\begin{itemize}
\item die effektive und effiziente Entwicklung und Weiterentwicklung komplexer SW-Systeme
\item sowie begleitender Dokumente
\item in einem bewusst arbeitsteilig gestalteten Prozess
\item unter Anwendung bewährter Prinzipien, Methoden und Modellen.
\end{itemize}
oder Vergleich:
\slidesScale{1}{25}
%\begin{itemize}
%\item $\Rightarrow$ nützt nichts, sondern verhindert vielmehr Schäden $\to$ sollte generell beachtet werden
%\item „Software Engineering ist -- wie die Hygiene in der Medizin -- langweilig und frustrierend für Leute, die die Abwehr von Fehlschlägen und Katastrophen nicht als positive Leistung betrachten.“
%\end{itemize}
\begin{itemize}
\item sauberes Arbeiten (vgl. Chirurg: kann auch so operieren, wenn er aber ohne Hygiene operiert, ist der Ausgang ungewiss)
\item „Wenn's nicht funktioniert, ist man immer wieder damit beschäftigt die gleichen Fehler zu bearbeiten.“
\end{itemize}

\setCustomSlideScale{.95}
\slidesScale{1}{26}

\chapter{System}

\paragraph{Wie werden Software-Systeme entwickelt?}
\begin{itemize}
\item Bevor mit der Implementierung begonnen wird, werden die Anforderungen ermittelt und beschrieben.
\item Danach wird ausgehend von der Beschreibung der Anforderungen eine Struktur festgelegt, nach der SW-System gebaut wird.
\item Erst dann beginnt die Implementierung.
\item Dabei begleiten ständig Tests und Dokumentation die Arbeit.
\end{itemize}
\paragraph{Was ist „Software‐Engineering“?} (siehe weitere Definitionen)\\
Software‐Engineering ist:
\begin{itemize}
\item die effektive und effiziente Entwicklung und Weiterentwicklung \emph{komplexer} SW-Systeme
\item sowie \emph{begleitender Dokumente}
\item in einem bewusst arbeitsteilig gestalteten \emph{Prozess}
\item unter Anwendung bewährter \emph{Prinzipien, Methoden und Modellen}.
\end{itemize}
\paragraph{Warum haben Analyse und Definition von Anforderungen an das SW‐System große Bedeutung
im Entwicklungsprozess?}
\begin{itemize}
\item Auch heute noch werden nur die Hälfte aller SW‐Projekte wie geplant realisiert.
\item 60\% der Fehler im SW‐system resultieren aus Fehlern in der Analysephase, d.h. aus Fehlern in der Definition der Anforderungen.
\item Die Behebung von Fehlern aus der Analysephase sind vergleichsweise sehr teuer.
\end{itemize}

% Vorlesung 20.10.2016
\setCustomSlideScale{.5}
% \slidesScale{2}{1}

\section{Entwicklung von SW-Entwicklungsstrategien}
Problem der SW-Entwicklung am Ende der 60‐ziger Jahre:\\
veränderte Rahmenbedingungen beim Einsatz und der Entwicklung von Software-Systemen\\
Also: Strategien bei der SW-Entwicklung ändern

\section{Definition Softwaresystem}
\subsection{Was kennzeichnet ein System?}
\begin{itemize}
\item Gesetzmäßigkeiten (Regeln, …) im Inneren, damit es funktioniert [Wenn es ein Innen gibt, muss es auch ein Außen geben]
\item $\Rightarrow$ Ein System besteht aus Komponenten, die Beziehung zueinander haben.
\end{itemize}
\begin{center}
\begin{tikzpicture}[thick, scale=0.6]
\draw  (-11,7) rectangle (11,-11);
\draw  (0,-2) ellipse (8 and 8);
\begin{scope} [xshift=-90, yshift=90,rotate=45]
\fill [white] (-2,2) ellipse (1 and 0.5);
\draw  (-2,2) ellipse (1 and 0.5);
\draw [-latex] (-2.5,1) -- (-2.5,3);
\draw [latex-] (-1.5,1) -- (-1.5,3);
\end{scope}
\begin{scope} [xshift=170, yshift=10,rotate=-45]
\fill [white] (-2,2) ellipse (1 and 0.5);
\draw  (-2,2) ellipse (1 and 0.5);
\draw [-latex] (-2.5,1) -- (-2.5,3);
\draw [latex-] (-1.5,1) -- (-1.5,3);
\end{scope}
\begin{scope} [xshift=250, yshift=-210,rotate=45]
\fill [white] (-2,2) ellipse (1 and 0.5);
\draw  (-2,2) ellipse (1 and 0.5);
\draw [-latex] (-2.5,1) -- (-2.5,3);
\draw [latex-] (-1.5,1) -- (-1.5,3);
\end{scope}
\begin{scope} [xshift=-170, yshift=-290,rotate=-45]
\fill [white] (-2,2) ellipse (1 and 0.5);
\draw  (-2,2) ellipse (1 and 0.5);
\draw [-latex] (-2.5,1) -- (-2.5,3);
\draw [latex-] (-1.5,1) -- (-1.5,3);
\end{scope}
\node at (-9,5) {Person A};
\node at (-8,-9) {anderes System};
\node at (8,-9) {Person B};
\node at (8,5) {…};
\draw  (-5,2) rectangle (-1,0) node[pos=.5]{Kompenente};
\draw  (2,1) rectangle (4,-1);
\draw  (-5,-4) rectangle (-2,-6);
\draw  (2,-5) rectangle (4,-7);
\draw  (-2,-7) rectangle (0,-9);
\draw (-4,0) -- (-4,-4);
\draw (-1,1) -- (2,0);
\draw (-2,0) -- (3,-5);
\draw (0,-8) -- (2,-6);
\draw (-3,-6) -- (-2,-8);
%\draw (-3,0) -- (-1,-4) -- (-1,-7);
\draw plot[smooth, tension=0.5] coordinates {(-3,0) (-1.5,-3.5) (-1,-7)};
\end{tikzpicture} % ABB SE2
\end{center}
\subparagraph{VDI 3633 Blatt 1: System} ( VDI = Verein Deutscher Ingenieure)
\slidesScale{2}{6}
\slidesScale{2}{7}

Also:
\begin{itemize}
\item Ein System besteht aus Komponenten, die miteinander in Beziehung stehen.
\item Eine Grenze trennt das System von seiner Umgebung (auch Kontext genannt).
\item Schnittstellen (Verbindungsstellen) verbinden das System mit seiner Umwelt.
\item Ein System kann Subsysteme enthalten. Ein System befindet sich zum Zeitpunkt t in einem def. Zustand.
\end{itemize}


\subsection{Warum sind Software-Systeme ganz spezielle Systeme?}
\begin{itemize}
\item SW-Systeme sind „immateriell“ und komplex; sie haben keine natürliche Lokalität.
\item SW-Systeme sind aus einem Werkstoff hergestellt, der von sich aus keine Strukturierung im „Großen“ erfordert (Sprich: niemand wird „gezwungen“ die Software zu strukturieren, es ist aber wohl hilfreich).
\end{itemize}

\subsection{Welchen Unterschied gibt es zwischen Modellen und Prototypen bei der Software-Entwicklung?}
\subparagraph{VDI 3633 Blatt 1: Modell}
\slidesScale{2}{10}
\slidesScale{2}{11}
Softwareprojekte können aufgrund ihrer Komplexität niemals nur durch ein Modell abgebildet werden, sondern brauchen mehrere!
\begin{itemize}
\item Modelle bilden ab.
\item Prototypen sind lauffähig.
\end{itemize}

\subsection{Welche Grundprinzipien finden in der SW-Entwicklung Anwendung?}
\begin{itemize}
\item Abstraktion
\slidesScale{2}{15}
\item Strukturierung
\item Zerlegung
\slidesScale{2}{17}
Entwicklerdokumentation braucht auch ein grobe Übersicht (nicht nur Doxygen generiertes)
\item Kapselung
\slidesScale{2}{18}
\item Hierarchisierung
\slidesScale{2}{16}
\item Standardisierung
\item (integrierte) Dokumentation\\
$\Rightarrow$ Weitergabe von Wissen
\slidesScale{2}{22}
\end{itemize}
weitere:
\begin{itemize}
\item Typisierung
\slidesScale{2}{19}
\item Nebenläufigkeit
\slidesScale{2}{20}
\item Persistenz
\slidesScale{2}{21}
\end{itemize}
Also: 
\begin{itemize}
\item Abstraktion 
\item Strukturierung
\item Zerlegung
\item Kapselung
\item Hierarchisierung
\item Standardisierung
\item integrierte Dokumentation
\end{itemize}
$\Rightarrow$ Mittel um Komplexität zu beherrschen

\section{Qualität von Software-Systemen}
\subsection{Was kennzeichnet qualitativ hochwertige Software-Systeme?}
\paragraph{Gedanken:}
\begin{itemize}
\item stabil laufen
\item wenig Fehler
\item erfüllt seine Funktion
\item intuitiv und einfach benutzbar
\item kompatibel
\item wenig Ressourcenverbrauch (effizient)
\item schöne Oberfläche (Achtung, hier wird klar: Qualität kann auch subjektiv sein)
\item transparenter Systemzustand (z.B. aussagefähige Statusmeldungen)
\item …
\end{itemize}
\paragraph{Lösung:}
\begin{itemize}
\item Funktionserfüllung
\item Benutzerfreundlichkeit
\item Wirtschaftlichkeit
\item Je nach gesetzten Prioritäten:
\begin{itemize}
\item Zuverlässigkeit
\item Sicherheit
\item Flexibilität (Änderbarkeit, Erweiterbarkeit, Portabilität, Kompatibilität)
Wiederverwendbarkeit
\end{itemize}
\end{itemize}

\slidesScale{3}{2}
Qualität hat mehrere Sichten:
\begin{itemize}
\item Benutzersicht (Außen)
\item Entwicklersicht (Innen)
\end{itemize}
\slidesScale{3}{3}
\slidesScale{3}{4}
\paragraph{Wann weiß man, ob das SW-System seine Funktion gut erfüllt?}
Übereinstimmung von \emph{geplanter} (spezifiziert z.B. im Pflichtenheft) und realisierter Funktionalität.

\subsection{Qualitätsmerkmale}

\subsubsection{Benutzungsschnittstelle} (DIN EN ISO 9241-110)\\
„Alle Bestandteile eines interaktiven Systems (Software oder Hardware), 
die Informationen und Steuerelemente zur Verfügung stellen,
die für den Benutzer notwendig sind,
um eine bestimmte Arbeitsaufgabe mit dem interaktiven System zu erledigen.“

\subsubsection{Usability} (DIN EN ISO 9241-110)\\
„Ausmaß, in dem ein System, ein Produkt oder eine Dienstleistung
durch bestimmte Benutzer in einem bestimmten Nutzungskontext genutzt werden kann,
um festgelegte Ziele effektiv, effizient und zufriedenstellend zu erreichen.“

\subsubsection{User Centered Design} (DIN EN ISO 9241-210)
\slidesScale{3}{6}
\begin{itemize}
\item Erreichbar gut über Prototypen, schon vor Beta-Tests
\end{itemize}
\paragraph{Benutzbarkeit} \parskp
Achtung: Subjektiv (es gibt auch objektive Regeln); gestützt durch Oberflächenprototyp (Check-Dummy)

\subsubsection{Effizienz} 
\begin{itemize}
\item Speicherausnutzung
\item Transport
\item Antwortzeit (durch bessere Algorithmen (bspw. Sortierung))
\end{itemize}

\subsubsection{Zuverlässigkeit}
\begin{itemize}
\item Fehlerfrei $\to$ Vermeidung von Ausfällen
\end{itemize}

\begin{tikzpicture}[scale=.5]
\draw [-latex] (0,0) -- (0,10) node[left]{$\lambda$ (Ausfallrate)};
\draw [-latex] (0,0) -- (10,0) node[below]{$t$};

\draw (1,7) -- (3,1) -- (8,1) -- (10,7);
\node at (3,0) [below] {$t_1$};
\node at (8,0) [below] {$t_2$};
\end{tikzpicture}
\begin{itemize}
\item Frühausfälle durch Produktionsfehler $t_1$
\item Spätausfälle durch physischen Verschleiß $t_2$
\end{itemize}

\begin{tikzpicture}[scale=.5]
\draw [-latex] (0,0) -- (0,10) node[left]{$\lambda$ (Ausfallrate)};
\draw [-latex] (0,0) -- (10,0) node[below]{$t$};

\draw (1,7) -- (3,1) -- (8,1) -- (10,7);
\node at (3,0) [below] {$t_1$};
\node at (8,0) [below] {$t_2$};
\draw (11,8) -- (12,8)  node[right]{Hardware $\to$ Badewannenkurve};
\draw [blue] (11,6) -- (12,6)  node[right]{Software (ideal)};
\draw [orange] (11,4) -- (12,4)  node[right, align=left]{Software (real möglich: \\ Fehler behoben oder „zu Tode gepflegt“)};
\draw [blue] plot[smooth, tension=0.5] coordinates { (1,7) (4,2) (7,1) (10,0.75)};
\draw [orange] (1,7) -- (3,5) -- (5,6);
\draw [orange] (3,5) -- (5,4);
\end{tikzpicture}

\subsubsection{Sicherheit}
\begin{itemize}
\item Vermeidung von gefährlichen Zuständen
\end{itemize}

Mögliche Lösung/Herangehensweise für Zuverlässigkeit und Sicherheit:\\
Redundanz!

\subsubsection{Erweiterbarkeit}
\begin{itemize}
\item neue Funktionalität kommt hinzu $\to$ welcher Aufwand resultiert daraus?
\end{itemize}

\subsubsection{Anwendbarkeit}
\begin{itemize}
\item gleiche Funktionalität mit anderen Eigenschaften z.B. kürzere Antwortzeit
\end{itemize}

\subsubsection{Kompatibilität}
Kompatibel $\to$ anschließbar
\begin{itemize}
\item Das SW-System A muss zusammen mit dem SW-System B arbeiten.
\end{itemize}

\subsubsection{Portabilität}
Portabel $\to$ „fortschleppbar“
\begin{itemize}
\item Das SW-System A läuft in der Umgebung von 1 bzw. 2 (Plattform) [Linux bzw. Windows]
\end{itemize}


\subsection{Wie kann hohe Software-Qualität erreicht werden?}
\slidesScale{3}{7}
Analyse: man hat schon etwas und schaut nach (Test: analytische Fehlersuche).\\
Konstruktion: es wird erst gebaut (im Vornherein gut Programmieren).
\slidesScale{3}{9}
SW-Qualität -- erfordert Kosten und Zeit
\slidesScale{3}{10}
\paragraph{Lösung:}
Durch gezielte Qualitätssicherung im 
\begin{itemize}
\item organisatorischen Bereich\\
(Vorgehen bei der Software-Entwicklung $\to$ Vorgehensmodelle)
\item im Rahmen der Anforderungsmodellierung
\item im konstruktiven Bereich\\
(Analyse, Entwurf, Implementierung $\to$ Muster, Schnittstellen, Werkzeuge…)
\item im analytischen Bereich (Test)
\end{itemize}





%\newpage
%\printbibliography

\end{document}