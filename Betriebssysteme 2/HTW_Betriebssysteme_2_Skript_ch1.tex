\lecdate{22.03.2017}
\section{Beispiel Bankautomat}
\slides{00-synchronisation}{2}
\subsection*{Möglicher (typischer) Ablauf}
\slides{00-synchronisation}{3}
\section{Race Condition}
\slides{00-synchronisation}{4}
\section{Kritischer Abschnitt}
\slides{00-synchronisation}{5}
$\to$ Lost-Update-Problem!
\begin{framed}
Zugriffsoperationen zur gemeinsam genutzten Variable
bilden einen so genannten \emph{kritischen Abschnitt} (critical
section).
\end{framed}
Zur Erinnerung: Scheduling (kooperativ $\to$ syscalls [$\Rightarrow$ Semaphor], …)
\section{Steuerung des kritischen Abschnitts}
\slides{00-synchronisation}{6}
\subsection*{Steuerung durch klammernde Funktionen}
\slides{00-synchronisation}{7}
$\to$ dezentrale Steuerung\\
Was müssen die Funktionen \lstinline$enter_cs()$ und \lstinline$leave_cs()$ tun?
\slides{00-synchronisation}{8}
\section{Realisierungsvarianten}
\slides{00-synchronisation}{9}
\section{Dezentrale Lösungsversuche}
\subsection{Naiver Versuch 1: „Ping-Pong“}
\slides{00-synchronisation}{11}
$\Rightarrow$ funktioniert daher nicht vernünftig.
\subsection{Versuch 2: Variable zeigt freien kritischen Abschnitt an}
\slides{00-synchronisation}{12}
\subsubsection*{Bewertung}
\slides{00-synchronisation}{13}
$\Rightarrow$ funktioniert gar nicht, da kritischer Abschnitt von beiden betreten werden kann!
\subsection{Versuch 3: Mehrere Variablen zeigen freien kritischen Abschnitt an}
\slides{00-synchronisation}{14}
\subsubsection*{Bewertung}
\slides{00-synchronisation}{15}
$\Rightarrow$ funktioniert auch nicht, aus selben Grund wie Versuch 2!
\subsection{Versuch 4: Mehrere Variablen zeigen Wunsch zum Eintreten in kritischen Abschnitt an}
\slides{00-synchronisation}{16}
\subsubsection*{Bewertung}
\slides{00-synchronisation}{17}
$\Rightarrow$ funktioniert nicht vernünftig, da Deadlock-Risiko.
\subsection{Versuch 5: Mehrere Variablen mit Eintrittswunsch und weiterer Prüfung}
\slides{00-synchronisation}{18}
\subsubsection*{Bewertung}
\slides{00-synchronisation}{19}
$\Rightarrow$ durch zufällige Verzögerung ist diese Lösung schon relativ gut. Statistisch kann trotzdem (bei Mehrprozessorsystemen) noch ein Livelock eintreten.
\subsection{Algorithmus von Peterson}
Vergleich auch: Algorithmus von Dekker und Dijkstra.
\slides{00-synchronisation}{20}
\subsubsection*{Bewertung}
\slides{00-synchronisation}{21}
Anmerkung: Bei Implementation kann es passieren, dass es nicht funktioniert $\to$ Der Algorithmus geht davon aus, dass Wertänderungen (bspw. von \lstinline$turn$) sofort von dem anderen Prozess gesehen wird. Tatsächlich passiert das (bei typischen Multiprozessorsystemen) erst zeitverzögert. Dafür müssten Schreibbarrieren eingeführt werden.
\subsection{Fazit}
\lecdate{29.03.2017}
\slides{00-synchronisation}{22}
„busy waiting“: Prozess verbraucht noch immer Ressourcen (durch \lstinline$while$-Schleife), obwohl er nur warten soll.

\section{Zentrale Lösung: Semaphore}
\subsubsection*{Motivation}
\slides{00-synchronisation}{23}

\subsection{Zustände eines Semaphors}
\slides{00-synchronisation}{24}

\subsection{Operationen eines Semaphors}
\slides{00-synchronisation}{25}
\slides{00-synchronisation}{26}

\subsubsection*{Implementierung der P()-Funktion}
\slides{00-synchronisation}{27}

\subsection{API für Semaphore}
\subsubsection*{Unter Unix}
\slides{00-synchronisation}{28}
\subsubsection*{Weitere}
\slides{00-synchronisation}{29}

\subsection{Anwendung: Zeitliche Synchronisation von Prozessen}
\slides{00-synchronisation}{31}
\subsubsection*{Rendezvous}
\slides{00-synchronisation}{34}
Das Vertauschen würde zu einem Deadlock führen.
\subsection{Weitere Aspekte}
\slides{00-synchronisation}{35}
Bekannte anschauliche Probleme:
\begin{itemize}
\item Leser-Schreiber-Problem (gleichzeitiges Lesen, nur einer darf immer schreiben)
\item Erzeuger-Verbraucher-Problem (beschränkter Puffer)
\item Problem des schlafenden Friseurs (verschieden exklusive Ressourcen)-
\item Problem der dinierenden Philosophen (Verteilung von begrenzten Ressourcen ohne Verhungern/Deadlock)
\end{itemize}

\subsubsection{Problem der dinierenden Philosophen}
\slides{00-synchronisation}{36}
\subsubsection*{Naive Lösung}
\begin{lstlisting}[language=C]
/* Code für den n-ten Philosophen, naive Loesung */
void philosopher(int n){
  while(1) {
    /* denken */
    take_fork(n);             /* rechte Gabel nehmen */
    take_fork((n+1)%5);       /* linke Gabel nehmen */
    /* essen */
    put_fork((n+1)%5);        /* linke Gabel weglegen */
    put_fork(n);              /* rechte Gabel weglegen */
  }
}

\end{lstlisting}
Problem: jeder greift nach seiner rechten Gabel $\to$ keiner kann mehr nach der linken greifen (Deadlock)!
\subsubsection*{Vorsichtiges Greifen nach 2. Gabel}
\begin{lstlisting}[language=C]
/* Code für den n-ten Philosophen, 'vorsichtiges' Greifen nach 2. Gabel */
void philosopher(int n){
  while(1) {
    /* denken */
  again:
    take_fork(n);             /* rechte Gabel nehmen */
    if (!available(fork((n+1)%5)) { /* falls linke Gabel nicht verfuegbar ... */
      put_fork(n);            /* ... rechte Gabel zuruecklegen ... */ 
      sleep(10);              /* ... eine Weile warten ... */ 
      goto again;             /* ... und von vorn probieren. */
    }
    take_fork((n+1)%5);       /* linke Gabel nehmen */
    /* essen */
    put_fork((n+1)%5);        /* linke Gabel weglegen */
    put_fork(n);              /* rechte Gabel weglegen */
  }
}
\end{lstlisting}
Problem: Kann zu einem Livelock führen (besser: sleep-Zeit zufällig auswählen).

\subsubsection*{1 Semaphor}
\begin{lstlisting}[language=C]
/* Code für den n-ten Philosophen, 1 Semaphor */
semaphore eat = 1;        /* Init: offen */
void philosopher(int n){
  while(1) {
    /* denken */
    P(eat);                   /* Erlaubnis zum Aufnehmen der Gabeln einholen */
    take_fork(n);             /* rechte Gabel nehmen */
    take_fork((n+1)%5);       /* linke Gabel nehmen */
    /* essen */
    put_fork((n+1)%5);        /* linke Gabel weglegen */
    put_fork(n);              /* rechte Gabel weglegen */
    V(eat);                   /* Erlaubnis zurückgeben */
  }
}

\end{lstlisting}
Problem: Maximale Anzahl an essenden Philosophen wird auf 1 reduziert!

\subsubsection*{Lösung nach Tanenbaum}
\begin{lstlisting}[language=C]
/* Loesung des Philosophenproblems nach Tanenbaum */
#define N 5
#define RIGHT(i) (((i)+1) % N)
#define LEFT(i) (((i)==N) ? 0: (i)+1)

enum phil_state {THINKING, HUNGRY, EATING};
enum phil_state state[N];     /* geeignet initialisiert */

semaphore mutex = 1;  
semaphore s[N];               /* müssen geschlossen initialisiert werden */

void test(int i){
  if ((state[i] == HUNGRY) &&
      (state[LEFT(i)] != EATING) &&
      (state[RIGHT(i)] != EATING)) {
    state[i] = EATING;
    V(s[i]);
  }
}

void get_forks(int i){
  P(mutex);
  state[i] = HUNGRY;
  test(i);
  V(mutex);
  P(s[i]);                /* hungrig blockieren oder verlassen, d.h., essen */
}

void put_forks(int i){
  P(mutex);
  state[i] = THINKING;
  test(LEFT(i));          /* linken Nachbarn ggf. wecken */
  test(RIGHT(i));         /* rechten Nachbarn ggf. wecken */
}

void philosopher(int n){
  while(1) {
    /* denken */
    get_forks(n);
    /* essen */
    put_forks(n); 
  }
}
\end{lstlisting}
Achtung: Programme mit vielen Semaphoren sind unübersichtlich!

\subsubsection*{Einfache pragmatische Lösung}
\begin{lstlisting}[language=C]
/* alternative Loesung des Philosophenproblems */
#define N 5
#define RIGHT(i) (((i)+1) % N)
#define LEFT(i) (((i)==N) ? 0: (i)+1)

semaphore free = N-1;                    /* nicht-binaerer Semaphor */  
semaphore fork[N] = { 1, 1, 1, 1, 1};    /* offen initialisiert */

void philosopher(int n){
  while(1) {
    /* denken */
    P(free);
    P(fork[RIGHT(n)]);
    P(fork[LEFT(n)]);
    /* essen */
    V(fork[LEFT(n)]);
    V(fork[RIGHT(n)]);
    V(free);
  }
}
\end{lstlisting}

\section{Fazit}
\slides{00-synchronisation}{38}

\section{Zusammenfassung Semaphore}
\lecdate{05.04.2017}
\slides{02-sync}{2}
\slides{02-sync}{3}
\section{Implementation der P()-Operation}
\slides{02-sync}{4}
\slides{02-sync}{5}
\section{Spinlocks (Schlossvariablen)}
\slides{02-sync}{6}
Spinlock $\to$ busy waiting
\subsection{Implementierung Ansatz 1}
Möglicher Syntax bei x86 (Unterschiede):
\begin{itemize}
\item Intel-Syntax:\\
\lstinline$instruction target, source$\\
\lstinline$mov$ (allgemein)
\item AT\&T Syntax\\
\lstinline$instruction source, target$\\
\lstinline$movb, movw, movl, movq$ (je nach Länge des Operants: Byte, Word, …)
\end{itemize}
Register: A, B, C, D
\begin{center}
\begin{tikzpicture}[scale=.7]
\draw  (0,0) rectangle (16,1) node[above]{0};
\node at (0,1) [above] {31};
\draw  (16,0) rectangle (12,1) node[above right]{7} node[above left]{8} node[pos=.5]{AL};
\draw  (12,0) rectangle (8,1) node[above]{15} node[pos=.5]{AH};
\node at (12,0.5) {AX};
\draw  (0,-0.5) rectangle (16,-1.5);
\node at (17.5,0.5) {EAX};
\node at (17.5,-1) {EBX};
\node at (17.5,-2.5) {ECX};
\node at (17.5,-4) {EDX};
\end{tikzpicture}
\end{center}
\begin{itemize}
\item ESI: Source Index
\item EDI: Destination Index
\item ESP: Stack Pointer (unteres Ende)
\item EBP: Base Pointer (oberes Ende)
\end{itemize}
\slides{02-sync}{7}
Ablauf:
\begin{itemize}
\item globale Funktionen bestimmen
\item definiere locked=0 und unlocked=1
\item data-Sektion: setze lock=unlocked(1)
\item text(Instruktions)-Sektion:
\begin{itemize}
\item kopiere lock in EAX
\item vergleiche, ob EAX gleich mit locked(0)
\item wenn ja, dann gehe erneut zu enter\_cs
\item sonst kopiere locked(0) in lock
\end{itemize}
\end{itemize}
\slides{02-sync}{12}
\subsection{Atomare Lese-Schreib-Instruktionen der IA32 (Intel 32bit Architektur)}
\slides{02-sync}{13}

\subsection{Implementierung Ansatz 2}
\slides{02-sync}{14}
Atomarer Exchange-Befehl wird genutzt, um sicher zu stellen, dass nur 1 Prozess in kritischen Abschnitt kann:\\
\begin{tabular}{c c | c}
EAX: & 0 & var\\
Lock: & var & 0
\end{tabular}\\
Also zu: \begin{tabular}{c c | c}
EAX: & 0 & 0\\
Lock: & 0 & 0
\end{tabular} 
und offen: \begin{tabular}{c c | c}
EAX: & 0 & 1\\
Lock: & 1 & 0
\end{tabular}
\slides{02-sync}{15}

\subsection{Warum nicht einfach Interrupts verbieten?}
\slides{02-sync}{16}

\section{API für Semaphore}
\subsubsection*{Unter Unix}
\slides{02-sync}{17}
\subsubsection*{Weitere}
\slides{02-sync}{18}
\subsection{Semaphor-API nach System V}
\slides{02-sync}{19}
\begin{itemize}
\item \lstinline$ipcs$: Anzeige der Semaphore (und Shared Memory Segments und Message Queues)
\item \lstinline$ipcrm$: löscht IPC-Ressource (bspw. Semaphor)
\end{itemize}

\subsubsection*{Erzeuger-Verbraucher-Problem}
Veranschaulichender Quellcode: \lstinline$erzver-sysv.c$

\subsection*{Weiter Aspekte}
\slides{02-sync}{22}

\section{Identifikation der Semaphore}
\slides{02-sync}{20}
Funktion \lstinline$ftok$ kann einen eindeutigen Key für den Semaphor generieren.

\section{Semaphorausgleichswert}
\slides{02-sync}{21}
Semaphorausgleichswert ist dafür da, dass beim Beenden eines Prozesses die Manipulation des Semaphores zurück gesetzt wird (wenn Prozess nach dem schließen des Semaphors beendet ohne ihn zu öffnen, schafft der Semaphorausgleichswert dafür, dass der kritische Abschnitt noch von anderen Programmen erreicht werden kann).\\
Der Semaphorausgleichswert zählt im Prinzip die Anzahl von \lstinline$P()$s und \lstinline$V()$s.


\section{Das Leser-Schreiber-Problem}
\lecdate{12.04.2017/1}
\slides{02-sync}{23}

\subsection{Triviallösung}
\slides{02-sync}{24}

\subsection{Standardlösung}
%\lstinputlisting[language=Pascal]{Vorlesung/Vorlesungsbeispiele/readwrite2.pfc}
\slides{02-sync}{26}
Im Pseudocode:
\begin{lstlisting}[language=C]
// Schreiben: 
P(wri)
/* schreibt */
V(wri)

// Lesen:
P(mutex)	// verhindert, dass mehre gleichzeitig Abfrage des P(wri) machen.
rc++	// zählt Leseprozesse
if (rc==1)
	P(wri)
V(mutex)
/* liest */
P(mutex)	// wie P(mutex) oben
rc--
if (rc==0)
	V(wri)
V(mutex)
\end{lstlisting}
\begin{itemize}
\item[$\checkmark$] exklusives Schreiben
\item[$\checkmark$] parallel Lesen möglich
\item[$\checkmark$] kein Lesen, wenn geschrieben wird
\item[$\times$] Lösung bevorzugt (ggf.) Leseprozesse
\end{itemize}

\subsection{Analoge bevorzugt Schreiber}
\slides{02-sync}{27}

\subsection{Umsetzung mit PEARL und Bolt-Variablen}
Bolt-Variablen in RTOS-UH und der EZ-Programmiersprache PEARL
\begin{itemize}
\item Boltvariable
\item Leser-Schreibe-\emph{Semaphore}
\begin{lstlisting}
„P“(n,Lesen)
„V“(n,Lesen)
„P“(n,Schreiben)
„V“(n,Schreiben)
\end{lstlisting}
\end{itemize}
\slides{02-sync}{28}
\subsubsection{Zustandsdiagramm einer Boltvariable}
\slides{02-sync}{29}
\subsubsection{Leser-Schreiber-Problem in PEARL}
\slides{02-sync}{30}

\subsection{Datenstrukturen im Linux-Kern}
\slides{02-sync}{31}
\slides{02-sync}{32}

\subsection{Leser-Schreiber-Spinlocks}
\slides{02-sync}{33}
Erläuterung:
\begin{lstlisting}
1:	rwlp++
		jump 3(forward), if sign bit not set
		rwlp--	(weil Schreiber anwesend)
2:	wenn rwlp noch nicht 0, dann
		jump 2(backward). Sonst:
		jumb 1(backward)	(vor vorn probieren)
3:
\end{lstlisting}
\lecdate{12.04.2017/2}
\slides{02-sync}{34}
Erläuterung:
\begin{lstlisting}
1:	btsl (bit test and set: aktueller Wert des referenzierten Bits geht ins Carry Flag(C), danach Bit auf 1 setzen) => Bit 31 von rwlp ins Carry-Flag C
		wenn C==1, Sprung nach 2 (wenn also schon Schreiber da war)
		testl: Operanden mit logischem UND verknüpfen (und prüfen, ob null ist)
		je 3(forward) wenn testl 0 ergibt, also die Anzahl Leser 0 ist. Sonst:
		btrl (bit test and reset: wie btsl, nur dass Bit auf 0 gesetzt wird) => Bit 31 von rwlp auf 0 setzen
2:	wenn rwlp noch nicht 0, dann
		jump 2(backward). Sonst:
		jump 1(backward)	(von vorn probieren)
3:
\end{lstlisting}

\subsection{Zwischenfazit}
\slides{02-sync}{35}

\section{Monitore}
\subsection{Idee}
\slides{02-sync}{36}
\subsection{Bedingungsvariablen}
\slides{02-sync}{37}
Bedingungsvariable = Prozesswarteschlange\\
delay $\to$ Prozess wird in (betreffende) Warteschlange eingeordnet\\
resume $\to$ (anderer) Prozess wird aus betreffender Warteschlange entfernt und fortgesetzt
\slides{02-sync}{38}
\subsubsection{Semantik von resume()}
\slides{02-sync}{39}
\slides{02-sync}{40}
\subsubsection{Bedingungsvariablen in Pthreads}
\slides{02-sync}{41}
\subsection{Struktur}
\slides{02-sync}{43}
\slides{02-sync}{42}
\subsection{Anwendungen}
\slides{02-sync}{44}
zu 1.:
%\lstinputlisting[language=Pascal]{Vorlesung/Vorlesungsbeispiele/monitormutex.pfc}
Von \lstinline$monitor mutex;$ bis \lstinline$end; (* monitor mutex *)$ sind die proceduren Monitor-Pro\-ze\-du\-ren. Daher kann bspw. die Prozedur \lstinline$enter$ nur von einem Prozess auf einmal ausgeführt werden.
\unimptnt{\slides{02-sync}{45}}

\subsection{Bewertung}
\slides{02-sync}{46}

\section{Prüfungsfragen}
\begin{itemize}
\item Semaphore Erklären (P/V)
\item P/V programmieren können
\item Abläufe erkennen (wer kommt zuerst dran)
\item Monitor als Theorie (nicht programmiertechnisch)
\end{itemize}