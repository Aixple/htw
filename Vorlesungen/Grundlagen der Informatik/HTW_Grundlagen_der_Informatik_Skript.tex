% Header aus der Vorlage
\documentclass[a4paper,11pt, footheight=26pt
%,twoside
]{scrreprt}
\usepackage[head=23pt]{geometry}	% head=23pt umgeht Fehlerwarnung, dafür größeres "top" in geometry
\geometry{a4paper, top=30mm, bottom=22mm,headsep=10mm, footskip=12mm
, left=20mm, right=20mm
%, inner=27mm, outer=13mm
}

% Zeile 2 (,twoside) und 7 (inner=...) für eine Druckversion (doppelseitig) ent-kommentieren (Rand für Hefter)

\setcounter{secnumdepth}{3}	% zählt auch subsubsection
\setcounter{tocdepth}{3}	% Inhaltsverzeichnis bis in subsubsection

% Input inkl. Umlaute, Silbentrennung
\usepackage[T1]{fontenc}
\usepackage[utf8]{inputenc}
\usepackage[ngerman]{babel}
\usepackage{csquotes}	% Anführungszeichen
\usepackage{eurosym}

% HTW Corporate Design: Arial (Helvetica)
\usepackage{helvet}
\renewcommand{\familydefault}{\sfdefault}

% Style-Aufhübschung
\usepackage{soul, color}	% Kapitälchen, Unterstrichen, Durchgestrichen usw. im Text
\usepackage{scrlayer-scrpage}	% Kopf-/Fußzeile
%\usepackage{titleref}
\usepackage[perpage]{footmisc}	% Fußnotenzählung Seitenweit, nicht Dokumentenweit
\renewcommand*{\thefootnote}{\fnsymbol{footnote}}	% Fußnoten-Symbole anstatt Zahlen
\renewcommand*{\titlepagestyle}{empty} % Keine Seitennummer auf Titelseite

% Mathe usw.
\usepackage{amssymb}
\usepackage[fleqn]{amsmath}	% fleqn: align-Umgebung rechtsbündig
\usepackage{xcolor}
\usepackage{esint}	% Schönere Integrale, \oiint vorhanden
\everymath=\expandafter{\the\everymath\displaystyle}	% Mathe Inhalte werden weniger verkleinert
\usepackage{wasysym}	% mehr Symbole, bspw \lightning
% Auch arcus-Hyperbolicus-Funktionen
\DeclareMathOperator{\arccot}{arccot}
\DeclareMathOperator{\arccosh}{arccosh}
\DeclareMathOperator{\arcsinh}{arcsinh}
\DeclareMathOperator{\arctanh}{arctanh}
\DeclareMathOperator{\arccoth}{arccoth} 
% Mathe in Anführungszeichen:
\newsavebox{\mathbox}\newsavebox{\mathquote}
\makeatletter
\newcommand{\mq}[1]{% \mathquotes{<stuff>}
  \savebox{\mathquote}{\text{"}}% Save quotes
  \savebox{\mathbox}{$\displaystyle #1$}% Save <stuff>
  \raisebox{\dimexpr\ht\mathbox-\ht\mathquote\relax}{"}#1\raisebox{\dimexpr\ht\mathbox-\ht\mathquote\relax}{''}
}
\makeatother

% tikz usw.
\usepackage{tikz}
\usepackage{pgfplots}
\pgfplotsset{compat=1.11}	% Umgeht Fehlermeldung
\usetikzlibrary{graphs}
%\usetikzlibrary{through}	% ???
\usetikzlibrary{arrows}
\usetikzlibrary{arrows.meta}	% Pfeile verändern / vergrößern: \draw[-{>[scale=1.5]}] (-3,5) -> (-3,3);
\usetikzlibrary{automata,positioning} % Zeilenumbruch im Node node[align=center] {Text\\nächste Zeile} automata für Graphen
\usetikzlibrary{matrix}
\usetikzlibrary{patterns}	% Schraffierte Füllung
\tikzstyle{reverseclip}=[insert path={	% Inverser Clip \clip
	(current page.north east) --
	(current page.south east) --
	(current page.south west) --
	(current page.north west) --
	(current page.north east)}
% Nutzen: 
%\begin{tikzpicture}[remember picture]
%\begin{scope}
%\begin{pgfinterruptboundingbox}
%\draw [clip] DIE FLÄCHE, IN DER OBJEKT NICHT ERSCHEINEN SOLL [reverseclip];
%\end{pgfinterruptboundingbox}
%\draw DAS OBJEKT;
%\end{scope}
%\end{tikzpicture}
]	% Achtung: dafür muss doppelt kompliert werden!
\usepackage{graphpap}	% Grid für Graphen
\tikzset{every state/.style={inner sep=2pt, minimum size=2em}}

% Tabular
\usepackage{longtable}	% Große Tabellen über mehrere Seiten
\usepackage{multirow}	% Multirow/-column: \multirow{2[Anzahl der Zeilen]}{*[Format]}{Test[Inhalt]} oder \multicolumn{7[Anzahl der Reihen]}{|c|[Format]}{Test2[Inhalt]}
\renewcommand{\arraystretch}{1.3} % Tabellenlinien nicht zu dicht
\usepackage{colortbl}
\arrayrulecolor{gray}	% heller Tabellenlinien
\usepackage{array}	% für folgende 3 Zeilen (für Spalten fester breite mit entsprechender Ausrichtung):
\newcolumntype{L}[1]{>{\raggedright\let\newline\\\arraybackslash\hspace{0pt}}m{\dimexpr#1\columnwidth-2\tabcolsep-1.5\arrayrulewidth}}
\newcolumntype{C}[1]{>{\centering\let\newline\\\arraybackslash\hspace{0pt}}m{\dimexpr#1\columnwidth-2\tabcolsep-1.5\arrayrulewidth}}
\newcolumntype{R}[1]{>{\raggedleft\let\newline\\\arraybackslash\hspace{0pt}}m{\dimexpr#1\columnwidth-2\tabcolsep-1.5\arrayrulewidth}}

% Nützliches
\usepackage{verbatim}	% u.a. zum auskommentieren via \begin{comment} \end{comment}
\usepackage{tabto}	% Tabs: /tab zum nächsten Tab oder /tabto{.5 \CurrentLineWidth} zur Stelle in der Linie
\NumTabs{6}	% Anzahl von Tabs pro Zeile zum springen
\usepackage{listings} % Source-Code mit Tabs
\usepackage{lstautogobble} 
\usepackage{enumitem}	% Anpassung der enumerates
\setlist[enumerate,1]{label=\arabic*.)}	% global andere Enum-Items
\newenvironment{anumerate}{\begin{enumerate}[label=\alph*.)]}{\end{enumerate}} % Alphabetische Aufzählung
\renewcommand{\labelitemiii}{$\scriptscriptstyle ^\blacklozenge$} % global andere 3. Item-Aufzählungszeichen
\usepackage{letltxmacro} % neue Definiton von Grundbefehlen
% Nutzen:
%\LetLtxMacro{\oldemph}{\emph}
%\renewcommand{\emph}[1]{\oldemph{#1}}

% Einrichtung von lst
\lstset{
basicstyle=\ttfamily, 
mathescape=true, 
%escapeinside=^^, 
autogobble, 
tabsize=2,
basicstyle=\footnotesize\sffamily\color{black},
frame=single,
rulecolor=\color{lightgray},
numbers=left,
numbersep=5pt,
numberstyle=\tiny\color{gray},
commentstyle=\color{gray},
keywordstyle=\color{green},
stringstyle=\color{orange},
morecomment=[l][\color{magenta}]{\#}
%showspaces=false,
showstringspaces=false,
breaklines=true,
literate=%
    {Ö}{{\"O}}1
    {Ä}{{\"A}}1
    {Ü}{{\"U}}1
    {ß}{{\ss}}1
    {ü}{{\"u}}1
    {ä}{{\"a}}1
    {ö}{{\"o}}1
    {~}{{\textasciitilde}}1
}
\usepackage{scrhack} % Fehler umgehen
\def\ContinueLineNumber{\lstset{firstnumber=last}} % vor lstlisting. Zum wechsel zum nicht-kontinuierlichen muss wieder \StartLineAt1 eingegeben werden
\def\StartLineAt#1{\lstset{firstnumber=#1}} % vor lstlisting \StartLineAt30 eingeben, um bei Zeile 30 zu starten
\let\numberLineAt\StartLineAt

% BibTeX
\usepackage[backend=bibtex, bibencoding=ascii]{biblatex}	% BibTeX
\usepackage{makeidx}
%\makeglossary
%\makeindex

% Grafiken
\usepackage{graphicx}
\usepackage{epstopdf}	% eps-Vektorgrafiken einfügen

% pdf-Setup
\usepackage{pdfpages}
\usepackage[bookmarks,%
bookmarksopen=false,% Klappt die Bookmarks in Acrobat aus
colorlinks=true,%
linkcolor=black,%
citecolor=red,%
urlcolor=green,%
]{hyperref}

% Titel, Autor usw. werden vor dem Anfang des Dokuments in einem Rutsch definiert…
\newcommand{\DTitel}[1]{\newcommand{\Dokumententitel}{#1}}
\newcommand{\DUntertitel}[1]{\newcommand{\Dokumentenuntertitel}{#1}}
\newcommand{\DAutor}[1]{\newcommand{\Dokumentenautor}{#1}}
\newcommand{\DNotiz}[1]{\newcommand{\Dokumentennotiz}{#1}}
\newcommand{\DSign}[1]{\newcommand{\Dokumentensignatur}{#1}}
\DSign{\footnotesize{\textcolor{darkgray}{Mitschrift von\\ \Dokumentenautor}}}
\newcommand{\Autorformat}[1]{\textcolor{darkgray}{Mitschrift von #1}}
\newcommand{\workingdir}{../}	% Arbeitsordner (in Abhängigkeit vom Master) Standard: LateX_master Ordner liegt im Eltern-Ordner
% … Deswegen folgendes erst Nach Dokumentenbeginn ausführen:
\AtBeginDocument{
	\hypersetup{
		pdfauthor={\Dokumentenautor},
		pdftitle={HTW Dresden | \Dokumententitel - \Dokumentenuntertitel},
	}
	\automark[section]{section}
	\automark*[subsection]{subsection}
	\pagestyle{scrheadings}
	\ihead{\includegraphics[height=1.7em]{\workingdir LaTeX_master/HTW-Logo.eps}}
	\ohead{\Dokumententitel}
	\cfoot{\pagemark}
	\ofoot{\Dokumentensignatur}
	% Titelseite
	\title{\includegraphics[width=0.35\textwidth]{\workingdir LaTeX_master/HTW-Logo.eps}\\\vspace{0.5em}
	\Huge\textbf{\Dokumententitel} \\\vspace*{0,5cm}
	\Large \Dokumentenuntertitel \\\vspace*{4cm}}
	\author{\Autorformat{\Dokumentenautor} \vspace*{1cm}\\\Dokumentennotiz}
}

%% EINFACHE BEFEHLE

% Abkürzungen Mathe
\newcommand{\EE}{\mathbb{E}}
\newcommand{\QQ}{\mathbb{Q}}
\newcommand{\RR}{\mathbb{R}}
\newcommand{\CC}{\mathbb{C}}
\newcommand{\NN}{\mathbb{N}}
\newcommand{\ZZ}{\mathbb{Z}}
\newcommand{\PP}{\mathbb{P}}
\renewcommand{\SS}{\mathbb{S}}
\newcommand{\cA}{\mathcal{A}}
\newcommand{\cB}{\mathcal{B}}
\newcommand{\cC}{\mathcal{C}}
\newcommand{\cD}{\mathcal{D}}
\newcommand{\cE}{\mathcal{E}}
\newcommand{\cF}{\mathcal{F}}
\newcommand{\cG}{\mathcal{G}}
\newcommand{\cH}{\mathcal{H}}
\newcommand{\cI}{\mathcal{I}}
\newcommand{\cJ}{\mathcal{J}}
\newcommand{\cM}{\mathcal{M}}
\newcommand{\cN}{\mathcal{N}}
\newcommand{\cP}{\mathcal{P}}
\newcommand{\cR}{\mathcal{R}}
\newcommand{\cS}{\mathcal{S}}
\newcommand{\cZ}{\mathcal{Z}}
\newcommand{\cL}{\mathcal{L}}
\newcommand{\cT}{\mathcal{T}}
\newcommand{\cU}{\mathcal{U}}
\newcommand{\cV}{\mathcal{V}}
\renewcommand{\phi}{\varphi}
\renewcommand{\epsilon}{\varepsilon}

% Farbdefinitionen
\definecolor{red}{RGB}{180,0,0}
\definecolor{green}{RGB}{75,160,0}
\definecolor{blue}{RGB}{0,75,200}
\definecolor{orange}{RGB}{255,128,0}
\definecolor{yellow}{RGB}{255,245,0}
\definecolor{purple}{RGB}{75,0,160}
\definecolor{cyan}{RGB}{0,160,160}
\definecolor{brown}{RGB}{120,60,10}

\definecolor{itteny}{RGB}{244,229,0}
\definecolor{ittenyo}{RGB}{253,198,11}
\definecolor{itteno}{RGB}{241,142,28}
\definecolor{ittenor}{RGB}{234,98,31}
\definecolor{ittenr}{RGB}{227,35,34}
\definecolor{ittenrp}{RGB}{196,3,125}
\definecolor{ittenp}{RGB}{109,57,139}
\definecolor{ittenpb}{RGB}{68,78,153}
\definecolor{ittenb}{RGB}{42,113,176}
\definecolor{ittenbg}{RGB}{6,150,187}
\definecolor{itteng}{RGB}{0,142,91}
\definecolor{ittengy}{RGB}{140,187,38}

% Textfarbe ändern
\newcommand{\tred}[1]{\textcolor{red}{#1}}
\newcommand{\tgreen}[1]{\textcolor{green}{#1}}
\newcommand{\tblue}[1]{\textcolor{blue}{#1}}
\newcommand{\torange}[1]{\textcolor{orange}{#1}}
\newcommand{\tyellow}[1]{\textcolor{yellow}{#1}}
\newcommand{\tpurple}[1]{\textcolor{purple}{#1}}
\newcommand{\tcyan}[1]{\textcolor{cyan}{#1}}
\newcommand{\tbrown}[1]{\textcolor{brown}{#1}}

% Umstellen der Tabellen Definition
\newcommand{\mpb}[1][.3]{\begin{minipage}{#1\textwidth}\vspace*{3pt}}
\newcommand{\mpe}{\vspace*{3pt}\end{minipage}}

\newcommand{\resultul}[1]{\underline{\underline{#1}}}
\newcommand{\parskp}{$ $\\}	% new line after paragraph
\newcommand{\corr}{\;\widehat{=}\;}
\newcommand{\mdeg}{^{\circ}}

\newcommand{\nok}[2]{\begin{pmatrix}#1\\#2\end{pmatrix}}	% n über k BESSER: \binom{n}{k}
\newcommand{\mtr}[1]{\begin{pmatrix}#1\end{pmatrix}}	% Matrix
\newcommand{\dtr}[1]{\begin{vmatrix}#1\end{vmatrix}}	% Determinante (Betragsmatrix)
\renewcommand{\vec}[1]{\underline{#1}}	% Vektorschreibweise
\newcommand{\imptnt}[1]{\colorbox{red!30}{#1}}	% Wichtiges
\newcommand{\intd}[1]{\,\mathrm{d}#1}

\bibliography{../Literatur/HTW_Literatur}

% Definition von Titel, Autor usw.
\DTitel{Grundlagen der Informatik}
\DUntertitel{Vorlesungsskript}
\DAutor{Falk Jonatan Strube}
\DNotiz{Vorlesung von Dr. Boris Hollas}

\begin{document}

\pagenumbering{Roman}

\maketitle
\newpage
\tableofcontents
\newpage

\pagenumbering{arabic}

\section*{Allgemeine Informationen}

Zugelassene Hilfsmittel Klausur: A-4 Blatt (doppelseitig, handbeschrieben)

Prüfungsvorleistung: alle paar Woche eine Lernabfrage in der Vorlesung (Bestanden wenn insgesamt im Schnitt 50\%)

Grundlage der Vorlesung: Grundkurs theoretische Informatik \cite{hollas2007grundkurs}

Lernkontrolle ab 23.10.2015 alle zwei Wochen.

\section{Aussagenlogik}

Mit der Aussagenlogik lassen sich Aussagen formulieren, die entweder wahr oder falsch sind. Aussagen sind atomare Aussagen wie „die Straße ist nass“ oder mit Hilfe von logischen Operatoren zusammengesetzte Aussagen.

\subsection{Syntax und Semantik}

\paragraph{Definition:} Die \emph{Formeln der Aussagenlogik} sind induktiv definiert.

\begin{itemize}
\item Jede atomare Aussage ist eine Formel der Aussagenlogik. Diese heißen Atomformeln oder Variablen. 
Atomformeln bezeichnen wir mit Kleinbuchstaben oder durch Wörter in Kleinbuchstaben.
\item Wenn $F$, $G$ Formeln der Aussagenlogik sind, dann auch $(F \wedge G)$, $(F\vee G)$, $(\neg F)$.
\end{itemize}

\subparagraph{Bsp.:} Formeln der Aussagenlogik sind $x$, $y$, $x\wedge y$, $( x\wedge (y \wedge z)) \vee (\neg x \wedge (y \wedge \neg z))$, $regnet$, $regnet \wedge nass$, da sie jeweils aus atomaren Aussagen die nach der Definition zusammensetzen lassen bestehen.\\
Keine Formeln der Aussagenlogik sind $x\wedge$, $\vee x$, $x\wedge \vee y$.\\

Um Klammern zu sparen, legen wir Prioritäten fest:

\begin{tabular}{c | c}
Operator & Priorität \\
\hline
$\neg$ & höchste\\
$\wedge ,\; \vee$ & \\
$\rightarrow , \; \leftrightarrow$ & niedrigste\\
\end{tabular}

\paragraph{Definition:} Eine \emph{Belegung} einer Formel $F$ der Aussagenlogik ist eine Zuordnung von Wahrheitswerten „wahr“ (1) oder „falsch“ (0) zu den Atomarformeln in $F$.
Daraus ergibt sich der \emph{Wahrheitswert} einer Formel:
\begin{itemize}
\item Eine Atomformel ist genau dann wahr, wenn sie mit „wahr“ belegt ist.
\item Die Formel $F\wedge G $ ist genau dann wahr, wenn $F$ „wahr“ ist und $G$ „wahr“ ist. \\
$F\vee G$ ist wahr, wenn $F$ wahr ist oder $G$ wahr.\\
$\neg F$ ist wahr, wenn $F$ falsch ist.\\

\begin{tabular}{c c | c c c c c}
$F$&$G$&$F\wedge G$&$F\vee G$&$\neg F$&$F\rightarrow G$& $F\leftrightarrow G$\\
\hline
0&0&0&0&1&1&1\\
0&1&0&1&1&1&0\\
1&0&0&1&0&0&0\\
1&1&1&1&0&1&1\\
\end{tabular}
\end{itemize}

\subparagraph{Bsp.:} Wenn $regnet$ bedeutet: „Es regnet.“\\
Wenn $nass$ bedeutet: „Die Straße ist nass.“\\
Dann bedeutet $regnet \wedge nass$: „Es regnet und die Straße ist nass.“\\

„Wenn es regnet, dann ist die Straße nass“ ($regnet \rightarrow nass$). Es muss nur der Fall ausgeschlossen werden, der nicht eintreffen kann: $\neg (regnet \wedge \neg nass)$ $\Rightarrow$ Folgendes darf nicht eintreffen: „Es regnet und die Straße ist nicht nass“. 

Alles andere („Es regnet nicht und die Straße ist nicht nass“, „Es regnet nicht und die Straße ist nass“ und „Es regnet und die Straße ist nass“) darf eintreffen.\\
\begin{tabular}{c c | c}
$regnet$ & $nass$ & $\neg (regnet \wedge \neg nass) = \neg regnet \vee nass$\\
\hline
0&0&1\\
0&1&1\\
1&0&0\\
1&1&1\\
\end{tabular}

\paragraph{Definition:} Die Operatoren $\rightarrow$ (\emph{Implikation}) und $\leftrightarrow$ (\emph{Äquivalenz}) sind definiert durch:
\begin{itemize}
\item $F\rightarrow G = \neg F \vee G$
\item $F\leftrightarrow G = (F \rightarrow G) \wedge (G \rightarrow F)$
\end{itemize}
(Siehe Tabelle oberhalb)

\subparagraph{Bsp.:} Berechnen des Betrags $y$ einer Zahl $x$:
\begin{lstlisting}
if (x>= 0)
	y=x;
else 
	y=-x;
\end{lstlisting}
Dargestellt als Formel der Assagenlogik: $((x\geq 0) \rightarrow y=x)\wedge(\neg (x\geq 0) \rightarrow y=-x)$

\paragraph{Definition:} Eine Formel $F$ der Aussagenlogik heißt
\begin{itemize}

\item \emph{erfüllbar}, wenn es eine Belegung gibt, sodass $F$ wahr ist, sonst \emph{unerfüllbar}. Mit $\bot$ bezeichnen wir eine unerfüllbare Formel (Widerspruch).
\item \emph{Tautologie} oder \emph{gültig}, wenn $F$ für jede Belegung wahr ist. Bezeichnung: $\top$
\end{itemize}

\subparagraph{Bsp.:}
\begin{itemize}
\item $x\wedge y$ ist erfüllbar.
\item $((\neg x \wedge y)\vee (x\wedge \neg y)) \wedge \neg (x\vee y)$ ist unerfüllbar (linke Seite: entweder x oder y falsch - rechte Seite: x oder y falsch)
\item $x \vee \neg x$ ist eine Tautologie
\end{itemize}

\paragraph{Definition:} Wir schreiben $F \equiv G$ („$F$ ist äquivalent zu $G$“), wenn für jede Belegung gilt: $F \leftrightarrow G$ wahr (d.h., $F\leftrightarrow G$ ist gültig).

\subsection{Rechenregeln}
siehe Mathematik I

\section{Beweistechniken}

ABB21

\paragraph{Direkter Beweis}
\subparagraph{Bsp.:} Wenn $a \in \mathbb{Z}$ gerade ist, dann ist auch $a^2$ gerade.\\
$(a \in \mathbb{Z} \text{ gerade } \Rightarrow a^2 \text{ gerade})$

\emph{Beweis:} 
\begin{itemize}
\item Wenn $a$ gerade ist, gibt es ein $n$ mit $a=2\cdot n$.
\item Dann gilt $a^2=4\cdot n^2=2\cdot 2 n^2$, 
\item woraus $a^2$ gerade folgt.
\end{itemize}

\paragraph{Indirekter Beweis} Mit einem indirekten Beweis wird $A\Rightarrow B$ bewiesen, indem die äquivalente Aussage $\neg B \Rightarrow \neg A$ bewiesen wird.

\subparagraph{Bsp.:} Wenn $a^2$ gerade ist, dann auch $a$.\\
$(a^2 \text{ gerade } \Rightarrow a \text{ gerade})$

\emph{Beweis:} Wir zeigen: Wenn $a$ ungerade ist, dann auch $a^2$.
\begin{itemize}
\item Aus $a$ ungerade folgt $a=2n-1$ für ein $n$. 
\item Dann ist $a^2=4 n^2 - 4 n + 1 = \underbrace{4( n^2-n)}_{gerade} + \underbrace{1}_{ungerade}$, 
\item Aus gerade + ungerade folgt ungerade, woraus $a^2$ ungerade folgt.
\end{itemize}

\paragraph{Beweis durch Widerspruch} Mit einem Beweis durch Widerspruch wird eine Aussage $A$ bewiesen, indem gezeigt wird, dass die Annahme „$A$ ist falsch“ zu einem Widerspruch führt.\\
(D.h., es wird $\neg A \rightarrow \bot$ gezeigt)

\subparagraph{Bsp.:} $\sqrt{2}$ ist irrational. Siehe Mathematik I.

\paragraph{Vollständige Induktion} Mit einer vollständigen Induktion lassen sich Aussagen der Art „für alle $n \in \mathbb{N}$ gilt …“ beweisen.

Prinzip: Gegeben eine Aussage der Form „für alle $n \in \mathbb{N}$ gilt $A(n)$“

\begin{itemize}
\item \emph{Induktionsanfang}: Man zeigt, die Wahrheit der Aussage für $n=1$ (mit anderen Worten: Man zeigt, dass $A(1)$ wahr ist) [$1$: die kleinste mögliche Zahl $\Rightarrow$ kann auch 0 oder eine andere sein]
\item \emph{Induktionsvorussetzung}: Die Aussage ist für $n$ wahr.
\item \emph{Induktionsschritt}: Wenn IV wahr ist, dann ist die Aussage auch für $n+1$ wahr.
\end{itemize}
In Formeln: Man zeigt
\begin{itemize}
\item IA: $A(1)$
\item IV: $A(n)$
\item IS: für alle $n$: $A(n)\Rightarrow A(n+1)$
\end{itemize}
ABB22
\subparagraph{Bsp.:} Für alle $n\geq 1$ gilt $\sum_{k=1}^{n}k=\frac{n(n+1)}{2}$

\emph{Beweis (Induktion):}
\begin{itemize}
\item[IA] $n=1$: $1=\frac{1 \cdot 2}{2}$ ist wahr.
\item[IV] Es gelte $\sum_{k=1}^{n}k=\frac{n(n+1)}{2}$ ist wahr.
\item[IS] $n \rightarrow n+1$: Zu zeigen: $\sum_{k=1}^{n+1}k=\frac{(n+1)(n+2)}{2}$ Es gilt:
\begin{align*}
\sum_{k=1}^{n+1}k&=(\sum_{k=1}^{n}k)+n+1\\
&\overset{IV}{=}\frac{n(n+1)}{2}+n+1\\
&=...=\frac{(n+1)(n+2)}{2}
\#
\end{align*}
\end{itemize}

\section{Elementare Kombinatorik}

Kreuzprodukt:\\
$A\times B = \{(a,b)|a \in A, b \in B\}$\\
$A^n=\underbrace{A\times ... \times A}_{n}$\\
Die \emph{Potenzmenge} einer Menge $M$ ist die Menge aller Teilmengen von $M$:
$\mathcal{P}(M)=\{A|A\subseteq M\}$
\subparagraph{Bsp.:} $\mathcal{P}(\{1,2\})=\{\emptyset, \{1\}, \{2\}, \{1,2\}\}$

\paragraph{Definition:} Die Mächtigkeit einer Menge $A$ ist die Anzahl ihrer Elemente. Notation: $|A|$

\paragraph{Satz:} Es gilt $|A^n|=|A|^n$
\subparagraph{Beweis:} 
Nach Def. ist $A^n=\{(a_1,...,a_n)|a_1,...,a_n \in A\}$. Um das n-Tupel $(a_1,...,a_n)$ zu erzeugen, gibt es $|A|$ viele Möglichkeiten. Insgesamt gibt es daher $|A|^n$ Möglichkeiten das n-Tupel $(a_1,...,a_n)$ auszuwählen.

\subparagraph{Bsp.:} Eine PIN bestehe aus 6 Ziffern. Mit $A=\{0,...,9\}$ ist $A^6$ die Menge aller PINs. Mit obigen Satz folgt: Die Anzahl aller PINs ist $|A^6|=|A|^6 = 10^6$

\subparagraph{Bsp.:} In dem Programm
\begin{lstlisting}
for (i=1 to n)
	for (j=1 to n)
		a[i][j]=i+j;
\end{lstlisting}
werden alle Paare (i,j) erzeugt. Die Anzahl der Paare ist $|\{1,...,n\}^2|=|\{1,...,n\}|^2=n^2$. Es gibt daher $n^2$ Schleifendurchläufe.

\paragraph{Satz:} $|\mathcal{P}(M)|=2^{|M|}$
\subparagraph{Beweis:} Für $M=\{m_1, ...,m_n\}$ identifizieren wir eine Teilmenge $A\subseteq M$ durch das n-Tupel $(a_1, ..., a_n)$ mit $a_k\begin{cases}0\text{ für }M_k \not \in A\\ 1\text{ für }m_k \in A\end{cases}$. Nach obigen Satz gibt es $|\{0,1\}^n|=2^n=2^{|M|}$ derartige Tupel.

\paragraph{Definition:} Für eine n-elementige Menge ist $\begin{pmatrix}n\\k\end{pmatrix}$ die Anzahl ihrer k-elementigen Teilmengen $(n\geq k \geq 0)$.

\subparagraph{Bsp.:} $\begin{pmatrix}n\\0\end{pmatrix}=1$, da $\emptyset$ die einzige 0-elementige Teilmenge ist.\\
$\begin{pmatrix}n\\n\end{pmatrix}=1$, da es nur eine n-elementige Teilmenge gibt (die Menge selber).\\
$\begin{pmatrix}n\\1\end{pmatrix}=n$, da es $n$ 1-elementige Teilmengen gibt.\\
$\begin{pmatrix}n\\2\end{pmatrix}=\frac{n(n-1)}{2}$, denn für das 1. Element gibt es $n$ Möglichkeiten, für das 2. Element $n-1$ Möglichkeiten. Da das Element $\{a,b\}=\{b,a\}$ hierbei doppelt gezählt wird, müssen wir durch 2 teilen.

\paragraph{Definition:} Eine Permutation der Folge $1,...,n$ ist eine neue Anordnung dieser Folge.

\subparagraph{Bsp.:} Alle Permutationen von $1,2,3$ sind $1,2,3$; $1,3,2$; $2,1,3$; $2,3,1$; $3,1,2$; $3,2,1$.

\paragraph{Definition:} $n!=1\cdot ... \cdot n \qquad 0! =1$.
\paragraph{Satz:} Es gibt $n!$ Permutationen von $n$ Zahlen.

\subparagraph{Beweis:} Für die 1. Stelle gibt es $n$ Möglichkeiten, für die 2. Stelle $n-1$ usw. Für die letzte Stelle nur noch eine Möglichkeit. Insgesamt also $n\cdot ... \cdot 1=n!$ Möglichkeiten.

\newpage
\printbibliography
\end{document}