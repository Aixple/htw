%
% API Documentation for API Documentation
% Module ledleuchtturm.ledleuchtturm
%
% Generated by epydoc 3.0.1
% [Tue Jul 18 00:55:24 2017]
%

%%%%%%%%%%%%%%%%%%%%%%%%%%%%%%%%%%%%%%%%%%%%%%%%%%%%%%%%%%%%%%%%%%%%%%%%%%%
%%                          Module Description                           %%
%%%%%%%%%%%%%%%%%%%%%%%%%%%%%%%%%%%%%%%%%%%%%%%%%%%%%%%%%%%%%%%%%%%%%%%%%%%

    \index{ledleuchtturm \textit{(package)}!ledleuchtturm.ledleuchtturm \textit{(module)}|(}
\section{Modul ledleuchtturm.ledleuchtturm}

    \label{ledleuchtturm:ledleuchtturm}
Main-Funktion des Paketes ledleuchtturm. Sie repräsentiert den Ablauf des 
Programms. Sie beschränkt sich auf das Aufrufen von Funktionen aus den 
Modulen \texttt{stdFunc}, \texttt{fileFunc}, \texttt{dbFunc} oder 
\texttt{ledFunc}. Des Weiteren behandelt  sie die Auswertung von 
Rückgabewerten und das Schreiben von Meldungen auf der Konsole.

\textbf{Autor:} Simon Retsch




%%%%%%%%%%%%%%%%%%%%%%%%%%%%%%%%%%%%%%%%%%%%%%%%%%%%%%%%%%%%%%%%%%%%%%%%%%%
%%                               Functions                               %%
%%%%%%%%%%%%%%%%%%%%%%%%%%%%%%%%%%%%%%%%%%%%%%%%%%%%%%%%%%%%%%%%%%%%%%%%%%%

  \subsection{Funktionen}

    \label{ledleuchtturm:ledleuchtturm:signal_handler}
    \index{ledleuchtturm \textit{(package)}!ledleuchtturm.ledleuchtturm \textit{(module)}!ledleuchtturm.ledleuchtturm.signal\_handler \textit{(function)}}

    \vspace{0.5ex}

\hspace{.8\funcindent}\begin{boxedminipage}{\funcwidth}

    \raggedright \textbf{signal\_handler}(\textit{signal}, \textit{frame})

    \vspace{-1.5ex}

    \rule{\textwidth}{0.5\fboxrule}
\setlength{\parskip}{2ex}
    Behandelt das Signal welches durch Strg+C ausgelöst wird (SIGINT) und 
    beendet alle Verbindungen oder schließt alle offenen Dateien damit das 
    Programm ordnungsgemäß beendet wird.

\setlength{\parskip}{1ex}
      \textbf{Parameter}
      \vspace{-1ex}

      \begin{quote}
        \begin{Ventry}{xxxxxx}

          \item[signal]

          Signalnummer

            {\it (type=int)}

          \item[frame]

          Unterbrochener Stack Frame

            {\it (type=stack frame)}

        \end{Ventry}

      \end{quote}

\textbf{Autor:} Simon Retsch



    \end{boxedminipage}


%%%%%%%%%%%%%%%%%%%%%%%%%%%%%%%%%%%%%%%%%%%%%%%%%%%%%%%%%%%%%%%%%%%%%%%%%%%
%%                               Variables                               %%
%%%%%%%%%%%%%%%%%%%%%%%%%%%%%%%%%%%%%%%%%%%%%%%%%%%%%%%%%%%%%%%%%%%%%%%%%%%

  \subsection{Variablen}

    \vspace{-1cm}
\hspace{\varindent}\begin{longtable}{|p{\varnamewidth}|p{\vardescrwidth}|l}
\cline{1-2}
\cline{1-2} \centering \textbf{Name} & \centering \textbf{Beschreibung}& \\
\cline{1-2}
\endhead\cline{1-2}\multicolumn{3}{r}{\small\ldots}\\\endfoot\cline{1-2}
\endlastfoot\raggedright D\-E\-B\-U\-G\- & \raggedright Entscheidung über Ausgabe der debug-Nachrichten.

\textbf{Wert:} 
{\tt 1}            {\it (type=int)}&\\
\cline{1-2}
\end{longtable}

    \index{ledleuchtturm \textit{(package)}!ledleuchtturm.ledleuchtturm \textit{(module)}|)}
