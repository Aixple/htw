%
% API Documentation for API Documentation
% Module ledleuchtturm.ledFunc
%
% Generated by epydoc 3.0.1
% [Tue Jul 18 00:55:24 2017]
%

%%%%%%%%%%%%%%%%%%%%%%%%%%%%%%%%%%%%%%%%%%%%%%%%%%%%%%%%%%%%%%%%%%%%%%%%%%%
%%                          Module Description                           %%
%%%%%%%%%%%%%%%%%%%%%%%%%%%%%%%%%%%%%%%%%%%%%%%%%%%%%%%%%%%%%%%%%%%%%%%%%%%

    \index{ledleuchtturm \textit{(package)}!ledleuchtturm.ledFunc \textit{(module)}|(}
\section{Modul ledleuchtturm.ledFunc}

    \label{ledleuchtturm:ledFunc}
Funktionen zur Arbeit mit LEDs.

\textbf{Autor:} Simon Retsch




%%%%%%%%%%%%%%%%%%%%%%%%%%%%%%%%%%%%%%%%%%%%%%%%%%%%%%%%%%%%%%%%%%%%%%%%%%%
%%                               Functions                               %%
%%%%%%%%%%%%%%%%%%%%%%%%%%%%%%%%%%%%%%%%%%%%%%%%%%%%%%%%%%%%%%%%%%%%%%%%%%%

  \subsection{Funktionen}

    \label{ledleuchtturm:ledFunc:toggleLED}
    \index{ledleuchtturm \textit{(package)}!ledleuchtturm.ledFunc \textit{(module)}!ledleuchtturm.ledFunc.toggleLED \textit{(function)}}

    \vspace{0.5ex}

\hspace{.8\funcindent}\begin{boxedminipage}{\funcwidth}

    \raggedright \textbf{toggleLED}(\textit{portnumber})

    \vspace{-1.5ex}

    \rule{\textwidth}{0.5\fboxrule}
\setlength{\parskip}{2ex}
    Wechselt die den Zustand (an/aus) der LED am angegebenen Port.

\setlength{\parskip}{1ex}
      \textbf{Parameter}
      \vspace{-1ex}

      \begin{quote}
        \begin{Ventry}{xxxxxxxxxx}

          \item[portnumber]

          Nummer des Ports am Rasperry Pi.

            {\it (type=int)}

        \end{Ventry}

      \end{quote}

      \textbf{R"uckgabewert}
    \vspace{-1ex}

      \begin{quote}
      1 bei Erfolg, -1 bei Misserfolg.

      {\it (type=int)}

      \end{quote}

\textbf{Autor:} Falk-Jonatan Strube



    \end{boxedminipage}

    \label{ledleuchtturm:ledFunc:enableLED}
    \index{ledleuchtturm \textit{(package)}!ledleuchtturm.ledFunc \textit{(module)}!ledleuchtturm.ledFunc.enableLED \textit{(function)}}

    \vspace{0.5ex}

\hspace{.8\funcindent}\begin{boxedminipage}{\funcwidth}

    \raggedright \textbf{enableLED}(\textit{portnumber})

    \vspace{-1.5ex}

    \rule{\textwidth}{0.5\fboxrule}
\setlength{\parskip}{2ex}
    Schaltet die LED am angegeben Port an.

\setlength{\parskip}{1ex}
      \textbf{Parameter}
      \vspace{-1ex}

      \begin{quote}
        \begin{Ventry}{xxxxxxxxxx}

          \item[portnumber]

          Nummer des Ports am Rasperry Pi.

            {\it (type=int)}

        \end{Ventry}

      \end{quote}

      \textbf{R"uckgabewert}
    \vspace{-1ex}

      \begin{quote}
      1 bei Erfolg, -1 bei Misserfolg.

      {\it (type=int)}

      \end{quote}

\textbf{Autor:} Falk-Jonatan Strube



    \end{boxedminipage}

    \label{ledleuchtturm:ledFunc:switchAllOff}
    \index{ledleuchtturm \textit{(package)}!ledleuchtturm.ledFunc \textit{(module)}!ledleuchtturm.ledFunc.switchAllOff \textit{(function)}}

    \vspace{0.5ex}

\hspace{.8\funcindent}\begin{boxedminipage}{\funcwidth}

    \raggedright \textbf{switchAllOff}()

    \vspace{-1.5ex}

    \rule{\textwidth}{0.5\fboxrule}
\setlength{\parskip}{2ex}
    Schaltet alle LEDs (definiert durch die globale Variablen der 
    entsprechenden Ports) aus.

\setlength{\parskip}{1ex}
      \textbf{R"uckgabewert}
    \vspace{-1ex}

      \begin{quote}
      1 bei Erfolg, -1 bei Misserfolg.

      {\it (type=int)}

      \end{quote}

\textbf{Autor:} Falk-Jonatan Strube



    \end{boxedminipage}

    \label{ledleuchtturm:ledFunc:blink}
    \index{ledleuchtturm \textit{(package)}!ledleuchtturm.ledFunc \textit{(module)}!ledleuchtturm.ledFunc.blink \textit{(function)}}

    \vspace{0.5ex}

\hspace{.8\funcindent}\begin{boxedminipage}{\funcwidth}

    \raggedright \textbf{blink}(\textit{portnumber})

    \vspace{-1.5ex}

    \rule{\textwidth}{0.5\fboxrule}
\setlength{\parskip}{2ex}
    Lässt die LED am angegebenen Port blinken, indem der ein Thread in der 
    globalen Variabel \texttt{blinkThread} als \texttt{blinkThreadFunction}
    gestartet wird.

\setlength{\parskip}{1ex}
      \textbf{Parameter}
      \vspace{-1ex}

      \begin{quote}
        \begin{Ventry}{xxxxxxxxxx}

          \item[portnumber]

          Nummer des Ports am Rasperry Pi.

            {\it (type=int)}

        \end{Ventry}

      \end{quote}

      \textbf{R"uckgabewert}
    \vspace{-1ex}

      \begin{quote}
      1 bei Erfolg, -1 bei Misserfolg.

      {\it (type=int)}

      \end{quote}

\textbf{Autor:} Falk-Jonatan Strube



    \end{boxedminipage}

    \label{ledleuchtturm:ledFunc:blinkThreadFunction}
    \index{ledleuchtturm \textit{(package)}!ledleuchtturm.ledFunc \textit{(module)}!ledleuchtturm.ledFunc.blinkThreadFunction \textit{(function)}}

    \vspace{0.5ex}

\hspace{.8\funcindent}\begin{boxedminipage}{\funcwidth}

    \raggedright \textbf{blinkThreadFunction}(\textit{portnumber})

    \vspace{-1.5ex}

    \rule{\textwidth}{0.5\fboxrule}
\setlength{\parskip}{2ex}
    Diese Funktion wird als Thread genutzt und lässt die LED am angegeben 
    Port blinken: Das Warten zwischen dem Wechseln des LED-Zustands wird 
    über das globale \texttt{blinkThreadStopEvent} realisiert. Dadurch kann
    der Zustand des Blinkens jederzeit beendet werden.

\setlength{\parskip}{1ex}
      \textbf{Parameter}
      \vspace{-1ex}

      \begin{quote}
        \begin{Ventry}{xxxxxxxxxx}

          \item[portnumber]

          Nummer des Ports am Rasperry Pi.

            {\it (type=int)}

        \end{Ventry}

      \end{quote}

      \textbf{R"uckgabewert}
    \vspace{-1ex}

      \begin{quote}
      1 bei Erfolg, -1 bei Misserfolg.

      {\it (type=int)}

      \end{quote}

\textbf{Autor:} Falk-Jonatan Strube



    \end{boxedminipage}

    \label{ledleuchtturm:ledFunc:killBlinkThread}
    \index{ledleuchtturm \textit{(package)}!ledleuchtturm.ledFunc \textit{(module)}!ledleuchtturm.ledFunc.killBlinkThread \textit{(function)}}

    \vspace{0.5ex}

\hspace{.8\funcindent}\begin{boxedminipage}{\funcwidth}

    \raggedright \textbf{killBlinkThread}()

    \vspace{-1.5ex}

    \rule{\textwidth}{0.5\fboxrule}
\setlength{\parskip}{2ex}
    Beendet den für das Blinken verantwortliche Thread 
    \texttt{blinkThread}. Es wird zuerst das \texttt{blinkThreadStopEvent} 
    aktiviert und der Thread dann zusammengeführt.

\setlength{\parskip}{1ex}
      \textbf{R"uckgabewert}
    \vspace{-1ex}

      \begin{quote}
      1 bei Erfolg, -1 bei Misserfolg.

      {\it (type=int)}

      \end{quote}

\textbf{Autor:} Falk-Jonatan Strube



    \end{boxedminipage}

    \label{ledleuchtturm:ledFunc:changeState}
    \index{ledleuchtturm \textit{(package)}!ledleuchtturm.ledFunc \textit{(module)}!ledleuchtturm.ledFunc.changeState \textit{(function)}}

    \vspace{0.5ex}

\hspace{.8\funcindent}\begin{boxedminipage}{\funcwidth}

    \raggedright \textbf{changeState}(\textit{status})

    \vspace{-1.5ex}

    \rule{\textwidth}{0.5\fboxrule}
\setlength{\parskip}{2ex}
    Wechselt den Status der LED-Anzeige. Dabei werden zuerst Blink-Threads 
    beendet (falls vorhanden). Dann wird die dem übergebenen Status 
    entsprechende LED eingeschaltet oder das Blinken gestartet. Wenn der 
    übergebene Status unbekannt ist, werden alle LEDs ausgeschaltet.

\setlength{\parskip}{1ex}
      \textbf{Parameter}
      \vspace{-1ex}

      \begin{quote}
        \begin{Ventry}{xxxxxx}

          \item[status]

          Status, zu dem gewechselt werden soll (bekannte Stati: 
          PROC/IDLE/MAINT/DOWN).

            {\it (type=string)}

        \end{Ventry}

      \end{quote}

      \textbf{R"uckgabewert}
    \vspace{-1ex}

      \begin{quote}
      1 bei Erfolg, -1 bei Misserfolg.

      {\it (type=int)}

      \end{quote}

\textbf{Autor:} Falk-Jonatan Strube



    \end{boxedminipage}

    \label{ledleuchtturm:ledFunc:setLEDsGlobal}
    \index{ledleuchtturm \textit{(package)}!ledleuchtturm.ledFunc \textit{(module)}!ledleuchtturm.ledFunc.setLEDsGlobal \textit{(function)}}

    \vspace{0.5ex}

\hspace{.8\funcindent}\begin{boxedminipage}{\funcwidth}

    \raggedright \textbf{setLEDsGlobal}(\textit{configData})

    \vspace{-1.5ex}

    \rule{\textwidth}{0.5\fboxrule}
\setlength{\parskip}{2ex}
    Setzt die Globalen Variablen für die LED Pins mit Hilfe der Daten aus 
    der Konfigurationsdatei.

\setlength{\parskip}{1ex}
      \textbf{Parameter}
      \vspace{-1ex}

      \begin{quote}
        \begin{Ventry}{xxxxxxxxxx}

          \item[configData]

          Beinhaltet die Daten aus der Konfigurationsdatei der 
          Konfigurationsdatei.

            {\it (type=Json Object)}

        \end{Ventry}

      \end{quote}

\textbf{Autor:} Simon Retsch



    \end{boxedminipage}

    \label{ledleuchtturm:ledFunc:initLEDs}
    \index{ledleuchtturm \textit{(package)}!ledleuchtturm.ledFunc \textit{(module)}!ledleuchtturm.ledFunc.initLEDs \textit{(function)}}

    \vspace{0.5ex}

\hspace{.8\funcindent}\begin{boxedminipage}{\funcwidth}

    \raggedright \textbf{initLEDs}()

    \vspace{-1.5ex}

    \rule{\textwidth}{0.5\fboxrule}
\setlength{\parskip}{2ex}
    Initialisiert die LEDs des Raspberry Pi: Setzen des Pinning-Layouts und
    Pins als Output definieren

\setlength{\parskip}{1ex}
      \textbf{R"uckgabewert}
    \vspace{-1ex}

      \begin{quote}
      1 falls das Initialisieren erfolgreich war, -1 wenn nicht.

      {\it (type=int)}

      \end{quote}

\textbf{Autor:} Simon Retsch



    \end{boxedminipage}


%%%%%%%%%%%%%%%%%%%%%%%%%%%%%%%%%%%%%%%%%%%%%%%%%%%%%%%%%%%%%%%%%%%%%%%%%%%
%%                               Variables                               %%
%%%%%%%%%%%%%%%%%%%%%%%%%%%%%%%%%%%%%%%%%%%%%%%%%%%%%%%%%%%%%%%%%%%%%%%%%%%

  \subsection{Variablen}

    \vspace{-1cm}
\hspace{\varindent}\begin{longtable}{|p{\varnamewidth}|p{\vardescrwidth}|l}
\cline{1-2}
\cline{1-2} \centering \textbf{Name} & \centering \textbf{Beschreibung}& \\
\cline{1-2}
\endhead\cline{1-2}\multicolumn{3}{r}{\small\ldots}\\\endfoot\cline{1-2}
\endlastfoot\raggedright G\-R\-E\-E\-N\-\_\-P\-O\-R\-T\- & \raggedright Port der grünen LED auf dem Raspberry Pi.

\textbf{Wert:} 
{\tt -1}            {\it (type=int)}&\\
\cline{1-2}
\raggedright Y\-E\-L\-L\-O\-W\-\_\-P\-O\-R\-T\- & \raggedright Port der gelben LED auf dem Raspberry Pi.

\textbf{Wert:} 
{\tt -1}            {\it (type=int)}&\\
\cline{1-2}
\raggedright R\-E\-D\-\_\-P\-O\-R\-T\- & \raggedright Port der roten LED auf dem Raspberry Pi.

\textbf{Wert:} 
{\tt -1}            {\it (type=int)}&\\
\cline{1-2}
\raggedright b\-l\-i\-n\-k\-T\-h\-r\-e\-a\-d\- & \raggedright Objekt des Threads der das Blinken einer LED realisiert.

\textbf{Wert:} 
{\tt threading.Thread()}            {\it (type=Thread object)}&\\
\cline{1-2}
\raggedright b\-l\-i\-n\-k\-T\-h\-r\-e\-a\-d\-S\-t\-o\-p\-E\-v\-e\-n\-t\- & \raggedright Objekt eines Thread Events, welches den Thread beendet.

\textbf{Wert:} 
{\tt threading.Event()}            {\it (type=Thread event object)}&\\
\cline{1-2}
\end{longtable}

    \index{ledleuchtturm \textit{(package)}!ledleuchtturm.ledFunc \textit{(module)}|)}
