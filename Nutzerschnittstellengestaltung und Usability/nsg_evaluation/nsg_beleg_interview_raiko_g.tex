\subsection*{Interview-Fragen}
\subsubsection*{Zur Person}
\begin{enumerate}
	\item Was ist dein Name?
	\interviewText{Raiko Glaser}
	\item Wie alt bist du?
	\interviewText{21}
	\item Was arbeitest/studierst du?
	\interviewText{Ich mache bei der deutschen Telekom AG eine Ausbildung zum Fachinformatiker für Anwendungsentwicklung.}
	\item Wie oft bzw. wie lange browst du Webseiten im Internet?
	\interviewText{Ich browse am Tag so ungefähr 2 Stunden privat auf Websiten, außerdem besuche ich für die Arbeit noch häufiger Websiten.}
	\item Auf welchen Seiten bist du regelmäßig unterwegs?
	\interviewText{Ich nutze vorallem Youtube und ntv}
	\item Wie schätzt Du Deine Kenntnisse im Umgang mit Webseiten ein?
	\interviewText{Im Zweifel Leihenhaft}
	\item Wofür nutzen Du das Internet (außer zum Browsen)?
	\interviewText{Computerspiele, Filme schauen (Netflix)}
	\item War Dir diese Webseite im vornherein bekannt? Wenn ja: Woher?
	\interviewText{Nein.}
\end{enumerate}

\subsubsection*{Allgemeines}
\begin{enumerate}
	\item Wie leicht fiel es Dir dich auf der Seite zu orientieren?
	%               -        ~        +  von bis
	\interviewScala{ }{ }{ }{ }{ }{6}{ }{schwer}{leicht}
	Was viel dir besonders schwer/leicht:
	\interviewText{wenig Kontent, daher gute Orientierung\\ Unterpunkte an den Punkten Code und Kreatives könnten direkt auf die Punkte dieser Seiten zeigen\\ AWS-Rundmail ist zu unübersichtlich.}
	\item Gab es Probleme beim Bedienen der Website?
	%               -        ~        +  von bis
	\interviewScala{ }{ }{ }{ }{ }{ }{7}{viele}{wenige}
	Anmerkungen:
	\interviewText{Bedienung ist in Ordnung}
	\item Konntest Du die Website auch gut auf dem Handy bedienen?
	%               -        ~        +  von bis
	\interviewScala{ }{ }{ }{ }{5}{ }{ }{schwer}{leicht}
	Anmerkungen:
	\interviewText{AWS ist zu mühevoll\\ Auf dem S3 konnte ich sie nicht öffnen.}
	\item Würdest Du etwas an der Webseite verändern?
	%               -        ~        +  von bis
	\interviewScala{ }{ }{ }{4}{ }{ }{ }{viel}{wenig}
	Wenn ja, was?
	\interviewText{Ich würde große Teile des Designs und Layouts ändern}
	\item Wie würdest du die Seite beschreiben? Was ist der Zweck der Seite?
	\interviewText{Der Zweck dieser Seite ist meiner Meinung nach Jonathans Zeug vorzustellen.}
	\item Kennst Du Webseiten, die mit dieser vergleichbar sind? Wenn ja: Wie lauten diese? In wie fern sind sie besser/schlechter/ähnlich gestaltet?
	\interviewText{Ja, manche Foren haben auch Sammlungen an Mods und Programmbeispiele}
\end{enumerate}

\subsubsection*{Zum Design}
\begin{enumerate}
	\item Wie bewertest du das Design der Seite?
	\begin{enumerate}
		\item Der Gesamteindruck war…
		%               -        ~        +  von bis
		\interviewScala{ }{ }{3}{ }{ }{ }{ }{schlecht}{gut}
		Anmerkungen:
		\interviewText{Es ist ein Design zu erkennen, aber es ist etwas fade.}
		\item Die Farbwahl und -gestaltung war…
		%               -        ~        +  von bis
		\interviewScala{ }{2}{ }{ }{ }{ }{ }{schlecht}{gut}
		Anmerkungen:
		\interviewText{Farbwahl ist geschmackssache.\\ Es wurden Farben verwendet. Es hätten aber ruhig auch etwas mehr mit Farbe gearbeitet werden können.}
		\item Die Anordnung der Inhalte war…
		%               -        ~        +  von bis
		\interviewScala{ }{ }{ }{ }{ }{6}{ }{schlecht}{gut}
		Anmerkungen:
		\interviewText{Ich sehe das AWS eher im Unterpunkt Kreativ, als im Unterpunkt Code.\\ Ich finde die Sidebar auf den Seiten Kreativ und Code überflüssig, da der Kontent auch so überschaubar ist.}
		\item Die Präsentation der Webseite war im Bezug auf den Inhalt…
		%               -        ~        +  von bis
		\interviewScala{ }{ }{ }{ }{ }{ }{7}{unangemessen}{angemessen}
		Anmerkungen:
		\interviewText{keine Anmerkung.}
		\item Die Lesbarkeit der Webseite war…
		%               -        ~        +  von bis
		\interviewScala{ }{ }{ }{ }{ }{ }{7}{schlecht}{gut}
		Anmerkungen:
		\interviewText{Keine Anmerkungen.}
	\end{enumerate}
\end{enumerate}

\subsubsection*{Automatisches Wichtel System}
\begin{enumerate}
	\item Sind Dir bei der Bedienung des Automatisierten Wichtel Systems Besonderheiten aufgefallen, die Du noch nicht erwähnt hast?
	\interviewText{Es gibt eine Englische Übersetzung.\\ Es ist keine Anmeldung im AWS notwendig und dort ist eine kleine Sicherheit gegen Missbrauch eingebaut.}
\end{enumerate}

\subsubsection*{Abschließende Fragen}
\begin{enumerate}
	\item Würdest du die Website weiter empfehlen? Warum? Warum nicht?
	\interviewText{Prinzipiell ja, aber aktuell fällt mir da niemand ein.}
	\item Würdest du die Funktionen der Seite erneut nutzen?
	\interviewText{nein}
\end{enumerate}
