\slides{03_Begriffe-KG_Eleganz-Kontrast}{2}
Mullet und Sano beschreiben sechs allgemeine Begriffe der Kommunikationsgestaltung. Sie wenden sich dabei gezielt an Designer interaktiver Systeme.

\subsubsection*{Aspekte der Begriffe}
\emph{Bedeutung und Funktion} beinhalten allgemeine Angaben zum Begriff.

\emph{Gestaltungsprinzipien} konkretisierten die Bedeutung des Begriffs, bleiben dennoch abstrakt in dem Sinne, dass man sie nicht unmittelbar umsetzen kann.

Erst bei der Diskussion der \emph{praktischen Techniken} und der \emph{typischen Fehler} bekommen Sie „Rezepte“, die sich direkt in ein Gestaltungsprojekt anwenden lassen. $\to$ Was tun? Was vermeiden?

Die Beispiele sollen Konzept, Prinzipien und Techniken (ggf. auch Fehler) illustrieren. Lassen Sie sich inspirieren!

\section{Eleganz und Einfachheit}
\slides{03_Begriffe-KG_Eleganz-Kontrast}{5}
Dieses Zitat des französischen Autors Antoine de Saint Exupèry verrät uns schon eine Technik für mehr Eleganz: Elemente sollen in einer Gestaltung gezielt nach und nach entfernt werden. Ist die Anwendung trotzdem gut verständlich, dann kann man getrost darauf verzichten. Die Gestaltung wird klarer und wirkt eleganter.

\subsection{Bedeutung und Funktion}
\subsubsection*{Bedeutung}
„Eleganz“ kommt aus dem Latein „eligere“. Es bedeutet: auswählen, aussuchen, auslesen.  

Eleganz bedeutet für den Screendesigner in der Tat zunächst einmal Auswahl. Der Designer entscheidet, welche Elemente auf der Oberfläche gezeigt werden sollen und was ausgelassen werden kann.

Eleganz ist unmittelbar mit Einfachheit (Simplicity) verbunden. Die elegantesten Lösungen sind oft die einfachsten: wir schätzen eine Gestaltung, die die gestellte Aufgabe in einer einfachen, ökonomischen Art und Weise löst, die \emph{mit wenigen Mitteln viel erreicht}.

\subsubsection*{Funktion}
Eleganz und Einfachheit wirken nicht nur \emph{ästhetisch}. Sie sind auch ein großer Pluspunkt, um Oberflächen leicht zu bedienen und benutzerfreundlich zu machen (\emph{Kommunikation}), weil unnötige Elemente, die verwirren könnten, erst gar nicht in Erscheinung treten.

\subsection{Prinzipien}
\subsubsection*{Angemessenheit (Fitness)}
Die Gestaltung sollte keine unnötigen Elemente aufweisen, alles was zu sehen ist, muss auf die Aufgabe gerichtet sein, muss eine Funktion erfüllen, nichts darf verzichtbar sein. 

MERKE: Deswegen ist Kitsch das Gegenteil vom Eleganz

$\to$ „Form follows function“
 
\subsubsection*{Einheitlichkeit (Unity)}
Die Elemente im Design müssen einheitlich wirken und ein kohärentes Ganzes bilden („aus einem Guss“)
 
\subsubsection*{Verfeinerung (Refinement)}
Die Gestaltung muss den Blick auf das Wesentliche lenken, alles Oberflächliche und Sekundäre tritt in den Hintergrund oder –- noch besser -– entfällt.

\subsection{Techniken}
\subsubsection*{Standardisierung (Regularization)}
Elemente standardisieren, wiederholen; Regelmäßigkeit und Rhythmus in die Gestaltung bringen, d.h. Vielfalt  zurücknehmen (natürlich nur dort, wo sie nicht sinnvoll ist)

\emph{Wiederholung} haben Sie als eine der vier grundlegende Designregeln von Robin Williams. Jetzt können wir ihre Bedeutung besser verstehen und sehen, dass Wiederholung kein Ziel  an sich ist, sondern u.a. ein Mittel für mehr Eleganz.
 
\subsubsection*{Reduktion (Reduction)}
Ein gelungenes Design sollte auf die wesentlichen Elemente reduziert werden; jedes Element soll wieder simplifiziert und abstrahiert werden. 

TIPP: Probieren Sie aus, Dinge weg zu lassen, die zunächst vielleicht wichtig erscheinen. Wie wirkt das reduziertere Design? Oft ist die einfachere Version auch die bessere. Erfahrene Designer müssen ebenfalls viel ausprobieren, bis sie herausfinden, was entfernt werden kann und was bleiben muss.
 
\subsubsection*{1 Element = 2 Funktionen (Leverage)}
Der englische Ausdruck „Leverage“ ist in diesem Zusammenhang schwer zu übersetzen, daher die Umschreibung. 

Es bedeutet folgendes: Prüfen Sie, ob ein Designelement mehrere Funktionen erfüllen kann, so dass die Anzahl der Designelemente weiter reduziert werden kann. Das wirkt immer besonders elegant.

\subsection{Fehler}
\slides{03_Begriffe-KG_Eleganz-Kontrast}{9}

\subsection{Beispiele}
\slides{03_Begriffe-KG_Eleganz-Kontrast}{11}
\slides{03_Begriffe-KG_Eleganz-Kontrast}{12}
Die Autorin dieser Gestaltung hat diverse Techniken eingesetzt, damit die Anwendung elegant anmutet:

\begin{itemize}
\item Wiederholung und Regelmäßigkeit: die Fotos im Kleinformat wiederholen sich auf beiden Seiten, der feinen Rahmen um die Seite wiederholt sich ebenfalls
\item Reduktion auf die wesentlichen Elemente: die Modelle und die Information. Auch farblich ist die Gestaltung zurückhaltend und elegant.
\item 1 Element = 2 Funktionen: Auf der ersten Seite bildet der Rahmen der 3 oberen Bilder gleichzeitig die Unterstreichung für die Titelzeile; auf der zweiten Seite bietet das Bild einer Etikette (passend zum Thema) den Rahmen für diverse Informationen.
\end{itemize}


\slides{03_Begriffe-KG_Eleganz-Kontrast}{14}
Diese Homepage „Hope Garden“ wirkt besonders elegant.

Hier wird die Technik „1 Element = 2 Funktionen“ wiederholt angewandt:
\begin{itemize}
\item Die Wiese dient gleichzeitig als Hintergrund für den Steuerungsbereich.
\item Der Stiel der Blume trennt zwei Bedeutungsbereiche im Steuerungsbereich: ein Informationsteil und die eigentliche interaktive Funktionen bzw. Links.
\item Die interaktive Funktion „Select a Flower“ ist gestalterisch durch Position und Farbe in den Himmel integriert.
\end{itemize}


\section{Proportion und Kontrast}
\slides{03_Begriffe-KG_Eleganz-Kontrast}{16}
Dieses Zitat verrät: visuelle Elemente sind  kein Ziel an sich: sie vermitteln Bedeutung. Das gilt besonders für Unterschiede bzw. Kontrast –- aber eigentlich gilt es für jedes Element einer Gestaltung.

\subsection{Bedeutung und Funktion}
\subsubsection*{Bedeutung}
Die Qualität eines Bildschirmdesigns hängt nicht nur von der ästhetischen Beschaffenheit einzelner Elemente ab, sondern vor allem von der Beziehung der Teile untereinander. Mit dem Begriff „Proportion und Kontrast“ wird diese Beziehung thematisiert.\bigskip

\emph{Kontrast}
Damit die Teile einer Gestaltung zur Geltung kommen, müssen sie sich voneinander unterscheiden. Unterschiede bilden die Basis für die Beziehungen zwischen den Teilen.

Kontrast kennen Sie als Grundregel der Gestaltung nach R. Williams. Kontrast kann erzeugt werden mit Größenverhältnissen, Formen, Ausrichtung, Farben, Bildern, Schrift.\bigskip

\emph{Proportion}
Gleichwohl sollen die Teile einer Gestaltung miteinander harmonieren und in Gleichgewicht sein. Auch Hintergrund und Vordergrund, Focus und Muster sollen in Gleichgewicht stehen und ein angenehmes und stimmiges Ganzes bilden.

Proportion und Kontrast müssen zueinander abgestimmt werden, um die  richtige und schwierige Balance zwischen interessantem, dynamischem Eindruck und gefälligen, \emph{harmonischen Verhältnissen} zu erreichen. Sie stehen auf keinem Fall in Wiederspruch zueinander.

\subsubsection*{Funktion}
Unterschiede helfen dabei, den Bildschirm zu organisieren und Bedeutung zu vermitteln. Zum Beispiel kann Kontrast benutzt werden, um Regionen mit verschiedener Bedeutung auf dem Bildschirm zu verdeutlichen (\emph{Kommunikation}).

Die gelungene Mischung von Proportion und Kontrast wirkt dynamisch, interessant, anregend und gleichzeitig gefällig und angenehm (\emph{Ästhetik, Emotion}). 


\subsection{Prinzipien}
\slides{03_Begriffe-KG_Eleganz-Kontrast}{18}
\subsubsection*{Harmonie (Harmony)}
Die Teile einer Gestaltung müssen so zueinander gesetzt werden, dass das Ganze einen gleichgewichtigen, ästhetisch befriedigenden Eindruck vermittelt. Dafür werden die Teile oft in bestimmten mathematischen Verhältnissen zueinander gesetzt. Es werden oft einfache rationale Proportionen benutz, wie z.B. bei den Buchseiten 1:2, 2:3 (sog. Oktav), 3:4 (sog. Quart), 5:8 (Annährung an den goldenen Schnitt), 5:9. Auch geometrisch definierbaren klaren irrationalen Proportionen ergeben Harmonie:
\begin{itemize}
\item 1:1,618 - goldener Schnitt
\item 1:1,414 - Quadratwurzel von 2 (DIN-Formate), 
\item 1:1,732 - Quadratwurzel von 3
\item 1: 2,236 - Quadratwurzel von 5
\end{itemize}
\slides{03_Begriffe-KG_Eleganz-Kontrast}{19}
Zum goldenen Schnitt:
\slides{03_Begriffe-KG_Eleganz-Kontrast}{20}
\slides{03_Begriffe-KG_Eleganz-Kontrast}{21}
\slides{03_Begriffe-KG_Eleganz-Kontrast}{22}
\subsubsection*{Klarheit (Clarity)}
Kontrast muss deutlich sein, Unterschiede dürfen nicht zufällig oder ungewollt erscheinen. Durch klaren Kontrast wird die Bedeutung der Elemente eindeutig vermittelt. 
 
\subsubsection*{Dynamik (Activity)}
Klare Unterschiede können dynamisch sogar dramatisch wirken. Sie können „Bewegung“ in die Gestaltung bringen.
 
\subsubsection*{Zurückhaltung (Restraint)}
Kontrast sollte deutlich und dramatisch sein, aber mit Bedacht eingesetzt werden. Viele Unterschiede stören sich gegenseitig und bewirken visuelles „Rauschen“. Besser sind WENIGE aber DEUTLICHE Kontraste. 

\subsection{Techniken}
\slides{03_Begriffe-KG_Eleganz-Kontrast}{24}
Sorgen Sie für Gleichgewicht zwischen Figur und Hintergrund. Setzen Sie Figur und Hintergrund bewusst in Beziehung, vermeiden Sie, dass Figur und Grund sich gegenseitig stören, vielmehr müssen sie harmonieren. Zum Beispiel geben Sie der Figur Raum, um sich zu entfalten, experimentieren Sie mit Größe und Position des Vordergrunds.
 
Bilden Sie Bedeutungsebenen (Layering), vermitteln Sie die Zusammengehörigkeit einer Ebene mit visuellen Mitteln: z.B. Nähe, selbe Farbe, selbe Hintergrund, selbe Form. Sorgen Sie für Proportion und Kontrast zwischen Ebenen.
 
Blinzen Sie mit den Augen und lehnen Sie sich zurück (Squint test). Welche Einheiten, welche Formen sind im Design noch zu erkennen? Bedenken Sie: nur diese grobe Formen werden vom „normalen“ Betrachter auf dem ersten Blick wahrgenommen.
 
Benutzen Sie nicht mehr oder weniger ähnliche Elemente, wenn Sie Kontrast erzeugen wollen, sondern übertreiben Sie bewusst die Unterschiede (Sharpening). Ermitteln Sie das ganze Spektrum, die ganze Skala der möglichen Unterschiede und probieren Sie bewusst extreme Werte aus. Konzentrieren Sie sich dabei auf wenigen Elementen, die deutlich kontrastieren z.B. nur die Farben oder nur die Schrift.

\slides{03_Begriffe-KG_Eleganz-Kontrast}{25}

\subsection{Fehler}
\slides{03_Begriffe-KG_Eleganz-Kontrast}{26}

\subsection{Beispiele}
\slides{03_Begriffe-KG_Eleganz-Kontrast}{27}
\slides{03_Begriffe-KG_Eleganz-Kontrast}{28}
Beispiel aus der Internetpräsenz der Medieninformatik bis 2010. 

Der Gestalter hat auf harmonische Proportionen geachtet:
\begin{itemize}
\item Der Steuerungsbereich an der linken Seite ist genau doppelt so breit wie das Bild an der rechen Seite.
\item Der Content-Bereich (grauer Hintergrund) ist genau doppelt so breit wie Steuerung und Bild zusammen. 
\item Dadurch ist der Content-Bereich dreimal so breit wie die Steuerung. Es sind klare, rationale Proportionen
\end{itemize}

\slides{03_Begriffe-KG_Eleganz-Kontrast}{29}
Dieses Beispiel stammt aus einer elektronischen Designzeitschrift, das Adobe mittlerweile eingestellt hat.

In dieser Gestaltung wird Kontrast gezielt und wiederholt eingesetzt. Man kann sagen, die Gestaltung „lebt“ von diesem Konzept, das meisterhaft verwirklicht wurde.

\begin{itemize}
\item Bunt/Unbunt-Kontrast: ein Rechteck ist schwarz, die andere sind bunt. Im Logo ist das Wort „Magazine“ schwarz/grau, der Buschtabe „a“ dagegen rot.
\item Komplementär-und Kalt-Warm-Kontrast: Rot-Grün-Blau der Quadrate; Magenta-Grün im „Rauch“.
\item Kontrast zwischen strengen, schweren Rechtecken, die das Auge sozusagen nach unten ziehen, und der Leichtigkeit der grünen Figur und des Rauches, die nach oben streben.
\item Im Logo arbeitet der Designer nicht nur mit Farbkontrast sondern auch mit typografischen Unterschieden und Unterschiede in der Ausrichtung der Buchstaben, dabei ist das Logo streng ausgerichtet.
\end{itemize}

\slides{03_Begriffe-KG_Eleganz-Kontrast}{30} 
Der Designer hat ganz klar mit der Technik der Verstärkung gearbeitet: die Wörter „Concept Media“ sind überdimensioniert. 

Die Technik „Figur und Grund“ ist interessant variiert bzw. direkt gebrochen: die Wörter „Concept Media“ haben keinen Raum, um sich zu entfalten und stoßen an den Rändern. Dass diese nicht Einhaltung der Regel nicht zufällig ist, sieht man darin, dass die einzelne Buchstaben selbst miteinander „kollidieren“, sie haben auch keinen Platz. Dadurch wirkt die Gestaltung umso stärker. 

\slides{03_Begriffe-KG_Eleganz-Kontrast}{31}
Das Bild ist gelungen aber man merkt, dass der Autor noch ein (talentierter) Anfänger ist.

\begin{itemize}
\item Der Bunt-Unbunt-Kontrast verliert durch die Rücknahme des Alpha-Wertes im Kind an Wirkung.
\item Die Proportionen der beiden Kreisen, (die durch den Blinzen-Test erkannt werden) Kopf und Steuerungsrad, harmonieren nicht miteinander, sie kontrastieren aber auch nicht wirklich; sie stehen in keinem klaren Verhältnis zueinander.
\end{itemize}


\section{Organisation und Struktur}
\slides{04_Begriffe-KG_Struktur-Programm}{5}
Dieses Zitat des Malers Paul Klee deutet an, wie wichtig die Organisation einer Gestaltung ist, denn sie bestimmt die Art, wie das Werk betrachtet und verstanden wird.

\subsection{Bedeutung und Funktion}
\subsubsection*{Bedeutung }
Organisation und Struktur führen die Augen des Betrachters und \emph{vermitteln ihm den Inhalt} in einer geeigneten Art und Weise. 

Ohne eine klare Struktur wirkt das Design chaotisch und ist schwer zu verstehen; man kommt sich verloren vor.

\subsubsection*{Funktion}
Organisation und Struktur sind wesentlich für die \emph{Kommunikation}. Eine gelungene Struktur macht den Bildschirm gut lesbar und die Information schnell auffindbar. 

Sie gibt dem Betrachter ein Gefühl der Kontrolle und der Zuversicht; er fühlt sich gut aufgehoben und fasst Vertrauen (\emph{Emotion}).

\subsection{Prinzipien}
\subsubsection*{Gruppieren (Grouping)}
Dieses Prinzip ist vergleichbar mit der Technik  „Layering“ oder Ebenen-Bildung zum Begriff „Proportion und Kontrast“. Besonders wichtig ist es aber für das Konzept der Struktur und Organisation, denn Organisieren bedeutet zunächst einmal Klassifizieren, Unterteilen, Gruppieren. Auch die Designerin Claudia Runk spricht davon, wie wichtig es ist, Gruppen zu bilden.

\subsubsection*{Beziehung und Hierarchie (Relationship, Hierarchy)}
Nachdem Gruppen gebildet wurden, müssen die Beziehungen untereinander geklärt und gezeigt werden:
\begin{itemize}
\item was zusammen gehört, muss in visueller Beziehung gebracht werden z.B. durch Nähe, Ausrichtung, Ähnlichkeit…
\item was wichtiger ist, muss deutlicher und schneller ins Auge fallen z.B. durch Größe, Farbe, Position, Hintergrund…

\end{itemize}
 
\subsubsection*{Gleichgewicht (Balance)}
Eine Struktur sollte harmonisch wirken. Gleichgewicht hat im visuellen Sinn eine ähnliche Bedeutung wie im physikalischen Sinn. Eine Komposition ist in Gleichgewicht, wenn das „Gewicht“ der Elemente auf beiden Seiten des Bildschirms ungefähr gleich ist.


\subsection{Techniken}
Benutzen Sie \emph{Nähe}, um Zusammengehörigkeit visuell deutlich zu machen und setzen Sie die entstandenen leeren Bereiche („negative space“ bzw. „white space“) bewusst ein. Abstand und leere Flächen sind wesentlich, um den Inhalt zu organisieren.\bigskip  
 
Benutzen Sie \emph{Ausrichtung} (Alignement), um visuelle Beziehungen herzustellen. Werfen Sie einen Blick auf die Ränder, stellen Sie die Elementen an den Ränder exakt bündig. Suchen Sie eine bestehende Bündigkeit und stellen Sie evtl. frei stehende Elemente bündig dazu.\bigskip

Benutzen Sie \emph{Symmetrie} (Symmetry), um Gleichgewicht zu gewährleisten. Symmetrie hat einen universellen ästhetischen Wert, Symmetrie bringt Harmonie und Schönheit in eine Gestaltung.

Merke: Symmetrie wirkt eher klassisch und gesetzt, sie vermittelt Stabilität und Ruhe. Daher ist Symmetrie in modernen, dynamischen Gestaltungen nicht „angesagt“. Außerdem muss man mit Symmetrie und mit zentrierter Ausrichtung vorsichtig umgehen, weil Anfänger und Leute, die von Gestaltung nichts verstehen, sich oft für eine zentrierte Ausrichtung entscheiden. Benutzen Sie Symmetrie und zentrierte Ausrichtung immer bewusst und betonen Sie sie.

\slides{04_Begriffe-KG_Struktur-Programm}{9}
Achtung! Bei der Bündigkeit ist nicht die physikalische, sondern die wahrgenommene Größe wichtig!

\subsection{Fehler}
\emph{Planlose und desorganisierte Gestaltung} ohne eine klare Struktur Dieser Fehler geht oft mit einem verwandten Problem einher: zu viel und desorganisierter Inhalt auf einer Fläche.

\emph{Falsche Struktur}, so dass der Inhalt falsch verstanden wird.

\emph{Konflikte in der Symmetrie oder in der Ausrichtung} entstehen, wenn verschiedene Ausrichtungen gemischt werden, die nicht genug kontrastieren.

\emph{Unklare Beziehungen}, die zu Missverständnisse führen können.

\subsection{Beispiele}
\slides{04_Begriffe-KG_Struktur-Programm}{11}
Der Bildschirm wirkt u.a. durch die verschiedene Ausrichtungen (zentriert und rechtsbündig) und die unklare räumliche Beziehungen desorganisiert. 
\slides{04_Begriffe-KG_Struktur-Programm}{12}
Die exakte Ausrichtung der kleinen grauen Bedienelemente außerhalb des Contents (Zum Fachbereich, Copyright, Impressum und die kleine Piktogramme) sorgt für die optische Verbindung der über den ganzen Bildschirm verteilten Elemente, die aber inhaltich eine Einheit bilden.
\slides{04_Begriffe-KG_Struktur-Programm}{13}
Symmetrie, Ausrichtung, Nähe, alles in dieser Gestaltung dient der Organisation. Der Bildschirm wirkt harmonisch und aufgeräumt, man findet sich schnell zurecht.
\slides{04_Begriffe-KG_Struktur-Programm}{14}
Diese Homepage ist wunderbar desorganisiert, man kann sie aber auch in Ordnung bringen (siehe unterhalb). Diese Gestaltung ist besonderes raffiniert.

An diesem Beispiel sieht man sehr deutlich: wenn ein Designer die Gestaltungsregel beherrscht, dann und nur dann kann er mit ihnen kreativ umgehen.
\slides{04_Begriffe-KG_Struktur-Programm}{15}

\section{Modul und Programm}
\slides{04_Begriffe-KG_Struktur-Programm}{17}
Albert Einstein war kein Designer aber er zeigt mit diesem Satz, dass er verstanden hat, wofür ein Design-Programm gut ist.
\subsection{Bedeutung und Funktion}
\subsubsection*{Bedeutung}
Die Gestaltung interaktiver Systeme ist nie Gestaltung eines einzigen Bildschirms sondern \emph{Gestaltung eines umfangreichen Systems}. Das nennen Mullet und Sano „Programm“.

Die Gestaltung einer Internetpräsenz  z.B. ist nicht nur Gestaltung eines Bildschirms sonder der ganzen Site. So unterscheidet Nielsen in „Designing Web Usability“ (2000) zwischen Page-Design und Site-Design.

Design-Programme basieren auf Wiederholungen von Größen oder Proportionen (Modul) oder von bestimmten Formen und Ideen (Themen). Dabei wird die Wiederholung von Proportionen entschieden durch die Gestaltung eines Rasters unterstützt.

Das schwierige bei Design-Programmen ist bei allen festen Vorgaben Flexibilität zu gewährleisten, damit die Anforderungen der verschiedenen Teile im Programm erfüllt werden können. 

\subsubsection*{Funktion}
Modul und Programm sind mit dem Konzept der Struktur verbunden. Sie gewährleisten eine durchgezogene Struktur durch verschiedene Screens oder gar durch eine Darstellung in Print und in interaktiven Medien.

Sie machen die Gestaltung besser verständlich, die Nutzung leichter (\emph{Kommunikation}) und unterstützen den Wiedererkennungswert und die mögliche Identifikation mit einer Marke (\emph{Emotion}).

\subsection{Prinzipien}
\subsubsection*{Fokus (Focus)}
Ein erfolgreiches Design-Programm sollte auf eine oder auf wenige wesentliche Ideen basieren, die die Grundlage des Programms bilden und sich durch das Programm durchziehen. 

\subsubsection*{Flexibilität (Flexibility)}
Ein erfolgreiches Design-Programm muss flexible genug sein, um extreme oder unvorhergesehene Darstellungen zu erlauben. Zum Beispiel dürfen Rastersysteme nicht starr sein sondern für Variabilität sorgen.  

\subsubsection*{Konsistente Anwendung (Consistent Application)}
Ein gutes Design-Programm muss schlüssig und entschieden an jeder Situation angewandt werden, damit es als solches wahrgenommen werden kann. 

\subsection{Techniken}
\subsubsection*{Wiederholung (Repetion)}
Unsere alte „Bekannte“, die Wiederholung, treffen wir hier noch einmal. Ein Design-Programm wird durch die Wiederholung von Designelementen bzw. von bestimmten Proportionen durch das ganze Programm verstärkt; Wiederholungen steigern den Wiedererkennungswert.
 
\subsubsection*{Rasterung (Grid-Based Layout)}
Die Aufteilung der zu gestaltenden Flächen in Bereichen durch ein Rastersystem ist wesentlich für die Strukturierung des Inhalts. Die Wiederholung des Rastersystems trägt entscheidend zum Charakter eines Designprogramms bei. Rastersysteme (obwohl nicht ausdrücklich sichtbar) sind wesentlich für die Gestaltung.

\slides{04_Begriffe-KG_Struktur-Programm}{21}
Der Satzspiegel wird gitterförmig in Felder unterteilt. Zwischen den Feldern lässt man einen kleinen Abstand.  So entsteht ein Raster.
 
Rastersysteme sind als Grundlage der grafischen Gestaltung erst in den 40er Jahren des 20er Jahrhundert aufgekommen. Eine der ersten Verfechter der Rasterung ist der bekannte Designer Josef Müller-Brockmann.

Text und Bilder werden innerhalb der Rasterfelder positioniert. Das bedeutet aber nicht, dass alles gleich aussieht, denn:
\begin{itemize}
\item Die Anzahl der Rasterfelder ist nicht im Voraus vorgegeben. Es kann u.U. viele kleine Rasterfelder geben.
\item Text und Bilder können sich über mehrere Felder strecken.
\item Die Rasterfelder können untereinander unterschiedlich groß sein.
\end{itemize}
\subsubsection{Rasterung}
\slides{04_Begriffe-KG_Struktur-Programm}{22}
\slides{04_Begriffe-KG_Struktur-Programm}{23}
\slides{04_Begriffe-KG_Struktur-Programm}{24}
\slides{04_Begriffe-KG_Struktur-Programm}{25}
\subsubsection{Funktionen der Rasterung}
\slides{04_Begriffe-KG_Struktur-Programm}{26}
\slides{04_Begriffe-KG_Struktur-Programm}{27}
Im rechten Beispiel ist die Rasterung um 45 Grad gedreht.
\slides{04_Begriffe-KG_Struktur-Programm}{28} 

\subsection{Fehler}
\slides{04_Begriffe-KG_Struktur-Programm}{29}

\subsection{Beispiele}
\slides{04_Begriffe-KG_Struktur-Programm}{30}
Auf dieser Homepage wiederholen sich die runde Form des Logos und die Papiermetapher (besonders gekonnt bei der Postkarte unter „Kontakt“). Auch Farben und Schriften wiederholen sich. Das Design-Programm ist durch die Wiederholungen nicht zu übersehen, jedoch nicht langweilig.
\slides{04_Begriffe-KG_Struktur-Programm}{31}
\slides{04_Begriffe-KG_Struktur-Programm}{32}
\slides{04_Begriffe-KG_Struktur-Programm}{33}
\slides{04_Begriffe-KG_Struktur-Programm}{34}
Auf dieser Homepage wiederholen sich untern anderen die Schrift und deren Farbe.

Darunter liegt aber auch ein Rastersystem, das entdeckt werden kann, wenn alle Seiten übereinander positioniert werden. Hier zeigt sich, wie Rasterung dabei hilft ein Designprogramm zu entwickeln – ohne jedoch langweilig zu wirken.
\slides{04_Begriffe-KG_Struktur-Programm}{35}
\slides{04_Begriffe-KG_Struktur-Programm}{36}
\slides{04_Begriffe-KG_Struktur-Programm}{37}
\slides{04_Begriffe-KG_Struktur-Programm}{38}


\section{Bild und Darstellung}
\slides{05_Begriffe-KG_Bild-Stil}{5}
\subsection{Bedeutung und Funktion}
\subsubsection*{Bedeutung}
\emph{Bildliche Darstellung ist unerlässlich} für die Gestaltung interaktiver Systeme: ohne Bilder existier ein „Graphical User Interface“ nicht.

Bilder stehen in einer GUI nicht für sich selbst, sondern für etwas, für das Objekt und die Funktion, die sie darstellen.

Der Betrachter muss das Bild entziffern, dabei sind zwei Leistungen notwendig: zunächst muss der Betrachter erkennen, was das Bild zeigt; anschließend muss der Betrachter verstehen, was das Gezeigte bedeutet.
 
\subsubsection*{Funktion}
Bilder sind wichtig für die \emph{Kommunikation}sfunktion des Designs, weil Bilder Bedeutung vermitteln (nach dem Motto „ein Bild sagt mehr als tausend Worte“). Außerdem werden Bilder eher international verstanden als Sprache.
Mit Bildern kann man einem Design mit einem speziellen Ausdruck und Persönlichkeit ausstatten, so dass es attraktiv und unverwechselbar ist und positive \emph{Emotionen} weckt.

\subsection{Prinzipien}
\subsubsection*{Charakterisierung (Characterization)}
Das Bild muss die wesentlichen Eigenschaften eines Objektes oder Begriffes treffen und dafür die richtige Perspektive finden, damit das Objekt zur Geltung kommt und richtig charakterisiert wird.
\slides{05_Begriffe-KG_Bild-Stil}{8}
Beispiel: Ein Stuhl von oben gesehen (links) ist fast nur ein Viereck, in dieser Perspektive ist der Stuhl nicht zu erkennen. In einer seitlichen Perspektive (rechts) lassen sich die charakteristische Merkmale eines Stuhls am besten zeigen.


\subsubsection*{Allgemeinheit (Generality)}
Dieses Prinzip hat mit der Idee der Charakterisierung zu tun und geht darüber hinaus. Das Objekt muss treffend und zusätzlich allgemein gültig charakterisiert werden.

Man versteht dieses Prinzip sehr gut am Beispiel: das Bild eines Druckers, das die Druckfunktion in einem DTP-Programm darstellen soll, steht nicht für einen bestimmten Drucker (etwa der Drucker in meinem Büro) sondern für ALLE Drucker der Welt, muss also allgemein sein. 
\slides{05_Begriffe-KG_Bild-Stil}{9}
Noch ein Beispiel: das Bild eines Zeugnis (in diesem Beispiel im Steuerungsbereich eines Lernprogramms unter der Glühbirne zu sehen) steht für alle Zeugnisse der Welt. In diesem Fall ähnelt das Piktogramm aber sehr einem bestimmten Zeugnis, das die Autorin in ihrer Schulzeit bekommen hatte; sie hat das Prinzip der Allgemeinheit nicht 100\% beachtet.


\subsubsection*{Kommunikation (Communicability)}
Ein Bild muss erfolgreich eine Bedeutung vermitteln, dafür ist nicht nur das Bild selbst entscheidend sondern auch der Kontext, in dem das Bild gedeutet wird. D.h. Designer und Betrachter müssen bestimmten Kenntnisse und Einstellungen teilen, ohne die das Bild nicht verstanden werden kann; der Designer muss den Kontext und die Vorkenntnisse des Betrachters kennen.
\slides{05_Begriffe-KG_Bild-Stil}{10}
Beispiel: ein typisch amerikanischer und ein europäischer Briefkasten sind sehr unterschiedlich, das muss bei der Gestaltung eines entsprechenden Piktogramms beachtet werden.

\subsubsection*{Unmittelbarkeit (Immediacy)}
„Last but not least“ ein gutes Bild prägt sich unmittelbar ein, es “fällt ins Auge”, man kann sich dem nicht entziehen. Das Bild spricht den Betrachter direkt an und vermittelt ohne Umwege eine klare Bedeutung. 

\subsection{Techniken}
\subsubsection*{Das richtige Medium wählen (Selecting the Right Vehicle)}
Ein Bild ist nicht immer besser als ein Wort. Bilder sind für die Vermittlung von Bedeutung geeignet, wenn der Begriff, der kommuniziert werden soll, ein konkretes bekanntes Objekt oder Icon ist. Wenn ein Begriff schwierig ist aber wiederholt kommuniziert werden soll, kann auch ein grafisches Symbol etabliert werden (z.B. das rote X für „Schließen“).
In allen anderen Fällen ist aber Text die bessere Wahl und das richtige Medium.
 
\subsubsection*{Verfeinerung durch zunehmende Abstraktion (Refinement Through Progressive Abstraction)}
Abstraktion bedeutet, die Eigenschaften eines Objektes von seiner konkreten physischen Realisierung zu trennen. Das Ziel ist die Botschaft auf das Wesentliche zu reduzieren. Der Designer sollte sich ein typischer Vertreter des Objektes besorgen, mehrere einfache Zeichnungen erstellen, komplexere Details vereinfachen und nach und nach alle Informationen entfernen, die für das Objekt nicht allgemein typisch und für die Kommunikation nicht wesentlich sind.
 
\subsubsection*{Koordination (Coordination)}
Eine Gruppe von Bildern, die zusammenarbeiten sollen, müssen dieselbe Bildsprache teilen: Abstraktionsgrad, Größe, Farbigkeit, Perspektive etc.

\subsubsection{Beispiel}
\slides{05_Begriffe-KG_Bild-Stil}{12}
Ein erster Entwurf der OPAL-Piktogramme  für das AnOpel-Projekt, 2011

Die Piktogramme für Drucker und Programm-Schließen (oben rechts) haben unnötig viele Details (sie sind also nicht verfeinert) und sind in 3D dargestellt. Sie passen nicht zu den 2-dimensionalen Piktogrammen im Contentbereich (Koordination ist auch verletzt). 

\slides{05_Begriffe-KG_Bild-Stil}{13}
Re-Design

Die Piktogramme für Drucker und Programm-Schließen (rechts oben) sind jetzt viel einfacher (Verfeinerung) und mit den anderen Bildern besser koordiniert, da alle frontal und fast flach gezeigt werden. Außerdem werden sie mit einem Text erläutert, so dass die Bedeutung nicht fehlt interpretiert werden kann („Das richtige Medium wählen“).

\slides{05_Begriffe-KG_Bild-Stil}{14}
Die neue Piktogramme sind weiter abstrahiert und verfeiner worden.

\subsection{Fehler}
\slides{05_Begriffe-KG_Bild-Stil}{15}

\subsection{Beispiele}
\slides{05_Begriffe-KG_Bild-Stil}{16}
Beispiel für Piktogramme im Lern- und Übungsprogramm  „Bürokaufmann, Bürokauffrau“  
\begin{itemize}
\item Ein Fragezeichen für Üben
\item Eine Glühbirne für Lösung
\item Ein Zeugnis für Lernstand
\item Ein blaues Buch für Lexikon
\item Ein gelbes Notizzettel für Notiz zur Seite
\item Ein grünes Heft für allgemeine Notizen
\item Ein Taschenrechner für Taschenrechner
\item Ein Megafon für Lautstärke
\end{itemize}
\slides{05_Begriffe-KG_Bild-Stil}{17}
Beispiel für Piktogramm in Google-Kalender

Die Bilder werden durch Text ergänzt. Ein schöner Einfall ist das Piktogramm für Terminerinnerung: eine Geburtstagstorte.
\slides{05_Begriffe-KG_Bild-Stil}{18}
Weitere Google-Piktogramme

Einfallsreich ist die Holz-Kiste für den Speicherplatz.
\slides{05_Begriffe-KG_Bild-Stil}{19}
Google-Piktogramme im aktuelleren „Flat Design“. Die M für Google-Mail  ist mittlerweile etabliert.
\slides{05_Begriffe-KG_Bild-Stil}{20}
Direkt, unverwechselbar, einprägsam und verwandelbar: das Logo von Apple.
\slides{05_Begriffe-KG_Bild-Stil}{21}
\slides{05_Begriffe-KG_Bild-Stil}{22}
\slides{05_Begriffe-KG_Bild-Stil}{23}
\slides{05_Begriffe-KG_Bild-Stil}{24}
Direkt, unverwechselbar, einprägsam und verwandelbar: das Logo der Deutschen Bank. Es repräsentiert ein abstrahiertes Prozent-Zeichen, das für den Zins steht. 
\slides{05_Begriffe-KG_Bild-Stil}{25}
\slides{05_Begriffe-KG_Bild-Stil}{26}
\slides{05_Begriffe-KG_Bild-Stil}{27}
\slides{05_Begriffe-KG_Bild-Stil}{28}
Direkt, unverwechselbar, einprägsam und verwandelbar: das Logo von Mercedes Benz, das Stern und Lenkrad auf genialer Art vereinigt. 
\slides{05_Begriffe-KG_Bild-Stil}{29}
\slides{05_Begriffe-KG_Bild-Stil}{30}
\slides{05_Begriffe-KG_Bild-Stil}{31}
\slides{05_Begriffe-KG_Bild-Stil}{32}
Direkt, unverwechselbar, einprägsam und verwandelbar: das Logo von Chanel.

Übrigens funktioniert hier der Buchstabe C als Bild und zwar durch die Verdoppelung und die Spiegelung, die aus dem C ein perfekt symmetrisches Bild machen (ähnlich wie im Abba-Logo das allerdings nicht ganz so raffiniert wirkt).
\slides{05_Begriffe-KG_Bild-Stil}{33}
\slides{05_Begriffe-KG_Bild-Stil}{34}
\slides{05_Begriffe-KG_Bild-Stil}{35}

\section{Stil}
\slides{05_Begriffe-KG_Bild-Stil}{37}

\subsection{Bedeutung und Funktion}
\subsubsection*{Bedeutung}
Derselbe Inhalt unter Einhaltung (oder ggf. bewusste und intelligente Veränderung) der Designtechniken kann auf verschiedenen Weisen gestaltet werden: der Designer steht immer vor der Wahl einer bestimmten Stilrichtung. 
Beispiel: Modernes Haus mit viel Glass, Landhausstil, Bauernhof ...

Stil hat mit der Kultur und mit den Werten zu tun, die die Gestaltung darstellen soll. Ein Stil ist eine optische Sprache, die die ästhetische, intellektuelle  sogar die moralische \emph{Werte der Kultur wiederspiegelt}, in der er entstanden ist.

Angesagte Stile ändern sich, denn der Mensch braucht das Neue, das Unbekannte. Deswegen wird auch Originalität im Design und in der Kunst hoch geschätzt. 
 
\subsubsection*{Funktion}
Stil ist eine Art sich auszudrucken, Werte und Gefühle zu vermitteln, Menschen anzusprechen, die sich im selben kulturellen Umfeld wohl fühlen sollen (emotionelle Funktion).

\subsection{Prinzipien}
\subsubsection*{Einmaligkeit (Distinctiveness)}
Ein Stil sollte neu, einmalig sein, in jeder Variante leicht wiederzuerkennen. Er muss für ein kulturelles Umfeld relevant und anregend sein. Der Erfolg eines Designs hängt nicht nur vom „look“ ab, sondern auch von „feel“. 
 
\subsubsection*{Integrität, Konsistenz (Integrity)}
Ein Stil muss konsistent sein, sich in den verschiedenen Situationen und Anwendungsbereichen treu bleiben. Aber er muss auch der Kultur treu sein, in der er entsteht. Treue kann auch bedeuten, die Gesellschaft weiter zu bringen, Neues zu entdecken. Treue und Ehrlichkeit sind wichtige Eigenschaften eines gelungenen Stils
 
\subsubsection*{Umfassend (Comprenhensiveness)}
Ein Stil muss sich auf verschiedene Situationen erweitern lassen, vielfältig sein. Er spiegelt sich in einer Schrift, aber auch in Farben, Formen ja in einem ganzen Lebensgefühl.


\subsection{Techniken}

Es gibt eigentlich keine Techniken um einen neuen Stil zu erschaffen, denn dies ist ein kreativer Prozess, der sehr eng mit der Persönlichkeit des Designers zu tun hat und mit seiner Art, das Leben und die Gesellschaft zu verstehen. Es gibt nur Techniken, um ein gegebenen Stil anzuwenden.\bigskip\\
\emph{Den Stil meistern}\\
setzt voraus, dass man die Style-Guide liest, bearbeitet, versteht und respektiert.
\emph{Den Stil erweitern}\\
Man muss besondere Achtung den Elementen schenken, die in der Style-Guide nicht direkt berücksichtigt wurden. Wie kann man die Stil-Philosophie weiter entwickeln, so dass sie sich an die neuen Situation optimal anpasst?

\subsection{Fehler}
\slides{05_Begriffe-KG_Bild-Stil}{37}

\subsection{Beispiele}
\slides{05_Begriffe-KG_Bild-Stil}{42}
Diese Homepage will gesellschaftlich unangepasst und alternativ wirken.
\slides{05_Begriffe-KG_Bild-Stil}{43}
Leo Burnett und Google bestechen durch Minimalismus und Leichtigkeit.
\slides{05_Begriffe-KG_Bild-Stil}{44}
\slides{05_Begriffe-KG_Bild-Stil}{45}
Die Google-Darstellung ist noch einfacher geworden!
\slides{05_Begriffe-KG_Bild-Stil}{46}
\slides{05_Begriffe-KG_Bild-Stil}{47}
Die Photo-Agentur Magnum will durch die Wahl der grafischen Mittel (insbesondere aber nicht nur die dunkle Farben) „cool“ und künstlerisch wirken.
\slides{05_Begriffe-KG_Bild-Stil}{48}















