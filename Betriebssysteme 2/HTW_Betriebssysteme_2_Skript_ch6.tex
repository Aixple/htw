\lecdate{28.06.2017}

\section{Grundbegriff}
\subsection{Ziele der Systemsicherheit}
\slides{07-security}{3}
\subsection{Bedrohungen}
\slides{07-security}{4}

\section{Bösartige Software}
\subsection{Überblick}
\slides{07-security}{5}
\begin{itemize}
\item lokaler Angriff (insider Angriff): Nutzer hat bereits Zugang zu System und versucht bspw. \lstinline`root` zu werden.
\item entfernter Angriff: Zugriff von außen (kein Account auf angegriffenen System vorhanden).
\item on-line-Angriff: Hacker ist verbunden ist mit angegriffenen System und „hackt“ (selten) [auch lokal möglich].
\item off-line-Angriff: bspw. durchforsten von Passwort-Dateien (die erklaut wurden).
\end{itemize}

\subsection{Logische Bomben}
\slides{07-security}{6}
\subsubsection*{Ausschnitt McAfee Aktivierungskalender}
\slides{07-security}{7}

\subsection{Hintertüren (Backdoors)}
\slides{07-security}{8}
\subsubsection*{Beispiel 1}
\slides{07-security}{9}
\subsubsection*{Beispiel 2}
\slides{07-security}{10}
Bei \lstinline`current->uid = 0` wird nicht verglichen, sondern \lstinline`uid` gesetzt! Wenn also die angegeben Option-Flags gesetzte werden, wird der ausführende Nutzer \lstinline`root`.
\slides{07-security}{11}

\subsection{Trojanisches Pferd}
\slides{07-security}{12}
\subsubsection*{Beispiel Unix}
\slides{07-security}{17}
\lstinline`u+s`: Setzen des Sticky-Bits (Programm läuft unter rechten des Eigentümers der Datei, nicht unter den rechten des Ausführenden)

\subsection{(Computer-)Viren}
\slides{07-security}{18}
\subsubsection*{Einfaches Beispiel}
\slides{07-security}{23}
\subsubsection*{Erweitertes Beispiel}
\slides{07-security}{24}
$\to$ keine Mehrfachinfektion mehr

\subsection{Würmer}
\slides{07-security}{25}
\subsubsection*{Komponenten eines Wurms}
\slides{07-security}{26}
\slides{07-security}{27}
Vokabeln:
\begin{itemize}
\item Vulnerability: Schwachstelle
\item Exploit: Ausnutzen einer Schwachstelle
\end{itemize}

\subsection{Rootkits}
\slides{07-security}{28}
Hinweis: Rootkit ist nicht dafür da, root zu werden, sondern das Schadprogramm zu verbergen.
\subsubsection*{Arten von Rootkits}
\slides{07-security}{29}
\subsubsection*{Gegenmaßnahmen}
\slides{07-security}{30}

\section{Authentifizierungsmechanismen}
\lecdate{05.07.2017}
\slides{07-security}{31}
Athentifizierung: Identifikation von einem Nutzer durch einen Host-Rechner.

\subsection{Angriffe auf den Authorisierungsvorgang}
\slides{07-security}{32}
\subsubsection{Erraten des Passworts}
\slides{07-security}{33}
\subsubsection*{Beispiel /var/log/auth.log}
\slides{07-security}{34}
\subsubsection{Wörterbuchangriff}
\slides{07-security}{35}
\subsubsection{Erschweren des Wörterbuchangriffs mittels Salz}
\slides{07-security}{36}
\subsubsection*{Beispiel: Bibliotheksfunktion crypt()}
\slides{07-security}{37}
\subsubsection*{Weitere Gegenmaßnahmen}
\slides{07-security}{38}

\subsection{Challenge-Response zur Authentifizierung}
\slides{07-security}{39}

\subsection{Beispiel: Authentifizierung in Windows}
\slides{07-security}{40}

\subsection{Sicherheit von NTLM}
\slides{07-security}{41}

\subsection{Authentifizierung mit physischen Objekten}
\slides{07-security}{42}

\section{Angriffstechniken}
\begin{itemize}
\item Heap Overflow (nicht regulär auszunutzen)
\item Integer Overflow (nicht regulär auszunutzen)
\item Stack Overflow: spezielle Form eines Buffer Overflows
\item Return-into-Libc-Exploit: Libc (return-)Aufruf umleiten
\item Return-Oriented-Programming (ROP)
\item Format-String-Attacke: Versuche \lstinline`printf` oder vergleichbares auszunutzen
\end{itemize}

\subsection{Buffer Overflow}
\slides{07-security}{44}
Sehr einfaches verwundbares Programm:\\
\lstinline`strcopy` interessiert sich nicht für Buffergröße $\to$ gesamtes \lstinline`argv[1]` wird in Speicher geschrieben. Dadurch kann die Rücksprungadresse überschrieben werden. Damit kann Schadcode ausgeführt werden (das wird von modernen Betriebssystemen allerdings verhindert):

\subsubsection{Prinzip}
\slides{07-security}{45}

\subsubsection{Ausschnitt des Stacks}
\slides{07-security}{46}
SFP: Saved Frame Pointer\\
<Ret>: Rückkehradresse
%\slides{07-security}{47}
%\slides{07-security}{48}
\slides{07-security}{49}
\slides{07-security}{50}
Return kehrt zu eigenem Code im Puffer zurück.

\subsubsection{Einfache Gegenmaßnahmen}
\slides{07-security}{51}

\subsubsection{Stackguard}
\slides{07-security}{52}
\subsubsection*{Wahl des Canary Words}
\slides{07-security}{53}
Canary word verhindert, dass es vom Angreifer eingegeben und kopiert werden kann -- wenn doch, dann würde beim Terminator Canary die Eingabe abbrechen.
\subsubsection*{Grenzen des Konzepts}
\slides{07-security}{54}

\subsubsection{StackShield}
\slides{07-security}{55}

\subsubsection[Ausführungsverbote beschreibbarer Seiten]{Ausführungsverbote beschreibbarer Seiten ($W \oplus X$)}
\slides{07-security}{56}
\slides{07-security}{58}
\slides{07-security}{59}
\begin{itemize}
\item Text: +X+R -W
\item Data: +R+W -X
\item Stack: +R+W -X
\end{itemize}
\subsubsection*{Adressumsetzung PAE (NX)}
\slides{07-security}{57}

\subsubsection{Address Space Layout Randomization (ASLR)}
\slides{07-security}{60}
Weiterhin negativ: tiefer Eingriff ins System; einige Programme müssen ohne ASLR laufen, weil sie sonst nicht mehr funktionieren (bleiben angreifbar)

\subsection{Return-into-Libc}
\slides{07-security}{61}
\subsubsection*{Stacklayout}
\slides{07-security}{62}
\subsubsection*{Verkettung zweier Libc-Aufrufe}
\slides{07-security}{66}
\slides{07-security}{67}
\subsubsection*{Anmerkungen}
\slides{07-security}{63}
\subsubsection{Bestimmung der Einsprungsadresse (statisch)}
\slides{07-security}{64}
\subsubsection{Dynamische Bestimmung der Einsprungadresse}
\slides{07-security}{65}
\subsubsection{Weitere Techniken}
\slides{07-security}{68}


\subsection{Format String Exploit}
\subsubsection{Funktionalität von printf()}
\slides{07-security}{70}
\subsubsection*{Beispiel}
\slides{07-security}{71}
\subsubsection{Explizite Adressierung von Argumenten}
\slides{07-security}{72}
\subsubsection*{Stacklayout}
\slides{07-security}{73}
\subsubsection{Beispiel für verwundbare Funktion}
\slides{07-security}{74}
\slides{07-security}{75}
%%% TODO: Folien

\subsection{Heap-Overflow}
\slides{07-security}{91}
Beispiel: 
\begin{lstlisting}[language=C]
#include <stdio.h>
#include <stdlib.h>

int main(int argc, char* argv[]){
	char *buf;
	char* num;
	
	buf = malloc(20);
	num = malloc(sizeof(int));
	
	*num = 42;
	printf("num = %d\n", *num);
	
	gets(buf);
	printf("num = %d\n", *num);
}
\end{lstlisting}
Wenn Eingabe zu Lang ist: Num wird überschrieben.
\subsection{Integer-Overflow}
\slides{07-security}{92}
\subsubsection*{Beispiel}
\slides{07-security}{93}

\section{Angriffscode (Shell)}
%%% TODO: Folien