\section{Multiplexverfahren}
\slides{32-multiplex_print}{3}

\subsection{Nutzertrennung}
\subsubsection{FDMA - Frequency Division Multiple Access (Frequenzmultiplex)}
\slides{32-multiplex_print}{4}

\subsubsection{TDMA - Time Division Multiple Access (Zeitmultiplex)}
\slides{32-multiplex_print}{5}
\subsubsection{FTDMA}
\slides{32-multiplex_print}{6}
\subsubsection{OFDMA}
\slides{32-multiplex_print}{7}

\subsubsection{WDM - Wellenlängenmultiplex}
\slides{32-multiplex_print}{8}

\subsubsection{SDMA - Space Division Multiple Access (Raummultiplex)}
\slides{32-multiplex_print}{9}

\subsubsection{CDMA - Code Division Multiple Access (Codemultiplex)}
\slides{32-multiplex_print}{10}

\subsubsection{FHCDMA}
\slides{32-multiplex_print}{11}
\subsubsection*{Beispiel: Bluetooth}
\slides{32-multiplex_print}{12}

\subsubsection*{CDMA-Sender}
\slides{32-multiplex_print}{13}
\subsubsection*{CDMA-Empfänger 1}
\slides{32-multiplex_print}{14}
\subsubsection*{CDMA-Empfänger 2}
\slides{32-multiplex_print}{15}

\subsubsection*{Orthogonale Codes}
\slides{32-multiplex_print}{16}
$H_1=(+1)$\\
$H_{2n}=\mtr{H_n & H_n \\ H_n & -H_n}$\\
also zum Beispiel:\\
$H_2=\begin{pmatrix*}[r]
1 & 1 \\ 1 & -1
\end{pmatrix*}
$\\
$H_4=\begin{pmatrix*}[r]H_2 & H_2 \\ H_2 & -H_2\end{pmatrix*}=\begin{pmatrix*}[r]1 & 1 & 1 & 1\\ 1 & -1 & 1 & -1 \\ 1 & 1 & -1 & -1 \\ 1 & -1 & -1 & 1\end{pmatrix*}$\\
Achtung! Es werden die Werte von $H_n$ eingesetzt, nicht die Matrix selber (Resultat ist also ein $2n \times 2n$ Matrix und keine Matrix mit $4$ Teilmatrizen).

\subsubsection*{GPS als Beispiel für CDMA}
\slides{32-multiplex_print}{17}

\subsection{Duplexverfahren}
\slides{32-multiplex_print}{18}
\subsubsection{Gleichlageverfahren}
\slides{32-multiplex_print}{19}

\subsection{Zusammenfassung}
\slides{32-multiplex_print}{20 }

