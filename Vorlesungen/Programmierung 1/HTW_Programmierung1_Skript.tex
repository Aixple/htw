% Header aus der Vorlage
\documentclass[a4paper,11pt, footheight=26pt
%,twoside
]{scrreprt}
\usepackage[head=23pt]{geometry}	% head=23pt umgeht Fehlerwarnung, dafür größeres "top" in geometry
\geometry{a4paper, top=30mm, bottom=22mm,headsep=10mm, footskip=12mm
, left=20mm, right=20mm
%, inner=27mm, outer=13mm
}

% Zeile 2 (,twoside) und 7 (inner=...) für eine Druckversion (doppelseitig) ent-kommentieren (Rand für Hefter)

\setcounter{secnumdepth}{3}	% zählt auch subsubsection
\setcounter{tocdepth}{3}	% Inhaltsverzeichnis bis in subsubsection

% Input inkl. Umlaute, Silbentrennung
\usepackage[T1]{fontenc}
\usepackage[utf8]{inputenc}
\usepackage[ngerman]{babel}
\usepackage{csquotes}	% Anführungszeichen
\usepackage{eurosym}

% HTW Corporate Design: Arial (Helvetica)
\usepackage{helvet}
\renewcommand{\familydefault}{\sfdefault}

% Style-Aufhübschung
\usepackage{soul, color}	% Kapitälchen, Unterstrichen, Durchgestrichen usw. im Text
\usepackage{scrlayer-scrpage}	% Kopf-/Fußzeile
%\usepackage{titleref}
\usepackage[perpage]{footmisc}	% Fußnotenzählung Seitenweit, nicht Dokumentenweit
\renewcommand*{\thefootnote}{\fnsymbol{footnote}}	% Fußnoten-Symbole anstatt Zahlen
\renewcommand*{\titlepagestyle}{empty} % Keine Seitennummer auf Titelseite

% Mathe usw.
\usepackage{amssymb}
\usepackage[fleqn]{amsmath}	% fleqn: align-Umgebung rechtsbündig
\usepackage{xcolor}
\usepackage{esint}	% Schönere Integrale, \oiint vorhanden
\everymath=\expandafter{\the\everymath\displaystyle}	% Mathe Inhalte werden weniger verkleinert
\usepackage{wasysym}	% mehr Symbole, bspw \lightning
% Auch arcus-Hyperbolicus-Funktionen
\DeclareMathOperator{\arccot}{arccot}
\DeclareMathOperator{\arccosh}{arccosh}
\DeclareMathOperator{\arcsinh}{arcsinh}
\DeclareMathOperator{\arctanh}{arctanh}
\DeclareMathOperator{\arccoth}{arccoth} 
% Mathe in Anführungszeichen:
\newsavebox{\mathbox}\newsavebox{\mathquote}
\makeatletter
\newcommand{\mq}[1]{% \mathquotes{<stuff>}
  \savebox{\mathquote}{\text{"}}% Save quotes
  \savebox{\mathbox}{$\displaystyle #1$}% Save <stuff>
  \raisebox{\dimexpr\ht\mathbox-\ht\mathquote\relax}{"}#1\raisebox{\dimexpr\ht\mathbox-\ht\mathquote\relax}{''}
}
\makeatother

% tikz usw.
\usepackage{tikz}
\usepackage{pgfplots}
\pgfplotsset{compat=1.11}	% Umgeht Fehlermeldung
\usetikzlibrary{graphs}
%\usetikzlibrary{through}	% ???
\usetikzlibrary{arrows}
\usetikzlibrary{arrows.meta}	% Pfeile verändern / vergrößern: \draw[-{>[scale=1.5]}] (-3,5) -> (-3,3);
\usetikzlibrary{automata,positioning} % Zeilenumbruch im Node node[align=center] {Text\\nächste Zeile} automata für Graphen
\usetikzlibrary{matrix}
\usetikzlibrary{patterns}	% Schraffierte Füllung
\tikzstyle{reverseclip}=[insert path={	% Inverser Clip \clip
	(current page.north east) --
	(current page.south east) --
	(current page.south west) --
	(current page.north west) --
	(current page.north east)}
% Nutzen: 
%\begin{tikzpicture}[remember picture]
%\begin{scope}
%\begin{pgfinterruptboundingbox}
%\draw [clip] DIE FLÄCHE, IN DER OBJEKT NICHT ERSCHEINEN SOLL [reverseclip];
%\end{pgfinterruptboundingbox}
%\draw DAS OBJEKT;
%\end{scope}
%\end{tikzpicture}
]	% Achtung: dafür muss doppelt kompliert werden!
\usepackage{graphpap}	% Grid für Graphen
\tikzset{every state/.style={inner sep=2pt, minimum size=2em}}

% Tabular
\usepackage{longtable}	% Große Tabellen über mehrere Seiten
\usepackage{multirow}	% Multirow/-column: \multirow{2[Anzahl der Zeilen]}{*[Format]}{Test[Inhalt]} oder \multicolumn{7[Anzahl der Reihen]}{|c|[Format]}{Test2[Inhalt]}
\renewcommand{\arraystretch}{1.3} % Tabellenlinien nicht zu dicht
\usepackage{colortbl}
\arrayrulecolor{gray}	% heller Tabellenlinien
\usepackage{array}	% für folgende 3 Zeilen (für Spalten fester breite mit entsprechender Ausrichtung):
\newcolumntype{L}[1]{>{\raggedright\let\newline\\\arraybackslash\hspace{0pt}}m{\dimexpr#1\columnwidth-2\tabcolsep-1.5\arrayrulewidth}}
\newcolumntype{C}[1]{>{\centering\let\newline\\\arraybackslash\hspace{0pt}}m{\dimexpr#1\columnwidth-2\tabcolsep-1.5\arrayrulewidth}}
\newcolumntype{R}[1]{>{\raggedleft\let\newline\\\arraybackslash\hspace{0pt}}m{\dimexpr#1\columnwidth-2\tabcolsep-1.5\arrayrulewidth}}

% Nützliches
\usepackage{verbatim}	% u.a. zum auskommentieren via \begin{comment} \end{comment}
\usepackage{tabto}	% Tabs: /tab zum nächsten Tab oder /tabto{.5 \CurrentLineWidth} zur Stelle in der Linie
\NumTabs{6}	% Anzahl von Tabs pro Zeile zum springen
\usepackage{listings} % Source-Code mit Tabs
\usepackage{lstautogobble} 
\usepackage{enumitem}	% Anpassung der enumerates
\setlist[enumerate,1]{label=\arabic*.)}	% global andere Enum-Items
\newenvironment{anumerate}{\begin{enumerate}[label=\alph*.)]}{\end{enumerate}} % Alphabetische Aufzählung
\renewcommand{\labelitemiii}{$\scriptscriptstyle ^\blacklozenge$} % global andere 3. Item-Aufzählungszeichen
\usepackage{letltxmacro} % neue Definiton von Grundbefehlen
% Nutzen:
%\LetLtxMacro{\oldemph}{\emph}
%\renewcommand{\emph}[1]{\oldemph{#1}}

% Einrichtung von lst
\lstset{
basicstyle=\ttfamily, 
mathescape=true, 
%escapeinside=^^, 
autogobble, 
tabsize=2,
basicstyle=\footnotesize\sffamily\color{black},
frame=single,
rulecolor=\color{lightgray},
numbers=left,
numbersep=5pt,
numberstyle=\tiny\color{gray},
commentstyle=\color{gray},
keywordstyle=\color{green},
stringstyle=\color{orange},
morecomment=[l][\color{magenta}]{\#}
%showspaces=false,
showstringspaces=false,
breaklines=true,
literate=%
    {Ö}{{\"O}}1
    {Ä}{{\"A}}1
    {Ü}{{\"U}}1
    {ß}{{\ss}}1
    {ü}{{\"u}}1
    {ä}{{\"a}}1
    {ö}{{\"o}}1
    {~}{{\textasciitilde}}1
}
\usepackage{scrhack} % Fehler umgehen
\def\ContinueLineNumber{\lstset{firstnumber=last}} % vor lstlisting. Zum wechsel zum nicht-kontinuierlichen muss wieder \StartLineAt1 eingegeben werden
\def\StartLineAt#1{\lstset{firstnumber=#1}} % vor lstlisting \StartLineAt30 eingeben, um bei Zeile 30 zu starten
\let\numberLineAt\StartLineAt

% BibTeX
\usepackage[backend=bibtex, bibencoding=ascii]{biblatex}	% BibTeX
\usepackage{makeidx}
%\makeglossary
%\makeindex

% Grafiken
\usepackage{graphicx}
\usepackage{epstopdf}	% eps-Vektorgrafiken einfügen

% pdf-Setup
\usepackage{pdfpages}
\usepackage[bookmarks,%
bookmarksopen=false,% Klappt die Bookmarks in Acrobat aus
colorlinks=true,%
linkcolor=black,%
citecolor=red,%
urlcolor=green,%
]{hyperref}

% Titel, Autor usw. werden vor dem Anfang des Dokuments in einem Rutsch definiert…
\newcommand{\DTitel}[1]{\newcommand{\Dokumententitel}{#1}}
\newcommand{\DUntertitel}[1]{\newcommand{\Dokumentenuntertitel}{#1}}
\newcommand{\DAutor}[1]{\newcommand{\Dokumentenautor}{#1}}
\newcommand{\DNotiz}[1]{\newcommand{\Dokumentennotiz}{#1}}
\newcommand{\DSign}[1]{\newcommand{\Dokumentensignatur}{#1}}
\DSign{\footnotesize{\textcolor{darkgray}{Mitschrift von\\ \Dokumentenautor}}}
\newcommand{\Autorformat}[1]{\textcolor{darkgray}{Mitschrift von #1}}
\newcommand{\workingdir}{../}	% Arbeitsordner (in Abhängigkeit vom Master) Standard: LateX_master Ordner liegt im Eltern-Ordner
% … Deswegen folgendes erst Nach Dokumentenbeginn ausführen:
\AtBeginDocument{
	\hypersetup{
		pdfauthor={\Dokumentenautor},
		pdftitle={HTW Dresden | \Dokumententitel - \Dokumentenuntertitel},
	}
	\automark[section]{section}
	\automark*[subsection]{subsection}
	\pagestyle{scrheadings}
	\ihead{\includegraphics[height=1.7em]{\workingdir LaTeX_master/HTW-Logo.eps}}
	\ohead{\Dokumententitel}
	\cfoot{\pagemark}
	\ofoot{\Dokumentensignatur}
	% Titelseite
	\title{\includegraphics[width=0.35\textwidth]{\workingdir LaTeX_master/HTW-Logo.eps}\\\vspace{0.5em}
	\Huge\textbf{\Dokumententitel} \\\vspace*{0,5cm}
	\Large \Dokumentenuntertitel \\\vspace*{4cm}}
	\author{\Autorformat{\Dokumentenautor} \vspace*{1cm}\\\Dokumentennotiz}
}

%% EINFACHE BEFEHLE

% Abkürzungen Mathe
\newcommand{\EE}{\mathbb{E}}
\newcommand{\QQ}{\mathbb{Q}}
\newcommand{\RR}{\mathbb{R}}
\newcommand{\CC}{\mathbb{C}}
\newcommand{\NN}{\mathbb{N}}
\newcommand{\ZZ}{\mathbb{Z}}
\newcommand{\PP}{\mathbb{P}}
\renewcommand{\SS}{\mathbb{S}}
\newcommand{\cA}{\mathcal{A}}
\newcommand{\cB}{\mathcal{B}}
\newcommand{\cC}{\mathcal{C}}
\newcommand{\cD}{\mathcal{D}}
\newcommand{\cE}{\mathcal{E}}
\newcommand{\cF}{\mathcal{F}}
\newcommand{\cG}{\mathcal{G}}
\newcommand{\cH}{\mathcal{H}}
\newcommand{\cI}{\mathcal{I}}
\newcommand{\cJ}{\mathcal{J}}
\newcommand{\cM}{\mathcal{M}}
\newcommand{\cN}{\mathcal{N}}
\newcommand{\cP}{\mathcal{P}}
\newcommand{\cR}{\mathcal{R}}
\newcommand{\cS}{\mathcal{S}}
\newcommand{\cZ}{\mathcal{Z}}
\newcommand{\cL}{\mathcal{L}}
\newcommand{\cT}{\mathcal{T}}
\newcommand{\cU}{\mathcal{U}}
\newcommand{\cV}{\mathcal{V}}
\renewcommand{\phi}{\varphi}
\renewcommand{\epsilon}{\varepsilon}

% Farbdefinitionen
\definecolor{red}{RGB}{180,0,0}
\definecolor{green}{RGB}{75,160,0}
\definecolor{blue}{RGB}{0,75,200}
\definecolor{orange}{RGB}{255,128,0}
\definecolor{yellow}{RGB}{255,245,0}
\definecolor{purple}{RGB}{75,0,160}
\definecolor{cyan}{RGB}{0,160,160}
\definecolor{brown}{RGB}{120,60,10}

\definecolor{itteny}{RGB}{244,229,0}
\definecolor{ittenyo}{RGB}{253,198,11}
\definecolor{itteno}{RGB}{241,142,28}
\definecolor{ittenor}{RGB}{234,98,31}
\definecolor{ittenr}{RGB}{227,35,34}
\definecolor{ittenrp}{RGB}{196,3,125}
\definecolor{ittenp}{RGB}{109,57,139}
\definecolor{ittenpb}{RGB}{68,78,153}
\definecolor{ittenb}{RGB}{42,113,176}
\definecolor{ittenbg}{RGB}{6,150,187}
\definecolor{itteng}{RGB}{0,142,91}
\definecolor{ittengy}{RGB}{140,187,38}

% Textfarbe ändern
\newcommand{\tred}[1]{\textcolor{red}{#1}}
\newcommand{\tgreen}[1]{\textcolor{green}{#1}}
\newcommand{\tblue}[1]{\textcolor{blue}{#1}}
\newcommand{\torange}[1]{\textcolor{orange}{#1}}
\newcommand{\tyellow}[1]{\textcolor{yellow}{#1}}
\newcommand{\tpurple}[1]{\textcolor{purple}{#1}}
\newcommand{\tcyan}[1]{\textcolor{cyan}{#1}}
\newcommand{\tbrown}[1]{\textcolor{brown}{#1}}

% Umstellen der Tabellen Definition
\newcommand{\mpb}[1][.3]{\begin{minipage}{#1\textwidth}\vspace*{3pt}}
\newcommand{\mpe}{\vspace*{3pt}\end{minipage}}

\newcommand{\resultul}[1]{\underline{\underline{#1}}}
\newcommand{\parskp}{$ $\\}	% new line after paragraph
\newcommand{\corr}{\;\widehat{=}\;}
\newcommand{\mdeg}{^{\circ}}

\newcommand{\nok}[2]{\begin{pmatrix}#1\\#2\end{pmatrix}}	% n über k BESSER: \binom{n}{k}
\newcommand{\mtr}[1]{\begin{pmatrix}#1\end{pmatrix}}	% Matrix
\newcommand{\dtr}[1]{\begin{vmatrix}#1\end{vmatrix}}	% Determinante (Betragsmatrix)
\renewcommand{\vec}[1]{\underline{#1}}	% Vektorschreibweise
\newcommand{\imptnt}[1]{\colorbox{red!30}{#1}}	% Wichtiges
\newcommand{\intd}[1]{\,\mathrm{d}#1}

\lstset{
basicstyle=\footnotesize\sffamily\color{black},
%frame=single,
numbers=left,
numbersep=5pt,
}

% Definition von Titel, Autor usw.
\DTitel{Programmierung I}
\DUntertitel{Vorlesungsskript}
\DAutor{Falk Jonatan Strube}
\DNotiz{Vorlesung von Prof. Dr.-Ing. Beck}

\begin{document}

\pagenumbering{Roman}

\maketitle
\newpage
\tableofcontents
\newpage

\pagenumbering{arabic}
\section*{Hinweise}

Zugelassene Hilfsmittel Klausur: Spickzettel A-4 Blatt, doppelseitig (, man-page c++.com)

\section{Einführung}

Bilde Durchschnitt aus folgender Notenübersicht:

\begin{tabular}{c c}
Index & Note\\
\hline
0&3\\
1&4\\
2&1\\
3&3\\
4&3\\
5&5\\
6&3\\
7&4\\
8&0\\
9&-\\
\end{tabular}

\subsection{Algorithmus}
\begin{enumerate}
\item Lösche Akku $\rightarrow$  2.
\item Lösche Counter $\rightarrow$  3.
\item Gibt es eine Zahl an Stelle Count? 
\begin{itemize}
\item Ja: $\rightarrow$  4.
\item Nein: $\rightarrow$  6.
\end{itemize}
\item Addiere markierte Zahl zu Akku $\rightarrow$  5.
\item Addiere 1 zu Counter $\rightarrow$  3.
\item Dividiere Wert in Akku durch Wert in Counter und speichere Akku $\rightarrow $ 7.
\item Ergebnis: Ausgabe des Akku $\rightarrow$ ENDE
\end{enumerate}

\subsection{Programmablaufplan (PAP)}

\begin{tikzpicture}[scale=0.3]
\draw (-3,6) ellipse (3 and 1) node{Start};
\draw[-latex] (-3,5) -> (-3,3);
\draw  (-7,3) rectangle (1,1) node[pos =.5]{Akk:=0};
\draw[-latex] (-3,1) -> (-3,-1);
\draw  (-7,-1) rectangle (1,-3) node[pos =.5]{Count:=0};
\draw[-latex] (-3,-3) -- (-3,-5);
\draw (-3,-5) -- (-10,-11) -- (-3,-17) -- (4,-11) -- cycle;
\draw  (-3,-11) node[align=center] {Gibt es eine\\ gültige Note an der\\ Stelle Count?};

\draw[-latex] (-3,-17) -- (-3,-19);
\draw (-2,-18) node{ja};
\draw  (-13,-19) rectangle (8,-21) node[pos =.5]{Akku := Akku + Noten[Count]};
\draw[-latex] (-3,-21) -- (-3,-23);
\draw (-10,-23) rectangle (5,-25) node[pos =.5]{Count := Count +1};
\draw[-latex] (-3,-25) -- (-3,-27) -- (10,-27) -- (10, -4) -- (-3,-4);

\draw[-latex] (-10,-11) -- (-19,-11) -- (-19,-13);
\draw (-19,-13) -- (-24,-17) -- (-19,-21) -- (-14,-17) -- cycle;
\draw  (-19,-17) node[align=center] {Count > 0?};
\draw[-latex] (-14,-17) -- (-12,-17);
\draw (-13,-16) node{nein};
\draw (-9,-17) ellipse (3 and 1) node {Fehler};
\draw[-latex] (-19,-21) -- (-19,-23);
\draw (-18,-22) node{ja};
\draw  (-27,-23) rectangle (-11,-25) node[pos =.5]{Akku := Akku/Count};
\draw[-latex] (-19,-25) -- (-19,-27);
\draw (-26,-27) -- (-10,-27) -- (-11,-29) -- (-27,-29) -- cycle;
\draw (-19,-28) node{Ausgabe Akku};
\draw[-latex] (-19,-29) -- (-19,-31);
\draw (-19,-32) ellipse (3 and 1) node{Erfolg};

\end{tikzpicture} 

\subsection[Struktogramm]{Struktogramm / Nassi-Shneiderman-Diagramm}

\begin{tikzpicture} [scale = 0.3]
\draw  (0,0) rectangle (28,2);
\draw (0,1) node[anchor=west]{Akku = 0}; 
\draw  (0,-6) rectangle (28,0);
\draw (0,-1) node[anchor=west]{while (Noten[Count]!=0)};
\draw  (4,-4) rectangle (28,-2);
\draw (4,-3) node[anchor=west]{Akku = Akku + Noten[Count]}; 
\draw  (4,-6) rectangle (28,-4);
\draw (4,-5) node[anchor=west]{Count :=  Count + 1}; 
\draw  (0,-8) rectangle (28,-6);
\draw (0,-7) node[anchor=west]{if (Count>0)}; 
\draw  (0,-12) rectangle (28,-8);
\draw (0,-11) node[anchor=west]{ja}; 
\draw (28,-11) node[anchor=east]{nein};
\draw (0,-8) -- (14,-12) -- (28,-8);
\draw (0,-12) rectangle (14,-16);
\draw (0,-14) node[anchor=west, align=left]{Akku = Akku/Count\\ Ausgabe Akku}; 
\draw (14,-12) rectangle (28,-16);
\draw (14,-14) node[anchor=west, align=left]{Fehler}; 
\end{tikzpicture}

\subsection{Quelltext in C}

\begin{lstlisting}[language=C]
#include <stdio.h>
#include <stdlib.h>

int Noten []={5,2,3,4,5,5,2,3,4,5,0}; //38/10

int main(){
	int Akku=0, Count=0;
	while (Noten[Count]!=0){
		Akku = Akku+Noten[Count];
		Count = Count+1;
	}
	if(Count>0){
		Akku = Akku/Count;
		printf("Durchschnitt: %d\n",Akku);
	} else
		printf("Fehler - Division durch 0\n");
	return 0;
}
\end{lstlisting}

Compilieren durch: 
\begin{lstlisting}
gcc SOURCE.c -0 DESTINATION
\end{lstlisting}

Ergebnis:\\
„10“ … aber: 38/10 = 3,8. Integer im Source-Code $\Rightarrow$ abgerundet

Lösung:
\begin{lstlisting}[language=C]
// Ergebnis mit Runden (innerhalb der if(Count>0)-Klammer)
Akku = Akku*10/Count;
printf("Durchschnitt: %d.%d\n",Akku/10,Akku%10);
\end{lstlisting}

Alternativ:
\begin{lstlisting}[language=C]
// anstatt int Akku=0, Count=0;
double Akku=0;
int Count =0;
\end{lstlisting}

\section{gcc}

gcc Ablauf für eine „hello.c“ Datei.

\begin{enumerate}
\item Pre-Prozessor (hello.c $\rightarrow$ hello.e $\Rightarrow$ gcc -E hello.c > hello.e) \\
Jede Zeile im Quelltext mit \emph{\#} werden hier interpretiert.
\item Compiler (hello.e $\rightarrow$ hello.o $\Rightarrow$ gcc -c hello.c)
\item Linker (hello.o $\rightarrow$ a.out / hello.exe | gcc hello.c $\rightarrow$ a.out $\Rightarrow$ gcc -o hello hello.c [oder auch gcc hello.c -o hello])\\
Bindet Objekt-Datei (xxx.o) mit Librarys zusammen.
\end{enumerate}

\section{Grundlagen von C}

FOLIE „Grundlagen von C“

\subsection{Datentypen}
was in Folien grau markiert ist, kann weggelassen auch werden $\Rightarrow$ „unsigned int i;“ = „unsigned i;“

\paragraph{Bsp.:}

\begin{lstlisting}[language=C]
unsigned int i;	// Variablen-Definition
i = 12;		// Wertzuweisung

printf("Wert von i: %d - Adresse von i: %p\n", i, &i);
// Hinweis: 
// %d - Dezimalwert, 
// %p - Adresswert, 
// &i - Adresse vor Variable
\end{lstlisting}
Erstellung einer Variablen (int i;): \emph{uninitialisierte Variable / Variablen-Definition}\\
Wertbelegung einer Variable während Definition einer Variablen (int i=0;): \emph{Initialisierung}\\
Wertbelegung zu späterem Zeitpunkt (i=2;): \emph{Wertzuweisung}

\subsection{Ausdrücke}
Programmiersprachliche Konstruktion zur Berechnung von Werten.
\subsubsection{Assoziativität}
(Folie Operatoren: Gewichtung der Operatoren von oben nach unten)\medskip\\
Unäre Operatoren (bspw. $-$ (negativ-Zeichen), $++$ (Inkrementierung) oder Klammern(cast))\\
Binäre Operatoren (bspw. $+$, $-$ (Rechenzeichen), $<=$ usw.)
\begin{lstlisting}[language=C]
int i;
long d;

i=(int)d;	// cast: Typwandlung

i++; // Postfixoperator (wird im Rahmen eines groesseren 
			Ausdrucks als letztes ausgefuehrt:)
i=1;
j=6;
k=j+i++; // k=7, i=2
++i; // Praefixoperator (wird im Rahmen eines groesseren 
			Ausdrucks als erstes ausgefuehrt:)
i=1;
j=6;
k=j+ ++i; // k=8, i=2

// Vorsicht! negativ-Bsp, wie ++ nicht zu verwenden ist:
i=2;
printf("%d\n", i++ + ++i);	\\ i= 6
printf("%d\n", i);	\\ i=4
\end{lstlisting}
Bei Division mit ganzen Zahlen wird der Rest abgeschnitten (nicht gerundet)!\medskip\\
Kurzschlussverfahren von Aneinanderkettung von Bedingungen ( i<0 || i<6) $\Rightarrow$ wenn die erste Prüfung wahr ist, wird der Test weiterer Bedingungen abgebrochen (bei \&\& wenn das erste falsch ist).\bigskip\\
\&\& im Vergleich zu \& (\& ist eine Bit-weise Operation): \\
01101100 \& 00001111 = 00001100 bzw. \\
01101100 | 11110000 = 11111100 \smallskip\\
Andere Zeichen: \\
$\wedge$ = XOR\\
$\sim$ = Bit-weise Negation\\
<\! < = shift (nach links) (bsp. i=4; i= i <\! < 2; $\Rightarrow$ i wird 16:\\
00000100 <\! < 2 $\Rightarrow$ 00010000\\
Achtung: bei negativen Zahlen (also Typ signed) bleibt bei Shift an der ersten Stelle das entsprechende Vorzeichenbit.\\
Bsp. für Abarbeitungsreihenfolge der Operatoren:\\
i*= 3+1		// i*(3+1)
\subsection{Anweisungen}
\begin{itemize}
\item Berechnungen
\item Alternative
\item Iteration
\item Sequenz
\end{itemize}
\subsubsection[Ausdrucksanweisung]{Ausdrucksanweisung (Expressionstatement)}
Eine Ausdrucksanweisung besteht aus einem Ausdruck gefolgt von einem Semikolon:
\begin{lstlisting}
<expr_stmnt>:: <expr> ';' .
\end{lstlisting}

Bsp.:
\begin{lstlisting}
printf("%d\n", i);
\end{lstlisting}
Zu Ausdrucksanweisungen gehören:
\begin{itemize}
\item Berechnungen
\item Aufrufe von Funktionen
\end{itemize}

\paragraph{Block}
Konstruktion, die Anweisungen kapselt -- nach außen einzelne Anweilungen enthält
\begin{itemize}
\item Vereinbarungen
\item Anweisungen
\end{itemize}
\begin{lstlisting}
<block>:: '{' { <statement> } '}' .
\end{lstlisting}

Bsp.:
\begin{lstlisting}[language=C]
#include <stdio.h>
#include <stdlib.h>

char vbuf[128];	// Vereinbarung

int main() {
	int i;	// Vereinbarung
	double x;	// Vereinbarung
	fgets(vbuf, 128, stdin);	// Anweisungen ...
	x=atof(vbuf);
	i = 1;
	x=x*10+i;
	printf("x: %lf\n",x);
}
\end{lstlisting}

\subsubsection[Alternativanweisung]{Alternativanweisung (if-statement)}

\begin{lstlisting}
<if-stmnt>:: 'if' '(' <condition> ')' 
				<statement>
			['else' <statement>] .
\end{lstlisting}

Bsp.:
\begin{lstlisting}[language=C]
#include <stdio.h>
#include <stdlib.h>

char vbuf[128];

int main() {
	double x;
	fgets(vbuf, 128, stdin);
	x=atof(vbuf);
	printf("x: %lf\n",x);
	if (x>1) printf("Groesser als 1\n");
	else printf("Kleiner als 1\n");	// optional
	puts("Hier geht es weiter");	
	// " " Strings (Zeichenketten), einzelnes Zeichen: '*'
}
\end{lstlisting}

\subsubsection{Leeranweisung}
\begin{lstlisting}
<empty_stmnt>:: ';'
\end{lstlisting}

\subsubsection[Iteration]{Iteration (Schleife/Loop)}
\paragraph{abweisende Schleife (kopfgesteuert)} while-Schleife
\begin{lstlisting}
<while_statement>:: 'while' '(' <condition> ')' <statement> .
\end{lstlisting}

Beispiel: $e^x$\\
e hoch x = 1 + x/1! + x*x /2! + x*x*x /3! …
\begin{enumerate} [label= \arabic* .]
\item  Summand: $x^0=1$
\item  Summand: $x^1/1! = x$
\item  Summand: $(x^1/1!)*x/2 = x^2/2!$
\item  Summand: $(x^2/2!) + x/3 = x^3/3!$
\end{enumerate}
Vereinfachung der Rechnung (für den Rechner) $\Rightarrow$ Nutzung des vorhergehenden Summanden.
\begin{lstlisting}[language=C]
#include <stdio.h>
#include <stdlib.h>

char vbuf[128];

int main() {
	int i=1;
	double x, y=1.0, summand = 1.0;
	printf("Eingabe von x: ");
	fgets(vbuf, 128, stdin);
	x=atof(vbuf);
	while (summand>0,00005){
		summand = summand *x/i;
		y += summand;
		printf("Summand %d: %lf\n", i, summand);
		i++;
	}
	printf("e^%lf: %lf\n", x, y);
	return 0;
}
\end{lstlisting}

\paragraph{Nicht abweisende Schleife (fußgesteuert)} do-while-Schleife
\begin{lstlisting}
<do_stmnt>:: 'do' <statement> 'while' '(' <condition> ')' ';' .
\end{lstlisting}

Bsp.:
\begin{lstlisting}[language=C]
#include <stdio.h>
#include <stdlib.h>

char vbuf[128];

int main() {
	int i=1;
	double x, y=1.0, summand = 1.0;
	printf("Eingabe von x: ");
	fgets(vbuf, 128, stdin);
	x=atof(vbuf);
	do{
		summand = summand *x/i;
		y += summand;
		printf("Summand %d: %lf\n", i, summand);
		i++;
	} while (summand>0,00005);
	printf("e^%lf: %lf\n", x, y);
	return 0;
}
\end{lstlisting}

\paragraph{for-Schleife}
\begin{lstlisting}
<for-stmnt>:: 'for' '(' <expr>';' <expr>';' <expr> ')' <statement> .
\end{lstlisting}

Bsp.:
\begin{lstlisting}[language=C]
#include <stdio.h>
#include <stdlib.h>

char vbuf[128];

int main() {
	int i=1;
	double x, y=1.0, summand = 1.0;
	printf("Eingabe von x: ");
	fgets(vbuf, 128, stdin);
	x=atof(vbuf);
	for (i=1; 	// Schleifeninitialisierung
	summand > 0.0005; 	// Abbruchbedingung / Condition
	i++){	// Iterationsausdruck
		summand = summand *x/i;
		y += summand;
		printf("Summand %d: %lf\n", i, summand);
	}
	printf("e^%lf: %lf\n", x, y);
	return 0;
}
\end{lstlisting}

Alternativ-Bsp der for-Schleife mit Komma-Operator:
\begin{lstlisting} [language=C]
int main() {
	int i=1;
	double x, y, summand;
	printf("Eingabe von x: ");
	fgets(vbuf, 128, stdin);
	x=atof(vbuf);
	for (i=1, y=1.0, summand=1.0;
	summand > 0.0005;
	summand*=x/i, y+=summand,
		 printf("Summand %d: %lf\n", i, summand), i++){
	}
	printf("e^%lf: %lf\n", x, y);
	return 0;
}
\end{lstlisting}

\paragraph{Verlassen der Schleife} break
\begin{lstlisting}[language=C]
#include <stdio.h>
#include <stdlib.h>

char vbuf[128];

int main() {
	int i=1;
	double x, y=1.0, summand = 1.0;
	printf("Eingabe von x: ");
	fgets(vbuf, 128, stdin);
	x=atof(vbuf);
	while (1){
		summand = summand *x/i;
		y += summand;
		printf("Summand %d: %lf\n", i, summand);
		if (summand<0,00005) break;
		i++;
	}
	printf("e^%lf: %lf\n", x, y);
	return 0;
}
\end{lstlisting}
\emph{break} bezieht sich auf die (von innen nach außen) nächste zu findende Schleife. Also auf die Schleife, in deren \emph{statement} sie vorkommt.

\paragraph{Neuberechnung der Bedingung} continue\\
Verlässt den Schleifenkörper (der eingebettete Anweisung).
\begin{lstlisting}[language=C]
#include <stdio.h>
#include <stdlib.h>

char vbuf[128];

int main() {
	int i=1;
	double x, y=1.0, summand = 1.0;
	printf("Eingabe von x: ");
	fgets(vbuf, 128, stdin);
	x=atof(vbuf);
	while (summand>0,00005){
		summand = summand *x/i;
		y += summand;		
		i++;
		if (summand > 0.00005) continue;
		printf("Summand %d: %lf\n", i-1, summand);
	}
	printf("e^%lf: %lf\n", x, y);
	return 0;
}
\end{lstlisting}
Wenn Summand größer als 0.00005 ist, startet er die Schleife neu. Die printf() wird erst ausgeführt, wenn er kleiner ist (also das letzte mal).

\paragraph{Fallunterscheidung} switch-Anweisung
\begin{lstlisting}[language=C]
switch (i){ // i ist ganzzahliger Ausdruck
	case 1:
		... break;
	case 2:
		... break;
	default:
		...
}
\end{lstlisting}
\begin{lstlisting}[language=C]
#include <stdio.h>
#include <stdlib.h>

char buf[128];

int main(){
	int wota;
	printf("Wochentag (1...7): ");
	fgets(buf, 128, stdin); wota=atoi(buf);
	switch (wota){
		case 1: puts("Montag");			break;
		case 2: puts("Dienstag");		break;
		case 3: puts("Mittwoch");		break;
		case 4: puts("Donnerstag");	break;
		case 5: puts("Freitag");		break;
		case 6: puts("Samstag");		break;
		case 7: puts("Sonntag");		break;
		
		default: puts("Die Woch hat nur 7 Tage!");
	}
	return 0;
}
\end{lstlisting}

\subsection{Zusammenfassendes Beispiel}

\begin{lstlisting} [language=C]
#include <stdio.h>
#include <stdlib.h>

char buf [128];

int main(){
	int result=0;
	char operator=0;
	int value;
	while (operator!=toupper('q')){
		printf("Eingabe Operator: ");
		fgets(buf, 128, stdin);
		operator = buf[0];
		printf("Eingabe Zahl: ");
		fgets(buf, 128, stdin);
		value = atoi(buf);
		switch (operator) {
			case '+':	// Erinnerung: kein "+" - nur '+' für einzelne Zeichenketten
				result += value;
				break;
			case '-':
				result -= value;
				break;
			case '*':
				result *= value;
				break;
			case '/':
				if (value) // bzw. value!=0 - aber !=0 kann in C weggelassen werden
					result /= value;
				else
					puts("Division durch 0 ist nicht erlaubt.");
				break;
			case '%':
				if (value) // bzw. value!=0 - aber !=0 kann in C weggelassen werden
					result %= value;
				else
					puts("Division durch 0 ist nicht erlaubt.");
				break;
			case 'q':
				break;		
			default: 
				printf("unerlaubte Operation %c\n", operator);
		}
		printf("result: %d\n", result);
	}
}
\end{lstlisting}

\subsection{Zeichenketten}
\begin{lstlisting} [language=C]
#include <stdio.h>
#include <stdlib.h>
#include <string.h>

char buf [128];

int main(){
	printf("Eingabe Zeichenkette: ");
	fgets(buf, 128,stdin);
	printf("Len von Str: %d\n", strlen(buf));
	buf[strlen(buf)-1]=0;
	puts(buf);
	while (buf[i]!=0)
		printf("%c", buf[i++]);
	printf("\n");
	return 0;
}
\end{lstlisting}
Bei der Eingabe „Max“ wird bei puts() sowohl die Eingabe der Zeichenkette, als auch die Eingabe der Eingabetaste (neue Zeile) ausgegeben.\\
buf \begin{tabular}{|l | l |l |l |l |l |l |l }
\hline
M & a & x & $\backslash$ n & $\emptyset$ &\; &\;&\;\\
\hline
\end{tabular}\\
mit $\emptyset$: binäre, terminierende Null ( 0000 0000 )\\
Hinweis: der Buffer muss immer noch Platz für „$\backslash$ n“ und „$\emptyset$“ haben, d.h. man hat in einem Buffer der Größe von 128 nur Platz für 126 zeichen.\\
mit \emph{buf[strlen(buf)-1]=0;} wird die Eingabetaste „$\backslash$ n“ raus gelöscht.\bigskip\\

Daraus ergibt sich eine Verbesserung für den Taschenrechner:
\begin{lstlisting} [language=C]
...
	printf("Eingabe Operator /Operand: ");
	fgets(buf, 128, stdin);
	operator = buf[0];
	value = atoi(buf+1);	// Buffer ab der Stelle 0+1: 1
...
\end{lstlisting}





\end{document}