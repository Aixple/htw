% Header aus der Vorlage
\documentclass[a4paper,11pt, footheight=26pt
%,twoside
]{scrreprt}
\usepackage[head=23pt]{geometry}	% head=23pt umgeht Fehlerwarnung, dafür größeres "top" in geometry
\geometry{a4paper, top=30mm, bottom=22mm,headsep=10mm, footskip=12mm
, left=20mm, right=20mm
%, inner=27mm, outer=13mm
}

% Zeile 2 (,twoside) und 7 (inner=...) für eine Druckversion (doppelseitig) ent-kommentieren (Rand für Hefter)

\setcounter{secnumdepth}{3}	% zählt auch subsubsection
\setcounter{tocdepth}{3}	% Inhaltsverzeichnis bis in subsubsection

% Input inkl. Umlaute, Silbentrennung
\usepackage[T1]{fontenc}
\usepackage[utf8]{inputenc}
\usepackage[ngerman]{babel}
\usepackage{csquotes}	% Anführungszeichen
\usepackage{eurosym}

% HTW Corporate Design: Arial (Helvetica)
\usepackage{helvet}
\renewcommand{\familydefault}{\sfdefault}

% Style-Aufhübschung
\usepackage{soul, color}	% Kapitälchen, Unterstrichen, Durchgestrichen usw. im Text
\usepackage{scrlayer-scrpage}	% Kopf-/Fußzeile
%\usepackage{titleref}
\usepackage[perpage]{footmisc}	% Fußnotenzählung Seitenweit, nicht Dokumentenweit
\renewcommand*{\thefootnote}{\fnsymbol{footnote}}	% Fußnoten-Symbole anstatt Zahlen
\renewcommand*{\titlepagestyle}{empty} % Keine Seitennummer auf Titelseite

% Mathe usw.
\usepackage{amssymb}
\usepackage[fleqn]{amsmath}	% fleqn: align-Umgebung rechtsbündig
\usepackage{xcolor}
\usepackage{esint}	% Schönere Integrale, \oiint vorhanden
\everymath=\expandafter{\the\everymath\displaystyle}	% Mathe Inhalte werden weniger verkleinert
\usepackage{wasysym}	% mehr Symbole, bspw \lightning
% Auch arcus-Hyperbolicus-Funktionen
\DeclareMathOperator{\arccot}{arccot}
\DeclareMathOperator{\arccosh}{arccosh}
\DeclareMathOperator{\arcsinh}{arcsinh}
\DeclareMathOperator{\arctanh}{arctanh}
\DeclareMathOperator{\arccoth}{arccoth} 
% Mathe in Anführungszeichen:
\newsavebox{\mathbox}\newsavebox{\mathquote}
\makeatletter
\newcommand{\mq}[1]{% \mathquotes{<stuff>}
  \savebox{\mathquote}{\text{"}}% Save quotes
  \savebox{\mathbox}{$\displaystyle #1$}% Save <stuff>
  \raisebox{\dimexpr\ht\mathbox-\ht\mathquote\relax}{"}#1\raisebox{\dimexpr\ht\mathbox-\ht\mathquote\relax}{''}
}
\makeatother

% tikz usw.
\usepackage{tikz}
\usepackage{pgfplots}
\pgfplotsset{compat=1.11}	% Umgeht Fehlermeldung
\usetikzlibrary{graphs}
%\usetikzlibrary{through}	% ???
\usetikzlibrary{arrows}
\usetikzlibrary{arrows.meta}	% Pfeile verändern / vergrößern: \draw[-{>[scale=1.5]}] (-3,5) -> (-3,3);
\usetikzlibrary{automata,positioning} % Zeilenumbruch im Node node[align=center] {Text\\nächste Zeile} automata für Graphen
\usetikzlibrary{matrix}
\usetikzlibrary{patterns}	% Schraffierte Füllung
\tikzstyle{reverseclip}=[insert path={	% Inverser Clip \clip
	(current page.north east) --
	(current page.south east) --
	(current page.south west) --
	(current page.north west) --
	(current page.north east)}
% Nutzen: 
%\begin{tikzpicture}[remember picture]
%\begin{scope}
%\begin{pgfinterruptboundingbox}
%\draw [clip] DIE FLÄCHE, IN DER OBJEKT NICHT ERSCHEINEN SOLL [reverseclip];
%\end{pgfinterruptboundingbox}
%\draw DAS OBJEKT;
%\end{scope}
%\end{tikzpicture}
]	% Achtung: dafür muss doppelt kompliert werden!
\usepackage{graphpap}	% Grid für Graphen
\tikzset{every state/.style={inner sep=2pt, minimum size=2em}}

% Tabular
\usepackage{longtable}	% Große Tabellen über mehrere Seiten
\usepackage{multirow}	% Multirow/-column: \multirow{2[Anzahl der Zeilen]}{*[Format]}{Test[Inhalt]} oder \multicolumn{7[Anzahl der Reihen]}{|c|[Format]}{Test2[Inhalt]}
\renewcommand{\arraystretch}{1.3} % Tabellenlinien nicht zu dicht
\usepackage{colortbl}
\arrayrulecolor{gray}	% heller Tabellenlinien
\usepackage{array}	% für folgende 3 Zeilen (für Spalten fester breite mit entsprechender Ausrichtung):
\newcolumntype{L}[1]{>{\raggedright\let\newline\\\arraybackslash\hspace{0pt}}m{\dimexpr#1\columnwidth-2\tabcolsep-1.5\arrayrulewidth}}
\newcolumntype{C}[1]{>{\centering\let\newline\\\arraybackslash\hspace{0pt}}m{\dimexpr#1\columnwidth-2\tabcolsep-1.5\arrayrulewidth}}
\newcolumntype{R}[1]{>{\raggedleft\let\newline\\\arraybackslash\hspace{0pt}}m{\dimexpr#1\columnwidth-2\tabcolsep-1.5\arrayrulewidth}}

% Nützliches
\usepackage{verbatim}	% u.a. zum auskommentieren via \begin{comment} \end{comment}
\usepackage{tabto}	% Tabs: /tab zum nächsten Tab oder /tabto{.5 \CurrentLineWidth} zur Stelle in der Linie
\NumTabs{6}	% Anzahl von Tabs pro Zeile zum springen
\usepackage{listings} % Source-Code mit Tabs
\usepackage{lstautogobble} 
\usepackage{enumitem}	% Anpassung der enumerates
\setlist[enumerate,1]{label=\arabic*.)}	% global andere Enum-Items
\newenvironment{anumerate}{\begin{enumerate}[label=\alph*.)]}{\end{enumerate}} % Alphabetische Aufzählung
\renewcommand{\labelitemiii}{$\scriptscriptstyle ^\blacklozenge$} % global andere 3. Item-Aufzählungszeichen
\usepackage{letltxmacro} % neue Definiton von Grundbefehlen
% Nutzen:
%\LetLtxMacro{\oldemph}{\emph}
%\renewcommand{\emph}[1]{\oldemph{#1}}

% Einrichtung von lst
\lstset{
basicstyle=\ttfamily, 
mathescape=true, 
%escapeinside=^^, 
autogobble, 
tabsize=2,
basicstyle=\footnotesize\sffamily\color{black},
frame=single,
rulecolor=\color{lightgray},
numbers=left,
numbersep=5pt,
numberstyle=\tiny\color{gray},
commentstyle=\color{gray},
keywordstyle=\color{green},
stringstyle=\color{orange},
morecomment=[l][\color{magenta}]{\#}
%showspaces=false,
showstringspaces=false,
breaklines=true,
literate=%
    {Ö}{{\"O}}1
    {Ä}{{\"A}}1
    {Ü}{{\"U}}1
    {ß}{{\ss}}1
    {ü}{{\"u}}1
    {ä}{{\"a}}1
    {ö}{{\"o}}1
    {~}{{\textasciitilde}}1
}
\usepackage{scrhack} % Fehler umgehen
\def\ContinueLineNumber{\lstset{firstnumber=last}} % vor lstlisting. Zum wechsel zum nicht-kontinuierlichen muss wieder \StartLineAt1 eingegeben werden
\def\StartLineAt#1{\lstset{firstnumber=#1}} % vor lstlisting \StartLineAt30 eingeben, um bei Zeile 30 zu starten
\let\numberLineAt\StartLineAt

% BibTeX
\usepackage[backend=bibtex, bibencoding=ascii]{biblatex}	% BibTeX
\usepackage{makeidx}
%\makeglossary
%\makeindex

% Grafiken
\usepackage{graphicx}
\usepackage{epstopdf}	% eps-Vektorgrafiken einfügen

% pdf-Setup
\usepackage{pdfpages}
\usepackage[bookmarks,%
bookmarksopen=false,% Klappt die Bookmarks in Acrobat aus
colorlinks=true,%
linkcolor=black,%
citecolor=red,%
urlcolor=green,%
]{hyperref}

% Titel, Autor usw. werden vor dem Anfang des Dokuments in einem Rutsch definiert…
\newcommand{\DTitel}[1]{\newcommand{\Dokumententitel}{#1}}
\newcommand{\DUntertitel}[1]{\newcommand{\Dokumentenuntertitel}{#1}}
\newcommand{\DAutor}[1]{\newcommand{\Dokumentenautor}{#1}}
\newcommand{\DNotiz}[1]{\newcommand{\Dokumentennotiz}{#1}}
\newcommand{\DSign}[1]{\newcommand{\Dokumentensignatur}{#1}}
\DSign{\footnotesize{\textcolor{darkgray}{Mitschrift von\\ \Dokumentenautor}}}
\newcommand{\Autorformat}[1]{\textcolor{darkgray}{Mitschrift von #1}}
\newcommand{\workingdir}{../}	% Arbeitsordner (in Abhängigkeit vom Master) Standard: LateX_master Ordner liegt im Eltern-Ordner
% … Deswegen folgendes erst Nach Dokumentenbeginn ausführen:
\AtBeginDocument{
	\hypersetup{
		pdfauthor={\Dokumentenautor},
		pdftitle={HTW Dresden | \Dokumententitel - \Dokumentenuntertitel},
	}
	\automark[section]{section}
	\automark*[subsection]{subsection}
	\pagestyle{scrheadings}
	\ihead{\includegraphics[height=1.7em]{\workingdir LaTeX_master/HTW-Logo.eps}}
	\ohead{\Dokumententitel}
	\cfoot{\pagemark}
	\ofoot{\Dokumentensignatur}
	% Titelseite
	\title{\includegraphics[width=0.35\textwidth]{\workingdir LaTeX_master/HTW-Logo.eps}\\\vspace{0.5em}
	\Huge\textbf{\Dokumententitel} \\\vspace*{0,5cm}
	\Large \Dokumentenuntertitel \\\vspace*{4cm}}
	\author{\Autorformat{\Dokumentenautor} \vspace*{1cm}\\\Dokumentennotiz}
}

%% EINFACHE BEFEHLE

% Abkürzungen Mathe
\newcommand{\EE}{\mathbb{E}}
\newcommand{\QQ}{\mathbb{Q}}
\newcommand{\RR}{\mathbb{R}}
\newcommand{\CC}{\mathbb{C}}
\newcommand{\NN}{\mathbb{N}}
\newcommand{\ZZ}{\mathbb{Z}}
\newcommand{\PP}{\mathbb{P}}
\renewcommand{\SS}{\mathbb{S}}
\newcommand{\cA}{\mathcal{A}}
\newcommand{\cB}{\mathcal{B}}
\newcommand{\cC}{\mathcal{C}}
\newcommand{\cD}{\mathcal{D}}
\newcommand{\cE}{\mathcal{E}}
\newcommand{\cF}{\mathcal{F}}
\newcommand{\cG}{\mathcal{G}}
\newcommand{\cH}{\mathcal{H}}
\newcommand{\cI}{\mathcal{I}}
\newcommand{\cJ}{\mathcal{J}}
\newcommand{\cM}{\mathcal{M}}
\newcommand{\cN}{\mathcal{N}}
\newcommand{\cP}{\mathcal{P}}
\newcommand{\cR}{\mathcal{R}}
\newcommand{\cS}{\mathcal{S}}
\newcommand{\cZ}{\mathcal{Z}}
\newcommand{\cL}{\mathcal{L}}
\newcommand{\cT}{\mathcal{T}}
\newcommand{\cU}{\mathcal{U}}
\newcommand{\cV}{\mathcal{V}}
\renewcommand{\phi}{\varphi}
\renewcommand{\epsilon}{\varepsilon}

% Farbdefinitionen
\definecolor{red}{RGB}{180,0,0}
\definecolor{green}{RGB}{75,160,0}
\definecolor{blue}{RGB}{0,75,200}
\definecolor{orange}{RGB}{255,128,0}
\definecolor{yellow}{RGB}{255,245,0}
\definecolor{purple}{RGB}{75,0,160}
\definecolor{cyan}{RGB}{0,160,160}
\definecolor{brown}{RGB}{120,60,10}

\definecolor{itteny}{RGB}{244,229,0}
\definecolor{ittenyo}{RGB}{253,198,11}
\definecolor{itteno}{RGB}{241,142,28}
\definecolor{ittenor}{RGB}{234,98,31}
\definecolor{ittenr}{RGB}{227,35,34}
\definecolor{ittenrp}{RGB}{196,3,125}
\definecolor{ittenp}{RGB}{109,57,139}
\definecolor{ittenpb}{RGB}{68,78,153}
\definecolor{ittenb}{RGB}{42,113,176}
\definecolor{ittenbg}{RGB}{6,150,187}
\definecolor{itteng}{RGB}{0,142,91}
\definecolor{ittengy}{RGB}{140,187,38}

% Textfarbe ändern
\newcommand{\tred}[1]{\textcolor{red}{#1}}
\newcommand{\tgreen}[1]{\textcolor{green}{#1}}
\newcommand{\tblue}[1]{\textcolor{blue}{#1}}
\newcommand{\torange}[1]{\textcolor{orange}{#1}}
\newcommand{\tyellow}[1]{\textcolor{yellow}{#1}}
\newcommand{\tpurple}[1]{\textcolor{purple}{#1}}
\newcommand{\tcyan}[1]{\textcolor{cyan}{#1}}
\newcommand{\tbrown}[1]{\textcolor{brown}{#1}}

% Umstellen der Tabellen Definition
\newcommand{\mpb}[1][.3]{\begin{minipage}{#1\textwidth}\vspace*{3pt}}
\newcommand{\mpe}{\vspace*{3pt}\end{minipage}}

\newcommand{\resultul}[1]{\underline{\underline{#1}}}
\newcommand{\parskp}{$ $\\}	% new line after paragraph
\newcommand{\corr}{\;\widehat{=}\;}
\newcommand{\mdeg}{^{\circ}}

\newcommand{\nok}[2]{\begin{pmatrix}#1\\#2\end{pmatrix}}	% n über k BESSER: \binom{n}{k}
\newcommand{\mtr}[1]{\begin{pmatrix}#1\end{pmatrix}}	% Matrix
\newcommand{\dtr}[1]{\begin{vmatrix}#1\end{vmatrix}}	% Determinante (Betragsmatrix)
\renewcommand{\vec}[1]{\underline{#1}}	% Vektorschreibweise
\newcommand{\imptnt}[1]{\colorbox{red!30}{#1}}	% Wichtiges
\newcommand{\intd}[1]{\,\mathrm{d}#1}

\bibliography{../Literatur/HTW_Literatur.bib}

% Definition von Titel, Autor usw.
\DTitel{Programmierung II}
\DUntertitel{Vorlesungsskript}
\DAutor{Falk-Jonatan Strube}
\DNotiz{Vorlesung von Dr. Arnold Beck}

\begin{document}

\maketitle
\newpage
\tableofcontents
\newpage

\section*{Hinweise}
Unterlagen unter:
\begin{lstlisting}[language=bash]
cd /home/rex/fi1/nestler/Programmierung_II_2016/
\end{lstlisting}
\paragraph{Compiler} 
\begin{itemize}
\item Intel i16, i13 (für Linux oder Visual Studio) \url{www.hocomputer.de} (kostenpflichtig)
\item gcc 5.3, 4.85 \url{gcc.gnu.org}
\end{itemize}
Zugriff auf Windows-Programme (Visual Studio 2013) in Linux-Laboren:
\begin{lstlisting}[language=bash]
rdesktop -f its56 # oder its59
\end{lstlisting}
Empfohlene Literatur: Breymann\cite{breymann2009c++}

\chapter{C++}
\section{Ein- und Ausgabe}
(siehe Folie CPP\_01\_stdio)
\begin{lstlisting}[language=C++]
#include <iostream>	// alternativ zu <stdio.h> in C
using namespace std;	// namespace: für bestimmte Abkürzungen (bpsw. cout anstatt std::cout)
// Hinweis: "::" zeigt, dass das davorstehende "static" ist (hier: std)

class integer {}; // class: Vergleichbar mit typedef

int main() {
	integer i0;	// i0: Instanz bzw. Objekt der Klasse integer
	cin.get();	// Eingabe abwarten
}
\end{lstlisting}
(vgl. integer.cpp)
\begin{lstlisting}[language=C++]
#include <iostream>
using namespace std;

class integer { // int - Variable in class verpacken
//	private:    // private ist default
		int i;  // this->i bzw. (*this).i
		// private nur für andere Klassen, andere Instanzen der gleichen Klasse können drauf zugreifen
	public: // wenn nichts steht, wird automatisch das vorherige angenommen. Hier: private
		integer(int i=0):i(i){  // Konstruktor und Defaultkonstruktor
			// i=0 default Wert, wenn keiner Angegeben
			// :i(i) übergebenes i wird dem i der Klasse zugewiesen: i_1(i_2) i_1 ist this->i, und i_2 ist das übergebene i
			cout<<"integer-Objekt i = "<<this->i<<endl; 
		}

		int get(){ return i; }

		void set(int i=0){ this->i = i; }

		// statische Methode: aufrufbar ohne Instanziierung der Klasse
		static integer add(integer i1, integer i2){  // Wertkopien von i1 und i2
			// return integer(i1.i + i2.i); // alternativ und explizit: Konstruktor-Aufruf
			// return erstellt eine Kopie (mit Konstruktur erstellt)!
			// return i1.get()+i2.get();    // Umwandlung int nach integer, Aufruf Konstruktor implizit
			return i1.i + i2.i;
			// i1.i Möglich, da innerhalb der Klasse integer und somit privates i sichtbar
		}
};

auto max(int x, int y) -> int { return x>y ? x : y; }	// Lambda-Funktion
// auto: Rückgabetyp ergibt sich aus dem Kontext bzw. über das "-> int"

template<typename Typ1, typename Typ2>	// Weiterentwicklung Makro: wählt automatisch Typ aus
auto quotient(Typ1 a, Typ2 b) -> decltype(a/b) { return a/b; }

auto main() -> int {
	auto k = 0;        // C++11: da 0 vom Typ int ist auch k vom Typ int
	decltype(k) j = 5; // C++11: da k vom Typ int ist auch j vom Typ int
	char *c = nullptr; //C++ 11: Zeigerliteral
	int *ip = NULL;

	integer i0(5), i1=4;     // 2 (alternative) Initialisiierungen von Objekten
	// i1=4 nur möglich, wenn 1 Parameter gefordert ist.
	cout<<"i0.i = "<<i0.get()<<endl;
	// cout im Vergleich zu printf() typsicher.
	cout<<"i0.i + i0.i = "<<integer::add(i0, i0).get()<<endl; // Aufruf static-Methode add
	integer i3 = integer::add(i0, i0);                        // Initialisierung von i3
	cout<<"i3.i     = "<<i3.get()<<endl;
	i0.set(22);
	cout<<"i0.i     = "<<i0.get()<<endl;
	cout<<"max(3,5) = "<<max(3,5)<<endl;
	cout<<"5/3      = "<<quotient(5, 3)<<endl;
	cout<<"5.0/3.0  = "<<quotient(5.0, 3.0)<<endl;
	cout<<"b / 1    = "<<quotient('b', '1')<<endl;
	cin.get();
}
\end{lstlisting}

(vgl iostream.pdf)
\begin{itemize}
\item nach jeder cin Eingabe: „cin.clear();“, damit Fehler ignoriert werden um weiter cin's abhandeln zu können (vgl. robust\_ea)
\end{itemize}
Einlesen:
\begin{lstlisting}[language=C++]
char sc;
cout << "sc=";
cin >> sc;
cin.clear();	// clear, um die Eingabezeile freizumachen, damit man nicht an Falscher Eingabe hängen bleibt
cin.ignore(INT_MAX, '\n');	// braucht #include <limits.h>
cout << "sc" << dec << (int) sc << endl;

// alternativ:
char vb[128];
cout << "s=";
cin.getline(vb, sizeof(vb), '\n');	// lesen als String, dann wieder umwandeln (liest auch Leerzeichen ein)
sc = atoi(vb);	// braucht #include <cstdlib>

// alternative zu getline:
cin.get(...);	// lässt aber \n im Strom
cin.get();

// alternativ
cin >> setw(sizeof(vb)) >> vb;	// verhindert Überlauf
sc = atoi(vb);

// alternativ
String s;	// braucht #include <string>
size_t ie=0;
cin >> s;
unsigned int ni = stoi(s, &ie, 10);

// alternativ
getline(cin, s, '\n');
double d = stod(s, &ie);

// zum compilieren: g++ p2a1.cpp -std=c++11 -o a.out
\end{lstlisting}

robust\_ea1.cpp:
\begin{lstlisting}[language=C++]
	do {
		cout<<"d = "; cin>>d; 	// einlesen
		if(cin.eof()) break;		// break bei Strg+D oä.
		if(cin.fail() || (cin.peek() != '\n')){		// ist nächstes Zeichen ungültig?
			cin.clear(); cin.ignore(INT_MAX,'\n');	// Strom zurücksetzen und zum \n gehen
			continue;
		}
		break; % Schleife verlassen, wenn korrekte Eingabe
	} while(true);
	
	if(cin.eof()){ cin.clear(); cout<<"eof\n"; }
	else {
		cin.clear(); cin.ignore(INT_MAX,'\n');
		cout<<"Wert d = "<<d<<endl;
	}
	cin.ignore();
\end{lstlisting}

\section{Defaultargumente}
Defaultargumente müssen immer von rechts angefangen definiert sein:
\begin{lstlisting}[language=C++]
myFunc(int i = 5, int j = 7)	// korrekt
myFunc(int i, int j = 7)			// korrekt
myFunc(int i = 5, int j)			// falsch!!!
\end{lstlisting}

\section{Überladen}
overload.pdf

Hinweis: cast auf zwei Möglichkeiten:
\begin{lstlisting}[language=C++]
int i = 5;
double d;
d = (long) i;
d = long(i);
\end{lstlisting}

\section{Typisierte Konstanten}

\section{Referenzen}
referenzen.pdf\\
Ein Speicherplatz wird mit mehreren Variablen-Namen beschrieben.

\section{String0}
\section{String}
\section{const}
const\_mutable.pdf\\
Faustregel: Alle Funktionen, die nichts ändern, immer das const anfügen.
\section{Folge}
Initfolge/Initfolge.pdf
\section{Kopierkonstrukturen}
Kopierkonstruktoren.pdf

\section{Friend}
Zugriff auf private Klassen von außerhalb:
\begin{itemize}
\item Funktionen
\item Alle Methoden anderer Klassen
\item spezifische Methoden anderer Klassen
\end{itemize}
Muss von Klasse mit privater Funktion festgelegt werden. Freundschaft wird nicht „erwidert“, ist nicht reflexiv (bsp. friendreflex).

\section{Vererbung}
vererben.pdf

\section{Virtuelle Funktionen}
\begin{itemize}
\item überschriebene Funktionen 
\item virtuelle Funktionen $\Rightarrow$Späte Bindung
\item rein virtuelle Funktionen\\
eine Klasse mit rein virtuellen Funktionen ist abstrakt
\item VMT virtual method table ABB 124
\end{itemize}
besch1.cpp (Beschäftigte)
\begin{lstlisting}[language=C++]
#include <iostream>
#include <string>
using namespace std;

class Besch{
	private:
		string name;
	public:	
		Besch(string name): name(name){}	// :name(name) --> Member-Initialisierer
		string getName() {return name;}
		void setName(string name) {this->name = name;}
		virtual void display(ostream& os) {os<<name;}	// virtuelle Funktion
		virtual double calc(){return 0.0;}	// normalerweise Geld nicht mit double, sondern ganzzahlig rechnen!
		// wenn Funktion eigentlich nicht benötigt wird: Polymorphes Interface:
		// virtual double calc() = 0; // rein virtuelle Funktion
		// diese rein virtuelle Funktion macht den Datentyp Besch abstract! D.h. davon kann keine Instanz erzeugt werden (wie in der Main passiert)!
};
ostream& operator<<(ostream& os, Besch& b) {b.display(os);}

class Arbeiter:public Besch{	// public: durchsichtige Vererbung
	private:
		int stunden;
		double stdLohn;
	public:
		Arbeiter(string name, double stdLohn):Besch(name){ // :Besch(name) --> Basis-Initialisierer
			this->stdLohn = stdLohn;
		}
		int getStunden() { return stunden; }
		void setStunden(int stunden) { this->stunden = stunden; }	// usw. getter und setter
		void display(ostream& os){ Besch::display(); os<<""<<stdLohn<<" "<<stunden<<" "<<calc();}
		double calc(){ return stdLohn * stunden; }
};
ostream& operator<<(ostream& os, Arbeiter& a) { a.display(os); }

class Angest:public Besch{
	private:
		double gehalt;
	public:
		Angest(string name, double stdLohn):Besch(name){
			this->gehalt = gehalt;
		}
		void display(ostream& os){ Besch::display()<<" "; os<<gehalt<<" ";}
		double calc(){ return gehalt; }
};
ostream& operator<<(ostream& os, Angest& a) { a.display(os); }

class Haendler:public Arbeiter{
	private:
		double prov;
		double ums;
	public:
		Haendler(string name, double stdLohn, double prov):Arbeiter(name, stdLohn){this->prov=prov;};
		void setUms(double ums){this->ums = ums;}
		double calc(){return Arbeiter::calc() ++ ums*prov;}
		void display(ostream &os){Arbeiter::display(os);}
}

int main() {
	Besch* b1 = new Besch("Hans Huckebein");
	b1->display(cout); cout<< " Euro"<<endl;
	cout<< (*b1) << " Euro" << endl;
	cout<< "=============="<<endl;
	Arbeiter* b2 = new Arbeiter("Moriz Lehmann",  8.99);
	b2->setStunden(140);
	b2->display(cout); cout<< " Euro"<<endl;
	cout<< (*b2) << " Euro"<< endl;
	cout<< "=============="<<endl;
	Angest* b3 = new Arbeiter("Friedrich Lempel",  3099.00);
	b3->display(cout); cout<< "Euro"<<endl;
	cout<< (*b3) << " Euro"<< endl;
	cout<< "=============="<<endl
	Haendler* b4 = new Handler("Bang Ohlufson",  15.80, 0.1);
	b4->setStunden(350);
	b4->setUms(12000);
	b4->display(cout); cout<< "Euro"<<endl;
	cout<< (*b3) << " Euro"<< endl;
	cout<< "=============="<<endl
	
	Besch* be[] = {b1,b2,b3,b4};
	for (int i=0; i<4; i++){
		cout<<i<<": "; 
		be[i]->display(cout); cout<<endl;	// gibt 0 für alle zurück, da nur calc() (und display()) vom Typ Besch ausgeführt, da das Array aus Besch besteht! Die Funktion calc() wurde überschrieben (Vgl. Überladen, wenn bei gleichen Namen verschiedene Paramater angegeben sind)!
		// Pointer im Array bestimmt die ausgeführten Funktionen!
		// Lösung: "virtual" als Schlüsselwort vor Funktion (s.o.). Angeben bei erstem Auftauchen in Vererbungshierarchie
		cout<<(*(be[i]))<<endl<<"+++++++++++++++"<<endl;	// geht auch, da im Operator (die ja nicht virtuell gemacht werden kann, weil sie kein Member ist) die virtuelle display-Fkt aufgerufen wird.
	}	
	
	return 0;
}
\end{lstlisting}
Vererbungs Hierarchie:\\
ABB 125\\
Potentielle Mehrfachvererbung (bei Manager):
\begin{itemize}
\item Hat mehrfache Namen (wegen unterschiedlichen Vererbungslinien)
\end{itemize}
Problemlösung: Arbeiter, Angest und Händler von virtual Besch erben lassen (dort, wo es sich aufzweigt):
\begin{lstlisting}[language=C++]
class Arbeiter: virtual public Besch{};
\end{lstlisting}
Hat immer noch Einschränkungen!












\newpage
\printbibliography
\end{document}