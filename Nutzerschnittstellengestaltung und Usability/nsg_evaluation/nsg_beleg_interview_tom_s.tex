\subsection*{Interview-Fragen}
\subsubsection*{Zur Person}
\begin{enumerate}
\item Was ist dein Name?
\interviewText{Tom Schmidtgen}
\item Wie alt bist du?
\interviewText{21}
\item Was arbeitest/studierst du?
\interviewText{Ich studiere Politikwissenschaft an der TU Dresden. }
\item Wie oft bzw. wie lange browst du Webseiten im Internet?
\interviewText{Ich benutze das Internet täglich. Dabei kommen ich bestimmt auf 4 bis 6 Stunden am Tag. Liegt natürlich auch daran, was ich für die Uni zu tun habe.}
\item Auf welchen Seiten bist du regelmäßig unterwegs?
\interviewText{Ich nutze viele sozialie Netzwerke wie Facebook, Twitter oder auch Messenger Dienste. Ich lese aber auch viele Nachrichten. Hier sind meine Quellen die 'DNN', 'SZ-Online, 'FAZ' und 'SPIEGEL-Online'.}
\item Wie schätzt Du Deine Kenntnisse im Umgang mit Webseiten ein?
\interviewText{Ziemlich gut.}
\item Wofür nutzen Du das Internet (außer zum Browsen)?
\interviewText{Ich bestelle Klamotten oder buche meinen Urlaub online.}
\item War Dir diese Webseite im vornherein bekannt? Wenn ja: Woher?
\interviewText{Nein.}
\end{enumerate}

\subsubsection*{Allgemeines}
\begin{enumerate}
\item Wie leicht fiel es Dir dich auf der Seite zu orientieren?
%               -        ~        +  von bis
\interviewScala{ }{ }{ }{ }{ }{6}{ }{schwer}{leicht}
Was viel dir besonders schwer/leicht:
\interviewText{Schwer fällt mir hier eigentlich nichts. [lacht] Die Seite hat drei deutliche Kategorien oder ich sage mal Anlaufpunkte, die man nutzen kann. Zudem sehe ich eine Suchleiste, die man nutzen kann. Sehr Einfach und übersichtlich.}
\item Gab es Probleme beim Bedienen der Website?
%               -        ~        +  von bis
\interviewScala{ }{ }{ }{ }{ }{ }{7}{viele}{wenige}
Anmerkungen:
\interviewText{Keine Anmerkungen.}
\item Konntest Du die Website auch gut auf dem Handy bedienen?
%               -        ~        +  von bis
\interviewScala{ }{ }{ }{ }{ }{ }{7}{schwer}{leicht}
Anmerkungen:
\interviewText{Ich benutze ein iPhone 5s, die Seite passt sich gut an mein Handy an. Es gab keine Probleme. Die Buttons und 'Linkfelder' sind gut zu erkennen und auch anzuklicken.}
\item Würdest Du etwas an der Webseite verändern?
%               -        ~        +  von bis
\interviewScala{1}{ }{ }{ }{ }{ }{ }{viel}{wenig}
Wenn ja, was?
\interviewText{Mir persönlich gefällt das Design nicht. Ich mag dieses sehr matte grün nicht, wenn es denn eins ist. Die Seite ist einfach, was wahrscheinlich bei euch Informatikern auch sein soll. Wenn man aber andere Leute damit auch ansprechen möchte, dann macht es keinen seriösen Eindruck. Ich würde mir hier wahrscheinlich nichts downloaden.}
\item Wie würdest du die Seite beschreiben? Was ist der Zweck der Seite?
\interviewText{Es ist eine Vorstellung von irgendwas. Man auch ein Wichtelsystem nutzen. Dafür wird es aber ziemlich wenig umworben oder darauf aufmerksam gemacht. Es steht da einfach so als Text. }
\item Kennst Du Webseiten, die mit dieser vergleichbar sind? Wenn ja: Wie lauten diese? In wie fern sind sie besser/schlechter/ähnlich gestaltet?
\interviewText{Mir sind Webseiten, die dieser ähneln keine bekannt. Ich nutze auch keine.}
\end{enumerate}

\subsubsection*{Zum Design}
\begin{enumerate}
\item Wie bewertest du das Design der Seite?
\begin{enumerate}
\item Der Gesamteindruck war…
%               -        ~        +  von bis
\interviewScala{ }{2}{ }{ }{ }{ }{ }{schlecht}{gut}
Anmerkungen:
\interviewText{Keine Anmerkungen.}
\item Die Farbwahl und -gestaltung war…
%               -        ~        +  von bis
\interviewScala{ }{2}{ }{ }{ }{ }{ }{schlecht}{gut}
Anmerkungen:
\interviewText{Wie schon gesagt, ich mag die gesamte Aufmachung der Seite nicht wirklich. Das grün gefällt mir auch nicht wirklich.}
\item Die Anordnung der Inhalte war…
%               -        ~        +  von bis
\interviewScala{ }{ }{ }{ }{5}{ }{ }{schlecht}{gut}
Anmerkungen:
\interviewText{Keine Anmerkungen.}
\item Die Präsentation der Webseite war im Bezug auf den Inhalt…
%               -        ~        +  von bis
\interviewScala{ }{ }{ }{ }{ }{6}{ }{unangemessen}{angemessen}
Anmerkungen:
\interviewText{Das was hier vorgestellt wird, davon habe ich keine Ahnung. Daher ist die Webseite auch angemessen für den Inhalt. Nichts spektakuläres.}
\item Die Lesbarkeit der Webseite war…
%               -        ~        +  von bis
\interviewScala{ }{ }{ }{ }{ }{6}{ }{schlecht}{gut}
Anmerkungen:
\interviewText{Keine Anmerkungen.}
\end{enumerate}
\end{enumerate}

\subsubsection*{Automatisches Wichtel System}
\begin{enumerate}
\item Sind Dir bei der Bedienung des Automatisierten Wichtel Systems Besonderheiten aufgefallen, die Du noch nicht erwähnt hast?
\interviewText{Auf einigen Seiten steht ziemlich viel. Ich würde mir das nicht alles durchlesen.}
\end{enumerate}

\subsubsection*{Abschließende Fragen}
\begin{enumerate}
\item Würdest du die Website weiter empfehlen? Warum? Warum nicht?
\interviewText{Für mich persönliche würde ich sie meinen Freunden nicht weiter empfehlen. Sowas brauchen wir als Politikwissenschaftler einfach nicht. Über das Wichelsystem könnte man noch einmal nachdenken, wenn es soweit ist.}
\item Würdest du die Funktionen der Seite erneut nutzen?
\interviewText{Wenn ich mit Freunden einmal Wichteln sollte, könnte man darüber nachdenken.}
\end{enumerate}
