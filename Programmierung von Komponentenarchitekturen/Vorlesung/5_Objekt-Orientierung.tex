\section{Vererbung}
Alle Member der Basisklasse werden quasi vom Compiler in die Subklasse kopiert.

\begin{center}
\href{https://commons.wikimedia.org/w/index.php?curid=6326887}{\includegraphics[width=.9\textwidth]{Vorlesung/a5ObjektOrientierung-img001.png}}
\end{center}

\begin{itemize}
\item In C\# nur einfach Vererbung (genau eine Mutterklasse)
\item dafür beliebige Anzahl von Schnittstellen
\item \emph{base}-Schlüsselwort zum Aufrufen der überschriebenen Basisklassenmember von abgeleiteten Klassen
\item \emph{sealed}-Schlüsselwort verhindert Vererbung von dieser Klasse
\end{itemize}

\section{Der Konstruktor}

\begin{itemize}
\item spezielle Methode die automatisch bei Speicherallokation mit \emph{new} aufgerufen wird
\item kann nicht explizit aufgerufen werden
\item wenn kein Konstruktor angegeben wird ein Standard-Konstruktor zur Verfügung gestellt 
\item spezielle Syntax: kein Rückgabewert, Name = Klassenname
\end{itemize}\bigskip
\begin{itemize}
\item sind mehrere Konstruktoren vorhanden kann man diese nutzen mit \emph{this}
\item auf Konstruktoren der Mutterklasse kann mit \emph{base} zugegriffen werden
\end{itemize}

\paragraph{Beispiel} Was tun die 3 Konstruktoren von Employee? 
\begin{lstlisting}[language={[Sharp]C}]
class Person {
	public string name;
	public Person(string name) {
		// perform some checks here 
		this.name = name;
	}
}

class Employee : Person {
	public int id;
	public Employee() : this("unbekannt", 0) { }
	public Employee(int id) : this("unbekannt", id) { }  
	public Employee(string name, int id) : base(name) {
		// perform some checks here
		this.id = id;
	}
}
\end{lstlisting}

\section{Polymorphie}
Auf einer Referenz der Basisklasse kann die spezifische Methode der Sub-klasse aufgerufen werden.\\
Funktioniert in C\# durch die Schlüsselwörter \lstinline$virtual$ und \lstinline$override$. 
\begin{lstlisting}[language={[Sharp]C}]
class Basisklasse { 
	public virtual void test() { 
		Console.WriteLine("Basis"); 
	} 
} 

class Subklasse : Basisklasse { 
	public override void test() { 
		Console.WriteLine("Sub"); 
	} 
} 

// Aufruf:
Basisklasse b = new Subklasse; 
b.test(); // Ausgabe: Sub
\end{lstlisting}

\section{Schlüsselwort-Optionen für Klassenmethoden}

\begin{itemize}
\item \emph{virtual}: Die Methode kann Inhalt enthalten und kann in einer Abgeleiteten Klasse polymorph überschrieben werden
\item \emph{override}: Muss bei einer überschriebenen Methode angegeben werden um eine virtual Methode der Basisklasse zu überschreiben
\item \emph{sealed}: Eine überschriebene virtuelle Methode darf in Tochterklassen nicht weiter überschrieben werden
\item \emph{abstract}: Die Methode ist leer und muss überschrieben werden
\item \emph{new}: Wenn die zu überschreibende Methode nicht virtual in der Basisklasse ist kann override nicht verwendet werden -{\textgreater} new nötig, Polymorphismus funktioniert dann nicht
\item \emph{static}: Methode muss von der Klasse aus aufgerufen werden, nicht von einer Instanz
\item \emph{const}: Der Rückgabewert kann nicht verändert werden (Aufruf function()=1 geht nicht)
\end{itemize}

\section{Optionen für Klassendeklaration}

\begin{itemize}
\item \emph{static}: Klasse darf nur statische Member enthalten und kann nicht instanziiert werden 
\item \emph{abstract}: wenn eine abstracte Methode enthalten ist muss die Klasse als abstract markiert werden, verhindert Instanziierung
\end{itemize}

\section{Optionen für Variablen}
\begin{itemize}
\item \emph{static}: die Variable oder Klasse existiert einmalig für die Funktion oder Klasse und nicht für eine Instanz)
\item \emph{const}: (die Variable ändert sich nicht)
\begin{itemize}
\item const vor Parametern: Die Parameter können nicht verändert werden
\item const vor Membervariable: Variablenwert muss sofort zugewiesen werden, Variable ist dann automatisch auch static, beide Einschränkungen sind aufgehoben wenn \emph{readonly} statt \emph{const} verwendet wird 
\end{itemize}
\end{itemize}

\section{Zugriffsmodifizierer für Sichtbarkeit}

\begin{tabular}{L{.15} | L{.15} L{.15} L{.15} L{.4}}
& Überall & Bibliothek & abgeleitete Klassen & \\\hline
private & ~ & ~ & ~ & $\to$ Standard in einem Typ\newline(z.B. in einer Klasse)\\
protected & ~ & ~ & ~ & ~\\
internal & ~ & ~ & ~ & ~\\
public & ~ & ~ & ~ & ~\\
\end{tabular}

\section{Casting von Klassen}
\begin{itemize}
\item implicit cast in eine Basisklasse: 
\item explicit cast in Subklasse: 
\item \emph{as} keyword: 
\item \emph{is} keyword: 
\end{itemize}
%%% TODO
\section{Properties (Eigenschaft)}
Property = Member-Variable die in einem Objekt definiert ist und auf die von außen zugegriffen werden soll.\\
Zugriff auf Eigenschaften: \\
\begin{tabular}{L{.5} L{.5}}
C++ Style &  C\#\\
\begin{lstlisting}[language={[Sharp]C}]
private int myVariable;
public int GetMyVariable() {
	return myVariable;
}
private void SetMyVariable(int value) {
	myVariable = value;
}
public void AccessVariables(int a) {
	int b = GetMyVariable(); SetMyVariable(a);
} 
\end{lstlisting} &
%%% TODO
\begin{lstlisting}[language={[Sharp]C}]

\end{lstlisting}\\
\end{tabular}

\begin{itemize}
\item Schreibaufwand reduziert
\item Direkter Zugriff im Aufrufer
\end{itemize}

\subsection{Auto-generated properties}
Will man nur den schreibenden Zugriff auf eine Variable einschränken, kann man noch kürzer schreiben:\\
\lstinline$public int MyVariable {get; private set;}$
\begin{itemize}
\item Dabei wird die zugrundeliegende Variable (vorher myVariable) nicht mehr benutzt
\item Intern wird eine eigene Variable angelegt auf die nun kein direkter Zugriff mehr möglich ist
\end{itemize}

\section{Object initialization Syntax}

Objekte / Felder einer Klasseninstanz können direkt zugewiesen werden, ohne benutzerdefinierten Konstruktor.
\begin{lstlisting}[language={[Sharp]C}]
class Person { 
	public string Nachname { get; set;} 
	public string Vorname { get; set;} 
	public Person() {} 
}

...

Person p1 = new Person() {Nachname = "Hook"}; 
Person p2 = new Person() {Vorname = "Wendy"}; 
Person p3 = new Person() {Vorname = "Peter", Nachname =  "Pan"};
\end{lstlisting}

\section{Anonyme Objekte}

\begin{itemize}
\item Klasse ohne Bezeichner-Namen
\item zur schnellen Gruppierung von Elementen 
\item Der Typ dieser Objekte ist kompliziert -{\textgreater} var keyword benutzen
\end{itemize}