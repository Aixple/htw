\section{Verzögerung}

\subsection{Verzögerung in paketorientierten Netzen}
\slides{12-grundlagen_print}{3}

\subsubsection*{Beispiele}
\slides{12-grundlagen_print}{4}

\subsubsection*{Grafik Verzögerung}
\slides{12-grundlagen_print}{5}
(Hinweis: Interessant für Prüfung)

\subsubsection{Datenrate-Verzögerungsprodukt}
bandwitdh-delay product, BDP (Begriff Bandbreite hier korrekt)
\slides{12-grundlagen_print}{6}
Wichtig: Es gibt einen Unterschied zwischen dem physischen Durchsatz und dem, der tatsächlich nur ankommt. In diesem Fall ist der Unterschied sehr groß: da muss die Fenstergröße größer gewählt werden.

\section{Standardisierung}
\subsection{Standardisierungskommision}
\subsubsection{International}
\slides{12-grundlagen_print}{7}
\subsubsection{National/Europäisch}
\slides{12-grundlagen_print}{8}
\subsection{Standards}
\slides{12-grundlagen_print}{9}

\section{OSI Modell}
\subsection*{ISO OSI 7-Schichtenprotokoll}
\slides{12-grundlagen_print}{10}
OSI: Open System Interconnection
\subsection*{Trennung in Anwendungs- und Transportverbindung}
\slides{12-grundlagen_print}{11}
\subsection{Bitübertragungsschicht (Physical Layer)}
\slides{12-grundlagen_print}{12}
\subsubsection*{Beispiel RS232}
\slides{12-grundlagen_print}{13}
\subsection{Sicherungsschicht (Data Link Layer)}
\slides{12-grundlagen_print}{14}
\subsection{Vermittlungsschicht (Network Layer)}
\slides{12-grundlagen_print}{15}
\subsection{Transportschicht (Transport Layer)}
\slides{12-grundlagen_print}{16}
\subsection{Sitzungsschicht (Session Layer)}
\slides{12-grundlagen_print}{17}
\subsection{Darstellungsschicht (Presentation Layer)}
\slides{12-grundlagen_print}{18}
\subsection{Anwendungsschicht (Application Layer)}
\slides{12-grundlagen_print}{19}

\section{Protokolle und Dienste}
\subsection{Protokolle}
\subsubsection{Protokollbegriff}
\slides{12-grundlagen_print}{20}
\subsubsection{Grundfunktionen in Protokollen}
\slides{12-grundlagen_print}{21}
\subsection{Dienste}
\slides{12-grundlagen_print}{22}
\subsubsection{Begriffe}
\slides{12-grundlagen_print}{23}
\subsection{Dienst und Protokoll}
\slides{12-grundlagen_print}{24}
\subsubsection*{Beispiel: Gespräch der Philosophen}
\slides{12-grundlagen_print}{25}
\begin{tabular}{c c c r}
Chinesischer Philosoph & $\overset{\text{Thema}}{\longleftrightarrow}$ & Spanischer Philosoph\\
$\downarrow$ &  &  $\downarrow$\\
Übersetzung Englisch & $\overset{\text{Englisch}}{\longleftrightarrow}$ & Übersetzung Englisch\\
$\downarrow$ & & $\downarrow$\\
Funker & $\overset{\text{Morse}}{\longleftrightarrow}$ & Funker
\end{tabular}\\
Veranschaulichung: Horizontal ist das Protokoll und vertikal der Dienst.
\subsubsection{Dienstelemente: Funktionsprinzip}
\slides{12-grundlagen_print}{26}
\subsubsection{Protokolldateneinheit / Interfacedateneinheit}
\slides{12-grundlagen_print}{27}
\subsubsection{Zusammenhang: SDU - PCI - PDU - IDU}
\slides{12-grundlagen_print}{28}
\subsubsection{Schichten und PDUs}
\slides{12-grundlagen_print}{29}
\subsubsection{Beispiel für Dienstelemente}
\slides{12-grundlagen_print}{30}
\subsubsection{Zeitdiagramm mit Dienstelementen}
Verbindungsaufbau zu Dienstelementen
\slides{12-grundlagen_print}{31}
\section{Protokollbeschreibungen}
(Möglichkeiten, vgl. Software Engineering)
\slides{12-grundlagen_print}{32}
\subsection{Systementwurf / Detailentwurf}
\slides{12-grundlagen_print}{33}
\subsubsection{TCP-Zustände des Servers}
\slides{12-grundlagen_print}{34}
\subsubsection{TCP-Zustandsmaschine}
\slides{12-grundlagen_print}{35}
\subsection{Zusammenfassung}
\slides{12-grundlagen_print}{36}
\subsubsection{Übung für Dienst-Primitiven (Operationen)}
\slides{12-grundlagen_print}{37}
\slides{12-grundlagen_print}{38}