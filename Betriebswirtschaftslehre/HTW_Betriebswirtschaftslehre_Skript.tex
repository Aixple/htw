% Header aus der Vorlage
\documentclass[a4paper,11pt, footheight=26pt
%,twoside
]{scrreprt}
\usepackage[head=23pt]{geometry}	% head=23pt umgeht Fehlerwarnung, dafür größeres "top" in geometry
\geometry{a4paper, top=30mm, bottom=22mm,headsep=10mm, footskip=12mm
, left=20mm, right=20mm
%, inner=27mm, outer=13mm
}

% Zeile 2 (,twoside) und 7 (inner=...) für eine Druckversion (doppelseitig) ent-kommentieren (Rand für Hefter)

\setcounter{secnumdepth}{3}	% zählt auch subsubsection
\setcounter{tocdepth}{3}	% Inhaltsverzeichnis bis in subsubsection

% Input inkl. Umlaute, Silbentrennung
\usepackage[T1]{fontenc}
\usepackage[utf8]{inputenc}
\usepackage[ngerman]{babel}
\usepackage{csquotes}	% Anführungszeichen
\usepackage{eurosym}

% HTW Corporate Design: Arial (Helvetica)
\usepackage{helvet}
\renewcommand{\familydefault}{\sfdefault}

% Style-Aufhübschung
\usepackage{soul, color}	% Kapitälchen, Unterstrichen, Durchgestrichen usw. im Text
\usepackage{scrlayer-scrpage}	% Kopf-/Fußzeile
%\usepackage{titleref}
\usepackage[perpage]{footmisc}	% Fußnotenzählung Seitenweit, nicht Dokumentenweit
\renewcommand*{\thefootnote}{\fnsymbol{footnote}}	% Fußnoten-Symbole anstatt Zahlen
\renewcommand*{\titlepagestyle}{empty} % Keine Seitennummer auf Titelseite

% Mathe usw.
\usepackage{amssymb}
\usepackage[fleqn]{amsmath}	% fleqn: align-Umgebung rechtsbündig
\usepackage{xcolor}
\usepackage{esint}	% Schönere Integrale, \oiint vorhanden
\everymath=\expandafter{\the\everymath\displaystyle}	% Mathe Inhalte werden weniger verkleinert
\usepackage{wasysym}	% mehr Symbole, bspw \lightning
% Auch arcus-Hyperbolicus-Funktionen
\DeclareMathOperator{\arccot}{arccot}
\DeclareMathOperator{\arccosh}{arccosh}
\DeclareMathOperator{\arcsinh}{arcsinh}
\DeclareMathOperator{\arctanh}{arctanh}
\DeclareMathOperator{\arccoth}{arccoth} 
% Mathe in Anführungszeichen:
\newsavebox{\mathbox}\newsavebox{\mathquote}
\makeatletter
\newcommand{\mq}[1]{% \mathquotes{<stuff>}
  \savebox{\mathquote}{\text{"}}% Save quotes
  \savebox{\mathbox}{$\displaystyle #1$}% Save <stuff>
  \raisebox{\dimexpr\ht\mathbox-\ht\mathquote\relax}{"}#1\raisebox{\dimexpr\ht\mathbox-\ht\mathquote\relax}{''}
}
\makeatother

% tikz usw.
\usepackage{tikz}
\usepackage{pgfplots}
\pgfplotsset{compat=1.11}	% Umgeht Fehlermeldung
\usetikzlibrary{graphs}
%\usetikzlibrary{through}	% ???
\usetikzlibrary{arrows}
\usetikzlibrary{arrows.meta}	% Pfeile verändern / vergrößern: \draw[-{>[scale=1.5]}] (-3,5) -> (-3,3);
\usetikzlibrary{automata,positioning} % Zeilenumbruch im Node node[align=center] {Text\\nächste Zeile} automata für Graphen
\usetikzlibrary{matrix}
\usetikzlibrary{patterns}	% Schraffierte Füllung
\tikzstyle{reverseclip}=[insert path={	% Inverser Clip \clip
	(current page.north east) --
	(current page.south east) --
	(current page.south west) --
	(current page.north west) --
	(current page.north east)}
% Nutzen: 
%\begin{tikzpicture}[remember picture]
%\begin{scope}
%\begin{pgfinterruptboundingbox}
%\draw [clip] DIE FLÄCHE, IN DER OBJEKT NICHT ERSCHEINEN SOLL [reverseclip];
%\end{pgfinterruptboundingbox}
%\draw DAS OBJEKT;
%\end{scope}
%\end{tikzpicture}
]	% Achtung: dafür muss doppelt kompliert werden!
\usepackage{graphpap}	% Grid für Graphen
\tikzset{every state/.style={inner sep=2pt, minimum size=2em}}

% Tabular
\usepackage{longtable}	% Große Tabellen über mehrere Seiten
\usepackage{multirow}	% Multirow/-column: \multirow{2[Anzahl der Zeilen]}{*[Format]}{Test[Inhalt]} oder \multicolumn{7[Anzahl der Reihen]}{|c|[Format]}{Test2[Inhalt]}
\renewcommand{\arraystretch}{1.3} % Tabellenlinien nicht zu dicht
\usepackage{colortbl}
\arrayrulecolor{gray}	% heller Tabellenlinien
\usepackage{array}	% für folgende 3 Zeilen (für Spalten fester breite mit entsprechender Ausrichtung):
\newcolumntype{L}[1]{>{\raggedright\let\newline\\\arraybackslash\hspace{0pt}}m{\dimexpr#1\columnwidth-2\tabcolsep-1.5\arrayrulewidth}}
\newcolumntype{C}[1]{>{\centering\let\newline\\\arraybackslash\hspace{0pt}}m{\dimexpr#1\columnwidth-2\tabcolsep-1.5\arrayrulewidth}}
\newcolumntype{R}[1]{>{\raggedleft\let\newline\\\arraybackslash\hspace{0pt}}m{\dimexpr#1\columnwidth-2\tabcolsep-1.5\arrayrulewidth}}

% Nützliches
\usepackage{verbatim}	% u.a. zum auskommentieren via \begin{comment} \end{comment}
\usepackage{tabto}	% Tabs: /tab zum nächsten Tab oder /tabto{.5 \CurrentLineWidth} zur Stelle in der Linie
\NumTabs{6}	% Anzahl von Tabs pro Zeile zum springen
\usepackage{listings} % Source-Code mit Tabs
\usepackage{lstautogobble} 
\usepackage{enumitem}	% Anpassung der enumerates
\setlist[enumerate,1]{label=\arabic*.)}	% global andere Enum-Items
\newenvironment{anumerate}{\begin{enumerate}[label=\alph*.)]}{\end{enumerate}} % Alphabetische Aufzählung
\renewcommand{\labelitemiii}{$\scriptscriptstyle ^\blacklozenge$} % global andere 3. Item-Aufzählungszeichen
\usepackage{letltxmacro} % neue Definiton von Grundbefehlen
% Nutzen:
%\LetLtxMacro{\oldemph}{\emph}
%\renewcommand{\emph}[1]{\oldemph{#1}}

% Einrichtung von lst
\lstset{
basicstyle=\ttfamily, 
mathescape=true, 
%escapeinside=^^, 
autogobble, 
tabsize=2,
basicstyle=\footnotesize\sffamily\color{black},
frame=single,
rulecolor=\color{lightgray},
numbers=left,
numbersep=5pt,
numberstyle=\tiny\color{gray},
commentstyle=\color{gray},
keywordstyle=\color{green},
stringstyle=\color{orange},
morecomment=[l][\color{magenta}]{\#}
%showspaces=false,
showstringspaces=false,
breaklines=true,
literate=%
    {Ö}{{\"O}}1
    {Ä}{{\"A}}1
    {Ü}{{\"U}}1
    {ß}{{\ss}}1
    {ü}{{\"u}}1
    {ä}{{\"a}}1
    {ö}{{\"o}}1
    {~}{{\textasciitilde}}1
}
\usepackage{scrhack} % Fehler umgehen
\def\ContinueLineNumber{\lstset{firstnumber=last}} % vor lstlisting. Zum wechsel zum nicht-kontinuierlichen muss wieder \StartLineAt1 eingegeben werden
\def\StartLineAt#1{\lstset{firstnumber=#1}} % vor lstlisting \StartLineAt30 eingeben, um bei Zeile 30 zu starten
\let\numberLineAt\StartLineAt

% BibTeX
\usepackage[backend=bibtex, bibencoding=ascii]{biblatex}	% BibTeX
\usepackage{makeidx}
%\makeglossary
%\makeindex

% Grafiken
\usepackage{graphicx}
\usepackage{epstopdf}	% eps-Vektorgrafiken einfügen

% pdf-Setup
\usepackage{pdfpages}
\usepackage[bookmarks,%
bookmarksopen=false,% Klappt die Bookmarks in Acrobat aus
colorlinks=true,%
linkcolor=black,%
citecolor=red,%
urlcolor=green,%
]{hyperref}

% Titel, Autor usw. werden vor dem Anfang des Dokuments in einem Rutsch definiert…
\newcommand{\DTitel}[1]{\newcommand{\Dokumententitel}{#1}}
\newcommand{\DUntertitel}[1]{\newcommand{\Dokumentenuntertitel}{#1}}
\newcommand{\DAutor}[1]{\newcommand{\Dokumentenautor}{#1}}
\newcommand{\DNotiz}[1]{\newcommand{\Dokumentennotiz}{#1}}
\newcommand{\DSign}[1]{\newcommand{\Dokumentensignatur}{#1}}
\DSign{\footnotesize{\textcolor{darkgray}{Mitschrift von\\ \Dokumentenautor}}}
\newcommand{\Autorformat}[1]{\textcolor{darkgray}{Mitschrift von #1}}
\newcommand{\workingdir}{../}	% Arbeitsordner (in Abhängigkeit vom Master) Standard: LateX_master Ordner liegt im Eltern-Ordner
% … Deswegen folgendes erst Nach Dokumentenbeginn ausführen:
\AtBeginDocument{
	\hypersetup{
		pdfauthor={\Dokumentenautor},
		pdftitle={HTW Dresden | \Dokumententitel - \Dokumentenuntertitel},
	}
	\automark[section]{section}
	\automark*[subsection]{subsection}
	\pagestyle{scrheadings}
	\ihead{\includegraphics[height=1.7em]{\workingdir LaTeX_master/HTW-Logo.eps}}
	\ohead{\Dokumententitel}
	\cfoot{\pagemark}
	\ofoot{\Dokumentensignatur}
	% Titelseite
	\title{\includegraphics[width=0.35\textwidth]{\workingdir LaTeX_master/HTW-Logo.eps}\\\vspace{0.5em}
	\Huge\textbf{\Dokumententitel} \\\vspace*{0,5cm}
	\Large \Dokumentenuntertitel \\\vspace*{4cm}}
	\author{\Autorformat{\Dokumentenautor} \vspace*{1cm}\\\Dokumentennotiz}
}

%% EINFACHE BEFEHLE

% Abkürzungen Mathe
\newcommand{\EE}{\mathbb{E}}
\newcommand{\QQ}{\mathbb{Q}}
\newcommand{\RR}{\mathbb{R}}
\newcommand{\CC}{\mathbb{C}}
\newcommand{\NN}{\mathbb{N}}
\newcommand{\ZZ}{\mathbb{Z}}
\newcommand{\PP}{\mathbb{P}}
\renewcommand{\SS}{\mathbb{S}}
\newcommand{\cA}{\mathcal{A}}
\newcommand{\cB}{\mathcal{B}}
\newcommand{\cC}{\mathcal{C}}
\newcommand{\cD}{\mathcal{D}}
\newcommand{\cE}{\mathcal{E}}
\newcommand{\cF}{\mathcal{F}}
\newcommand{\cG}{\mathcal{G}}
\newcommand{\cH}{\mathcal{H}}
\newcommand{\cI}{\mathcal{I}}
\newcommand{\cJ}{\mathcal{J}}
\newcommand{\cM}{\mathcal{M}}
\newcommand{\cN}{\mathcal{N}}
\newcommand{\cP}{\mathcal{P}}
\newcommand{\cR}{\mathcal{R}}
\newcommand{\cS}{\mathcal{S}}
\newcommand{\cZ}{\mathcal{Z}}
\newcommand{\cL}{\mathcal{L}}
\newcommand{\cT}{\mathcal{T}}
\newcommand{\cU}{\mathcal{U}}
\newcommand{\cV}{\mathcal{V}}
\renewcommand{\phi}{\varphi}
\renewcommand{\epsilon}{\varepsilon}

% Farbdefinitionen
\definecolor{red}{RGB}{180,0,0}
\definecolor{green}{RGB}{75,160,0}
\definecolor{blue}{RGB}{0,75,200}
\definecolor{orange}{RGB}{255,128,0}
\definecolor{yellow}{RGB}{255,245,0}
\definecolor{purple}{RGB}{75,0,160}
\definecolor{cyan}{RGB}{0,160,160}
\definecolor{brown}{RGB}{120,60,10}

\definecolor{itteny}{RGB}{244,229,0}
\definecolor{ittenyo}{RGB}{253,198,11}
\definecolor{itteno}{RGB}{241,142,28}
\definecolor{ittenor}{RGB}{234,98,31}
\definecolor{ittenr}{RGB}{227,35,34}
\definecolor{ittenrp}{RGB}{196,3,125}
\definecolor{ittenp}{RGB}{109,57,139}
\definecolor{ittenpb}{RGB}{68,78,153}
\definecolor{ittenb}{RGB}{42,113,176}
\definecolor{ittenbg}{RGB}{6,150,187}
\definecolor{itteng}{RGB}{0,142,91}
\definecolor{ittengy}{RGB}{140,187,38}

% Textfarbe ändern
\newcommand{\tred}[1]{\textcolor{red}{#1}}
\newcommand{\tgreen}[1]{\textcolor{green}{#1}}
\newcommand{\tblue}[1]{\textcolor{blue}{#1}}
\newcommand{\torange}[1]{\textcolor{orange}{#1}}
\newcommand{\tyellow}[1]{\textcolor{yellow}{#1}}
\newcommand{\tpurple}[1]{\textcolor{purple}{#1}}
\newcommand{\tcyan}[1]{\textcolor{cyan}{#1}}
\newcommand{\tbrown}[1]{\textcolor{brown}{#1}}

% Umstellen der Tabellen Definition
\newcommand{\mpb}[1][.3]{\begin{minipage}{#1\textwidth}\vspace*{3pt}}
\newcommand{\mpe}{\vspace*{3pt}\end{minipage}}

\newcommand{\resultul}[1]{\underline{\underline{#1}}}
\newcommand{\parskp}{$ $\\}	% new line after paragraph
\newcommand{\corr}{\;\widehat{=}\;}
\newcommand{\mdeg}{^{\circ}}

\newcommand{\nok}[2]{\begin{pmatrix}#1\\#2\end{pmatrix}}	% n über k BESSER: \binom{n}{k}
\newcommand{\mtr}[1]{\begin{pmatrix}#1\end{pmatrix}}	% Matrix
\newcommand{\dtr}[1]{\begin{vmatrix}#1\end{vmatrix}}	% Determinante (Betragsmatrix)
\renewcommand{\vec}[1]{\underline{#1}}	% Vektorschreibweise
\newcommand{\imptnt}[1]{\colorbox{red!30}{#1}}	% Wichtiges
\newcommand{\intd}[1]{\,\mathrm{d}#1}

\bibliography{../Literatur/HTW_Literatur.bib}

% Definition von Titel, Autor usw.
\DTitel{Betriebswirtschaftslehre}
\DUntertitel{Vorlesungsskript}
\DAutor{Falk-Jonatan Strube}
\DNotiz{Vorlesung von Dr. Wolf-Eckart Grüning}

\newcommand{\folie}[1]{\begin{center}
\includegraphics[page=#1, width=.7\columnwidth]{Vorlesung/I370_SS2016_crop.pdf}
%\includegraphics[page=#1, width=.7\columnwidth, trim={2.5cm 1.8cm 2.5cm 4.35cm},clip]{Vorlesung/I370_SS2016_EineFolieProSeite.pdf}
\end{center}}

\begin{document}

\maketitle
\newpage
\tableofcontents
\newpage

%\chapter*{Vorbemerkung}

\chapter{BWL als Wissenschaft}
\section{Angewandte- vs Grundwissenschaften}
\folie{10}
\begin{itemize}
\item BWL ist Anwendungswissenschaft
\item Praxis verändert sich stets (bspw. durch Internet)
\end{itemize}
\section{Gliederung der BWL}
\subsection{Funktionale Gliederung}
\folie{11}
Einteilung nach Funktion im Betrieb
\begin{itemize}
\item Grundfunktionen
\begin{itemize}
\item Beschaffung (Materialwirtschaft) $\rightarrow$ Produktion $\rightarrow$ Absatz
\end{itemize}
\end{itemize}
Wertschöpfung: es soll wenig Wert in die „Beschaffung“ einfließen, der Absatz soll maximiert werden.
\subsection{Institutionelle Gliederung}
\folie{12}
Einteilung nach Zweck des Betriebs

\subsection{Genetische Gliederung}
\folie{13}
Einteilung nach Lebenszeit des Betriebs\\
Liquidation muss nicht „Bankrott“ heißen, kann auch bewusste entscheidung sein.

\chapter{Management}
\section{Managementzyklus}
\folie{16}
Manegment deswegen so gut bezahlt, wegen: Entscheidungen\\
Entscheidungen sind die Herausforderungen des Managers im Vergleich zum Ausführenden, der weniger signifikant entscheiden muss.
\begin{itemize}
\item Planung\\
Planung der (eigenen) Tätigkeit. Eine gute Planung besteht aus:
\begin{itemize}
\item Zielfindung\\
Bsp.: „Kundenbeziehung schlecht, Software entwickeln“ $\rightarrow$ herausfinden, wie man die Zufriedenheit messen kann, um sie entsprechend \emph{quantitativ} verbessern zu können.
\end{itemize}
\item Organisation\\
Maßnahmen, um Ziel umsetzen zu können.
\item Personaleinsatz\\
Zuteilung des Personals zu den Maßnahmen.
\item Führung\\
Realisierung der Maßnahmen und eingreifen, damit sie entsprechend des Ziels umgesetzt werden.
\item Kontrolle\\
Ist-Stand prüfen und mit Ziel abgleichen.
\end{itemize}
\folie{17}
Problem mit einfachem Zyklus: Teilweise sind nicht alle Probleme nicht von Anfang an bekannt, wodurch die Planung fehlerhaft sein kann (Bsp.: Prüfungsplanung am Anfang vom Semester, obwohl die Modul-Inhalte noch gar nicht abzuschätzen sind).
\begin{itemize}
\item Planung:
\begin{itemize}
\item Strategische Planung
\item Operative Planung
\end{itemize}
\end{itemize}

\section{Managementkritik}
\folie{18}
\begin{itemize}
\item Kontrollillusion
\begin{itemize}
\item unbeabsichtigte Auswirkungen (bspw. leidendes soziales Umfeld bei großem betrieblichen Engagement)
\item ausbleiben von beabsichtigten Effekten (Überschätzung der eigenen Fähigkeiten $\rightarrow$ Lernziel kann nicht erreicht werden)
\end{itemize}
\item Mikromanagement
\item „Goldenes Pony“\\
Problem ist nicht zwangsläufig universell
\end{itemize}
\section{Merkmale eines Managers} \label{sec:merkmale_eines_managers}
\folie{19}
\begin{itemize}
\item technische Kompetenzen\\
Beherrschung des Fachgebiets (für Management oft nicht so entscheidend). Aber auch mentales Problem: Auswahl des Werkzeugs, was das Beste für den Zweck ist - nicht, was am einfachsten bzw. bekannt ist.
\item konzeptionelle Kompetenzen\\
Feingefühl für Planung; Planungsgeschick $\Rightarrow$ Lösungsfindung
\item soziale Kompetenzen
\end{itemize}
\paragraph{Management und Ethik}
\begin{itemize}
\item Der rechtschaffene Manager\\
Beispiel Entlassungen: Wird der sozial Benachteiligte behalten und der kompetentere Mitarbeiter entlassen, wird ggf. gegen das Unternehmen gehandelt -- aber moralisch.\\
Im Zweifelsfall gegen das Unternehmen.
\item Corporate Social Responsibility\\
Beispielsweise Sponsoring bei Fußballklubs, wo die Verantwortung gegenüber des Sponsors besteht.\\
Im Zweifelsfall gegen den Manager.
\end{itemize}
\folie{20}

\chapter{Grundlagen der Wirtschaft}
\section{Bedürfnisse, Bedarf, Markt, Wirtschaft}
\folie{23}
\folie{24}
Grundfrage: Was ist Wirtschaft?
\begin{itemize}
\item Beginnend bei: Bedürfniss\\
„Es fehlt etwas.“\\
Ist unendlich: Wird eines erfüllt, entstehen neue.
\begin{itemize}
\item Existenzbedürfnisse: Wohnen, Essen usw.
\item Grundbedürfnisse: „normale“ Bedürfnisse in der entsprechenden Gesellschaft (bpsw. Auto, Versicherung, …)
\item Luxusbedürfnisse: Motivation bspw. auch Statussymbol
\item komplementäre Bedürfnisse: abhängige Bedürfnisse: Bedürfnisse, die sich aus dem Erfüllen anderer Bedürfnisse ergeben. Bsp.: Man kauft sich einen Laserdrucker, und hat das neue Bedürfnis nach Tonern.
\end{itemize}
\item Bedarf\\
Beschreibt Menge, die durch Mittel an Bedürfnissen abgedeckt ist (bspw. wie viel Geld ist vorhanden um Bedürfniss zu stellen $\rightarrow$ Bedarf). Mittel sind immer begrenzt.
\item Wirtschaft
\item Güter\\
Wirtschaft produziert Güter (physische Waren, Dienstleistungen). In Qualität und Quantität nur begrenzt herstellbar.
\end{itemize}
Bedarf beeinflusst die Nachfrage, die Güter das Angebot auf den Markt.\\
Dieses Prinzip gilt, seit Menschen sich spezialisiert haben und dadurch jeweils eigene Güter für den Markt hatten.

\section{Wirtschaftsgüter}
\begin{itemize}
\item freie Güter
\begin{itemize}
\item bpsw. Wasser -- aber: Grundwasser ist knappes Gut
\end{itemize}
\item knappe Güter
\begin{itemize}
\item Waren
\begin{itemize}
\item Produktionsgüter
\item Konsumgüter
\end{itemize}
\item Rechte
\item Dienstleistungen
\end{itemize}
\end{itemize}
\folie{25}

\section{Markt- und Wettbewerbsformen}
\folie{26}
\paragraph{Übung} Beispiele:\\
\begin{tabular}{l l}
Begriff & Beispiel\\
\hline
Polypol & Lebensmittelproduktion\\
bileterales Monopol & \\
bilaterales Oligopol & Kriegswaffen, Großschiffbau, Spezialausrüstung
\end{tabular}\\
Ein bileterales Monopol darf es eigentlich nicht geben.

\section{Rechtsrahmen}
\folie{27}
Rechtsnormen:
\begin{itemize}
\item Normgeber (setzen der Standards)\\
Verschiedene Typen von Normen:
\begin{itemize}
\item \emph{Gesetze}\\
Normgeber: Parlamentarisch $\to$ Länder (Landtage) / Bund (Bundesrat, -tag)
\item \emph{Verordnungen}\\
Rechtsnormen, die auf Verwaltungswege entstehen\\
Normgeber: Ministier (Bundes-/Landes-). Bedarf Ermächtigungsgrundlage durch Gesetz. Ermöglicht dann schnellere Veränderungen.
\item \emph{Satzungen} (nicht Vereinssatzung (dies sind eher Statute), sondern Rechtsnormen)\\
Normgeber: Landkreise/Kommunen\\
Bsp.: Gemeindesatzung (bspw. für den korrekten Ortsnamen: Frankfurt am Main, Frankfurt (Oder))
\end{itemize}
Geschriebene Verhaltensregeln von Menschen und Menschengruppen.\\
Ziel: Zusammenleben von Menschen und -gruppen regeln.
\item Normen:
\begin{itemize}
\item StVO
\item Grundgesetz
\item BDSG
\item StGB
\item HGB\\
usw.
\end{itemize}
\end{itemize}
\paragraph{Übung:} Rechtsnormen im unternehmerischen Handeln
\begin{itemize}
\item BGB: Schuldrecht, Arbeitsverträge
\item HGB: Verträge zwischen Kaufleuten, Buchführungspflicht, …
\item Arbeitsschutzgesetz: bspw. Sichtheitsbeauftragten.
\item Bundesarbeitszeitgesetz
\item EStG: Einkommenssteuer (fur Privatperson)
\item KöStG: Körperschaftssteuergesetz (für Unternehmen)
\item UStG: Umsatzsteuer
\end{itemize}
Gesetze brauchen Kontrolle und Sanktionierungen.
\folie{28}
Unterschied:\\
\begin{tabular}{p{0.2\columnwidth} p{0.7\columnwidth}}
Privatrecht & ist einvernehmlich zwischen zwei gleichberechtigten Parteien. Ohne Einvernehmlichkeit, kein Vertrag\\
Öffentliches Recht & Gesetze ohne einvernehmlichkeit von übergeordneten Regierung
\end{tabular}\\
Arbeitsrecht und Wettbewerbsrecht sind sich nicht eindeutig einen dieser Kategorien zuzuordnen.

\section{Produktionsfaktoren}
\folie{29}
\begin{itemize}
\item Elementarfunktionen
\begin{itemize}
\item Arbeit: körperlich und geistig (in die Produkte selbst)
\item Rechte: Lizenzen für Software, Codecs usw.\\
Konzessionen: Rechte zur Nutzung von Naturschätzen (durch Staat)
\end{itemize}
\item Dispositive Faktoren
\begin{itemize}
\item Wissen: Weitergabe von Wissen und Erfahrung von älteren auf jüngeren Mitarbeitern
\end{itemize}
\end{itemize}
\paragraph{Übung} Möglichkeiten der Substitionen von Produktionsfaktoren\\
Produktionsfaktoren sind Input für Wertschöpfung.\\
Substitionen: 
\begin{itemize}
\item Maschinen/Roboter für Handarbeit
\item Patentinhaber einkaufen, anstatt Patent zu mieten
\item Patent imitieren anstatt zu mieten (vgl. Audio-/Video-Codecs)
\item Materialeinsparung durch Wissen
\end{itemize}

\section{Betriebliche Funktionen: Wertschöpfungskette}
\folie{30}
Vgl. Wertschöpfungskette $\Leftrightarrow$ Güterkreislauf\\
Wertschöpfung hauptsächlich durch roten Bereich, gelber Bereich unterstützend.

\chapter{Das Unternehmen}
\section{Was ist ein Unternehmen?}
\folie{33}
\begin{itemize}
\item Soziales System: nicht rein rationales System (Entscheidungen), nicht alle haben gleiche Voraussetzungen
\item Planvoll organisiert: Zielvorstellung mit Maßnahmen für Erfüllung der Ziele
\item Kombination von Produktionsfaktoren: zur Wertschöpfung
\item Marktausrichtung: $\downarrow$
\item Befriedigung von Bedürfnissen auf dem Markt
\end{itemize}
\paragraph{Übung} System? Planvolle Tätigkeit?
\begin{itemize}
\item System: abgeschlossener Betrachtungsbereich mit mehreren Bestandteilen in Wechselwirkung zueinander.\\
geregelte Abläufe (in künstlichen Systemen)
\item Planvolle Tätigkeit: 
\end{itemize}

\section{Rechtsformen}
\folie{34}
\begin{itemize}
\item privatrechliche Unternehmen
\begin{itemize}
\item Einzelunternehmen
\item Gesellschaftsunternehmen: Zusammenschluss von Unternehmern
\begin{itemize}
\item Personengesellschaften
\item Kapitalgesellschaft: bilden eigene „juristische Person“
\end{itemize}
\end{itemize}
\item öffentlich-rechtliche Unternehmen: Rechtsnormen(Gesetz, Verordnung, Satzung) regeln Tätigkeit
\end{itemize}
\folie{35}
\begin{itemize}
\item Haftung
\begin{itemize}
\item unbeschränkt
\item beschränkt: beschränkte Haftung limitiert auch Kreditwürdigkeit
\end{itemize}
\item Leitungsbefugnis
\begin{itemize}
\item Wer ist Chef? Bspw. in AG nur limitiert, selbst wenn man Anteil hat.
\end{itemize}
\item Gewinn- und Verlustbeteiligung
\item Kapitalbeschaffung
\begin{itemize}
\item Eigen-
\item Fremd- : braucht Sicherung (durch Haftung oder anderem)
\end{itemize}
\item Steuerbelastung
\begin{itemize}
\item Personengesellschaften: Einkünfte fließen in Einkommenssteuer ein
\item Kapitalgesellschaften: Kapitalgesellschaft wird separat besteuert, Person zusätzlich auch Einkommenssteuer
\end{itemize}
\item Publizitätspflicht
\end{itemize}
\subsection{Einzelunternehmen}
\folie{36}
\folie{37}
Aufnahme Gewerbebetrieb braucht:
\begin{itemize}
\item Gewerbeerlaubnis
\item beim Finanzamt anzeigen (passiert auch automatisch)
\end{itemize}
\subsection{Gesellschaft bürgerlichen Rechts (GbR)}
auch ARGE: Arbeitsgemeinschaft (Zusammenschluss von Bauunternehmen zum erreichten eines größen Bauwerks)
\folie{38}
Klage besteht aus:
\begin{itemize}
\item wen beklagt man?
\item was will man haben?
\end{itemize}
\folie{39}
\subsection{Offene Handelsgesellschaft (OHG)}
Bsp.: 
\begin{itemize}
\item Verlag C. H. Beck OHG (Verlag für Rechtsbücher)
\item Misch \& Goebel OHG
\end{itemize}
Firma: der Name\\
Unternehmen: Unternehmen, dass sich an Wirtschaftsbetrieb beteiligt.\\
$ \Rightarrow $ Zwei Firmen können ein Unternehmen bilden.
\folie{41}

\subsection{Kommanditgesellschaft (KG)}
OHG mit zwei Gesellschaftern:
\begin{itemize}
\item Komplementäre: wie OHG
\item Kommanditisten: von Geschäftsführung ausgeschlossen, hat Kontrollrecht -- Haften aber auch nur mit Kapitaleinlagen
\end{itemize}
Komamanditisten sind Investoren, die direktere Einsicht in ihre Investition haben (im Vgl. zu Aktien)\\
Bsp.:
\begin{itemize}
\item Bauer Vertriebs KG
\item SchwörerHaus KG
\end{itemize}
\folie{42}
\folie{43}

\subsection{Aktiengesellschaft (AG)}
Bsp.:
\begin{itemize}
\item Daimler, BMW usw.
\end{itemize}
$\Rightarrow$ die meisten großen Unternehmen sind AGs.
\folie{44}
\folie{45}
\folie{46}
DAX: 30 größten AGs in Deutschland (Deutsche Aktien Index)\\
Unterschied: AG an der Börse oder auf privatem Markt.\\
Anhang AG \emph{iL}: in Liquidation (in der Auflösung: Begleichung aller Schulden usw.)\\
AG: Kapitalbeschaffung relativ einfach (hat breite Basis an potentiellen Gesellschaftern)\\
Anzeigen von 25\% Grundkapital: Sperrminorität (ab 25\% könnte man Veto bei Beschlüssen in Aktionärversammlungen einlegen)
\folie{47}
\subsubsection{Kleine Aktiengesellschaft}
\folie{48}
\subsubsection{Europäische Aktiengesellschaft (SE)}
\folie{49}
Bsp.:\\
MAN SE, Conrad SE

\subsection{Gesellschaft mit beschränkter Haftung (GmbH)}
\folie{50}
\folie{51}
\folie{52}
\folie{53}

\subsection{GmbH \& Co. KG}
Wirkt als KG, Komplementär ist aber GmbH.
\folie{54}
\folie{55}

\subsection{Kommanditgesellschaft auf Aktion (KGaA)}
Grundstruktur KG, aber in Aktien zerlegt.
\folie{56}
\folie{57}
Warum eine GmbH \& Co. KG in Co. KGaA anstatt in AG übergehen wollen würde: in KGaA bleibt (alleiniges) Bestimmungsrecht durch Komplementär (im Gegensatz zu AG).
\folie{58}
\folie{59}

\subsection{Genossenschaft (eG)}
\folie{60}
\folie{61}
\folie{62}

\subsection{Unternehmensverfassung}
\folie{63}

\section{Unternehmenszusammenschlüsse}
\folie{64}
\begin{itemize}
\item horizontal
\begin{itemize}
\item Banken (Fusionierung): Commerzbank + Dresdner Bank
\item VW-Konzern: Zusammenschluss mehrerer Autohersteller
\end{itemize}
\item vertikal
\begin{itemize}
\item 
\end{itemize}
\item anorganisch
\begin{itemize}
\item Mitsubishi, Sony (mehrere Geschäftsfelder für finanziellen Ausgleich)
\item RWE (Energieerzeuger erweitert in den 90ern zu Kohle, Netzausbau, Abfallentsorgung, …)
\end{itemize}
\end{itemize}

\folie{65}
\begin{itemize}
\item Kartell
\begin{itemize}
\item eher loser Zusammenschluss
\item Preis-/ Gebietsabsprachen usw.
\item Syndikate: zwei Unternehmen bieten zusammen zwei Angebote an
\end{itemize}
\item Konzern
\begin{itemize}
\item gemeinsame Leitung
\item rechtliche Unabhängigkeit
\item $\Rightarrow$ bspw. VW (Teilunternehmen: unterschiedliche Marken), Media Markt u. Saturn usw.,  private Fernsehsender, …
\end{itemize}
\item Trust
\begin{itemize}
\item verschmelzen zweier Unternehmen (entweder ein Unternehmen nimmt das andere auf, oder sie verschmelzen zu einer neuen juristischen Person)
\end{itemize}
\end{itemize}
\paragraph{Übung} 
\begin{itemize}
\item Hardware-Hersteller (Mobiltelefone) $\to$ Kartell (strategische Allianz) $\to$ einheitliches OS (Android)
\end{itemize}
Derartige zusammenschlüsse können rechtmäßig und nicht-rechtmäßig sein.
\folie{66}
Mögliche Probleme bei Zusammenschlüssen:
\begin{itemize}
\item Gesetze, die Monopolbildung verbieten, damit kein Zusammenschluss von Unternehmen eine Marktbeherrschung erlangt
\item Zusammenschlüsse müssen vom Bundeskartellamt abgesegnet werden
\end{itemize}

\section{Unternehmensziele}
\folie{67}
Hauptziel: Gewinnmaximierung\\
$\text{Gewinn} = \text{Ertrag} - \text{Aufwand}$
\paragraph{Übung} CSR (siehe \ref{sec:merkmale_eines_managers}). \\
Beispiel: Unternehmesziel „maximale Gewinne“, CSR „Arbeitnehmer gut behandeln“ (vgl. Lohn).\\
Weiteres Bsp.: Bettler vor Bockwurst-Stand: Bockwurst geben oder nicht? Wenn ja, dann ggf. nach Feierabend, falls etwas übrig bleibt. Oder während des Dienstes mit möglichst viel Zuschauern (als Werbung).

\folie{68}
$\text{Rentabilität} = \frac{\text{Gewinn}}{\text{Einsatz}}$\\
$\text{Eigenkapital-Rentabilität} = \frac{\text{Gewinn}}{\text{Eigenkapital}}$ (in \%)
\folie{69}
$\text{Produktivität} = \frac{\text{Ergebnis}}{\text{Faktoreinsatz}}$
\folie{70}
\folie{71}
\folie{72}
\paragraph{Übung}\parskp
\begin{tabular}{L{0.3} L{0.7}}
Humanziele & Messgröße\tabularnewline
\hline
Mitarbeiterzufriedenheit erhöhen & $\Delta$ Arbeitsproduktivität oder $\Delta$ Einhaltung Arbeitsziele oder Mitarbeiter-Zufriedenheitsumfrage\tabularnewline
\end{tabular}

%\newpage
%\printbibliography
\end{document}