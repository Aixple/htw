\section{Digitale Unterschriften}
\slides{it1-53-krypto-zert_print}{3}
\subsection{Haschwert}
\slides{it1-53-krypto-zert_print}{4}
(Verschlüsselter Hash ergibt Unterschrift fester Länge)
\subsection{Prüfung der Unterschrift}
\slides{it1-53-krypto-zert_print}{5}
(Entschlüsselter Hash muss zur Nachricht passen)
\subsection{Nutzung öffentlicher Schlüssel}
\slides{it1-53-krypto-zert_print}{6}

\section{Zertifikate}
\subsection{X.509-Zertifikat: Namensbestandteile}
\slides{it1-53-krypto-zert_print}{7}
\subsection{Zertifizierung eines öffentlichen Schlüssels}
\slides{it1-53-krypto-zert_print}{8}
\subsection{Schlüsselzertifikate}
\slides{it1-53-krypto-zert_print}{9}

\section{Public-Key-Infrastructure (PKI)}
\slides{it1-53-krypto-zert_print}{10}
Beispiel: zweistufiges PKI-Modell (Wurzel-CA und Zwischen-CA)
\subsection{Verifikation einer Signatur (gleiche CAs)}
\slides{it1-53-krypto-zert_print}{11}
(TLN1 erhält Nachricht von TLN2)
\subsection{Verifikation einer Signatur (verschiedene CAs)}
\slides{it1-53-krypto-zert_print}{12}
\subsection{Beispiele}
\subsubsection{Zertifikatsverwaltung der HTW}
\slides{it1-53-krypto-zert_print}{13}
\subsubsection{Zertifikat: Deutsche Telekom Root CA 2}
\slides{it1-53-krypto-zert_print}{14}
\subsubsection{Zertifikat: DFN-Verein PCA Global - G01}
\slides{it1-53-krypto-zert_print}{15}
\subsubsection{Zertifikat: HTW-Dresden CA - G02}
\slides{it1-53-krypto-zert_print}{16}
\subsubsection{Serverzertifikat: webmail.htw-dresden.de}
\slides{it1-53-krypto-zert_print}{17}
\subsection{Zertifikate: Formate}
\slides{it1-53-krypto-zert_print}{18}
\subsection{Zertifizierungsstellen (CA)}
\slides{it1-53-krypto-zert_print}{19}
\subsection{Rückruflisten (Certificate Revocation List (CRL))}
\slides{it1-53-krypto-zert_print}{20}
\subsection{Zeitstempeldienst}
\slides{it1-53-krypto-zert_print}{21}
\subsection{Alternativen zu PKI}
\slides{it1-53-krypto-zert_print}{22}

\section{Web of Trust}
\slides{it1-53-krypto-zert_print}{23}












