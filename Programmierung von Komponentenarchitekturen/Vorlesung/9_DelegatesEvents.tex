\section{Delegate}
Delegate (Delegierter = Weiterleiter eines Auftrags) ist ein Typ der den Zeiger auf eine Methode beschreibt
\begin{lstlisting}[language={[Sharp]C}]
	public delegate double CalculateHandler(double value1, double value2); 
\end{lstlisting}
Danach kann eine Variable vom Typ des Delegate deklariert werden:\\
\lstinline$CalculateHandler calculate;$\\
Diese ist ein Zeiger auf eine beliebige Methode die in Argumentliste und Rückgabewert mit dem Delegate übereinstimmt.\\
Der Zeiger kann wie folgt auf eine vorhandene Methode gesetzt werden:\\
\lstinline$calculate = someMatchingFunction;$\\
Danach wird die zugewiesene Methode aufgerufen durch :\\
\lstinline$calculate(value1, value2)$

\subsubsection*{Beispiel}

\begin{lstlisting}[language={[Sharp]C}]
public delegate double CalculateHandler(double value1, double value2);

class Functions {
	public static double Add(double x, double y)  {  return x + y;  }
	public static double Subtract(double x, double y)  {  return x {}- y;  }
}
// ...
static void Main(string[] args) {
	CalculateHandler calculate;
	Console.Write("Geben Sie den ersten Operanden ein: ");
	double input1 = Convert.ToDouble(Console.ReadLine());
	Console.Write("Geben Sie den zweiten Operanden ein: ");
	double input2 = Convert.ToDouble(Console.ReadLine());  
	Console.Write("Operation: Addition - (A) oder Subtraktion - (S)? ");
	string wahl = Console.ReadLine().ToUpper();

	if (wahl == "A")
		calculate = Functions.Add;
	else if (wahl == "S") 
		calculate = Functions.Subtract;
	else 
		return;

	double result = calculate(input1, input2);
	Console.WriteLine("Ergebnis = {0}{\textbackslash}n{\textbackslash}n", result);
}
\end{lstlisting}

Soweit ist das ganze erstmal äquivalent zu: 
\begin{lstlisting}[language={[Sharp]C}]
double result;
if(wahl == "A")	
	result = Functions.Add(input1, input2);
else if(wahl == "S")
	result = Functions.Subtract(input1, input2);
\end{lstlisting}

Nur mit dem Unterschied das die Ausführung der Funktion innerhalb der if-Abfrage steht. 

\section{Multicast Delegates}
Mehrere Delegates werden in einem kombiniert.\\
Aufruf des Delegates führt also mehrere Funktionen hintereinander aus:
\begin{lstlisting}[language={[Sharp]C}]
	calculate = Functions.Subtract;
	calculate += Functions.Add;
	double result = calculate(input1, input2);
\end{lstlisting}

\subsection{Generische Delegates}
C\# stellt die meistbenutzten Delegates zur Verfügung

\begin{lstlisting}[language={[Sharp]C}]
delegate void Action();
delegate TReturn Func<TReturn>();
delegate TReturn Func<TArg, TReturn>(TArg arg);
delegate TReturn Func<TArg1, TArg2, TReturn>(TArg1 arg1, TArg2 arg2); 
\end{lstlisting}
Diese sind vordefiniert und können überall benutzt werden.\\
Man könnte also auf die Deklaration von CalculateHandler verzichten und diesen überall ersetzen durch:

%%% TODO

\subsection{Lambda Funktionen}
(anonyme Funktionen)\\
Syntax:  (Argumente) => Funktionsaufruf

\subsubsection*{Beispiel}
\begin{lstlisting}[language={[Sharp]C}]
if (wahl == "A")
	calculate = (x, y) => x+y;
else if (wahl == "S")
	calculate = (x, y) => x+y;
\end{lstlisting}

Einschränkungen für lambda Funktionen:
\begin{itemize}
\item Zugriff nur auf lokale Variablen
\item keine Sprunganweisungen (break, goto)
\end{itemize}

\subsubsection*{Weitere Beispiele}
\begin{lstlisting}[language={[Sharp]C}]
x => x * x;

() => 42;

() => Console.WriteLine("Hallo");

() => { 
	Console.WriteLine("Hallo");
	return 42;
}
\end{lstlisting}

Was ist der Typ dieser Funktionen? 

\section{Ereignisse}

\begin{center}
\begin{tabular}{L{.3}L{.3}}
Aufrufer\newline (Client) &  Objekt\newline (Server)
\end{tabular}
\end{center}
Ereignis: Aufruf einer Methode im Client\\
$\to$herausragende Rolle bei GUI Programmierung

\subsubsection*{Beispiel}
\begin{lstlisting}[language={[Sharp]C}]
public class Circle {
	private double radius;
	public double Radius {
		get { return radius; }
		set { radius = value; }
	}
}

public class Program {  
	static void Main(string[] args) {
		Circle kreis = new Circle();
		kreis.Radius = -1;
	}
	
	static void kreis_InvalidRadius() {
		Console.WriteLine("Invalid Radius.");  
	}
}
\end{lstlisting}

\emph{Ziel}: Änderung des Radius zu einem negativen Wert soll die Methode \lstinline$kreis_InvalidRadius()$ aufrufen.

\begin{itemize}
\item \lstinline$kreis_InvalidRadius()$ ist ein EventHandler (behandelt ein auftretendes Ereignis)
\item Dazu muss Circle die Methode \lstinline$kreis_InvalidRadius()$ übergeben werden
\item Wie kann man eine Methode übergeben? 
\end{itemize}

\begin{lstlisting}[language={[Sharp]C}]
public class Circle {
	// ...
	public delegate void InvalidRadiusEventHandler();
	public event InvalidRadiusEventHandler InvalidRadius;
	public double Radius {
		get { return radius; }
		set {
			if (value >= 0) Radius = value;
			else if (InvalidRadius != null)  InvalidRadius();
		}
	}
}

public class Program {  
	static void Main(string[] args) {
		Circle kreis = new Circle();
		kreis.InvalidRadius += kreis_InvalidRadius;
		kreis.Radius = -1;
	}

	static void kreis_InvalidRadius() {
		Console.WriteLine("Invalid Radius.");  
	}
}
\end{lstlisting}

Schlüsselwort \emph{event}: kann prinzipiell auch weggelassen werden, ändert nur den Zugriff auf den delegate: für events nur mit += oder -=  möglich ist

\subsection{Ereignisse mit Übergabeparametern}

Oft sollte der Ereignishandler noch Argumente übergeben bekommen
\begin{itemize}
\item wer hat das Ereignis ausgelöst (hier: kreis): \emph{Sender}
\item zusätzliche ereignisspezifische Daten: \emph{EventArgs}
\end{itemize}
Standard-Signatur für .NET Ereignishandler enthält diese 2 Argumente:\\
\lstinline$static void kreis_InvalidRadius(object sender, EventArgs e)$

\subsubsection*{Beispiel}
\begin{lstlisting}[language={[Sharp]C}]
static void kreis_InvalidRadius(object sender, InvalidRadiusEventArgs e) {
	Console.Write("Ein Radius von {0} ist nicht zulässig.", e.invalidRadius);
	Console.Write("Neueingabe: ");
	((Circle)sender).Radius = Convert.ToDouble(Console.ReadLine());
}

public class Circle {
	public delegate void InvalidRadiusEventHandler(object o, InvalidRadiusEventArgs e);
	// ...
	public double Radius {
		get { return radius; }
		set {
			if (value >= 0)
				radius = value;
			else 
				InvalidRadius(  );
		}
	}
}

public class InvalidRadiusEventArgs : EventArgs {
	public double invalidRadius;
	public InvalidRadiusEventArgs(double invalidRadius_) {
		invalidRadius = invalidRadius_;
	}
}
\end{lstlisting}

Frage: Was passiert wenn nach einem falschen Radius, bei Neueingabe wieder etwas negatives eingegeben wird?

%%% TODO
