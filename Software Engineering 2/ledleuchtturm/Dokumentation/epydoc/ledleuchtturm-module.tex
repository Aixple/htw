%
% API Documentation for API Documentation
% Package ledleuchtturm
%
% Generated by epydoc 3.0.1
% [Tue Jul 18 00:55:24 2017]
%

%%%%%%%%%%%%%%%%%%%%%%%%%%%%%%%%%%%%%%%%%%%%%%%%%%%%%%%%%%%%%%%%%%%%%%%%%%%
%%                          Module Description                           %%
%%%%%%%%%%%%%%%%%%%%%%%%%%%%%%%%%%%%%%%%%%%%%%%%%%%%%%%%%%%%%%%%%%%%%%%%%%%

    \index{ledleuchtturm \textit{(package)}|(}
\section{Paket ledleuchtturm}

    \label{ledleuchtturm}
Das Paket ledleuchtturm verbindet sich mit einer postgresql Datenbank und 
wartet auf Aktualisierungen einer definierten Maschine. Bei einer 
Aktualisierung werden definierte Ports eines Raspberry Pi ein- oder 
ausgeschaltet. Direkt nach dem Verbindungsaufbau wird zuerst der aktuelle 
Status gelesen und dann werden die entsprechenden Pins geschaltet. Danach 
wird auf Updates gewartet. Durch das Signal Strg-C werden die Ports des 
Raspberry Pi in ihren Anfangszustand versetzt und alle Verbindungen oder 
geöffneten Dateien geschlossen.


%%%%%%%%%%%%%%%%%%%%%%%%%%%%%%%%%%%%%%%%%%%%%%%%%%%%%%%%%%%%%%%%%%%%%%%%%%%
%%                                Modules                                %%
%%%%%%%%%%%%%%%%%%%%%%%%%%%%%%%%%%%%%%%%%%%%%%%%%%%%%%%%%%%%%%%%%%%%%%%%%%%

\subsection{Module}

\begin{itemize}
\setlength{\parskip}{0ex}
\item \textbf{dbFunc}: Funktionen zur Arbeit mit der Datenbank.



  \textit{(Unterabschnitt \ref{ledleuchtturm:dbFunc}, S.~\pageref{ledleuchtturm:dbFunc})}

\item \textbf{fileFunc}: Funktionen zur Arbeit mit Dateien.



  \textit{(Unterabschnitt \ref{ledleuchtturm:fileFunc}, S.~\pageref{ledleuchtturm:fileFunc})}

\item \textbf{ledFunc}: Funktionen zur Arbeit mit LEDs.



  \textit{(Unterabschnitt \ref{ledleuchtturm:ledFunc}, S.~\pageref{ledleuchtturm:ledFunc})}

\item \textbf{ledleuchtturm}: Main-Funktion des Paketes ledleuchtturm.



  \textit{(Unterabschnitt \ref{ledleuchtturm:ledleuchtturm}, S.~\pageref{ledleuchtturm:ledleuchtturm})}

\item \textbf{stdFunc}: Funktionen, die beim Entwicklungsprozess benutzt werden.



  \textit{(Unterabschnitt \ref{ledleuchtturm:stdFunc}, S.~\pageref{ledleuchtturm:stdFunc})}

\end{itemize}


%%%%%%%%%%%%%%%%%%%%%%%%%%%%%%%%%%%%%%%%%%%%%%%%%%%%%%%%%%%%%%%%%%%%%%%%%%%
%%                               Variables                               %%
%%%%%%%%%%%%%%%%%%%%%%%%%%%%%%%%%%%%%%%%%%%%%%%%%%%%%%%%%%%%%%%%%%%%%%%%%%%

  \subsection{Variablen}

    \vspace{-1cm}
\hspace{\varindent}\begin{longtable}{|p{\varnamewidth}|p{\vardescrwidth}|l}
\cline{1-2}
\cline{1-2} \centering \textbf{Name} & \centering \textbf{Beschreibung}& \\
\cline{1-2}
\endhead\cline{1-2}\multicolumn{3}{r}{\small\ldots}\\\endfoot\cline{1-2}
\endlastfoot\raggedright \_\-\_\-p\-a\-c\-k\-a\-g\-e\-\_\-\_\- & \raggedright \textbf{Wert:} 
{\tt None}&\\
\cline{1-2}
\end{longtable}

    \index{ledleuchtturm \textit{(package)}|)}
