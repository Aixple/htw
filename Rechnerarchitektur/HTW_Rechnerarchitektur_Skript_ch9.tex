\section{Bus: Einführung}
\subsection{Definition}
\slides{09}{2}
\subsection{Generelle Struktur}
\slides{09}{3}
\subsection*{Was definiert einen Bus?}
\slides{09}{4}
\subsection{Vor- und Nachteile}
\slides{09}{5}
\subsection{Terminologie}
\slides{09}{6}
\slides{09}{7}
\slides{09}{8}
\slides{09}{9}
\subsection{Physikalische Struktur: parallel und seriell}
\slides{09}{10}

\section{Hierarchie}
\subsection*{Hierarchische Organisation}
\slides{09}{11}
\subsection{Prozessorbus}
\slides{09}{12}
\subsection{Speicherbus}
\slides{09}{13}
\subsection{Peripheriebus}
\slides{09}{14}
\subsection{Ein-/Ausgabebus}
\slides{09}{15}

\unimptnt{
\subsection{Alternative Einteilung}
\slides{09}{16}
}

\section{Topologie und Kopplung}
\subsection{Topologien}
\slides{09}{17}
\subsubsection{Repeater}
\slides{09}{18}
\subsubsection{Hub (Nabe/Mittelpunkt)}
\slides{09}{19}
\subsubsection{Bridge (Brücke)}
\slides{09}{20}

\section{Dedizierung}
\subsection{Dedizierter Bus}
\slides{09}{21}
\subsection{Nichtdedizierter Bus}
\slides{09}{22}

\section{Partitionierung}
\subsection{Ressourcenpartitionierter Bus}
\slides{09}{23}
\unimptnt{
\subsubsection*{Charakteristika}
\slides{09}{24}
}
\subsection{Funktionspartitionierter Bus}
\slides{09}{25}

\section{Transaktion und Übertragungsarten}
\subsection{Überblick}
\slides{09}{26}
\subsubsection*{(Nicht-)multiplexiert}
\slides{09}{27}
\unimptnt{
\subsection{Multiplexbus}
\slides{09}{28}
}


